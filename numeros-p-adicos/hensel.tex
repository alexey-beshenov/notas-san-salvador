\documentclass{article}

% TODO : CLEAN UP THIS MESS
% (AND MAKE SURE ALL TEXTS STILL COMPILE)
\usepackage[leqno]{amsmath}
\usepackage{amssymb}
\usepackage{graphicx}

\usepackage{diagbox} % table heads with diagonal lines
\usepackage{relsize}

\usepackage{wasysym}
\usepackage{scrextend}
\usepackage{epigraph}
\setlength\epigraphwidth{.6\textwidth}

\usepackage[utf8]{inputenc}

\usepackage{titlesec}
\titleformat{\chapter}[display]
  {\normalfont\sffamily\huge\bfseries}
  {\chaptertitlename\ \thechapter}{5pt}{\Huge}
\titleformat{\section}
  {\normalfont\sffamily\Large\bfseries}
  {\thesection}{1em}{}
\titleformat{\subsection}
  {\normalfont\sffamily\large\bfseries}
  {\thesubsection}{1em}{}
\titleformat{\part}[display]
  {\normalfont\sffamily\huge\bfseries}
  {\partname\ \thepart}{0pt}{\Huge}

\usepackage[T1]{fontenc}
\usepackage{fourier}
\usepackage{paratype}

\usepackage[symbol,perpage]{footmisc}

\usepackage{perpage}
\MakePerPage{footnote}

\usepackage{array}
\newcolumntype{x}[1]{>{\centering\hspace{0pt}}p{#1}}

% TODO: the following line causes conflict with new texlive (!)
% \usepackage[english,russian,polutonikogreek,spanish]{babel}
% \newcommand{\russian}[1]{{\selectlanguage{russian}#1}}

% Remove conflicting options for the moment:
\usepackage[english,polutonikogreek,spanish]{babel}

\AtBeginDocument{\shorthandoff{"}}
\newcommand{\greek}[1]{{\selectlanguage{polutonikogreek}#1}}

% % % % % % % % % % % % % % % % % % % % % % % % % % % % % %
% Limit/colimit symbols (with accented i: lím / colím)

\usepackage{etoolbox} % \patchcmd

\makeatletter
\patchcmd{\varlim@}{lim}{\lim}{}{}
\makeatother
\DeclareMathOperator*{\colim}{co{\lim}}
\newcommand{\dirlim}{\varinjlim}
\newcommand{\invlim}{\varprojlim}

% % % % % % % % % % % % % % % % % % % % % % % % % % % % % %

\usepackage[all,color]{xy}

\usepackage{pigpen}
\newcommand{\po}{\ar@{}[dr]|(.4){\text{\pigpenfont I}}}
\newcommand{\pb}{\ar@{}[dr]|(.3){\text{\pigpenfont A}}}
\newcommand{\polr}{\ar@{}[dr]|(.65){\text{\pigpenfont A}}}
\newcommand{\pour}{\ar@{}[ur]|(.65){\text{\pigpenfont G}}}
\newcommand{\hstar}{\mathop{\bigstar}}

\newcommand{\bigast}{\mathop{\Huge \mathlarger{\mathlarger{\ast}}}}

\newcommand{\term}{\textbf}

\usepackage{stmaryrd}

\usepackage{cancel}

\usepackage{tikzsymbols}

\newcommand{\open}{\underset{\mathrm{open}}{\hookrightarrow}}
\newcommand{\closed}{\underset{\mathrm{closed}}{\hookrightarrow}}

\newcommand{\tcol}[2]{{#1 \choose #2}}

\newcommand{\homot}{\simeq}
\newcommand{\isom}{\cong}
\newcommand{\cH}{\mathcal{H}}
\renewcommand{\hom}{\mathrm{hom}}
\renewcommand{\div}{\mathop{\mathrm{div}}}
\renewcommand{\Im}{\mathop{\mathrm{Im}}}
\renewcommand{\Re}{\mathop{\mathrm{Re}}}
\newcommand{\id}[1]{\mathrm{id}_{#1}}
\newcommand{\idid}{\mathrm{id}}

\newcommand{\ZG}{{\ZZ G}}
\newcommand{\ZH}{{\ZZ H}}

\newcommand{\quiso}{\simeq}

\newcommand{\personality}[1]{{\sc #1}}

\newcommand{\mono}{\rightarrowtail}
\newcommand{\epi}{\twoheadrightarrow}
\newcommand{\xepi}[1]{\xrightarrow{#1}\mathrel{\mkern-14mu}\rightarrow}

% % % % % % % % % % % % % % % % % % % % % % % % % % % % % %

\DeclareMathOperator{\Ad}{Ad}
\DeclareMathOperator{\Aff}{Aff}
\DeclareMathOperator{\Ann}{Ann}
\DeclareMathOperator{\Aut}{Aut}
\DeclareMathOperator{\Br}{Br}
\DeclareMathOperator{\CH}{CH}
\DeclareMathOperator{\Cl}{Cl}
\DeclareMathOperator{\Coeq}{Coeq}
\DeclareMathOperator{\Coind}{Coind}
\DeclareMathOperator{\Cop}{Cop}
\DeclareMathOperator{\Corr}{Corr}
\DeclareMathOperator{\Cor}{Cor}
\DeclareMathOperator{\Cov}{Cov}
\DeclareMathOperator{\Der}{Der}
\DeclareMathOperator{\Div}{Div}
\DeclareMathOperator{\D}{D}
\DeclareMathOperator{\Ehr}{Ehr}
\DeclareMathOperator{\End}{End}
\DeclareMathOperator{\Eq}{Eq}
\DeclareMathOperator{\Ext}{Ext}
\DeclareMathOperator{\Frac}{Frac}
\DeclareMathOperator{\Frob}{Frob}
\DeclareMathOperator{\Funct}{Funct}
\DeclareMathOperator{\Fun}{Fun}
\DeclareMathOperator{\GL}{GL}
\DeclareMathOperator{\Gal}{Gal}
\DeclareMathOperator{\Gr}{Gr}
\DeclareMathOperator{\Hol}{Hol}
\DeclareMathOperator{\Hom}{Hom}
\DeclareMathOperator{\Ho}{Ho}
\DeclareMathOperator{\Id}{Id}
\DeclareMathOperator{\Ind}{Ind}
\DeclareMathOperator{\Inn}{Inn}
\DeclareMathOperator{\Isom}{Isom}
\DeclareMathOperator{\Ker}{Ker}
\DeclareMathOperator{\Lan}{Lan}
\DeclareMathOperator{\Lie}{Lie}
\DeclareMathOperator{\Map}{Map}
\DeclareMathOperator{\Mat}{Mat}
\DeclareMathOperator{\Max}{Max}
\DeclareMathOperator{\Mor}{Mor}
\DeclareMathOperator{\Nat}{Nat}
\DeclareMathOperator{\Nrd}{Nrd}
\DeclareMathOperator{\Ob}{Ob}
\DeclareMathOperator{\Out}{Out}
\DeclareMathOperator{\PGL}{PGL}
\DeclareMathOperator{\PSL}{PSL}
\DeclareMathOperator{\PSU}{PSU}
\DeclareMathOperator{\Pic}{Pic}
\DeclareMathOperator{\RHom}{RHom}
\DeclareMathOperator{\Rad}{Rad}
\DeclareMathOperator{\Ran}{Ran}
\DeclareMathOperator{\Rep}{Rep}
\DeclareMathOperator{\Res}{Res}
\DeclareMathOperator{\SL}{SL}
\DeclareMathOperator{\SO}{SO}
\DeclareMathOperator{\SU}{SU}
\DeclareMathOperator{\Sh}{Sh}
\DeclareMathOperator{\Sing}{Sing}
\DeclareMathOperator{\Specm}{Specm}
\DeclareMathOperator{\Spec}{Spec}
\DeclareMathOperator{\Sp}{Sp}
\DeclareMathOperator{\Stab}{Stab}
\DeclareMathOperator{\Sym}{Sym}
\DeclareMathOperator{\Tors}{Tors}
\DeclareMathOperator{\Tor}{Tor}
\DeclareMathOperator{\Tot}{Tot}
\DeclareMathOperator{\UUU}{U}

\DeclareMathOperator{\adj}{adj}
\DeclareMathOperator{\ad}{ad}
\DeclareMathOperator{\af}{af}
\DeclareMathOperator{\card}{card}
\DeclareMathOperator{\cm}{cm}
\DeclareMathOperator{\codim}{codim}
\DeclareMathOperator{\cod}{cod}
\DeclareMathOperator{\coeq}{coeq}
\DeclareMathOperator{\coim}{coim}
\DeclareMathOperator{\coker}{coker}
\DeclareMathOperator{\cont}{cont}
\DeclareMathOperator{\conv}{conv}
\DeclareMathOperator{\cor}{cor}
\DeclareMathOperator{\depth}{depth}
\DeclareMathOperator{\diag}{diag}
\DeclareMathOperator{\diam}{diam}
\DeclareMathOperator{\dist}{dist}
\DeclareMathOperator{\dom}{dom}
\DeclareMathOperator{\eq}{eq}
\DeclareMathOperator{\ev}{ev}
\DeclareMathOperator{\ex}{ex}
\DeclareMathOperator{\fchar}{char}
\DeclareMathOperator{\fr}{fr}
\DeclareMathOperator{\gr}{gr}
\DeclareMathOperator{\im}{im}
\DeclareMathOperator{\infl}{inf}
\DeclareMathOperator{\interior}{int}
\DeclareMathOperator{\intrel}{intrel}
\DeclareMathOperator{\inv}{inv}
\DeclareMathOperator{\length}{length}
\DeclareMathOperator{\mcd}{mcd}
\DeclareMathOperator{\mcm}{mcm}
\DeclareMathOperator{\multideg}{multideg}
\DeclareMathOperator{\ord}{ord}
\DeclareMathOperator{\pr}{pr}
\DeclareMathOperator{\rel}{rel}
\DeclareMathOperator{\res}{res}
\DeclareMathOperator{\rkred}{rkred}
\DeclareMathOperator{\rkss}{rkss}
\DeclareMathOperator{\rk}{rk}
\DeclareMathOperator{\sgn}{sgn}
\DeclareMathOperator{\sk}{sk}
\DeclareMathOperator{\supp}{supp}
\DeclareMathOperator{\trdeg}{trdeg}
\DeclareMathOperator{\tr}{tr}
\DeclareMathOperator{\vol}{vol}

\newcommand{\iHom}{\underline{\Hom}}

\renewcommand{\AA}{\mathbb{A}}
\newcommand{\CC}{\mathbb{C}}
\renewcommand{\SS}{\mathbb{S}}
\newcommand{\TT}{\mathbb{T}}
\newcommand{\PP}{\mathbb{P}}
\newcommand{\BB}{\mathbb{B}}
\newcommand{\RR}{\mathbb{R}}
\newcommand{\ZZ}{\mathbb{Z}}
\newcommand{\FF}{\mathbb{F}}
\newcommand{\HH}{\mathbb{H}}
\newcommand{\NN}{\mathbb{N}}
\newcommand{\QQ}{\mathbb{Q}}
\newcommand{\KK}{\mathbb{K}}

% % % % % % % % % % % % % % % % % % % % % % % % % % % % % %

\usepackage{amsthm}

\newcommand{\legendre}[2]{\left(\frac{#1}{#2}\right)}

\newcommand{\examplesymbol}{$\blacktriangle$}
\renewcommand{\qedsymbol}{$\blacksquare$}

\newcommand{\dfn}{\mathrel{\mathop:}=}
\newcommand{\rdfn}{=\mathrel{\mathop:}}

\usepackage{xcolor}
\definecolor{mylinkcolor}{rgb}{0.0,0.4,1.0}
\definecolor{mycitecolor}{rgb}{0.0,0.4,1.0}
\definecolor{shadecolor}{rgb}{0.79,0.78,0.65}
\definecolor{gray}{rgb}{0.6,0.6,0.6}

\usepackage{colortbl}

\definecolor{myred}{rgb}{0.7,0.0,0.0}
\definecolor{mygreen}{rgb}{0.0,0.7,0.0}
\definecolor{myblue}{rgb}{0.0,0.0,0.7}

\definecolor{redshade}{rgb}{0.9,0.5,0.5}
\definecolor{greenshade}{rgb}{0.5,0.9,0.5}

\usepackage[unicode,colorlinks=true,linkcolor=mylinkcolor,citecolor=mycitecolor]{hyperref}
\newcommand{\refref}[2]{\hyperref[#2]{#1~\ref*{#2}}}
\newcommand{\eqnref}[1]{\hyperref[#1]{(\ref*{#1})}}

\newcommand{\tos}{\!\!\to\!\!}

\usepackage{framed}

\newcommand{\cequiv}{\simeq}

\makeatletter
\newcommand\xleftrightarrow[2][]{%
  \ext@arrow 9999{\longleftrightarrowfill@}{#1}{#2}}
\newcommand\longleftrightarrowfill@{%
  \arrowfill@\leftarrow\relbar\rightarrow}
\makeatother

\newcommand{\bsquare}{\textrm{\ding{114}}}

% % % % % % % % % % % % % % % % % % % % % % % % % % % % % %

\newtheoremstyle{myplain}
  {\topsep}   % ABOVESPACE
  {\topsep}   % BELOWSPACE
  {\itshape}  % BODYFONT
  {0pt}       % INDENT (empty value is the same as 0pt)
  {\bfseries} % HEADFONT
  {.}         % HEADPUNCT
  {5pt plus 1pt minus 1pt} % HEADSPACE
  {\thmnumber{#2}. \thmname{#1}\thmnote{ (#3)}}   % CUSTOM-HEAD-SPEC

\newtheoremstyle{myplainnameless}
  {\topsep}   % ABOVESPACE
  {\topsep}   % BELOWSPACE
  {\normalfont}  % BODYFONT
  {0pt}       % INDENT (empty value is the same as 0pt)
  {\bfseries} % HEADFONT
  {.}         % HEADPUNCT
  {5pt plus 1pt minus 1pt} % HEADSPACE
  {\thmnumber{#2}}   % CUSTOM-HEAD-SPEC 

\newtheoremstyle{sectionexercise}
  {\topsep}   % ABOVESPACE
  {\topsep}   % BELOWSPACE
  {\normalfont}  % BODYFONT
  {0pt}       % INDENT (empty value is the same as 0pt)
  {\bfseries} % HEADFONT
  {.}         % HEADPUNCT
  {5pt plus 1pt minus 1pt} % HEADSPACE
  {Ejercicio \thmnumber{#2}\thmnote{ (#3)}}   % CUSTOM-HEAD-SPEC

\newtheoremstyle{mydefinition}
  {\topsep}   % ABOVESPACE
  {\topsep}   % BELOWSPACE
  {\normalfont}  % BODYFONT
  {0pt}       % INDENT (empty value is the same as 0pt)
  {\bfseries} % HEADFONT
  {.}         % HEADPUNCT
  {5pt plus 1pt minus 1pt} % HEADSPACE
  {\thmnumber{#2}. \thmname{#1}\thmnote{ (#3)}}   % CUSTOM-HEAD-SPEC

% EN ESPAÑOL

\newtheorem*{hecho*}{Hecho}
\newtheorem*{corolario*}{Corolario}
\newtheorem*{teorema*}{Teorema}
\newtheorem*{conjetura*}{Conjetura}
\newtheorem*{proyecto*}{Proyecto}
\newtheorem*{observacion*}{Observación}

\newtheorem*{lema*}{Lema}
\newtheorem*{resultado-clave*}{Resultado clave}
\newtheorem*{proposicion*}{Proposición}

\theoremstyle{definition}
\newtheorem*{ejercicio*}{Ejercicio}
\newtheorem*{definicion*}{Definición}
\newtheorem*{comentario*}{Comentario}
\newtheorem*{definicion-alternativa*}{Definición alternativa}
\newtheorem*{ejemploxs}{Ejemplo}
\newenvironment{ejemplo*}
  {\pushQED{\qed}\renewcommand{\qedsymbol}{\examplesymbol}\ejemploxs}
  {\popQED\endejemploxs}

\theoremstyle{myplain}
\newtheorem{proposicion}{Proposición}[section]

\newtheorem{proyecto}[proposicion]{Proyecto}
\newtheorem{teorema}[proposicion]{Teorema}
\newtheorem{corolario}[proposicion]{Corolario}
\newtheorem{hecho}[proposicion]{Hecho}
\newtheorem{lema}[proposicion]{Lema}

\newtheorem{observacion}[proposicion]{Observación}

\newenvironment{observacionejerc}
    {\pushQED{\qed}\renewcommand{\qedsymbol}{$\square$}\csname inner@observacionejerc\endcsname}
    {\popQED\csname endinner@observacionejerc\endcsname}
\newtheorem{inner@observacionejerc}[proposicion]{Observación}

\newenvironment{proposicionejerc}
    {\pushQED{\qed}\renewcommand{\qedsymbol}{$\square$}\csname inner@proposicionejerc\endcsname}
    {\popQED\csname endinner@proposicionejerc\endcsname}
\newtheorem{inner@proposicionejerc}[proposicion]{Proposicion}

\newenvironment{lemaejerc}
    {\pushQED{\qed}\renewcommand{\qedsymbol}{$\square$}\csname inner@lemaejerc\endcsname}
    {\popQED\csname endinner@lemaejerc\endcsname}
\newtheorem{inner@lemaejerc}[proposicion]{Lema}

\newtheorem{calculo}[proposicion]{Cálculo}

\theoremstyle{myplainnameless}
\newtheorem{nameless}[proposicion]{}

\theoremstyle{mydefinition}
\newtheorem{comentario}[proposicion]{Comentario}
\newtheorem{comentarioast}[proposicion]{Comentario ($\clubsuit$)}
\newtheorem{construccion}[proposicion]{Construcción}
\newtheorem{aplicacion}[proposicion]{Aplicación}
\newtheorem{definicion}[proposicion]{Definición}
\newtheorem{definicion-alternativa}[proposicion]{Definición alternativa}
\newtheorem{notacion}[proposicion]{Notación}
\newtheorem{advertencia}[proposicion]{Advertencia}
\newtheorem{digresion}[proposicion]{Digresión}
\newtheorem{ejemplox}[proposicion]{Ejemplo}
\newenvironment{ejemplo}
  {\pushQED{\qed}\renewcommand{\qedsymbol}{\examplesymbol}\ejemplox}
  {\popQED\endejemplox}
\newtheorem{contraejemplox}[proposicion]{Contraejemplo}
\newenvironment{contraejemplo}
  {\pushQED{\qed}\renewcommand{\qedsymbol}{\examplesymbol}\contraejemplox}
  {\popQED\endcontraejemplox}
\newtheorem{noejemplox}[proposicion]{No-ejemplo}
\newenvironment{noejemplo}
  {\pushQED{\qed}\renewcommand{\qedsymbol}{\examplesymbol}\noejemplox}
  {\popQED\endnoejemplox}
 
\newtheorem{ejemploastx}[proposicion]{Ejemplo ($\clubsuit$)}
\newenvironment{ejemploast}
  {\pushQED{\qed}\renewcommand{\qedsymbol}{\examplesymbol}\ejemploastx}
  {\popQED\endejemploastx}

\ifdefined\exercisespersection
  \theoremstyle{sectionexercise}
  \newtheorem{ejercicio}{}[section]
  \theoremstyle{mydefinition}
\else
  \ifdefined\exercisesglobal
    \theoremstyle{sectionexercise}
    \newtheorem{ejercicio}{}
    \theoremstyle{mydefinition}
  \else
    \ifdefined\exercisespersection
      \newtheorem{ejercicio}[proposicion]{Ejercicio}
    \fi
  \fi
\fi

% % % % % % % % % % % % % % % % % % % % % % % % % % % % % %

\theoremstyle{myplain}
\newtheorem{proposition}{Proposition}[section]
\newtheorem*{fact*}{Fact}
\newtheorem*{proposition*}{Proposition}
\newtheorem{lemma}[proposition]{Lemma}
\newtheorem*{lemma*}{Lemma}

\newtheorem{exercise}{Exercise}
\newtheorem*{hint}{Hint}

\newtheorem{theorem}[proposition]{Theorem}
\newtheorem{conjecture}[proposition]{Conjecture}
\newtheorem*{theorem*}{Theorem}
\newtheorem{corollary}[proposition]{Corollary}
\newtheorem{fact}[proposition]{Fact}
\newtheorem*{claim}{Claim}
\newtheorem{definition-theorem}[proposition]{Definition-theorem}

\theoremstyle{mydefinition}
\newtheorem{examplex}[proposition]{Example}
\newenvironment{example}
  {\pushQED{\qed}\renewcommand{\qedsymbol}{\examplesymbol}\examplex}
  {\popQED\endexamplex}

\newtheorem*{examplexx}{Example}
\newenvironment{example*}
  {\pushQED{\qed}\renewcommand{\qedsymbol}{\examplesymbol}\examplexx}
  {\popQED\endexamplexx}

\newtheorem{definition}[proposition]{Definition}
\newtheorem*{definition*}{Definition}
\newtheorem{wrong-definition}[proposition]{Wrong definition}
\newtheorem{remark}[proposition]{Remark}

\makeatletter
\newcommand{\xRightarrow}[2][]{\ext@arrow 0359\Rightarrowfill@{#1}{#2}}
\makeatother

% % % % % % % % % % % % % % % % % % % % % % % % % % % % % %

\newcommand{\Et}{\mathop{\text{\rm Ét}}}

\newcommand{\categ}[1]{\text{\bf #1}}
\newcommand{\vcateg}{\mathcal}
\newcommand{\bone}{{\boldsymbol 1}}
\newcommand{\bDelta}{{\boldsymbol\Delta}}
\newcommand{\bR}{{\mathbf{R}}}

\newcommand{\univ}{\mathfrak}

\newcommand{\TODO}{\colorbox{red}{\textbf{*** TODO ***}}}
\newcommand{\proofreadme}{\colorbox{red}{\textbf{*** NEEDS PROOFREADING ***}}}

\makeatletter
\def\iddots{\mathinner{\mkern1mu\raise\p@
\vbox{\kern7\p@\hbox{.}}\mkern2mu
\raise4\p@\hbox{.}\mkern2mu\raise7\p@\hbox{.}\mkern1mu}}
\makeatother

\newcommand{\ssincl}{\reflectbox{\rotatebox[origin=c]{45}{$\subseteq$}}}
\newcommand{\vsupseteq}{\reflectbox{\rotatebox[origin=c]{-90}{$\supseteq$}}}
\newcommand{\vin}{\reflectbox{\rotatebox[origin=c]{90}{$\in$}}}

\newcommand{\Ga}{\mathbb{G}_\mathrm{a}}
\newcommand{\Gm}{\mathbb{G}_\mathrm{m}}

\renewcommand{\U}{\UUU}

\DeclareRobustCommand{\Stirling}{\genfrac\{\}{0pt}{}}
\DeclareRobustCommand{\stirling}{\genfrac[]{0pt}{}}

% % % % % % % % % % % % % % % % % % % % % % % % % % % % % %
% tikz

\usepackage{tikz-cd}
\usetikzlibrary{babel}
\usetikzlibrary{decorations.pathmorphing}
\usetikzlibrary{arrows}
\usetikzlibrary{calc}
\usetikzlibrary{fit}
\usetikzlibrary{hobby}

% % % % % % % % % % % % % % % % % % % % % % % % % % % % % %
% Banners

\newcommand\mybannerext[3]{{\normalfont\sffamily\bfseries\large\noindent #1

\noindent #2

\noindent #3

}\noindent\rule{\textwidth}{1.25pt}

\vspace{1em}}

\newcommand\mybanner[2]{{\normalfont\sffamily\bfseries\large\noindent #1

\noindent #2

}\noindent\rule{\textwidth}{1.25pt}

\vspace{1em}}

\renewcommand{\O}{\mathcal{O}}


\numberwithin{equation}{section}

\usepackage[numbers]{natbib}

\usepackage{fullpage}

\usepackage{tocloft}

\renewcommand\cftsecfont{\normalfont}
\renewcommand\cfttoctitlefont{\sffamily\large\bfseries}
\renewcommand\cftsecpagefont{\normalfont}
\renewcommand{\cftsecleader}{\cftdotfill{\cftsecdotsep}}
\renewcommand\cftsecdotsep{\cftdot}
\renewcommand\cftsubsecdotsep{\cftdot}

\author{Alexey Beshenov (cadadr@gmail.com)}
\title{El lema de Hensel y sus aplicaciones}
\date{Universidad de El Salvador. Septiembre de 2018}

\usepackage{multicol}

\setlength{\columnseprule}{0.4pt}

\theoremstyle{definition}
\newtheorem{ejerc}{Ejercicio}

\newif\ifsolutions
% \solutionstrue
\solutionsfalse

\usepackage{multirow}

\begin{document}

{\normalfont\sffamily\bfseries \maketitle}

\section{Lema de Hensel}

Sea $K$ un cuerpo completo respecto a una norma no arquimediana
$\|\cdot\|$. Denotemos por
$$\O_K \dfn \{ x\in K \mid \|x\| \le 1 \}$$
el anillo de enteros correspondiente. Hay que tener en mente los siguientes dos
ejemplos principales.

\begin{enumerate}
\item[1)] Sobre los números racionales $\QQ$ consideremos la norma $p$-ádica
$$\left|\frac{m}{n}\right|_p \dfn 1/p^{v_p (m) - v_p (n)}.$$
La completación de $\QQ$ respecto a $\QQ_p$ es el cuerpo de los números
$p$-ádicos. El anillo $\ZZ_p\dfn \O_{\QQ_p}$ es el anillo de los enteros
$p$-ádicos.

\item[2)] Sea $k$ un cuerpo. Sobre el anillo de polinomios $k [X]$ definamos la
  valuación $X$-ádica mediante
  \[ v_X (0) \dfn \infty, \quad
     v_X \left(\sum_{i\ge 0} a_i\,X^i\right) \dfn \min \{ i \mid a_i \ne 0 \},
     \text{ si }\sum_{i\ge 0} a_i\,X^i \ne 0 \]
  y la norma correspondiente sobre el cuerpo de funciones racionales $k (X)$
  $$\left|\frac{f}{g}\right|_X \dfn \rho^{v_X (f) - v_X (g)}$$
  para algún parametro fijo $0 < \rho < 1$. La completación de $k (X)$ respecto
  a $|\cdot|_X$ es el cuerpo de las series de Laurent
  \[ k (\!(X)\!) = \Bigl\{ \sum_{i \ge -n} a_i\,X^i \Bigm|
                           a_i \in k, ~ n = 0,1,2,3,\ldots \Bigr\}, \]
  mientras que $\O_{k (\!(X)\!)}$ es el anillo de las series formales
  \[ k [\![X]\!] = \Bigl\{ \sum_{i \ge 0} a_i\,X^i \Bigm|
                           a_i \in k, ~ n = 0,1,2,3,\ldots \Bigr\}. \]
\end{enumerate}

Para cualquier cuerpo completo no arquimediano se cumple el siguiente resultado
importante.

\begin{teorema}[Lema de
  Hensel\footnote{\label{footnote:Hensel}\href{http://www-history.mcs.st-andrews.ac.uk/Biographies/Hensel.html}{\personality{Kurt
        Hensel}} (1861--1941), matemático alemán, estudiante de Kronecker y
    Weierstrass. Descubrió los números $p$-ádicos en 1897.}]
\label{teorema:Hensel}
Sea $f (X) \in \O_K [X]$ un polinomio con coeficientes en $\O_K$. Supongamos que
existe $x_0 \in \O_K$ tal que
$$\|f (x_0)\| < \|f' (x_0)\|^2.$$
Entonces existe único $x \in \O_K$ tal que $f (x) = 0$ y
$\|x-x_0\| < \|f' (x_0)\|$.
\end{teorema}

Antes de demostrar el lema de Hensel, necesitamos un par de sencillos lemas.

\begin{lema}
  \label{lema:Hensel-lema-2}
  Sea $h (X) \in \O_K [X]$ algún polinomio con coeficientes en $\O_K$. Entonces
  para cualesquiera $x,y\in \O_K$ se tiene
  $$\|h (x) - h (y)\| \le \|x-y\|.$$

  \begin{proof}
    Si $h (X) = \sum_{0 \le i \le d} a_i\,X^i$ para algunos $a_i\in \O_K$,
    entonces
    \[ h (x) - h (y) = \sum_{0 \le i \le d} a_i\,x^i - \sum_{0 \le i \le d}
       a_i\,y^i = \sum_{1 \le i \le d} a_i\,(x^i - y^i) = (x-y)\,\phi (x,y), \]
    donde $\phi (X,Y) \in \O_K [X,Y]$ es algún polinomio con coeficientes en
    $\O_K$. Luego,
    $$\|h (x) - h (y)\| = \|x-y\|\cdot \|\phi (x,y)\| \le \|x-y\|,$$
    ya que $\phi (x,y) \in \O_K$ y $\|\phi (x,y)\| \le 1$.
  \end{proof}
\end{lema}

\begin{lema}
  \label{lema:Hensel-lema-1}
  Sea $f (X) \in \O_K [X]$ algún polinomio con coeficientes en $\O_K$. Entonces
  para cualesquiera $x,y\in \O_K$
  $$f (x+y) = f (x) + f' (x)\,y + z\,y^2$$
  para algún $z\in \O_K$.

  \begin{proof}
    Si $f (X) = \sum_{0 \le i \le d} a_i\,X^i$, entonces la fórmula del binomio
    nos da
    \[ f (X+Y) =
       \sum_{0 \le i \le d} a_i\,(X+Y)^i =
       \sum_{0 \le i \le d} a_i\,X^i + \sum_{1 \le i \le d} a_i\,i\,X^{i-1}\,Y +
           \sum_{1 \le i \le d} a_i\,g_i (X,Y)\,Y^2, \]
    donde $g_i (X,Y) \in \ZZ [X,Y]$ son ciertos polinomios con coeficientes
    enteros. Notemos que $\sum_{1 \le i \le d} a_i\,i\,X^{i-1} = f' (X)$,
    y la identidad de arriba puede ser escrita como
    $$f (X+Y) = f (X) + f' (X)\,Y + g (X,Y)\,Y^2,$$
    donde $g (X,Y) \dfn \sum_{1 \le i \le d} a_i\,g_i (X,Y) \in \O_K [X,Y]$ es
    algún polinomio con coeficientes en $\O_K$.
  \end{proof}
\end{lema}

\begin{proof}[Demostración del lema de Hensel (el método de Newton $p$-ádico)]
  Notamos primero que si $\|x-x_0\| < \|f' (x_0)\|$, entonces según
  \ref{lema:Hensel-lema-2} se cumple
  $$\|f' (x) - f' (x_0)\| \le \|x - x_0\| < \|f' (x_0)\|,$$
  así que
  \[ \|f' (x)\| =
     \|f' (x) - f' (x_0) + f' (x_0)\| =
     \max \{ \|f' (x) - f' (x_0)\|, \|f' (x_0)\| \} = \|f' (x_0)\|. \]

  Demostremos que $f$ puede tener solo una raíz $x$ tal que
  $$\|x-x_0\| < \|f' (x_0)\|.$$
  Supongamos que para algunos $x, x' \in \O_K$ se cumple
  \[ f (x) = f (x') = 0, \quad
     \|x-x_0\| < \|f' (x_0)\|, \quad
     \|x'-x_0\| < \|f' (x_0)\|. \]
  Tenemos
  \[ \|x' - x\| = \|(x' - x_0) - (x - x_0)\| \le
     \max \{ \|x' - x_0\|, \|x - x_0\| \} < \|f' (x_0)\|. \]
  Escribamos $x' = x + y$. Según \ref{lema:Hensel-lema-1} tenemos
  $$0 = f (x') = f (x + y) = f (x) + f' (x)\,(x'-x) + z\,(x' - x)^2$$
  para algún $z \in \O_K$. Aquí $f (x) = 0$, entonces nos queda la identidad
  $$f' (x)\,y = -z\,y^2.$$
  Si $y \ne 0$, tenemos
  $$f' (x) = -z\,y,$$
  y en particular
  \[ \|f' (x)\| = \|z\|\cdot \|y\| \le \|y\| =
     \|x' - x\| < \|f' (x_0)\| = \|f' (x)\|, \]
  que es una contradicción. Entonces, la única opción es $y = 0$ y $x = x'$.

  \vspace{1em}

  Ahora para demostrar que la raíz $x$ existe, consideremos la sucesión de los
  elementos definidos a partir de $x_0$ por
  $$x_{n+1} \dfn x_n - \frac{f (x_n)}{f' (x_n)}.$$
  Note que es la misma fórmula que se usa en el análisis real en el método de
  Newton. Denotemos
  \[ \delta \dfn \left\|\frac{f (x_0)}{f' (x_0)^2}\right\| =
     \|f (x_0)\|\cdot \|f' (x_0)\|^{-2} < 1. \]
  Vamos a demostrar por inducción que los números $x_n$ satisfacen las
  siguientes propiedades:

  \begin{enumerate}
  \item[$1_n$)] $\|x_n\| \le 1$; es decir, $x_n \in \O_K$,
  \item[$2_n$)] $\|f' (x_n)\| = \|f' (x_0)\|$,
  \item[$3_n$)] $\|f (x_n)\| \le \|f' (x_0)\|^2 \, \delta^{2^n}$.
  \end{enumerate}

  \vspace{1em}

  Antes de verificar $1_n$), $2_n$), $3_n$), veamos por qué nuestra sucesión
  $\{ x_n \}$ demuestra el teorema. Notemos que las desigualdades $2_n$) y
  $3_n$) nos dan
  \begin{equation}
    \label{eqn:Hensel-proof-Cauchy-bound}
     \|x_{n+1} - x_n\| =
     \left\|\frac{f (x_n)}{f' (x_n)}\right\| =
     \frac{\|f (x_n)\|}{\|f' (x_0)\|} \le \|f' (x_0)\| \, \delta^{2^n},
  \end{equation}
  lo que implica que $\{ x_n \}$ es una sucesión de Cauchy:
  \[ \|x_m - x_n\| =
     \|(x_m - x_{m-1}) + (x_{m-1} - x_{m-2}) + \cdots + (x_{n+1} - x_n) \| \le
     \max \{ \|x_m - x_{m+1}\|, \ldots, \|x_{n+1} - x_n\| \}, \]
  y por lo tanto podemos pasar al límite
  $$x \dfn \lim_{n\to \infty} x_n.$$
  Ya que $x_n \in \O_K$ para todo $n$, se tiene $x \in \O_K$. Luego,
  $$\|f' (x)\| = \lim_{n\to \infty} \|f' (x_n)\| = \|f' (x_0)\|$$
  gracias a $2_n$) y
  \[ \|f (x)\| = \lim_{n\to \infty} \|f (x_n)\| \le
     \lim_{n\to \infty} \|f' (x_0)\|^2\,\delta^{2^n} = 0 \]
  gracias a $3_n$), y entonces $f (x) = 0$.

  \vspace{1em}

  Demostremos que
  $$\|x - x_0\| = \left\|\frac{f (x_0)}{f' (x_0)}\right\|.$$
  Será suficiente demostrar que para todo $n = 1,2,3,\ldots$ se cumple
  $$\|x_n - x_0\| = \left\|\frac{f (x_0)}{f' (x_0)}\right\|,$$
  y luego pasar al límite $n \to \infty$. Para $n = 1$ esto se cumple por
  la definición $x_1 \dfn x_0 - \frac{f (x_0)}{f' (x_0)}$. Luego, notemos que
  según \eqnref{eqn:Hensel-proof-Cauchy-bound}, tenemos la desigualdad estricta
  \[ \|x_{n+1} - x_n\| \le \|f' (x_0)\| \, \delta^{2^n} <
     \|f' (x_0)\| \, \delta =
     \left\|\frac{f (x_0)}{f' (x_0)}\right\|. \]

  Esto nos da el paso inductivo: si
  $\|x_n - x_0\| = \left\|\frac{f (x_0)}{f' (x_0)}\right\|$, entonces
  \[ \|x_{n+1} - x_0\| =
     \|(x_{n+1} - x_n) + (x_n - x_0)\| \le
     \max \{ \|x_{n+1} - x_n\|, \|x_n - x_0\| \} =
     \left\|\frac{f (x_0)}{f' (x_0)}\right\|. \]

  \vspace{1em}

  Para terminar la demostración, nos queda solo ver que los números $x_n$
  cumplen las propiedades $1_n$), $2_n$), $3_n$). Para $n = 0$, estas son
  nuestras hipótesis sobre $x_0$. Para el paso inductivo, tenemos que ver que
  $1_n$), $2_n$), $3_n$) implican $1_{n+1}$), $2_{n+1}$), $3_{n+1}$). Primero,
  $x_{n+1}$ está bien definido, ya que según $2_n$),
  $$\|f' (x_n)\| = \|f' (x_0)\| \ne 0,$$
  y por lo tanto $f' (x_n) \ne 0$ (en efecto, la hipótesis
  $\|f (x_0)\| < \|f' (x_0)\|^2$ implica que $\|f' (x_0)\| \ne 0$). Para
  demostrar $1_{n+1}$):
  \[ \|x_{n+1}\| = \left\|x_n - \frac{f (x_n)}{f' (x_n)}\right\| \le
     \max \{\|x_n\|, \left\|\frac{f (x_n)}{f' (x_n)}\right\| \} \stackrel{???}{\le} 1 \]
  es suficiente ver que $\left\|\frac{f (x_n)}{f' (x_n)}\right\| \le 1$. Esto
  se sigue de las propiedades $2_n$) y $3_n$):
  \[ \left\|\frac{f (x_n)}{f' (x_n)}\right\| =
     \left\|\frac{f (x_n)}{f' (x_0)}\right\| \le
     \|f' (x_0)\|\,\delta^{2^n} \le 1. \]

  Para demostrar $2_{n+1}$), notemos que
  $\|f (x_n)\| \le \|f' (x_0)\|^2 \, \delta^{2^n}$ según $3_n$), y ya que
  $\delta < 1$, tenemos $\|f (x_n)\| < \|f' (x_0)\|^2$. La desigualdad
  \ref{lema:Hensel-lema-2} para $f' (X)$ nos da
  \[ \|f' (x_{n+1}) - f' (x_n)\| \le
     \|x_{n+1} - x_n\| =
     \left\|\frac{f (x_n)}{f' (x_n)}\right\| =
     \left\|\frac{f (x_n)}{f' (x_0)}\right\| <
     \|f' (x_0)\|. \]
  Pero la desigualdad estricta
  $$\|f' (x_{n+1}) - f' (x_n)\| < \max \{ \|f' (x_{n+1})\|, \|f' (x_n)\| \}$$
  es posible solo si $\|f' (x_{n+1})\| = \|f' (x_n)\|$. Esto demuestra
  $2_{n+1}$). Para ver $3_{n+1}$), podemos usar la identidad de
  \ref{lema:Hensel-lema-1} para $x_n$ y $-\frac{f (x_n)}{f' (x_n)}$:
  \[ f (x_{n+1}) =
     f \left(x_n - \frac{f (x_n)}{f' (x_n)}\right) =
     f (x_n) + f' (x_n)\,\left(- \frac{f (x_n)}{f' (x_n)}\right) + z\,\left(-\frac{f (x_n)}{f' (x_n)}\right)^2 =
     z\,\left(- \frac{f (x_n)}{f' (x_n)}\right)^2 \]
  para algún $z\in \O_K$ (es decir, $\|z\| \le 1$). Entonces, usando $2_n$) y la
  desigualdad $3_n$), tenemos
  \[ \|f (x_{n+1})\| \le
     \left\|\frac{f (x_n)}{f' (x_n)}\right\|^2 \le
     \frac{(\|f' (x_0)\|^2\,\delta^{2^n})^2}{\|f' (x_0)\|^2} =
     \|f' (x_0)\|^2\,\delta^{2^{n+1}}. \]
  Esto termina la demostración de $1_n$), $2_n$), $3_n$) para todo $n$.
\end{proof}

\begin{ejemplo}
  Recordemos el método de Newton del análisis real. Para resolver una ecuación
  $f (x) = 0$, se empieza por algún $x_0$ y luego se consideran las
  aproximaciones consecutivas
  $$x_{n+1} = x_n - \frac{f (x_n)}{f' (x_n)}.$$
  Geométricamente, $x_{n+1}$ es la intersección del eje $x$ con la tangente
  a $f (x)$ en $x = x_n$. Si la aproximación inicial $x_0$ es buena, los $x_n$
  convergen a una raíz de $f$. Por ejemplo, si $f (x) = x^2 - 2$, entonces
  $f' (x) = 2\,x$, y empezando por $x_0 = 1$, se obtiene
  \[ x_1 = 1.5, \quad
     x_2 = 1.416666\ldots, \quad
     x_3 = 1.414215\ldots, \quad
     x_4 = 1.414213\ldots, \quad
     \ldots \]
  lo que converge rápidamente a $\sqrt{2}$.

  \begin{framed}\small
\begin{verbatim}
? x = 1.0;
? for (i=0,6, printf ("x%d = %s\n", i,x); x = x - (x^2-2)/(2*x));
x0 = 1.0000000000000000000000000000000000000
x1 = 1.5000000000000000000000000000000000000
x2 = 1.4166666666666666666666666666666666667
x3 = 1.4142156862745098039215686274509803922
x4 = 1.4142135623746899106262955788901349101
x5 = 1.4142135623730950488016896235025302436
x6 = 1.4142135623730950488016887242096980786

? sqrt (2)                                                       
% =  1.4142135623730950488016887242096980786
\end{verbatim}
  \end{framed}

  Sin embargo, en algunos casos el método no converge. Por ejemplo,
  si $f (x) = x^3 - x$ y empezamos por $x_0 = 1/\sqrt{5}$, entonces

  \begin{align*}
    x_1 & = \frac{1}{\sqrt{5}} - \frac{(1/\sqrt{5})^3 - 1/\sqrt{5}}{3\,(1/\sqrt{5})^2 - 1} =
          \frac{1}{\sqrt{5}} - \frac{1}{\sqrt{5}}\cdot \frac{1/5 - 1}{3/5 - 1} =
          -\frac{1}{\sqrt{5}},\\
    x_2 & = -\frac{1}{\sqrt{5}} - \frac{(-1/\sqrt{5})^3 + 1/\sqrt{5}}{3\,(-1/\sqrt{5})^2 - 1} =
          -\frac{1}{\sqrt{5}} + \frac{1}{\sqrt{5}}\cdot \frac{1/5 - 1}{3/5 - 1} =
          \frac{1}{\sqrt{5}} = x_0,\\
    x_2 & = x_1 = -\frac{1}{\sqrt{5}},\\
        & \cdots
  \end{align*}

  \begin{center}
    \begin{tikzpicture}[x=4cm,y=4cm]
      \draw[domain=-1.1:1.1,samples=100,variable=\x] plot ({\x},{\x*\x*\x - \x});

      \draw ({1/sqrt(5)}, {-4/(5*sqrt(5))}) -- ({-1/sqrt(5)}, 0);
      \draw ({-1/sqrt(5)}, {4/(5*sqrt(5))}) -- ({1/sqrt(5)}, 0);
      \draw[dashed] ({-1/sqrt(5)}, {4/(5*sqrt(5))}) -- ({-1/sqrt(5)}, 0);
      \draw[dashed] ({1/sqrt(5)}, {-4/(5*sqrt(5))}) -- ({1/sqrt(5)}, 0);

      \draw ({1/sqrt(5)}, 0) node[above] {$x_0$};
      \draw ({-1/sqrt(5)}, 0) node[below] {$x_1$};

      \draw[->] (-1.5,0) -- (1.5,0) node[right] {$x$};
      \draw[->] (0,-0.8) -- (0,0.8) node[above] {$f (x)$};
    \end{tikzpicture}
  \end{center}

  En el caso no-arquimediano, el método de Newton siempre converge: la única
  condición para la aproximación inicial es $\|f (x_0)\| < \|f' (x_0)\|^2$.
\end{ejemplo}

La demostración de arriba nos da un algoritmo concreto para calcular
el resultado con una dada precisión. La desigualdad
$$\|x_{n+1} - x_n\| \le \|f' (x_0)\| \, \delta^{2^n}$$
de \eqnref{eqn:Hensel-proof-Cauchy-bound} implica que para todo $m > n$
se cumple
$$\|x_m - x_n\| \le \|f' (x_0)\| \, \delta^{2^n}$$
(use la desigualdad ultramétrica). Para $m \to \infty$ esto nos da
\[ \|x - x_n\| \le \|f' (x_0)\| \, \delta^{2^n} =
   \|f' (x_0)\| \cdot \left\|\frac{f (x_0)}{f' (x_0)^2}\right\|^{2^n}, \]
lo que significa que a cada paso la precisión por lo menos se dobla y
el lgoritmo es bastante eficaz.

\begin{ejemplo}
  \label{ejemplo:raiz-cuadrada-3-adica-de-7}
  Calculemos una raíz cuadrada $3$-ádica de $7$. Buscamos entonces las raíces
  del polinomio $f (X) = X^2 - 7$ en $\ZZ_3$. Módulo $3$ tenemos dos soluciones:
  $1^2 - 7 \equiv 0 \pmod{3}$ y $2^2 - 7 \equiv 0 \pmod{3}$. Consideremos, por
  ejemplo, $x_0 = 1$. Tenemos
  $$|f (x_0)|_3 = |-6|_3 = \frac{1}{3}, \quad |f' (x_0)|_3^2 = |2|_3^2 = 1,$$
  entonces la condición $|f (x_0)|_3 < |f' (x_0)|_3^2$ se cumple, y el lema de
  Hensel nos dice que existe único $x \in \ZZ_3$ tal que $f (x) = 0$ y
  $|x - x_0|_3 < 1$; es decir, $x \equiv x_0 \pmod{3}$. Además, la demostración
  nos da una sucesión específica $x_n$ tal que $x = \lim_{n\to\infty}
  x_n$. Calculemos algunos de estos $x_n$ por la fórmula
  $$x_{n+1} = x_n - \frac{f (x_n)}{f' (x_n)}.$$
  Tenemos
  \begin{align*}
    x_1 & = 1 - \frac{1^2 - 7}{2\cdot 1} = 4,\\
    x_2 & = 4 - \frac{4^2 - 7}{2\cdot 4} = \frac{23}{8},\\
    x_3 & = \frac{977}{368},\\
        & \ldots
  \end{align*}

  La expansión $3$-ádica de $x_3$ es dada por
  $$1 + 3 + 3^2 + 2\cdot 3^4 + 2\cdot 3^7 + 3^9 + 3^{10} + 2\cdot 3^{11} + 2\cdot 3^{13} + \cdots$$
  (esto se puede calcular en PARI/GP). Además, sabemos que el número de los
  dígitos $p$-ádicos correctos en las aproximaciones $x_n$ por lo menos se dobla
  a cada paso; es decir,
  $$v_p (x-x_n) \ge 2^n \iff x \equiv x_n \pmod{p^{2^n}}.$$
  En particular, en $x_3$ los dígitos hasta $a_7$ coinciden con los dígitos de
  la verdadera raíz cuadrada $x$ de $7$ tal que $x \equiv 1 \pmod{3}$. Tenemos
  $$x = 1 + 3 + 3^2 + 2\cdot 3^4 + 2\cdot 3^7 + \cdots$$
  Si empezamos por $x_0' = 2$, se obtiene otro número $x' \in \ZZ_p$ tal que
  $x'^2 = 7$ y $x' \equiv x_0' \pmod{3}$. Es la otra raíz cuadrada de $7$. Por
  supuesto, $x = -x'$:
  $$-x = 2 + 3 + 3^2 + 2\cdot 3^3 + 2\cdot 3^5 + 2\cdot 3^6 + 0\cdot 3^7 + \cdots$$

  \begin{framed}
    \noindent Podemos verificar nuestro cálculo en PARI/GP:

    {\small
\begin{verbatim}
? (1 + 3 + 3^2 + 2*3^4 + 2*3^7 + O (3^8))^2
% = 1 + 2*3 + O(3^8)

? (2 + 3 + 3^2 + 2*3^3 + 2*3^5 + 2*3^6 + O(3^8))^2
% = 1 + 2*3 + O(3^8)
\end{verbatim}}

      \noindent PARI/GP puede calcular raíces cuadradas de números $p$-ádicos directamente:

      {\small
\begin{verbatim}
? sqrt (7 + O (3^10))
% = 1 + 3 + 3^2 + 2*3^4 + 2*3^7 + 3^8 + 3^9 + O(3^10)
\end{verbatim}}
      \end{framed}
\end{ejemplo}

Vamos a escribir simplemente ``$\sqrt{x}$'' para denotar una raíz cuadrada de
$x\in\QQ_p$. Esta notación viene del análisis real, donde $\sqrt{x}$ para
$x > 0$ normalmente denota el número \emph{positivo} tal que
$(\sqrt{x})^2 = x$. En el caso $p$-ádico, también hay dos posibilidades,
y la diferencia es el signo $\pm 1$, pero ya no hay un modo tan canónico de
elegir uno. Por ejemplo, la función \texttt{sqrt($x$)} en PARI/GP devuelve la
raíz cuadrada con el primer dígito $p$-ádico $0 \le a_0 \le p/2$.

\begin{framed}
  \noindent En PARI/GP, la función \verb|padicappr(f,a)| devuelve las raíces
  $p$-ádicas del polinomio $f$ congruentes a $a$ módulo $p$.

  {\small
\begin{verbatim}
? padicappr(x^2-7, 1 + O (3^10))
% = [1 + 3 + 3^2 + 2*3^4 + 2*3^7 + 3^8 + 3^9 + O(3^10)]~
? padicappr(x^2-7, 2 + O (3^10))
% = [2 + 3 + 3^2 + 2*3^3 + 2*3^5 + 2*3^6 + 3^8 + 3^9 + O(3^10)]~
\end{verbatim}}
\end{framed}

\begin{ejerc}
  Usando el método de Newton, calcule los primeros $8$ dígitos de las raíces
  cuadradas $\pm\sqrt{-1}$ en $\QQ_5$ y $\pm\sqrt{-3}$ en $\QQ_7$.
  (Use PARI/GP para hacer cálculos con números racionales y encontrar sus
  expansiones $p$-ádicas).
\end{ejerc}

\begin{ejemplo}
  \label{ejemplo:raiz-cuadrada-de-1+X}
  Calculemos la raíz cuadrada $\sqrt{1+X}$ en $\QQ [\![X]\!]$; es decir,
  encontremos las raíces del polinomio
  $F (Z) = Z^2 - 1 - X \in \QQ [\![X]\!] [Z]$. Para $f_0 = 1$ tenemos
  $$v_X (F (1)) = v_X (-X) = 1, \quad v_X (F' (1)) = v_X (2) = 0.$$
  Entonces, se cumple $|F (1)|_X < |F' (1)|_X^2$ y podemos calcular las
  aproximaciones
  $$f_{n+1} = f_n - \frac{F (f_n)}{F' (f_n)} = f_n - \frac{f_n^2 - 1 - X}{2\,f_n}.$$

  \pagebreak

  Esta vez sería mejor hacerlo con ayuda de PARI/GP:

  \begin{framed}\small
\begin{verbatim}
? f = 1;
? for (n=1,3, f = f - (f^2-1-X)/(2*f); printf ("f_%d = %s\n", n,f));
f_1 = 1/2*X + 1
f_2 = (X^2 + 8*X + 8)/(4*X + 8)
f_3 = (X^4 + 32*X^3 + 160*X^2 + 256*X + 128)/(8*X^3 + 80*X^2 + 192*X + 128)

? f + O (X^8)
% = 1 + 1/2*X - 1/8*X^2 + 1/16*X^3 - 5/128*X^4 + 7/256*X^5 - 21/1024*X^6
+ 33/2048*X^7 + O(X^8)
\end{verbatim}
  \end{framed}

  Como sabemos, los coeficientes de $f_3$ coinciden con los coeficientes de
  $\sqrt{X+1} = \sum_{i \ge 0} a_i\,X^i$ por lo menos hasta $a_7$. Entonces,
  \[ \sqrt{X+1} = 1 + \frac{1}{2}\,X - \frac{1}{8}\,X^2 + \frac{1}{16}\,X^3 -
                  \frac{5}{128}\,X^4 + \frac{7}{256}\,X^5 - \frac{21}{1024}\,X^6 +
                  \frac{33}{2048}\,X^7 + \cdots\]
  Lo que acabamos de calcular son los primeros coeficientes de la serie
  binomial
  $$(1+X)^{1/m} = \sum_{i \ge 0} {1/m \choose i}\,X^i,$$
  donde
  \[ {Y \choose i} \dfn \frac{Y\,(Y-1)\cdots (Y-i+1)}{i!} \in \QQ [Y]. \qedhere \]
\end{ejemplo}

\begin{framed}\small
\begin{verbatim}
? (1+X)^(1/2) + O (X^8)
% = 1 + 1/2*X - 1/8*X^2 + 1/16*X^3 - 5/128*X^4 + 7/256*X^5 - 21/1024*X^6
+ 33/2048*X^7 + O(X^8)

? vector (7,i,binomial(1/2,i))
% = [1/2, -1/8, 1/16, -5/128, 7/256, -21/1024, 33/2048]
\end{verbatim}
\end{framed}

El resto de este texto está dedicado a algunas aplicaciones típicas del lema de
Hensel para los números $p$-ádicos.

% % % % % % % % % % % % % % % % % % % % % % % % % % % % % %

\section{Cuadrados en $\QQ_p$}

\begin{proposicion}
  \label{prop:cuadrados-en-Zp}
  ~

  \begin{enumerate}
  \item[1)] Para $p \ne 2$, un número $u \in \ZZ_p^\times$ es un cuadrado en
    $\QQ_p$ (es decir, $u = x^2$ para algún $x \in \QQ_p$) si y solamente si $u$
    es un cuadrado módulo $p$.

  \item[2)] Un número $u\in\ZZ_2^\times$ es un cuadrado en $\QQ_2$ si y solamente
    si $u \equiv 1 \pmod{8}$.
  \end{enumerate}

  \begin{proof}
    Supongamos que $p\ne 2$. Si $u = x^2$ en $\QQ_p$, notamos que
    $|x|_p^2 = |u|_p = 1$, y por lo tanto $x \in \ZZ_p^\times$. Luego, se tiene
    $u \equiv x^2 \pmod{p}$. En la otra dirección, si la ecuación
    $f (X) = X^2 - u$ tiene solución módulo $p$, esto significa que existe
    un número $0 \le x_0 \le p-1$ tal que $x_0^2 - u \equiv 0 \pmod{p}$.
    En términos de normas, $|f (x_0)|_p \le 1/p$, mientras que
    $|f' (x_0)^2|_p = |2\,x_0|^2_p = 1$, puesto que $p \ne 2$. Entonces, podemos
    aplicar el lema de Hensel que nos da un elemento $x\in \ZZ_p$ tal que
    $f (x) = x^2 - u = 0$.

    En el caso de $p = 2$, de la misma manera, $u = x^2$ para algún $x\in\QQ_2$
    implica $x\in \ZZ_2^\times$ y podemos considerar la reducción módulo $8$.
    En el anillo $\ZZ_2/8\ZZ_2 \isom \ZZ/8\ZZ$, los elementos invertibles son
    $1, 3, 5, 7$, y todos sus cuadrados son congruentes a $1$ módulo $8$:
    $$1^2 \equiv 3^2 \equiv 5^2 \equiv 7^2 \equiv 1 \pmod{8}.$$
    Entonces, se tiene necesariamente $x^2 \equiv 1 \pmod{8}$. En la otra
    dirección, si $u\in\ZZ_2^\times$ satisface $u \equiv 1 \pmod{8}$, esto quiere
    decir que $|1-u|_2 \le 1/8$.  Apliquemos el lema de Hensel al polinomio
    $f (X) = X^2 - u$. Para $x_0 = 1$ tenemos
    $$|f (x_0)|_2 = |1-u|_2 \le 1/8 < |f' (x_0)|_2^2 = |2|_2^2 = 1/4,$$
    así que existe $x\in\ZZ_2$ tal que $u = x^2$.
  \end{proof}
\end{proposicion}

Hemos encontrado los cuadrados en $\ZZ_p^\times$, pero ¿qué sucede con
los cuadrados en $\QQ_p$? Pues, todo elemento $x\in \QQ_p^\times$ puede ser
escrito como $x = p^n\,u$, donde $n\in\ZZ$ y $u\in\ZZ_p^\times$. Luego,
$|x|_p = |p^n\,u|_p = 1/p^n$. Note que si $x$ es un cuadrado, entonces $n$ tiene
que ser par. Esto nos dice que los cuadrados en $\QQ_p^\times$ son precisamente
los números $p^n\,u$ donde $n\in\ZZ$ es par y $u$ es un cuadrado en
$\ZZ_p^\times$.

\begin{ejerc}
  Denotemos por $(\QQ_p^\times)^2$ el grupo multiplicativo de cuadrados en
  $\QQ_p^\times$. Demuestre que
  \begin{align*}
    \RR^\times / (\RR^\times)^2 & \isom C_2,\\
    \QQ_p^\times / (\QQ_p^\times)^2 & \isom C_2\times C_2, \quad\text{si } p\ne 2,\\
    \QQ_2^\times / (\QQ_2^\times)^2 & \isom C_2\times C_2\times C_2.
  \end{align*}

  \ifsolutions\begin{solucion}
    Para $p = \infty$ tenemos $\QQ_\infty = \RR$ y un número $x \in \RR^\times$
    es un cuadrado si y solamente si $x > 0$. Así que
    $$[\RR^\times : (\RR^\times)^2] = 2.$$

    Luego, para $p < \infty$ tenemos
    $$\QQ_p^\times \isom \left<p\right>\times\ZZ_p^\times,$$
    donde $\left<p\right> = \{ p^n \mid n\in\ZZ \}$ es el grupo cíclico generado
    por $p$. En efecto, todo número $x\in\QQ_p^\times$ puede ser escrito de modo
    único como $x = p^m\,u$ donde $u \in \ZZ_p^\times$ y $m\in\ZZ$. Es un
    cuadrado si y solamente si $m$ es par y $u$ es un cuadrado en
    $\ZZ_p^\times$. Entonces,
    $$\QQ_p^\times/(\QQ_p^\times)^2 \isom C_2\times \ZZ_p^\times/(\ZZ_p^\times)^2.$$
    Para $p \ne 2$, hemos visto en \ref{prop:cuadrados-en-Zp} que
    $u\in\ZZ_p^\times$ es un cuadrado en $\QQ_p$ si y solamente si es un
    cuadrado módulo $p$. Entonces,
    \[ [\ZZ_p^\times : (\ZZ_p^\times)^2] =
       [(\ZZ/p\ZZ)^\times : ((\ZZ/p\ZZ)^\times)^2] = 2. \]
    Para $p = 2$, hemos visto en \ref{aplicacion:cuadrados-en-Z2} que
    $u\in\ZZ_2^\times$ es un cuadrado si y solamente si $u\equiv
    1\pmod{8}$. Tenemos $(\ZZ/8\ZZ)^\times \isom C_2\times C_2$, así que
    $$\ZZ_2^\times / (\ZZ_2^\times)^2 \isom C_2\times C_2.$$
  \end{solucion}\fi
\end{ejerc}

\begin{ejerc}
  Demuestre que si $p \equiv 2 \pmod{3}$, entonces para todo $a\in\ZZ$ tal que
  $p\nmid a$ tenemos $\sqrt[3]{a} \in \QQ_p$. Para $p\not\equiv 2 \pmod{3}$,
  encuentre algún $a$ tal que $p\nmid a$ y $\sqrt[3]{a} \notin \QQ_p$ .

  \ifsolutions\begin{solucion}
    De nuevo, el lema de Hensel nos dice que las raíces cúbicas en
    $\ZZ_p^\times$ pueden ser levantadas de $\FF_p^\times$. La hipótesis
    $p \equiv 2 \pmod{3}$ implica que $3 \nmid (p-1) = \#\FF_p^\times$, así que
    para todo generador $a_0 \in \FF_p^\times$ el elemento $a_0^3$ es también
    un generador de $\FF_p^\times$, y $a \equiv a_0^{3n} = (a_0^n)^3 \pmod{p}$.

    Para $p = 7$ los cubos en $\FF_7^\times$ son $1$ y $6$. Por ejemplo,
    $\sqrt[3]{2} \notin \QQ_7$.
  \end{solucion}\fi
\end{ejerc}

El lector que todavía empieza a estudiar los números $p$-ádicos probablemente
se había planteado la siguiente pregunta. El cuerpo $\QQ_p$ contiene el cuerpo
$\QQ$ donde existe la noción de números positivos y negativos. ¿Se puede definir
algo parecido para $\QQ_p$?

Recordemos la siguiente definición. Se dice que un cuerpo $F$ es \term{ordenado}
si está definido un subconjunto $P\subset F$ de \term{elementos positivos} con
las siguientes propiedades:

\begin{enumerate}
\item[1)] Para $x\in F$ se cumple precisamente una de las siguientes relaciones:
$$x\in P, \quad x = 0, \quad -x\in P.$$

\item[2)] Si $x,y\in P$, entonces $x+y \in P$ y $xy\in P$.
\end{enumerate}

Normalmente si $x\ne 0$ y $x\in P$, se escribe ``$x > 0$''; si $x\ne 0$
y $-x\in P$, se escribe ``$x < 0$''. De los axiomas se sigue que $1 > 0$
y $-1 < 0$: de hecho, $1 = 1^2 = (-1)^2$. Luego, la propiedad 2) implica que
para todo $n = 1,2,3,\ldots$ tenemos
$$\underbrace{1 + \cdots + 1}_n > 0, \quad -\underbrace{(1 + \cdots + 1)}_n < 0.$$
En particular, un cuerpo de característica positiva no puede ser ordenado.
Para $x\ne 0$ tenemos $x^2 = (-x)^2$, lo que significa que $x^2 > 0$ para todo
$x\ne 0$: \emph{todos los cuadrados de números no nulos tienen que ser
  positivos}.

Por ejemplo, $\QQ$ y $\RR$ son cuerpos ordenados. El cuerpo $\CC$ no es
ordenado: tenemos $i^2 = -1 < 0$, un cuadrado que no es positivo. De la misma
manera, se puede ver que los cuerpos $p$-ádicos $\QQ_p$ no son ordenados, ya que
en cada uno de ellos se puede encontrar muchos cuadrados negativos.

\begin{observacion}
  $\QQ_p$ no puede ser ordenado para ningún primo finito $p$.

  \begin{proof}
    Por ejemplo, el cuerpo $\QQ_2$ contiene la raíz cuadrada $\pm\sqrt{-7}$,
    y luego $(\sqrt{-7})^2 = -7 < 0$. Para $p > 2$, el cuerpo $\QQ_p$ contiene,
    por ejemplo, $\pm\sqrt{1-p}$, y luego $(\sqrt{1-p})^2 = 1-p < 0$.
  \end{proof}
\end{observacion}

% % % % % % % % % % % % % % % % % % % % % % % % % % % % % %

\section{Raíces de la unidad en $\QQ_p$}

Una \term{raíz $n$-ésima de la unidad} es un número $\zeta \in \QQ_p$ tal que
$\zeta^n = 1$. Notamos que esto implica que $|\zeta|_p = 1$, así que las raíces
de la unidad en $\QQ_p$ forman un subgrupo $\mu (\QQ_p)$ del grupo
multiplicativo $\ZZ_p^\times$. La reducción módulo $p$ nos da un homomorfismo
de grupos
\begin{equation}
  \label{eqn:reduccion-mod-p-de-raices-de-unidad}
  \phi\colon \mu (\QQ_p) \hookrightarrow \ZZ_p^\times \epi (\ZZ/p\ZZ)^\times.
\end{equation}

\begin{proposicion}
  El homomorfismo $\phi$ es sobreyectivo; específicamente, en $\QQ_p$ hay raíces
  de orden $p-1$ que dan diferentes restos módulo $p$.

  \begin{proof}
    Consideremos el polinomio
    $$f (X) = X^p - X = X\,(X^{p-1} - 1) \in \ZZ_p [X]$$
    cuyas raíces no nulas corresponden a las raíces de la unidad de orden
    $p-1$. Según el pequeño teorema de Fermat la ecuación
    $$x_0^p - x_0 \equiv 0 \pmod{p}$$
    tiene $p$ soluciones $0 \le x_0 \le p-1$. Tenemos
    $$|f (x_0)|_p \le 1/p < |f' (x_0)|_p^2 = |p\,x_0^{p-1} - 1|_p^2 = 1,$$
    así que el lema de Hensel funciona y para cada $x_0$ produce un elemento
    único $x\in\ZZ_p$ que satisface $x^p - x = 0$ y $x \equiv x_0 \pmod{p}$.
  \end{proof}
\end{proposicion}

\begin{ejerc}
  He aquí otro modo de encontrar las raíces de la unidad de orden
  $p-1$. Demuestre que para todo $x\in\ZZ_p$ se cumple
  $$x^{p^{n+1}} \equiv x^{p^n} \pmod{p^{n+1}}$$
  y que el límite $\lim_{n\to\infty} x^{p^n}$ existe, y es precisamente la raíz
  del polinomio $X^p - X$ que es congruente a $x$ modulo $p$.

  \ifsolutions\begin{solucion}
    Primero, por el pequeño teorema de Fermat,
    $$x^p \equiv x \pmod{p}.$$
    Esto implica que
    $$x^{p^{n+1}} = (x^p)^{p^n} \equiv x^{p^n} \pmod{p^{n+1}}$$
    para todo $n = 1,2,3,\ldots$ En términos de normas
    $p$-ádicas, tenemos desigualdades
    $$|x^{p^{n+1}} - x^{p^n}|_p \le \frac{1}{p^{n+1}},$$
    lo que demuestra que la sucesión es de Cauchy y su límite
    $$y = \lim_{n\to\infty} x^{p^n}$$
    existe. Tenemos $y^n = y$. En efecto,
    \[ y^p - y =
       \left(\lim_{n\to\infty} x^{p^n}\right)^p - \lim_{n\to\infty} x^{p^n} =
       \lim_{n\to\infty} (x^{p^n})^p - \lim_{n\to\infty} x^{p^n} =
       \lim_{n\to\infty} (x^{p^{n+1}} - x^{p^n}) = 0 \]
    por lo que acabamos de ver. Ahora para todo $n$
    $$x^{p^n} \equiv x \pmod{p}.$$
    ---si $n = 1$, es el pequeño teorema de Fermat, y luego por inducción
    $$x^{p^n} = (x^{p^{n-1}})^p \equiv x^p \equiv x \pmod{p}.$$
    Así que
    $$y \equiv x \pmod{p}.$$
  \end{solucion}\fi
\end{ejerc}

En $\QQ_p$ no hay otras raíces de la unidad.

\begin{proposicion}
  Las únicas raíces de la unidad en $\QQ_2$ son $\pm 1$.

  Las únicas raíces de la unidad en $\QQ_p$ para $p \ne 2$ son las
  $(p-1)$-ésimas raíces que acabamos de encontrar.

  \begin{proof}
    \noindent\textbf{Primero examinemos el caso $p = 2$}. Toda raíz de la unidad
    $\zeta \in \mu (\QQ_2)$ es un producto de una raíz de la unidad de orden
    $2^k$ y una raíz de la unidad de orden impar $n$, así que será suficiente
    considerar las raíces de orden $2^k$ y orden impar por separado.

    La raíz de la unidad primitiva de orden $2$ es $-1$, pero $-1$ no es un
    cuadrado en $\ZZ_2^\times$, así que en $\ZZ_2^\times$ no hay raíces
    de la unidad de orden $2^k$ para $k \ge 2$.

    Ahora sea $\zeta \in \mu (\QQ_2)$ una raíz de la unidad tal que
    $\zeta^n = 1$ para $n$ impar. Consideremos el polinomio
    $f (X) = X^n - 1$. Tenemos $f (1) = 0$ y $f' (1) = n \not\equiv 0 \pmod{2}$,
    y entonces el lema de Hensel nos dice que existe \emph{único} $x \in \ZZ_2$
    tal que $f (x) = 0$ y $x \equiv 1 \pmod{2}$. Pero \emph{todo} elemento
    de $\ZZ_2^\times$ se reduce a $1$ módulo $2$, así que $\zeta = 1$. Esto
    demuestra que en $\ZZ_2^\times$ no hay raíces de la unidad no triviales
    de orden impar.

    \vspace{1em}

    \noindent\textbf{Ahora examinemos el caso $p \ne 2$}. Hemos probado usando
    el lema de Hensel que el homomorfismo
    $\phi\colon \mu (\QQ_p) \to (\ZZ/p\ZZ)^\times$
    de \eqnref{eqn:reduccion-mod-p-de-raices-de-unidad} es sobreyectivo, y hay
    que demostrar que $\phi$ es también inyectivo. Sea $\zeta \in \mu (\QQ_p)$
    una $n$-ésima raíz de la unidad tal que $\phi (\zeta) = 1$; a saber,
    $\zeta = 1 + p\,x$ para algún $x\in\ZZ_p$. Necesitamos ver que
    $x = 0$. Tenemos
    $$\zeta^n = (1 + p\,x)^n = 1,$$
    y entonces, aplicando el teorema del binomio,
    $$n\,p\,x + {n\choose 2}\,p^2\,x^2 + {n\choose 3}\,p^3\,x^3 + \cdots + p^n\,x^n = 0.$$
    Luego,
    $$x\,\left(n + {n\choose 2}\,p\,x + {n\choose 3}\,p^2\,x^2 + \cdots + p^{n-1}\,x^{n-1}\right) = 0.$$
    Si $p\nmid n$, entonces la expresión en paréntesis no puede ser nula y
    $x = 0$. Si $p\mid n$, podemos reemplazar $\zeta$ por $\zeta^p$ y $n$ por
    $n/p$. De nuevo, el mismo argumento demuestra que $x = 0$ o $p\mid
    n$. Repitiendo este proceso, todo se reduce al caso $n = p$. Tenemos
    entonces
    $$x\,\left(p + {p\choose 2}\,p\,x + {p\choose 3}\,p^2\,x^2 + \cdots + p^{p-1}\,x^{p-1}\right) = 0.$$
    Sin embargo, $p \ne 2$, así que todos los términos en la suma
    $${p\choose 2}\,p\,x + {p\choose 3}\,p^2\,x^2 + \cdots + p^{p-1}\,x^{p-1}$$
    son divisibles por $p^2$, la expresión en paréntesis no puede ser nula y $x = 0$.
  \end{proof}
\end{proposicion}

\begin{corolario}
  Los cuerpos $\QQ_p$ no son isomorfos para diferentes $p$.

  \begin{proof}
    El orden del grupo de las raíces de la unidad nos permite distinguir todos
    los $\QQ_p$, excepto $\QQ_2$ y $\QQ_3$, donde las raíces de la unidad son
    $\pm 1$. En este caso excepcional podemos notar, por ejemplo, que
    $\sqrt{-2} \in \QQ_3$, mientras que $\sqrt{-2} \notin \QQ_2$.
  \end{proof}
\end{corolario}

\begin{ejerc}
  Sea $p$ un número primo.

  \begin{enumerate}
  \item[1)] Sea $n$ un número entero, posiblemente negativo. Demuestre que si
    $p\nmid n$ y $x\in\ZZ_p$ satisface $x\equiv 1\pmod{p}$, entonces $x$ tiene
    una raíz $n$-ésima: existe $y$ tal que $y^n = x$.

  \item[2)] Demuestre que $1 + p$ no tiene raíces $p$-ésimas en $\QQ_p$.

  \item[3)] Demuestre que $x\in\ZZ_p$ tiene una raíz $p$-ésima si
    $x \equiv 1 \pmod{p^2}$ y $p \ne 2$.
  \end{enumerate}

  \noindent \emph{Indicación: use el lema de Hensel en 1). En 2), si
    $y = b_0 + b_1\,p + b_2\,p^2 + \cdots$, calcule los primeros dígitos
    $p$-ádicos de $y^p$. En 3), use el lema de Hensel con una buena aproximación
    inicial.}

  \ifsolutions\begin{solucion}
    En 1), para $n > 0$, podemos aplicar el lema de Hensel al polinomio
    $f (X) = X^n - x$. Su derivada es $f' (X) = n\,X^{n-1}$. Tenemos
    $f (1) \equiv 1^n - 1 = 0 \pmod{p}$ y $f' (1) = n \not\equiv 0 \pmod{p}$ por
    nuestra hipótesis. Entonces, existe único $y\in\ZZ_p$ tal que $y^n = x$ y
    $y\equiv 1\pmod{p}$. Para $n < 0$, podemos aplicar el lema de Hensel
    al polinomio $f (X) = x\,X^{-n} - 1$. Tenemos $f (1) \equiv 0 \pmod{p}$ y
    $f' (1) \equiv -n \not\equiv 0 \pmod{p}$.

    En 2), notemos que si
    $y = b_0 + b_1\,p + b_2\,p^2 + b_3\,p^3 + \cdots\in\ZZ_p$, tenemos
    $$y^p = b_0^p + \left(\sum_{i_1 + i_2 + \cdots + i_p = 1} b_{i_1} \, b_{i_2} \cdots b_{i_p}\right)\,p + \cdots,$$
    pero el término
    $\sum\limits_{i_1 + i_2 + \cdots + i_p = 1} b_{i_1}\,b_{i_2}\cdots b_{i_p}$
    es igual a $p\cdot b_1\,b_0^{p-1}$, y entonces
    $y^p = b_0^p + O (p^2)$, así que

    \begin{align*}
      y^p & \equiv b_0^p \equiv b_0 \pmod{p},\\
      y^p & \equiv b_0^p \pmod{p^2}.
    \end{align*}
    Estas consideraciones demuestran en particular que para
    $x = 1 + p$ no existe $y\in\ZZ_p$ tal que $x = y^p$.

    En 3), para $x = 1 + a_2\,p^2 + a_3\,p^3 + \cdots$, podemos aplicar el lema
    de Hensel al polinomio $f (X) = X^p - x$. Para la approximation inicial
    $x_0 = 1 + a_2\,p$ tenemos
    \[ x_0^p =
       \sum_{0 \le k \le p} {p \choose k} \, a_2^k\,p^k =
       1 + {p \choose 1}\,a_2\,p + {p \choose 2}\,a_2^2\,p^2 + \cdots + a_2^p\,p^p. \]
    Ahora si $p > 2$ (!!!), entonces
    $$x_0^p = 1 + a_2\,p^2 + O (p^3),$$
    así que $x_0^p \equiv x \pmod{p^3}$; es decir,
    $|f (x_0)|_p \le \frac{1}{p^3}$. Luego,
    $$|f' (x_0)|_p = |p\,x_0^{p-1}|_p = |p|_p\cdot |x_0|_p^{p-1} = \frac{1}{p},$$
    y las hipótesis del lema de Hensel se cumplen:
    $$|f (x_0)|_p \le \frac{1}{p^3} < \frac{1}{p^2} = |f' (x_0)|_p^2,$$
    y por lo tanto existe único $y\in\ZZ_p$ tal que $y^p = x$ y
    $|y-x_0|_p < |f' (x_0)|_p$; es decir,
    $y \equiv x_0 \equiv 1 + a_2\,p \pmod{p^2}$.
  \end{solucion}\fi
\end{ejerc}

\begin{ejerc}
  He aquí otro modo de demostrar que $\QQ_p$ no tiene raíces $p$-ésimas
  de la unidad no triviales.

  \begin{enumerate}
  \item[1)] Recuerde el \term{criterio de Eisenstein} en la siguiente
    versión. Sea
    $$f (X) = a_n\,X^n + a_{n-1}\,X^{n-1} + \cdots + a_1\,X + a_0 \in R [X]$$
    un polinomio con coeficientes en un dominio de factorización única $R$. Sea
    $p\in R$ un elemento primo tal que

    \begin{itemize}
    \item $p\mid a_i$ para todo $i = 0,\ldots,n-1$,
    \item $p\mid a_n$,
    \item $p^2 \nmid a_0$.
    \end{itemize}

    Entonces $f (X)$ es irreducible en $K [X]$ donde $K$ es el cuerpo de
    fracciones de $R$. Es decir, $f (X)$ no puede ser expresado como un producto
    de dos polinomios de grado $< n$.

  \item[2)] Usando el criterio de Eisenstein para $\ZZ_p [X]$, demuestre que
    el polinomio ciclotómico
    $$f (X) = \frac{X^p - 1}{X - 1} = X^{p-1} + X^{p-2} + \cdots + X + 1$$
    es irreducible en $\QQ_p [X]$. Concluya que la única $p$-ésima raíz
    de la unidad en $\QQ_p$ es $1$.
  \end{enumerate}

  \noindent \emph{Indicación: en 2), considere el polinomio $f (X+1)$.}

  \ifsolutions\begin{solucion}
    Las $p$-ésimas raíces de la unidad son las raíces
    del polinomio
    \[ X^p - 1 = (X - 1)\,\frac{X^p - 1}{X - 1}
               = (X-1)\,(X^{p-1} + X^{p-2} + \cdots + X + 1), \]
    así que en $\QQ_p$ no hay raíces de la unidad no triviales precisamente
    cuando el polinomio
    $$f (X) = \frac{X^p - 1}{X - 1} = X^{p-1} + X^{p-2} + \cdots + X + 1$$
    es irreducible en $\QQ_p [X]$. El criterio de Eisenstein no se aplica
    a $f (X)$, pero se aplica a
    \[ f (X+1) = \frac{(X+1)^p - 1}{X}
               = \sum_{0 \le k \le p-1} {p\choose k+1}\,X^k. \]
    Efectivamente, $p\mid {p\choose k+1}$ para $k = 0,\ldots,p-2$,
    $p \nmid {p\choose p}$, y $p^2 \nmid {p\choose 1}$, así que $f (X+1)$ es
    irreducible y $f (X)$ es también irreducible.
  \end{solucion}\fi
\end{ejerc}

% % % % % % % % % % % % % % % % % % % % % % % % % % % % % %

\section{Automorfismos de $\QQ_p$}

\begin{proposicion}
  \label{aplicacion:automorfismos-de-Qp}
  El único automorfismo $f\colon \QQ_p\to\QQ_p$ es la aplicación identidad.
\end{proposicion}

\textbf{Primero notemos que todo automorfismo $f\colon \QQ_p\to\QQ_p$ deja $\QQ$
  fijo}. Primero, para todo $n\in\NN$
$$f (\pm n) = f (\pm \underbrace{1 + \cdots + 1}_n) = \pm n\,f (1) = \pm n.$$
Luego, para $m/n \in \QQ$ tenemos
$$n\,f \left(\frac{m}{n}\right) = f \left(n\,\frac{m}{n}\right) = f (m) = m,$$
así que $f (m/n) = m/n$.

\vspace{1em}

Se sigue que \term{el único automorfismo \underline{continuo}
  $f\colon \QQ_p\to\QQ_p$ es la aplicación identidad.} De hecho, ya que $\QQ$ es
denso en $\QQ_p$, todo elemento $x\in \QQ_p$ puede ser representado como
$x = \lim_{n\to\infty} x_n$ para $x_n\in \QQ$, y si $f$ es una aplicación
continua,
\[ f (x) =
   f \left(\lim_{n\to\infty} x_n\right) =
   \lim_{n\to\infty} f (x_n) = \lim_{n\to\infty} x_n = x, \]
puesto que $f$ deja $\QQ$ fijo.

\vspace{1em}

Entonces, para demostrar \ref{aplicacion:automorfismos-de-Qp}, sería suficiente
ver que todo automorfismo $f\colon \QQ_p\to\QQ_p$ es automáticamente
continuo. Para esto nos va a servir una caracterización algebraica del grupo
de unidades $\ZZ_p^\times$.

\begin{lema}
  Las siguientes condiciones son equivalentes para $x\in\QQ_p^\times$:

  \begin{enumerate}
  \item[1)] $x^{p-1}$ tiene $n$-ésimas raíces para un número infinito de $n$;

  \item[2)] $x\in\ZZ_p^\times$.
  \end{enumerate}

  \begin{proof}
    Si $x^{p-1} = y^n$ para algún $n$, entonces
    $$(p-1)\,v_p (x) = n\,v_p (y).$$
    Si esto se cumple para un número infinito de $n$, entonces las relaciones
    $n \mid (p-1)\,v_p (x)$ implican que $v_p (x) = 0$; es decir, si
    $x\in\ZZ_p^\times$.

    En la otra dirección, si $x\in\ZZ_p^\times$, tenemos
    $x\not\equiv 0 \pmod{p}$, y luego $x^{p-1} \equiv 1 \pmod{p}$ por el pequeño
    teorema de Fermat. Luego, para $f (X) = X^n - x^{p-1} \in \ZZ_p [X]$ tenemos
    $f (1) \equiv 0 \pmod{p}$ y $f' (1) \not\equiv 0$ si $p\nmid n$. El lema
    de Hensel nos da entonces $y\in\ZZ_p$ tal que $y^n = x^{p-1}$.
  \end{proof}
\end{lema}

El lema que acabamos de demostrar implica que
$f (\ZZ_p^\times) \subseteq \ZZ_p^\times$. Ahora todo $x\in\QQ_p^\times$ puede
ser escrito como $x = p^n\,u$, donde $n\in\ZZ$ y $u\in\ZZ_p^\times$. Tenemos
$$f (x) = f (p^n\,u) = f (p^n)\,f (u) = p^n\,f (u).$$
Luego,
$$v_p (f (x)) = v_p (p^n\,f (u)) = n = v_p (x).$$
Esto significa que para todo $x,y\in \QQ_p$
$$|f (x) - f (y)|_p = |f (x-y)|_p = |x-y|_p.$$
Entonces, $f$ es una aplicación continua, y esto termina la demostración. \qed

\begin{ejerc}
  Demuestre que el único automorfismo $f\colon \RR\to \RR$ es la aplicación
  identidad.

  \begin{enumerate}
  \item[1)] Note que $f (x^2) = f (x)^2$, así que $f$ aplica números positivos
    en números positivos y en general, preserva el orden:
    $$x \le y \Longrightarrow f (x) \le f (y).$$

  \item[2)] De nuevo, $f$ es identidad sobre $\QQ$. Demuestre que esto junto con
    preservación del orden implica que $f$ es identidad sobre todo $\RR$.
  \end{enumerate}

  Note que $\CC$ ya tiene un número infinito de automorfismos no contínuos.

  \ifsolutions\begin{solucion}
    $f$ preserva productos, entonces $f (x^2) = f (x)^2$ y $f$ aplica cuadrados
    a cuadrados. Pero $x$ es un cuadrado en $\RR$ si y solamente si $x\ge 0$,
    así que
    $$x\ge 0 \Longrightarrow f (x)\ge 0.$$
    Luego,
    \[ x \le y \iff y-x \ge 0 \iff f (y-x) =
       f (y) - f (x) \ge 0 \iff f (x) \le f (y), \]
    así que $f$ preserva el orden. Ahora para todo número real $z\in \RR$ y
    $\epsilon > 0$ podemos encontrar $x,y\in \QQ$ tales que $x \le z \le y$ y
    $y-x \le \epsilon$. Luego,
    $$x \le z \le y \Longrightarrow x \le f (z) \le y$$
    Entonces, $|f (z) - z| \le y-x \le \epsilon$. Esto implica que $f (z) = z$.
  \end{solucion}\fi
\end{ejerc}

\end{document}
