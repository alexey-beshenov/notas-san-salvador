\documentclass{article}

% TODO : CLEAN UP THIS MESS
% (AND MAKE SURE ALL TEXTS STILL COMPILE)
\usepackage[leqno]{amsmath}
\usepackage{amssymb}
\usepackage{graphicx}

\usepackage{diagbox} % table heads with diagonal lines
\usepackage{relsize}

\usepackage{wasysym}
\usepackage{scrextend}
\usepackage{epigraph}
\setlength\epigraphwidth{.6\textwidth}

\usepackage[utf8]{inputenc}

\usepackage{titlesec}
\titleformat{\chapter}[display]
  {\normalfont\sffamily\huge\bfseries}
  {\chaptertitlename\ \thechapter}{5pt}{\Huge}
\titleformat{\section}
  {\normalfont\sffamily\Large\bfseries}
  {\thesection}{1em}{}
\titleformat{\subsection}
  {\normalfont\sffamily\large\bfseries}
  {\thesubsection}{1em}{}
\titleformat{\part}[display]
  {\normalfont\sffamily\huge\bfseries}
  {\partname\ \thepart}{0pt}{\Huge}

\usepackage[T1]{fontenc}
\usepackage{fourier}
\usepackage{paratype}

\usepackage[symbol,perpage]{footmisc}

\usepackage{perpage}
\MakePerPage{footnote}

\usepackage{array}
\newcolumntype{x}[1]{>{\centering\hspace{0pt}}p{#1}}

% TODO: the following line causes conflict with new texlive (!)
% \usepackage[english,russian,polutonikogreek,spanish]{babel}
% \newcommand{\russian}[1]{{\selectlanguage{russian}#1}}

% Remove conflicting options for the moment:
\usepackage[english,polutonikogreek,spanish]{babel}

\AtBeginDocument{\shorthandoff{"}}
\newcommand{\greek}[1]{{\selectlanguage{polutonikogreek}#1}}

% % % % % % % % % % % % % % % % % % % % % % % % % % % % % %
% Limit/colimit symbols (with accented i: lím / colím)

\usepackage{etoolbox} % \patchcmd

\makeatletter
\patchcmd{\varlim@}{lim}{\lim}{}{}
\makeatother
\DeclareMathOperator*{\colim}{co{\lim}}
\newcommand{\dirlim}{\varinjlim}
\newcommand{\invlim}{\varprojlim}

% % % % % % % % % % % % % % % % % % % % % % % % % % % % % %

\usepackage[all,color]{xy}

\usepackage{pigpen}
\newcommand{\po}{\ar@{}[dr]|(.4){\text{\pigpenfont I}}}
\newcommand{\pb}{\ar@{}[dr]|(.3){\text{\pigpenfont A}}}
\newcommand{\polr}{\ar@{}[dr]|(.65){\text{\pigpenfont A}}}
\newcommand{\pour}{\ar@{}[ur]|(.65){\text{\pigpenfont G}}}
\newcommand{\hstar}{\mathop{\bigstar}}

\newcommand{\bigast}{\mathop{\Huge \mathlarger{\mathlarger{\ast}}}}

\newcommand{\term}{\textbf}

\usepackage{stmaryrd}

\usepackage{cancel}

\usepackage{tikzsymbols}

\newcommand{\open}{\underset{\mathrm{open}}{\hookrightarrow}}
\newcommand{\closed}{\underset{\mathrm{closed}}{\hookrightarrow}}

\newcommand{\tcol}[2]{{#1 \choose #2}}

\newcommand{\homot}{\simeq}
\newcommand{\isom}{\cong}
\newcommand{\cH}{\mathcal{H}}
\renewcommand{\hom}{\mathrm{hom}}
\renewcommand{\div}{\mathop{\mathrm{div}}}
\renewcommand{\Im}{\mathop{\mathrm{Im}}}
\renewcommand{\Re}{\mathop{\mathrm{Re}}}
\newcommand{\id}[1]{\mathrm{id}_{#1}}
\newcommand{\idid}{\mathrm{id}}

\newcommand{\ZG}{{\ZZ G}}
\newcommand{\ZH}{{\ZZ H}}

\newcommand{\quiso}{\simeq}

\newcommand{\personality}[1]{{\sc #1}}

\newcommand{\mono}{\rightarrowtail}
\newcommand{\epi}{\twoheadrightarrow}
\newcommand{\xepi}[1]{\xrightarrow{#1}\mathrel{\mkern-14mu}\rightarrow}

% % % % % % % % % % % % % % % % % % % % % % % % % % % % % %

\DeclareMathOperator{\Ad}{Ad}
\DeclareMathOperator{\Aff}{Aff}
\DeclareMathOperator{\Ann}{Ann}
\DeclareMathOperator{\Aut}{Aut}
\DeclareMathOperator{\Br}{Br}
\DeclareMathOperator{\CH}{CH}
\DeclareMathOperator{\Cl}{Cl}
\DeclareMathOperator{\Coeq}{Coeq}
\DeclareMathOperator{\Coind}{Coind}
\DeclareMathOperator{\Cop}{Cop}
\DeclareMathOperator{\Corr}{Corr}
\DeclareMathOperator{\Cor}{Cor}
\DeclareMathOperator{\Cov}{Cov}
\DeclareMathOperator{\Der}{Der}
\DeclareMathOperator{\Div}{Div}
\DeclareMathOperator{\D}{D}
\DeclareMathOperator{\Ehr}{Ehr}
\DeclareMathOperator{\End}{End}
\DeclareMathOperator{\Eq}{Eq}
\DeclareMathOperator{\Ext}{Ext}
\DeclareMathOperator{\Frac}{Frac}
\DeclareMathOperator{\Frob}{Frob}
\DeclareMathOperator{\Funct}{Funct}
\DeclareMathOperator{\Fun}{Fun}
\DeclareMathOperator{\GL}{GL}
\DeclareMathOperator{\Gal}{Gal}
\DeclareMathOperator{\Gr}{Gr}
\DeclareMathOperator{\Hol}{Hol}
\DeclareMathOperator{\Hom}{Hom}
\DeclareMathOperator{\Ho}{Ho}
\DeclareMathOperator{\Id}{Id}
\DeclareMathOperator{\Ind}{Ind}
\DeclareMathOperator{\Inn}{Inn}
\DeclareMathOperator{\Isom}{Isom}
\DeclareMathOperator{\Ker}{Ker}
\DeclareMathOperator{\Lan}{Lan}
\DeclareMathOperator{\Lie}{Lie}
\DeclareMathOperator{\Map}{Map}
\DeclareMathOperator{\Mat}{Mat}
\DeclareMathOperator{\Max}{Max}
\DeclareMathOperator{\Mor}{Mor}
\DeclareMathOperator{\Nat}{Nat}
\DeclareMathOperator{\Nrd}{Nrd}
\DeclareMathOperator{\Ob}{Ob}
\DeclareMathOperator{\Out}{Out}
\DeclareMathOperator{\PGL}{PGL}
\DeclareMathOperator{\PSL}{PSL}
\DeclareMathOperator{\PSU}{PSU}
\DeclareMathOperator{\Pic}{Pic}
\DeclareMathOperator{\RHom}{RHom}
\DeclareMathOperator{\Rad}{Rad}
\DeclareMathOperator{\Ran}{Ran}
\DeclareMathOperator{\Rep}{Rep}
\DeclareMathOperator{\Res}{Res}
\DeclareMathOperator{\SL}{SL}
\DeclareMathOperator{\SO}{SO}
\DeclareMathOperator{\SU}{SU}
\DeclareMathOperator{\Sh}{Sh}
\DeclareMathOperator{\Sing}{Sing}
\DeclareMathOperator{\Specm}{Specm}
\DeclareMathOperator{\Spec}{Spec}
\DeclareMathOperator{\Sp}{Sp}
\DeclareMathOperator{\Stab}{Stab}
\DeclareMathOperator{\Sym}{Sym}
\DeclareMathOperator{\Tors}{Tors}
\DeclareMathOperator{\Tor}{Tor}
\DeclareMathOperator{\Tot}{Tot}
\DeclareMathOperator{\UUU}{U}

\DeclareMathOperator{\adj}{adj}
\DeclareMathOperator{\ad}{ad}
\DeclareMathOperator{\af}{af}
\DeclareMathOperator{\card}{card}
\DeclareMathOperator{\cm}{cm}
\DeclareMathOperator{\codim}{codim}
\DeclareMathOperator{\cod}{cod}
\DeclareMathOperator{\coeq}{coeq}
\DeclareMathOperator{\coim}{coim}
\DeclareMathOperator{\coker}{coker}
\DeclareMathOperator{\cont}{cont}
\DeclareMathOperator{\conv}{conv}
\DeclareMathOperator{\cor}{cor}
\DeclareMathOperator{\depth}{depth}
\DeclareMathOperator{\diag}{diag}
\DeclareMathOperator{\diam}{diam}
\DeclareMathOperator{\dist}{dist}
\DeclareMathOperator{\dom}{dom}
\DeclareMathOperator{\eq}{eq}
\DeclareMathOperator{\ev}{ev}
\DeclareMathOperator{\ex}{ex}
\DeclareMathOperator{\fchar}{char}
\DeclareMathOperator{\fr}{fr}
\DeclareMathOperator{\gr}{gr}
\DeclareMathOperator{\im}{im}
\DeclareMathOperator{\infl}{inf}
\DeclareMathOperator{\interior}{int}
\DeclareMathOperator{\intrel}{intrel}
\DeclareMathOperator{\inv}{inv}
\DeclareMathOperator{\length}{length}
\DeclareMathOperator{\mcd}{mcd}
\DeclareMathOperator{\mcm}{mcm}
\DeclareMathOperator{\multideg}{multideg}
\DeclareMathOperator{\ord}{ord}
\DeclareMathOperator{\pr}{pr}
\DeclareMathOperator{\rel}{rel}
\DeclareMathOperator{\res}{res}
\DeclareMathOperator{\rkred}{rkred}
\DeclareMathOperator{\rkss}{rkss}
\DeclareMathOperator{\rk}{rk}
\DeclareMathOperator{\sgn}{sgn}
\DeclareMathOperator{\sk}{sk}
\DeclareMathOperator{\supp}{supp}
\DeclareMathOperator{\trdeg}{trdeg}
\DeclareMathOperator{\tr}{tr}
\DeclareMathOperator{\vol}{vol}

\newcommand{\iHom}{\underline{\Hom}}

\renewcommand{\AA}{\mathbb{A}}
\newcommand{\CC}{\mathbb{C}}
\renewcommand{\SS}{\mathbb{S}}
\newcommand{\TT}{\mathbb{T}}
\newcommand{\PP}{\mathbb{P}}
\newcommand{\BB}{\mathbb{B}}
\newcommand{\RR}{\mathbb{R}}
\newcommand{\ZZ}{\mathbb{Z}}
\newcommand{\FF}{\mathbb{F}}
\newcommand{\HH}{\mathbb{H}}
\newcommand{\NN}{\mathbb{N}}
\newcommand{\QQ}{\mathbb{Q}}
\newcommand{\KK}{\mathbb{K}}

% % % % % % % % % % % % % % % % % % % % % % % % % % % % % %

\usepackage{amsthm}

\newcommand{\legendre}[2]{\left(\frac{#1}{#2}\right)}

\newcommand{\examplesymbol}{$\blacktriangle$}
\renewcommand{\qedsymbol}{$\blacksquare$}

\newcommand{\dfn}{\mathrel{\mathop:}=}
\newcommand{\rdfn}{=\mathrel{\mathop:}}

\usepackage{xcolor}
\definecolor{mylinkcolor}{rgb}{0.0,0.4,1.0}
\definecolor{mycitecolor}{rgb}{0.0,0.4,1.0}
\definecolor{shadecolor}{rgb}{0.79,0.78,0.65}
\definecolor{gray}{rgb}{0.6,0.6,0.6}

\usepackage{colortbl}

\definecolor{myred}{rgb}{0.7,0.0,0.0}
\definecolor{mygreen}{rgb}{0.0,0.7,0.0}
\definecolor{myblue}{rgb}{0.0,0.0,0.7}

\definecolor{redshade}{rgb}{0.9,0.5,0.5}
\definecolor{greenshade}{rgb}{0.5,0.9,0.5}

\usepackage[unicode,colorlinks=true,linkcolor=mylinkcolor,citecolor=mycitecolor]{hyperref}
\newcommand{\refref}[2]{\hyperref[#2]{#1~\ref*{#2}}}
\newcommand{\eqnref}[1]{\hyperref[#1]{(\ref*{#1})}}

\newcommand{\tos}{\!\!\to\!\!}

\usepackage{framed}

\newcommand{\cequiv}{\simeq}

\makeatletter
\newcommand\xleftrightarrow[2][]{%
  \ext@arrow 9999{\longleftrightarrowfill@}{#1}{#2}}
\newcommand\longleftrightarrowfill@{%
  \arrowfill@\leftarrow\relbar\rightarrow}
\makeatother

\newcommand{\bsquare}{\textrm{\ding{114}}}

% % % % % % % % % % % % % % % % % % % % % % % % % % % % % %

\newtheoremstyle{myplain}
  {\topsep}   % ABOVESPACE
  {\topsep}   % BELOWSPACE
  {\itshape}  % BODYFONT
  {0pt}       % INDENT (empty value is the same as 0pt)
  {\bfseries} % HEADFONT
  {.}         % HEADPUNCT
  {5pt plus 1pt minus 1pt} % HEADSPACE
  {\thmnumber{#2}. \thmname{#1}\thmnote{ (#3)}}   % CUSTOM-HEAD-SPEC

\newtheoremstyle{myplainnameless}
  {\topsep}   % ABOVESPACE
  {\topsep}   % BELOWSPACE
  {\normalfont}  % BODYFONT
  {0pt}       % INDENT (empty value is the same as 0pt)
  {\bfseries} % HEADFONT
  {.}         % HEADPUNCT
  {5pt plus 1pt minus 1pt} % HEADSPACE
  {\thmnumber{#2}}   % CUSTOM-HEAD-SPEC 

\newtheoremstyle{sectionexercise}
  {\topsep}   % ABOVESPACE
  {\topsep}   % BELOWSPACE
  {\normalfont}  % BODYFONT
  {0pt}       % INDENT (empty value is the same as 0pt)
  {\bfseries} % HEADFONT
  {.}         % HEADPUNCT
  {5pt plus 1pt minus 1pt} % HEADSPACE
  {Ejercicio \thmnumber{#2}\thmnote{ (#3)}}   % CUSTOM-HEAD-SPEC

\newtheoremstyle{mydefinition}
  {\topsep}   % ABOVESPACE
  {\topsep}   % BELOWSPACE
  {\normalfont}  % BODYFONT
  {0pt}       % INDENT (empty value is the same as 0pt)
  {\bfseries} % HEADFONT
  {.}         % HEADPUNCT
  {5pt plus 1pt minus 1pt} % HEADSPACE
  {\thmnumber{#2}. \thmname{#1}\thmnote{ (#3)}}   % CUSTOM-HEAD-SPEC

% EN ESPAÑOL

\newtheorem*{hecho*}{Hecho}
\newtheorem*{corolario*}{Corolario}
\newtheorem*{teorema*}{Teorema}
\newtheorem*{conjetura*}{Conjetura}
\newtheorem*{proyecto*}{Proyecto}
\newtheorem*{observacion*}{Observación}

\newtheorem*{lema*}{Lema}
\newtheorem*{resultado-clave*}{Resultado clave}
\newtheorem*{proposicion*}{Proposición}

\theoremstyle{definition}
\newtheorem*{ejercicio*}{Ejercicio}
\newtheorem*{definicion*}{Definición}
\newtheorem*{comentario*}{Comentario}
\newtheorem*{definicion-alternativa*}{Definición alternativa}
\newtheorem*{ejemploxs}{Ejemplo}
\newenvironment{ejemplo*}
  {\pushQED{\qed}\renewcommand{\qedsymbol}{\examplesymbol}\ejemploxs}
  {\popQED\endejemploxs}

\theoremstyle{myplain}
\newtheorem{proposicion}{Proposición}[section]

\newtheorem{proyecto}[proposicion]{Proyecto}
\newtheorem{teorema}[proposicion]{Teorema}
\newtheorem{corolario}[proposicion]{Corolario}
\newtheorem{hecho}[proposicion]{Hecho}
\newtheorem{lema}[proposicion]{Lema}

\newtheorem{observacion}[proposicion]{Observación}

\newenvironment{observacionejerc}
    {\pushQED{\qed}\renewcommand{\qedsymbol}{$\square$}\csname inner@observacionejerc\endcsname}
    {\popQED\csname endinner@observacionejerc\endcsname}
\newtheorem{inner@observacionejerc}[proposicion]{Observación}

\newenvironment{proposicionejerc}
    {\pushQED{\qed}\renewcommand{\qedsymbol}{$\square$}\csname inner@proposicionejerc\endcsname}
    {\popQED\csname endinner@proposicionejerc\endcsname}
\newtheorem{inner@proposicionejerc}[proposicion]{Proposicion}

\newenvironment{lemaejerc}
    {\pushQED{\qed}\renewcommand{\qedsymbol}{$\square$}\csname inner@lemaejerc\endcsname}
    {\popQED\csname endinner@lemaejerc\endcsname}
\newtheorem{inner@lemaejerc}[proposicion]{Lema}

\newtheorem{calculo}[proposicion]{Cálculo}

\theoremstyle{myplainnameless}
\newtheorem{nameless}[proposicion]{}

\theoremstyle{mydefinition}
\newtheorem{comentario}[proposicion]{Comentario}
\newtheorem{comentarioast}[proposicion]{Comentario ($\clubsuit$)}
\newtheorem{construccion}[proposicion]{Construcción}
\newtheorem{aplicacion}[proposicion]{Aplicación}
\newtheorem{definicion}[proposicion]{Definición}
\newtheorem{definicion-alternativa}[proposicion]{Definición alternativa}
\newtheorem{notacion}[proposicion]{Notación}
\newtheorem{advertencia}[proposicion]{Advertencia}
\newtheorem{digresion}[proposicion]{Digresión}
\newtheorem{ejemplox}[proposicion]{Ejemplo}
\newenvironment{ejemplo}
  {\pushQED{\qed}\renewcommand{\qedsymbol}{\examplesymbol}\ejemplox}
  {\popQED\endejemplox}
\newtheorem{contraejemplox}[proposicion]{Contraejemplo}
\newenvironment{contraejemplo}
  {\pushQED{\qed}\renewcommand{\qedsymbol}{\examplesymbol}\contraejemplox}
  {\popQED\endcontraejemplox}
\newtheorem{noejemplox}[proposicion]{No-ejemplo}
\newenvironment{noejemplo}
  {\pushQED{\qed}\renewcommand{\qedsymbol}{\examplesymbol}\noejemplox}
  {\popQED\endnoejemplox}
 
\newtheorem{ejemploastx}[proposicion]{Ejemplo ($\clubsuit$)}
\newenvironment{ejemploast}
  {\pushQED{\qed}\renewcommand{\qedsymbol}{\examplesymbol}\ejemploastx}
  {\popQED\endejemploastx}

\ifdefined\exercisespersection
  \theoremstyle{sectionexercise}
  \newtheorem{ejercicio}{}[section]
  \theoremstyle{mydefinition}
\else
  \ifdefined\exercisesglobal
    \theoremstyle{sectionexercise}
    \newtheorem{ejercicio}{}
    \theoremstyle{mydefinition}
  \else
    \ifdefined\exercisespersection
      \newtheorem{ejercicio}[proposicion]{Ejercicio}
    \fi
  \fi
\fi

% % % % % % % % % % % % % % % % % % % % % % % % % % % % % %

\theoremstyle{myplain}
\newtheorem{proposition}{Proposition}[section]
\newtheorem*{fact*}{Fact}
\newtheorem*{proposition*}{Proposition}
\newtheorem{lemma}[proposition]{Lemma}
\newtheorem*{lemma*}{Lemma}

\newtheorem{exercise}{Exercise}
\newtheorem*{hint}{Hint}

\newtheorem{theorem}[proposition]{Theorem}
\newtheorem{conjecture}[proposition]{Conjecture}
\newtheorem*{theorem*}{Theorem}
\newtheorem{corollary}[proposition]{Corollary}
\newtheorem{fact}[proposition]{Fact}
\newtheorem*{claim}{Claim}
\newtheorem{definition-theorem}[proposition]{Definition-theorem}

\theoremstyle{mydefinition}
\newtheorem{examplex}[proposition]{Example}
\newenvironment{example}
  {\pushQED{\qed}\renewcommand{\qedsymbol}{\examplesymbol}\examplex}
  {\popQED\endexamplex}

\newtheorem*{examplexx}{Example}
\newenvironment{example*}
  {\pushQED{\qed}\renewcommand{\qedsymbol}{\examplesymbol}\examplexx}
  {\popQED\endexamplexx}

\newtheorem{definition}[proposition]{Definition}
\newtheorem*{definition*}{Definition}
\newtheorem{wrong-definition}[proposition]{Wrong definition}
\newtheorem{remark}[proposition]{Remark}

\makeatletter
\newcommand{\xRightarrow}[2][]{\ext@arrow 0359\Rightarrowfill@{#1}{#2}}
\makeatother

% % % % % % % % % % % % % % % % % % % % % % % % % % % % % %

\newcommand{\Et}{\mathop{\text{\rm Ét}}}

\newcommand{\categ}[1]{\text{\bf #1}}
\newcommand{\vcateg}{\mathcal}
\newcommand{\bone}{{\boldsymbol 1}}
\newcommand{\bDelta}{{\boldsymbol\Delta}}
\newcommand{\bR}{{\mathbf{R}}}

\newcommand{\univ}{\mathfrak}

\newcommand{\TODO}{\colorbox{red}{\textbf{*** TODO ***}}}
\newcommand{\proofreadme}{\colorbox{red}{\textbf{*** NEEDS PROOFREADING ***}}}

\makeatletter
\def\iddots{\mathinner{\mkern1mu\raise\p@
\vbox{\kern7\p@\hbox{.}}\mkern2mu
\raise4\p@\hbox{.}\mkern2mu\raise7\p@\hbox{.}\mkern1mu}}
\makeatother

\newcommand{\ssincl}{\reflectbox{\rotatebox[origin=c]{45}{$\subseteq$}}}
\newcommand{\vsupseteq}{\reflectbox{\rotatebox[origin=c]{-90}{$\supseteq$}}}
\newcommand{\vin}{\reflectbox{\rotatebox[origin=c]{90}{$\in$}}}

\newcommand{\Ga}{\mathbb{G}_\mathrm{a}}
\newcommand{\Gm}{\mathbb{G}_\mathrm{m}}

\renewcommand{\U}{\UUU}

\DeclareRobustCommand{\Stirling}{\genfrac\{\}{0pt}{}}
\DeclareRobustCommand{\stirling}{\genfrac[]{0pt}{}}

% % % % % % % % % % % % % % % % % % % % % % % % % % % % % %
% tikz

\usepackage{tikz-cd}
\usetikzlibrary{babel}
\usetikzlibrary{decorations.pathmorphing}
\usetikzlibrary{arrows}
\usetikzlibrary{calc}
\usetikzlibrary{fit}
\usetikzlibrary{hobby}

% % % % % % % % % % % % % % % % % % % % % % % % % % % % % %
% Banners

\newcommand\mybannerext[3]{{\normalfont\sffamily\bfseries\large\noindent #1

\noindent #2

\noindent #3

}\noindent\rule{\textwidth}{1.25pt}

\vspace{1em}}

\newcommand\mybanner[2]{{\normalfont\sffamily\bfseries\large\noindent #1

\noindent #2

}\noindent\rule{\textwidth}{1.25pt}

\vspace{1em}}

\renewcommand{\O}{\mathcal{O}}


\numberwithin{equation}{section}

\usepackage[numbers]{natbib}

\usepackage{fullpage}

\usepackage{tocloft}

\renewcommand\cftsecfont{\normalfont}
\renewcommand\cfttoctitlefont{\sffamily\large\bfseries}
\renewcommand\cftsecpagefont{\normalfont}
\renewcommand{\cftsecleader}{\cftdotfill{\cftsecdotsep}}
\renewcommand\cftsecdotsep{\cftdot}
\renewcommand\cftsubsecdotsep{\cftdot}

\hypersetup{
  pdftitle = {Introducción al lenguaje funtorial},
  pdfauthor = {Alexey Beshenov (cadadr@gmail.com)},
  pdfdisplaydoctitle = true
}

\author{Alexey Beshenov (cadadr@gmail.com)}
\title{Introducción al lenguaje funtorial}
\date{Universidad de El Salvador. Ciclo impar 2018}

\usepackage{multicol}

\setlength{\columnseprule}{0.4pt}

\theoremstyle{definition}
\newtheorem{ejerc}{Ejercicio}

\newif\ifsolutions
\solutionstrue
% \solutionsfalse

\usepackage{multirow}

\begin{document}

{\normalfont\sffamily\bfseries \maketitle}

\epigraph{...mathematics will perish before the end of this century if the
  present trend for senseless abstraction---I call it: theory of the empty
  set---cannot be blocked up.}{Carta de Siegel a Mordell sobre el libro de Lang
  ``Diophantine Geometry'' (1964)}

En los presentes apuntes no vamos a estudiar la teoría de categorías como tal,
sino que veremos algunas definiciones básicas e indispensables para cursos de
álgebra y geometría.

La literatura adicional recomendada son los libros de texto
\cite{Leinster-Basic} y \cite{Riehl-Context}, el libro clásico \cite{MacLane}
(escrito por uno de los fundadores de la teoría) y el manual en tres volúmenes
\cite{Borceux-1,Borceux-2,Borceux-3}. El lector interesado en la perspectiva
histórico-filosófica, puede consultar \cite{Marquis-geometrical-POV} y
\cite{Kromer-Tool-and-Object}.

\tableofcontents

\pagebreak

\section{Definición de categoría}

%\marginpar{\footnotesize lección\\31.05.18}

\begin{definicion}
  Una \term{categoría} $\vcateg{C}$ está definida por los siguientes datos.

  \begin{enumerate}
  \item[1)] Una clase de \term{objetos} $\Ob (\vcateg{C})$ que vamos a denotar
    por las letras $X, Y, Z, \ldots$

  \item[2)] Conjuntos $\Hom_\vcateg{C} (X,Y)$ de \term{morfismos}
    $f\colon X\to Y$ para cada par de objetos fijos $X, Y\in \Ob (\vcateg{C})$.

  \item[3)] Una ley de composición de morfismos
    \begin{align*}
      \circ\colon \Hom_\vcateg{C} (X,Y) \times \Hom_\vcateg{C} (Y,Z) & \to \Hom_\vcateg{C} (X,Z),\\
      (f, g) & \mapsto g\circ f.
    \end{align*}
  \end{enumerate}

  Estos datos tienen que satisfacer los siguientes axiomas.

  \begin{enumerate}
  \item[i)] Para cada objeto $X$ existe el morfismo identidad
    $\id{X}\colon X\to X$ que se comporta como la identidad respecto a la
    composición:
    \begin{align*}
      (X\xrightarrow{f} Y)\circ (X\xrightarrow{\id{X}} X) & = (X\xrightarrow{f} Y),\\
      (Y\xrightarrow{\id{Y}} Y) \circ (X\xrightarrow{f} Y) & = (X\xrightarrow{f} Y).
    \end{align*}

  \item[ii)] La composición es asociativa: para cualesquiera
    $X,Y,Z,W\in\Ob (\vcateg{C})$ y $f\colon X\to Y$, $g\colon Y\to Z$,
    $h\colon Z\to W$ se cumple
    $$h\circ (g\circ f) = (h\circ g)\circ f.$$
  \end{enumerate}
\end{definicion}

\begin{comentario}
  \label{comentario:clases-y-categorias-pequenas}
  Hemos dicho que tenemos una \emph{clase} de objetos porque normalmente los
  objetos no forman un conjunto sino un ``conjunto grande''. El ejemplo más
  básico es la categoría donde los objetos son conjuntos y los morfismos
  $X\to Y$ son aplicaciones entre conjuntos. Todos los conjuntos no forman un
  conjunto. Por esto en una categoría se trata de una \emph{clase} de
  objetos. Cuando los objetos forman un conjunto, se dice que $\vcateg{C}$ es
  una categoría \term{pequeña}. La mayoría de las categoría interesantes que
  aparecen en la naturaleza no son pequeñas.
\end{comentario}

\begin{definicion}
  Se dice que una categoría $\vcateg{C}$ es una \term{subcategoría} de
  $\vcateg{D}$ si

  \begin{enumerate}
  \item[1)] los objetos de $\vcateg{C}$ forman una subclase de los objetos de
    $\vcateg{D}$: para todo $X\in \Ob{C}$ se tiene $X\in \Ob{D}$;

  \item[2)] los morfismos en $\vcateg{C}$ son también morfismos en $\vcateg{D}$:
    se tiene $\Hom_\vcateg{C} (X,Y) \subseteq \Hom_\vcateg{D} (X,Y)$ para
    cualesquiera $X,Y \in \Ob(\vcateg{C})$;

  \item[3)] la composición de morfismos en $\vcateg{C}$ es la misma que en
    $\vcateg{D}$.
  \end{enumerate}

  Si además $\Hom_\vcateg{C} (X,Y) = \Hom_\vcateg{D} (X,Y)$ para cualesquiera
  $X,Y \in \Ob(\vcateg{C})$, entonces se dice que $\vcateg{C}$ es una
  \term{subcategoría plena} de $\vcateg{D}$.
\end{definicion}

\begin{ejemplo}
  He aquí algunos ejemplos básicos de categorías que ya conocemos bien.

  \begin{center}
    \begin{tabular}{ccc}
      categoría & objetos & morfismos \\
      \hline
      $\categ{Set}$ & conjuntos & aplicaciones entre conjuntos \\
      $\categ{Top}$ & espacios topológicos & aplicaciones continuas \\
      $\categ{Haus}$ & espacios topológicos Hausdorff & aplicaciones continuas \\
      $\categ{Grp}$ & grupos & homomorfismos de grupos \\
      $\categ{Ab}$ & grupos abelianos & homomorfismos de grupos \\
      $\categ{Ring}$ & anillos & homomorfismos de anillos \\
      $\categ{CRing}$ & anillos conmutativos & homomorfismos de anillos \\
      $k\categ{-Vect}$ & espacios vectoriales sobre un cuerpo $k$ & aplicaciones $k$-lineales \\
      $k\categ{-vect}$ & espacios vectoriales de dimensión finita sobre $k$ & aplicaciones $k$-lineales \\
      $\cdots$ & $\cdots$ & $\cdots$ \\
    \end{tabular}
  \end{center}

  Todas las categorías de arriba pueden ser interpretadas como subcategorías de
  $\categ{Set}$: sus objetos son conjuntos con cierta estructura especial y los
  morfismos son aplicaciones que preservan las estructuras. La categoría
  $\categ{Ab}$ es una subcategoría plena de $\categ{Grp}$, la categoría de
  anillos conmutativos $\categ{CRing}$ es una subcategoría plena de la categoría
  de todos los anillos $\categ{Ring}$. La categoría $\categ{Haus}$ es una
  subcategoría plena de $\categ{Top}$.

  Los espacios vectoriales de dimensión finita forman una subcategoría plena
  $k\categ{-vect}\subset k\categ{-Vect}$.
\end{ejemplo}

\begin{ejemplo}
  \label{ejemplo:preorden-como-categoria}
  Un \term{conjunto preordenado} es un conjunto $X$ dotado de una relación
  $\preceq$ que satisface los siguientes axiomas.

  \begin{enumerate}
  \item[P1)] La relación es \term{simétrica}: se tiene $x\preceq x$ para todo
    $x\in X$.

  \item[P2)] La relación es \term{transitiva}: si $x\preceq y$ e $y\preceq z$,
    entonces $x\preceq z$.
  \end{enumerate}

  En este caso se puede definir una categoría donde los objetos son los
  elementos de $X$ y los morfismos son
  \[ \Hom (x,y) \dfn \begin{cases}
      \{ x\to y \}, & \text{ si }x\preceq y,\\
      \emptyset, & \text{ en caso contrario.}
    \end{cases} \]
  El axioma P1) nos da los morfismos identidad, mientras que el
  axioma P2) nos permite componer los morfismos.
\end{ejemplo}

\begin{ejemplo}
  \label{ejemplo:categoria-de-abiertos}
  Si $X$ es un espacio topológico, se puede definir la categoría
  $\categ{Abiertos} (X)$ donde los objetos son los subconjuntos abiertos
  $U\subseteq X$ y los morfismos son
  \[ \Hom (U,V) \dfn \begin{cases}
      \{ U\to V \}, & \text{ si }U\subseteq V,\\
      \emptyset, & \text{ en caso contrario.}
    \end{cases}. \]
\end{ejemplo}

\begin{ejemplo}
  Los espacios métricos forman una categoría $\categ{Met}$. Un objeto de esta
  categoría es un conjunto $X$ dotado de una distancia
  $d_X\colon X\times X\to \RR$ que satisface los axiomas habituales. Los
  morfismos $f\colon X\to Y$ son las aplicaciones que preservan la métrica en el
  sentido de que
  $$d_Y (f (x), f (y)) \le d_X (x,y)$$
  para cualesquiera $x,y\in X$. A veces se considera la categoría más grande
  $\categ{Lipsch}$ donde los morfismos son las
  \term{aplicaciones Lipschitz}. Una aplicación $f\colon X\to Y$ es Lipschitz si
  existe una constante $C > 0$ tal que
  $$d_Y (f (x), f (y)) \le C\,d_X (x,y)$$
  para cualesquiera $x,y\in X$. Note que $\categ{Met}$ es una subcategoría de
  $\categ{Lipsch}$. Puesto que toda aplicación Lipschitz es continua, la
  categoría $\categ{Lipsch}$ puede ser vista como una subcategoría de
  $\categ{Top}$.
\end{ejemplo}

\begin{ejemplo}
  Sea $G$ un grupo fijo y $X$ un conjunto. Se dice que $G$ \term{actúa} sobre
  $X$ por la izquierda, o que $X$ es un \term{$G$-conjunto} si está definida una
  aplicación
  \begin{align*}
    G\times X & \to X,\\
    (g,x) & \mapsto g\cdot x
  \end{align*}
  que satiface
  \begin{enumerate}
  \item[1)] $1\cdot x = x$ para todo $x\in X$,
  \item[2)] $g\cdot (h\cdot x) = (gh)\cdot x$ para cualesquiera $g,h\in G$ y
    $x\in X$.
  \end{enumerate}

  Un morfismo de $G$-conjuntos es una aplicación $f\colon X\to Y$ que es
  compatible con la acción de $G$: se cumple $f (g\cdot x) = g\cdot f (x)$ para
  cualesauiera $g\in G$ y $x\in X$. En otras palabras, se pide que el diagrama
  \[ \begin{tikzcd}
      G\times X\ar{r}{\idid\times f}\ar{d} & G\times Y\ar{d} \\
      X\ar{r}[swap]{f} & Y \\
    \end{tikzcd} \]
  sea conmutativo. Esto nos da una categoría $G\categ{-Set}$.
\end{ejemplo}

La aceptación del lenguaje categórico se debe a que casi todas las estructuras
en álgebra y geometría dan lugar a categorías, y muchas propiedades básicas son
consecuencias formales de la teoría de categorías.

% % % % % % % % % % % % % % % % % % % % % % % % % % % % % %

\section{Isomorfismos, monomorfismos y epimorfismos}

\begin{definicion}
  \label{definicion:iso-mono-epi}
  Sea $f\colon X\to Y$ un morfismo en cualquier categoría.

  \begin{enumerate}
  \item[1)] $f$ es un \term{isomorfismo} si existe otro morfismo
    $f^{-1}\colon Y\to X$ tal que $f^{-1}\circ f = \id{X}$ y
    $f\circ f^{-1} = \id{Y}$. (Ejercicio: si existe, este $f^{-1}$ es únicamente
    definido por $f$.) En este caso escribimos $X\isom Y$.

  \item[2)] $f$ es un \term{monomorfismo} si para todo par de morfismos
    $g,g'\colon Z \to X$ tenemos
    $$f\circ g = f\circ g' \Rightarrow g = g';$$
    en otras palabras, si para todo $Z\in \Ob(\vcateg{C})$ la aplicación
    \begin{align*}
      f_*\colon \Hom_\vcateg{C} (Z,X) & \to \Hom_\vcateg{C} (Z,Y),\\
      g & \mapsto f\circ g
    \end{align*}
    es inyectiva.

  \item[3)] $f$ es un \term{epimorfismo} si para todo par de morfismos
    $g,g'\colon Y\to Z$ tenemos
    $$g\circ f = g'\circ f \Rightarrow g = g';$$
    en otras palabras, si para todo $Z\in \Ob(\vcateg{C})$ la aplicación
    \begin{align*}
      f^*\colon \Hom_\vcateg{C} (Y,Z) & \to \Hom_\vcateg{C} (X,Z),\\
      g & \mapsto g\circ f
    \end{align*}
    es inyectiva.
  \end{enumerate}

  Un monomorfismo se denota por $\mono$ y un epimorfismo se denota por $\epi$.
\end{definicion}

\begin{ejemplo}
  \label{ejemplo:grupos-como-categorias}
  A todo grupo $G$ se puede asociar una categoría $\categ{BG}$ donde hay un solo
  objeto $\ast$, hay un morfismo $g\colon \ast\to\ast$ por cada elemento
  $g\in G$ y la composición $g\circ f$ corresponde al producto $gf\in G$. En la
  categoría $\categ{BG}$ todos los morfismos son isomorfismos. Recíprocamente,
  una categoría donde hay solo un objeto y todos los morfismos son invertibles
  corresponde a un grupo.

  Si no asumimos la existencia de morfismos inversos, una categoría con un
  objeto corresponde a un \term{monoide}. Una categoría donde todas las flechas
  son invertibles, pero puede haber más de un objeto, corresponde a un
  \term{grupoide}. Los grupoides tienen papel fundamental en la topología
  algebraica: en lugar del \term{grupo fundamental} $\pi_1 (X,x_0)$ (construido
  a partir de un punto fijo $x_0\in X$) es más natural considerar el
  \term{grupoide fundamental} $\Pi_1 (X)$ (sin fijar $x_0\in X$).
\end{ejemplo}

\begin{observacion}
  Las composiciones de iso-, mono-, epimorfismos satisfacen las siguientes
  propiedades.

  \begin{enumerate}
  \item[1)] Si $f\colon X\to Y$ e $g\colon Y\to Z$ son isomorfismos, entonces
    $g\circ f\colon X\to Z$ es un isomorfismo.

  \item[2)] Si $m\colon X\mono Y$ y $m'\colon Y\mono Z$ son monomorfismos,
    entonces $m'\circ m\colon X\to Z$ es un monomorfismo.

  \item[3)] Si $e\colon X\epi Y$ y $e'\colon Y\epi Z$ son epimorfismos, entonces
    $e'\circ e\colon X\to Z$ es un epimorfismo.

  \item[4)] Si para $m\colon X\to Y$, $f\colon Y\to Z$ la composición $f\circ m$
    es un monomorfismo, entonces $m$ es un monomorfismo.

  \item[5)] Si para $f\colon X\to Y$, $e\colon Y\to Z$ la composición $e\circ f$
    es un epimorfismo, entonces $e$ es un epimorfismo.
  \end{enumerate}

  \begin{proof}
    Ejercicio para el lector.
  \end{proof}
\end{observacion}

\begin{observacion}
  Todo isomorfismo es automáticamente mono y epi.

  \begin{proof}
    Ejercicio para el lector.
  \end{proof}
\end{observacion}

\begin{ejemplo}
  En la categoría $\categ{Set}$ los isomorfismos, monomorfismos, epimorfismos
  son las aplicaciones biyectivas, inyectivas, sobreyectivas respectivamente.
\end{ejemplo}

En cualquier categoría donde los objetos son conjuntos con estructura adicional,
y los morfismos son aplicaciones que preservan esta estructura, un morfismo
$f\colon X\to Y$ que es inyectivo en los conjuntos subyacentes es un
monomorfismo, y un morfismo $f\colon X\to Y$ que es sobreyectivo en los
conjuntos subyacentes es un epimorfismo. El recíproco no siempre se cumple:
puede haber monomorfismos no inyectivos y epimorfismos no sobreyectivos.

\begin{ejemplo}
  En la categoría de anillos $\categ{Ring}$ (resp. anillos conmutativos
  $\categ{CRing}$) los isomorfismos son precisamente los homomorfismos
  biyectivos. En efecto, si tenemos un homomorfismo de anillos $f\colon R\to S$
  que es biyectivo, entonces la aplicación inversa $f^{-1}\colon S\to R$ es
  automáticamente un homomorfismo de anillos.

  Los monomorfismos de anillos (resp. anillos conmutativos) son precisamente los
  homomorfismos inyectivos. Por ejemplo, sea $f\colon R\to S$ un homomorfismo de
  anillos que \emph{no es} inyectivo. Esto quiere decir que $f (x) = f (x')$
  para diferentes $x,x'\in R$. Definamos dos homomorfismos
  $$g\colon \ZZ [X] \to R, \quad X\mapsto x$$
  y
  $$g'\colon \ZZ [X] \to R, \quad X\mapsto x'$$
  (el anillo de polinomios $\ZZ [X]$ tiene la siguiente propiedad particular:
  para cualquier anillo $R$ un homomorfismo $\ZZ [X] \to R$ se define de manera
  única por la imagen de $X$ y esta puede ser cualquier elemento de $R$). Luego
  tenemos $f\circ g = f\circ g'$, aun cuando $g\ne g'$. Esto demuestra que $f$
  no es un monomorfismo.

  Sin embargo, no todo epimorfismo de anillos (resp. anillos conmuativos) es
  necesariamente sobreyectivo. Consideremos por ejemplo la inclusión
  $f\colon \ZZ \to \QQ$. Supongamos que $g\circ f = g'\circ f$ para algunos
  homomorfismos de anillos $g,g'\colon \QQ \to R$. Luego, para todo
  $\frac{a}{b} \in \QQ$ se tiene
  \begin{multline*}
    g \left(\frac{a}{b}\right) =
    g (a)\cdot g \left(\frac{1}{b}\right) =
    g' (a)\cdot g \left(\frac{1}{b}\right) =
    g' \left(\frac{a}{b}\cdot b\right)\cdot g \left(\frac{1}{b}\right) =
    g' \left(\frac{a}{b}\right)\cdot g' (b)\cdot g \left(\frac{1}{b}\right) \\
  = g' \left(\frac{a}{b}\right)\cdot g (b)\cdot g \left(\frac{1}{b}\right) =
    g' \left(\frac{a}{b}\right)\cdot g \left(b\cdot \frac{1}{b}\right) =
    g' \left(\frac{a}{b}\right)\cdot g (1) =
    g' \left(\frac{a}{b}\right).
  \end{multline*}
  Entonces, $g = g'$. Esto nos da un ejemplo de un morfismo que es epi y mono
  pero no es iso.
\end{ejemplo}

\begin{ejemplo}
  Sería interesante ver algún ejemplo de un monomorfismo que no es inyectivo.
  Recordemos que grupo $A$ es \term{divisible} si para todo $x\in A$ y para todo
  $n = 1,2,3,\ldots$ existe $y\in A$ tal que
  $$n\,y \dfn \underbrace{y + \cdots + y}_n = x.$$
  Por ejemplo, el grupo aditivo $\QQ$ es divisible. Por otro lado, para todo
  homomorfismo de grupos abelianos $f\colon A\to B$ donde $A$ es divisible, se
  cumple que la imagen $\im f$ es también un grupo divisible. Esto implica que
  si $A$ es divisible, todo cociente $A/B$ es divisible. Por ejemplo, el grupo
  $\QQ/\ZZ$ es divisible.

  Esto nos da la categoría $\categ{AbDiv}$ de grupos abelianos divisibles, que
  es una subcategoría plena de la categoría $\categ{Ab}$. Dentro de esta
  categoría podemos considerar el homomorfismo canónico
  $f\colon \QQ\to \QQ/\ZZ$. Sea $A$ otro grupo divisible y sean
  $g,g'\colon A\to \QQ$ dos homomorfismos tales que $f\circ g = f\circ
  g'$. Afirmamos que $g = g'$. En efecto, supongamos que $g \ne g'$, es decir
  $h\dfn g - g' \ne 0$, mientras que $f\circ h = 0$. Entonces, existe algún
  $x\in A$ tal que $h (x) \ne 0$. Sin pérdida de generalidad,
  $h (x) \in \ZZ\setminus \{ 0 \}$ es un entero positivo (en el caso contrario
  podemos reemplazar $x$ por $-x$ o multiplicarlo por un entero). Gracias a la
  divisibilidad de $A$, existe algún $y\in A$ tal que
  $$2\cdot h (x)\cdot y = x.$$
  Aplicando $h$ a la igualdad anterior, se obtiene la siguiente igualdad en
  $\QQ$:
  $$2\cdot h (x)\cdot h (y) = h (x).$$
  De aquí se sigue que $h (y) = 1/2$. Por tanto $f (h (y)) \ne 0$ en $\QQ/\ZZ$,
  lo que contradice el hecho de que $f\circ h = 0$.
\end{ejemplo}

\begin{ejemplo}
  En la categoría de grupos $\categ{Grp}$ (resp. grupos abelianos $\categ{Ab}$)
  los isomorfismos son precisamente los homomorfismos biyectivos. Si
  $f\colon G\to H$ es un homomorfismo biyectivo de grupos, entonces la
  aplicación inversa $f^{-1}\colon H\to G$ es automáticamente un homomorfismo.

  Los monomorfismos de grupos (resp. grupos abelianos) son los homomorfismos
  inyectivos. En efecto, si $f\colon G\to H$ no es inyectivo, entonces
  $f (x) = f (x')$ para diferentes $x,x'\in G$. Podemos considerar los
  homomorfismos
  $$g\colon \ZZ \to G, \quad 1\mapsto x$$
  y
  $$g'\colon \ZZ \to G, \quad 1\mapsto x'$$
  (todo homomorfismo de grupos $\ZZ\to G$ está definido de modo único por la
  imagen de $1$ que puede ser cualquier elemento de $G$). Ahora,
  $f\circ g = f\circ g'$, aunque $g\ne g'$, así que $f$ no es un monomorfismo.

  Los epimorfismos de grupos (resp. grupos abelianos) son precisamente los
  homomorfismos sobreyectivos; esto no es tan fácil y no lo voy a
  probar\footnote{Véase por ejemplo \emph{Linderholm, C.E. A group epimorphism
      is surjective. Amer. Math. Monthly 77, pp. 176–177.}}. En la categoría de
  grupos abelianos $\categ{Ab}$ funciona el siguiente truco. Sea
  $f\colon A\to B$ un homomorfismo de grupos abelianos que no es
  sobreyectivo. Lo último significa que $\im f \ne B$ y $B/\im f \ne 0$. Podemos
  considerar el homomorfismo $p\colon B\to B/\im f$ (la proyección sobre el
  grupo cociente) y el homomorfismo nulo $0\colon B\to B/\im f$ que aplica todo
  en $0$. Luego, $p\circ f = 0\circ f$, aunque $p\ne 0$, así que $f$ no es un
  epimorfismo. Para los grupos no abelianos, este argumento falla: para un
  homomorfismo $f\colon G\to H$ la imagen $\im f$ no tiene por qué ser un
  subgrupo normal de $G$ si $G$ no es abeliano.
\end{ejemplo}

\begin{ejemplo}
  Los isomorfismos en la categoría $\categ{Top}$ de espacios topológicos tienen
  nombre especial: son los \term{homeomorfismos}. Note que no son simplemente
  morfismos biyectivos: una aplicación continua $f\colon X\to Y$ puede tener una
  aplicación inversa $f^{-1}\colon Y\to X$ que no es continua.

  Los monomorfismos de espacios topológicos son precisamente las aplicaciones
  continuas inyectivas. En efecto, dada una aplicación continua $f\colon X\to Y$
  tal que $f (x) = f (x')$ para algunos $x\ne x'$, podemos considerar el espacio
  unipuntual $\{ \ast \}$ y dos aplicaciones $g,g '\colon \{ \ast \}\to X$ dadas
  por
  $$g\colon \ast \mapsto x, \quad g'\colon \ast \mapsto x'.$$
  Tenemos $g\circ f = g'\circ f$, mientras que $g\ne g'$.

  Los epimorfismos de espacios topológicos son las aplicaciones continuas
  sobreyectivas. Para verlo, consideremos una aplicación continua
  $f\colon X\to Y$ tal que $\im f\ne Y$. Sea $\{ 0,1 \}$ el espacio topológico
  de dos puntos dotado de la topología trivial (indiscreta). Sea
  $g\colon Y\to \{ 0, 1 \}$ la aplicación que envía todo en $1$ y sea
  $g'\colon Y\to \{ 0, 1 \}$ la aplicación definida por
  \[ g' (y) \dfn \begin{cases}
      1, & y\in \im f,\\
      0, & y\notin \im f.
    \end{cases} \]
  Ambas aplicaciones $g$ y $g'$ son continuas, puesto que la topología sobre
  $\{ 0, 1 \}$ es trivial. Ahora $g\circ f = g'\circ f$, pero $g\ne g'$.
\end{ejemplo}

\begin{ejemplo}
  Consideremos ahora la categoría $\categ{Haus}$ de espacios Hausdorff. El mismo
  argumento de arriba demuestra que los monomorfismos son precisamente las
  aplicaciones continuas inyectivas. Sin embargo, el argumento sobre los
  epimorfismos no funciona: el espacio $\{ 0, 1 \}$ con la topología indiscreta
  no es Hausdorff.

  En realidad, los epimorfismos en $\categ{Haus}$ son las aplicaciones continuas
  $f\colon X\to Y$ con imagen densa
  (es decir, $\overline{\im f} = Y$). Efectivamente, si $f\colon X\to Y$ es una
  aplicación continua con esta propiedad, entonces si dos aplicaciones
  $g,g'\colon Y\to Z$ satisfacen $g\circ f = g'\circ f$, estas coinciden sobre
  $\im f$, y por ende $g = g'$. Aquí estamos usando la siguiente propiedad:
  cuando $g,g'\colon Y\to Z$ son dos aplicaciones continuas, $Y$ es Hausdorff y
  $g$ y $g'$ coinciden sobre un subconjunto denso de $Y$, entonces $g = g'$.

  En la otra dirección, dado un epimorfismo $f\colon X\to Y$ entre dos espacios
  de Hausdorff, se puede comprobar que $\overline{\im f} = Y$, pero el argumento
  es un poco más sutíl\footnote{Véase
    \url{http://math.stackexchange.com/questions/214045/}} y prefiero omitirlo.
\end{ejemplo}

\begin{ejemplo}
  En la categoría $k\categ{-Vect}$ de espacios vectoriales sobre $k$, los
  isomorfismos, monomorfismos, epimorfismos son las aplicaciones $k$-lineales
  biyectivas, inyectivas, sobreyectivas respectivamente.
\end{ejemplo}

% % % % % % % % % % % % % % % % % % % % % % % % % % % % % %

\section{Funtores entre categorías}

\epigraph{Les mathématiciens n'étudient pas des objets, mais des relations entre
  les objets}{Poincaré}

% \epigraph{Comme tu sais, mon honorable collègue Mac Lane soutient que toute
% notion de structure comporte necessairement une notion d'homomorphisme,
% consistant à indiquer, pour chacune des données constituant la structure,
% celles qui se comportent de manière covariante et celles qui se comportent de
% manière contravariante... Que penses-tu qu'il y ait à tirer de ce genre de
% considerations?}{Carta de Weil a Chevalley, el 15 de octubre de 1950}

Para relacionar diferentes categorías, sirve la noción de funtor.

\begin{definicion}
  Sean $\vcateg{C}$ y $\vcateg{D}$ dos categorías. Un \term{funtor (covariante)}
  $F\colon \vcateg{C} \to \vcateg{D}$ es una regla que

  \begin{enumerate}
  \item[1)] a cada objeto $X$ de $\vcateg{C}$ asigna un objeto $F (X)$ de
    $\vcateg{D}$,

  \item[2)] a cada morfismo $f\colon X\to Y$ en $\vcateg{C}$ asigna un morfismo
    $F (f) \rdfn f_*\colon F (X) \to F (Y)$ en $\vcateg{D}$;
  \end{enumerate}
  de modo que se cumplen los siguientes axiomas:
  \begin{enumerate}
  \item[a)] $F$ preserva los morfismos identidad: $(\idid_{X})_* = \id{F (X)}$
    para cada $X$;
  \item[b)] $F$ preserva la composición de morfismos: $(g\circ f)_* = g_*\circ f_*$.
  \end{enumerate}
\end{definicion}

\begin{observacion}
  Si $f$ es un isomorfismo en $\vcateg{C}$ y $F$ es un funtor
  $\vcateg{C}\to \vcateg{D}$, entonces $F (f)$ es un isomorfismo en
  $\vcateg{D}$.

  \begin{proof}
    Ejercicio para el lector.
  \end{proof}
\end{observacion}

\begin{observacion}
  Si $F\colon \vcateg{C}\to\vcateg{D}$ y $G\colon \vcateg{D}\to\vcateg{E}$ son
  dos funtores, entonces la composición
  $G\circ F\colon \vcateg{C} \to \vcateg{E}$ (definida como la aplicación de $F$
  seguida por la aplicación de $G$) es también un funtor. Esta composición es
  asociativa. También para cada categoría tenemos el funtor identidad
  $\Id_\vcateg{C}\colon \vcateg{C} \to \vcateg{C}$ (que aplica un objeto $X$ en
  el mismo objeto $X$ y un morfismo $f$ en el mismo morfismo $f$). Esto es la
  identidad respecto a la composición de funtores.
\end{observacion}

\begin{comentario}
  Intuitivamente, todo esto significa que las categorías forman una categoría
  donde los objetos son categorías y los morfismos son funtores entre ellas. Sin
  embargo, hay un pequeño detalle fastidioso: según nuestra definición, para dos
  objetos fijos $X$ e $Y$ los morfismos $X\to Y$ forman un conjunto. Sin
  embargo, los objetos de una categoría no necesariamente forman un conjunto y
  por esto para dos categorías $\vcateg{C}$ y $\vcateg{D}$ los funtores
  $\vcateg{C}\to\vcateg{D}$ no necesariamente forman un conjunto.

  Sin embargo, las categorías pequeñas (véase
  \ref{comentario:clases-y-categorias-pequenas}) forman una categoría. Esta se
  denota por $\categ{Cat}$.
\end{comentario}

\begin{ejemplo}
  \label{ejemplo:hom-covariante}
  Sea $\vcateg{C}$ una categoría y $X$ un objeto fijo. Para cada objeto $Y$
  podemos considerar el conjunto de morfismos $X\to Y$:
  $$Y \quad\rightsquigarrow\quad \Hom_\vcateg{C} (X,Y).$$
  Cada morfismo $f\colon Y\to Y'$ induce una aplicación $f_*$ definida por la
  composición con $f$:
  $$(Y\xrightarrow{f} Y') \quad\rightsquigarrow\quad (\Hom_\vcateg{C} (X,Y) \xrightarrow{f_*} \Hom_\vcateg{C} (X,Y')).$$

  \[ \begin{tikzcd}
      X\ar{r}{g}\ar[dashed]{d}[swap]{f\circ g} & Y\ar{dl}{f} \\
      Y'
    \end{tikzcd} \]

  Todo esto define un funtor
  $$\Hom_\vcateg{C} (X,-)\colon \vcateg{C} \to \categ{Set}$$
  ---el morfismo identidad $\idid\colon Y\to Y$ obviamente induce el morfismo
  identidad
  $$\idid_*\colon \Hom_\vcateg{C} (X,Y) \to \Hom_\vcateg{C} (X,Y)$$
  y para la composición de morfismos $Y \xrightarrow{f} Y' \xrightarrow{f'} Y''$
  tenemos $(f'\circ f)_* = f'_* \circ f_*$.
\end{ejemplo}

\begin{ejemplo}
  Para los espacios vectoriales sobre $k$, las aplicaciones $k$-lineales
  $U\to V$ también forman un espacio vectorial sobre $k$, y en este caso se
  tiene un funtor
  $$\iHom_k (U, -)\colon k\categ{-Vect} \to k\categ{-Vect}.$$
  ---escribimos ``$\iHom_k$'' en lugar de ``$\Hom_{k\categ{-Vect}}$'' para
  subrayar que los valores no están en $\categ{Set}$, sino en $k\categ{-Vect}$.
\end{ejemplo}

\begin{ejemplo}
  Hay una serie de funtores tontos pero muy importantes: los
  \term{funtores olvidadizos}. Estos funtores olvidan una parte de la
  estructura. Por ejemplo, todo espacio topológico, grupo, anillo, espacio
  vectorial, $G$-conjunto, etc. es un conjunto. Esto nos da funtores como
  \[ \categ{Top} \to \categ{Set}, \quad
     \categ{Grp} \to \categ{Set}, \quad
     \categ{Ring} \to \categ{Set}, \quad
     k\categ{-Vect} \to \categ{Set}, \quad
     G\categ{-Set}\to \categ{Set}, \quad
     \text{etc.} \]
  que asignan a cada objeto el conjunto subyacente y a cada morfismo la
  aplicación entre conjuntos correspondiente.

  También se puede olvidar solo una parte de la estructura: por ejemplo, olvidar
  que un espacio es Hausdorff, que un grupo es abeliano, que un anillo es
  conmutativo, que en un espacio vectorial hay una multiplicación por escalares:
  \[ \categ{Haus} \to \categ{Top}, \quad
     \categ{Ab} \to \categ{Grp}, \quad
     \categ{CRing} \to \categ{Ring}, \quad
     k\categ{-Vect} \to \categ{Ab}, \quad
     \text{etc.} \qedhere \]
\end{ejemplo}

\begin{ejemplo}
  Sea $k$ un cuerpo. A partir de un conjunto $X$ se puede considerar el espacio
  vectorial $k \langle X\rangle$ generado por los elementos de $X$. Esto define
  un funtor
  $$\categ{Set} \to k\categ{-Vect}.$$
  En efecto, cada aplicación entre conjuntos $X\to Y$ induce
  \emph{de modo funtorial} la aplicación $k$-lineal correspondiente
  $k \langle X\rangle \to k \langle Y\rangle$.
\end{ejemplo}

\begin{ejemplo}
  Para un anillo $R$ consideremos el grupo de unidades (elementos invertibles)
  $R^\times$. Es fácil comprobar que para todo homomorfismo de anillos
  $f\colon R\to S$ se cumple $f (R^\times) \subseteq S^\times$, así que tenemos
  un funtor
  \[ (-)^\times\colon \categ{Ring} \to \categ{Grp}. \qedhere \]
\end{ejemplo}

\begin{ejemplo}
  Si consideramos grupos como categorías (véase
  \ref{ejemplo:grupos-como-categorias}), funtores corresponden a homomorfismos.
\end{ejemplo}

\begin{ejemplo}
  En topología algebraica, a cada espacio $X$ se puede asociar el conjunto
  $\pi_0 (X)$ formado por las componentes conexas (por caminos). Esto es un
  funtor $\pi_0\colon \categ{Top} \to \categ{Set}$: una aplicación continua
  $X\to Y$ necesariamente envia los puntos de la misma componente conexa de $X$
  a la misma componente conexa de $Y$.

  El grupo fundamental $\pi_1 (X,x_0)$ es un funtor
  $\categ{Top}_\ast \to \categ{Grp}$. Aquí $\categ{Top}_\ast$ es la categoría
  cuyos objetos son los \term{espacios topológicos con un punto marcado}; es
  decir, pares $(X,x_0)$ donde $X$ es un espacio topológico y $x_0\in X$. Los
  morfismos $f\colon (X,x_0) \to (Y,y_0)$ en esta categoría son las aplicaciones
  continuas $f\colon X\to Y$ tales que $f (x_0) = y_0$. Resulta que el grupo
  fundamental es un invariante \term{homotópico} de $(X,x_0)$. Esto puede ser
  expresado diciendo que $\pi_1$ es un funtor definido sobre la categoría
  $\categ{HoTop}_\ast$ donde los objetos son los mismos, pero los morfismos son
  las clases de equivalencia homotópica de aplicaciones continuas
  $(X,x_0)\to (Y,y_0)$.

  Si queremos deshacernos de la elección del punto marcado, hay que considerar
  el funtor del grupoide fundamental $\Pi_1$ que a cada espacio asocia un
  grupoide.

  También se pueden definir los \term{grupos de homotopía superiores}
  $\pi_n (X,x_0)$ para cualquier $n = 2,3,4,5,\ldots$ En este caso son todos
  grupos abelianos, y se trata de una serie de funtores
  $\pi_n\colon \categ{Top}_\ast \to \categ{Ab}$ o
  ${\categ{HoTop}_* \to \categ{Ab}}$. Sin embargo, el cálculo de estos grupos es
  extremadamente difícil incluso para espacios sencillos como las esferas: la
  estructura de los grupos $\pi_n (S^k, \ast)$ es un problema muy profundo de
  matemáticas, sin respuesta sencilla y completa.

  Los grupos de \term{homología} $H_n (X)$ y \term{cohomología} $H_n (X;A)$
  (donde $A$ es el \term{grupo de coeficientes}) son invariantes homotópicos más
  manejables y son también funtores $\categ{Top}_\ast \to \categ{Ab}$ o
  $\categ{HoTop}_* \to \categ{Ab}$.
\end{ejemplo}

Muy a menudo es útil considerar funtores que cambian la dirección de los
morfismos.

\begin{definicion}
  Sean $\vcateg{C}$ y $\vcateg{D}$ dos categorías. Un \term{funtor
    contravariante} $F\colon \vcateg{C} \to \vcateg{D}$ es una regla que a cada
  objeto $X$ de $\vcateg{C}$ asigna un objeto $F (X)$ de $\vcateg{D}$, y a cada
  morfismo $f\colon X\to Y$ en $\vcateg{C}$ asigna un morfismo
  $F (f) \rdfn f^* \colon F (Y) \to F (X)$ en $\vcateg{D}$, de modo que
  $$\id{X}^* = \id{F (X)}, \quad (g\circ f)^* = f^*\circ g^*.$$
\end{definicion}

Para tratar los funtores covariantes y contravariantes de la misma manera, es
útil introducir la noción de categoría opuesta.

\begin{definicion}
  Sea $\vcateg{C}$ una categoría. Entonces su \term{categoría opuesta}
  $\vcateg{C}^\mathrm{op}$ es la categoría donde los objetos son los mismos,
  pero un morfismo $f^\mathrm{op}\colon X\to Y$ en $\vcateg{C}^\mathrm{op}$ es
  un morfismo $f\colon Y\to X$ en $\vcateg{C}$ y la composición está definida
  por
  $$g^\mathrm{op} \circ f^\mathrm{op} \dfn (f\circ g)^\mathrm{op}.$$

  \[ \begin{tikzcd}
      \vcateg{C}^\mathrm{op}\colon & X\ar{r}[swap]{f^\mathrm{op}}\ar[bend left]{rr}{g^\mathrm{op}\circ f^\mathrm{op}} & Y\ar{r}[swap]{g^\mathrm{op}} & Z \\
      \vcateg{C}\colon & X & Y\ar{l}[swap]{f} & Z\ar{l}[swap]{g}\ar[bend left]{ll}{f\circ g}
    \end{tikzcd} \]
\end{definicion}

\begin{observacion}
  Un funtor contravariante $F\colon \vcateg{C} \to \vcateg{D}$ puede ser visto
  como un funtor covariante $F\colon \vcateg{C}^\mathrm{op} \to \vcateg{D}$ o
  $F\colon \vcateg{C} \to \vcateg{D}^\mathrm{op}$.
\end{observacion}

\begin{comentario}
  Muy a menudo se dice que un funtor $F$ es contravariante y se escribe
  $F\colon \vcateg{C}^\mathrm{op} \to \vcateg{D}$ para subrayar que es
  contravariante. En realidad, como funtor definido sobre
  $\vcateg{C}^\mathrm{op}$, es covariante.
\end{comentario}

\begin{ejemplo}
  \label{ejemplo:hom-contravariante}
  Sea $\vcateg{C}$ una categoría e $Y$ un objeto fijo. Para cada objeto $X$
  podemos considerar el conjunto de morfismos $X\to Y$:
  $$Y \quad\rightsquigarrow\quad \Hom_\vcateg{C} (X,Y).$$
  Cada morfismo $f\colon X\to X'$ induce una aplicación $f^*$ por la composición
  con $f$. Notemos que el único modo sensato de definirla es de tomar un
  morfismo $g\colon X' \to Y$ y componerlo con $f\colon X \to X'$:
  \[ (X\xrightarrow{f} X') \quad\rightsquigarrow\quad
     (\Hom_\vcateg{C} (X',Y) \xrightarrow{f^*} \Hom_\vcateg{C} (X,Y)). \]

  \[ \begin{tikzcd}
      Y & X\ar{dl}{f}\ar[dashed]{l}[swap]{g\circ f} \\
      X'\ar{u}{g}
    \end{tikzcd} \]

  Vemos que esto define un funtor
  \[ \Hom_\vcateg{C} (-,Y)\colon \vcateg{C}^\mathrm{op} \to
     \categ{Set}. \qedhere \]
\end{ejemplo}

\begin{ejemplo}
  Para dos conjuntos preordenados $X$ y $Y$ considerados como categorías (véase
  \ref{ejemplo:preorden-como-categoria}) un funtor es una aplicación
  $f\colon X\to Y$ que preserva la relación $\preceq$.
\end{ejemplo}

\begin{ejemplo}
  En \ref{ejemplo:categoria-de-abiertos} asociamos a todo espacio topológico $X$
  la categoría $\categ{Abiertos} (X)$. Si tenemos una aplicación continua
  $f\colon X\to Y$, entonces por la definición, $f^{-1} (U)$ es abierto en $X$
  para todo abierto $U\subseteq Y$. Esto define un funtor
  $f^{-1}\colon \categ{Abiertos} (Y)\to \categ{Abiertos} (X)$.
\end{ejemplo}

\begin{ejemplo}
  Para los espacios vectoriales sobre $k$, tenemos un funtor
  $$\iHom_k (-, V)\colon k\categ{-Vect}^\mathrm{op} \to k\categ{-Vect}.$$
  En particular, el espacio vectorial dual define un funtor
  \[ (-)^* \dfn \iHom_k (-, k)\colon k\categ{-Vect}^\mathrm{op} \to
     k\categ{-Vect}. \qedhere \]
\end{ejemplo}

\begin{observacion}
  La composición de dos funtores contravariantes es un funtor covariante.

  \begin{proof}
    La dirección de los morfismos cambia a la opuesta dos veces.

    A saber, si tenemos dos funtores contravariantes
    $F\colon \vcateg{C}\to\vcateg{D}$ y $G\colon \vcateg{D}\to\vcateg{E}$, estos
    pueden ser vistos como funtores
    $F\colon \vcateg{C}\to\vcateg{D}^\mathrm{op}$ y
    $G\colon \vcateg{D}^\mathrm{op}\to\vcateg{E}$. Al componerlos, nos queda un
    funtor $\vcateg{C}\to\vcateg{E}$.
  \end{proof}
\end{observacion}

\begin{ejemplo}
  El funtor del espacio doble dual
  $$(-)^{**} \dfn (-)^*\circ (-)^* \colon k\categ{-Vect} \to k\categ{-Vect}$$
  es covariante.
\end{ejemplo}

\begin{definicion}
  Sea $F\colon \vcateg{C}\to\vcateg{D}$ un funtor. Para cualesquera
  $X,Y\in\Ob(\vcateg{C})$ podemos considerar la aplicación de conjuntos
  \begin{align*}
    \Hom_\vcateg{C} (X,Y) & \to \Hom_\vcateg{C} (F (X), F (Y)),\\
    f & \mapsto F (f).
  \end{align*}

  \begin{enumerate}
  \item[1)] Si la aplicación de arriba es inyectiva para cualesquiera
    $X,Y\in\Ob(\vcateg{C})$, se dice que $F$ es \term{fiel}.

  \item[2)] Si es sobreyectiva para cualesquiera $X,Y\in\Ob(\vcateg{C})$, se
    dice que $F$ es \term{pleno}.

  \item[3)] Si es biyectiva para cualesquiera $X,Y\in\Ob(\vcateg{C})$, se dice
    que $F$ es \term{fielmente pleno} o bien \term{fiel y
      pleno}\footnote{``Fully faithful'' o ``full and faithful'' en inglés.}.
  \end{enumerate}
\end{definicion}

\begin{ejemplo}
  El funtor olvidadizo $k\categ{-Vect} \to \categ{Set}$ es fiel: diferentes
  morfismos entre espacios vectoriales corresponden a diferentes aplicaciones
  entre conjuntos.

  El funtor olvidadizo $\categ{Ab} \to \categ{Grp}$ es fiel y pleno: un
  homomorfismo de grupos abelianos es un homomorfismo de grupos.

  Si consideramos grupos como categorías (véase
  \ref{ejemplo:grupos-como-categorias}), funtores fieles, plenos y fielmente
  plenos corresponden a monomorfismos, epimorfismos e isomorfismos.
\end{ejemplo}

Hay miles de ejemplos interesantes de funtores. Toda construcción buena y
canónica corresponde a un funtor.

% % % % % % % % % % % % % % % % % % % % % % % % % % % % % %

\section{Transformaciones naturales entre funtores}

%\marginpar{\footnotesize lección\\1.06.18}

\epigraph{If topology were publicly defined as the study of families of sets
  closed under finite intersection and infinite unions a serious disservice
  would be perpetrated on embryonic students of topology. The mathematical
  correctness of such a definition reveals nothing about topology except that
  its basic axioms can be made quite simple. And with the category theory we are
  confronted with the same pedagogical problem. The basic axioms <...> are much
  too simple.

  A better (albeit not perfect) description of topology is that it is the study
  of continuous maps; and category theory is likewise better described as the
  theory of functors. <...> It is not too misleading, at least historically, to
  say that categories are what one must define in order to define functors, and
  that functors are what one must define in order to define natural
  transformations.}{Peter Freyd, introducción al libro\\``Abelian categories''
  (1964)}

Resumamos nuestra situación. En las matemáticas clásicas normalmente se
estudiaban conjuntos dotados de cierta estructura. Por ejemplo, un espacio
vectorial sobre $k$ es un conjunto con estructura de grupo abeliano y
multiplicación por los elementos de $k$ que satisfacen ciertos axiomas, etc. El
punto de vista categórico supone que, en vez de estudiar elementos de cada
objeto, hay que estudiar las aplicaciones entre objetos. La definición de
categorías no menciona elementos que ``pertenecen'' a objetos y en realidad no
habla de aplicaciones: se trata solo de algunas flechas $X\to Y$ que se pueden
componer formalmente. Otro nivel de razonamiento es de dar más importancia no a
las categorías particulares, sino a los funtores entre categorías. Pero en
realidad, la noción más importante no es la de categoría ni la de funtor, sino
la de morfismo entre funtores; estas son las transformaciones naturales.

\begin{definicion}
  Sean $\vcateg{C}$ y $\vcateg{D}$ dos categorías y sean
  $F,G\colon \vcateg{C} \to \vcateg{D}$ dos funtores entre ellas. Una
  \term{transformación natural} $\alpha\colon F\Rightarrow G$ entre $F$ y $G$ es
  una colección de morfismos $\alpha_X\colon F(X) \to G(X)$ en $\vcateg{D}$ para
  cada objeto $X$ de $\vcateg{C}$, tal que para cada morfismo $X\to Y$ en
  $\vcateg{C}$ el siguiente diagrama en $\vcateg{D}$ es conmutativo:

  \[ \begin{tikzcd}
      F (X)\ar{r}{\alpha_X}\ar{d}[swap]{F (f)} & G (X)\ar{d}{G (f)} \\
      F (Y)\ar{r}[swap]{\alpha_Y} & G (Y)
    \end{tikzcd} \]
\end{definicion}

Si $F$ y $G$ son dos funtores $\vcateg{C} \to \vcateg{D}$, vamos a usar la
notación
$$\Nat (F,G) \dfn \{ \text{transformaciones naturales }F\Rightarrow G \}.$$
En general $\Nat (F,G)$ no es un conjunto porque a priori los objetos de
$\vcateg{C}$ y $\vcateg{D}$ no forman conjuntos.

\begin{observacion}
  Si $F,G,H$ son tres funtores $\vcateg{C} \to \vcateg{D}$ y
  $\alpha\colon F\Rightarrow G$ y $\beta\colon G\Rightarrow H$ son
  transformaciones naturales, entonces la composición
  $\beta\circ\alpha\colon F\Rightarrow H$ definida como
  $$(\beta\circ\alpha)_X \dfn \beta_X \circ \alpha_X$$
  es también una transformación natural:

  \[ \begin{tikzcd}
      F (X)\ar{r}{\alpha_X}\ar{d}{F (f)} & G (X)\ar{d}{G (f)}\ar{r}{\beta_X} & H (X)\ar{d}{H (f)} \\
      F (Y)\ar{r}{\alpha_Y} & G (Y)\ar{r}{\beta_Y} & H (Y)
    \end{tikzcd} \]

  Además, para cada funtor $F$ existe la transformación natural identidad
  $\Id_F\colon F\Rightarrow F$ definida por
  $(\Id_F)_X \dfn F (\id{X}) = \id{F (X)}$ para todo $X$. Es la identidad
  respecto a la composición de transformaciones naturales.
\end{observacion}

Todo esto nos permite decir cuándo dos funtores son isomorfos.

\begin{definicion}
  Se dice que dos funtores $F,G\colon \vcateg{C} \to \vcateg{D}$ son
  \term{isomorfos} si existen transformaciones naturales
  $\alpha\colon F\Rightarrow G$ y $\alpha^{-1}\colon G\Rightarrow F$ tales que
  $\alpha^{-1}\circ\alpha = \Id_F$ y $\alpha\circ\alpha^{-1} = \Id_G$.
\end{definicion}

La situación de arriba es equivalente a tener una transformación natural
$\alpha\colon F\Rightarrow G$ donde cada $\alpha_X\colon F (X)\to G (X)$ es un
isomorfismo.

\begin{ejemplo}
  Para un espacio vectorial $U$ tenemos la aplicación de evaluación que va de
  $U$ a su doble dual:
  \begin{align*}
    ev_U\colon U & \to U^{**},\\
    u & \mapsto (\phi \mapsto \phi (u)).
  \end{align*}
  Esta aplicación es natural en el sentido de que define una transformación
  natural entre el funtor identidad sobre $k\categ{-Vect}$ y el funtor
  $(-)^{**}\colon k\categ{-Vect}\to k\categ{-Vect}$. En efecto, dado una
  aplicación lineal $f\colon U\to V$, esta induce primero la aplicación
  \begin{align*}
    f^*\colon V^* & \to U^*,\\
    (\phi\colon V\to k) & \mapsto (\phi\circ f\colon U\to k).
  \end{align*}

  Luego, pasando otra vez a espacios duales, se obtiene
  \begin{align*}
    f^{**}\colon U^{**} & \to V^{**},\\
    (\phi\colon U^* \to k) & \mapsto (\phi\circ f^*\colon V^*\to k).
  \end{align*}

  Tenemos que comprobar que el siguiente cuadrado conmuta:

  \[ \begin{tikzcd}
      U\ar{d}[swap]{f}\ar{r}{ev_U} & U^{**}\ar{d}{f^{**}} \\
      V\ar{r}[swap]{ev_V} & V^{**}
    \end{tikzcd} \]

  De hecho, ambas aplicaciones $ev_V\circ f$ y $f^{**}\circ ev_U$ asocian a un
  vector $u\in U$ el mismo elemento de $V^{**}$ dado por
  \[ \phi \mapsto \phi (f (u)). \qedhere \]
\end{ejemplo}

\begin{ejemplo}
  Sea $R$ un anillo conmutativo y sea $\GL_n (R)$ el grupo de las matrices
  invertibles de $n\times n$ con coeficientes en $R$. Un homomorfismo de anillos
  $f\colon R\to S$ induce de modo funtorial un homomorfismo de grupos
  $$\GL_n (f)\colon \GL_n (R)\to \GL_n (S).$$
  En otras palabras, se aplica $f$ a cada entrada de una matriz. Entonces,
  tenemos un funtor
  $$\GL_n\colon \categ{CRing} \to \categ{Grp}.$$
  Una matriz es invertible si y solamente si su determinante es invertible, y el
  determinante es un homomorfismo de grupos
  $$\det\colon \GL_n (R) \to R^\times.$$
  Además, el determinante es natural en el sentido de que cada homomorfismo de
  anillos $f\colon R\to S$ induce un cuadrado conmutativo

  \[ \begin{tikzcd}
      \GL_n (R)\ar{d}[swap]{\GL_n (f)}\ar{r}{\det_R} & R^\times\ar{d}{f} \\
      \GL_n (S)\ar{r}[swap]{\det_S} & S^\times
    \end{tikzcd} \]

  En conclusión, el determinante define una transformación natural entre los
  funtores
  \[ \GL_n, (-)^\times\colon \categ{CRing} \to \categ{Grp}. \qedhere \]
\end{ejemplo}

\begin{ejemplo}
  Tenemos funtores $\Hom_\vcateg{C} (X,-)\colon \vcateg{C} \to \categ{Set}$ y
  $\Hom_\vcateg{C} (-,Y)\colon \vcateg{C}^\mathrm{op} \to \categ{Set}$. Cada
  morfismo
  $$X \to X' \quad\text{y}\quad Y \to Y'$$
  induce transformaciones naturales
  \[ \Hom_\vcateg{C} (X',-) \Rightarrow \Hom_\vcateg{C} (X,-) \quad\text{y}\quad
     \Hom_\vcateg{C} (-, Y) \Rightarrow \Hom_\vcateg{C} (-, Y'). \]

  Por ejemplo, en el primer caso, tenemos diagramas conmutativos

  \[ \begin{tikzcd}
      \Hom_\vcateg{C} (X',Y)\ar{r}{\alpha_Y}\ar{d} & \Hom_\vcateg{C} (X,Y)\ar{d} & (X'\to Y) \ar[|->]{r}\ar[|->]{d} & (X \to X' \to Y) \ar[|->]{d} \\
      \Hom_\vcateg{C} (X',Y')\ar{r}{\alpha_{Y'}} & \Hom_\vcateg{C} (X,Y') & (X'\to Y\to Y') \ar[|->]{r} & (X\to X' \to Y \to Y') \\
    \end{tikzcd} \]
  y el segundo caso corresponde a los mismos diagramas, salvo que la
  transformación natural corresponde a las flechas verticales. Todo esto
  significa que $\Hom_\vcateg{C} (-,-)$ es un \term{bifuntor}, es decir es un
  funtor en dos argumentos, y su comportamiento en un argumento es compatible
  con su comportamiento en el otro.
\end{ejemplo}

\begin{ejemplo}
  Sea $\{ \ast \}$ un conjunto unipuntual. El funtor
  $\Hom_\categ{Set} (\{ \ast \}, -)\colon \categ{Set} \to \categ{Set}$ es
  isomorfo a $\Id\colon \categ{Set} \to \categ{Set}$. En efecto, se ve que la
  biyección em
  \begin{align*}
    \Hom_\categ{Set} (\{ \ast \}, X) & \xrightarrow{\isom} X,\\
    f & \mapsto f (\ast)
  \end{align*}
  es natural en $X$.

  De la misma manera, para espacios vectoriales, el funtor $\iHom_k (k, -)$ es
  isomorfo a $\Id\colon k\categ{-Vect} \to k\categ{-Vect}$, gracias al
  isomorfismo natural $\iHom_k (k, V) \isom V$ definido por $f \mapsto f (1)$.
\end{ejemplo}

% % % % % % % % % % % % % % % % % % % % % % % % % % % % % %

\section{Producto de Godement}
\label{sec:producto-de-godement}

%\marginpar{\footnotesize lo añadí\\
%para\\
%explicar las\\
%identidades\\
%triangulares}

\begin{definicion}
  ~

  \begin{itemize}
  \item Sean $\vcateg{C}, \vcateg{D}, \vcateg{E}$ tres categorías;

  \item sean $F_1, G_1$ dos funtores entre $\vcateg{C}$ y $\vcateg{D}$;

  \item sean $F_2, G_2$ dos funtores entre $\vcateg{D}$ y $\vcateg{E}$;

  \item sean $\alpha\colon F_1\Rightarrow G_1$ y
    $\beta\colon F_2\Rightarrow G_2$ dos transformaciones naturales.
  \end{itemize}

  Todo esto puede ser expresado mediante el diagrama

  \[ \begin{tikzpicture}
      \matrix[matrix of math nodes,column sep=2cm] (m)
      {
        \vcateg{C} & \vcateg{D} & \vcateg{E} \\
      };
      \draw[->] (m-1-1) to[bend left=50] node[label=above:$F_1$] (F1) {} (m-1-2);
      \draw[->] (m-1-1) to[bend right=50] node[label=below:$G_1$] (G1) {} (m-1-2);
      \draw[->] (m-1-2) to[bend left=50] node[label=above:$F_2$] (F2) {} (m-1-3);
      \draw[->] (m-1-2) to[bend right=50] node[label=below:$G_2$] (G2) {} (m-1-3);

      \draw[double,double equal sign distance,-implies,shorten >=10pt,shorten <=10pt] 
      (F1) -- node[label=right:$\alpha$] {} (G1);
      \draw[double,double equal sign distance,-implies,shorten >=10pt,shorten <=10pt] 
      (F2) -- node[label=right:$\beta$] {} (G2);
    \end{tikzpicture} \]

  Entonces, el \term{producto de Godement}\footnote{\personality{Roger Godement}
    (1921--2016), matemático francés, miembro del grupo Bourbaki, conocido por
    sus contribuciones en análisis funcional y libros expositivos, por ejemplo
    ``Topologie Algébrique et Théorie des Faisceaux'' (1958).} de $\alpha$ y
  $\beta$ es la transformación natural
  $$\beta\ast\alpha\colon F_2\circ F_1\Rightarrow G_2\circ G_1$$
  definida como
  \[ (\beta\ast\alpha)_X \dfn \beta_{G_1 (X)} \circ F_2 (\alpha_X) =
     G_2 (\alpha_X)\circ \beta_{F_1(X)}. \]

  \[ \begin{tikzcd}[row sep=4em, column sep=4em]
      F_2 (F_1(X)) \ar{r}{F_2 (\alpha_X)}\ar{d}[swap]{\beta_{F_1(X)}}\ar{dr}{(\beta\ast\alpha)_X} & F_2 (G_1(X)) \ar{d}{\beta_{G_1(X)}} \\
      G_2 (F_1(X)) \ar{r}[swap]{G_2 (\alpha_X)} & G_2 (G_1 (X))
    \end{tikzcd} \]
  (Este diagrama conmuta gracias a la naturalidad de $\beta$.)

  \[ \begin{tikzpicture}
      \matrix[matrix of math nodes,column sep=3cm] (m)
      {
        \vcateg{C} & \vcateg{E} \\
      };
      \draw[->] (m-1-1) to[bend left=50] node[label=above:$F_2\circ F_1$] (F) {} (m-1-2);
      \draw[->] (m-1-1) to[bend right=50] node[label=below:$G_2\circ G_1$] (G) {} (m-1-2);

      \draw[double,double equal sign distance,-implies,shorten >=10pt,shorten <=10pt] 
      (F) -- node[label=right:$\beta\ast\alpha$] {} (G);
    \end{tikzpicture} \]
\end{definicion}

\begin{observacion}
  $\beta\ast\alpha\colon F_2\circ F_1\Rightarrow G_2\circ G_1$ es una
  transformación natural: para todo morfismo $f\colon X\to X'$ en $\vcateg{C}$
  el diagrama
  \[ \begin{tikzcd}
      F_2 (F_1 (X)) \ar{r}{(\beta\ast\alpha)_X}\ar{d}[swap]{F_2 (F_1 (f))} & G_2 (G_1 (X))\ar{d}{G_2 (G_1 (f))} \\
      F_2 (F_1 (X')) \ar{r}[swap]{(\beta\ast\alpha)_{X'}} & G_2 (G_1 (X')) \\
    \end{tikzcd} \]
  conmuta.

  \begin{proof}
    Consideremos el siguiente cubo:
    \[ \begin{tikzcd}[row sep=4em, column sep=4em]
        && F_2 (F_1(X)) \ar{rrr}{F_2 (\alpha_X)}\ar[dashed,shift left=0.4em]{ddrrr}[near end]{(\beta\ast\alpha)_X}\ar[]{dd}[near end]{\beta_{F_1(X)}}\ar{dll}[swap]{F_2 (F_1 (f))} &&& F_2 (G_1(X))\ar{dd}{\beta_{G_1(X)}}\ar{dll}[swap]{F_2 (G_1 (f))} \\
        F_2 (F_1(X')) \ar[dashed,shift right=0.35em]{ddrrr}[near end]{(\beta\ast\alpha)_{X'}} \ar[crossing over]{rrr}[near end]{F_2 (\alpha_{X'})}\ar{dd}[swap]{\beta_{F_1(X')}} &&& F_2 (G_1(X')) \\
        && G_2 (F_1(X)) \ar{rrr}[near start]{G_2 (\alpha_X)}\ar{dll}[swap]{G_2 (F_1 (f))} &&& G_2 (G_1 (X))\ar{dll}{G_2 (G_1 (f))} \\
        G_2 (F_1(X')) \ar{rrr}[swap]{G_2 (\alpha_{X'})} &&& G_2 (G_1 (X'))\ar[leftarrow, crossing over]{uu}[near start,swap]{\beta_{G_1(X')}}
      \end{tikzcd} \]

    Las caras izquierda y derecha, posterior y delantera conmutan por la
    naturalidad de $\beta$. La cara de arriba y de abajo conmutan por la
    naturalidad de $\alpha$ (estas corresponden a los funtores $F_2$ y $G_2$
    aplicados al cuadrado de naturalidad). Podemos concluir que la sección
    diagonal del cubo también conmuta.
  \end{proof}
\end{observacion}

\begin{observacion}
  El producto de Godement es asociativo: para un diagrama
  \[ \begin{tikzpicture}
      \matrix[matrix of math nodes,column sep=2cm] (m)
      {
        \vcateg{A} & \vcateg{B} & \vcateg{C} & \vcateg{D} \\
      };
      \draw[->] (m-1-1) to[bend left=50] node[label=above:$F_1$] (F1) {} (m-1-2);
      \draw[->] (m-1-1) to[bend right=50] node[label=below:$G_1$] (G1) {} (m-1-2);
      \draw[->] (m-1-2) to[bend left=50] node[label=above:$F_2$] (F2) {} (m-1-3);
      \draw[->] (m-1-2) to[bend right=50] node[label=below:$G_2$] (G2) {} (m-1-3);
      \draw[->] (m-1-3) to[bend left=50] node[label=above:$F_3$] (F3) {} (m-1-4);
      \draw[->] (m-1-3) to[bend right=50] node[label=below:$G_3$] (G3) {} (m-1-4);

      \draw[double,double equal sign distance,-implies,shorten >=10pt,shorten <=10pt] 
      (F1) -- node[label=right:$\alpha$] {} (G1);
      \draw[double,double equal sign distance,-implies,shorten >=10pt,shorten <=10pt] 
      (F2) -- node[label=right:$\beta$] {} (G2);
      \draw[double,double equal sign distance,-implies,shorten >=10pt,shorten <=10pt] 
      (F3) -- node[label=right:$\gamma$] {} (G3);
    \end{tikzpicture} \]
  se cumple
  $$(\gamma\ast\beta)\ast\alpha = \gamma\ast(\beta\ast\alpha).$$

  \begin{proof}
    Ejercicio para el lector.
  \end{proof}
\end{observacion}

\begin{observacion}
  Sean $\vcateg{C},\vcateg{D},\vcateg{E}$ tres categorías. Sean $F,G,H$ funtores
  $\vcateg{C} \to \vcateg{D}$ y sean $I,J,K$ tres funtores
  $\vcateg{D}\to \vcateg{E}$. Consideremos transformaciones naturales
  \[ \alpha\colon F\Rightarrow G, \quad
     \beta\colon G\Rightarrow H, \quad
     \sigma\colon I\Rightarrow J, \quad
     \tau\colon J\Rightarrow K. \]

   \[ \begin{tikzpicture}
       \matrix[matrix of math nodes,column sep=3cm,row sep=5em] (m)
       {
         \vcateg{C} & \vcateg{D} & \vcateg{E} \\
       };
       \draw[->] (m-1-1) to[bend left=50] node[label=above:$F$] (F) {} (m-1-2);
       \draw[] (m-1-1) to node (G) {} (m-1-2);
       \draw[->] (m-1-1) to node[label=above:$G$,near end] {} (m-1-2);
       \draw[->] (m-1-1) to[bend right=50] node[label=below:$H$] (H) {} (m-1-2);
       \draw[->] (m-1-2) to[bend left=50] node[label=above:$I$] (I) {} (m-1-3);
       \draw[->] (m-1-2) to node[label=above:$J$,near end] {} (m-1-3);
       \draw[] (m-1-2) to node (J) {} (m-1-3);
       \draw[->] (m-1-2) to[bend right=50] node[label=below:$K$] (K) {} (m-1-3);

       \draw[double,double equal sign distance,-implies] (F) -- node[label=left:$\alpha$] {} (G);
       \draw[double,double equal sign distance,-implies] (G) -- node[label=left:$\beta$] {} (H);
       \draw[double,double equal sign distance,-implies] (I) -- node[label=left:$\sigma$] {} (J);
       \draw[double,double equal sign distance,-implies] (J) -- node[label=left:$\tau$] {} (K);

     \end{tikzpicture} \]

   Entonces,
   \[ (\tau\circ \sigma)\ast (\beta\circ \alpha) =
      (\tau\ast\beta)\circ (\sigma\ast\alpha). \]

   \begin{proof}
     Ejercicio para el lector.
   \end{proof}
\end{observacion}

Un caso muy especial es cuando una de las transformaciones naturales en cuestión
es la identidad:
\[ \begin{tikzpicture}
    \matrix[matrix of math nodes,column sep=2cm] (m)
    {
      \vcateg{C} & \vcateg{D} & \vcateg{E} \\
    };
    \draw[->] (m-1-1) to[bend left=50] node[label=above:$F$] (F1) {} (m-1-2);
    \draw[->] (m-1-1) to[bend right=50] node[label=below:$F$] (F2) {} (m-1-2);
    \draw[->] (m-1-2) to[bend left=50] node[label=above:$G$] (G) {} (m-1-3);
    \draw[->] (m-1-2) to[bend right=50] node[label=below:$H$] (H) {} (m-1-3);
    
    \draw[double,double equal sign distance,-implies,shorten >=10pt,shorten <=10pt] 
    (F1) -- node[label=right:$\Id$] {} (F2);
    \draw[double,double equal sign distance,-implies,shorten >=10pt,shorten <=10pt] 
    (G) -- node[label=right:$\beta$] {} (H);
  \end{tikzpicture} \]
en este caso se escribe simplemente
$$\beta\circ F\colon G\circ F\Rightarrow F\circ H.$$
Es la transformación natural definida por
$$(\beta\circ F)_X \dfn \beta_{F (X)}.$$

De la misma manera, si $\beta = \Id$
\[ \begin{tikzpicture}
    \matrix[matrix of math nodes,column sep=2cm] (m)
    {
      \vcateg{C} & \vcateg{D} & \vcateg{E} \\
    };
    \draw[->] (m-1-1) to[bend left=50] node[label=above:$F$] (F) {} (m-1-2);
    \draw[->] (m-1-1) to[bend right=50] node[label=below:$G$] (G) {} (m-1-2);
    \draw[->] (m-1-2) to[bend left=50] node[label=above:$H$] (H1) {} (m-1-3);
    \draw[->] (m-1-2) to[bend right=50] node[label=below:$H$] (H2) {} (m-1-3);

    \draw[double,double equal sign distance,-implies,shorten >=10pt,shorten <=10pt] 
    (F) -- node[label=right:$\alpha$] {} (G);
    \draw[double,double equal sign distance,-implies,shorten >=10pt,shorten <=10pt] 
    (H1) -- node[label=right:$\Id$] {} (H2);
  \end{tikzpicture} \]
entonces se trata de una transformación natural
$$H\circ\alpha\colon H\circ F\Rightarrow H\circ G$$
definida por
$$(H\circ\alpha)_X\dfn H (\alpha_X).$$

% % % % % % % % % % % % % % % % % % % % % % % % % % % % % %

\section{Funtores representables y el lema de Yoneda}

Por fin estamos listos para demostrar el hecho más importante de la teoría de
categorías básica.

\begin{proposicion}[{Lema de Yoneda\footnote{\personality{Nobuo Yoneda}
      (1930--1996) fue un matemático japonés, conocido principalmente por el
      lema que lleva su nombre. La leyenda dice que Yoneda y Saunders Mac Lane,
      uno de los fundadores de la teoría de categorías, se encontraron en un
      café de la estación de París Norte y su conversación continuó en el tren
      de Yoneda justo antes de que este partiera. Fue en esta ocasión que Yoneda
      explicó a Mac Lane su famoso lema. Gracias a esta coincidencia, Mac Lane
      formuló el lema como lo conocimos hoy en día y lo popularizó. Después de
      su regreso a Japón, Yoneda trabajó en informática, en particular en el
      lenguaje de programación Algol.}}]
  \label{lema-de-yoneda}
  ~

  \begin{enumerate}
  \item[1)] Para cada $X\in \Ob(\vcateg{C})$ y cada funtor covariante
    $F\colon \vcateg{C} \to \categ{Set}$ tenemos una biyección natural
    $$\Nat (\Hom_\vcateg{C} (X,-), F) \xrightarrow{\isom} F (X).$$
    Aquí la naturalidad quiere decir que para cada morfismo $f\colon X\to X'$ y
    cada transformación natural $\alpha\colon F\Rightarrow G$ los siguientes
    diagramas son conmutativos:

    \[ \begin{tikzcd}
        \Nat (\Hom_\vcateg{C} (X, -), F)\ar{d}[swap]{f^*}\ar{r}{\isom} & F (X)\ar{d}{F (f)} & \Nat (\Hom_\vcateg{C} (X, -), F)\ar{d}[swap]{\alpha\circ -}\ar{r}{\isom} & F (X)\ar{d}{\alpha_X} \\
        \Nat (\Hom_\vcateg{C} (X', -), F) \ar{r}{\isom} & F (X') & \Nat (\Hom_\vcateg{C} (X, -), G) \ar{r}{\isom} & G (X)
      \end{tikzcd} \]

    ($f\colon X\to X'$ induce una transformación natural
    $f^*\colon \Hom_\vcateg{C} (X', -) \Rightarrow \Hom_\vcateg{C} (X, -)$, y a
    su vez una aplicación entre conjuntos
    $f^*\colon \Nat (\Hom_\vcateg{C} (X, -), F) \to \Nat (\Hom_\vcateg{C} (X', -), F)$
    por la precomposición; la transformación natural
    $\alpha\colon F\Rightarrow G$ induce un morfismo entre conjuntos
    $\Nat (\Hom_\vcateg{C} (X,-), F) \Rightarrow \Nat (\Hom_\vcateg{C} (X,-),
    G)$ por la poscomposición con $\alpha$.)

  \item[2)] Para cada $Y\in \Ob(\vcateg{C})$ y cada funtor contravariante
    $F\colon \vcateg{C}^\mathrm{op} \to \categ{Set}$ tenemos una biyección
    natural
    $$\Nat (\Hom_\vcateg{C} (-,Y), F) \xrightarrow{\isom} F (Y).$$
    Aquí la naturalidad quiere decir que para cada morfismo $f\colon Y\to Y'$ y
    cada transformación natural $\alpha\colon F\Rightarrow G$ los siguientes
    diagramas son conmutativos:

    \[ \begin{tikzcd}
        \Nat (\Hom_\vcateg{C} (-,Y), F)\ar{r}{\isom} & F (Y) & \Nat (\Hom_\vcateg{C} (-, Y), F)\ar{d}[swap]{\alpha\circ -}\ar{r}{\isom} & F (Y)\ar{d}{\alpha_Y} \\
        \Nat (\Hom_\vcateg{C} (-,Y'), F) \ar{r}{\isom} \ar{u}{f^*} & F (Y')\ar{u}[swap]{F (f)} & \Nat (\Hom_\vcateg{C} (-,Y), G) \ar{r}{\isom} & G (Y)
      \end{tikzcd} \]
  \end{enumerate}

  \begin{proof}
    Veamos, por ejemplo, el caso contravariante. Dejo el caso covariante como un
    ejercicio al lector. Tenemos que definir una biyección
    $$\Nat (\Hom_\vcateg{C} (-,Y), F) \isom F (Y).$$

    A partir de una transformación natural
    $\alpha\colon \Hom_\vcateg{C} (-,Y) \Rightarrow F$ debemos producir un
    elemento de $F (Y)$. De hecho, hay solo una posibilidad obvia: tenemos la
    aplicación $\alpha_Y\colon \Hom_\vcateg{C} (Y,Y) \to F (Y)$ y el único
    elemento que seguramente contiene el conjunto $\Hom_\vcateg{C} (Y,Y)$ es el
    morfismo identidad $\id{Y}$. Entonces, podemos considerar
    $$\alpha_Y (\id{Y}) \in F (Y).$$

    Ahora bien, a partir de un elemento $y\in F (Y)$ tenemos que definir una
    transformación natural
    $$\alpha^y\colon \Hom_\vcateg{C} (-,Y) \Rightarrow F,$$
    es decir, una familia de morfismos
    $$\alpha^y_X\colon \Hom_\vcateg{C} (X,Y) \to F (X).$$
    Si tenemos un elemento $f\in \Hom_\vcateg{C} (X,Y)$, entonces $F$ nos da una
    aplicación entre conjuntos $F (f)\colon F (Y) \to F (X)$. Podemos aplicar
    $F (f)$ a $y \in F (Y)$ para obtener un elemento de $F (X)$:
    $$\alpha^y_X (f) \dfn F (f) (y).$$

    Hay que verificar varios detalles.

    \begin{enumerate}
    \item[1)] \textbf{$\alpha^y_X$ define una transformación natural
        $\alpha^y\colon \Hom_\vcateg{C} (-,Y) \Rightarrow F$.} En efecto, para
      cada $f\colon X\to X'$ tenemos un diagrama conmutativo

      \[ \begin{tikzcd}
          \Hom_\vcateg{C} (X,Y) \ar{r}{\alpha_X^y} & F (X) & h\circ f\ar[|->]{r} & F (h\circ f) (y) = F (f) \circ F (h) (y) \\
          \Hom_\vcateg{C} (X',Y) \ar{r}[swap]{\alpha_{X'}^y}\ar{u}{-\circ f} & F (X')\ar{u}[swap]{F (f)} & h\ar[|->]{u}\ar[|->]{r} & F (h) (y) \ar[|->]{u}
        \end{tikzcd} \]

    \item[2)] \textbf{Las correspondencias que hemos definido dan una biyección
        entre los conjuntos $\Nat (\Hom_\vcateg{C} (-,Y), F)$ y $F (Y)$.}

      Si tenemos una transformación natural
      $\alpha\colon \Hom_\vcateg{C} (-,Y) \Rightarrow F$, entonces el elemento
      correspondiente de $F (Y)$ es $\alpha_Y (\id{Y})$. Luego, a partir de
      $\alpha_Y (\id{Y})$ se construye la transformación natural
      \begin{align*}
  \Hom_\vcateg{C} (X,Y) & \to F (X),\\
        f & \mapsto F (f) \circ \alpha_Y (\id{Y}).
      \end{align*}
      Pero $\alpha$ es natural, de donde
      $$F (f) \circ \alpha_Y (\id{Y}) = \alpha_X\circ f^* (\id{Y}) = \alpha_X (f).$$

      \[ \begin{tikzcd}
          \Hom_\vcateg{C} (Y,Y) \ar{r}{\alpha_Y}\ar{d}[swap]{f^*} & F (Y)\ar{d}{F (f)} \\
          \Hom_\vcateg{C} (X,Y) \ar{r}[swap]{\alpha_X} & F (X) \\
\end{tikzcd} \]

A partir de $y\in F (Y)$ se construye una transformación natural
$\alpha^y_X\colon \Hom_\vcateg{C} (X,Y) \to F (X)$ definida por
$\alpha^y_X (f) = F (f) (y)$. Luego, se recupera el elemento
$\alpha^y_Y (\id{Y}) = F (\id{Y}) (y) = \id{F (y)} (y) = y$.

\item[3)] \textbf{La biyección es natural.} Primero, hay que ver que el
  siguiente diagrama conmuta:

  \[ \begin{tikzcd}
      \Nat (\Hom_\vcateg{C} (-,Y), F)\ar{r}{\isom} & F (Y) \\
      \Nat (\Hom_\vcateg{C} (-,Y'), F) \ar{r}{\isom} \ar{u}{f^*} & F (Y')\ar{u}[swap]{F (f)}
    \end{tikzcd} \]

  En efecto, para una transformación natural
  $\alpha\colon \Hom_\vcateg{C} (-,Y') \Rightarrow F$ tenemos

  \[ \begin{tikzcd}
      \alpha (f\circ -) \ar[|->]{r} & \alpha_Y (f\circ \id{Y}) = \alpha_Y \circ f^* = F (f) \circ \alpha_{Y'} (\id{Y'}) \\
      \alpha\ar[|->]{u}\ar[|->]{r} & \alpha_{Y'} (\id{Y'})\ar[|->]{u}
    \end{tikzcd} \]
  Aquí una vez más usamos la naturalidad de $\alpha$:
  \[ \begin{tikzcd}
      \Hom_\vcateg{C} (Y',Y)\ar{r}{\alpha_Y} & F (Y) \\
      \Hom_\vcateg{C} (Y',Y')\ar{u}{f^*}\ar{r}[swap]{\alpha_{Y'}} & F (Y') \ar{u}[swap]{F (f)}
    \end{tikzcd} \]

  Comprobemos ahora la conmutatividad del diagrama

  \[ \begin{tikzcd}
      \Nat (\Hom_\vcateg{C} (-,Y), F)\ar{d}[swap]{\alpha\circ -}\ar{r}{\isom} & F (Y)\ar{d}{\alpha_Y} \\
      \Nat (\Hom_\vcateg{C} (-,Y), G) \ar{r}{\isom} & G (Y)
    \end{tikzcd} \]

  En efecto, para una transformación natural
  $\beta\colon \Hom_\vcateg{C} (-,Y) \Rightarrow F$ tenemos

  \[ \begin{tikzcd}
      \beta\ar[|->]{d}\ar[|->]{r} & \beta_X (\id{X})\ar[|->]{d} \\
      \alpha\circ\beta \ar{r} & (\alpha\circ\beta)_X (\id{X}) = \alpha_X (\beta_X (\id{X}))
    \end{tikzcd} \]
\end{enumerate}
\end{proof}
\end{proposicion}

El siguiente es un caso particular del lema de Yoneda cuando $F$ es el funtor
covariante $\Hom_\vcateg{C} (X,-)$ o el funtor contravariante
$\Hom_\vcateg{C} (-,Y)$.

\begin{corolario}[Encajamiento de Yoneda]
  \label{encajamiento-de-yoneda}
  ~

  \begin{enumerate}
  \item[1)] Para cualesquiera $X, X' \in \Ob(\vcateg{C})$ tenemos una biyección
    natural
    $$\Nat (\Hom_\vcateg{C} (X,-), \Hom_\vcateg{C} (X',-)) \isom \Hom_\vcateg{C} (X',X).$$

  \item[2)] Para cualesquiera $Y, Y' \in \Ob(\vcateg{C})$ tenemos una biyección
    natural
    $$\Nat (\Hom_\vcateg{C} (-,Y), \Hom_\vcateg{C} (-,Y')) \isom \Hom_\vcateg{C} (Y,Y').$$
  \end{enumerate}
\end{corolario}

El último resultado puede ser interpretado de la manera siguiente. Si
$\vcateg{C}$ es una categoría pequeña, entonces para dos funtores
$F,G\colon \vcateg{C}^\mathrm{op} \to \categ{Set}$ las transformaciones
naturales $\Nat (F,G)$ forman un conjunto. Por lo tanto podemos considerar la
categoría
$$\widehat{\vcateg{C}} \dfn \categ{Fun} (\vcateg{C}^\mathrm{op}, \categ{Set}),$$
llamada la \term{categoría de prehaces sobre $\vcateg{C}$}, cuyos objetos son
los funtores $F\colon \vcateg{C}^\mathrm{op} \to \categ{Set}$ y cuyos morfismos
son las transformaciones naturales $F\Rightarrow G$. El lema de Yoneda nos dice
que tenemos un funtor

\begin{align*}
  \mathcal{Y}\colon \vcateg{C} & \to \widehat{\vcateg{C}},\\
  X & \mapsto \Hom_\vcateg{C} (-,X),
\end{align*}
que es fiel y pleno, lo cual quiere decir que para cada par de objetos $X, X'$
tenemos una biyección
\begin{align*}
  \Hom_\vcateg{C} (X, X') & \isom \Hom_{\widehat{\vcateg{C}}} (\mathcal{Y} (X), \mathcal{Y} (X')),\\
  f & \mapsto \mathcal{Y} (f).
\end{align*}

En consecuencia la categoría $\vcateg{C}$ puede ser vista como una subcategoría
plena de la categoría más grande $\widehat{\vcateg{C}}$. El funtor $\mathcal{Y}$
recibe el nombre de \term{encajamiento de Yoneda} y tiene un rol fundamental en
álgebra y geometría. De modo similar, tenemos la versión contravariante

\begin{align*}
  \mathcal{Y}\colon \vcateg{C}^\mathrm{op} & \to \categ{Fun} (\vcateg{C}, \categ{Set}),\\
  X & \mapsto \Hom_\vcateg{C} (X,-).
\end{align*}

\begin{corolario}
  \label{funtores-representados-objeto-unico-salvo-iso}
  ~

  \begin{enumerate}
  \item[1)] Si $F \isom \Hom_\vcateg{C} (X,-)\colon \vcateg{C} \to \categ{Set}$
    para algún objeto $X$, entonces $X$ está definido de manera única salvo
    isomorfismo único.

  \item[2)] Si
    $F \isom \Hom_\vcateg{C} (-,Y)\colon \vcateg{C}^\mathrm{op} \to \categ{Set}$
    para algún objeto $Y$, entonces $Y$ está definido de manera única salvo
    isomorfismo único.
  \end{enumerate}

  \begin{proof}
    Por ejemplo, supongamos que
    $$\Hom_\vcateg{C} (X,-) \isom F \isom \Hom_\vcateg{C} (X',-).$$
    Este isomorfismo de funtores corresponde a un par de transformaciones naturales
    \[ \alpha\colon \Hom_\vcateg{C} (X,-) \Rightarrow \Hom_\vcateg{C} (X',-), \quad
      \beta\colon \Hom_\vcateg{C} (X',-) \Rightarrow \Hom_\vcateg{C} (X,-) \]
    tales que
    \[ \beta\circ\alpha = \Id_{\Hom_\vcateg{C} (X,-)}, \quad
      \alpha\circ\beta = \Id_{\Hom_\vcateg{C} (X',-)}. \]
    Por el encajamiento de Yoneda, $\alpha$ y $\beta$ corresponden a morfismos
    $f\colon X'\to X$ y $g\colon X \to X'$ tales que $g\circ f = \id{X'}$ y
    $f\circ g = \id{X}$.

    \[ \begin{tikzpicture}
        \matrix(m)[matrix of math nodes, row sep=3em, column sep=2em, text height=1.5ex, text depth=0.25ex]
        {\Hom_\vcateg{C} (X,-) & \Hom_\vcateg{C} (X',-)\\
          X & X'\\};

        \draw[->] (m-1-1) edge[double,loop left] node[left,font=\scriptsize] {$\Id$} (m-1-1);
        \draw[->] (m-1-2) edge[double,loop right] node[right,font=\scriptsize] {$\Id$} (m-1-2);

        \draw[double,->] ([yshift=0.5em]m-1-1.east) -- node[above,font=\scriptsize] {$\alpha$} ([yshift=0.5em]m-1-2.west);
        \draw[double,->] ([yshift=-0.5em]m-1-2.west) -- node[below,font=\scriptsize] {$\beta$} ([yshift=-0.5em]m-1-1.east);

        \draw[double,->] (m-2-1) edge[loop left] node[left,font=\scriptsize] {$\idid$} (m-2-1);
        \draw[double,->] (m-2-2) edge[loop right] node[right,font=\scriptsize] {$\idid$} (m-2-2);

        \draw[<-] ([yshift=0.5em]m-2-1.east) -- node[above,font=\scriptsize] {$f$} ([yshift=0.5em]m-2-2.west);
        \draw[<-] ([yshift=-0.5em]m-2-2.west) -- node[below,font=\scriptsize] {$g$} ([yshift=-0.5em]m-2-1.east);
      \end{tikzpicture} \qedhere \]
  \end{proof}
\end{corolario}

\begin{definicion}
  Si para un funtor $F\colon \vcateg{C} \to \categ{Set}$ (resp.
  $F\colon \vcateg{C}^\mathrm{op} \to \categ{Set}$) tenemos un isomorfismo
  $F \isom \Hom_\vcateg{C} (X,-)$ (resp. $F \isom \Hom_\vcateg{C} (-,X)$) para
  algún objeto $X \in \Ob(\vcateg{C})$, se dice que $F$ es un funtor
  \term{representable}\footnote{En el caso contravariante
    $F \isom \Hom_\vcateg{C} (-,X)\colon \vcateg{C}^\mathrm{op} \to \categ{Set}$
    algunos autores dicen que $F$ es \term{correpresentado} por $X$, pero yo no
    voy a usar esta terminología.}, y que $F$ es \term{representado} por
  $X$. (Por el lema de Yoneda, este $X$ es único salvo isomorfismo.)
\end{definicion}

Los funtores representables son un pilar de las matemáticas modernas porque
muchas construcciones universales pueden ser formuladas en términos de
representabilidad. El lema de Yoneda tiene el privilegio de ser uno de los
resultados más tautológicos y profundos al mismo tiempo.

Ejemplos interesantes de funtores representables van a surgir más adelante.

% % % % % % % % % % % % % % % % % % % % % % % % % % % % % %

\section{Funtores adjuntos}

%\marginpar{\footnotesize lección\\4.06.2018}

Sea $V$ un espacio vectorial dotado de un producto escalar
$\langle \cdot,\cdot \rangle$. Recordemos que dos operadores $A\colon V\to V$ y
$A^*\colon V\to V$ son \emph{adjuntos} si para cualesquiera $v_1,v_2 \in V$ se
cumple
$$\langle A v_1, v_2\rangle = \langle v_1, A^* v_2\rangle.$$
Algo parecido existe en la teoría de categorías: es la nocion de funtores adjuntos.

\begin{definicion}
  Sean $F\colon \vcateg{C} \to \vcateg{D}$ y $G\colon \vcateg{D} \to \vcateg{C}$
  dos funtores. Se dice que $F$ es \term{adjunto por la izquierda} a $G$ y que
  $G$ es \term{adjunto por la derecha} a $F$ si para cualesquiera
  $X \in \Ob (\vcateg{C})$ e $Y \in \Ob (\vcateg{D})$ tenemos una biyección
  natural
  $$\Phi_{XY}\colon \Hom_\vcateg{D} (F (X), Y) \xrightarrow{\isom} \Hom_\vcateg{C} (X, G (Y)).$$
  La naturalidad quiere decir que para $X$ fijo la biyección
  $$\Phi_{X,-}\colon \Hom_\vcateg{D} (F (X), -) \xrightarrow{\isom} \Hom_\vcateg{C} (X, G (-))$$
  es un isomorfismo de funtores $\vcateg{D} \to \categ{Set}$, y para $Y$ fijo la
  biyección
  $$\Phi_{-,Y}\colon \Hom_\vcateg{D} (F (-), Y) \xrightarrow{\isom} \Hom_\vcateg{C} (-, G (Y))$$
  es también un isomorfismo de funtores
  $\vcateg{C}^\mathrm{op} \to \categ{Set}$.
\end{definicion}

\begin{ejemplo}
  Los funtores adjuntos aparecen en varios contextos en álgebra y
  geometría. Lamentablemente, no tenemos bastante tiempo para ver muchos
  ejemplos; aquí sugiero algunos y dejo los detalles como ejercicios al lector.

  \begin{enumerate}
  \item[1)] Tenemos el funtor olvidadizo $k\categ{-Vect} \to \categ{Set}$ que
    para cada espacio vectorial $V$ ``olvida'' su estructura y le asocia el
    conjunto subyacente. A cada conjunto $X$ se puede asociar el espacio
    vectorial $k\langle X\rangle$ generado por $X$; es decir, el espacio cuya
    base corresponde a los elementos de $X$. En este caso toda aplicación lineal
    $f\colon k \langle X\rangle\to V$ se define de manera única por los imágenes
    de los elementos de la base.

    \[ \begin{tikzcd}
        X \ar[hookrightarrow]{r}\ar{d}[swap]{f} & k\langle X\rangle \\
        V\ar[dashed]{ur}[swap]{\exists !}
      \end{tikzcd} \]

    Esto nos da una biyección natural
    $$\Hom_{k\categ{-Vect}} (k \langle X\rangle, V) \isom \Hom_\categ{Set} (X,V).$$
    Luego, el funtor $k \langle -\rangle$ es adjunto por la izquierda al funtor
    olvidadizo $k\categ{-Vect} \to \categ{Set}$.

  \item[2)] El funtor olvidadizo $G\categ{-Set}\to \categ{Set}$ tiene un funtor
    adjunto por la izquierda $\categ{Set}\to G\categ{-Set}$ que a cada conjunto
    $X$ asocia el conjunto $G\times X$ con la acción de $G$ mediante
    $g\cdot (h,x) \dfn (gh, x)$. Hay una biyección natural
    $$\Hom_{G\categ{-Set}} (G\times X, Y) \isom \Hom_\categ{Set} (X, Y).$$

  \item[3)] Para cada conjunto fijo $X$ tenemos el funtor
  $$-\times X\colon \categ{Set} \to \categ{Set}$$
  que asocia a un conjunto $Y$ el producto $Y\times X$ y a una aplicación entre
  conjuntos $f\colon Y_1\to Y_2$ la aplicación
  $f\times\idid\colon Y_1\times X\to Y_2\times X$. Este funtor es adjunto por la
  izquierda al funtor
  $$(-)^X \dfn \Hom_\categ{Set} (X,-)\colon \categ{Set} \to \categ{Set}.$$
  En otras palabras, tenemos una biyección natural
  $$\Hom_\categ{Set} (Y\times X, Z) \isom \Hom_\categ{Set} (Y, Z^X).$$
  En efecto, a una aplicación $f\colon Y\times X\to Z$ se puede asociar una aplicación $Y\to Z^X$ definida por
  $$y \mapsto (x \mapsto f (y,x)).$$
  Viceversa, toda aplicación $g\colon Y\to Z^X$ nos da una aplicación $Y\times X\mapsto Z$ definida por
  $$(y,x) \mapsto (g (y)) (x).$$

\item[4)] Si $X$ es un espacio topológico, podemos olvidar su topología y
  considerar a $X$ como un conjunto. Esto define el funtor olvidadizo
  $$Olv\colon \categ{Top} \to \categ{Set}.$$
  Un funtor adjunto a $Olv$ debe ir en la otra dirección: para un conjunto $X$
  definir alguna topología sobre el mismo. De hecho, hay dos modos canónicos de
  hacerlo: definir sobre $X$ la \term{topología discreta}, donde cada
  subconjunto $U\subseteq X$ es abierto, o la \term{topología indiscreta}, donde
  los únicos subconjuntos abiertos son $\emptyset$ y $X$. Esto define dos
  funtores diferentes
  $$Discr, Indiscr\colon \categ{Set} \to \categ{Top}.$$
  Resulta que $Olv$ es adjunto por la izquierda a $Indiscr$ y por la derecha a $Discr$:
  \begin{align*}
    \Hom_\categ{Set} (Olv (X), Y) & \isom \Hom_\categ{Top} (X, Indiscr (Y)),\\
    \Hom_\categ{Top} (Discr (X), Y) & \isom \Hom_\categ{Set} (X, Olv (Y)).
  \end{align*}
  Esto es otro modo de decir que si $Y$ lleva la topología indiscreta, entonces
  cualquier aplicación $X\to Y$ es continua. Si $X$ lleva la topología discreta,
  entonces cualquier aplicación $X\to Y$ es continua.

\item[5)] Tenemos el funtor de inclusión de la categoría de grupos abelianos en
  la categoría de grupos:
  $$i\colon \categ{Ab} \to \categ{Grp}.$$
  Un funtor adjunto a $i$ debe construir un grupo abeliano a partir de un grupo
  $G$ de manera canónica. Como sabemos, tenemos que considerar la
  \term{abelianización}:
  $$G^\mathrm{ab} \dfn G/[G,G].$$
  Aquí $[G,G]$ es el \term{subgrupo conmutador}: el subgrupo generado por todos
  los conmutadores:
  $$[G,G] \dfn \langle [x,y] \dfn xyx^{-1}y^{-1} \mid x,y\in G \rangle.$$
  Es un subgrupo normal, y por ende se puede pasar al grupo cociente $G/[G,G]$
  que será abeliano. La abelianización tiene la siguiente propiedad universal:
  todo homomorfismo $f\colon G\to A$ donde $A$ es un grupo abeliano se factoriza
  de modo único por $G^{ab}$:
  \[ \begin{tikzcd}
      G \ar[twoheadrightarrow]{d}\ar{r}{f} & A \\
      G^{ab}\ar[dashed]{ur}[swap]{\exists !}
    \end{tikzcd} \]
  (ya que $A$ es abeliano, todos los conmutadores están en el núcleo de $f$).

  La abelianización es un funtor $\categ{Grp} \to \categ{Ab}$ que es adjunto por
  la izquierda a $i$:
  $$\Hom_\categ{Ab} (G^\mathrm{ab}, A) \isom \Hom_\categ{Grp} (G, i (A)).$$

\item[6)] Si $R$ es un anillo, entonces sus \term{unidades} (elementos
  invertibles) forman un grupo $R^\times$. Esto nos da un funtor
  $$(-)^\times\colon \categ{Ring} \to \categ{Grp}.$$
  Un funtor adjunto debe construir cierto anillo a partir de un grupo $G$ de
  manera canónica. Es la construcción del anillo $\ZZ [G]$ que consiste de las
  sumas finitas formales $\sum_{g\in G} n_g \, g$, donde la multiplicación está
  definida por la distributividad y la multiplicación en $G$. Esto es un funtor
  $$\ZZ [-]\colon \categ{Grp} \to \categ{Ring},$$
  que es adjunto por la izquierda a $(-)^\times$:
  \[ \Hom_\categ{Ring} (\ZZ [G], R) \isom
     \Hom_\categ{Grp} (G, R^\times). \qedhere \]
  \end{enumerate}
\end{ejemplo}

Los funtores adjuntos están relacionados con los funtores representables de la
siguiente manera.

\begin{proposicion}
  \label{representabilidad-y-adjunciones}
  ~

  \begin{enumerate}
  \item[1)] Para un funtor $F\colon \vcateg{C} \to \vcateg{D}$ existe un adjunto
    por la derecha si y solamente si para todo $Y \in \Ob (\vcateg{D})$ el
    funtor
    \begin{align*}
  \Hom_\vcateg{D} (F (-), Y)\colon \vcateg{C}^\mathrm{op} & \to \categ{Set},\\
      X & \mapsto \Hom_\vcateg{D} (F (X), Y)
    \end{align*}
    es representable, es decir isomorfo a $\Hom_\vcateg{C} (-, X')$ para algún
    $X'\in \Ob (\vcateg{C})$.

  \item[2)] Para un funtor $G\colon \vcateg{D} \to \vcateg{C}$ existe un adjunto
    por la izquierda si y solamente si para todo $X \in \Ob (\vcateg{C})$ el
    funtor
    \begin{align*}
      \Hom_\vcateg{C} (X, G (-))\colon \vcateg{D} & \to \categ{Set},\\
      Y & \mapsto \Hom_\vcateg{C} (X, G (Y))
    \end{align*}
    es representable, es decir isomorfo a $\Hom_\vcateg{D} (Y', -)$ para algún
    $Y'\in \Ob (\vcateg{D})$.
  \end{enumerate}

  \begin{proof}
    Por ejemplo, veamos la primera parte. Si $F$ es adjunto por la izquierda a
    $G$, entonces para todo $Y\in \Ob (\vcateg{D})$ tenemos un isomorfismo
    natural
    $$\Phi_{-,Y}\colon \Hom_\vcateg{D} (F (-), Y) \isom \Hom_\vcateg{C} (-, G(Y))$$
    y $X' \dfn G (Y)$ representa el funtor
    $\Hom_\vcateg{D} (F (-), Y)$. Recíprocamente, supongamos que para todo
    objeto $Y \in \Ob (\vcateg{D})$ tenemos isomorfismos de funtores
    $$\Phi_{-,Y}\colon \Hom_\vcateg{D} (F (-), Y) \isom \Hom_\vcateg{C} (-, X').$$
    Sea $G (Y) \dfn X'$. Un morfismo $f\colon Y_1 \to Y_2$ en $\vcateg{D}$
    induce una transformación natural entre funtores
    \[ \Hom_\vcateg{C} (-, X'_1) \isom \Hom_\vcateg{D} (F (-), Y_1) \xRightarrow{f\circ -}
      \Hom_\vcateg{D} (F (-), Y_2) \isom \Hom_\vcateg{C} (-, X'_2), \]
    que por el encajamiento de Yoneda (\ref{encajamiento-de-yoneda}) corresponde
    a un único morfismo $X'_1 \to X'_2$. Esto define un morfismo
    $G (f)\colon G (X_1) \to G (X_2)$, y se ve que definido de esta manera, $G$
    es un funtor $\vcateg{D} \to \vcateg{C}$.
  \end{proof}
\end{proposicion}

\begin{proposicion}[Uno de los adjuntos define al otro, salvo isomorfismo]
  ~

  \begin{enumerate}
  \item[1)] Si $F\colon \vcateg{C} \to \vcateg{D}$ es adjunto por la izquierda a
    dos funtores $G, G'\colon \vcateg{D} \to \vcateg{C}$, entonces $G \isom G'$.

  \item[2)] Si $G\colon \vcateg{D} \to \vcateg{C}$ es adjunto por la derecha a
    dos funtores $F, F'\colon \vcateg{C} \to \vcateg{D}$, entonces $F \isom F'$.
  \end{enumerate}

  \begin{proof}
    Demostremos la primera parte; la segunda es idéntica. Si tenemos biyecciones
    naturales
    \[ \Phi_{XY}\colon \Hom_\vcateg{D} (F (X), Y) \xrightarrow{\isom} \Hom_\vcateg{C} (X, G (Y)), \quad
       \Phi'_{XY}\colon \Hom_\vcateg{D} (F (X), Y) \xrightarrow{\isom} \Hom_\vcateg{C} (X, G' (Y)), \]
    entonces los isomorfismos de funtores
    $$\Phi'_{-,Y}\circ\Phi_{-,Y}^{-1}\colon \Hom_\vcateg{C} (-, G (Y))\Rightarrow \Hom_\vcateg{C} (-, G' (Y))$$
    bajo el lema de Yoneda
    (\ref{funtores-representados-objeto-unico-salvo-iso}) corresponden a
    isomorfismos $\alpha_Y\colon G (Y) \xrightarrow{\isom} G' (Y)$ para cada
    $Y$. Notamos que los diagramas
    \begin{equation}
      \label{eqn:alpha-conmuta-con-biyecciones}
      \begin{tikzcd}
        & \Hom_\vcateg{D} (F(X), Y) \ar{dl}[swap]{\Phi_{XY}}{\isom}\ar{dr}{\Phi'_{XY}}[swap]{\isom} \\
        \Hom_\vcateg{C} (X,G (Y)) \ar{rr}[swap]{\alpha_{Y*}} & & \Hom_\vcateg{C} (X,G' (Y))
      \end{tikzcd}
    \end{equation}
    conmutan.

    Para obtener un isomorfismo de funtores $G \isom G'$, falta verificar que
    los $\alpha_Y$ definen una transformación natural, es decir que para cada
    morfismo $f\colon Y\to Y'$ el siguiente diagrama es conmutativo:

    \[ \begin{tikzcd}
        G (Y) \ar{r}{\alpha_Y}\ar{d}[swap]{G (f)} & G' (Y)\ar{d}{G' (f)} \\
        G (Y') \ar{r}[swap]{\alpha_{Y'}} & G' (Y') \\
      \end{tikzcd} \]

    Pero, también gracias a Yoneda, este diagrama es conmutativo si y solamente si
    para todo $X$ el diagrama

    \[ \begin{tikzcd}
        \Hom_\vcateg{C} (X,G (Y)) \ar{r}{\alpha_{Y*}}\ar{d}[swap]{G (f)_*} & \Hom_\vcateg{C} (X,G' (Y))\ar{d}{G' (f)_*} \\
        \Hom_\vcateg{C} (X,G (Y')) \ar{r}[swap]{\alpha_{Y' *}} & \Hom_\vcateg{C} (X,G' (Y')) \\
      \end{tikzcd} \]
    es conmutativo. Gracias a la naturalidad de las biyecciones $\Phi_{X,-}$ y
    $\Phi'_{X,-}$, en el siguiente cubo
    \[ \begin{tikzcd}
        & \Hom_\vcateg{D} (F(X), Y) \ar{rr}{\idid}\ar[]{dd}[near end]{f_*}\ar{dl}{\Phi_{XY}}[swap]{\isom} && \Hom_\vcateg{D} (F (X), Y)\ar{dd}{f_*}\ar{dl}{\Phi'_{XY}}[swap]{\isom} \\
        \Hom_\vcateg{C} (X,G (Y)) \ar[crossing over]{rr}[near end]{\alpha_{Y*}}\ar{dd}[swap]{G (f)_*} && \Hom_\vcateg{C} (X,G' (Y)) \\
        & \Hom_\vcateg{D} (F (X), Y') \ar{rr}[near start]{\idid}\ar{dl}{\Phi_{XY'}}[swap]{\isom} && \Hom_\vcateg{D} (F (X), Y')\ar{dl}{\Phi'_{XY'}}[swap]{\isom} \\
        \Hom_\vcateg{C} (X,G (Y')) \ar{rr}[swap]{\alpha_{Y' *}} && \Hom_\vcateg{C} (X,G' (Y'))\ar[leftarrow, crossing over]{uu}[near start,swap]{G' (f)_*} \\
      \end{tikzcd} \]
    la cara izquierda y derecha conmutan; las caras de arriba y de abajo
    conmutan por \eqnref{eqn:alpha-conmuta-con-biyecciones}, y la cara posterior
    conmuta trivialmente. Entonces, se puede concluir que la cara delantera
    conmuta.
  \end{proof}
\end{proposicion}

Los funtores adjuntos tienen muchas propiedades útiles. Una de estas es
preservación de mono- y epimorfismos.

\begin{proposicion}
  Sea $F\colon \vcateg{C} \to \vcateg{D}$ un funtor adjunto por la izquierda a
  $G\colon \vcateg{D}\to \vcateg{C}$. Entonces,

  \begin{enumerate}
  \item[1)] $F$ \term{preserva epimorfismos}: para todo epimorfismo
    $f\colon X\epi X'$ en $\vcateg{C}$ el morfismo $F (f)\colon F(X) \to F(X')$
    es un epimorfismo en $\vcateg{D}$;

  \item[2)] $G$ \term{preserva monomorfismos}: para todo monomorfismo
    $g\colon Y\mono Y'$ en $\vcateg{D}$ el morfismo $G (g)\colon G(Y) \to G(Y')$
    es un monomorfismo en $\vcateg{C}$.
  \end{enumerate}

  \begin{proof}
    Por ejemplo, en 1), para un objeto $Z \in \Ob (\mathcal{D})$, consideremos
    el diagrama conmutativo
    $$\begin{tikzcd}
      \Hom_\vcateg{C} (X', G (Z)) \ar{r}{f^*}\ar{d}[swap]{\isom} & \Hom_\vcateg{C} (X, G (Z))\ar{d}{\isom} \\
      \Hom_\vcateg{D} (F (X'), Z) \ar{r}{F (f)^*} & \Hom_\vcateg{D} (F (X), Z)
    \end{tikzcd}$$
    Si $f$ es un epimorfismo, entonces la flecha $f^*$ es inyectiva, y luego
    $F (f)^*$ es también inyectiva. La inyectividad de $F (f)^*$ para cualquier
    $Z$ precisamente significa que $F (f)$ es un epimorfismo. La parte 2) se
    deja como un ejercicio.
  \end{proof}
\end{proposicion}

% % % % % % % % % % % % % % % % % % % % % % % % % % % % % %

\section{Unidad y counidad de una adjunción}

A veces es útil otra descripción de adjunción de funtores.

\begin{proposicion}
  \label{unidad-y-counidad}
  Consideremos una adjunción
  $$\Hom_\vcateg{D} (F (X), Y) \isom \Hom_\vcateg{C} (X, G (Y)).$$
  En particular, tenemos
  \begin{align*}
    \Hom_\vcateg{D} (F (X), F(X)) & \isom \Hom_\vcateg{C} (X, GF (X)),\\
    \Hom_\vcateg{D} (FG (Y), Y) & \isom \Hom_\vcateg{C} (G (Y), G (Y)).
  \end{align*}

  \begin{enumerate}
  \item[1)] Sea $\eta_X\colon X\to GF (X)$ el morfismo que corresponde al
    morfismo identidad $\idid\colon F(X) \to F(X)$ bajo la primera biyección.

  \item[2)] Sea $\epsilon_Y\colon FG (Y) \to Y$ el morfismo que corresponde al
    morfismo identidad $\idid\colon G (Y) \to G (Y)$ bajo la segunda biyección.
  \end{enumerate}

  Entonces los $\eta_X$ definen una transformación natural
  $\Id_\vcateg{C} \Rightarrow G\circ F$ (la \term{unidad de la adjunción}) y los
  $\epsilon_Y$ definen una transformación natural
  $F\circ G\Rightarrow \Id_\vcateg{D}$ (la \term{counidad de la adjunción}), y
  la adjunción puede ser escrita como
  \begin{align*}
    \Hom_\vcateg{D} (F (X), Y) & \isom \Hom_\vcateg{C} (X, G (Y)),\\
    (F (X) \xrightarrow{f} Y) & \mapsto (X\xrightarrow{\eta_X} GF (X) \xrightarrow{G (f)} G (Y)),\\
    (F (X)\xrightarrow{F (g)} FG (Y) \xrightarrow{\epsilon_Y} Y) & \mapsfrom (X\xrightarrow{g} G (Y)).
  \end{align*}

  \begin{proof}
    Por ejemplo, para ver que $\eta_X\colon X\to GF (X)$ define una
    transformación natural, tenemos que ver que los siguientes diagramas son
    conmutativos para cada morfismo $g\colon X\to X'$:
    \[ \begin{tikzcd}
        X\ar{r}{\eta_X}\ar{d}[swap]{g} & GF (X)\ar{d}{GF (g)} \\
        X'\ar{r}[swap]{\eta_{X'}} & GF (X') \\
      \end{tikzcd} \]

    De hecho, por la definición de $\eta_X$ y la naturalidad de adjunción,
    tenemos el diagrama conmutativo
    \[ \begin{tikzcd}
        \Hom_\vcateg{D} (F (X), F (X))\ar{r}{\isom}\ar{d}[swap]{F (g)\circ -} & \Hom_\vcateg{C} (X, GF (X))\ar{d}{GF (g)\circ -} & \id{F(X)}\ar[|->]{d}\ar[|->]{r} & \eta_X\ar[|->]{d} \\
        \Hom_\vcateg{D} (F (X), F (X'))\ar{r}{\isom} & \Hom_\vcateg{C} (X, GF (X')) & F (g) \ar[|->]{r} & GF (g) \circ \eta_X = \eta_{X'}\circ g \\
        \Hom_\vcateg{D} (F (X'), F (X'))\ar{r}{\isom}\ar{u}{-\circ F (g)} & \Hom_\vcateg{C} (X', GF (X'))\ar{u}[swap]{-\circ g} & \id{F(X')}\ar[|->]{r}\ar[|->]{u} & \eta_{X'} \ar[|->]{u} \\
      \end{tikzcd} \]

    De modo similar, todo $f\colon F (X)\to Y$ nos da el diagrama conmutativo
    \[ \begin{tikzcd}
        \Hom_\vcateg{D} (F (X), F (X))\ar{r}{\isom}\ar{d}[swap]{f\circ -} & \Hom_\vcateg{C} (X, GF (X))\ar{d}{G (f)\circ -} & \id{F(X)}\ar[|->]{d}\ar[|->]{r} & \eta_X\ar[|->]{d} \\
        \Hom_\vcateg{D} (F (X), Y)\ar{r}{\isom} & \Hom_\vcateg{C} (X, G (Y)) & f \ar[|->]{r} & G (f) \circ \eta_X
      \end{tikzcd} \]

    Luego, $f$ corresponde a $G (f) \circ \eta_X$. La verificación de que
    $\epsilon$ es una transformación natural
    $F\circ G\Rightarrow \Id_\vcateg{D}$ y que $g\colon X \to G (Y)$ corresponde
    a $\epsilon_Y\circ F (g)$ es similar.
  \end{proof}
\end{proposicion}

%\marginpar{\footnotesize lección\\5.06.2018}

\begin{proposicion}
  Sea $F\colon \vcateg{C} \to \vcateg{D}$ un funtor adjunto por la izquierda a
  $G\colon \vcateg{D} \to \vcateg{C}$. Entonces

  \begin{enumerate}
  \item[1)] la unidad de la adjunción $\eta_X\colon X\to GF (X)$ es un
    monomorfismo para todo $X\in \Ob (\vcateg{C})$ si y solamente si $F$ es
    fiel;

  \item[2)] la counidad de la adjunción $\epsilon_Y\colon FG (Y)\to Y$ es un
    epimorfismo para todo $Y\in \Ob (\vcateg{D})$ si y solamente si $G$ es fiel.
  \end{enumerate}

  \begin{proof}
    Para todo morfismo $f\colon X'\to X$, bajo la adjunción a $F (f)$
    corresponde el morfismo $\eta_X\circ f$:
    \[ \begin{tikzcd}
        \Hom_\vcateg{D} (F (X), F (X))\ar{r}{\isom}\ar{d}[swap]{-\circ F (f)} & \Hom_\vcateg{C} (X, GF (X))\ar{d}{-\circ f} & \id{F(X)}\ar[|->]{d}\ar[|->]{r} & \eta_X\ar[|->]{d} \\
        \Hom_\vcateg{D} (F (X'), F (X))\ar{r}[swap]{\isom} & \Hom_\vcateg{C} (X', GF (X)) & F (f) \ar[|->]{r} & \eta_X\circ f
      \end{tikzcd} \]

    De la misma manera, para todo $g\colon Y\to Y'$, a $G (g)$ corresponde el
    morfismo $g\circ \epsilon_Y$:
    \[ \begin{tikzcd}
        \Hom_\vcateg{D} (FG (Y), Y)\ar{r}{\isom}\ar{d}[swap]{g\circ -} & \Hom_\vcateg{C} (G(Y), G (Y))\ar{d}{G (g)\circ -} & \epsilon_Y\ar[|->]{d}\ar[|->]{r} & \id{G (Y)}\ar[|->]{d} \\
        \Hom_\vcateg{D} (FG (Y), Y')\ar{r}[swap]{\isom} & \Hom_\vcateg{C} (G (Y), G (Y')) & g\circ \epsilon_Y \ar[|->]{r} & G (g)
      \end{tikzcd} \]

    Tenemos aplicaciones entre conjuntos
    \[ \begin{tikzcd}[row sep=0.1em]
        \Hom_\vcateg{C} (X', X) \ar{r}{\eta_{X*}} & \Hom_\vcateg{C} (X', GF (X)) \ar{r}{\isom} & \Hom_\vcateg{D} (F (X'), F (X)) \\
        f \ar[|->]{r} & \eta_X\circ f \ar[|->]{r} & F (f) \\
        ~ & ~ & ~ \\
        \Hom_\vcateg{D} (Y, Y') \ar{r}{\epsilon_Y^*} & \Hom_\vcateg{D} (FG (Y), Y') \ar{r}{\isom} & \Hom_\vcateg{C} (G (Y), G (Y')) \\
        g \ar[|->]{r} & g\circ \epsilon_Y \ar[|->]{r} & G (g)
      \end{tikzcd} \]
    $\eta_X$ es mono para todo $X\in \Ob (\vcateg{C})$ si y solamente si la
    aplicación $\eta_{X*}$ de arriba es inyectiva para todo
    $X,X'\in \Ob (\vcateg{C})$, si y solamente si $F$ es fiel. Asimismo,
    $\epsilon_Y$ es epi para todo $Y\in \Ob (\vcateg{D})$ si y solamente si la
    aplicación $\epsilon_Y^*$ es inyectiva para todo $Y,Y'\in \Ob (\vcateg{D})$,
    si y solamente si $G$ es fiel.
  \end{proof}
\end{proposicion}

\begin{ejemplo}
  Consideremos la adjunción entre la consrucción de $k$-espacios vectoriales
  generados por un conjunto y el funtor olvidadizo
  $k\categ{-Vect} \to \categ{Set}$. El funtor
  $X \rightsquigarrow k \langle X\rangle$ es obviamente fiel: diferentes
  aplicaciónes entre conjuntos $X\to Y$ inducen diferentes aplicaciones lineales
  $k\langle X\rangle \to k\langle Y\rangle$. Esto implica que la unidad de la
  adjunción $\eta_X\colon X\to k\langle X\rangle$ es inyectiva para todo $X$.
  En efecto, $\eta_X$ es simplemente la inclusión de los generadores en
  $k\langle X\rangle$.
\end{ejemplo}

% % % % % % % % % % % % % % % % % % % % % % % % % % % % % %

\section{Caracterización de adjunciones por las identidades triangulares}

\begin{proposicion}[Identidades triangulares]
  Sean $F\colon \vcateg{C}\to\vcateg{D}$ y $G\colon \vcateg{D}\to\vcateg{C}$ dos
  funtores adjuntos y sean $\eta\colon \Id_\vcateg{C} \Rightarrow G\circ F$ y
  $\epsilon\colon F\circ G\Rightarrow \Id_\vcateg{D}$ la unidad y counidad
  correspondiente. Entonces, se cumplen las identidades

  \begin{equation}
    \label{eqn:identidades-triangulares}
    (\epsilon\circ F)\circ (F\circ \eta) = \Id_F, \quad
    (G\circ \epsilon)\circ (\eta\circ G) = \Id_G.
  \end{equation}

  \begin{equation}
    \label{eqn:identidades-triangulares-diag}
    \begin{tikzcd}
      F \ar[double,->]{r}{F\circ\eta}\ar[double,->]{dr}[swap]{\Id} & FGF\ar[double,->]{d}{\epsilon\circ F} & & G \ar[double,->]{r}{\eta\circ G}\ar[double,->]{dr}[swap]{\Id} & GFG\ar[double,->]{d}{G\circ \epsilon} \\
      & F && & G
    \end{tikzcd}
  \end{equation}

  Aquí $(\epsilon\circ F)_X \dfn \epsilon_{F (X)}$,
  $(F\circ \eta)_X \dfn F (\eta_X)$, $(\eta\circ G)_Y \dfn \eta_{G (Y)}$,
  $(G\circ\epsilon)_Y \dfn G (\epsilon_Y)$ (véase
  \S\ref{sec:producto-de-godement}).
\end{proposicion}

Las fórmulas \eqnref{eqn:identidades-triangulares} se conocen como las
\term{identidades triangulares}.

\begin{proof}
  Según \ref{unidad-y-counidad}, la biyección de adjunción viene dada por
  \begin{align*}
    \Phi_{XY}\colon\Hom_\vcateg{D} (F (X), Y) & \isom \Hom_\vcateg{C} (X, G (Y)),\\
    (F (X) \xrightarrow{f} Y) & \mapsto (X\xrightarrow{\eta_X} GF (X) \xrightarrow{G (f)} G (Y)),\\
    (F (X)\xrightarrow{F (g)} FG (Y) \xrightarrow{\epsilon_Y} Y) & \mapsfrom (X\xrightarrow{g} G (Y)).
  \end{align*}

  Los diagramas \eqnref{eqn:identidades-triangulares-diag} significan que para
  todo objeto $X\in \Ob (\vcateg{C})$ y todo objeto $Y\in \Ob (\vcateg{D})$ se
  tiene

  \[ \begin{tikzcd}
      F (X) \ar{r}{F (\eta_X)}\ar{dr}[swap]{\idid} & FGF (X)\ar{d}{\epsilon_{F (X)}} & & G (Y) \ar{r}{\eta_{G (Y)}}\ar{dr}[swap]{\idid} & GFG (Y)\ar{d}{G (\epsilon_Y)} \\
      & F (X) && & G (Y)
    \end{tikzcd} \]

  Efectivamente, para el morfismo $\id{F (X)}\colon F (X)\to F (X)$ se cumple
  \begin{multline*}
    \id{F (X)} =
    \Phi_{X,F(X)}^{-1}\circ\Phi_{X,F(X)} (\id{F (X)}) =
    \Phi_{X,F(X)}^{-1} (G (\id{F (X)}) \circ \eta_{F (X)}) =
    \Phi_{X,F(X)}^{-1} (\id{GF (X)} \circ \eta_{F (X)}) \\
  = \Phi_{X,F(X)}^{-1} (\eta_{F (X)}) = \eta_{F (X)}\circ F (\eta_X),
  \end{multline*}
  y para $\id{G (Y)}\colon G (Y)\to G (Y)$ se cumple
  \begin{multline*}
    \id{G (Y)} =
    \Phi_{G(Y),Y}\circ\Phi_{G(Y),Y}^{-1} (\id{G (Y)}) =
    \Phi_{G(Y),Y} (\epsilon_Y\circ F (\id{G (Y)})) =
    \Phi_{G(Y),Y} (\epsilon_Y\circ \id{FG (X)}) \\
  = \Phi_{G(Y),Y} (\epsilon_Y) = G (\epsilon_Y)\circ \eta_{G (Y)}.
  \end{multline*}
\end{proof}

Las unidades triangulares nos dan una caracterización de adjunciones que puede
ser usada como una definición alternativa.

\begin{proposicion}
  Sean $F\colon \vcateg{C}\to\vcateg{D}$ y $G\colon \vcateg{D}\to\vcateg{C}$ dos
  funtores y sean $\eta\colon \Id_\vcateg{C} \Rightarrow G\circ F$ y
  $\epsilon\colon F\circ G\Rightarrow \Id_\vcateg{D}$ dos transformaciones
  naturales que cumplen las identidades triangulares
  \[ (\epsilon\circ F)\circ (F\circ \eta) = \Id_F, \quad
     (G\circ \epsilon)\circ (\eta\circ G) = \Id_G; \]
  es decir,
  \[ \epsilon_{F (X)}\circ F (\eta_X) = \id{X}, \quad
     G (\epsilon_Y)\circ \eta_{G (Y)} = \id{Y} \]
  para cualesquiera $X\in \Ob (\vcateg{C})$, $Y\in \Ob (\vcateg{D})$.
  Entonces, $F$ es adjunto por la izquierda a $G$.

\begin{proof}
  Necesitamos definir una biyección natural
  $$\Phi_{XY}\colon\Hom_\vcateg{D} (F (X), Y) \isom \Hom_\vcateg{C} (X, G (Y)).$$
  Las consideraciones de arriba sugieren que hay que poner
  $$\Phi_{XY} (f) \dfn G (f)\circ \eta_X, \quad \Phi_{XY}^{-1} (g) \dfn \epsilon_Y\circ F (g).$$
  Estas aplicaciones son mutualmente inversas:
  \[ \Phi_{XY}^{-1} (\Phi_{XY} (f)) =
     \Phi_{XY}^{-1} (G (f)\circ \eta_X) =
     \epsilon_Y\circ F (G (f)\circ \eta_X) =
     \epsilon_Y\circ FG (f) \circ F (\eta_X) =
     f\circ \epsilon_{F (X)} \circ F (\eta_X) = f. \]
  Aquí hemos usado la naturalidad de $\epsilon$:
  \[ \begin{tikzcd}
      FGF (X)\ar{r}{\epsilon_{F (X)}}\ar{d}[swap]{FG (f)} & F (X)\ar{d}{f}\\
      FG (Y)\ar{r}[swap]{\epsilon_Y} & Y
    \end{tikzcd} \]
  De modo similar,
  \[ \Phi_{XY} (\Phi_{XY}^{-1} (g)) =
     \Phi_{XY} (\epsilon_Y\circ F (g)) =
     G (\epsilon_Y\circ F (g))\circ\eta_X =
     G (\epsilon_Y)\circ GF (g)\circ\eta_X =
     G (\epsilon_Y)\circ \eta_{G (Y)}\circ g = g, \]
  usando la naturalidad de $\eta$:
  \[ \begin{tikzcd}
      X\ar{d}[swap]{g}\ar{r}{\eta_X} & GF (X)\ar{d}{GF (g)}\\
      G (Y)\ar{r}[swap]{\eta_{G (Y)}} & GFG (Y)
    \end{tikzcd} \]
  Falta ver que la biyección $\Phi_{XY}$ definida de esta manera es natural. La
  naturalidad en $X$ quiere decir que para todo morfismo $h\colon X'\to X$ el
  diagrama
  \[ \begin{tikzcd}
      \Hom_\vcateg{D} (F (X), Y)\ar{r}{\Phi_{XY}}\ar{d}[swap]{F (h)^*} & \Hom_\vcateg{C} (X, G (Y))\ar{d}{h^*} \\
      \Hom_\vcateg{D} (F (X'), Y)\ar{r}[swap]{\Phi_{X'Y}} & \Hom_\vcateg{C} (X', G (Y))
    \end{tikzcd} \]
  conmuta. En efecto,
  $$h^* \circ \Phi_{XY} (h) = h^* (G (f)\circ \eta_X) = G (f)\circ \eta_X\circ h$$
  y
  \[ \Phi_{X'Y}\circ F (h)^* (f) =
     \Phi_{X'Y} (f\circ F (h)) =
     G (f\circ F (h))\circ \eta_{X'} =
     G (f)\circ GF (h) \circ \eta_{X'}. \]
   Tenemos $\eta_X\circ h = GF (h) \circ \eta_{X'}$ gracias a la naturalidad de
   $\eta$:
   \[ \begin{tikzcd}
       X' \ar{r}{\eta_{X'}}\ar{d}[swap]{h} & GF (X')\ar{d}{GF (h)} \\
       X \ar{r}[swap]{\eta_X} & GF (X) \\
     \end{tikzcd} \]

   Para comprobar la naturalidad en $Y$, será mejor usar la aplicación inversa
   $\Phi_{XY}^{-1}$ y comprobar que para todo $h\colon Y\to Y'$ el diagrama
   \[ \begin{tikzcd}
       \Hom_\vcateg{C} (X, G (Y))\ar{d}[swap]{G (h)_*}\ar{r}{\Phi_{XY}^{-1}} & \Hom_\vcateg{D} (F (X), Y)\ar{d}{h_*}  \\
       \Hom_\vcateg{C} (X, G (Y'))\ar{r}[swap]{\Phi_{XY'}^{-1}} & \Hom_\vcateg{D} (F (X), Y')
     \end{tikzcd} \]
   conmuta. Tenemos
   $$h_*\circ \Phi_{XY}^{-1} (g) = h_* (\epsilon_Y\circ F (g)) = h\circ \epsilon_Y\circ F (g)$$
   y
   \[ \Phi_{XY'}^{-1}\circ G (h)_* (g) =
      \Phi_{XY'}^{-1} (G (h) \circ g) =
      \epsilon_{Y'}\circ F (G (h) \circ g) =
      \epsilon_{Y'}\circ FG (h) \circ F (g), \]
   y nos queda notar que $h\circ \epsilon_Y = \epsilon_{Y'}\circ FG (h)$:
   \[ \begin{tikzcd}
       FG (Y)\ar{r}{\epsilon_Y}\ar{d}[swap]{FG (h)} & Y\ar{d}{h} \\
       FG (Y')\ar{r}[swap]{\epsilon_{Y'}} & Y' \\
     \end{tikzcd} \]
 \end{proof}
\end{proposicion}

% % % % % % % % % % % % % % % % % % % % % % % % % % % % % %

\section{Caracterización de funtores adjuntos en términos de propiedades universales}

\begin{observacion}[Propiedad universal de la unidad y counidad]
  Sean $F\colon \vcateg{C}\to\vcateg{D}$ y $G\colon \vcateg{D}\to\vcateg{C}$ dos
  funtores adjuntos y sean $\eta\colon \Id_\vcateg{C} \Rightarrow G\circ F$ y
  $\epsilon\colon F\circ G\Rightarrow \Id_\vcateg{D}$ la unidad y counidad
  correspondientes.

\begin{enumerate}
\item[1)] Para todo morfismo $f\colon F (X) \to Y$ existe un morfismo único
  $\widetilde{f}\colon X\to G (Y)$ tal que
  $\epsilon_Y \circ F (\widetilde{f}) = f$:

  \[ \begin{tikzcd}
      F (X)\ar{rr}{f}\ar{dr}[swap]{F (\widetilde{f})} & & Y \\
      & FG (Y)\ar{ur}[swap]{\epsilon_Y}
    \end{tikzcd} \]

\item[2)] Para todo morfismo $g\colon X\to G (Y)$ existe un morfismo único
  $\widehat{g}\colon F (X) \to Y$ tal que $G (\widehat{g})\circ \eta_X = g$:

  \[ \begin{tikzcd}
      X\ar{rr}{g}\ar{dr}[swap]{\eta_X} & & G (Y) \\
      & GF (Y)\ar{ur}[swap]{G (\widehat{g})}
    \end{tikzcd} \]
\end{enumerate}

\begin{proof}
  Sigue del hecho de que la biyección de adjunción viene dada por
  \begin{align*}
    \Hom_\vcateg{D} (F (X), Y) & \isom \Hom_\vcateg{C} (X, G (Y)),\\
    f & \mapsto \widetilde{f} \dfn G (f)\circ \eta_X,\\
    \widehat{g} \dfn \epsilon_Y\circ F (g) & \mapsfrom g
  \end{align*}
  ---véase \ref{unidad-y-counidad}.
\end{proof}
\end{observacion}

Resulta que cada una de las propiedades universales de arriba caracteriza una
adjunción.

\begin{proposicion}
  ~

  \begin{enumerate}
  \item[1)] Sea $F\colon \vcateg{C}\to\vcateg{D}$ un funtor. Supongamos que para
    todo $Y\in\Ob (\vcateg{D})$ existe un objeto $G (Y)$ y un morfismo
    $\epsilon_Y\colon F (G (Y)) \to Y$ que satisfacen la siguiente propiedad
    universal: para todo morfismo $f\colon F (X) \to Y$ existe un morfismo único
    $\widetilde{f}\colon X\to G (Y)$ tal que el diagrama de abajo conmuta:
    \[ \begin{tikzcd}
        F (X) \ar{rr}{f}\ar[dashed]{dr}[swap]{F (\widetilde{f})} && Y \\
        & F (G (Y))\ar{ur}[swap]{\epsilon_Y}
      \end{tikzcd} \]
    Entonces, $G\colon Y\rightsquigarrow G (Y)$ define un funtor adjunto por la
    derecha a $F$ y $\epsilon$ es la counidad de la adjunción.

  \item[2)] Sea $G\colon \vcateg{D}\to\vcateg{C}$ un funtor tal que para todo
    $X\in \Ob (\vcateg{C})$ existe un objeto $F (X)\in \Ob (\vcateg{D})$ y un
    morfismo $\eta_X\colon X\to G (F (X))$ que satisfacen la siguiente propiedad
    universal: para todo morfismo $g\colon X \to G (Y)$ existe un morfismo único
    $\widehat{g}\colon F (X) \to Y$ tal que el diagrama de abajo conmuta:
    \[ \begin{tikzcd}
        X \ar{rr}{g}\ar{dr}[swap]{\eta_X} && G (Y) \\
        & G (F (X))\ar[dashed]{ur}[swap]{G (\widehat{g})}
      \end{tikzcd} \]
    Entonces, $F\colon X\rightsquigarrow F (X)$ define un funtor adjunto por la
    izquierda a $G$ y $\eta$ es la unidad de la adjunción.
  \end{enumerate}

  \begin{proof}
    Por ejemplo, analicemos el segundo caso.

    Primero, hay que ver que $F$ es un funtor. Para definir $F$ sobre las
    flechas, para un morfismo $\phi\colon X\to X'$ consideremos la composición
    $\eta_{X'}\circ \phi\colon X\to GF (X')$. Entonces, existe un morfismo único
    $$F (\phi) \dfn \widehat{\eta_{X'}\circ \phi}\colon F(X) \to F (X')$$
    tal que
    $$G (F (\phi))\circ \eta_X = \eta_{X'}\circ \phi.$$
    Esto será nuestra definición de $F (\phi)$.
    \[ \begin{tikzcd}
        X \ar{rr}{\eta_{X'}\circ \phi}\ar{dr}[swap]{\eta_X} && G (F (X')) \\
        & G (F (X))\ar[dashed]{ur}[swap]{G (F (\phi))}
      \end{tikzcd} \]
    En particular, si $X = X'$ y $\phi = \id{X}$, entonces
    $$F (\id{X}) = \id{F (X)}$$
    por la unicidad. Ahora para $\phi\colon X\to X'$ y $\psi\colon X'\to X''$ el
    morfismo $F (\psi)\colon F (X') \to F (X'')$ es el único morfismo
    caracterizado por
    $$G (F (\psi))\circ \eta_{X'} = \eta_{X''}\circ \psi$$
    y $F (\psi\circ\phi)\colon F(X) \to F (X'')$ es el único morfismo
    caracterizado por
    $$G (F (\psi\circ\phi))\circ \eta_{X} = \eta_{X''}\circ \psi\circ\phi.$$
    Pero las fórmulas de arriba y la funtorialidad de $G$ nos dan
    \[ \eta_{X''}\circ \psi\circ\phi =
       G (F (\psi))\circ \eta_{X'}\circ \phi =
       G (F (\psi))\circ G (F (\phi))\circ \eta_X =
       G (F (\psi)\circ F (\phi))\circ \eta_X, \]
    así que
    $$F (\psi\circ\phi) = F (\psi)\circ F (\phi).$$

    Para ver que $\eta_X\colon X\to GF (X)$, hay que comprobar que para todo
    morfismo $\phi\colon X\to X'$ el cuadrado
    \[ \begin{tikzcd}
        X \ar{r}{\eta_X}\ar{d}[swap]{\phi} & GF (X)\ar{d}{GF (\phi)} \\
        X' \ar{r}[swap]{\eta_{X'}} & GF (X') \\
      \end{tikzcd} \]
    conmuta. Pero esto se cumple por nuestra \emph{definición} de $F (\phi)$.

    Por fin, hay que verificar que
    \begin{align*}
      \Hom_\vcateg{C} (X, G (Y)) & \to \Hom_\vcateg{C} (F (X), Y),\\
      g & \mapsto \widehat{g}
    \end{align*}
    es una biyección natural.

    Si tenemos $\widehat{g_1} = \widehat{g_2}$, entonces
    $G (\widehat{g_1}) = G (\widehat{g_2})$ y por lo tanto $g_1 = g_2$:
    \[ \begin{tikzcd}
        X \ar[shift left=0.25em]{rr}{g_1}\ar[shift right=0.25em]{rr}[swap]{g_2}\ar{dr}[swap]{\eta_X} && G (Y) \\
        & G (F (X))\ar[dashed]{ur}[swap]{G (\widehat{g_1}) = G (\widehat{g_2})}
      \end{tikzcd} \]
    Esto prueba que la aplicación $g\mapsto \widehat{g}$ es inyectiva. Para ver
    que es sobreyectiva, notamos que para $f\colon F (X)\to Y$ se cumple
    $\widehat{G(f)\circ \eta_X} = f$:
    \[ \begin{tikzcd}
        X \ar{rr}{G(f)\circ \eta_X}\ar{dr}[swap]{\eta_X} && G (Y) \\
        & G (F (X))\ar[dashed]{ur}[swap]{G (f) = G (\widehat{G(f)\circ \eta_X})}
      \end{tikzcd} \]
    La naturalidad de la biyección en $X$ significa que para todo
    $\phi\colon X\to X'$ hay diagramas conmutativos
    \[ \begin{tikzcd}
        \Hom_\vcateg{C} (X', G (Y))\ar{r}{\isom}\ar{d}[swap]{\phi^*} & \Hom_\vcateg{C} (F (X'), Y)\ar{d}{F (\phi)^*} \\
        \Hom_\vcateg{C} (X, G (Y))\ar{r}[swap]{\isom} & \Hom_\vcateg{C} (F (X), Y) \\
      \end{tikzcd} \]
    Es decir, para todo $g\colon X'\to G (Y)$ debe cumplirse
    $$\widehat{g\circ\phi} = \widehat{g}\circ F (\phi).$$
    En efecto, $\widehat{g\circ\phi}$ es el único morfismo tal que
    $$G (\widehat{g\circ\phi})\circ \eta_X = g\circ\phi.$$
    y $\widehat{g}$ es el único morfismo tal que
    $$G (\widehat{g})\circ \eta_{X'} = g.$$
    La segunda fórmula y la naturalidad de $\eta$ nos da
    \[ g\circ \phi =
       G (\widehat{g})\circ \eta_{X'}\circ \phi =
       G (\widehat{g})\circ GF (\phi)\circ \eta_X =
       G (\widehat{g}\circ F (\phi))\circ \eta_X. \]
     Ahora para $\psi\colon Y\to Y'$ hay que comprobar que
     \[ \begin{tikzcd}
         \Hom_\vcateg{C} (X, G (Y))\ar{r}{\isom}\ar{d}[swap]{G (\psi)_*} & \Hom_\vcateg{C} (F (X), Y)\ar{d}{\psi_*} \\
         \Hom_\vcateg{C} (X, G (Y'))\ar{r}[swap]{\isom} & \Hom_\vcateg{C} (F (X), Y') \\
       \end{tikzcd} \]
     Es decir, que para todo $g\colon X\to G (Y)$ se cumple
     $$\widehat{G (\psi)\circ g} = \psi\circ \widehat{g}.$$
     La flecha $\widehat{G (\psi)\circ g}$ se caracteriza por
     $$G (\widehat{G (\psi)\circ g})\circ \eta_X = G (\psi)\circ g,$$
     mientras que
     $$G (\widehat{g})\circ \eta_X = g.$$
     Luego
     \[ G (\psi)\circ g =
        G (\psi)\circ G (\widehat{g})\circ \eta_X =
        G (\psi\circ \widehat{g})\circ \eta_X. \qedhere \]
  \end{proof}
\end{proposicion}

\begin{ejemplo}
  Sea $i\colon \categ{Ab} \to \categ{Grp}$ la inclusión de grupos abelianos en
  grupos. Para todo grupo $G$ tenemos su abelianización $G^{ab} \dfn G/[G,G]$
  junto con el homomorfismo canónico $\eta_G\colon G\to i (G^{ab})$. Luego, para
  todo grupo abeliano $A$ y todo homomorfismo $f\colon G \to A$ existe un
  homomorfismo único $f^{ab}\colon G^{ab} \to A$ que hace conmutar el diagrama
  \[ \begin{tikzcd}
    G \ar{rr}{f}\ar{dr}[swap]{\eta_G} && i (A) \\
    & i (G^{ab})\ar[dashed]{ur}[swap]{i (f^{ab})}
  \end{tikzcd} \]
  El resultado precedente nos dice que esto es suficiente para deducir que
  $G \rightsquigarrow G^{ab}$ es un funtor, es adjunto por la izquierda a
  $i\colon \categ{Ab} \to \categ{Grp}$, y la unidad de la adjunción es
  $\eta_G\colon G\to i (G^{ab})$.
\end{ejemplo}

% % % % % % % % % % % % % % % % % % % % % % % % % % % % % %

\section{Objetos terminales e iniciales}

\begin{definicion}
  Sea $\vcateg{C}$ una categoría.

  Se dice que un objeto $T\in \Ob (\vcateg{C})$ es \term{terminal} si para todo
  $X \in \Ob (\vcateg{C})$ existe morfismo único $X\to T$.

  Se dice que un objeto $I\in \Ob (\vcateg{C})$ es \term{inicial} si para todo
  $X \in \Ob (\vcateg{C})$ existe morfismo único $I\to X$.
\end{definicion}

En otras palabras, la definción quiere decir que para todo
$X \in \Ob (\vcateg{C})$ se tiene $|\Hom_\vcateg{C} (X, T)| = 1$ y
$|\Hom_\vcateg{C} (I, X)| = 1$. En particular, se sigue que
$\Hom_\vcateg{C} (T,T) = \{ \id{T} \}$ y $\Hom_\vcateg{C} (I,I) = \{ \id{I} \}$.

\begin{ejemplo}
  En la categoría de conjuntos $\categ{Set}$ un conjunto $\{ \ast \}$ que consta
  de un solo elemento es un objeto terminal (para todo $X$ hay exactamente una
  aplicación $X\to \{ \ast \}$) y el conjunto vacío $\emptyset$ es un objeto
  inicial (para todo $X$ hay exactamente una aplicación
  $\emptyset \to X$\footnote{En particular, hay exactamente una aplicación
    $\emptyset\to\emptyset$, entonces $0^0 = 1$.}).

  En la categoría de espacios topológicos $\categ{Top}$ los objetos terminales e
  iniciales son los mismos que en $\categ{Set}$.
\end{ejemplo}

Los objetos terminales e iniciales no siempre existen: por ejemplo, los
conjuntos no vacíos forman una categoría que no tiene objeto inicial.

\begin{ejemplo}
  En la categoría de anillos $\categ{Ring}$ el anillo nulo $0$ es un objeto
  terminal y el anillo $\ZZ$ es un objeto inicial (como anillo, $\ZZ$ está
  generado por $1\in \ZZ$, y para todo anillo $R$ un homomorfismo
  $f\colon \ZZ \to R$ debe preservar la identidad).
\end{ejemplo}

\begin{ejemplo}
  En la categoría de grupos $\categ{Grp}$ un grupo trivial $\{ e \}$ es un
  objeto terminal e inicial al mismo tiempo. En la categoría de grupos abelianos
  $\categ{Ab}$ un grupo nulo $\{ 0 \}$ es un objeto terminal e inicial al mismo
  tiempo.
\end{ejemplo}

\begin{observacion}
  Si un objeto terminal (resp. inicial) existe, es único salvo isomorfismo
  único.

  \begin{proof}
    Si existen dos objetos terminales $T, T' \in \Ob (\vcateg{C})$, entonces
    existen morfismos únicos $f\colon T \to T'$ y $g\colon T'\to T$, y las
    composiciones $g\circ f$ y $f\circ g$ necesariamente coinciden con $\id{T}$ y
    $\id{T'}$ respectivamente.
  \end{proof}
\end{observacion}

\begin{observacion}
  Todo funtor adjunto por la izquierda (resp. derecha) preserva objetos
  iniciales (resp. terminales).

  \begin{proof}
    Si $F\colon \vcateg{C}\to \vcateg{D}$ es adjunto por la izquierda a
    $G\colon \vcateg{D}\to \vcateg{C}$ y $I$ es un objeto inicial en
    $\vcateg{C}$, entonces para todo $X\in \Ob (\vcateg{D})$ se tiene una
    biyección
    $$\Hom_\vcateg{D} (F (I), X) \cong \Hom_\vcateg{C} (I, G (X))$$
    y en particular
    $$|\Hom_\vcateg{D} (F (I), X)| = |\Hom_\vcateg{C} (I, G (X))| = 1.$$
    Si $T$ es un objeto terminal en $\vcateg{D}$, entonces para todo
    $X\in \Ob (\vcateg{C})$ se tiene una biyección
    $$\Hom_\vcateg{C} (X, G (T)) \isom \Hom_\vcateg{D} (F (X), T)$$
    y en particular
    \[ |\Hom_\vcateg{C} (X, G (T))| = |\Hom_\vcateg{D} (F (X), T)| = 1. \qedhere \]
  \end{proof}
\end{observacion}

\begin{ejemplo}
  Para todo conjunto fijo $X$ el funtor
  $-\times X\colon \categ{Set}\to \categ{Set}$ es adjunto por la izquierda a
  $(-)^X\colon \categ{Set}\to \categ{Set}$. Tenemos
  $\emptyset \times X = \emptyset$ y $\{ \ast \}^X \isom \{ \ast \}$.
\end{ejemplo}

\begin{ejemplo}
  El funtor del grupo de unidades $(-)^\times\colon \categ{Ring}\to \categ{Grp}$
  es adjunto por la izquierda al funtor
  ${\ZZ [-]\colon \categ{Grp} \to \categ{Ring}}$. Para un anillo trivial $0$ el
  grupo de unidades es trivial. Para un grupo trivial $G = \{ e \}$ el anillo
  correspondiente $\ZZ [G]$ es isomorfo a $\ZZ$.
\end{ejemplo}

% % % % % % % % % % % % % % % % % % % % % % % % % % % % % %

\section{Productos y coproductos}

\begin{definicion}
  Para dos objetos $X_1, X_2\in \Ob (\vcateg{C})$ su \term{producto} es un
  objeto $X_1\times X_2\in \Ob (\vcateg{C})$ dotado de dos morfismos
  $p_1\colon X_1\times X_2\to X_1$ y $p_2\colon X_1\times X_2 \to X_2$ que
  satisfacen la siguiente propiedad universal: si tenemos otro objeto
  $Z \in \Ob (\vcateg{C})$ junto con morfismos $f_1 \colon Z\to X_1$ y
  $f_2\colon Z\to X_2$, entonces existe un único morfismo
  $\tcol{f_1}{f_2}\colon Z\to X_1\times X_2$ tal que
  $$p_1\circ \tcol{f_1}{f_2} = f_1 \quad\text{y}\quad p_2\circ \tcol{f_1}{f_2} = f_2.$$
  De modo similar, un \term{coproducto} es un objeto
  $X_1\sqcup X_2\in \Ob (\vcateg{C})$ dotado de dos morfismos
  $i_1\colon X_1\to X_1\sqcup X_2$ y $i_2\colon X_2\to X_1\sqcup X_2$ que
  satisfacen la propiedad universal: si tenemos otro objeto
  $Z \in \Ob (\vcateg{C})$ junto con morfismos $f_1\colon X_1\to Z$ y
  $f_2\colon X_2\to Z$, entonces existe un único morfismo
  $(f_1,f_2)\colon Z\to X_1\times X_2$ tal que
  $$(f_1,f_2)\circ i_1 = f_1 \quad\text{y}\quad (f_1,f_2)\circ i_2 = f_2.$$

  \[ \begin{tikzcd}
      & Z\ar[dashed]{d}{\tcol{f_1}{f_2}}[swap]{\exists !}\ar{dl}[swap]{f_1}\ar{dr}{f_2} \\
      X_1 & X_1\times X_2\ar{l}[swap]{p_1}\ar{r}{p_2} & X_2
    \end{tikzcd} \quad\quad
    \begin{tikzcd}
      X_1\ar{r}{i_1}\ar{dr}[swap]{f_1} & X_1\sqcup X_2\ar[dashed]{d}[swap]{\exists !}{(f_1,f_2)} & X_2\ar{l}[swap]{i_2}\ar{dl}{f_2} \\
      & Z
    \end{tikzcd} \]
\end{definicion}

De modo similar, se pueden definir productos y coproductos infinitos.

\begin{definicion}
  Sea $\{ X_k \}_{k \in I}$ una familia de objetos indexada por un conjunto $I$.

  \begin{enumerate}
  \item[1)] Un \term{producto} $\prod_k X_k$ es un objeto junto con morfismos
    $p_k\colon \prod_k X_k \to X_k$ tal que para cualquier otro objeto $Z$ junto
    con morfismos $\{ f_k\colon Z\to X_k \}_{k\in I}$ existe un único morfismo
    $f\colon Z \to \prod_k X_k$ que satisface $p_k\circ f = f_k$ para cada
    $k\in I$.

  \item[2)] Un \term{coproducto} $\coprod_k X_k$ es un objeto junto con
    morfismos $i_k\colon X_k\to \coprod_k X_k$ tal que para cualquier otro
    objeto $Z$ junto con morfismos $\{ f_k\colon X_k\to Z \}_{k\in I}$ existe un
    único morfismo $f\colon \coprod_k X_k \to Z$ que satisface
    $f\circ i_k = f_k$ para cada $k\in I$.
  \end{enumerate}

  \[ \begin{tikzcd}
      Z \ar{dr}{f_k}\ar[dashed]{d}[swap]{\exists !}{f} \\
      \prod_k X_k\ar{r}[swap]{p_k} & X_k
    \end{tikzcd} \quad\quad \begin{tikzcd}
      X_k\ar{r}{i_k}\ar{dr}[swap]{f_k} & \coprod_k X_k\ar[dashed]{d}[swap]{\exists !}{f} \\
      & Z
    \end{tikzcd} \]
\end{definicion}

\begin{ejemplo}
  En la categoría de conjuntos $\categ{Set}$ el producto $\prod_k X_k$ es (salvo
  isomorfismo) el producto cartesiano y el coproducto $\coprod_k X_k$ es
  la unión disjunta.

  En categoría $\categ{Top}$ el producto de espacios topológicos $\prod_k X_k$
  es el producto de conjuntos subyacentes dotado de la topología producto.
  El coproducto $\coprod_k X_k$ es la unión disjunta donde cada componente viene
  con su topología original.
\end{ejemplo}

\begin{ejemplo}
  En la categoría de anillos con identidad $\categ{Ring}$ el producto
  $\prod_k R_k$ es el producto cartesiano con las operaciones definidas por
  \[ (r_k)_k + (r'_k)_k \dfn (r_k+r'_k)_k, \quad
     (r_k)_k \cdot (r'_k)_k \dfn (r_k\cdot r'_k)_k. \]
  El elemento nulo es $(0_{R_k})_k$ y la identidad es $(1_{R_k})_k$. Las
  proyecciones $p_k\colon (r_k)_k \mapsto r_k$ son visiblemente homomorfismos de
  anillos.

  El coproducto en la categoría de anillos con identidad es algo diferente: note
  que las inclusiones $r \mapsto (0,\ldots,0,r,0,\ldots,0)$ no preservan
  la identidad. En la categoría de anillos conmutativos $\categ{CRing}$
  el coproducto es el producto tensorial.
\end{ejemplo}

\begin{ejemplo}
  No todas categorías tienen productos y coproductos. Notemos que en
  la categoría de cuerpos, $\Hom (L,K) = \emptyset$ si $L$ y $K$ son cuerpos
  de característica diferente. Si $\fchar L_1 \ne \fchar L_2$, su producto
  $L_1\times L_2$ (resp. coproducto $L_1\sqcup L_2$) no existe, ya que no pueden
  existir ambos morfismos canónicos $L_1\times L_2\to L_1$ y
  $L_1\times L_2\to L_2$ (resp. $L_1\to L_1\sqcup L_2$ y
  $L_2\to L_1\sqcup L_2$).
\end{ejemplo}

\begin{ejemplo}
  En la categoría de grupos $\categ{Grp}$ el producto $\prod_k G_k$ es
  el producto de conjuntos dotado de la multiplicación término por término
  $$(g_k)_k\cdot (h_k)_k \dfn (g_k\cdot h_k)_k.$$
  El coproducto de grupos es algo muy diferente: es el producto libre $\bigast_k G_k$.

  En la categoría de grupos abelianos $\categ{Ab}$ el producto es el mismo,
  mientras que el coproducto es la suma directa (compuesta de sumas formales
  finitas de elementos).
  \[ \bigoplus_k A_k \dfn
       \Bigl\{ \sum_k a_k \Bigm| a_k\in A_k, ~
         a_k = 0, \text{ salvo un número finito de }k \Bigr\}. \qedhere \]
\end{ejemplo}

He aquí otro modo de definir productos y coproductos en términos de funtores
representables.

\begin{observacion}
  \label{hom-y-productos-y-coproductos}
  Sea $Z\in \Ob (\vcateg{C})$ un objeto fijo. El funtor covariante
  $\Hom_\vcateg{C} (Z,-)$ preserva productos y el funtor contravariante
  $\Hom_\vcateg{C} (-,Z)$ convierte coproductos en productos. Es decir, si
  existe el producto $\prod_{k\in I} X_k$, entonces hay una biyección natural
  entre conjuntos
  \begin{align}
    \label{eqn:hom-preserva-productos} \Hom_\vcateg{C} (Z, \prod_{k\in I} X_k) & \isom \prod_{k\in I} \Hom_\vcateg{C} (Z, X_k),\\
    f & \mapsto (p_k\circ f)_{k\in I},\notag
  \end{align}

  Si existe el coproducto $\coprod_{k\in I} X_k$, entonces hay una biyección natural entre conjuntos
  \begin{align}
    \label{eqn:hom-convierte-coproductos-en-productos} \Hom_\vcateg{C} (\coprod_{k\in I} X_k, Z) & \isom \prod_{k\in I} \Hom_\vcateg{C} (X_k, Z),\\
    f & \mapsto (f\circ i_k)_{k\in I}.\notag
  \end{align}

  En otras palabras, $\prod_{k\in I} X_k$ es un objeto que representa al funtor
  \begin{align*}
    \prod_{k\in I} \Hom_\vcateg{C} (-,X_k)\colon \vcateg{C}^\mathrm{op} & \to \categ{Set},\\
    Z & \mapsto \prod_{k\in I} \Hom_\vcateg{C} (Z,X_k).
  \end{align*}
  y $X\sqcup Y$ es un objeto que representa al funtor
  \begin{align*}
    \prod_{k\in I} \Hom_\vcateg{C} (X_k,-)\colon \vcateg{C} & \to \categ{Set},\\
    Z & \mapsto \prod_{k\in I} \Hom_\vcateg{C} (X_k,Z).
  \end{align*}

  \begin{proof}
    Las biyecciones \eqnref{eqn:hom-preserva-productos} y
    \eqnref{eqn:hom-convierte-coproductos-en-productos} son otro modo de
    formular la definición de producto y coproducto. La aplicación inversa viene
    dada por la propiedad universal. La naturalidad en $Z$ es también evidente:
    para todo morfismo $h\colon Z \to Z'$ se tienen diagramas conmuativos
    \[ \begin{tikzcd}
        \Hom_\vcateg{C} (Z, \prod_k X_k) \ar{r}{\isom} & \prod_k \Hom_\vcateg{C} (Z, X_k) & \Hom_\vcateg{C} (\coprod_k X_k, Z) \ar{r}{\isom}\ar{d} & \prod_k \Hom_\vcateg{C} (X_k, Z)\ar{d} \\
        \Hom_\vcateg{C} (Z', \prod_k X_k) \ar{r}{\isom}\ar{u} & \prod_k \Hom_\vcateg{C} (Z', X_k)\ar{u} & \Hom_\vcateg{C} (\coprod_k X_k, Z') \ar{r}{\isom} & \prod_k \Hom_\vcateg{C} (X_k, Z') \\
        f\circ h\ar[|->]{r} & (p_k\circ f\circ h)_k & f\ar[|->]{d}\ar[|->]{r} & (f\circ i_k)_k\ar[|->]{d} \\
        f \ar[|->]{r}\ar[|->]{u} & (p_k\circ f)_k \ar[|->]{u} & h\circ f \ar[|->]{r} & (h\circ f\circ i_k)_k
      \end{tikzcd} \]
  \end{proof}
\end{observacion}

Entonces, los productos y coproductos pueden ser \emph{definidos} como objetos
que representan a los funtores $\prod_k \Hom_\vcateg{C} (-,X_k)$ y
$\prod_k \Hom_\vcateg{C} (X_k,-)$.

\begin{observacion}
  Si un producto $\prod_k X_k$ (resp. coproducto $\coprod_k X_k$) existe, es
  único salvo isomorfismo único.

  \begin{proof}
    Es una propiedad general de funtores representables que hemos probado en
    \ref{funtores-representados-objeto-unico-salvo-iso}. El lector también puede
    deducirla directamente de la definición de producto y coproducto.
  \end{proof}
\end{observacion}

\begin{observacion}
  Los productos y coproductos satisfacen las siguientes propiedades.

  \begin{enumerate}
  \item[1)] Los productos y coproductos son conmutativos (cuando existen):
    $$X_1\times X_2\isom X_2\times X_1 \quad\text{y}\quad X_1\sqcup X_2\isom X_2\sqcup X_1.$$

  \item[2)] Los productos y coproductos son asociativos (cuando existen):
  $$(X_1\times X_2)\times X_3\isom X_1\times (X_2\times X_3) \isom X_1\times X_2\times X_3$$
  y
  $$(X_1\sqcup X_2)\sqcup X_3\isom X_1\sqcup (X_2\sqcup X_3) \isom X_1\sqcup X_2\sqcup X_3.$$
\end{enumerate}

\begin{proof}
  Sigue de la propiedad universal del producto y coproducto (ejercicio para el lector), o de isomorfismos de funtores
  \begin{gather*}
    \Hom_\vcateg{C} (-,X_1)\times \Hom_\vcateg{C} (-,X_2) \isom \Hom_\vcateg{C} (-,X_2)\times \Hom_\vcateg{C} (-,X_1),\\
    \Hom_\vcateg{C} (X_1,-)\times \Hom_\vcateg{C} (X_2,-) \isom \Hom_\vcateg{C} (X_2,-)\times \Hom_\vcateg{C} (X_1,-),
  \end{gather*}
  etcétera.
\end{proof}
\end{observacion}

Terminamos esta sección con una propiedad importante de los funtores adjuntos.

\begin{observacion}
  \label{adjunto-a-la-izquierda-preserva-coproductos}
  Sea $F\colon \vcateg{C} \to \vcateg{D}$ un funtor adjunto por la izquierda a
  $G\colon \vcateg{D} \to \vcateg{C}$. Entonces $F$ preserva coproductos y $G$
  preserva productos, es decir para cualesquiera $X_k \in \Ob (\vcateg{C})$ e
  $Y_k \in \Ob (\vcateg{D})$
  \begin{align*}
    F (\coprod_k X_k) & \isom \coprod_k F (X_k),\\
    G (\prod_k Y_k) & \isom \prod_k G (Y_k)
  \end{align*}
  (siempre y cuando estos productos y coproductos existan).

  \begin{proof}
    Como hemos observado en \ref{hom-y-productos-y-coproductos}, para cualquier
    objeto $Z$ tenemos isomorfismos de funtores
    \begin{align*}
      \Hom (\coprod_k X_k, -) & \isom \prod_k \Hom (X_k, -),\\
      \Hom (-, \prod_k Y_k) & \isom \prod_k \Hom (-, Y_k).
    \end{align*}
    Ahora tenemos
    \begin{multline*}
      \Hom_\vcateg{D} (F (\coprod_k X_k), -) \isom \Hom_\vcateg{C} (\coprod_k X_k, G (-)) \isom \prod_k \Hom_\vcateg{C} (X_k, G (-)) \isom \prod_k \Hom_\vcateg{D} (F(X_k), -) \\
      \isom \Hom_\vcateg{D} (\coprod_k F(X_k), -)
    \end{multline*}
    y el lema de Yoneda implica que $F (\coprod_k X_k) \isom \coprod_k
    F(X_k)$. De modo similar se demuestra el isomorfismo
    $G (\prod_k Y_k) \isom \prod_k G (Y_k)$.
  \end{proof}
\end{observacion}

\begin{ejemplo}
  Hay biyecciones canónicas de conjuntos
  \[ \left( \coprod_k Y_k \right)\times X \isom \coprod_k (Y_k \times X), \quad
     \left(\prod_k Z_k\right)^X \isom \prod_k (Z_k)^X. \qedhere \]
\end{ejemplo}

\begin{ejemplo}
  El grupo de unidades de un producto de anillos es el producto de grupos de unidades:
  \[ \left(\prod_k R_k\right)^\times \isom \prod_k R_k^\times. \qedhere \]
\end{ejemplo}

\begin{ejemplo}
  La abelianización del producto libre de grupos es la suma directa de las
  abelianizaciones correspondientes:
  $$\left(\bigast_k G_k\right)^{ab} \isom \bigoplus_k (G_k)^{ab}.$$
  El producto en la categoría de grupos abealianos $\categ{Ab}$ coincide con el
  producto en la categoría de grupos $\categ{Grp}$ porque la inclusión
  $\categ{Ab} \hookrightarrow \categ{Grp}$ es un funtor adjunto por la derecha.
\end{ejemplo}

% % % % % % % % % % % % % % % % % % % % % % % % % % % % % %

%\marginpar{\footnotesize lección\\6.06.2018}

\section{Productos y coproductos fibrados}

En álgebra y geometría (y sobre todo en geometría algebraica) tiene mucha
importancia el punto de vista \term{relativo}. En cualquier categoría
$\vcateg{C}$ podemos fijar un objeto $Z$ y considerar los
\term{objetos sobre $Z$}, es decir, objetos $X$ junto con un morfismo
especificado $X \to Z$, y los \term{morfismos sobre $Z$} que son diagramas
conmutativos
$$\begin{tikzcd}
X\ar{rr}\ar{dr} & & Y\ar{dl} \\
& Z
\end{tikzcd}$$
De la misma manera, los \term{objetos bajo $Z$} vienen con morfismos $Z \to X$ y
los \term{morfismos bajo $Z$} son diagramas conmutativos
$$\begin{tikzcd}
 & Z\ar{dl}\ar{dr} \\
X\ar{rr} & & Y
\end{tikzcd}$$
Para objetos sobre $Z$ o bajo $Z$ existen construcciones universales denominadas
\term{producto fibrado} y \term{coproducto fibrado}, respectivamente.

\begin{definicion}
  Si tenemos morfismos $X\to Z$ y $Y\to Z$, entonces el \term{producto fibrado
    de $X$ e $Y$ sobre $Z$} es un objeto $X\times_Z Y$ con morfismos
  $X\times_Z Y \to X$ y $X\times_Z Y\to Y$ que forman parte del diagrama
  conmutativo
  \[ \begin{tikzpicture}
      \matrix(m)[matrix of math nodes, row sep=2em, column sep=2em,
      text height=1.5ex, text depth=0.25ex]{
        X\times_Z Y & Y \\
        X & Z \\};
      \path[->] (m-1-1) edge (m-1-2);
      \path[->] (m-2-1) edge (m-2-2);
      \path[->] (m-1-1) edge (m-2-1);
      \path[->] (m-1-2) edge (m-2-2);

      \begin{scope}[shift=($(m-1-1)!.4!(m-2-2)$)]
        \draw +(-.2,0) -- +(0,0)  -- +(0,.2);
      \end{scope}
    \end{tikzpicture} \]

  Además se pide la siguiente propiedad universal: si $V$ es otro objeto junto
  con morfismos $V\to X$ y $V\to Y$ que conmutan con los morfismos $X\to Z$ y
  $Y\to Z$, entonces existe un único morfismo $V \to X\times_Z Y$ tal que el
  siguiente diagrama es conmutativo:
  \[ \begin{tikzpicture}
      \matrix(m)[matrix of math nodes, row sep=2em, column sep=2em,
      text height=1.5ex, text depth=0.25ex]{
        V \\
        & X\times_Z Y & Y \\
        & X & Z \\};
      \path[->] (m-2-2) edge (m-2-3);
      \path[->] (m-3-2) edge (m-3-3);
      \path[->] (m-2-2) edge (m-3-2);
      \path[->] (m-2-3) edge (m-3-3);
      \path[->,dashed,font=\scriptsize] (m-1-1) edge node[above] {$\exists!$} (m-2-2);
      \path[->,bend left] (m-1-1) edge (m-2-3);
      \path[->,bend right] (m-1-1) edge (m-3-2);

      \begin{scope}[shift=($(m-2-2)!.4!(m-3-3)$)]
        \draw +(-.2,0) -- +(0,0)  -- +(0,.2);
      \end{scope}
    \end{tikzpicture} \]
\end{definicion}

\begin{comentario}
  En inglés el producto fibrado también se llama ``pullback''. El símbolo
  ``$\begin{tikzpicture}\draw +(-.2,0) -- +(0,0) -- +(0,.2);\end{tikzpicture}$''
  en el cuadrado de arriba significa que es el cuadrado del producto fibrado, es
  decir que no es simplemente un cuadrado conmutativo, sino también
  universal. También se dice que es un \term{cuadrado cartesiano}, pero no vamos
  a usar esta terminología.

  La notación $X\times_Z Y$ es un poco ambigua: este objeto depende no solamente
  de $X,Y,Z$, sino de morfismos específicos $X\to Z$ e $Y\to Z$. Además, este
  objeto está bien definido solo salvo isomorfismo.
\end{comentario}

\begin{ejemplo}
  En la categoría de conjuntos, el producto fibrado de $f\colon X\to Z$ y
  $g\colon Y\to Z$ es el conjunto
  $$X\times_Z Y = \{ (x,y) \in X\times Y \mid f (x) = g (y) \}$$
  y los morfismos $X\times_Z Y \to X$ y $X\times_Z Y \to Y$ están inducidos por
  las proyecciones.

  En particular, si $X$ e $Y$ son subconjuntos de $Z$, entonces $X\times_Z Y$
  puede ser identificado con $X\cap Y \subseteq Z$:
  \[ \begin{tikzpicture}
      \matrix(m)[matrix of math nodes, row sep=2em, column sep=2em,
      text height=1.5ex, text depth=0.25ex]{
        X\cap Y & Y \\
        X & Z \\};
      \path[right hook->] (m-1-1) edge (m-1-2);
      \path[right hook->] (m-2-1) edge (m-2-2);
      \path[right hook->] (m-1-1) edge (m-2-1);
      \path[right hook->] (m-1-2) edge (m-2-2);

      \begin{scope}[shift=($(m-1-1)!.4!(m-2-2)$)]
        \draw +(-.2,0) -- +(0,0)  -- +(0,.2);
      \end{scope}
    \end{tikzpicture} \]

  En las categorías de conjuntos con estructura adicional los productos fibrados
  tienen una descripción parecida.
\end{ejemplo}

A partir de la propiedad universal, se deduce que

\begin{enumerate}
\item[1)] si el producto fibrado $X\times_Z Y$ existe, es único salvo
  isomorfismo;

\item[2)] el producto fibrado es conmutativo: $X\times_Z Y \isom Y\times_Z X$;

\item[3)] el producto fibrado es asociativo:
  $(X_1 \times_Z X_2) \times_Z X_3 \isom X_1 \times_Z (X_2 \times_Z X_3)$.

\item[4)] los productos fibrados preservan isomorfismos: si la flecha $Y\to Z$
  es un isomorfismo, entonces $X\times_Z Y\to X$ es también un isomorfismo:
  \[ \begin{tikzpicture}
      \matrix(m)[matrix of math nodes, row sep=2em, column sep=2em,
      text height=1.5ex, text depth=0.25ex]{
        X\times_Z Y & Y \\
        X & Z \\};
      \path[->] (m-1-1) edge (m-1-2);
      \path[->] (m-2-1) edge (m-2-2);
      \path[->,font=\scriptsize] (m-1-1) edge node[left] {$\isom$} (m-2-1);
      \path[->,font=\scriptsize] (m-1-2) edge node[right] {$\isom$} (m-2-2);

      \begin{scope}[shift=($(m-1-1)!.4!(m-2-2)$)]
        \draw +(-.2,0) -- +(0,0)  -- +(0,.2);
      \end{scope}
    \end{tikzpicture} \]
\end{enumerate}

Cada concepto en la teoría de categorías tiene su dual: basta cambiar
las direcciones de las flechas en la definición. Así tenemos monomorfismos
y epimorfismos, productos y coproductos, etc. De la misma manera, la noción dual
a los productos fibrados son los coproductos fibrados.

\begin{definicion}
  Si tenemos morfismos $Z\to X$ y $Z\to Y$, entonces el
  \term{coproducto fibrado de $X$ e $Y$ bajo $Z$} es un objeto $X\sqcup_Z Y$
  junto con morfismos $X\to X\sqcup_Z Y$ y $Y\to X\sqcup_Z Y$ que forman parte
  del diagrama conmutativo

  \[ \begin{tikzpicture}
      \matrix(m)[matrix of math nodes, row sep=2em, column sep=2em,
      text height=1.5ex, text depth=0.25ex]{
        Z & Y \\
        X & X\sqcup_Z Y \\};
      \path[->] (m-1-1) edge (m-1-2);
      \path[->] (m-2-1) edge (m-2-2);
      \path[->] (m-1-1) edge (m-2-1);
      \path[->] (m-1-2) edge (m-2-2);

      \begin{scope}[shift=($(m-1-1)!.7!(m-2-2)$)]
        \draw +(-.2,0) -- +(0,0)  -- +(0,.2);
      \end{scope}
    \end{tikzpicture} \]

  Además se pide la siguiente propiedad universal: si $V$ es otro objeto junto
  con morfismos $X\to V$ y $Y\to V$ que conmutan con los morfismos $Z\to X$ y
  $Z\to Y$, entonces existe un único morfismo $X\sqcup_Z Y \to V$ tal que el
  diagrama siguiente es conmutativo:

  \[ \begin{tikzpicture}
      \matrix(m)[matrix of math nodes, row sep=2em, column sep=2em,
      text height=1.5ex, text depth=0.25ex]{
        Z & Y \\
        X & X\sqcup_Z Y \\
        & & V\\};
      \path[->] (m-1-1) edge (m-1-2);
      \path[->] (m-2-1) edge (m-2-2);
      \path[->] (m-1-1) edge (m-2-1);
      \path[->] (m-1-2) edge (m-2-2);
      \path[dashed,->,font=\scriptsize] (m-2-2) edge node[above] {$\exists!$} (m-3-3);
      \path[->,bend left] (m-1-2) edge (m-3-3);
      \path[->,bend right] (m-2-1) edge (m-3-3);

      \begin{scope}[shift=($(m-1-1)!.7!(m-2-2)$)]
        \draw +(-.2,0) -- +(0,0)  -- +(0,.2);
      \end{scope}
    \end{tikzpicture} \]
\end{definicion}

\begin{comentario}
  En inglés el coproducto fibrado también se llama ``pushout''. El símbolo
  ``$\begin{tikzpicture}\draw +(-.2,0) -- +(0,0) -- +(0,.2);\end{tikzpicture}$''
  en el cuadrado significa que es universal en el sentido de arriba. También se
  dice que es un \term{cuadrado cocartesiano}. Otro término posible es
  \term{suma amalgamada} (!).
\end{comentario}

\begin{ejemplo}
  En la categoría de conjuntos, el coproducto fibrado de $f\colon Z\to X$ y
  $g\colon Z\to Y$ es el conjunto
  $$X\sqcup_Z Y = (X\sqcup Y)/\!\sim,$$
  donde $\sim$ es la relación de equivalencia generada\footnote{Es decir, se
    considera la relación de equivalencia más pequeña que satisface
    $i_1\circ f (z) \sim i_2 \circ g (z)$.} por
  $$i_1\circ f (z) \sim i_2 \circ g (z).$$
  En particular, si $X$ e $Y$ son subconjuntos de otro conjunto $W$ y $Z$ es su
  intersección, entonces $X\sqcup_Z Y$ puede ser identificado con $X\cup Y$:
  \[ \begin{tikzpicture}
      \matrix(m)[matrix of math nodes, row sep=2em, column sep=2em,
      text height=1.5ex, text depth=0.25ex]{
        X\cap Y & Y \\
        X & X\cup Y \\};
      \path[right hook->] (m-1-1) edge (m-1-2);
      \path[right hook->] (m-2-1) edge (m-2-2);
      \path[right hook->] (m-1-1) edge (m-2-1);
      \path[right hook->] (m-1-2) edge (m-2-2);

      \begin{scope}[shift=($(m-1-1)!.7!(m-2-2)$)]
        \draw +(-.2,0) -- +(0,0)  -- +(0,.2);
      \end{scope}
\end{tikzpicture} \]

  En categorías de conjuntos con estructura adicional los coproductos fibrados
  tienen descripción parecida.
\end{ejemplo}

A partir de la propiedad universal, se deduce que

\begin{enumerate}
\item[1)] si el coproducto fibrado $X\sqcup_Z Y$ existe, es único salvo isomorfismo;

\item[2)] el coproducto fibrado es conmutativo: $X\sqcup_Z Y \isom Y\sqcup_Z X$;

\item[3)] el coproducto fibrado es asociativo:
  $(X_1 \sqcup_Z X_2) \sqcup_Z X_3 \isom X_1 \sqcup_Z (X_2 \sqcup_Z X_3)$.

\item[4)] los coproductos fibrados preservan isomorfismos: si la flecha $Z\to X$
  es un isomorfismo, entonces $Y\to X\sqcup_Z Y$ es también un isomorfismo:
  \[ \begin{tikzpicture}
      \matrix(m)[matrix of math nodes, row sep=2em, column sep=2em,
      text height=1.5ex, text depth=0.25ex]{
        Z & Y \\
        X & X\sqcup_Z Y \\};
      \path[->] (m-1-1) edge (m-1-2);
      \path[->] (m-2-1) edge (m-2-2);
      \path[->,font=\scriptsize] (m-1-1) edge node[left] {$\isom$} (m-2-1);
      \path[->,font=\scriptsize] (m-1-2) edge node[right] {$\isom$} (m-2-2);

      \begin{scope}[shift=($(m-1-1)!.7!(m-2-2)$)]
        \draw +(-.2,0) -- +(0,0)  -- +(0,.2);
      \end{scope}
    \end{tikzpicture} \]
\end{enumerate}

\begin{observacion}
  \label{productos-como-productos-fibrados-sobre-el-objeto-terminal}
  Si $T$ es un objeto terminal (resp. $I$ es un objeto inicial), entonces
  $X\times_T Y \isom X\times Y$ (resp. $X\sqcup_I Y \isom X\sqcup Y$).

  \begin{proof}
    Por ejemplo, en el caso de productos fibrados, para todo $X\in \vcateg{C}$
    automáticamente tenemos un único morfismo $X\to T$, y la propiedad universal
    de $X\times_T Y$ corresponde a la propiedad universal de $X\times Y$.
  \end{proof}
\end{observacion}

\begin{observacion}
  Los productos fibrados preservan monomorfismos y los coproductos fibrados
  preservan epimorfismos:

  \begin{enumerate}
  \item[1)] si $m\colon X\to Y$ es un monomorfismo, entonces
    $\overline{m}\colon X\times_Y Z\to Z$ es también un monomorfismo;

  \item[2)] si $e\colon X\to Y$ es un epimorfismo, entonces
    $\overline{e}\colon Z\to Z\sqcup_X Y$ es también un epimorfismo.
  \end{enumerate}

  \[ \begin{tikzpicture}
      \matrix(m)[matrix of math nodes, row sep=2em, column sep=2em,
      text height=1.5ex, text depth=0.25ex]{
        X\times_Y Z & Z && X & Y \\
        X & Y && Z & Z\sqcup_X Y \\};
      \path[>->,font=\scriptsize] (m-1-1) edge node[above] {$\overline{m}$} (m-1-2);
      \path[>->,font=\scriptsize] (m-2-1) edge node[below] {$m$} (m-2-2);
      \path[->,font=\scriptsize] (m-1-1) edge node[left] {$\overline{f}$} (m-2-1);
      \path[->,font=\scriptsize] (m-1-2) edge node[right] {$f$} (m-2-2);

      \path[->>,font=\scriptsize] (m-1-4) edge node[above] {$e$} (m-1-5);
      \path[->>,font=\scriptsize] (m-2-4) edge node[below] {$\overline{e}$} (m-2-5);
      \path[->,font=\scriptsize] (m-1-4) edge node[left] {$f$} (m-2-4);
      \path[->,font=\scriptsize] (m-1-5) edge node[right] {$\overline{f}$} (m-2-5);

      \begin{scope}[shift=($(m-1-1)!.4!(m-2-2)$)]
        \draw +(-.2,0) -- +(0,0)  -- +(0,.2);
      \end{scope}
      \begin{scope}[shift=($(m-1-4)!.7!(m-2-5)$)]
        \draw +(-.2,0) -- +(0,0)  -- +(0,.2);
      \end{scope}
    \end{tikzpicture} \]

  \begin{proof}
    Para ver 1), si tenemos dos morfismos $g,h\colon V \to X\times_Y Z$ tales
    que $\overline{m}\circ g = \overline{m}\circ h$, entonces por la
    conmutatividad del cuadrado $m\circ \overline{f}\circ g = m\circ f\circ h$,
    pero $m$ es mono, así que $\overline{f}\circ g = \overline{f}\circ h$. Ahora
    podemos aplicar la propiedad universal de productos fibrados:
    \[ \begin{tikzpicture}
        \matrix(m)[matrix of math nodes, row sep=2em, column sep=2em,
        text height=1.5ex, text depth=0.25ex]{
          V \\
          & X\times_Y Z & Z \\
          & X & Y \\};
        \path[>->,font=\scriptsize] (m-2-2) edge node[above] {$\overline{m}$} (m-2-3);
        \path[>->,font=\scriptsize] (m-3-2) edge node[below] {$m$} (m-3-3);
        \path[->,font=\scriptsize] (m-2-2) edge node[left] {$\overline{f}$} (m-3-2);
        \path[->,font=\scriptsize] (m-2-3) edge node[right] {$f$} (m-3-3);
        \path[->,dashed,font=\scriptsize] (m-1-1) edge node[above] {$\exists!$} (m-2-2);
        \path[->,font=\scriptsize,bend left] (m-1-1) edge node[above] {$\overline{m}\circ g = \overline{m}\circ h$} (m-2-3);
        \path[->,font=\scriptsize,bend right] (m-1-1) edge node[left] {$\overline{f}\circ g = \overline{f}\circ h$} (m-3-2);

        \begin{scope}[shift=($(m-2-2)!.4!(m-3-3)$)]
          \draw +(-.2,0) -- +(0,0)  -- +(0,.2);
        \end{scope}
      \end{tikzpicture} \]

    Aquí la flecha punteada es única, de donde $g = h$. Para 2) el argumento es
    idéntico, solo hay que cambiar la dirección de todas las flechas.
  \end{proof}
\end{observacion}

% % % % % % % % % % % % % % % % % % % % % % % % % % % % % %

\section{Ecualizadores y coecualizadores}

\begin{definicion}
  Sean $f$ y $g$ un par de morfismos $X\to Y$.

  Un \term{ecualizador} de $f$ y $g$ es un objeto $\eq (f,g)$ junto con un
  morfismo $i\colon \eq (f,g) \to X$ tal que $f\circ i = g\circ i$ (de aquí el
  nombre ``ecualizador'') y además $i$ es universal entre todos los morfismos
  que ecualizan a $f$ y $g$: a saber, si $j\colon Z \to X$ es otro morfismo tal
  que $f\circ j = g\circ j$, entonces existe un único morfismo $Z\to \eq (f,g)$
  que hace parte del siguiente diagrama conmutativo:
  \[ \begin{tikzcd}
      \eq (f,g)\ar{r}{i} & X\ar[shift left=0.25em]{r}{f}\ar[shift right=0.25em]{r}[swap]{g} & Y \\
      Z\ar{ur}[swap]{j}\ar[dashed]{u}{\exists !}
    \end{tikzcd} \]

  Un \term{coecualizador} de $f$ y $g$ es un objeto $\coeq (f,g)$ junto con un
  morfismo $p\colon Y\to \coeq (f,g)$ tal que $p\circ f = p\circ g$ y que
  satisface la propiedad universal
  \[ \begin{tikzcd}
      X\ar[shift left=0.25em]{r}{f}\ar[shift right=0.25em]{r}[swap]{g} & Y\ar{r}{p}\ar{dr}[swap]{q} & \coeq (f,g)\ar[dashed]{d}{\exists !} \\
      & & Z
    \end{tikzcd} \]
\end{definicion}

\begin{observacion}
  ~

  \begin{enumerate}
  \item[1)] Si $\eq (f,g)$ existe, entonces el morfismo
    $i\colon \eq (f,g) \to X$ es mono y el objeto $\eq (f,g)$ es único salvo
    isomorfismo.

  \item[2)] Si $\coeq (f,g)$ existe, entonces el morfismo
    $p\colon Y\to \coeq (f,g)$ es epi y el objeto $\coeq (f,g)$ es único salvo
    isomorfismo.
  \end{enumerate}

  \begin{proof}
    Vamos a demostrar la parte sobre $\eq (f,g)$ y dejo la parte sobre
    $\coeq (f,g)$ como un ejercicio (invirtiendo las flechas).

    Sean $h,h'\colon Z \to \eq (f,g)$ dos flechas tales que
    $i\circ h = i\circ h'$. Tenemos $f\circ i\circ h = g\circ i\circ h$ y
    $f\circ i\circ h' = g\circ i\circ h'$, y por la propiedad universal del
    ecualizador, existen únicos morfismos $\ell, \ell'\colon Z \to \eq (f,g)$
    tales que $i\circ h = i\circ \ell$ y $i\circ h' = i\circ \ell'$. Entonces
    $\ell = \ell' = h = h'$.

    \[ \begin{tikzcd}
        Z\ar[shift left=0.25em]{r}{h}\ar[shift right=0.25em]{r}[swap]{h'} & \eq (f,g)\ar{r}{i} & X\ar[shift left=0.25em]{r}{f}\ar[shift right=0.25em]{r}[swap]{g} & Y
      \end{tikzcd} \]

    Ahora sean $E$ y $E'$ dos objetos con morfismos $i\colon E \to X$ y
    $i'\colon E' \to M$ que satisfacen la propiedad universal del
    ecualizador. Entonces existen morfismos únicos $k\colon E'\to E$ y
    $\ell\colon E \to E'$ tales que $i\circ k = i'$ y $i'\circ \ell =
    i$. Tenemos $i\circ k\circ \ell = i\circ \id{E}$, pero $i$ es un
    monomorfismo, y por lo tanto $k\circ \ell = \id{E}$. De modo similar,
    $i'\circ \ell\circ k = i'\circ \id{E'}$ y $\ell\circ k = \id{E'}$. Las
    flechas $k$ y $\ell$ definen un isomorfismo $E\isom E'$.
  \end{proof}
\end{observacion}

\begin{ejemplo}
  Para $f = g\colon X\to Y$

  \begin{enumerate}
  \item[1)] el morfismo $\eq (f,f)\to X$ debe ser (salvo isomorfismo) el
    morfismo identidad $\id{X}$;

  \item[2)] el morfismo $Y\to \coeq (f,f)$ debe ser (salvo isomorfismo) el
    morfismo identidad $\id{Y}$. \qedhere
  \end{enumerate}
\end{ejemplo}

\begin{ejemplo}
  En $\categ{Set}$ para un par de aplicaciones $f,g\colon X\to Y$ el conjunto
  $$\eq (f,g) = \{ x\in X \mid f(x) = g(x) \}$$
  con la inclusión $i\colon \eq (f,g) \hookrightarrow X$ satisface la propiedad
  universal del ecualizador. Luego sea $\sim$ la relación de equivalencia sobre
  $Y$ generada por
  $$f (x) \sim g (x) \text{ para cada }x\in X.$$
  Entonces el conjunto
  $$\coeq (f,g) = Y/\sim$$
  con el morfismo de proyección $p\colon Y \twoheadrightarrow \coeq (f,g)$
  satisface la propiedad universal del coecualizador.
\end{ejemplo}

\begin{observacion}
  \label{ecualizadores-como-productos-fibrados}
  Todo ecualizador (resp. coecualizador) puede ser expresado como un producto
  (resp. coproducto) fibrado.

  \begin{proof}
    Por ejemplo, en el caso de ecualizadores, para morfismos $f,g\colon X\to Y$
    podemos considerar uno de los siguientes productos fibrados.

    Primera construcción:
    \[ \begin{tikzpicture}
        \matrix(m)[matrix of math nodes, row sep=2em, column sep=2em, text height=1.5ex, text depth=0.25ex]{
          E & X \\
          X & X\times Y\\};
        \path[->,font=\scriptsize] (m-1-1) edge node[above] {$e'$} (m-1-2);
        \path[->,font=\scriptsize] (m-2-1) edge node[below] {$\tcol{\idid}{f}$} (m-2-2);
        \path[->,font=\scriptsize] (m-1-1) edge node[left] {$e$} (m-2-1);
        \path[->,font=\scriptsize] (m-1-2) edge node[right] {$\tcol{\idid}{g}$} (m-2-2);

        \begin{scope}[shift=($(m-1-1)!.4!(m-2-2)$)]
          \draw +(-.2,0) -- +(0,0)  -- +(0,.2);
        \end{scope}
      \end{tikzpicture} \]
    La conmutatividad de este cuadrado
    ($\tcol{\idid}{f} \circ e = \tcol{\idid}{g} \circ e'$) equivale a $e = e'$ y
    $f\circ e = g\circ e$.

    Segunda construcción:
    \[ \begin{tikzpicture}
        \matrix(m)[matrix of math nodes, row sep=2em, column sep=2em, text height=1.5ex, text depth=0.25ex]{
          E & X \\
          X & Y\times Y\\};
        \path[->,font=\scriptsize] (m-1-1) edge node[above] {$e$} (m-1-2);
        \path[->,font=\scriptsize] (m-2-1) edge node[below] {$\tcol{\idid}{\idid}$} (m-2-2);
        \path[->,font=\scriptsize] (m-1-1) edge node[left] {$e'$} (m-2-1);
        \path[->,font=\scriptsize] (m-1-2) edge node[right] {$\tcol{f}{g}$} (m-2-2);

        \begin{scope}[shift=($(m-1-1)!.4!(m-2-2)$)]
          \draw +(-.2,0) -- +(0,0)  -- +(0,.2);
        \end{scope}
      \end{tikzpicture} \]

    La conmutatividad de este cuadrado equivale a $f\circ e = g\circ e = e'$.

    En ambos casos, la propiedad universal del producto fibrado equivale a la
    propiedad universal del ecualizador $e\colon E\to X$.
  \end{proof}
\end{observacion}

\begin{observacion}
  \label{productos-fibrados-como-ecualizadores}
  Todo producto (resp. coproducto) fibrado puede ser expresado como un
  ecualizador (resp. coecualizador).

  \begin{proof}
    Necesitamos construir un producto (resp. coproducto) fibrado
    \[ \begin{tikzpicture}
        \matrix(m)[matrix of math nodes, row sep=2em, column sep=2em,
        text height=1.5ex, text depth=0.25ex]{
          X\times_Z Y & Y && Z & Y \\
          X & Z && X & X\sqcup_Z Y \\};
        \path[->,font=\scriptsize] (m-1-1) edge node[above] {$\overline{f}$} (m-1-2);
        \path[->,font=\scriptsize] (m-2-1) edge node[below] {$f$} (m-2-2);
        \path[->,font=\scriptsize] (m-1-1) edge node[left] {$\overline{g}$} (m-2-1);
        \path[->,font=\scriptsize] (m-1-2) edge node[right] {$g$} (m-2-2);

        \path[->,font=\scriptsize] (m-1-4) edge node[above] {$g$} (m-1-5);
        \path[->,font=\scriptsize] (m-2-4) edge node[below] {$\overline{g}$} (m-2-5);
        \path[->,font=\scriptsize] (m-1-4) edge node[left] {$f$} (m-2-4);
        \path[->,font=\scriptsize] (m-1-5) edge node[right] {$\overline{f}$} (m-2-5);

        \begin{scope}[shift=($(m-1-1)!.4!(m-2-2)$)]
          \draw +(-.2,0) -- +(0,0)  -- +(0,.2);
        \end{scope}
        \begin{scope}[shift=($(m-1-4)!.7!(m-2-5)$)]
          \draw +(-.2,0) -- +(0,0)  -- +(0,.2);
        \end{scope}
      \end{tikzpicture} \]

    Lo más obvio que podemos hacer es tomar el producto y coproducto
    \[ X \xleftarrow{p_X} X\times Y \xrightarrow{p_Y} Y \quad\quad
       X \xrightarrow{i_X} X\sqcup Y \xleftarrow{i_Y} Y \]
    y considerar los cuadrados
    \[ \begin{tikzpicture}
        \matrix(m)[matrix of math nodes, row sep=2em, column sep=2em,
        text height=1.5ex, text depth=0.25ex]{
          X\times Y & Y && Z & Y \\
          X & Z && X & X\sqcup Y \\};
        \path[->,font=\scriptsize] (m-1-1) edge node[above] {$p_Y$} (m-1-2);
        \path[->,font=\scriptsize] (m-2-1) edge node[below] {$f$} (m-2-2);
        \path[->,font=\scriptsize] (m-1-1) edge node[left] {$p_X$} (m-2-1);
        \path[->,font=\scriptsize] (m-1-2) edge node[right] {$g$} (m-2-2);

        \path[->,font=\scriptsize] (m-1-4) edge node[above] {$g$} (m-1-5);
        \path[->,font=\scriptsize] (m-2-4) edge node[below] {$i_X$} (m-2-5);
        \path[->,font=\scriptsize] (m-1-4) edge node[left] {$f$} (m-2-4);
        \path[->,font=\scriptsize] (m-1-5) edge node[right] {$i_Y$} (m-2-5);

        \begin{scope}[shift=($(m-1-1)!.4!(m-2-2)$)]
          \draw +(-.2,0) -- +(0,0)  -- +(0,.2);
        \end{scope}
        \begin{scope}[shift=($(m-1-4)!.7!(m-2-5)$)]
          \draw +(-.2,0) -- +(0,0)  -- +(0,.2);
        \end{scope}
      \end{tikzpicture} \]
    pero estos no tienen por qué ser conmutativos. Entonces, podemos tomar el
    ecualizador (resp. coecualizador) correspondiente:
    \[ \begin{tikzcd}
        X\times_Z Y \ar{r}{\tcol{\overline{f}}{\overline{g}}} & X\times Y \ar[shift left=0.25em]{r}{f\circ p_X}\ar[shift right=0.25em]{r}[swap]{g\circ p_Y} & Z && Z \ar[shift left=0.25em]{r}{i_X\circ f}\ar[shift right=0.25em]{r}[swap]{i_Y\circ g} & X\sqcup Y \ar{r}{(\overline{f}, \overline{g})} & X \sqcup_Z Y
      \end{tikzcd} \]
    El lector puede comprobar que en este caso la propiedad universal de
    (co)equializadores corresponde a la propiedad universal de (co)productos
    fibrados.
  \end{proof}
\end{observacion}

% % % % % % % % % % % % % % % % % % % % % % % % % % % % % %

\section{Límites y colímites en general}

%\marginpar{\footnotesize lección\\7.06.2018}

\begin{definicion}
  Sea $\vcateg{I}$ una categoría pequeña y sea
  $F\colon \vcateg{I} \to \vcateg{C}$ un funtor. Un \term{límite} de $F$ es un
  objeto $\lim_\vcateg{I} F \in \Ob (\vcateg{C})$ con un conjunto de morfismos
  $\{ \lim_\vcateg{I} F \to F (i) \}_{i\in \Ob (\vcateg{I})}$ tales que cada
  morfismo $i\to j$ en $\vcateg{I}$ induce un diagrama conmutativo
  $$\begin{tikzcd}
    & \lim_\vcateg{I} F\ar{dl}\ar{dr}\\
    F (i)\ar{rr} & & F (j) \\
  \end{tikzcd}$$
  Además, pedimos la siguiente propiedad universal: si para cualquier otro
  objeto $X\in \Ob (\vcateg{C})$ con un conjunto de morfismos
  $\{ X \to F (i) \}_{i\in \Ob (\vcateg{I})}$ que conmuta con los morfismos de
  $\vcateg{I}$ en el sentido de arriba existe un único morfismo
  $X \to \lim_\vcateg{I} F$ que da diagramas conmutativos
  $$\begin{tikzcd}
    & \lim_\vcateg{I} F\ar{ddl}\ar{ddr} \\
    & X\ar{dl}\ar{dr}\ar[dashed]{u}[swap]{\exists !} \\
    F (i)\ar{rr} & & F (j) \\
  \end{tikzcd}$$

  Un \term{colímite} de $F$ es un objeto
  $\colim_\vcateg{I} F \in \Ob (\vcateg{C})$ con un conjunto de morfismos
  $\{ F(i) \to \colim_\vcateg{I} F \}_{i\in \vcateg{I}}$ tales que para cada
  morfismo $i\to j$ en $\vcateg{I}$ hay un diagrama conmutativo
  $$\begin{tikzcd}
    F (i)\ar{rr}\ar{dr} & & F (j)\ar{dl} \\
    & \colim_\vcateg{I} F
  \end{tikzcd}$$
  Y se pide que $\colim_\vcateg{I} F$ satisfaga la propiedad universal
  $$\begin{tikzcd}
    F (i)\ar{rr}\ar{dr}\ar{ddr} & & F (j)\ar{dl}\ar{ddl} \\
    & \colim_\vcateg{I} F\ar[dashed]{d}{\exists !}\\
    & X
  \end{tikzcd}$$
\end{definicion}

\begin{observacion}
  Si un límite o colímite existe, es único salvo isomorfismo único.
\end{observacion}

\begin{ejemplo}
  Ya conocemos algunos ejemplos básicos de límites y colímites.

  \begin{enumerate}
  \item[1)] La \term{categoría vacía} $\emptyset$ es la categoría que no tiene
    ningún objeto. Trivialmente, para cualquier categoría $\vcateg{C}$ hay un
    funtor único $\emptyset\to\vcateg{C}$. El límite de este funtor es un objeto
    terminal y el colímite es un objeto inicial.

  \item[2)] Sea $\vcateg{I}$ una categoría que tiene solo dos objetos $1$ y $2$
    y ningún morfismo salvo $\id{1}$ y $\id{2}$:

    \[ \begin{tikzpicture}
        \matrix(m)[matrix of math nodes, row sep=2em, column sep=2em, text height=1.5ex, text depth=0.25ex]
        {1 & 2\\};

        \draw[->] (m-1-1) edge[loop above] node[above,font=\scriptsize] {$\idid$} (m-1-1);
        \draw[->] (m-1-2) edge[loop above] node[above,font=\scriptsize] {$\idid$} (m-1-2);
      \end{tikzpicture} \]

    Un funtor $F\colon \vcateg{I}\to \vcateg{C}$ corresponde a una elección de
    objetos $X_1 = F (1), X_2 = F (2)\in \Ob (\vcateg{C})$. El límite de $F$ es
    el producto $X_1\times X_2$ y el colímite es el coproducto $X_1\sqcup X_2$.

    De la misma manera, todo conjunto $I$ puede ser interpretado como una
    categoría pequeña $\vcateg{I}$ cuyos objetos corresponden a los elementos de
    $I$ y cuyos morfismos son $\id{i}$ para todo $i\in I$. Un funtor
    $F\colon \vcateg{I}\to\vcateg{C}$ corresponde a una elección de una familia
    de objetos $X_i$. El límite de este funtor es el producto $\prod_i X_i$ y el
    colímite es el coproducto $\coprod_i X_i$.

  \item[3)] Sea $I$ la categoría
    $$\begin{tikzpicture}
      \matrix(m)[matrix of math nodes, row sep=2em, column sep=2em, text height=1.5ex, text depth=0.25ex]
      { & 2\\
        1 & 3\\};

      \draw[->] (m-2-1) edge (m-2-2);
      \draw[->] (m-1-2) edge (m-2-2);

      \draw[->] (m-2-2) edge[loop right] node[right,font=\scriptsize] {$\idid$} (m-2-2);
      \draw[->] (m-2-1) edge[loop above] node[above,font=\scriptsize] {$\idid$} (m-2-1);
      \draw[->] (m-1-2) edge[loop right] node[right,font=\scriptsize] {$\idid$} (m-1-2);
    \end{tikzpicture}$$
    Un funtor $F\colon \vcateg{I}\to \vcateg{C}$ corresponde a un diagrama
    $$\begin{tikzcd}
      & Y\ar{d} \\
      X\ar{r} & Z
    \end{tikzcd}$$
    El límite de $F$ es el producto fibrado de $X$ e $Y$ sobre $Z$
    $$\begin{tikzpicture}
      \matrix(m)[matrix of math nodes, row sep=2em, column sep=2em, text height=1.5ex, text depth=0.25ex]{
        X\times_Z Y & Y \\
        X & Z \\};
      \path[->] (m-1-1) edge (m-1-2);
      \path[->] (m-2-1) edge (m-2-2);
      \path[->] (m-1-1) edge (m-2-1);
      \path[->] (m-1-2) edge (m-2-2);

      \begin{scope}[shift=($(m-1-1)!.4!(m-2-2)$)]
        \draw +(-.2,0) -- +(0,0)  -- +(0,.2);
      \end{scope}
    \end{tikzpicture}$$

    De la misma manera, si $I$ es la categoría
    $$\begin{tikzpicture}
      \matrix(m)[matrix of math nodes, row sep=2em, column sep=2em, text height=1.5ex, text depth=0.25ex]
      {3 & 2\\
        1\\};

      \draw[->] (m-1-1) edge (m-1-2);
      \draw[->] (m-1-1) edge (m-2-1);

      \draw[->] (m-1-1) edge[loop left] node[left,font=\scriptsize] {$\idid$} (m-2-2);
      \draw[->] (m-1-2) edge[loop right] node[right,font=\scriptsize] {$\idid$} (m-2-1);
      \draw[->] (m-2-1) edge[loop left] node[left,font=\scriptsize] {$\idid$} (m-1-2);
    \end{tikzpicture}$$
    entonces un funtor $F\colon \vcateg{I}\to\vcateg{C}$ corresponde a un diagrama
    $$\begin{tikzcd}
      Z\ar{r}\ar{d} & Y \\
      X
    \end{tikzcd}$$
    y el colímite de $F$ es el coproducto fibrado de $X$ e $Y$ bajo $Z$
    \[ \begin{tikzpicture}
        \matrix(m)[matrix of math nodes, row sep=2em, column sep=2em,
        text height=1.5ex, text depth=0.25ex]{
          Z & Y \\
          X & X\sqcup_Z Y \\};
        \path[->] (m-1-1) edge (m-1-2);
        \path[->] (m-2-1) edge (m-2-2);
        \path[->] (m-1-1) edge (m-2-1);
        \path[->] (m-1-2) edge (m-2-2);

        \begin{scope}[shift=($(m-1-1)!.7!(m-2-2)$)]
          \draw +(-.2,0) -- +(0,0)  -- +(0,.2);
        \end{scope}
      \end{tikzpicture} \]

  \item[4)] Sea $\vcateg{I}$ la categoría
    $$\begin{tikzpicture}
      \matrix(m)[matrix of math nodes, row sep=2em, column sep=2em, text height=1.5ex, text depth=0.25ex]
      {1 & 2\\};

      \draw[->,transform canvas={yshift=0.3em}] (m-1-1) edge (m-1-2);
      \draw[->,transform canvas={yshift=-0.3em}] (m-1-1) edge (m-1-2);

      \draw[->] (m-1-1) edge[loop above] node[above,font=\scriptsize] {$\idid$} (m-1-1);
      \draw[->] (m-1-2) edge[loop above] node[above,font=\scriptsize] {$\idid$} (m-1-2);
    \end{tikzpicture}$$
    Un funtor $F\colon\vcateg{I}\to\vcateg{C}$ corresponde a un diagrama
    $$\begin{tikzcd}
      X\ar[shift left=0.3em]{r}{f}\ar[shift right=0.3em]{r}[swap]{g} & Y
    \end{tikzcd}$$
    El límite de $F$ es el ecualizador de $f$ y $g$ y el colímite es el coecualizador. \qedhere
  \end{enumerate}
\end{ejemplo}

Muchas categorías importantes son \term{completas y cocompletas}, es decir
tienen límites y colímites para cualquier categoría pequeña
$\vcateg{I}$. Entre ellas, la categoría de conjuntos
$\categ{Set}$ y muchas categorías de conjuntos con estructura adicional:
espacios topológicos $\categ{Top}$, grupos
$\categ{Grp}$, anillos $\categ{Ring}$, espacios vectoriales
$k\categ{-Vect}$, etc. En estos casos los límites y colímites tienen una
descripción explícita:
\begin{align*}
  \lim_\vcateg{I} F & = \{ (x_i) \in \prod_{i\in \Ob (\vcateg{I})} F(i) \mid F (f) (x_i) = x_j \text{ para cada } f\colon i\to j \}.\\
  \colim_\vcateg{I} F & = \left.\coprod_{i\in \Ob (\vcateg{I})} F (i) \right/\!\sim,
\end{align*}
donde la relación de equivalencia $\sim$ es generada por
$$F (i) \ni x_i \sim F(f) (x_i) \in F (j) \text{ para todo } f\colon i\to j.$$
El lector puede hacer los ajustes (obvios) para el caso de conjuntos con
estructura adicional. Por ejemplo, para espacios topológicos por
$\left.\coprod_{i\in \Ob (\vcateg{I}} F (i)
\right/\!\sim$ se entiende el espacio con la topología cociente.

\begin{ejemplo}
  Consideremos los números naturales $\NN$ con la relación $\ge$. Esto
  corresponde a una categoría
  $$\begin{tikzcd}
    \cdots\ar{r} & 2\ar{r} & 1\ar{r} & 0
  \end{tikzcd}$$
  Consideremos el funtor $F\colon \NN\to \categ{CRing}$ que a cada $i$ asocia el
  anillo conmutativo $\ZZ/p^{i+1}\ZZ$ y a cada morfismo $j\to i$ donde $j \ge i$
  asocia el homomorfismo de anillos $\ZZ/p^{j+1}\ZZ \to \ZZ/p^{i+1}\ZZ$ que toma
  el resto módulo $p^{i+1}$. El límite de este funtor es el anillo de enteros
  $p$-ádicos:
  $$\ZZ_p = \lim_{i\ge 0} \ZZ/p^{i+1}\ZZ.$$
  En efecto, la construcción general nos dice que
  \[ \lim_{i\ge 0} \ZZ/p^{i+1}\ZZ =
     \{ (x_i) \in \prod_{i\ge 0} \ZZ/p^{i+1}\ZZ \mid
            x_j \equiv x_i \pmod{p^{i+1}} \text{ para cada }j\ge i \}. \]
  Entonces, los elementos del límite son sucesiones
  $(x_0,x_1,x_2,x_3,\ldots)$ de restos módulo $p^{i+1}$ que satisfacen
  \begin{align*}
    x_1 & \equiv x_0 \pmod{p},\\
    x_2 & \equiv x_1 \pmod{p^2},\\
    x_3 & \equiv x_2 \pmod{p^3},\\
        & \cdots
  \end{align*}
  Esto significa que existen números únicos $0\le a_i\le p-1$ tales que
  \begin{align*}
    x_0 & = a_0,\\
    x_1 & = a_0 + a_1\,p,\\
    x_2 & = a_0 + a_1\,p + a_2\,p^2,\\
    x_3 & = a_0 + a_1\,p + a_2\,p^2 + a_3\,p^3,\\
        & \cdots
  \end{align*}
  De esta manera obtenemos las expansiones $p$-ádicas
  \[ (x_0,x_1,x_2,x_3,\ldots) \leftrightarrow
     a_0 + a_1\,p + a_2\,p^2 + a_3\,p^3 + \cdots. \qedhere \]
\end{ejemplo}

\begin{ejemplo}
  Consideremos los números enteros positivos $\ZZ_+$ ordenados por la relación
  de divisibilidad $n \succeq m \iff m \mid n$. Esto da lugar a una categoría
  \[ \begin{tikzcd}\cdots & 6\ar[bend right]{rrr}\ar[bend right]{rrrrr} & 5\ar[bend left]{rrrr} & 4\ar[bend left]{rr}\ar[bend left]{rrr} & 3\ar[bend right]{rr} & 2\ar{r} & 1 \end{tikzcd} \]

  Cuando $m\mid n$, está bien definido el homomorfismo $\ZZ/n\ZZ\to \ZZ/m\ZZ$
  que a cada resto módulo $n$ asocia el resto módulo $m$ correspondiente. Esto
  define un funtor $n \rightsquigarrow \ZZ/n\ZZ$ y su límite es
  $$\widehat{\ZZ} = \lim_{n\ge 1} \ZZ/n\ZZ$$
  llamado el \term{anillo de enteros profinitos}. Por el teorema chino del resto, hay isomorfismos naturales
  \[ \ZZ/n\ZZ \isom \ZZ/p_1^{k_1}\ZZ\times\cdots\times  \ZZ/p_s^{k_s}\ZZ, \quad
     \text{donde }n = p_1^{k_1}\cdots p_s^{k_s}. \]
  Luego,
  \[ \lim_{n\ge 1} \ZZ/n\ZZ \isom
     \prod_{p\text{ primo}} \lim_i \ZZ/p^i\ZZ \isom
     \prod_p \ZZ_p. \qedhere \]
\end{ejemplo}
  
\begin{ejemplo}
  ~

  \begin{enumerate}
  \item[1)] Consideremos el funtor que a cada entero positivo $n$ asocia el
    grupo simétrico $S_n$ y a $m\le n$ asocia el monomorfismo $S_m \mono S_n$
    que extiende toda permutación $\sigma$ de $\{ 1,2,\ldots,m \}$ a una
    permutación de $\{ 1,2,\ldots,n \}$ poniendo $\sigma (i) = i$ si $i > m$. El
    colímite correspondiente es el grupo simétrico infinito
    $$S_\infty = \colim_{n\ge 1} S_n.$$
    Los elementos de $S_\infty$ son permutaciones de números naturales $\NN$ que
    tienen un número finito de puntos no fijos.

  \item[2)] De la misma manera, a cada $m \le n$ se puede asociar un
    monomorfismo de grupos de matrices invertibles $\GL_m (R) \to \GL_n (R)$
    poniendo $1$ en la diagonal:
    $$A \mapsto \begin{pmatrix}
      A \\
      & 1 \\
      && \ddots \\
      &&& 1
    \end{pmatrix}.$$
    El colímite es el grupo
    $$\GL_\infty (R) = \colim_{n\ge 1} \GL_n (R).$$
    Sus elementos son las transformaciones lineales invertibles
    $R^\NN \to R^\NN$ donde cada una afecta solamente un número finito de
    vectores de la base.

  \item[3)] Los grupos de las raíces de la unidad
    $$\mu_n (\CC) \dfn \{ z\in \CC \mid z^n = 1 \} \isom \ZZ/n\ZZ$$
    vienen con inclusiones naturales $\mu_m (\CC) \subset \mu_n (\CC)$ para
    $m\mid n$. El colímite correspondiente es el grupo de todas las raíces de la
    unidad
    $$\mu_\infty (\CC) = \colim_{n\ge 1} \mu_n (\CC) = \bigcup_{n\ge 1} \mu_n (\CC).$$
    Este grupo es isomorfo al grupo cociente $\QQ/\ZZ$: el homomorfismo
    \begin{align*}
      \QQ & \to \mu_\infty (\CC),\\
      m/n & \to e^{2\pi i\,m/n}
    \end{align*}
    es sobreyectivo y tiene $\ZZ$ como su núcleo. Tenemos
    \[ \QQ/\ZZ = \colim_{n\ge 1} \ZZ/n\ZZ. \qedhere \]
  \end{enumerate}
\end{ejemplo}

%\begin{observacion}
%\label{limites-y-colimites-como-funtores-representables}
%Sea $\vcateg{I}$ una categoría pequeña. Consideremos el funtor
%\begin{align*}
%\Delta\colon \vcateg{C} & \to \Fun (\vcateg{I}, \vcateg{C}),\\
%X & \rightsquigarrow \Delta_X
%\end{align*}
%que a cada objeto $X\in \Ob (\vcateg{C})$ asocia el funtor constante $\Delta_X \colon \vcateg{I} \to \vcateg{C}$ (tal que $\Delta_X (i) = X$ para cada $i\in \Ob (\vcateg{I})$).
%
%Para todo funtor $F\colon \vcateg{I} \to \vcateg{C}$ tenemos biyecciones naturales
%\begin{align*}
%\Nat (\Delta_X, F) & \isom \Hom_\vcateg{C} (X, \lim_\vcateg{I} F),\\
%\Nat (F, \Delta_X) & \isom \Hom_\vcateg{C} (\colim_\vcateg{I} F, X).
%\end{align*}
%
%En otras palabras, el límite $\lim_\vcateg{I} F$ es un objeto que representa al funtor contravariante
%\begin{align*}
%\vcateg{C}^\circ & \to \categ{Set},\\
%X & \mapsto \Nat (\Delta_X, F),\\
%\end{align*}
%y el colímite $\colim_\vcateg{I} F$ es un objeto que representa al funtor covariante
%\begin{align*}
%\vcateg{C} & \to \categ{Set},\\
%X & \mapsto \Nat (F, \Delta_X).
%\end{align*}
%
%\begin{proof}
%Notamos que una colección de diagramas conmutativos
%$$\begin{tikzcd}
% & \lim_\vcateg{I} F\ar{dl}\ar{dr}\\
%F (i)\ar{rr} & & F (j) \\
%\end{tikzcd}$$
%es precisamente una transformación natural $\Delta_X \Rightarrow F$ y una colección de diagramas conmutativos
%$$\begin{tikzcd}
%F (i)\ar{rr}\ar{dr} & & F (j)\ar{dl} \\
% & \colim_\vcateg{I} F
%\end{tikzcd}$$
%es una transformación natural $F\Rightarrow \Delta_X$. Las propiedades universales del límite y colímite corresponden a las biyecciónes $\Nat (\Delta_X, F) \isom \Hom_\vcateg{C} (X, \lim_\vcateg{I} F)$ y $\Nat (F, \Delta_X) \isom \Hom_\vcateg{C} (\colim_\vcateg{I} F, X)$.
%\end{proof}
%\end{observacion}

\begin{observacion}
  \label{obs:lim-colim-y-hom}
  El $\Hom_\vcateg{C} (X,-)$ covariante preserva límites: hay una biyección
  natural
  \[ \Hom_\vcateg{C} (X, \lim_{i\in \Ob (\vcateg{I})} F (i)) \isom
     \lim_{i\in \Ob (\vcateg{I})} \Hom_\vcateg{C} (X,F (i)). \]
   El $\Hom_\vcateg{C} (-, X)$ contravariante convierte colímites en límites:
   hay una biyección natural
   \[ \Hom_\vcateg{C} (\colim_{i\in \Ob (\vcateg{I})} F (i), X) \isom
      \lim_{i\in \Ob (\vcateg{I})} \Hom_\vcateg{C} (F (i), X). \]

    En otras palabras, $\lim_\vcateg{I} F$ es el objeto que representa al funtor
    contravariante $\lim_{i\in \Ob (\vcateg{I})} \Hom_\vcateg{C} (-,F (i))$ y
    $\colim_\vcateg{I} F$ es el objeto que representa al funtor covariante
    $\lim_{i\in \Ob (\vcateg{I})} \Hom_\vcateg{C} (F (i), -)$.

    \begin{proof}
      Se verifica directamente a partir de las definiciones.
    \end{proof}
\end{observacion}

En particular, si $\lim_\vcateg{I} F$ (resp. $\colim_\vcateg{I} F$) existe, es
único salvo isomorfismo por las propiedades generales de funtores adjuntos.

\begin{observacion}
  \label{adjuntos-preservan-co-limites}
  Todo funtor adjunto por la izquierda preserva colímites, y todo funtor adjunto
  por la derecha preserva límites. A saber, sea
  $L\colon \vcateg{C} \to \vcateg{D}$ adjunto por la izquierda a
  $R\colon \vcateg{D}\to \vcateg{C}$:
  $$\Hom_\vcateg{D} (L (X), Y) \isom \Hom_\vcateg{C} (X, R (Y)).$$
  Sea $F\colon \vcateg{I} \to \vcateg{C}$ algún funtor y sea
  $\colim_\vcateg{I} F \in \Ob (\vcateg{C})$ su colímite. Entonces, hay un
  isomorfismo natural
  $$L (\colim_\vcateg{I} F) \isom \colim_\vcateg{I} (L\circ F).$$
  De modo similar, para un funtor $F\colon \vcateg{I} \to \vcateg{D}$ hay un
  isomorfismo natural
  $$R (\lim_\vcateg{I} F) \isom \lim_\vcateg{I} (R\circ F).$$
\end{observacion}

En particular, un funtor adjunto por la izquierda preserva coproductos,
coproductos fibrados, coecualizadores, etc., y un funtor adjunto por la derecha
preserva productos, productos fibrados, ecualizadores, etc. Es una
generalización de \ref{adjunto-a-la-izquierda-preserva-coproductos},
y la demostración es idéntica.

\begin{proof}
  Por ejemplo, en el caso de límites, tenemos para cada $X \in \Ob (\vcateg{C})$
  isomorfismos naturales
  \begin{align*}
    \Hom_\vcateg{C} (X, R (\lim_\vcateg{I} F)) & \isom \Hom_\vcateg{D} (L (X), \lim_\vcateg{I} F) \quad \text{por la adjunción entre }L\text{ y }R \\
                                               & \isom \lim_{i\in\Ob (\vcateg{I})} \Hom_\vcateg{D} (L (X), F (i)) \quad \text{por \ref{obs:lim-colim-y-hom}} \\
                                               & \isom \lim_{i\in\Ob (\vcateg{I})} \Hom_\vcateg{D} (X, R (F (i))) \quad \text{por la adjunción} \\
                                               & \isom \Hom_\vcateg{D} (X, \lim_\vcateg{I} (R \circ F)) \quad \text{por \ref{obs:lim-colim-y-hom}}.
  \end{align*}

  Entonces el lema de Yoneda implica
  \[ R (\lim_\vcateg{I} F) \isom \lim_\vcateg{I} (R\circ F). \qedhere \]
\end{proof}

\begin{proposicion}
  Todo límite (resp. límite finito) puede ser expresado por productos y
  ecualizadores (resp. productos finitos y ecualizadores). Todo colímite
  (resp. colímite finito) puede ser expresado por coproductos y coecualizadores
  (resp. coproductos finitos y coecualizadores).
\end{proposicion}

Aquí el producto (resp. coproducto) vacío se considera como un objeto terminal
(resp. inicial).

\begin{proof}
  Sea $\vcateg{I}$ una categoría pequeña y sea $F\colon \vcateg{I}\to\vcateg{C}$
  un funtor. Sea $\Hom (\vcateg{I})$ el conjunto de morfismos de
  $\vcateg{I}$. Para todo morfismo $f\colon i\to j$ denotemos
  $$\cod (f) \dfn j.$$
  Consideremos los siguientes productos sobre todos los objetos en $\vcateg{I}$
  y todos los morfismos en $\vcateg{I}$:
  \begin{align*}
    \Pi & \dfn \prod_{i\in \Ob(\vcateg{I})} F(i),\\
    \vec{\Pi} & \dfn \prod_{f\in \Hom (\vcateg{I})} F (\cod (f))
  \end{align*}
  Estos productos vienen con las proyecciones
  $$p_i\colon \Pi \to F(i), \quad \vec{p}_f\colon \vec{\Pi} \to F (\cod (f)).$$
  Por la propiedad universal de $\vec{\Pi}$ existes morfismos únicos
  $\alpha,\beta\colon \Pi \to \vec{\Pi}$ tales que para todo $f\colon i\to j$ en
  $\Hom (\vcateg{I})$ se cumple
  \begin{align*}
    \vec{p}_f\circ\alpha & = p_j,\\
    \vec{p}_f\circ \beta & = F(f)\circ p_i.
  \end{align*}

  \[ \begin{tikzcd}
      \Pi\ar[dashed]{d}[swap]{\alpha}{\exists !}\ar{dr}{p_j} & & \Pi\ar[dashed]{d}[swap]{\beta}{\exists !} \ar{r}{p_i} & F (i)\ar{d}{F (f)} \\
      \vec{\Pi} \ar{r}[swap]{\vec{p}_f} & F (j) & \vec{\Pi} \ar{r}[swap]{\vec{p}_f} & F (j)
    \end{tikzcd} \]

  Consideremos el ecualizador
  \[ \begin{tikzcd}
      E\ar{r}{e} & \Pi\ar[shift left=0.3em]{r}{\alpha}\ar[shift right=0.3em]{r}[swap]{\beta} & \vec{\Pi}
    \end{tikzcd} \]

  Vamos a probar que $E$ junto con morfismos $p_i\circ e\colon E\to F (i)$ es el límite de $F$. Primero, notamos que los diagramas
  $$\begin{tikzcd}
    & E\ar{dl}[swap]{p_i\circ e}\ar{dr}{p_j\circ e} \\
    F (i)\ar{rr}[swap]{F (f)} & & F (j)
  \end{tikzcd}$$
  conmutan para todo $f\colon i\to j$. En efecto,
  $$F(f)\circ p_i\circ e = \vec{p}_f\circ\beta\circ e = \vec{p}_f\circ\alpha\circ e = p_j\circ e.$$

  Ahora sea $X$ otro objeto con morfismos $q_i\colon X\to F (i)$ que hacen conmutar los diagramas
  $$\xymatrix{
    & X\ar[dl]_{q_i}\ar[dr]^{q_j} \\
    F(i)\ar[rr]_{F(f)} & & F(j)
  }$$
  Por la propiedad universal de $\Pi$, existe un morfismo único $
  \gamma\colon X \to \Pi$ tal que
  $$q_i = p_i\circ\gamma$$
  para todo $i\in\Ob(\vcateg{I})$.

  \[ \begin{tikzcd}
      X\ar[dashed]{d}[swap]{\gamma}{\exists !}\ar{dr}{q_i} \\
      \Pi \ar{r}[swap]{p_i} & F (i)
    \end{tikzcd} \]

  Ahora para todo $f\colon i\to j$ tenemos
  \[ \vec{p}_f\circ \alpha\circ \gamma =
     p_j\circ \gamma =
     q_j =
     F(f)\circ q_i =
     F(f)\circ p_i\circ\gamma =
     \vec{p}_f\circ\beta\circ\gamma. \]

  Esto significa que $\alpha\circ\gamma = \beta\circ\gamma$, y la propiedad
  universal del ecualizador $E$ implica que existe un morfismo único
  $\delta\colon X\to E$ tal que $e\circ \delta = \gamma$.

  \[ \begin{tikzcd}
      E\ar{r}{e} & \Pi\ar[shift left=0.25em]{r}{\alpha}\ar[shift right=0.25em]{r}[swap]{\beta} & \vec{\Pi} \\
      X\ar{ur}[swap]{\delta}\ar[dashed]{u}{\exists !}
    \end{tikzcd} \]

  Luego, $q_i = p_i\circ\gamma = p_i\circ e\circ\delta$, así que los diagramas
  \[ \begin{tikzcd}[row sep=4em, column sep=4em]
      & X \ar{ddl}[swap]{q_i}\ar{ddr}{q_j}\ar{d}{\delta} \\
      & E\ar{dl}{p_i\circ e}\ar{dr}[swap]{p_j\circ e} \\
      F (i)\ar{rr}[swap]{F (f)} & & F (j)
    \end{tikzcd} \]
  conmutan. Falta solo ver que el morfismo $\delta$ es único.
  Sea $\delta'\colon X\to E$ otro morfismo tal que
  $q_i = p_i\circ e \circ \delta'$ para todo $i\in \Ob(\vcateg{I})$.
  Entonces,
  $$p_i\circ e\circ\delta = p_i\circ\gamma = q_i = p_i\circ e\circ\delta',$$
  lo que implica que $e\circ\delta = e\circ\delta^\prime$. El morfismo $e$ es
  mono, siendo un ecualizador, y por lo tanto $\delta = \delta'$.
\end{proof}

\begin{corolario}
  Todo límite finito (resp. colímite finito) puede ser expresado por productos
  fibrados y objetos terminales (resp. coproductos fibrados y objetos
  iniciales).

  \begin{proof}
    Como hemos notado en
    \ref{productos-como-productos-fibrados-sobre-el-objeto-terminal},
    los productos (resp. coproductos) son productos (resp. coproductos) fibrados
    sobre un objeto terminal (resp. inicial).
    En \ref{ecualizadores-como-productos-fibrados} hemos visto que
    los ecualizadores (resp. coecualizadores) pueden ser expresados como
    productos (resp. coproductos) fibrados.
  \end{proof}
\end{corolario}

% % % % % % % % % % % % % % % % % % % % % % % % % % % % % %

\section{Equivalencias de categorías}

%\marginpar{\footnotesize lección\\8.06.2018}

Dos categorías comparten las mismas propiedades cuando son equivalentes.

\begin{definicion}
  Se dice que un funtor $F\colon \vcateg{C}\to \vcateg{D}$ es una
  \term{equivalencia de categorías} si existe otro funtor
  $G\colon \vcateg{D}\to \vcateg{C}$ tal que
  $$\Id_\vcateg{C} \isom G\circ F \quad\text{y}\quad F\circ G \isom \Id_\vcateg{D}.$$
  Esto significa que para todo $X\in \Ob (\vcateg{C})$ tenemos isomorfismos
  $\eta_X\colon X\to G F (X)$ y para todo $A\in \Ob (\vcateg{D})$ tenemos
  isomorfismos $\epsilon_A\colon FG (A) \to A$, de tal modo que para todo
  morfismo $f\colon X\to Y$ en $\vcateg{C}$ y todo morfismo $h\colon A\to B$ en
  $\vcateg{D}$ los siguientes cuadrados conmutan:

  \[ \begin{tikzcd}
      X \ar{r}{\eta_X}[swap]{\isom}\ar{d}[swap]{f} & G F (X) \ar{d}{GF (f)} & & F G (A)\ar{r}{\epsilon_A}[swap]{\isom}\ar{d}[swap]{FG (h)} & A\ar{d}{h} \\
      Y \ar{r}{\eta_Y}[swap]{\isom} & G F (Y) & & F G (B) \ar{r}{\epsilon_B}[swap]{\isom} & B
    \end{tikzcd} \]
\end{definicion}

Aquí se pide que las composiciones $G\circ F$ y $F\circ G$ sean \emph{isomorfas}
a funtores identidad. Cuando tenemos \emph{igualdades}
$G\circ F = \Id_\vcateg{C}$ y $F\circ G = \Id_\vcateg{D}$, se dice que las
categorías $\vcateg{C}$ y $\vcateg{D}$ son \term{isomorfas}. Aunque el
isomorfismo de categorías parece algo más natural, en practica es inútil ya que
exige una biyección entre objetos de $\vcateg{C}$ y $\vcateg{D}$ que preserva
los morfismos, es decir que $\vcateg{C}$ y $\vcateg{D}$ son en esencia
idénticas. La equivalencia de categorías es algo mucho más interesante y quiere
decir que existe una biyección entre \emph{clases de isomorfismo} de objetos de
$\vcateg{C}$ y $\vcateg{D}$ y biyecciones entre morfismos.

\begin{ejemplo}
  Consideremos la categoría de espacios vectoriales sobre $k$ de dimensión
  finita $k\categ{-vect}$. Como sabemos del álgebra lineal, dos espacios
  vectoriales $U$ y $V$ son isomorfos si y solamente si
  $\dim_k U = \dim_k V$. Así, escogiendo una base, para cada espacio vectorial
  $V$ podemos escoger un isomorfismo $V \isom k^d$. Esto define una equivalencia
  entre la categoría $k\categ{-vect}$ y la categoría cuyos objetos son $k^n$
  para $n\in \NN$ y los morfismos son las aplicaciones lineales $k^n\to k^m$;
  es decir matrices de $m\times n$. La categoría $k\categ{-vect}$ no es pequeña
  porque para cada espacio vectorial el conjunto subyacente puede ser cualquier
  cosa, entonces todos los espacios vectoriales (aún de dimensión finita) no
  forman un conjunto. Sin embargo, los espacios vectoriales $k^d$ forman
  un conjunto, entonces una categoría pequeña. Los cursos básicos del álgebra
  lineal tratan precisamente de esta equivalencia de categorías.
\end{ejemplo}

\begin{ejemplo}
  Para la categoría $k\categ{-vect}$ de espacios vectoriales de dimensión finita
  el funtor del espacio vectorial dual nos da un isomorfismo natural
  $$\ev_V\colon V \isom (V^*)^*.$$
  Aquí uno de los ``$*$'' denota el funtor
  $k\categ{-vect} \to k\categ{-vect}^\mathrm{op}$ y otro ``$*$'' denota el
  funtor $k\categ{-vect}^\mathrm{op} \to k\categ{-vect}$. El isomorfismo de
  arriba corresponde a los isomorfismos de funtores
  \[ \eta\colon \Id_{k\categ{-vect}} \Rightarrow *\circ *, \quad
     \epsilon\colon *\circ* \Rightarrow \Id_{k\categ{-vect}^\mathrm{op}}. \]
  Entonces tenemos una equivalencia de categorías
  $$k\categ{-vect} \cequiv k\categ{-vect}^\mathrm{op}.$$
  No existe ningún isomorfismo \emph{natural} $V \isom V^*$, porque $V$ y $V^*$
  son objetos de diferentes categorías. De hecho, el isomorfismo $V \isom V^*$
  que normalmente se estudia en cursos del álgebra lineal depende de una base
  fija de $V$, o de otros datos suplementarios (por ejemplo, una forma bilineal
  no degenerada fija $\langle\cdot,\cdot\rangle\colon V\times V \to k$).
\end{ejemplo}

He aquí una caracterización útil de las equivalencias.

\begin{proposicion}
  \label{equivalencia-fielmente-pleno-y-esencialmente-sobreyectivo}
  Un funtor $F\colon \vcateg{C} \to \vcateg{D}$ define una equivalencia de
  categorías si y solamente si

  \begin{enumerate}
  \item[1)] $F$ es fielmente pleno;

  \item[2)] $F$ es \term{esencialmente sobreyectivo}: para cada objeto
    $Y \in \Ob (\vcateg{D})$ existe algún objeto $X \in \Ob (\vcateg{C})$ tal
    que $F (X) \isom Y$.
  \end{enumerate}

\begin{proof}
  Si $F$ es una equivalencia de categorías, es fácil comprobar que $F$ cumple
  las condiciones 1) y 2). De hecho, por la definición de equivalencia, tenemos
  otro funtor $G\colon \vcateg{D} \to \vcateg{C}$ junto con un isomorfismos de
  funtores $\Id_\vcateg{C} \isom G\circ F$ y $F\circ G \isom \Id_\vcateg{D}$, es
  decir, familias de isomorfismos naturales
  $\eta_X\colon X\xrightarrow{\isom} G F (X)$ y
  $\epsilon_Y\colon FG (Y) \xrightarrow{\isom} Y$.

  El hecho de que cada equivalencia de categorías sea esencialmente sobreyectiva
  es inmediato: para todo $Y\in \Ob (\vcateg{D})$ tenemos isomorfismo
  $F G (Y) \isom Y$.

  Para ver que la aplicación $f \mapsto F (f)$ es inyectiva, notemos que
  $F (f) = F (f')$ implica en particular $GF (f) = GF (f')$, y luego se ve que
  $f = f'$ desde el cuadrado conmutativo de abajo:

  \[ \begin{tikzcd}
      X\ar{r}{\eta_X}[swap]{\isom}\ar[shift left=0.25em]{d}{f}\ar[shift right=0.25em]{d}[swap]{f'} & G F (X)\ar{d}{GF (f) = GF (f')} \\
      X'\ar{r}{\eta_{X'}}[swap]{\isom} & G F (X') \\
    \end{tikzcd} \]

  \[ f = \eta_{X'}^{-1} \circ GF (f) \circ \eta_X =
     \eta_{X'}^{-1} \circ GF (f') \circ \eta_X = f'. \]

   Para ver que la aplicación $f \mapsto F (f)$ es sobreyectiva, será suficiente
   notar que para todo morfismo $g\colon F(X)\to F(X')$ en $\vcateg{D}$ existe
   un morfismo $f$ en $\vcateg{C}$ tal que $GF (f) = G (g)$, porque, como
   acabamos de ver, la aplicación $g \mapsto G (g)$ es inyectiva (por el mismo
   argumento aplicado a $G$). Se ve que podemos considerar
   $f \dfn \eta_{X'}^{-1} \circ G (g) \circ \eta_X$.

   \[ \begin{tikzcd}
       X\ar{r}{\eta_X}[swap]{\isom}\ar{d}[swap]{f} & G F (X)\ar{d}{G (g) = GF (f)} \\
       X'\ar{r}{\eta_{X'}}[swap]{\isom} & G F (X') \\
     \end{tikzcd} \]

   \[ GF (f) =
      \eta_{X'}\circ f\circ \eta_X^{-1} =
      \cancel{\eta_{X'}\circ \eta_{X'}^{-1}} \circ G (g) \circ \cancel{\eta_X\circ \eta_X^{-1}} =
      G (g). \]

    \vspace{1em}

    Ahora tenemos que ver que las condiciones 1) y 2) nos permiten construir un
    funtor $G\colon \vcateg{D} \to \vcateg{C}$ tal que
    $G\circ F\isom \Id_\vcateg{C}$ y $F\circ G \isom \Id_\vcateg{D}$.

    Si $F$ es fielmente pleno y esencialmente sobreyectivo, para cada
    $Y\in \Ob (\vcateg{D})$ tenemos una biyección natural
    \[ \Hom_\vcateg{D} (F (-), Y) \isom
       \Hom_\vcateg{D} (F (-), F (X)) \isom
       \Hom_\vcateg{C} (-, X). \]
    Entonces el funtor $\Hom_\vcateg{D} (F (-), Y)$ está representado por algún
    $X \in \Ob (\vcateg{C})$, y por lo tanto existe un funtor $G$ que es adjunto
    por la derecha a $F$ (véase \ref{representabilidad-y-adjunciones}): existe
    una biyección natural
    $$\Hom_\vcateg{D} (F (X), Y) \isom \Hom_\vcateg{C} (X, G(Y)).$$
    Tenemos las transformaciones naturales correspondientes (véase
    \ref{unidad-y-counidad})
    \[ \eta\colon \Id_\vcateg{C} \Rightarrow G\circ F, \quad
       \epsilon\colon F\circ G \Rightarrow \Id_\vcateg{D}. \]
    De hecho $\eta$ y $\epsilon$ son isomorfismos de funtores. Por ejemplo,
    veamos que $\eta_X\colon X \to GF (X)$ es un isomorfismo. Para
    $X'\in \Ob (\vcateg{C})$ apliquemos el funtor $\Hom_\vcateg{C}
    (X',-)$. Tenemos un morfismo
    $$\eta_X^*\colon \Hom_\vcateg{C} (X',X) \to \Hom_\vcateg{C} (X', GF (X)).$$
    Pero este no es otra cosa que la composición de las biyecciones naturales
    \[ \Hom_\vcateg{C} (X',X) \isom
       \Hom_\vcateg{D} (F (X'), F (X)) \isom
       \Hom_\vcateg{C} (X', GF (X)) \]
    (la primera porque $F$ es fielmente pleno, la segunda por la
    adjunción). Entonces $\eta_X^*$ es un isomorfismo natural para cada
    $X' \in \Ob (\vcateg{C})$, y el lema de Yoneda implica que $\eta_X$ es un
    isomorfismo. De modo similar, para ver que $\epsilon_Y\colon FG (Y) \to Y$
    es un isomorfismo, apliquemos el funtor $\Hom_\vcateg{D} (F (X), -)$:
    $$\epsilon_Y^*\colon \Hom_\vcateg{D} (F (X), FG (Y)) \to \Hom_\vcateg{D} (F (X),Y).$$
    Este es la composición de las biyecciones naturales
    \[ \Hom_\vcateg{D} (F (X), FG (Y)) \isom
       \Hom_\vcateg{C} (X, G (Y)) \isom
       \Hom_\vcateg{D} (F (X), Y). \qedhere \]
  \end{proof}
\end{proposicion}

\begin{observacion}
  \label{obs:equivalencia-adjunta-por-ambos-lados}
  Sean $F\colon \vcateg{C} \to \vcateg{D}$ y $G\colon \vcateg{D} \to \vcateg{C}$
  dos funtores que definen una equivalencia de categorías, junto con
  transformaciones naturales $\eta\colon \Id_\vcateg{C} \Rightarrow G\circ F$ y
  $\epsilon\colon F\circ G \Rightarrow \Id_\vcateg{D}$. Entonces

  \begin{enumerate}
  \item[1)] $F$ es adjunto por la izquierda a $G$, la unidad de la adjunción
    siendo $\eta_X$,

  \item[2)] $G$ es adjunto por la izquierda a $F$, la counidad de la adjunción
    siendo $\eta^{-1}_X$.
  \end{enumerate}

  \begin{proof}
    Por ejemplo, en el caso 1) tenemos que compobar que la aplicación
    \begin{align*}
      \Hom_\vcateg{D} (F (X), Y) & \to \Hom_\vcateg{C} (X, G (Y)),\\
      f & \mapsto G (f) \circ \eta_X
    \end{align*}
    es una biyección natural. En efecto, $G (-)$ es una biyección porque $G$ es
    fielmente pleno, y $-\circ \eta_X$ es una biyección porque $\eta_X$ es
    un isomorfismo. La naturalidad en $X$ e $Y$ es un ejercicio para el lector
    (use la funtorialidad de $G$ y naturalidad de $\eta_X$).
  \end{proof}
\end{observacion}

\begin{corolario}
  Una equivalencia de categorías $F\colon \vcateg{C} \to \vcateg{D}$ preserva
  todos los límites y colímites.
\end{corolario}

% % % % % % % % % % % % % % % % % % % % % % % % % % % % % %

\section{El teorema de los ceros como una equivalencia de categorías}

En esta sección me gustaría presentar un ejemplo detallado y no trivial de
equivalencia de categorías. Recordemos un poco de la geometría algebraica
elemental. Sea $k$ un cuerpo.

\begin{enumerate}
\item[1)] Para un ideal $\mathfrak{a} \subset k [X_1,\ldots,X_n]$
  el \term{conjunto algebraico afín} correspondiente es dado por los ceros
  comunes de los polinomios en $\mathfrak{a}$:
  $$V (\mathfrak{a}) \dfn \{ \underline{x} \in \AA^n (k) \mid f (\underline{x}) = 0 \text{ para todo } f\in \mathfrak{a} \} \subseteq \AA^n (k).$$

\item[2)] Para un subconjunto $X \subset \AA^n (k)$, se puede considerar el
  ideal de los polinomios que se anulan sobre $X$:
  $$I (X) \dfn \{ f\in k[X_1,\ldots,X_n] \mid f (\underline{x}) = 0 \text{ para todo }\underline{x}\in X \}.$$

\item[3)] El \term{teorema de los ceros de Hilbert} dice que cuando $k$ es un
  cuerpo algebraicamente cerrado, se cumple
  $$I (V (\mathfrak{a})) = \sqrt{\mathfrak{a}} \dfn \{ f\in k[X_1,\ldots,X_n] \mid f^n \in \mathfrak{a}\text{ para algún }n = 1,2,3,\ldots \}.$$
\end{enumerate}

A priori, un conjunto algebraico afín es nada más un subconjunto
$V\subset \AA^n (k)$. Para tener una estructura más interesante, podemos definir
qué es un morfismo entre conjuntos algebraicos afines. Primero, para un conjunto
algebraico afín $V\subset \AA^n (k)$ se puede considerar las
\term{funciones polinomiales} sobre $V$. Son los polinomios restringidos a
$V$. En este caso nos gustaría decir que $f = g$ si la diferencia $f-g$ es nula
sobre $V$. Esto explica la siguiente definición.

\begin{definicion}
  Para un conjunto algebraico afín $V \subseteq \AA^n (k)$ su \term{álgebra de
    funciones polinomiales} es dada por
  $$k [V] \dfn k [X_1,\ldots,X_n]/I (V).$$
\end{definicion}

Recordemos que un ideal $\mathfrak{a} \subseteq A$ es \term{radical} si
$\sqrt{\mathfrak{a}} = A$. Esto es equivalente a decir que el anillo cociente
$A/\mathfrak{a}$ es \term{reducido} (no tiene nilpotentes no nulos). El ideal
$I (V)$ es siempre radical, así que $k [V]$ es un anillo reducido.

\begin{definicion}
  Una \term{$k$-álgebra afín} $A$ es un anillo reducido de la forma
  $A = k [X_1,\ldots,X_n] / I$ para algún ideal $I \subset k [X_1,\ldots,X_n]$.
\end{definicion}

Ahora podemos considerar las aplicaciones entre conjuntos algebraicos afines
definidos por las funciones polinomiales.

\begin{definicion}
  Sean $V\subset \AA^n (k)$ y $W\subset \AA^m (k)$ dos conjuntos algebraicos
  afines. Una \term{aplicación polinomial} $\phi\colon V\to W$ es una aplicación
  que tiene coordenadas polinomiales; es decir,
  $$\phi (\underline{x}) = (\phi_1 (\underline{x}), \ldots, \phi_m (\underline{x})), \quad \text{donde }\phi_i\in k[V].$$
\end{definicion}

Una aplicación polinomial entre dos conjuntos algebraicos afines
$\phi\colon V\colon\to W$ donde $V \subset \AA^n (k)$ y $W \subset \AA^m (k)$
induce un homomorfismo de álgebras $k [W]\to k [V]$ que va
\emph{en la dirección opuesta}:
$$\phi\colon V\to W\quad\rightsquigarrow\quad\phi^*\colon k [W]\to k [V].$$
A saber, un elemento $f\in k[W]$ es un polinomio en $m$ variables
$f (Y_1,\ldots,Y_m)$ considerado módulo $I(W)$. Luego, la aplicación $\phi$ es
dada por $m$ polinomios
$$\phi_1 (X_1,\ldots,X_n), \ldots, \phi_m (X_1,\ldots,X_n) \in k[X_1,\ldots,X_n]/I(V).$$
Así que $f$ puede ser evaluado en $Y_i = \phi_i$. Esto nos da una aplicación
bien definida
\begin{align*}
  \phi^*\colon k[Y_1,\ldots,Y_m]/I(W) & \to k[X_1,\ldots,X_n]/I(V),\\
  f (Y_1,\ldots,Y_m) & \mapsto f (\phi_1 (X_1,\ldots,X_n), \ldots, \phi_m (X_1,\ldots,X_n)).
\end{align*}

\begin{observacion}
  Los conjuntos algebraicos afines y aplicaciones polinomiales forman una
  categoría $k\categ{-ConjAf}$. Las $k$-álgebras afines forman una categoría
  $k\categ{-ÁlgAf}$, que es una subcategoría plena de la categoría
  $k\categ{-Álg}$ de todas las $k$-álgebras. La formación del álgebra de
  funciones polinomiales es un funtor contravariante
  $$k\categ{-ConjAf}^\mathrm{op} \to k\categ{-ÁlgAf}.$$
  Este funtor es fielmente pleno.

  \begin{proof}
    Ejercicio para el lector.
  \end{proof}
\end{observacion}

\begin{ejemplo}
  Para un conjunto algebraico afín $V \subset \AA^n (k)$ especificar un punto
  $x_0\in V$ es lo mismo que especificar una aplicación
  $i\colon \AA^0 (k) \hookrightarrow V$ entre un punto $\AA^0 (k) = \{ \ast \}$
  y $V$ tal que $i (\ast) = x_0$. Estos morfismos son obviamente polinomiales y
  corresponden de modo biyectivo a homomorfismos de $k$-álgebras
  $k [V]\to k$. El núcleo de este homomorfismo es un ideal maximal
  $\mathfrak{m}\subset k[V]$.

  Si $k$ es algebraicamente cerrado (!), una de las versiones del teorema de los
  ceros nos dice que todos los ideales maximales de $k[V]$ surgen de esta
  manera, mas esto es falso si $k \ne \overline{k}$.
\end{ejemplo}

\begin{ejemplo}
  Consideremos el conjunto algebraico afín $V (Y - X^2) \subset \AA^2 (k)$.
  Esta es una parábola. La proyección al eje $x$ definida por
  $\phi\colon (x,y)\mapsto x$ es una aplicación polinomial $V\to \AA^1 (k)$.
  Se ve que esta posee una aplicación inversa $x\mapsto (x,x^2)$ que es también
  polinomial. Entonces, la parábola y la recta son isomorfas en la categoría de
  conjuntos algebraicos afines.

  \begin{center}
    \begin{tikzpicture}[x=0.5cm, y=0.5cm]
      \foreach \i [count=\j from 1] in {-5, ..., 100}
      \draw ({-2.5 + 0.05*\i}, {(-2.5 + 0.05*\i)^2}) -- ({-2.5 + 0.05*\j)}, {(-2.5 + 0.05*\j)^2});

      \draw[->] (-3,0) -- (3,0) node[right] {$x$};
      \draw[->] (0,-0.5) -- (0,8) node[above] {$y$};

      \foreach \i in {0, ..., 3}
      \draw[->] ({-2.5 + 0.5*\i}, {(-2.5 + 0.5*\i)^2}) -- ({-2.5 + 0.5*\i)}, 0);
      \foreach \i in {7, ..., 10}
      \draw[->] ({-2.5 + 0.5*\i}, {(-2.5 + 0.5*\i)^2}) -- ({-2.5 + 0.5*\i)}, 0);
    \end{tikzpicture}
  \end{center}

  Esto se traduce al isomorfismo de anillos
  $$k [X,Y] / (Y- X^2) \isom k [T].$$
  En efecto, la aplicación $t \mapsto (t,t^2)$ entre $\AA^1 (k)$ y $V$ induce un
  isomorfismo
  \begin{align*}
    k [X,Y] / (Y- X^2) & \to k [T],\\
    X & \mapsto T,\\
    Y & \mapsto T^2.
  \end{align*}
\end{ejemplo}

\begin{ejemplo}
  Consideremos el conjunto algebraico afín $V = V (Y^2 - X^3)$ en
  $\AA^2 (k)$. Es una curva cúbica con una cúspide en el punto $(0,0)$:

  \begin{center}
    \begin{tikzpicture}[x=0.75cm, y=0.75cm]
      \foreach \i [count=\j from 1] in {0, ..., 50}
      \draw ({0.05*\i}, {sqrt ((0.05*\i)^3)}) -- ({0.05*\j}, {sqrt ((0.05*\j)^3)});
      \foreach \i [count=\j from 1] in {0, ..., 50}
      \draw ({0.05*\i}, {-sqrt ((0.05*\i)^3)}) -- ({0.05*\j}, {-sqrt ((0.05*\j)^3)});

      \draw[->] (-1,0) -- (3,0) node[right] {$x$};
      \draw[->] (0,-5) -- (0,5) node[above] {$y$};
    \end{tikzpicture}
  \end{center}

  Podemos considerar una aplicación polinomial $\phi\colon \AA^1 (k) \to V$ por
  $t \mapsto (t^2,t^3)$. Esta induce un homomorfismo de $k$-álgebras
  correspondientes

  \begin{align*}
    \phi^*\colon k[X,Y] / (Y^2-X^3) & \to k [T],\\
    X & \mapsto T^2,\\
    Y & \mapsto T^3.
  \end{align*}

  Note que $\phi$ es una \emph{biyección de conjuntos}, pero no es un
  isomorfismo. Por ejemplo, se puede notar que la aplicación $\phi^*$ es
  inyectiva pero no es sobreyectiva: su imagen es el álgebra $k [T^2,T^3]$
  formada por los polinomios en $T^2$ y $T^3$.

  De hecho, las álgebras $k[X,Y] / (Y^2-X^2) \isom k [T^2,T^3]$ y $k [T]$ no son
  isomorfas y por lo tanto $V$ y $\AA^1 (k)$ no son isomorfos en la categoría
  de conjuntos algebraicos afines. Primero, notemos que gracias a la magnifica
  identidad
  $$T^3\cdot\frac{1}{T^2} = T$$
  ambas álgebras $k [T^2,T^3]$ y $k [T]$ tienen el mismo cuerpo de fracciones
  $k (T)$: el cuerpo de funciones racionales en una variable $T$.

  Recordemos que un dominio de integridad $A$ es \term{integralmente cerrado} si
  en su cuerpo de fracciones $K \dfn \Frac A$, si un elemento $x\in K$ satisface
  una ecuación polinomial mónica
  $$x^n + a_{n-1}\,x^{n-1} + \cdots + a_1\,x + a_0 = 0$$
  con $a_0,\ldots,a_{n-1} \in A$, entonces $x\in A$.

  El anillo $k [T]$ es integralmente cerrado, pero $k [T^2,T^3]$ no lo es:
  por ejemplo, $T^3/T^2$ es un elemento en su cuerpo de fracciones que satisface
  una ecuación mónica con coeficientes en $k[T^2,T^3]$:
  \[ (T^3/T^2)^2 - T^2 = 0. \qedhere \]
\end{ejemplo}

\begin{proposicion}
  Si $k$ un cuerpo algebraicamente cerrado, entonces el funtor
  $V \rightsquigarrow k[V]$ es esencialmente sobreyectivo.

\begin{proof}
  Sea $A$ una $k$-álgebra afín; es decir, $A = k [X_1,\ldots,X_n]/\mathfrak{a}$
  para algún ideal $\mathfrak{a}$. Puesto que $A$ es un anillo reducido,
  se tiene $\mathfrak{a} = \sqrt{\mathfrak{a}}$. Podemos considerar
  $V \dfn V (\mathfrak{a})$, y luego el teorema de los ceros nos dice que
  \[ k [V] \dfn k[X_1,\ldots,X_n]/I (V (\mathfrak{a})) =
     k[X_1,\ldots,X_n]/\sqrt{\mathfrak{a}} =
     k[X_1,\ldots,X_n]/\mathfrak{a} \isom A. \qedhere \]
 \end{proof}
\end{proposicion}

\begin{corolario}
  Si $k$ es un cuerpo algebraicamente cerrado, existe una equivalencia de
  categorías
  $$k\categ{-ConjAf}^\mathrm{op} \cequiv k\categ{-ÁlgAf}.$$
\end{corolario}

\subsection*{El punto de vista moderno}

Como acabamos de ver, un análisis de las soluciones de ecuaciones polinomiales
sobre un cuerpo algebraicamente cerrado $k$ nos lleva naturalmente a las
$k$-álgebras afines. Esta algebraización es un buen logro, y usando este punto
de vista, se puede desarrollar mucha teoría.

Sin embargo, a veces nos interesan cuerpos $k$ que no son algebraicamente
cerrados, o casos en los que $k$ se reemplaza por un anillo conmutativo
arbitrario. También en muchos problemas es útil variar $k$. No existe ningún
cuerpo que puede ser encajado en todos los cuerpos, así que, por ejemplo, un
conjunto algebraico sobre $\CC$ y un conjunto algebraico sobre
$\overline{\FF_p}$ viven en dos mundos diferentes. Es algo lamentable: a veces
la misma ecuación, por ejemplo
$$Y^2 = X^3 - X + 1$$
puede ser considerada sobre cualquier cuerpo.

Una solución extrema de todos de estos problemas es reemplazar las $k$-álgebras
afines con $k$ fijo por los anillos conmutativos arbitrarios (posiblemente con
nilpotentes). Esta idea fue sistemáticamente desarrollada por el matemático
francés \personality{Alexander Grothendieck} (1928--2014) con ayuda de su
colaborador \personality{Jean Dieudonné} (1906--1992) en el tratado
``Éléments de géométrie algébrique'' (1960--1967).

Las ideas de Grothendieck fueron revolucionarias en los años 50, pero ahora
están universalmente reconocidas.

Como hemos visto, los conjuntos algebraicos afines sobre $k = \overline{k}$ son
duales a las $k$-álgebras afines. Los objetos duales a los anillos conmutativos
se llaman \term{esquemas afines}.

\[ \begin{tikzcd}[row sep=0.1em, column sep=0.1em]
    \text{geometría} & & \text{álgebra} \\
    \\
    {\left\{\begin{array}{c}\text{conjuntos algebraicos afines sobre}\\\text{un cuerpo algebraicamente cerrado }k\end{array}\right\}} & \cequiv & \{k\text{-álgebras afines}\}^\circ\\
    \cap & & \cap \\
    \{\text{esquemas afines}\} & \cequiv & \{\text{anillos conmutativos}\}^\circ
  \end{tikzcd} \]

En este sentido, se puede pensar en un esquema afín nada más como en un anillo
conmutativo y morfismos entre esquemas afines como en homomorfismos de anillos
en la dirección opuesta. Obviamente, la verdadera definición de esquemas afines
es un poco más sofisticada. En general, un \term{esquema} es el resultado
de pegamiento de esquemas afines.

% % % % % % % % % % % % % % % % % % % % % % % % % % % % % %

\pagebreak

\appendix
\section{Algunos ejercicios}

% \subsection*{Funtores}
%
% \begin{ejerc}
% Para un conjunto $X$ sea
% $$P^+ (X) \dfn P^- (X) \dfn 2^X$$
% el conjunto de subconjuntos de $X$. Para una aplicación $f\colon X\to Y$
% definamos
% \begin{align*}
% P^+ (f)\colon 2^X & \to 2^Y,\\
% Z & \mapsto f (Z),
% \end{align*}
% y
% \begin{align*}
% P^- (f)\colon 2^Y & \to 2^X,\\
% Z & \mapsto f^{-1} (Z).
% \end{align*}
%
% Demuestre que $P^+$ y $P^-$ son funtores $\categ{Set} \to \categ{Set}$, uno
% covariante y el otro contravariante.
% \end{ejerc}

\subsection*{Iso-, epi-, mono-}

\begin{ejerc}
  Demuestre que si $f$ es un isomorfismo en $\vcateg{C}$ y $F$ es un funtor
  $\vcateg{C}\to \vcateg{D}$, entonces $F (f)$ es un isomorfismo en
  $\vcateg{D}$.
\end{ejerc}

\begin{ejerc}
  Demuestre que las composiciones de iso-, mono-, epimorfismos satisfacen las
  siguientes propiedades.

  \begin{enumerate}
  \item[1)] Si $f\colon X\to Y$ e $g\colon Y\to Z$ son isomorfismos, entonces
    $g\circ f\colon X\to Z$ es un isomorfismo.

  \item[2)] Si $m\colon X\mono Y$ y $m'\colon Y\mono Z$ son monomorfismos,
    entonces $m'\circ m\colon X\to Z$ es un monomorfismo.

  \item[3)] Si $e\colon X\epi Y$ y $e'\colon Y\epi Z$ son epimorfismos, entonces
    $e'\circ e\colon X\to Z$ es un epimorfismo.

  \item[4)] Si para $m\colon X\to Y$, $f\colon Y\to Z$ la composición $f\circ m$
    es un monomorfismo, entonces $m$ es un monomorfismo.

  \item[5)] Si para $f\colon X\to Y$, $e\colon Y\to Z$ la composición $e\circ f$
    es un epimorfismo, entonces $e$ es un epimorfismo.
  \end{enumerate}
\end{ejerc}

\begin{ejerc}
  Demuestre que en la categoría $k\categ{-Vect}$ los isomorfismos,
  monomorfismos, epimorfismos son las aplicaciones $k$-lineales biyectivas,
  inyectivas, sobreyectivas respectivamente.
\end{ejerc}

\subsection*{Lema de Yoneda}

\begin{ejerc}
  Demuestre con todos los detalles la versión covariante del lema de Yoneda.
\end{ejerc}

\begin{ejerc}
  Sea $G$ un grupo. Consideremos $G$ como una categoría. Note que un funtor
  $F\colon G\to \categ{Set}$ corresponde a un $G$-conjunto y una transformación
  natural entre tales funtores es una aplicación $G$-equivariante. ¿Qué es un
  funtor representable en este caso? ¿Qué significa el encajamiento de Yoneda?
\end{ejerc}

\begin{ejerc}
  Sea $R$ un anillo.

  \begin{enumerate}
  \item[a)] Demuestre que el funtor olvidadizo $R\categ{-Alg} \to \categ{Set}$
    es representable.

  \item[b)] Supongamos que para cada $R$-álgebra $A$ está especificada una
    aplicación entre conjuntos $\alpha_A\colon A \to A$ de tal manera que para
    todo homomorfismo de $R$-álgebras $\phi\colon A \to B$ se cumple
    $\phi \circ \alpha_A = \alpha_B \circ \phi$.
    \[ \begin{tikzcd}
        A \ar{r}{\alpha_A}\ar{d}[swap]{\phi} & A\ar{d}{\phi} \\
        B \ar{r}{\alpha_B} & B
      \end{tikzcd} \]

    Usando el lema de Yoneda, demuestre que existe un polinomio $f \in R[x]$ tal
    que para toda $R$-álgebra $A$ se tiene
    $$\alpha_A\colon a \mapsto f (a).$$

  \item[c)] Demuestre lo mismo sin recurrir a Yoneda.
  \end{enumerate}
\end{ejerc}

\subsection*{Límites y colímites}

\begin{ejerc}
  Consideremos la categoría $\categ{Top}_*$ cuyos objetos $(X,x_0)$ son espacios
  topológicos con un punto marcado $x_0\in X$ y cuyos morfismos
  $f\colon (X,x_0)\to (Y,y_0)$ son aplicaciones continuas $f\colon X\to Y$ tales
  que $f (x_0) = y_0$. Describa objetos terminales e iniciales, productos y
  coproductos en $\categ{Top}_*$.
\end{ejerc}

\begin{ejerc}
  Describa objetos terminales e iniciales, productos y coproductos en la
  categoría $\categ{Cat}$ de categorías pequeñas.
\end{ejerc}

\begin{ejerc}
  Para un grupo fijo $G$, consideremos la categoría $G\categ{-Set}$ cuyos
  objetos son $G$-conjuntos (conjuntos con acción de $G$) y cuyos morfismos
  $f\colon X\to Y$ son aplicaciones $G$-equivariantes (que satisfacen la
  condición $f (g\cdot x) = g\cdot f (x)$ para cualesauiera $g\in G$ y
  $x\in X$). Describa objetos terminales e iniciales, productos y coproductos en
  $G\categ{-Set}$.
\end{ejerc}

\begin{ejerc}
  Para una categoría pequeña sea $\widehat{\vcateg{C}}$ la categoría de funtores
  $F\colon \vcateg{C}^\mathrm{op} \to \categ{Set}$. Describa los objetos
  terminales e iniciales, productos y coproductos en $\widehat{\vcateg{C}}$.
\end{ejerc}

\begin{ejerc}
  Demuestre que los productos fibrados son funtoriales en el siguiente sentido:
  un diagrama conmutativo
  \[ \begin{tikzcd}
      X_1\ar{dr}\ar[d] & & Y_1\ar{dl}\ar{d} \\
      X_2\ar{dr} & Z_1\ar{d} & Y_2\ar{dl} \\
      & Z_2
    \end{tikzcd} \]
  induce un morfismo canónico
  $X_1\times_{Z_1} Y_1 \to X_2 \times_{Z_2} Y_2$.
\end{ejerc}

\begin{ejerc}
  Demuestre que en la categoría $k\categ{-Vect}$ se tiene
  $$\eq (f,g) = \ker (f-g)\quad\text{y}\quad\coeq (f,g) = \coker (f-g).$$
\end{ejerc}

\begin{ejerc}
  Demuestre que los productos fibrados preservan isomorfismos: si la flecha
  $Y\to Z$ es un isomorfismo, entonces $X\times_Z Y\to X$ es también un
  isomorfismo:

  \[ \begin{tikzpicture}
      \matrix(m)[matrix of math nodes, row sep=2em, column sep=2em,
      text height=1.5ex, text depth=0.25ex]{
        X\times_Z Y & Y \\
        X & Z \\};
      \path[->] (m-1-1) edge (m-1-2);
      \path[->] (m-2-1) edge (m-2-2);
      \path[->,font=\scriptsize] (m-1-1) edge node[left] {$\isom$} (m-2-1);
      \path[->,font=\scriptsize] (m-1-2) edge node[right] {$\isom$} (m-2-2);

      \begin{scope}[shift=($(m-1-1)!.4!(m-2-2)$)]
        \draw +(-.2,0) -- +(0,0)  -- +(0,.2);
      \end{scope}
    \end{tikzpicture} \]
\end{ejerc}

\begin{ejerc}
  Demuestre que los productos fibrados

  \[ \begin{tikzpicture}
      \matrix(m)[matrix of math nodes, row sep=2em, column sep=2em,
      text height=1.5ex, text depth=0.25ex]{
        E & X \\
        X & X\times Y\\};
      \path[->,font=\scriptsize] (m-1-1) edge node[above] {$e'$} (m-1-2);
      \path[->,font=\scriptsize] (m-2-1) edge node[below] {$\tcol{\idid}{f}$} (m-2-2);
      \path[->,font=\scriptsize] (m-1-1) edge node[left] {$e$} (m-2-1);
      \path[->,font=\scriptsize] (m-1-2) edge node[right] {$\tcol{\idid}{g}$} (m-2-2);

      \begin{scope}[shift=($(m-1-1)!.4!(m-2-2)$)]
        \draw +(-.2,0) -- +(0,0)  -- +(0,.2);
      \end{scope}
    \end{tikzpicture} \quad\quad \begin{tikzpicture}
      \matrix(m)[matrix of math nodes, row sep=2em, column sep=2em,
      text height=1.5ex, text depth=0.25ex]{
        E & X \\
        X & Y\times Y\\};
      \path[->,font=\scriptsize] (m-1-1) edge node[above] {$e$} (m-1-2);
      \path[->,font=\scriptsize] (m-2-1) edge node[below] {$\tcol{\idid}{\idid}$} (m-2-2);
      \path[->,font=\scriptsize] (m-1-1) edge node[left] {$e'$} (m-2-1);
      \path[->,font=\scriptsize] (m-1-2) edge node[right] {$\tcol{f}{g}$} (m-2-2);

      \begin{scope}[shift=($(m-1-1)!.4!(m-2-2)$)]
        \draw +(-.2,0) -- +(0,0)  -- +(0,.2);
      \end{scope}
    \end{tikzpicture} \]
  calculan el ecualizador de $f,g\colon X\to Y$. Formule y demuestre la
  propiedad dual para coecualizadores y coproductos fibrados.
\end{ejerc}

\subsection*{Transformaciones naturales}

\begin{ejerc}
  Sean $\vcateg{C},\vcateg{D},\vcateg{E}$ tres categorías. Sean $F,G,H$ funtores
  $\vcateg{C} \to \vcateg{D}$ y sean $I,J,K$ tres funtores
  $\vcateg{D}\to \vcateg{E}$. Consideremos transformaciones naturales
  $$\alpha\colon F\Rightarrow G, \quad \beta\colon G\Rightarrow H, \quad
    \sigma\colon I\Rightarrow J, \quad \tau\colon J\Rightarrow K.$$

  \[ \begin{tikzpicture} \matrix[matrix of math nodes,column sep=3cm,row sep=5em] (m) {
        \vcateg{C} & \vcateg{D} & \vcateg{E} \\
      };
      \draw[->] (m-1-1) to[bend left=50]
      node[label=above:$F$] (F) {} (m-1-2); \draw[] (m-1-1) to node (G) {}
      (m-1-2); \draw[->] (m-1-1) to
      node[label=above:$G$,near end] {} (m-1-2); \draw[->] (m-1-1) to[bend right=50]
      node[label=below:$H$] (H) {} (m-1-2); \draw[->] (m-1-2) to[bend left=50]
      node[label=above:$I$] (I) {} (m-1-3); \draw[->] (m-1-2) to
      node[label=above:$J$,near end] {} (m-1-3); \draw[] (m-1-2) to node (J) {}
      (m-1-3); \draw[->] (m-1-2) to[bend right=50]
      node[label=below:$K$] (K) {} (m-1-3);

      \draw[double,double equal sign distance,-implies] (F) -- node[label=left:$\alpha$] {} (G);
      \draw[double,double equal sign distance,-implies] (G) -- node[label=left:$\beta$] {} (H);
      \draw[double,double equal sign distance,-implies] (I) -- node[label=left:$\sigma$] {} (J);
      \draw[double,double equal sign distance,-implies] (J) -- node[label=left:$\tau$] {} (K);

    \end{tikzpicture} \]

  Demuestre que
  $$(\tau\circ \sigma)\ast (\beta\circ \alpha) =
    (\tau\ast\beta)\circ (\sigma\ast\alpha),$$
    donde $\ast$ denota el producto de Godement.
  \end{ejerc}

\begin{ejerc}
  Demuestre que el producto de Godement es asociativo: para un diagrama
  \[ \begin{tikzpicture}
      \matrix[matrix of math nodes,column sep=2cm] (m)
      {
        \vcateg{A} & \vcateg{B} & \vcateg{C} & \vcateg{D} \\
      };
      \draw[->] (m-1-1) to[bend left=50] node[label=above:$F_1$] (F1) {} (m-1-2);
      \draw[->] (m-1-1) to[bend right=50] node[label=below:$G_1$] (G1) {} (m-1-2);
      \draw[->] (m-1-2) to[bend left=50] node[label=above:$F_2$] (F2) {} (m-1-3);
      \draw[->] (m-1-2) to[bend right=50] node[label=below:$G_2$] (G2) {} (m-1-3);
      \draw[->] (m-1-3) to[bend left=50] node[label=above:$F_3$] (F3) {} (m-1-4);
      \draw[->] (m-1-3) to[bend right=50] node[label=below:$G_3$] (G3) {} (m-1-4);

      \draw[double,double equal sign distance,-implies,shorten >=10pt,shorten <=10pt] 
      (F1) -- node[label=right:$\alpha$] {} (G1);
      \draw[double,double equal sign distance,-implies,shorten >=10pt,shorten <=10pt] 
      (F2) -- node[label=right:$\beta$] {} (G2);
      \draw[double,double equal sign distance,-implies,shorten >=10pt,shorten <=10pt] 
      (F3) -- node[label=right:$\gamma$] {} (G3);
    \end{tikzpicture} \]
  se cumple
  $$(\gamma\ast\beta)\ast\alpha = \gamma\ast(\beta\ast\alpha).$$
\end{ejerc}

\begin{ejerc}
  Sea $\vcateg{I}$ una categoría pequeña. Consideremos el funtor
  \begin{align*}
    \Delta\colon \vcateg{C} & \to \Fun (\vcateg{I}, \vcateg{C}),\\
    X & \rightsquigarrow \Delta_X
  \end{align*}
  que a cada objeto $X\in \Ob (\vcateg{C})$ asocia el funtor constante
  $\Delta_X \colon \vcateg{I} \to \vcateg{C}$ (tal que $\Delta_X (i) = X$ para
  cada $i\in \Ob (\vcateg{I})$). Demuestre que para todo funtor
  $F\colon\vcateg{I}\to\vcateg{C}$ hay biyecciones naturales
  \begin{align*}
    \Nat (\Delta_X, F) & \isom \Hom_\vcateg{C} (X, \lim_\vcateg{I} F),\\
    \Nat (F, \Delta_X) & \isom \Hom_\vcateg{C} (\colim_\vcateg{I} F, X).
  \end{align*}
\end{ejerc}

\subsection*{Adjunciones}

\begin{ejerc}
  Sea $\categ{Ring}_1$ la categoría de anillos con identidad donde los morfismos
  son los homomorfismos $f\colon R\to S$ que satisfacen $f (1_R) \to 1_S$ y
  $\categ{Ring}$ la categoría de anillos que no necesariamente tienen identidad.

  Para un anillo $R$ consideremos el conjunto $\widehat{R} \dfn \ZZ\times R$ con
  la multiplicación
  $$(n_1, r_1)\cdot (n_2, r_2) \dfn (n_1 n_2, n_1 r_1 + n_2 r_1 + r_1 r_2).$$
  Note que es un anillo con identidad $(1,0)$. Demuestre que
  $R \rightsquigarrow \widehat{R}$ es un funtor
  $\categ{Ring} \to \categ{Ring}_1$ y es adjunto por la izquierda a la inclusión
  $\categ{Ring}_1 \hookrightarrow \categ{Ring}$.
\end{ejerc}

\begin{ejerc}
  Digamos que en una categoría $\vcateg{C}$ dos objetos $X$ e $Y$ están en la
  misma componente conexa si existe una cadena de morfismos de $X$ a $Y$, que no
  necesariamente van en la misma dirección, por ejemplo
  $$X \rightarrow \bullet \rightarrow \bullet \leftarrow
    \bullet \rightarrow \bullet \rightarrow \bullet \leftarrow Y$$
  Para una categoría pequeña $\vcateg{C}$ sea $\pi_0 (\vcateg{C})$ el conjunto
  de sus componentes conexas. Demuestre que $\pi_0$ es un funtor
  $\categ{Cat} \to \categ{Set}$. Demuestre que es adjunto por la izquierda al
  funtor $\categ{Set} \to \categ{Cat}$ que a cada conjunto $X$ asocia la
  categoría donde los objetos son los elementos de $X$ y los únicos morfismos
  son los morfismos identidad.
\end{ejerc}

\begin{ejerc}
  Sean $X$ e $Y$ dos conjuntos. Consideremos $2^X$ y $2^Y$ como conjuntos
  parcialmente ordenados por la relación $\subseteq$, y en particular como
  categorías.

  Sea $f\colon X\to Y$ una aplicación. Para $A\in 2^X$ definamos
  \begin{align*}
    f_* (A) & \dfn \{ y\in Y \mid f^{-1} (y) \subseteq A \},\\
    \im (A) & \dfn \{ f (x) \mid x\in A \},
  \end{align*}
  y para $B\in 2^Y$ definamos
  $$f^{-1} (B) \dfn \{ x\in X \mid f (x) \in B \}.$$
  Demuestre que $f_*$ e $\im$ son funtores $2^X\to 2^Y$ y $f^{-1}$ es un funtor
  $2^Y \to 2^X$. Demuestre que

  \begin{enumerate}
  \item[1)] $\im$ es adjunto por la izquierda a $f^{-1}$,

  \item[2)] $f^{-1}$ es adjunto por la izquierda a $f_*$.
  \end{enumerate}

  ¿Qué significa en este caso la preservación de objetos iniciales y coproductos
  (resp. objetos terminales y productos) por adjunto por la izquierda
  (resp. adjunto por la derecha)?
\end{ejerc}

\subsection*{Equivalencias de categorías}

\begin{ejerc}
  Demuestre que si $F\colon \vcateg{C}\to \vcateg{D}$ es una equivalencia de
  categorías, entonces $F$ envía un objeto terminal (resp. inicial) de
  $\vcateg{C}$ en un objeto terminal (resp. inicial) de $\vcateg{D}$.

  Supongamos que existe una equivalencia de categorías
  $F\colon \categ{Set}\to \categ{Set}^\mathrm{op}$. Note que en este caso para
  cada conjunto $X$ tendríamos
  $$X \isom \Hom_\categ{Set} (\{ \ast \},X) \isom
            \Hom_\categ{Set} (F (X), \emptyset).$$
  Concluya que las categorías $\categ{Set}$ y $\categ{Set}^\mathrm{op}$ no son
  equivalentes.
\end{ejerc}

% % % % % % % % % % % % % % % % % % % % % % % % % % % % % %

\pagebreak
\bibliographystyle{../amsalpha-cust}
{\small\bibliography{../salvador}}

\end{document}
