\documentclass{article}

% TODO : CLEAN UP THIS MESS
% (AND MAKE SURE ALL TEXTS STILL COMPILE)
\usepackage[leqno]{amsmath}
\usepackage{amssymb}
\usepackage{graphicx}

\usepackage{diagbox} % table heads with diagonal lines
\usepackage{relsize}

\usepackage{wasysym}
\usepackage{scrextend}
\usepackage{epigraph}
\setlength\epigraphwidth{.6\textwidth}

\usepackage[utf8]{inputenc}

\usepackage{titlesec}
\titleformat{\chapter}[display]
  {\normalfont\sffamily\huge\bfseries}
  {\chaptertitlename\ \thechapter}{5pt}{\Huge}
\titleformat{\section}
  {\normalfont\sffamily\Large\bfseries}
  {\thesection}{1em}{}
\titleformat{\subsection}
  {\normalfont\sffamily\large\bfseries}
  {\thesubsection}{1em}{}
\titleformat{\part}[display]
  {\normalfont\sffamily\huge\bfseries}
  {\partname\ \thepart}{0pt}{\Huge}

\usepackage[T1]{fontenc}
\usepackage{fourier}
\usepackage{paratype}

\usepackage[symbol,perpage]{footmisc}

\usepackage{perpage}
\MakePerPage{footnote}

\usepackage{array}
\newcolumntype{x}[1]{>{\centering\hspace{0pt}}p{#1}}

% TODO: the following line causes conflict with new texlive (!)
% \usepackage[english,russian,polutonikogreek,spanish]{babel}
% \newcommand{\russian}[1]{{\selectlanguage{russian}#1}}

% Remove conflicting options for the moment:
\usepackage[english,polutonikogreek,spanish]{babel}

\AtBeginDocument{\shorthandoff{"}}
\newcommand{\greek}[1]{{\selectlanguage{polutonikogreek}#1}}

% % % % % % % % % % % % % % % % % % % % % % % % % % % % % %
% Limit/colimit symbols (with accented i: lím / colím)

\usepackage{etoolbox} % \patchcmd

\makeatletter
\patchcmd{\varlim@}{lim}{\lim}{}{}
\makeatother
\DeclareMathOperator*{\colim}{co{\lim}}
\newcommand{\dirlim}{\varinjlim}
\newcommand{\invlim}{\varprojlim}

% % % % % % % % % % % % % % % % % % % % % % % % % % % % % %

\usepackage[all,color]{xy}

\usepackage{pigpen}
\newcommand{\po}{\ar@{}[dr]|(.4){\text{\pigpenfont I}}}
\newcommand{\pb}{\ar@{}[dr]|(.3){\text{\pigpenfont A}}}
\newcommand{\polr}{\ar@{}[dr]|(.65){\text{\pigpenfont A}}}
\newcommand{\pour}{\ar@{}[ur]|(.65){\text{\pigpenfont G}}}
\newcommand{\hstar}{\mathop{\bigstar}}

\newcommand{\bigast}{\mathop{\Huge \mathlarger{\mathlarger{\ast}}}}

\newcommand{\term}{\textbf}

\usepackage{stmaryrd}

\usepackage{cancel}

\usepackage{tikzsymbols}

\newcommand{\open}{\underset{\mathrm{open}}{\hookrightarrow}}
\newcommand{\closed}{\underset{\mathrm{closed}}{\hookrightarrow}}

\newcommand{\tcol}[2]{{#1 \choose #2}}

\newcommand{\homot}{\simeq}
\newcommand{\isom}{\cong}
\newcommand{\cH}{\mathcal{H}}
\renewcommand{\hom}{\mathrm{hom}}
\renewcommand{\div}{\mathop{\mathrm{div}}}
\renewcommand{\Im}{\mathop{\mathrm{Im}}}
\renewcommand{\Re}{\mathop{\mathrm{Re}}}
\newcommand{\id}[1]{\mathrm{id}_{#1}}
\newcommand{\idid}{\mathrm{id}}

\newcommand{\ZG}{{\ZZ G}}
\newcommand{\ZH}{{\ZZ H}}

\newcommand{\quiso}{\simeq}

\newcommand{\personality}[1]{{\sc #1}}

\newcommand{\mono}{\rightarrowtail}
\newcommand{\epi}{\twoheadrightarrow}
\newcommand{\xepi}[1]{\xrightarrow{#1}\mathrel{\mkern-14mu}\rightarrow}

% % % % % % % % % % % % % % % % % % % % % % % % % % % % % %

\DeclareMathOperator{\Ad}{Ad}
\DeclareMathOperator{\Aff}{Aff}
\DeclareMathOperator{\Ann}{Ann}
\DeclareMathOperator{\Aut}{Aut}
\DeclareMathOperator{\Br}{Br}
\DeclareMathOperator{\CH}{CH}
\DeclareMathOperator{\Cl}{Cl}
\DeclareMathOperator{\Coeq}{Coeq}
\DeclareMathOperator{\Coind}{Coind}
\DeclareMathOperator{\Cop}{Cop}
\DeclareMathOperator{\Corr}{Corr}
\DeclareMathOperator{\Cor}{Cor}
\DeclareMathOperator{\Cov}{Cov}
\DeclareMathOperator{\Der}{Der}
\DeclareMathOperator{\Div}{Div}
\DeclareMathOperator{\D}{D}
\DeclareMathOperator{\Ehr}{Ehr}
\DeclareMathOperator{\End}{End}
\DeclareMathOperator{\Eq}{Eq}
\DeclareMathOperator{\Ext}{Ext}
\DeclareMathOperator{\Frac}{Frac}
\DeclareMathOperator{\Frob}{Frob}
\DeclareMathOperator{\Funct}{Funct}
\DeclareMathOperator{\Fun}{Fun}
\DeclareMathOperator{\GL}{GL}
\DeclareMathOperator{\Gal}{Gal}
\DeclareMathOperator{\Gr}{Gr}
\DeclareMathOperator{\Hol}{Hol}
\DeclareMathOperator{\Hom}{Hom}
\DeclareMathOperator{\Ho}{Ho}
\DeclareMathOperator{\Id}{Id}
\DeclareMathOperator{\Ind}{Ind}
\DeclareMathOperator{\Inn}{Inn}
\DeclareMathOperator{\Isom}{Isom}
\DeclareMathOperator{\Ker}{Ker}
\DeclareMathOperator{\Lan}{Lan}
\DeclareMathOperator{\Lie}{Lie}
\DeclareMathOperator{\Map}{Map}
\DeclareMathOperator{\Mat}{Mat}
\DeclareMathOperator{\Max}{Max}
\DeclareMathOperator{\Mor}{Mor}
\DeclareMathOperator{\Nat}{Nat}
\DeclareMathOperator{\Nrd}{Nrd}
\DeclareMathOperator{\Ob}{Ob}
\DeclareMathOperator{\Out}{Out}
\DeclareMathOperator{\PGL}{PGL}
\DeclareMathOperator{\PSL}{PSL}
\DeclareMathOperator{\PSU}{PSU}
\DeclareMathOperator{\Pic}{Pic}
\DeclareMathOperator{\RHom}{RHom}
\DeclareMathOperator{\Rad}{Rad}
\DeclareMathOperator{\Ran}{Ran}
\DeclareMathOperator{\Rep}{Rep}
\DeclareMathOperator{\Res}{Res}
\DeclareMathOperator{\SL}{SL}
\DeclareMathOperator{\SO}{SO}
\DeclareMathOperator{\SU}{SU}
\DeclareMathOperator{\Sh}{Sh}
\DeclareMathOperator{\Sing}{Sing}
\DeclareMathOperator{\Specm}{Specm}
\DeclareMathOperator{\Spec}{Spec}
\DeclareMathOperator{\Sp}{Sp}
\DeclareMathOperator{\Stab}{Stab}
\DeclareMathOperator{\Sym}{Sym}
\DeclareMathOperator{\Tors}{Tors}
\DeclareMathOperator{\Tor}{Tor}
\DeclareMathOperator{\Tot}{Tot}
\DeclareMathOperator{\UUU}{U}

\DeclareMathOperator{\adj}{adj}
\DeclareMathOperator{\ad}{ad}
\DeclareMathOperator{\af}{af}
\DeclareMathOperator{\card}{card}
\DeclareMathOperator{\cm}{cm}
\DeclareMathOperator{\codim}{codim}
\DeclareMathOperator{\cod}{cod}
\DeclareMathOperator{\coeq}{coeq}
\DeclareMathOperator{\coim}{coim}
\DeclareMathOperator{\coker}{coker}
\DeclareMathOperator{\cont}{cont}
\DeclareMathOperator{\conv}{conv}
\DeclareMathOperator{\cor}{cor}
\DeclareMathOperator{\depth}{depth}
\DeclareMathOperator{\diag}{diag}
\DeclareMathOperator{\diam}{diam}
\DeclareMathOperator{\dist}{dist}
\DeclareMathOperator{\dom}{dom}
\DeclareMathOperator{\eq}{eq}
\DeclareMathOperator{\ev}{ev}
\DeclareMathOperator{\ex}{ex}
\DeclareMathOperator{\fchar}{char}
\DeclareMathOperator{\fr}{fr}
\DeclareMathOperator{\gr}{gr}
\DeclareMathOperator{\im}{im}
\DeclareMathOperator{\infl}{inf}
\DeclareMathOperator{\interior}{int}
\DeclareMathOperator{\intrel}{intrel}
\DeclareMathOperator{\inv}{inv}
\DeclareMathOperator{\length}{length}
\DeclareMathOperator{\mcd}{mcd}
\DeclareMathOperator{\mcm}{mcm}
\DeclareMathOperator{\multideg}{multideg}
\DeclareMathOperator{\ord}{ord}
\DeclareMathOperator{\pr}{pr}
\DeclareMathOperator{\rel}{rel}
\DeclareMathOperator{\res}{res}
\DeclareMathOperator{\rkred}{rkred}
\DeclareMathOperator{\rkss}{rkss}
\DeclareMathOperator{\rk}{rk}
\DeclareMathOperator{\sgn}{sgn}
\DeclareMathOperator{\sk}{sk}
\DeclareMathOperator{\supp}{supp}
\DeclareMathOperator{\trdeg}{trdeg}
\DeclareMathOperator{\tr}{tr}
\DeclareMathOperator{\vol}{vol}

\newcommand{\iHom}{\underline{\Hom}}

\renewcommand{\AA}{\mathbb{A}}
\newcommand{\CC}{\mathbb{C}}
\renewcommand{\SS}{\mathbb{S}}
\newcommand{\TT}{\mathbb{T}}
\newcommand{\PP}{\mathbb{P}}
\newcommand{\BB}{\mathbb{B}}
\newcommand{\RR}{\mathbb{R}}
\newcommand{\ZZ}{\mathbb{Z}}
\newcommand{\FF}{\mathbb{F}}
\newcommand{\HH}{\mathbb{H}}
\newcommand{\NN}{\mathbb{N}}
\newcommand{\QQ}{\mathbb{Q}}
\newcommand{\KK}{\mathbb{K}}

% % % % % % % % % % % % % % % % % % % % % % % % % % % % % %

\usepackage{amsthm}

\newcommand{\legendre}[2]{\left(\frac{#1}{#2}\right)}

\newcommand{\examplesymbol}{$\blacktriangle$}
\renewcommand{\qedsymbol}{$\blacksquare$}

\newcommand{\dfn}{\mathrel{\mathop:}=}
\newcommand{\rdfn}{=\mathrel{\mathop:}}

\usepackage{xcolor}
\definecolor{mylinkcolor}{rgb}{0.0,0.4,1.0}
\definecolor{mycitecolor}{rgb}{0.0,0.4,1.0}
\definecolor{shadecolor}{rgb}{0.79,0.78,0.65}
\definecolor{gray}{rgb}{0.6,0.6,0.6}

\usepackage{colortbl}

\definecolor{myred}{rgb}{0.7,0.0,0.0}
\definecolor{mygreen}{rgb}{0.0,0.7,0.0}
\definecolor{myblue}{rgb}{0.0,0.0,0.7}

\definecolor{redshade}{rgb}{0.9,0.5,0.5}
\definecolor{greenshade}{rgb}{0.5,0.9,0.5}

\usepackage[unicode,colorlinks=true,linkcolor=mylinkcolor,citecolor=mycitecolor]{hyperref}
\newcommand{\refref}[2]{\hyperref[#2]{#1~\ref*{#2}}}
\newcommand{\eqnref}[1]{\hyperref[#1]{(\ref*{#1})}}

\newcommand{\tos}{\!\!\to\!\!}

\usepackage{framed}

\newcommand{\cequiv}{\simeq}

\makeatletter
\newcommand\xleftrightarrow[2][]{%
  \ext@arrow 9999{\longleftrightarrowfill@}{#1}{#2}}
\newcommand\longleftrightarrowfill@{%
  \arrowfill@\leftarrow\relbar\rightarrow}
\makeatother

\newcommand{\bsquare}{\textrm{\ding{114}}}

% % % % % % % % % % % % % % % % % % % % % % % % % % % % % %

\newtheoremstyle{myplain}
  {\topsep}   % ABOVESPACE
  {\topsep}   % BELOWSPACE
  {\itshape}  % BODYFONT
  {0pt}       % INDENT (empty value is the same as 0pt)
  {\bfseries} % HEADFONT
  {.}         % HEADPUNCT
  {5pt plus 1pt minus 1pt} % HEADSPACE
  {\thmnumber{#2}. \thmname{#1}\thmnote{ (#3)}}   % CUSTOM-HEAD-SPEC

\newtheoremstyle{myplainnameless}
  {\topsep}   % ABOVESPACE
  {\topsep}   % BELOWSPACE
  {\normalfont}  % BODYFONT
  {0pt}       % INDENT (empty value is the same as 0pt)
  {\bfseries} % HEADFONT
  {.}         % HEADPUNCT
  {5pt plus 1pt minus 1pt} % HEADSPACE
  {\thmnumber{#2}}   % CUSTOM-HEAD-SPEC 

\newtheoremstyle{sectionexercise}
  {\topsep}   % ABOVESPACE
  {\topsep}   % BELOWSPACE
  {\normalfont}  % BODYFONT
  {0pt}       % INDENT (empty value is the same as 0pt)
  {\bfseries} % HEADFONT
  {.}         % HEADPUNCT
  {5pt plus 1pt minus 1pt} % HEADSPACE
  {Ejercicio \thmnumber{#2}\thmnote{ (#3)}}   % CUSTOM-HEAD-SPEC

\newtheoremstyle{mydefinition}
  {\topsep}   % ABOVESPACE
  {\topsep}   % BELOWSPACE
  {\normalfont}  % BODYFONT
  {0pt}       % INDENT (empty value is the same as 0pt)
  {\bfseries} % HEADFONT
  {.}         % HEADPUNCT
  {5pt plus 1pt minus 1pt} % HEADSPACE
  {\thmnumber{#2}. \thmname{#1}\thmnote{ (#3)}}   % CUSTOM-HEAD-SPEC

% EN ESPAÑOL

\newtheorem*{hecho*}{Hecho}
\newtheorem*{corolario*}{Corolario}
\newtheorem*{teorema*}{Teorema}
\newtheorem*{conjetura*}{Conjetura}
\newtheorem*{proyecto*}{Proyecto}
\newtheorem*{observacion*}{Observación}

\newtheorem*{lema*}{Lema}
\newtheorem*{resultado-clave*}{Resultado clave}
\newtheorem*{proposicion*}{Proposición}

\theoremstyle{definition}
\newtheorem*{ejercicio*}{Ejercicio}
\newtheorem*{definicion*}{Definición}
\newtheorem*{comentario*}{Comentario}
\newtheorem*{definicion-alternativa*}{Definición alternativa}
\newtheorem*{ejemploxs}{Ejemplo}
\newenvironment{ejemplo*}
  {\pushQED{\qed}\renewcommand{\qedsymbol}{\examplesymbol}\ejemploxs}
  {\popQED\endejemploxs}

\theoremstyle{myplain}
\newtheorem{proposicion}{Proposición}[section]

\newtheorem{proyecto}[proposicion]{Proyecto}
\newtheorem{teorema}[proposicion]{Teorema}
\newtheorem{corolario}[proposicion]{Corolario}
\newtheorem{hecho}[proposicion]{Hecho}
\newtheorem{lema}[proposicion]{Lema}

\newtheorem{observacion}[proposicion]{Observación}

\newenvironment{observacionejerc}
    {\pushQED{\qed}\renewcommand{\qedsymbol}{$\square$}\csname inner@observacionejerc\endcsname}
    {\popQED\csname endinner@observacionejerc\endcsname}
\newtheorem{inner@observacionejerc}[proposicion]{Observación}

\newenvironment{proposicionejerc}
    {\pushQED{\qed}\renewcommand{\qedsymbol}{$\square$}\csname inner@proposicionejerc\endcsname}
    {\popQED\csname endinner@proposicionejerc\endcsname}
\newtheorem{inner@proposicionejerc}[proposicion]{Proposicion}

\newenvironment{lemaejerc}
    {\pushQED{\qed}\renewcommand{\qedsymbol}{$\square$}\csname inner@lemaejerc\endcsname}
    {\popQED\csname endinner@lemaejerc\endcsname}
\newtheorem{inner@lemaejerc}[proposicion]{Lema}

\newtheorem{calculo}[proposicion]{Cálculo}

\theoremstyle{myplainnameless}
\newtheorem{nameless}[proposicion]{}

\theoremstyle{mydefinition}
\newtheorem{comentario}[proposicion]{Comentario}
\newtheorem{comentarioast}[proposicion]{Comentario ($\clubsuit$)}
\newtheorem{construccion}[proposicion]{Construcción}
\newtheorem{aplicacion}[proposicion]{Aplicación}
\newtheorem{definicion}[proposicion]{Definición}
\newtheorem{definicion-alternativa}[proposicion]{Definición alternativa}
\newtheorem{notacion}[proposicion]{Notación}
\newtheorem{advertencia}[proposicion]{Advertencia}
\newtheorem{digresion}[proposicion]{Digresión}
\newtheorem{ejemplox}[proposicion]{Ejemplo}
\newenvironment{ejemplo}
  {\pushQED{\qed}\renewcommand{\qedsymbol}{\examplesymbol}\ejemplox}
  {\popQED\endejemplox}
\newtheorem{contraejemplox}[proposicion]{Contraejemplo}
\newenvironment{contraejemplo}
  {\pushQED{\qed}\renewcommand{\qedsymbol}{\examplesymbol}\contraejemplox}
  {\popQED\endcontraejemplox}
\newtheorem{noejemplox}[proposicion]{No-ejemplo}
\newenvironment{noejemplo}
  {\pushQED{\qed}\renewcommand{\qedsymbol}{\examplesymbol}\noejemplox}
  {\popQED\endnoejemplox}
 
\newtheorem{ejemploastx}[proposicion]{Ejemplo ($\clubsuit$)}
\newenvironment{ejemploast}
  {\pushQED{\qed}\renewcommand{\qedsymbol}{\examplesymbol}\ejemploastx}
  {\popQED\endejemploastx}

\ifdefined\exercisespersection
  \theoremstyle{sectionexercise}
  \newtheorem{ejercicio}{}[section]
  \theoremstyle{mydefinition}
\else
  \ifdefined\exercisesglobal
    \theoremstyle{sectionexercise}
    \newtheorem{ejercicio}{}
    \theoremstyle{mydefinition}
  \else
    \ifdefined\exercisespersection
      \newtheorem{ejercicio}[proposicion]{Ejercicio}
    \fi
  \fi
\fi

% % % % % % % % % % % % % % % % % % % % % % % % % % % % % %

\theoremstyle{myplain}
\newtheorem{proposition}{Proposition}[section]
\newtheorem*{fact*}{Fact}
\newtheorem*{proposition*}{Proposition}
\newtheorem{lemma}[proposition]{Lemma}
\newtheorem*{lemma*}{Lemma}

\newtheorem{exercise}{Exercise}
\newtheorem*{hint}{Hint}

\newtheorem{theorem}[proposition]{Theorem}
\newtheorem{conjecture}[proposition]{Conjecture}
\newtheorem*{theorem*}{Theorem}
\newtheorem{corollary}[proposition]{Corollary}
\newtheorem{fact}[proposition]{Fact}
\newtheorem*{claim}{Claim}
\newtheorem{definition-theorem}[proposition]{Definition-theorem}

\theoremstyle{mydefinition}
\newtheorem{examplex}[proposition]{Example}
\newenvironment{example}
  {\pushQED{\qed}\renewcommand{\qedsymbol}{\examplesymbol}\examplex}
  {\popQED\endexamplex}

\newtheorem*{examplexx}{Example}
\newenvironment{example*}
  {\pushQED{\qed}\renewcommand{\qedsymbol}{\examplesymbol}\examplexx}
  {\popQED\endexamplexx}

\newtheorem{definition}[proposition]{Definition}
\newtheorem*{definition*}{Definition}
\newtheorem{wrong-definition}[proposition]{Wrong definition}
\newtheorem{remark}[proposition]{Remark}

\makeatletter
\newcommand{\xRightarrow}[2][]{\ext@arrow 0359\Rightarrowfill@{#1}{#2}}
\makeatother

% % % % % % % % % % % % % % % % % % % % % % % % % % % % % %

\newcommand{\Et}{\mathop{\text{\rm Ét}}}

\newcommand{\categ}[1]{\text{\bf #1}}
\newcommand{\vcateg}{\mathcal}
\newcommand{\bone}{{\boldsymbol 1}}
\newcommand{\bDelta}{{\boldsymbol\Delta}}
\newcommand{\bR}{{\mathbf{R}}}

\newcommand{\univ}{\mathfrak}

\newcommand{\TODO}{\colorbox{red}{\textbf{*** TODO ***}}}
\newcommand{\proofreadme}{\colorbox{red}{\textbf{*** NEEDS PROOFREADING ***}}}

\makeatletter
\def\iddots{\mathinner{\mkern1mu\raise\p@
\vbox{\kern7\p@\hbox{.}}\mkern2mu
\raise4\p@\hbox{.}\mkern2mu\raise7\p@\hbox{.}\mkern1mu}}
\makeatother

\newcommand{\ssincl}{\reflectbox{\rotatebox[origin=c]{45}{$\subseteq$}}}
\newcommand{\vsupseteq}{\reflectbox{\rotatebox[origin=c]{-90}{$\supseteq$}}}
\newcommand{\vin}{\reflectbox{\rotatebox[origin=c]{90}{$\in$}}}

\newcommand{\Ga}{\mathbb{G}_\mathrm{a}}
\newcommand{\Gm}{\mathbb{G}_\mathrm{m}}

\renewcommand{\U}{\UUU}

\DeclareRobustCommand{\Stirling}{\genfrac\{\}{0pt}{}}
\DeclareRobustCommand{\stirling}{\genfrac[]{0pt}{}}

% % % % % % % % % % % % % % % % % % % % % % % % % % % % % %
% tikz

\usepackage{tikz-cd}
\usetikzlibrary{babel}
\usetikzlibrary{decorations.pathmorphing}
\usetikzlibrary{arrows}
\usetikzlibrary{calc}
\usetikzlibrary{fit}
\usetikzlibrary{hobby}

% % % % % % % % % % % % % % % % % % % % % % % % % % % % % %
% Banners

\newcommand\mybannerext[3]{{\normalfont\sffamily\bfseries\large\noindent #1

\noindent #2

\noindent #3

}\noindent\rule{\textwidth}{1.25pt}

\vspace{1em}}

\newcommand\mybanner[2]{{\normalfont\sffamily\bfseries\large\noindent #1

\noindent #2

}\noindent\rule{\textwidth}{1.25pt}

\vspace{1em}}

\renewcommand{\O}{\mathcal{O}}


\numberwithin{equation}{section}

\usepackage[numbers]{natbib}

\usepackage{fullpage}

\author{Alexey Beshenov (cadadr@gmail.com)}
\title{Ejercicios sobre categorías}
\date{Universidad de El Salvador. Ciclo impar 2018}

\usepackage{multicol}

\setlength{\columnseprule}{0.4pt}

\theoremstyle{definition}
\newtheorem{ejerc}{Ejercicio}

\newif\ifsolutions
% \solutionstrue
\solutionsfalse

\usepackage{multirow}

\begin{document}

{\normalfont\sffamily\bfseries \maketitle}

% \subsection*{Funtores}
%
% \begin{ejerc}
% Para un conjunto $X$ sea
% $$P^+ (X) \dfn P^- (X) \dfn 2^X$$
% el conjunto de subconjuntos de $X$. Para una aplicación $f\colon X\to Y$
% definamos
% \begin{align*}
% P^+ (f)\colon 2^X & \to 2^Y,\\
% Z & \mapsto f (Z),
% \end{align*}
% y
% \begin{align*}
% P^- (f)\colon 2^Y & \to 2^X,\\
% Z & \mapsto f^{-1} (Z).
% \end{align*}
%
% Demuestre que $P^+$ y $P^-$ son funtores $\categ{Set} \to \categ{Set}$, uno
% covariante y el otro contravariante.
% \end{ejerc}

\subsection*{Iso-, epi-, mono-}

\begin{ejerc}
  Demuestre que si $f$ es un isomorfismo en $\vcateg{C}$ y $F$ es un funtor
  $\vcateg{C}\to \vcateg{D}$, entonces $F (f)$ es un isomorfismo en
  $\vcateg{D}$.
\end{ejerc}

\begin{ejerc}
  Demuestre que las composiciones de iso-, mono-, epimorfismos satisfacen las
  siguientes propiedades.

  \begin{enumerate}
  \item[1)] Si $f\colon X\to Y$ e $g\colon Y\to Z$ son isomorfismos, entonces
    $g\circ f\colon X\to Z$ es un isomorfismo.

  \item[2)] Si $m\colon X\mono Y$ y $m'\colon Y\mono Z$ son monomorfismos,
    entonces $m'\circ m\colon X\to Z$ es un monomorfismo.

  \item[3)] Si $e\colon X\epi Y$ y $e'\colon Y\epi Z$ son epimorfismos, entonces
    $e'\circ e\colon X\to Z$ es un epimorfismo.

  \item[4)] Si para $m\colon X\to Y$, $f\colon Y\to Z$ la composición $f\circ m$
    es un monomorfismo, entonces $m$ es un monomorfismo.

  \item[5)] Si para $f\colon X\to Y$, $e\colon Y\to Z$ la composición $e\circ f$
    es un epimorfismo, entonces $e$ es un epimorfismo.
  \end{enumerate}
\end{ejerc}

\begin{ejerc}
  Demuestre que en la categoría $k\categ{-Vect}$ los isomorfismos,
  monomorfismos, epimorfismos son las aplicaciones $k$-lineales biyectivas,
  inyectivas, sobreyectivas respectivamente.
\end{ejerc}

\subsection*{Lema de Yoneda}

\begin{ejerc}
  Demuestre con todos los detalles la versión covariante del lema de Yoneda.
\end{ejerc}

\begin{ejerc}
  Sea $G$ un grupo. Consideremos $G$ como una categoría. Note que un funtor
  $F\colon G\to \categ{Set}$ corresponde a un $G$-conjunto y una transformación
  natural entre tales funtores es una aplicación $G$-equivariante. ¿Qué es un
  funtor representable en este caso? ¿Qué significa el encajamiento de Yoneda?
\end{ejerc}

\begin{ejerc}
  Sea $R$ un anillo.

  \begin{enumerate}
  \item[a)] Demuestre que el funtor olvidadizo $R\categ{-Alg} \to \categ{Set}$
    es representable.

  \item[b)] Supongamos que para cada $R$-álgebra $A$ está especificada una
    aplicación entre conjuntos $\alpha_A\colon A \to A$ de tal manera que para
    todo homomorfismo de $R$-álgebras $\phi\colon A \to B$ se cumple
    $\phi \circ \alpha_A = \alpha_B \circ \phi$.
    \[ \begin{tikzcd}
        A \ar{r}{\alpha_A}\ar{d}[swap]{\phi} & A\ar{d}{\phi} \\
        B \ar{r}{\alpha_B} & B
      \end{tikzcd} \]

    Usando el lema de Yoneda, demuestre que existe un polinomio $f \in R[x]$ tal
    que para toda $R$-álgebra $A$ se tiene
    $$\alpha_A\colon a \mapsto f (a).$$

  \item[c)] Demuestre lo mismo sin recurrir a Yoneda.
  \end{enumerate}
\end{ejerc}

\subsection*{Límites y colímites}

\begin{ejerc}
  Consideremos la categoría $\categ{Top}_*$ cuyos objetos $(X,x_0)$ son espacios
  topológicos con un punto marcado $x_0\in X$ y cuyos morfismos
  $f\colon (X,x_0)\to (Y,y_0)$ son aplicaciones continuas $f\colon X\to Y$ tales
  que $f (x_0) = y_0$. Describa objetos terminales e iniciales, productos y
  coproductos en $\categ{Top}_*$.
\end{ejerc}

\begin{ejerc}
  Describa objetos terminales e iniciales, productos y coproductos en la
  categoría $\categ{Cat}$ de categorías pequeñas.
\end{ejerc}

\begin{ejerc}
  Para un grupo fijo $G$, consideremos la categoría $G\categ{-Set}$ cuyos
  objetos son $G$-conjuntos (conjuntos con acción de $G$) y cuyos morfismos
  $f\colon X\to Y$ son aplicaciones $G$-equivariantes (que satisfacen la
  condición $f (g\cdot x) = g\cdot f (x)$ para cualesauiera $g\in G$ y
  $x\in X$). Describa objetos terminales e iniciales, productos y coproductos en
  $G\categ{-Set}$.
\end{ejerc}

\begin{ejerc}
  Para una categoría pequeña sea $\widehat{\vcateg{C}}$ la categoría de funtores
  $F\colon \vcateg{C}^\mathrm{op} \to \categ{Set}$. Describa los objetos
  terminales e iniciales, productos y coproductos en $\widehat{\vcateg{C}}$.
\end{ejerc}

\begin{ejerc}
  Demuestre que los productos fibrados son funtoriales en el siguiente sentido:
  un diagrama conmutativo
  \[ \begin{tikzcd}
      X_1\ar{dr}\ar[d] & & Y_1\ar{dl}\ar{d} \\
      X_2\ar{dr} & Z_1\ar{d} & Y_2\ar{dl} \\
      & Z_2
    \end{tikzcd} \]
  induce un morfismo canónico
  $X_1\times_{Z_1} Y_1 \to X_2 \times_{Z_2} Y_2$.
\end{ejerc}

\begin{ejerc}
  Demuestre que en la categoría $k\categ{-Vect}$ se tiene
  $$\eq (f,g) = \ker (f-g)\quad\text{y}\quad\coeq (f,g) = \coker (f-g).$$
\end{ejerc}

\begin{ejerc}
  Demuestre que los productos fibrados preservan isomorfismos: si la flecha
  $Y\to Z$ es un isomorfismo, entonces $X\times_Z Y\to X$ es también un
  isomorfismo:

  \[ \begin{tikzpicture}
      \matrix(m)[matrix of math nodes, row sep=2em, column sep=2em,
      text height=1.5ex, text depth=0.25ex]{
        X\times_Z Y & Y \\
        X & Z \\};
      \path[->] (m-1-1) edge (m-1-2);
      \path[->] (m-2-1) edge (m-2-2);
      \path[->,font=\scriptsize] (m-1-1) edge node[left] {$\isom$} (m-2-1);
      \path[->,font=\scriptsize] (m-1-2) edge node[right] {$\isom$} (m-2-2);

      \begin{scope}[shift=($(m-1-1)!.4!(m-2-2)$)]
        \draw +(-.2,0) -- +(0,0)  -- +(0,.2);
      \end{scope}
    \end{tikzpicture} \]
\end{ejerc}

\begin{ejerc}
  Demuestre que los productos fibrados

  \[ \begin{tikzpicture}
      \matrix(m)[matrix of math nodes, row sep=2em, column sep=2em,
      text height=1.5ex, text depth=0.25ex]{
        E & X \\
        X & X\times Y\\};
      \path[->,font=\scriptsize] (m-1-1) edge node[above] {$e'$} (m-1-2);
      \path[->,font=\scriptsize] (m-2-1) edge node[below] {$\tcol{\idid}{f}$} (m-2-2);
      \path[->,font=\scriptsize] (m-1-1) edge node[left] {$e$} (m-2-1);
      \path[->,font=\scriptsize] (m-1-2) edge node[right] {$\tcol{\idid}{g}$} (m-2-2);

      \begin{scope}[shift=($(m-1-1)!.4!(m-2-2)$)]
        \draw +(-.2,0) -- +(0,0)  -- +(0,.2);
      \end{scope}
    \end{tikzpicture} \quad\quad \begin{tikzpicture}
      \matrix(m)[matrix of math nodes, row sep=2em, column sep=2em,
      text height=1.5ex, text depth=0.25ex]{
        E & X \\
        X & Y\times Y\\};
      \path[->,font=\scriptsize] (m-1-1) edge node[above] {$e$} (m-1-2);
      \path[->,font=\scriptsize] (m-2-1) edge node[below] {$\tcol{\idid}{\idid}$} (m-2-2);
      \path[->,font=\scriptsize] (m-1-1) edge node[left] {$e'$} (m-2-1);
      \path[->,font=\scriptsize] (m-1-2) edge node[right] {$\tcol{f}{g}$} (m-2-2);

      \begin{scope}[shift=($(m-1-1)!.4!(m-2-2)$)]
        \draw +(-.2,0) -- +(0,0)  -- +(0,.2);
      \end{scope}
    \end{tikzpicture} \]
  calculan el ecualizador de $f,g\colon X\to Y$. Formule y demuestre la
  propiedad dual para coecualizadores y coproductos fibrados.
\end{ejerc}

\subsection*{Transformaciones naturales}

\begin{ejerc}
  Sean $\vcateg{C},\vcateg{D},\vcateg{E}$ tres categorías. Sean $F,G,H$ funtores
  $\vcateg{C} \to \vcateg{D}$ y sean $I,J,K$ tres funtores
  $\vcateg{D}\to \vcateg{E}$. Consideremos transformaciones naturales
  $$\alpha\colon F\Rightarrow G, \quad \beta\colon G\Rightarrow H, \quad
    \sigma\colon I\Rightarrow J, \quad \tau\colon J\Rightarrow K.$$

  \[ \begin{tikzpicture} \matrix[matrix of math nodes,column sep=3cm,row sep=5em] (m) {
        \vcateg{C} & \vcateg{D} & \vcateg{E} \\
      };
      \draw[->] (m-1-1) to[bend left=50]
      node[label=above:$F$] (F) {} (m-1-2); \draw[] (m-1-1) to node (G) {}
      (m-1-2); \draw[->] (m-1-1) to
      node[label=above:$G$,near end] {} (m-1-2); \draw[->] (m-1-1) to[bend right=50]
      node[label=below:$H$] (H) {} (m-1-2); \draw[->] (m-1-2) to[bend left=50]
      node[label=above:$I$] (I) {} (m-1-3); \draw[->] (m-1-2) to
      node[label=above:$J$,near end] {} (m-1-3); \draw[] (m-1-2) to node (J) {}
      (m-1-3); \draw[->] (m-1-2) to[bend right=50]
      node[label=below:$K$] (K) {} (m-1-3);

      \draw[double,double equal sign distance,-implies] (F) -- node[label=left:$\alpha$] {} (G);
      \draw[double,double equal sign distance,-implies] (G) -- node[label=left:$\beta$] {} (H);
      \draw[double,double equal sign distance,-implies] (I) -- node[label=left:$\sigma$] {} (J);
      \draw[double,double equal sign distance,-implies] (J) -- node[label=left:$\tau$] {} (K);

    \end{tikzpicture} \]

  Demuestre que
  $$(\tau\circ \sigma)\ast (\beta\circ \alpha) =
    (\tau\ast\beta)\circ (\sigma\ast\alpha),$$
    donde $\ast$ denota el producto de Godement.
  \end{ejerc}

\begin{ejerc}
  Demuestre que el producto de Godement es asociativo: para un diagrama
  \[ \begin{tikzpicture}
      \matrix[matrix of math nodes,column sep=2cm] (m)
      {
        \vcateg{A} & \vcateg{B} & \vcateg{C} & \vcateg{D} \\
      };
      \draw[->] (m-1-1) to[bend left=50] node[label=above:$F_1$] (F1) {} (m-1-2);
      \draw[->] (m-1-1) to[bend right=50] node[label=below:$G_1$] (G1) {} (m-1-2);
      \draw[->] (m-1-2) to[bend left=50] node[label=above:$F_2$] (F2) {} (m-1-3);
      \draw[->] (m-1-2) to[bend right=50] node[label=below:$G_2$] (G2) {} (m-1-3);
      \draw[->] (m-1-3) to[bend left=50] node[label=above:$F_3$] (F3) {} (m-1-4);
      \draw[->] (m-1-3) to[bend right=50] node[label=below:$G_3$] (G3) {} (m-1-4);

      \draw[double,double equal sign distance,-implies,shorten >=10pt,shorten <=10pt] 
      (F1) -- node[label=right:$\alpha$] {} (G1);
      \draw[double,double equal sign distance,-implies,shorten >=10pt,shorten <=10pt] 
      (F2) -- node[label=right:$\beta$] {} (G2);
      \draw[double,double equal sign distance,-implies,shorten >=10pt,shorten <=10pt] 
      (F3) -- node[label=right:$\gamma$] {} (G3);
    \end{tikzpicture} \]
  se cumple
  $$(\gamma\ast\beta)\ast\alpha = \gamma\ast(\beta\ast\alpha).$$
\end{ejerc}

\begin{ejerc}
  Sea $\vcateg{I}$ una categoría pequeña. Consideremos el funtor
  \begin{align*}
    \Delta\colon \vcateg{C} & \to \Fun (\vcateg{I}, \vcateg{C}),\\
    X & \rightsquigarrow \Delta_X
  \end{align*}
  que a cada objeto $X\in \Ob (\vcateg{C})$ asocia el funtor constante
  $\Delta_X \colon \vcateg{I} \to \vcateg{C}$ (tal que $\Delta_X (i) = X$ para
  cada $i\in \Ob (\vcateg{I})$). Demuestre que para todo funtor
  $F\colon\vcateg{I}\to\vcateg{C}$ hay biyecciones naturales
  \begin{align*}
    \Nat (\Delta_X, F) & \isom \Hom_\vcateg{C} (X, \lim_\vcateg{I} F),\\
    \Nat (F, \Delta_X) & \isom \Hom_\vcateg{C} (\colim_\vcateg{I} F, X).
  \end{align*}
\end{ejerc}

\subsection*{Adjunciones}

\begin{ejerc}
  Sea $\categ{Ring}_1$ la categoría de anillos con identidad donde los morfismos
  son los homomorfismos $f\colon R\to S$ que satisfacen $f (1_R) \to 1_S$ y
  $\categ{Ring}$ la categoría de anillos que no necesariamente tienen identidad.

  Para un anillo $R$ consideremos el conjunto $\widehat{R} \dfn \ZZ\times R$ con
  la multiplicación
  $$(n_1, r_1)\cdot (n_2, r_2) \dfn (n_1 n_2, n_1 r_1 + n_2 r_1 + r_1 r_2).$$
  Note que es un anillo con identidad $(1,0)$. Demuestre que
  $R \rightsquigarrow \widehat{R}$ es un funtor
  $\categ{Ring} \to \categ{Ring}_1$ y es adjunto por la izquierda a la inclusión
  $\categ{Ring}_1 \hookrightarrow \categ{Ring}$.
\end{ejerc}

\begin{ejerc}
  Digamos que en una categoría $\vcateg{C}$ dos objetos $X$ e $Y$ están en la
  misma componente conexa si existe una cadena de morfismos de $X$ a $Y$, que no
  necesariamente van en la misma dirección, por ejemplo
  $$X \rightarrow \bullet \rightarrow \bullet \leftarrow
    \bullet \rightarrow \bullet \rightarrow \bullet \leftarrow Y$$
  Para una categoría pequeña $\vcateg{C}$ sea $\pi_0 (\vcateg{C})$ el conjunto
  de sus componentes conexas. Demuestre que $\pi_0$ es un funtor
  $\categ{Cat} \to \categ{Set}$. Demuestre que es adjunto por la izquierda al
  funtor $\categ{Set} \to \categ{Cat}$ que a cada conjunto $X$ asocia la
  categoría donde los objetos son los elementos de $X$ y los únicos morfismos
  son los morfismos identidad.
\end{ejerc}

\begin{ejerc}
  Sean $X$ e $Y$ dos conjuntos. Consideremos $2^X$ y $2^Y$ como conjuntos
  parcialmente ordenados por la relación $\subseteq$, y en particular como
  categorías.

  Sea $f\colon X\to Y$ una aplicación. Para $A\in 2^X$ definamos
  \begin{align*}
    f_* (A) & \dfn \{ y\in Y \mid f^{-1} (y) \subseteq A \},\\
    \im (A) & \dfn \{ f (x) \mid x\in A \},
  \end{align*}
  y para $B\in 2^Y$ definamos
  $$f^{-1} (B) \dfn \{ x\in X \mid f (x) \in B \}.$$
  Demuestre que $f_*$ e $\im$ son funtores $2^X\to 2^Y$ y $f^{-1}$ es un funtor
  $2^Y \to 2^X$. Demuestre que

  \begin{enumerate}
  \item[1)] $\im$ es adjunto por la izquierda a $f^{-1}$,

  \item[2)] $f^{-1}$ es adjunto por la izquierda a $f_*$.
  \end{enumerate}

  ¿Qué significa en este caso la preservación de objetos iniciales y coproductos
  (resp. objetos terminales y productos) por adjunto por la izquierda
  (resp. adjunto por la derecha)?
\end{ejerc}

\subsection*{Equivalencias de categorías}

\begin{ejerc}
  Demuestre que si $F\colon \vcateg{C}\to \vcateg{D}$ es una equivalencia de
  categorías, entonces $F$ envía un objeto terminal (resp. inicial) de
  $\vcateg{C}$ en un objeto terminal (resp. inicial) de $\vcateg{D}$.

  Supongamos que existe una equivalencia de categorías
  $F\colon \categ{Set}\to \categ{Set}^\mathrm{op}$. Note que en este caso para
  cada conjunto $X$ tendríamos
  $$X \isom \Hom_\categ{Set} (\{ \ast \},X) \isom
            \Hom_\categ{Set} (F (X), \emptyset).$$
  Concluya que las categorías $\categ{Set}$ y $\categ{Set}^\mathrm{op}$ no son
  equivalentes.
\end{ejerc}

\subsection*{Equivalencias de categorías}

\begin{ejerc}
Demuestre que si $F\colon \vcateg{C}\to \vcateg{D}$ es una equivalencia de categorías, entonces $F$ envía un objeto terminal (resp. inicial) de $\vcateg{C}$ en un objeto terminal (resp. inicial) de $\vcateg{D}$.

Supongamos que existe una equivalencia de categorías $F\colon \categ{Set}\to \categ{Set}^\mathrm{op}$. Note que en este caso para cada conjunto $X$ tendríamos
$$X \isom \Hom_\categ{Set} (\{ \ast \},X) \isom \Hom_\categ{Set} (F (X), \emptyset).$$
Concluya que las categorías $\categ{Set}$ y $\categ{Set}^\mathrm{op}$ no son equivalentes.
\end{ejerc}

\iffalse
\pagebreak

\section*{Preguntas teóricas}

\begin{enumerate}
\item Defina qué es una categoría y dé algunos ejemplos.

\item Defina qué es un isomorfismo, epimorfismo, monomorfismo y dé algunos ejemplos.

\item Defina qué es un funtor y dé algunos ejemplos.

\item Defina qué es la categoría opuesta a una categoría.

\item Defina qué es una transformación natural y dé algunos ejemplos.

\item Formule el lema de Yoneda y describa las aplicaciones mutualmente inversas involucradas.

\item Defina qué es una adjunción entre funtores y dé algunos ejemplos.

\item Formule el criterio de funtor adjunto en términos de representabilidad.

\item Defina qué es la unidad y counidad de una adjunción y dé algunos ejemplos.

\item Defina qué es un objeto terminal e inicial y dé algunos ejemplos.

\item Defina qué es un producto y coproducto y dé algunos ejemplos.

\item Demuestre que todo funtor adjunto por la izquierda preserva coproductos y todo funtor adjunto por la derecha preserva productos. Dé algunos ejemplos.

\item Defina qué es un producto y coproducto fibrado y dé algunos ejemplos.

\item Defina qué es un ecualizador y coecualizador y dé algunos ejemplos.

\item Defina qué es un límite y colímite y dé algunos ejemplos.

\item Defina qué es una equivalencia de categorías y dé algunos ejemplos.

\item Demuestre que un funtor es una equivalencia si y solamente si es fielmente pleno y esencialmente sobreyectivo.
\end{enumerate}

En cada pregunta ``algunos ejemplos'' significa ``al menos tres ejemplos interesantes''.
\fi

\end{document}
