\documentclass{article}

% TODO : CLEAN UP THIS MESS
% (AND MAKE SURE ALL TEXTS STILL COMPILE)
\usepackage[leqno]{amsmath}
\usepackage{amssymb}
\usepackage{graphicx}

\usepackage{diagbox} % table heads with diagonal lines
\usepackage{relsize}

\usepackage{wasysym}
\usepackage{scrextend}
\usepackage{epigraph}
\setlength\epigraphwidth{.6\textwidth}

\usepackage[utf8]{inputenc}

\usepackage{titlesec}
\titleformat{\chapter}[display]
  {\normalfont\sffamily\huge\bfseries}
  {\chaptertitlename\ \thechapter}{5pt}{\Huge}
\titleformat{\section}
  {\normalfont\sffamily\Large\bfseries}
  {\thesection}{1em}{}
\titleformat{\subsection}
  {\normalfont\sffamily\large\bfseries}
  {\thesubsection}{1em}{}
\titleformat{\part}[display]
  {\normalfont\sffamily\huge\bfseries}
  {\partname\ \thepart}{0pt}{\Huge}

\usepackage[T1]{fontenc}
\usepackage{fourier}
\usepackage{paratype}

\usepackage[symbol,perpage]{footmisc}

\usepackage{perpage}
\MakePerPage{footnote}

\usepackage{array}
\newcolumntype{x}[1]{>{\centering\hspace{0pt}}p{#1}}

% TODO: the following line causes conflict with new texlive (!)
% \usepackage[english,russian,polutonikogreek,spanish]{babel}
% \newcommand{\russian}[1]{{\selectlanguage{russian}#1}}

% Remove conflicting options for the moment:
\usepackage[english,polutonikogreek,spanish]{babel}

\AtBeginDocument{\shorthandoff{"}}
\newcommand{\greek}[1]{{\selectlanguage{polutonikogreek}#1}}

% % % % % % % % % % % % % % % % % % % % % % % % % % % % % %
% Limit/colimit symbols (with accented i: lím / colím)

\usepackage{etoolbox} % \patchcmd

\makeatletter
\patchcmd{\varlim@}{lim}{\lim}{}{}
\makeatother
\DeclareMathOperator*{\colim}{co{\lim}}
\newcommand{\dirlim}{\varinjlim}
\newcommand{\invlim}{\varprojlim}

% % % % % % % % % % % % % % % % % % % % % % % % % % % % % %

\usepackage[all,color]{xy}

\usepackage{pigpen}
\newcommand{\po}{\ar@{}[dr]|(.4){\text{\pigpenfont I}}}
\newcommand{\pb}{\ar@{}[dr]|(.3){\text{\pigpenfont A}}}
\newcommand{\polr}{\ar@{}[dr]|(.65){\text{\pigpenfont A}}}
\newcommand{\pour}{\ar@{}[ur]|(.65){\text{\pigpenfont G}}}
\newcommand{\hstar}{\mathop{\bigstar}}

\newcommand{\bigast}{\mathop{\Huge \mathlarger{\mathlarger{\ast}}}}

\newcommand{\term}{\textbf}

\usepackage{stmaryrd}

\usepackage{cancel}

\usepackage{tikzsymbols}

\newcommand{\open}{\underset{\mathrm{open}}{\hookrightarrow}}
\newcommand{\closed}{\underset{\mathrm{closed}}{\hookrightarrow}}

\newcommand{\tcol}[2]{{#1 \choose #2}}

\newcommand{\homot}{\simeq}
\newcommand{\isom}{\cong}
\newcommand{\cH}{\mathcal{H}}
\renewcommand{\hom}{\mathrm{hom}}
\renewcommand{\div}{\mathop{\mathrm{div}}}
\renewcommand{\Im}{\mathop{\mathrm{Im}}}
\renewcommand{\Re}{\mathop{\mathrm{Re}}}
\newcommand{\id}[1]{\mathrm{id}_{#1}}
\newcommand{\idid}{\mathrm{id}}

\newcommand{\ZG}{{\ZZ G}}
\newcommand{\ZH}{{\ZZ H}}

\newcommand{\quiso}{\simeq}

\newcommand{\personality}[1]{{\sc #1}}

\newcommand{\mono}{\rightarrowtail}
\newcommand{\epi}{\twoheadrightarrow}
\newcommand{\xepi}[1]{\xrightarrow{#1}\mathrel{\mkern-14mu}\rightarrow}

% % % % % % % % % % % % % % % % % % % % % % % % % % % % % %

\DeclareMathOperator{\Ad}{Ad}
\DeclareMathOperator{\Aff}{Aff}
\DeclareMathOperator{\Ann}{Ann}
\DeclareMathOperator{\Aut}{Aut}
\DeclareMathOperator{\Br}{Br}
\DeclareMathOperator{\CH}{CH}
\DeclareMathOperator{\Cl}{Cl}
\DeclareMathOperator{\Coeq}{Coeq}
\DeclareMathOperator{\Coind}{Coind}
\DeclareMathOperator{\Cop}{Cop}
\DeclareMathOperator{\Corr}{Corr}
\DeclareMathOperator{\Cor}{Cor}
\DeclareMathOperator{\Cov}{Cov}
\DeclareMathOperator{\Der}{Der}
\DeclareMathOperator{\Div}{Div}
\DeclareMathOperator{\D}{D}
\DeclareMathOperator{\Ehr}{Ehr}
\DeclareMathOperator{\End}{End}
\DeclareMathOperator{\Eq}{Eq}
\DeclareMathOperator{\Ext}{Ext}
\DeclareMathOperator{\Frac}{Frac}
\DeclareMathOperator{\Frob}{Frob}
\DeclareMathOperator{\Funct}{Funct}
\DeclareMathOperator{\Fun}{Fun}
\DeclareMathOperator{\GL}{GL}
\DeclareMathOperator{\Gal}{Gal}
\DeclareMathOperator{\Gr}{Gr}
\DeclareMathOperator{\Hol}{Hol}
\DeclareMathOperator{\Hom}{Hom}
\DeclareMathOperator{\Ho}{Ho}
\DeclareMathOperator{\Id}{Id}
\DeclareMathOperator{\Ind}{Ind}
\DeclareMathOperator{\Inn}{Inn}
\DeclareMathOperator{\Isom}{Isom}
\DeclareMathOperator{\Ker}{Ker}
\DeclareMathOperator{\Lan}{Lan}
\DeclareMathOperator{\Lie}{Lie}
\DeclareMathOperator{\Map}{Map}
\DeclareMathOperator{\Mat}{Mat}
\DeclareMathOperator{\Max}{Max}
\DeclareMathOperator{\Mor}{Mor}
\DeclareMathOperator{\Nat}{Nat}
\DeclareMathOperator{\Nrd}{Nrd}
\DeclareMathOperator{\Ob}{Ob}
\DeclareMathOperator{\Out}{Out}
\DeclareMathOperator{\PGL}{PGL}
\DeclareMathOperator{\PSL}{PSL}
\DeclareMathOperator{\PSU}{PSU}
\DeclareMathOperator{\Pic}{Pic}
\DeclareMathOperator{\RHom}{RHom}
\DeclareMathOperator{\Rad}{Rad}
\DeclareMathOperator{\Ran}{Ran}
\DeclareMathOperator{\Rep}{Rep}
\DeclareMathOperator{\Res}{Res}
\DeclareMathOperator{\SL}{SL}
\DeclareMathOperator{\SO}{SO}
\DeclareMathOperator{\SU}{SU}
\DeclareMathOperator{\Sh}{Sh}
\DeclareMathOperator{\Sing}{Sing}
\DeclareMathOperator{\Specm}{Specm}
\DeclareMathOperator{\Spec}{Spec}
\DeclareMathOperator{\Sp}{Sp}
\DeclareMathOperator{\Stab}{Stab}
\DeclareMathOperator{\Sym}{Sym}
\DeclareMathOperator{\Tors}{Tors}
\DeclareMathOperator{\Tor}{Tor}
\DeclareMathOperator{\Tot}{Tot}
\DeclareMathOperator{\UUU}{U}

\DeclareMathOperator{\adj}{adj}
\DeclareMathOperator{\ad}{ad}
\DeclareMathOperator{\af}{af}
\DeclareMathOperator{\card}{card}
\DeclareMathOperator{\cm}{cm}
\DeclareMathOperator{\codim}{codim}
\DeclareMathOperator{\cod}{cod}
\DeclareMathOperator{\coeq}{coeq}
\DeclareMathOperator{\coim}{coim}
\DeclareMathOperator{\coker}{coker}
\DeclareMathOperator{\cont}{cont}
\DeclareMathOperator{\conv}{conv}
\DeclareMathOperator{\cor}{cor}
\DeclareMathOperator{\depth}{depth}
\DeclareMathOperator{\diag}{diag}
\DeclareMathOperator{\diam}{diam}
\DeclareMathOperator{\dist}{dist}
\DeclareMathOperator{\dom}{dom}
\DeclareMathOperator{\eq}{eq}
\DeclareMathOperator{\ev}{ev}
\DeclareMathOperator{\ex}{ex}
\DeclareMathOperator{\fchar}{char}
\DeclareMathOperator{\fr}{fr}
\DeclareMathOperator{\gr}{gr}
\DeclareMathOperator{\im}{im}
\DeclareMathOperator{\infl}{inf}
\DeclareMathOperator{\interior}{int}
\DeclareMathOperator{\intrel}{intrel}
\DeclareMathOperator{\inv}{inv}
\DeclareMathOperator{\length}{length}
\DeclareMathOperator{\mcd}{mcd}
\DeclareMathOperator{\mcm}{mcm}
\DeclareMathOperator{\multideg}{multideg}
\DeclareMathOperator{\ord}{ord}
\DeclareMathOperator{\pr}{pr}
\DeclareMathOperator{\rel}{rel}
\DeclareMathOperator{\res}{res}
\DeclareMathOperator{\rkred}{rkred}
\DeclareMathOperator{\rkss}{rkss}
\DeclareMathOperator{\rk}{rk}
\DeclareMathOperator{\sgn}{sgn}
\DeclareMathOperator{\sk}{sk}
\DeclareMathOperator{\supp}{supp}
\DeclareMathOperator{\trdeg}{trdeg}
\DeclareMathOperator{\tr}{tr}
\DeclareMathOperator{\vol}{vol}

\newcommand{\iHom}{\underline{\Hom}}

\renewcommand{\AA}{\mathbb{A}}
\newcommand{\CC}{\mathbb{C}}
\renewcommand{\SS}{\mathbb{S}}
\newcommand{\TT}{\mathbb{T}}
\newcommand{\PP}{\mathbb{P}}
\newcommand{\BB}{\mathbb{B}}
\newcommand{\RR}{\mathbb{R}}
\newcommand{\ZZ}{\mathbb{Z}}
\newcommand{\FF}{\mathbb{F}}
\newcommand{\HH}{\mathbb{H}}
\newcommand{\NN}{\mathbb{N}}
\newcommand{\QQ}{\mathbb{Q}}
\newcommand{\KK}{\mathbb{K}}

% % % % % % % % % % % % % % % % % % % % % % % % % % % % % %

\usepackage{amsthm}

\newcommand{\legendre}[2]{\left(\frac{#1}{#2}\right)}

\newcommand{\examplesymbol}{$\blacktriangle$}
\renewcommand{\qedsymbol}{$\blacksquare$}

\newcommand{\dfn}{\mathrel{\mathop:}=}
\newcommand{\rdfn}{=\mathrel{\mathop:}}

\usepackage{xcolor}
\definecolor{mylinkcolor}{rgb}{0.0,0.4,1.0}
\definecolor{mycitecolor}{rgb}{0.0,0.4,1.0}
\definecolor{shadecolor}{rgb}{0.79,0.78,0.65}
\definecolor{gray}{rgb}{0.6,0.6,0.6}

\usepackage{colortbl}

\definecolor{myred}{rgb}{0.7,0.0,0.0}
\definecolor{mygreen}{rgb}{0.0,0.7,0.0}
\definecolor{myblue}{rgb}{0.0,0.0,0.7}

\definecolor{redshade}{rgb}{0.9,0.5,0.5}
\definecolor{greenshade}{rgb}{0.5,0.9,0.5}

\usepackage[unicode,colorlinks=true,linkcolor=mylinkcolor,citecolor=mycitecolor]{hyperref}
\newcommand{\refref}[2]{\hyperref[#2]{#1~\ref*{#2}}}
\newcommand{\eqnref}[1]{\hyperref[#1]{(\ref*{#1})}}

\newcommand{\tos}{\!\!\to\!\!}

\usepackage{framed}

\newcommand{\cequiv}{\simeq}

\makeatletter
\newcommand\xleftrightarrow[2][]{%
  \ext@arrow 9999{\longleftrightarrowfill@}{#1}{#2}}
\newcommand\longleftrightarrowfill@{%
  \arrowfill@\leftarrow\relbar\rightarrow}
\makeatother

\newcommand{\bsquare}{\textrm{\ding{114}}}

% % % % % % % % % % % % % % % % % % % % % % % % % % % % % %

\newtheoremstyle{myplain}
  {\topsep}   % ABOVESPACE
  {\topsep}   % BELOWSPACE
  {\itshape}  % BODYFONT
  {0pt}       % INDENT (empty value is the same as 0pt)
  {\bfseries} % HEADFONT
  {.}         % HEADPUNCT
  {5pt plus 1pt minus 1pt} % HEADSPACE
  {\thmnumber{#2}. \thmname{#1}\thmnote{ (#3)}}   % CUSTOM-HEAD-SPEC

\newtheoremstyle{myplainnameless}
  {\topsep}   % ABOVESPACE
  {\topsep}   % BELOWSPACE
  {\normalfont}  % BODYFONT
  {0pt}       % INDENT (empty value is the same as 0pt)
  {\bfseries} % HEADFONT
  {.}         % HEADPUNCT
  {5pt plus 1pt minus 1pt} % HEADSPACE
  {\thmnumber{#2}}   % CUSTOM-HEAD-SPEC 

\newtheoremstyle{sectionexercise}
  {\topsep}   % ABOVESPACE
  {\topsep}   % BELOWSPACE
  {\normalfont}  % BODYFONT
  {0pt}       % INDENT (empty value is the same as 0pt)
  {\bfseries} % HEADFONT
  {.}         % HEADPUNCT
  {5pt plus 1pt minus 1pt} % HEADSPACE
  {Ejercicio \thmnumber{#2}\thmnote{ (#3)}}   % CUSTOM-HEAD-SPEC

\newtheoremstyle{mydefinition}
  {\topsep}   % ABOVESPACE
  {\topsep}   % BELOWSPACE
  {\normalfont}  % BODYFONT
  {0pt}       % INDENT (empty value is the same as 0pt)
  {\bfseries} % HEADFONT
  {.}         % HEADPUNCT
  {5pt plus 1pt minus 1pt} % HEADSPACE
  {\thmnumber{#2}. \thmname{#1}\thmnote{ (#3)}}   % CUSTOM-HEAD-SPEC

% EN ESPAÑOL

\newtheorem*{hecho*}{Hecho}
\newtheorem*{corolario*}{Corolario}
\newtheorem*{teorema*}{Teorema}
\newtheorem*{conjetura*}{Conjetura}
\newtheorem*{proyecto*}{Proyecto}
\newtheorem*{observacion*}{Observación}

\newtheorem*{lema*}{Lema}
\newtheorem*{resultado-clave*}{Resultado clave}
\newtheorem*{proposicion*}{Proposición}

\theoremstyle{definition}
\newtheorem*{ejercicio*}{Ejercicio}
\newtheorem*{definicion*}{Definición}
\newtheorem*{comentario*}{Comentario}
\newtheorem*{definicion-alternativa*}{Definición alternativa}
\newtheorem*{ejemploxs}{Ejemplo}
\newenvironment{ejemplo*}
  {\pushQED{\qed}\renewcommand{\qedsymbol}{\examplesymbol}\ejemploxs}
  {\popQED\endejemploxs}

\theoremstyle{myplain}
\newtheorem{proposicion}{Proposición}[section]

\newtheorem{proyecto}[proposicion]{Proyecto}
\newtheorem{teorema}[proposicion]{Teorema}
\newtheorem{corolario}[proposicion]{Corolario}
\newtheorem{hecho}[proposicion]{Hecho}
\newtheorem{lema}[proposicion]{Lema}

\newtheorem{observacion}[proposicion]{Observación}

\newenvironment{observacionejerc}
    {\pushQED{\qed}\renewcommand{\qedsymbol}{$\square$}\csname inner@observacionejerc\endcsname}
    {\popQED\csname endinner@observacionejerc\endcsname}
\newtheorem{inner@observacionejerc}[proposicion]{Observación}

\newenvironment{proposicionejerc}
    {\pushQED{\qed}\renewcommand{\qedsymbol}{$\square$}\csname inner@proposicionejerc\endcsname}
    {\popQED\csname endinner@proposicionejerc\endcsname}
\newtheorem{inner@proposicionejerc}[proposicion]{Proposicion}

\newenvironment{lemaejerc}
    {\pushQED{\qed}\renewcommand{\qedsymbol}{$\square$}\csname inner@lemaejerc\endcsname}
    {\popQED\csname endinner@lemaejerc\endcsname}
\newtheorem{inner@lemaejerc}[proposicion]{Lema}

\newtheorem{calculo}[proposicion]{Cálculo}

\theoremstyle{myplainnameless}
\newtheorem{nameless}[proposicion]{}

\theoremstyle{mydefinition}
\newtheorem{comentario}[proposicion]{Comentario}
\newtheorem{comentarioast}[proposicion]{Comentario ($\clubsuit$)}
\newtheorem{construccion}[proposicion]{Construcción}
\newtheorem{aplicacion}[proposicion]{Aplicación}
\newtheorem{definicion}[proposicion]{Definición}
\newtheorem{definicion-alternativa}[proposicion]{Definición alternativa}
\newtheorem{notacion}[proposicion]{Notación}
\newtheorem{advertencia}[proposicion]{Advertencia}
\newtheorem{digresion}[proposicion]{Digresión}
\newtheorem{ejemplox}[proposicion]{Ejemplo}
\newenvironment{ejemplo}
  {\pushQED{\qed}\renewcommand{\qedsymbol}{\examplesymbol}\ejemplox}
  {\popQED\endejemplox}
\newtheorem{contraejemplox}[proposicion]{Contraejemplo}
\newenvironment{contraejemplo}
  {\pushQED{\qed}\renewcommand{\qedsymbol}{\examplesymbol}\contraejemplox}
  {\popQED\endcontraejemplox}
\newtheorem{noejemplox}[proposicion]{No-ejemplo}
\newenvironment{noejemplo}
  {\pushQED{\qed}\renewcommand{\qedsymbol}{\examplesymbol}\noejemplox}
  {\popQED\endnoejemplox}
 
\newtheorem{ejemploastx}[proposicion]{Ejemplo ($\clubsuit$)}
\newenvironment{ejemploast}
  {\pushQED{\qed}\renewcommand{\qedsymbol}{\examplesymbol}\ejemploastx}
  {\popQED\endejemploastx}

\ifdefined\exercisespersection
  \theoremstyle{sectionexercise}
  \newtheorem{ejercicio}{}[section]
  \theoremstyle{mydefinition}
\else
  \ifdefined\exercisesglobal
    \theoremstyle{sectionexercise}
    \newtheorem{ejercicio}{}
    \theoremstyle{mydefinition}
  \else
    \ifdefined\exercisespersection
      \newtheorem{ejercicio}[proposicion]{Ejercicio}
    \fi
  \fi
\fi

% % % % % % % % % % % % % % % % % % % % % % % % % % % % % %

\theoremstyle{myplain}
\newtheorem{proposition}{Proposition}[section]
\newtheorem*{fact*}{Fact}
\newtheorem*{proposition*}{Proposition}
\newtheorem{lemma}[proposition]{Lemma}
\newtheorem*{lemma*}{Lemma}

\newtheorem{exercise}{Exercise}
\newtheorem*{hint}{Hint}

\newtheorem{theorem}[proposition]{Theorem}
\newtheorem{conjecture}[proposition]{Conjecture}
\newtheorem*{theorem*}{Theorem}
\newtheorem{corollary}[proposition]{Corollary}
\newtheorem{fact}[proposition]{Fact}
\newtheorem*{claim}{Claim}
\newtheorem{definition-theorem}[proposition]{Definition-theorem}

\theoremstyle{mydefinition}
\newtheorem{examplex}[proposition]{Example}
\newenvironment{example}
  {\pushQED{\qed}\renewcommand{\qedsymbol}{\examplesymbol}\examplex}
  {\popQED\endexamplex}

\newtheorem*{examplexx}{Example}
\newenvironment{example*}
  {\pushQED{\qed}\renewcommand{\qedsymbol}{\examplesymbol}\examplexx}
  {\popQED\endexamplexx}

\newtheorem{definition}[proposition]{Definition}
\newtheorem*{definition*}{Definition}
\newtheorem{wrong-definition}[proposition]{Wrong definition}
\newtheorem{remark}[proposition]{Remark}

\makeatletter
\newcommand{\xRightarrow}[2][]{\ext@arrow 0359\Rightarrowfill@{#1}{#2}}
\makeatother

% % % % % % % % % % % % % % % % % % % % % % % % % % % % % %

\newcommand{\Et}{\mathop{\text{\rm Ét}}}

\newcommand{\categ}[1]{\text{\bf #1}}
\newcommand{\vcateg}{\mathcal}
\newcommand{\bone}{{\boldsymbol 1}}
\newcommand{\bDelta}{{\boldsymbol\Delta}}
\newcommand{\bR}{{\mathbf{R}}}

\newcommand{\univ}{\mathfrak}

\newcommand{\TODO}{\colorbox{red}{\textbf{*** TODO ***}}}
\newcommand{\proofreadme}{\colorbox{red}{\textbf{*** NEEDS PROOFREADING ***}}}

\makeatletter
\def\iddots{\mathinner{\mkern1mu\raise\p@
\vbox{\kern7\p@\hbox{.}}\mkern2mu
\raise4\p@\hbox{.}\mkern2mu\raise7\p@\hbox{.}\mkern1mu}}
\makeatother

\newcommand{\ssincl}{\reflectbox{\rotatebox[origin=c]{45}{$\subseteq$}}}
\newcommand{\vsupseteq}{\reflectbox{\rotatebox[origin=c]{-90}{$\supseteq$}}}
\newcommand{\vin}{\reflectbox{\rotatebox[origin=c]{90}{$\in$}}}

\newcommand{\Ga}{\mathbb{G}_\mathrm{a}}
\newcommand{\Gm}{\mathbb{G}_\mathrm{m}}

\renewcommand{\U}{\UUU}

\DeclareRobustCommand{\Stirling}{\genfrac\{\}{0pt}{}}
\DeclareRobustCommand{\stirling}{\genfrac[]{0pt}{}}

% % % % % % % % % % % % % % % % % % % % % % % % % % % % % %
% tikz

\usepackage{tikz-cd}
\usetikzlibrary{babel}
\usetikzlibrary{decorations.pathmorphing}
\usetikzlibrary{arrows}
\usetikzlibrary{calc}
\usetikzlibrary{fit}
\usetikzlibrary{hobby}

% % % % % % % % % % % % % % % % % % % % % % % % % % % % % %
% Banners

\newcommand\mybannerext[3]{{\normalfont\sffamily\bfseries\large\noindent #1

\noindent #2

\noindent #3

}\noindent\rule{\textwidth}{1.25pt}

\vspace{1em}}

\newcommand\mybanner[2]{{\normalfont\sffamily\bfseries\large\noindent #1

\noindent #2

}\noindent\rule{\textwidth}{1.25pt}

\vspace{1em}}

\renewcommand{\O}{\mathcal{O}}


\numberwithin{equation}{section}

\usepackage[numbers]{natbib}

\usepackage[
  top=2cm,
  bottom=2cm,
  left=3cm,
  right=2cm,
  marginparwidth=1.5cm,
  headheight=17pt,
  includehead,includefoot,
  heightrounded,
]{geometry}

\hypersetup{
  pdftitle = {Teorema fundamental del álgebra},
  pdfauthor = {Alexey Beshenov (cadadr@gmail.com)},
  pdfdisplaydoctitle = true
}

\author{Alexey Beshenov (cadadr@gmail.com)}
\title{Teorema fundamental del álgebra}
\date{Universidad de El Salvador. 2018}

\begin{document}

{\normalfont\sffamily\bfseries \maketitle}

El propósito de esta nota es demostrar el
\term{teorema fundamental del álgebra}.

\begin{teorema}
  \index{teorema!fundamental del álgebra}
  Sea $f \in \CC [X]$ un polinomio no constante. Entonces, existe $z \in \CC$
  tal que ${f (z) = 0}$.
\end{teorema}

Este resultado aparece en el tratado de d'Alembert\footnote{\personality{Jean le
    Rond D'Alembert} (1717--1783), matemático, filósofo y enciclopedista
  francés, conocido por sus contribuciones en análisis, particularmente las
  ecuaciones diferenciales.} ``Recherches sur le calcul intégral'' (1748) y fue
probado de manera rigurosa en la tesis de doctorado de Gauss, publicada en
1799. El nombre ``teorema fundamental del álgebra'' parece un poco ridículo en
un curso del álgebra moderna, pero es histórico y bastante común. Sin duda,
es uno de los resultados más importantes en las matemáticas.

Aunque el estudio de raíces de polinomios pertenece al terreno del álgebra,
la misma construcción de los números complejos es analítica y por ende cualquier
prueba del teorema fundamental del álgebra debe usar análisis. Para una prueba
estándar puramente analítica, refiero a \cite[\S 3.3]{Vinberg-2003}.
El argumento de abajo es \emph{topológico}, basado implícitamente en
el \term{grupo fundamental} del círculo. Para más detalles, véase
\cite[Chapter~1]{May-1999-Concise}.

% % % % % % % % % % % % % % % % % % % % % % % % % % % % % %

% HTML: Grado de aplicación <strong>S</strong><sup>1</sup> → <strong>S</strong><sup>1</sup>
\section{Grado de aplicación $\SS^1 \to \SS^1$}

Consideremos el circulo unitario en el plano complejo
$$\SS^1 \dfn \{ z\in \CC \mid |z| = 1 \}$$
y la aplicación
$$\exp\colon \RR \to \SS^1, \quad x \mapsto e^{2\pi i x}.$$

\begin{figure}[h]
  \begin{center}
    \includegraphics{helice.mps}
  \end{center}

  \caption{La aplicación $x \mapsto e^{2\pi i x}$}
\end{figure}

Esta es una aplicación continua y es un homomorfismo sobreyectivo de grupos
$(\RR,+)$ y $(\SS^1,\times)$. Su núcleo es precisamente $\ZZ \subset \RR$:
$$\exp (x) = e^{2\pi i x} = 1 \iff x \in \ZZ.$$
Para obtener una aplicación inyectiva, se puede restringir $\exp$ al intervalo
abierto $\left(-\frac{1}{2}, +\frac{1}{2}\right)$. Como se sabe del análisis
complejo, esta restricción tiene una aplicación inversa
$$\log\colon \SS^1 \setminus \{ -1 \} \to \left(-\frac{1}{2}, +\frac{1}{2}\right)$$
que es también continua.

\begin{nameless}\textbf{Lema del levantamiento}.
  \index{lema!del levantamiento}
  \label{lema-del-levantamiento}
  \emph{Para toda aplicación continua $f\colon [0,1]^n \to \SS^1$ con
    $f (\underline{0}) = 1$ existe una aplicación continua
    $\widetilde{f}\colon [0,1]^n \to \RR$ que satisface
    $$\exp (\widetilde{f} (\underline{x})) = f (\underline{x}), \quad \widetilde{f} (\underline{0}) = 0.$$
    Además, estas condiciones definen a $\widetilde{f}$ de modo único.}

  \[ \begin{tikzcd}
      & \RR \ar{d}{\exp} && 0\ar[|->]{d} \\
      {[0,1]^n}\ar{r}[swap]{f}\ar[dashed]{ur}{\widetilde{f}} & \SS^1 & \underline{0} \ar[|->]{r}\ar[|->]{ur} & 1
    \end{tikzcd} \]

  \begin{proof}
    Primero, probemos la unicidad de $\widetilde{f}$. Supongamos que hay dos
    aplicaciones continuas $\widetilde{f}_1$ y $\widetilde{f}_2$ que cumplen
    \[ \exp (\widetilde{f}_1 (\underline{x})) = \exp (\widetilde{f}_2 (\underline{x})) = f (\underline{x}), \quad
      \widetilde{f}_1 (\underline{0}) = \widetilde{f}_2 (\underline{0}) = 0. \]
    La primera ecuación implica que
    $$\widetilde{f}_1 (\underline{x}) - \widetilde{f}_2 (\underline{x}) \in \ZZ.$$
    Entonces, la aplicación
    $\widetilde{f}_1 (\underline{x}) - \widetilde{f}_2 (\underline{x})$ es
    continua sobre el conjunto conexo $[0,1]^n$ y toma valores enteros, pero
    esto significa que es constante. Luego, para cualquier
    $\underline{x} \in [0,1]^n$
    $$\widetilde{f}_1 (\underline{x}) - \widetilde{f}_2 (\underline{x}) = \widetilde{f}_1 (\underline{0}) - \widetilde{f}_2 (\underline{0}) = 0.$$

    \vspace{1em}

    Ahora tenemos que establecer la existencia de $\widetilde{f}$. La aplicación
    $f\colon [0,1]^n \to \SS^1$ es continua y el cubo $[0,1]^n$ es compacto,
    entonces $f$ es uniformemente continua. Gracias a esto, existe $\delta > 0$
    tal que
    \[ \|\underline{x} - \underline{y}\| < \delta \Longrightarrow
      |f (\underline{x}) - f (\underline{y})| < 2 \Longrightarrow
      f (\underline{x}) \ne - f (\underline{y}). \]
    Fijemos un número natural $N$ tal que
    $\frac{1}{N} \, \|\underline{x}\| < \delta$ para todo $x \in [0,1]^n$ (esto
    es posible gracias a la compacidad de $[0,1]^n$). Pongamos
    $$\widetilde{f} (\underline{x}) \dfn \sum_{0 \le k \le N-1} \log \left(\cfrac{f \left(\frac{k+1}{N}\,\underline{x}\right)}{f \left(\frac{k}{N}\,\underline{x}\right)}\right).$$
    Por nuestra elección de $N$, tenemos
    $$\cfrac{f \left(\frac{k+1}{N}\,\underline{x}\right)}{f \left(\frac{k}{N}\,\underline{x}\right)} \ne -1$$
    para ningún $\underline{x} \in [0,1]^n$, así que $\widetilde{f}$ es una
    aplicación continua bien definida. Luego,
    $$\exp (\widetilde{f} (\underline{x})) = \cfrac{f \left(\frac{1}{N}\,\underline{x}\right)}{f (\underline{0})} \, \cfrac{f \left(\frac{2}{N}\,\underline{x}\right)}{f \left(\frac{1}{N}\,\underline{x}\right)} \cdots \cfrac{f \left(\frac{N-1}{N}\,\underline{x}\right)}{f \left(\frac{N-2}{n}\,\underline{x}\right)}\,\cfrac{f (\underline{x})}{f \left(\frac{N-1}{N}\,\underline{x}\right)} = f (\underline{x}),$$
    y claramente,
    $$\widetilde{f} (\underline{0}) = \log (f (\underline{0})) = \log (1) = 0.$$
  \end{proof}
\end{nameless}

\begin{definicion}
  Un \term{lazo}\index{lazo} en $\SS^1$ es una aplicacion continua
  $\gamma\colon \SS^1 \to \SS^1$ que satisface $\gamma (1) = 1$.

  Una \term{homotopía}\index{homotopía} entre dos lazos $\gamma_0$ y $\gamma_1$
  es una aplicación continua
  $$h\colon [0,1]\times \SS^1 \to \SS^1$$
  tal que
  $$h (0,z) = \gamma_0 (z), \quad h (1,z) = \gamma_1 (z), \quad h (t,1) = 1$$
  para cualesquiera $t \in [0,1]$, $z \in \SS^1$.
\end{definicion}

En otras palabras, una homotopía define una familia de lazos
$\gamma_t\colon z \mapsto h (t,z)$ que dependen de manera continua del parámetro
$t \in [0,1]$.

\begin{definicion}
  Dado un lazo $\gamma\colon \SS^1 \to \SS^1$, consideremos la aplicación
  \begin{align*}
    f\colon [0,1] & \to \SS^1,\\
    x & \mapsto \gamma (\exp (x)).
  \end{align*}
  Ahora según \ref{lema-del-levantamiento}, existe una aplicación continua única
  $f\colon [0,1] \to \RR$ que satisface
  $$\exp (\widetilde{f} (x)) = f (x), \quad \widetilde{f} (0) = 0.$$
  En particular,
  $$\exp (\widetilde{f} (1)) = f (1) = \gamma (1) = 1,$$
  así que $\widetilde{f} (1) \in \ZZ$. El número $\widetilde{f} (1)$ se llama
  el \term{grado}\index{grado!de lazo} del lazo $\gamma$ y se denota por
  $\deg \gamma$.
\end{definicion}

Intuitivamente, $\deg \gamma$ nos dice cuántas vueltas en el sentido antihorario
da $\gamma$ alrededor del circulo $\SS^1$.

\begin{ejemplo}
  El lazo constante
  $$\gamma\colon \SS^1 \to \SS^1, \quad z \mapsto 1$$
  tiene grado nulo: en efecto, a este lazo corresponde la aplicación constante
  $$f\colon [0,1] \to \SS^1, \quad x \mapsto 1,$$
  que se levanta a la aplicación constante
  $$\widetilde{f}\colon [0,1] \to \RR, \quad x \mapsto 0.$$
\end{ejemplo}

\begin{ejemplo}
  Consideremos el lazo
  $$\gamma\colon \SS^1 \to \SS^1, \quad z \mapsto z^n.$$
  A este corresponde la aplicación
  $$f\colon [0,1] \to \SS^1, \quad x \mapsto e^{2\pi i n x}$$
  que se levanta a la aplicación
  $$\widetilde{f} (x) = nx.$$
  En efecto, tenemos
  \[ \exp (\widetilde{f} (x)) = e^{2\pi i n x} = f (x), \quad
    \widetilde{f} (0) = 0. \]
  Podemos concluir que
  $$\deg \gamma = \widetilde{f} (1) = n.$$
\end{ejemplo}

Es fácil convencerse intuitivamente que al deformar un lazo de manera continua,
el número de vueltas que este da alrededor del círculo $\SS^1$ no cambia. Esto
se refleja en el siguiente resultado.

\begin{lema}
  Si entre dos lazos $\gamma_0$ y $\gamma_1\colon \SS^1 \to \SS^1$ existe una
  homotopía, entonces
  $$\deg \gamma_0 = \deg \gamma_1.$$

  \begin{proof}
    Consideremos una homotopía
    $$h\colon [0,1]\times \SS^1 \to \SS^1$$
    tal que
    $$h (0,z) = \gamma_0 (z), \quad h (1,z) = \gamma_1 (z).$$
    Definamos una aplicación
    \begin{align*}
      f\colon [0,1] \times [0,1] & \to \SS^1,\\
      (t,x) & \mapsto h (t, \exp (x)).
    \end{align*}
    De nuevo, podemos invocar el lema del levantamiento
    \ref{lema-del-levantamiento} para concluir que existe una aplicación
    continua
    $$\widetilde{f}\colon [0,1]\times [0,1] \to \RR$$
    que satisface
    $$\exp (\widetilde{f} (t,x)) = f (t,x), \quad \widetilde{f} (0,0) = 0.$$
    En particular, se tiene
    $$\exp (\widetilde{f} (t,0)) = h (t,1) = 1,$$
    lo que implica que $\widetilde{f} (t,0) \in \ZZ$. La aplicación
    $t \mapsto \widetilde{f} (t,0)$ es continua, definida sobre el intervalo
    conexo $[0,1]$, y dado que sus valores son enteros, esta debe ser
    constante. Tenemos $\widetilde{f} (0,0) = 0$, de donde podemos concluir que
    $$\widetilde{f} (t,0) = 0\text{ para todo }t\in [0,1].$$
    Además,
    \[ \exp (\widetilde{f} (0,x)) = \gamma_0 (\exp (x)), \quad
      \exp (\widetilde{f} (1,x)) = \gamma_1 (\exp (x)). \]
    Esto significa que $\widetilde{f} (0,x)$ y $\widetilde{f} (0,y)$ son
    levantamientos de los lazos $\gamma_0$ y $\gamma_1$ respectivamente, y luego
    \[ \deg \gamma_0 = \widetilde{f} (0,1), \quad
      \deg \gamma_1 = \widetilde{f} (1,1). \]
    Notamos que
    $$\exp (\widetilde{f} (t,1)) = h (t, \exp (1)) = h (t, 1) = 1,$$
    así que $\widetilde{f} (t,1) \in \ZZ$. De nuevo, se tiene una aplicación
    continua $t \mapsto \widetilde{f} (t,1)$ definida sobre el intervalo conexo
    $[0,1]$ que toma valores enteros, entonces es constante. En particular,
    $$\deg \gamma_0 = \widetilde{f} (0,1) = \widetilde{f} (1,1) = \deg \gamma_1.$$
  \end{proof}
\end{lema}

\begin{comentario}
  De hecho, se puede probar que si $\deg \gamma_0 = \deg \gamma_1$, entonces
  entre los lazos $\gamma_0$ y $\gamma_1$ hay una homotopía. Sin embargo,
  no lo vamos a necesitar en la prueba de abajo.
\end{comentario}

\begin{comentario}
  Para una curva $\gamma\colon \SS^1 \to \CC \setminus \{ 0 \}$ se puede definir
  el grado (también conocido como el \term{índice} o \term{número de rotación})
  mediante la integral
  $$\deg \gamma = \frac{1}{2\pi i}\,\oint_\gamma \frac{dz}{z}.$$
  Con esta definición, para deducir que $\deg \gamma \in \ZZ$ y $\deg \gamma$ es
  invariante respecto a homotopía, se usa la
  \term{fórmula integral de Cauchy}. Véase por ejemplo
  \cite[Chapter III]{Lang-GTM-103} para los detalles.
\end{comentario}

% % % % % % % % % % % % % % % % % % % % % % % % % % % % % %

\section{Prueba del teorema}

Consideremos un polinomio complejo
$$f = z^n + a_{n-1}\,z^{n-1} + a_{n-2}\,z^{n-2} + \cdots + a_1\,z + a_0.$$
Asumamos que $f$ no tiene raíces: $f (z) \ne 0$ para ningún
$z \in \CC$. Definamos un lazo
\begin{align*}
  \gamma\colon \SS^1 & \to \SS^1,\\
  z & \mapsto \frac{f (z)}{|f (z)|}\,\frac{|f (1)|}{f (1)}.
\end{align*}
Para $t \in [0,1]$ pongamos
$$h_1 (t,z) \dfn \begin{cases}
  \cfrac{f (z/t)\,t^n}{|f (z/t)\,t^n|}\,\cfrac{|f (1/t)\,t^n|}{f (1/t)\,t^n}, & t \ne 0,\\
  z^n, & t = 0.
\end{cases}$$
Notamos que
$$\lim_{t\to 0} h_1 (t,z) = z^n$$
y
$$h_1 (1,z) = \gamma (z),$$
así que $h_1\colon [0,1] \to \SS^1$ es una aplicación continua que define una
homotopía entre el lazo $z \mapsto z^n$ y $\gamma$. Entonces,
$$\deg \gamma = \deg (z \mapsto z^n) = n.$$

Por otro lado, podemos definir
$$h_2 (t,z) \dfn \frac{f (tz)}{|f (tz)|}\,\frac{|f (t)|}{f (t)}.$$
Tenemos
$$h_2 (0,z) = \gamma (0) = 1, \quad h_2 (1,z) = \gamma (z),$$
así que $h_2$ define una homotopía entre el lazo constante $z \mapsto 1$ y
$\gamma$, así que
$$\deg \gamma = 0.$$
Entonces, si $n > 0$, tendríamos una contradicción. Podemos concluir que todo
polinomio complejo no constante debe tener una raíz. \qed

\bibliographystyle{../../amsalpha-cust}
{\small\bibliography{../../salvador}}

\end{document}
