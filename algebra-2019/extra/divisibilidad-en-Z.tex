\documentclass{article}

% TODO : CLEAN UP THIS MESS
% (AND MAKE SURE ALL TEXTS STILL COMPILE)
\usepackage[leqno]{amsmath}
\usepackage{amssymb}
\usepackage{graphicx}

\usepackage{diagbox} % table heads with diagonal lines
\usepackage{relsize}

\usepackage{wasysym}
\usepackage{scrextend}
\usepackage{epigraph}
\setlength\epigraphwidth{.6\textwidth}

\usepackage[utf8]{inputenc}

\usepackage{titlesec}
\titleformat{\chapter}[display]
  {\normalfont\sffamily\huge\bfseries}
  {\chaptertitlename\ \thechapter}{5pt}{\Huge}
\titleformat{\section}
  {\normalfont\sffamily\Large\bfseries}
  {\thesection}{1em}{}
\titleformat{\subsection}
  {\normalfont\sffamily\large\bfseries}
  {\thesubsection}{1em}{}
\titleformat{\part}[display]
  {\normalfont\sffamily\huge\bfseries}
  {\partname\ \thepart}{0pt}{\Huge}

\usepackage[T1]{fontenc}
\usepackage{fourier}
\usepackage{paratype}

\usepackage[symbol,perpage]{footmisc}

\usepackage{perpage}
\MakePerPage{footnote}

\usepackage{array}
\newcolumntype{x}[1]{>{\centering\hspace{0pt}}p{#1}}

% TODO: the following line causes conflict with new texlive (!)
% \usepackage[english,russian,polutonikogreek,spanish]{babel}
% \newcommand{\russian}[1]{{\selectlanguage{russian}#1}}

% Remove conflicting options for the moment:
\usepackage[english,polutonikogreek,spanish]{babel}

\AtBeginDocument{\shorthandoff{"}}
\newcommand{\greek}[1]{{\selectlanguage{polutonikogreek}#1}}

% % % % % % % % % % % % % % % % % % % % % % % % % % % % % %
% Limit/colimit symbols (with accented i: lím / colím)

\usepackage{etoolbox} % \patchcmd

\makeatletter
\patchcmd{\varlim@}{lim}{\lim}{}{}
\makeatother
\DeclareMathOperator*{\colim}{co{\lim}}
\newcommand{\dirlim}{\varinjlim}
\newcommand{\invlim}{\varprojlim}

% % % % % % % % % % % % % % % % % % % % % % % % % % % % % %

\usepackage[all,color]{xy}

\usepackage{pigpen}
\newcommand{\po}{\ar@{}[dr]|(.4){\text{\pigpenfont I}}}
\newcommand{\pb}{\ar@{}[dr]|(.3){\text{\pigpenfont A}}}
\newcommand{\polr}{\ar@{}[dr]|(.65){\text{\pigpenfont A}}}
\newcommand{\pour}{\ar@{}[ur]|(.65){\text{\pigpenfont G}}}
\newcommand{\hstar}{\mathop{\bigstar}}

\newcommand{\bigast}{\mathop{\Huge \mathlarger{\mathlarger{\ast}}}}

\newcommand{\term}{\textbf}

\usepackage{stmaryrd}

\usepackage{cancel}

\usepackage{tikzsymbols}

\newcommand{\open}{\underset{\mathrm{open}}{\hookrightarrow}}
\newcommand{\closed}{\underset{\mathrm{closed}}{\hookrightarrow}}

\newcommand{\tcol}[2]{{#1 \choose #2}}

\newcommand{\homot}{\simeq}
\newcommand{\isom}{\cong}
\newcommand{\cH}{\mathcal{H}}
\renewcommand{\hom}{\mathrm{hom}}
\renewcommand{\div}{\mathop{\mathrm{div}}}
\renewcommand{\Im}{\mathop{\mathrm{Im}}}
\renewcommand{\Re}{\mathop{\mathrm{Re}}}
\newcommand{\id}[1]{\mathrm{id}_{#1}}
\newcommand{\idid}{\mathrm{id}}

\newcommand{\ZG}{{\ZZ G}}
\newcommand{\ZH}{{\ZZ H}}

\newcommand{\quiso}{\simeq}

\newcommand{\personality}[1]{{\sc #1}}

\newcommand{\mono}{\rightarrowtail}
\newcommand{\epi}{\twoheadrightarrow}
\newcommand{\xepi}[1]{\xrightarrow{#1}\mathrel{\mkern-14mu}\rightarrow}

% % % % % % % % % % % % % % % % % % % % % % % % % % % % % %

\DeclareMathOperator{\Ad}{Ad}
\DeclareMathOperator{\Aff}{Aff}
\DeclareMathOperator{\Ann}{Ann}
\DeclareMathOperator{\Aut}{Aut}
\DeclareMathOperator{\Br}{Br}
\DeclareMathOperator{\CH}{CH}
\DeclareMathOperator{\Cl}{Cl}
\DeclareMathOperator{\Coeq}{Coeq}
\DeclareMathOperator{\Coind}{Coind}
\DeclareMathOperator{\Cop}{Cop}
\DeclareMathOperator{\Corr}{Corr}
\DeclareMathOperator{\Cor}{Cor}
\DeclareMathOperator{\Cov}{Cov}
\DeclareMathOperator{\Der}{Der}
\DeclareMathOperator{\Div}{Div}
\DeclareMathOperator{\D}{D}
\DeclareMathOperator{\Ehr}{Ehr}
\DeclareMathOperator{\End}{End}
\DeclareMathOperator{\Eq}{Eq}
\DeclareMathOperator{\Ext}{Ext}
\DeclareMathOperator{\Frac}{Frac}
\DeclareMathOperator{\Frob}{Frob}
\DeclareMathOperator{\Funct}{Funct}
\DeclareMathOperator{\Fun}{Fun}
\DeclareMathOperator{\GL}{GL}
\DeclareMathOperator{\Gal}{Gal}
\DeclareMathOperator{\Gr}{Gr}
\DeclareMathOperator{\Hol}{Hol}
\DeclareMathOperator{\Hom}{Hom}
\DeclareMathOperator{\Ho}{Ho}
\DeclareMathOperator{\Id}{Id}
\DeclareMathOperator{\Ind}{Ind}
\DeclareMathOperator{\Inn}{Inn}
\DeclareMathOperator{\Isom}{Isom}
\DeclareMathOperator{\Ker}{Ker}
\DeclareMathOperator{\Lan}{Lan}
\DeclareMathOperator{\Lie}{Lie}
\DeclareMathOperator{\Map}{Map}
\DeclareMathOperator{\Mat}{Mat}
\DeclareMathOperator{\Max}{Max}
\DeclareMathOperator{\Mor}{Mor}
\DeclareMathOperator{\Nat}{Nat}
\DeclareMathOperator{\Nrd}{Nrd}
\DeclareMathOperator{\Ob}{Ob}
\DeclareMathOperator{\Out}{Out}
\DeclareMathOperator{\PGL}{PGL}
\DeclareMathOperator{\PSL}{PSL}
\DeclareMathOperator{\PSU}{PSU}
\DeclareMathOperator{\Pic}{Pic}
\DeclareMathOperator{\RHom}{RHom}
\DeclareMathOperator{\Rad}{Rad}
\DeclareMathOperator{\Ran}{Ran}
\DeclareMathOperator{\Rep}{Rep}
\DeclareMathOperator{\Res}{Res}
\DeclareMathOperator{\SL}{SL}
\DeclareMathOperator{\SO}{SO}
\DeclareMathOperator{\SU}{SU}
\DeclareMathOperator{\Sh}{Sh}
\DeclareMathOperator{\Sing}{Sing}
\DeclareMathOperator{\Specm}{Specm}
\DeclareMathOperator{\Spec}{Spec}
\DeclareMathOperator{\Sp}{Sp}
\DeclareMathOperator{\Stab}{Stab}
\DeclareMathOperator{\Sym}{Sym}
\DeclareMathOperator{\Tors}{Tors}
\DeclareMathOperator{\Tor}{Tor}
\DeclareMathOperator{\Tot}{Tot}
\DeclareMathOperator{\UUU}{U}

\DeclareMathOperator{\adj}{adj}
\DeclareMathOperator{\ad}{ad}
\DeclareMathOperator{\af}{af}
\DeclareMathOperator{\card}{card}
\DeclareMathOperator{\cm}{cm}
\DeclareMathOperator{\codim}{codim}
\DeclareMathOperator{\cod}{cod}
\DeclareMathOperator{\coeq}{coeq}
\DeclareMathOperator{\coim}{coim}
\DeclareMathOperator{\coker}{coker}
\DeclareMathOperator{\cont}{cont}
\DeclareMathOperator{\conv}{conv}
\DeclareMathOperator{\cor}{cor}
\DeclareMathOperator{\depth}{depth}
\DeclareMathOperator{\diag}{diag}
\DeclareMathOperator{\diam}{diam}
\DeclareMathOperator{\dist}{dist}
\DeclareMathOperator{\dom}{dom}
\DeclareMathOperator{\eq}{eq}
\DeclareMathOperator{\ev}{ev}
\DeclareMathOperator{\ex}{ex}
\DeclareMathOperator{\fchar}{char}
\DeclareMathOperator{\fr}{fr}
\DeclareMathOperator{\gr}{gr}
\DeclareMathOperator{\im}{im}
\DeclareMathOperator{\infl}{inf}
\DeclareMathOperator{\interior}{int}
\DeclareMathOperator{\intrel}{intrel}
\DeclareMathOperator{\inv}{inv}
\DeclareMathOperator{\length}{length}
\DeclareMathOperator{\mcd}{mcd}
\DeclareMathOperator{\mcm}{mcm}
\DeclareMathOperator{\multideg}{multideg}
\DeclareMathOperator{\ord}{ord}
\DeclareMathOperator{\pr}{pr}
\DeclareMathOperator{\rel}{rel}
\DeclareMathOperator{\res}{res}
\DeclareMathOperator{\rkred}{rkred}
\DeclareMathOperator{\rkss}{rkss}
\DeclareMathOperator{\rk}{rk}
\DeclareMathOperator{\sgn}{sgn}
\DeclareMathOperator{\sk}{sk}
\DeclareMathOperator{\supp}{supp}
\DeclareMathOperator{\trdeg}{trdeg}
\DeclareMathOperator{\tr}{tr}
\DeclareMathOperator{\vol}{vol}

\newcommand{\iHom}{\underline{\Hom}}

\renewcommand{\AA}{\mathbb{A}}
\newcommand{\CC}{\mathbb{C}}
\renewcommand{\SS}{\mathbb{S}}
\newcommand{\TT}{\mathbb{T}}
\newcommand{\PP}{\mathbb{P}}
\newcommand{\BB}{\mathbb{B}}
\newcommand{\RR}{\mathbb{R}}
\newcommand{\ZZ}{\mathbb{Z}}
\newcommand{\FF}{\mathbb{F}}
\newcommand{\HH}{\mathbb{H}}
\newcommand{\NN}{\mathbb{N}}
\newcommand{\QQ}{\mathbb{Q}}
\newcommand{\KK}{\mathbb{K}}

% % % % % % % % % % % % % % % % % % % % % % % % % % % % % %

\usepackage{amsthm}

\newcommand{\legendre}[2]{\left(\frac{#1}{#2}\right)}

\newcommand{\examplesymbol}{$\blacktriangle$}
\renewcommand{\qedsymbol}{$\blacksquare$}

\newcommand{\dfn}{\mathrel{\mathop:}=}
\newcommand{\rdfn}{=\mathrel{\mathop:}}

\usepackage{xcolor}
\definecolor{mylinkcolor}{rgb}{0.0,0.4,1.0}
\definecolor{mycitecolor}{rgb}{0.0,0.4,1.0}
\definecolor{shadecolor}{rgb}{0.79,0.78,0.65}
\definecolor{gray}{rgb}{0.6,0.6,0.6}

\usepackage{colortbl}

\definecolor{myred}{rgb}{0.7,0.0,0.0}
\definecolor{mygreen}{rgb}{0.0,0.7,0.0}
\definecolor{myblue}{rgb}{0.0,0.0,0.7}

\definecolor{redshade}{rgb}{0.9,0.5,0.5}
\definecolor{greenshade}{rgb}{0.5,0.9,0.5}

\usepackage[unicode,colorlinks=true,linkcolor=mylinkcolor,citecolor=mycitecolor]{hyperref}
\newcommand{\refref}[2]{\hyperref[#2]{#1~\ref*{#2}}}
\newcommand{\eqnref}[1]{\hyperref[#1]{(\ref*{#1})}}

\newcommand{\tos}{\!\!\to\!\!}

\usepackage{framed}

\newcommand{\cequiv}{\simeq}

\makeatletter
\newcommand\xleftrightarrow[2][]{%
  \ext@arrow 9999{\longleftrightarrowfill@}{#1}{#2}}
\newcommand\longleftrightarrowfill@{%
  \arrowfill@\leftarrow\relbar\rightarrow}
\makeatother

\newcommand{\bsquare}{\textrm{\ding{114}}}

% % % % % % % % % % % % % % % % % % % % % % % % % % % % % %

\newtheoremstyle{myplain}
  {\topsep}   % ABOVESPACE
  {\topsep}   % BELOWSPACE
  {\itshape}  % BODYFONT
  {0pt}       % INDENT (empty value is the same as 0pt)
  {\bfseries} % HEADFONT
  {.}         % HEADPUNCT
  {5pt plus 1pt minus 1pt} % HEADSPACE
  {\thmnumber{#2}. \thmname{#1}\thmnote{ (#3)}}   % CUSTOM-HEAD-SPEC

\newtheoremstyle{myplainnameless}
  {\topsep}   % ABOVESPACE
  {\topsep}   % BELOWSPACE
  {\normalfont}  % BODYFONT
  {0pt}       % INDENT (empty value is the same as 0pt)
  {\bfseries} % HEADFONT
  {.}         % HEADPUNCT
  {5pt plus 1pt minus 1pt} % HEADSPACE
  {\thmnumber{#2}}   % CUSTOM-HEAD-SPEC 

\newtheoremstyle{sectionexercise}
  {\topsep}   % ABOVESPACE
  {\topsep}   % BELOWSPACE
  {\normalfont}  % BODYFONT
  {0pt}       % INDENT (empty value is the same as 0pt)
  {\bfseries} % HEADFONT
  {.}         % HEADPUNCT
  {5pt plus 1pt minus 1pt} % HEADSPACE
  {Ejercicio \thmnumber{#2}\thmnote{ (#3)}}   % CUSTOM-HEAD-SPEC

\newtheoremstyle{mydefinition}
  {\topsep}   % ABOVESPACE
  {\topsep}   % BELOWSPACE
  {\normalfont}  % BODYFONT
  {0pt}       % INDENT (empty value is the same as 0pt)
  {\bfseries} % HEADFONT
  {.}         % HEADPUNCT
  {5pt plus 1pt minus 1pt} % HEADSPACE
  {\thmnumber{#2}. \thmname{#1}\thmnote{ (#3)}}   % CUSTOM-HEAD-SPEC

% EN ESPAÑOL

\newtheorem*{hecho*}{Hecho}
\newtheorem*{corolario*}{Corolario}
\newtheorem*{teorema*}{Teorema}
\newtheorem*{conjetura*}{Conjetura}
\newtheorem*{proyecto*}{Proyecto}
\newtheorem*{observacion*}{Observación}

\newtheorem*{lema*}{Lema}
\newtheorem*{resultado-clave*}{Resultado clave}
\newtheorem*{proposicion*}{Proposición}

\theoremstyle{definition}
\newtheorem*{ejercicio*}{Ejercicio}
\newtheorem*{definicion*}{Definición}
\newtheorem*{comentario*}{Comentario}
\newtheorem*{definicion-alternativa*}{Definición alternativa}
\newtheorem*{ejemploxs}{Ejemplo}
\newenvironment{ejemplo*}
  {\pushQED{\qed}\renewcommand{\qedsymbol}{\examplesymbol}\ejemploxs}
  {\popQED\endejemploxs}

\theoremstyle{myplain}
\newtheorem{proposicion}{Proposición}[section]

\newtheorem{proyecto}[proposicion]{Proyecto}
\newtheorem{teorema}[proposicion]{Teorema}
\newtheorem{corolario}[proposicion]{Corolario}
\newtheorem{hecho}[proposicion]{Hecho}
\newtheorem{lema}[proposicion]{Lema}

\newtheorem{observacion}[proposicion]{Observación}

\newenvironment{observacionejerc}
    {\pushQED{\qed}\renewcommand{\qedsymbol}{$\square$}\csname inner@observacionejerc\endcsname}
    {\popQED\csname endinner@observacionejerc\endcsname}
\newtheorem{inner@observacionejerc}[proposicion]{Observación}

\newenvironment{proposicionejerc}
    {\pushQED{\qed}\renewcommand{\qedsymbol}{$\square$}\csname inner@proposicionejerc\endcsname}
    {\popQED\csname endinner@proposicionejerc\endcsname}
\newtheorem{inner@proposicionejerc}[proposicion]{Proposicion}

\newenvironment{lemaejerc}
    {\pushQED{\qed}\renewcommand{\qedsymbol}{$\square$}\csname inner@lemaejerc\endcsname}
    {\popQED\csname endinner@lemaejerc\endcsname}
\newtheorem{inner@lemaejerc}[proposicion]{Lema}

\newtheorem{calculo}[proposicion]{Cálculo}

\theoremstyle{myplainnameless}
\newtheorem{nameless}[proposicion]{}

\theoremstyle{mydefinition}
\newtheorem{comentario}[proposicion]{Comentario}
\newtheorem{comentarioast}[proposicion]{Comentario ($\clubsuit$)}
\newtheorem{construccion}[proposicion]{Construcción}
\newtheorem{aplicacion}[proposicion]{Aplicación}
\newtheorem{definicion}[proposicion]{Definición}
\newtheorem{definicion-alternativa}[proposicion]{Definición alternativa}
\newtheorem{notacion}[proposicion]{Notación}
\newtheorem{advertencia}[proposicion]{Advertencia}
\newtheorem{digresion}[proposicion]{Digresión}
\newtheorem{ejemplox}[proposicion]{Ejemplo}
\newenvironment{ejemplo}
  {\pushQED{\qed}\renewcommand{\qedsymbol}{\examplesymbol}\ejemplox}
  {\popQED\endejemplox}
\newtheorem{contraejemplox}[proposicion]{Contraejemplo}
\newenvironment{contraejemplo}
  {\pushQED{\qed}\renewcommand{\qedsymbol}{\examplesymbol}\contraejemplox}
  {\popQED\endcontraejemplox}
\newtheorem{noejemplox}[proposicion]{No-ejemplo}
\newenvironment{noejemplo}
  {\pushQED{\qed}\renewcommand{\qedsymbol}{\examplesymbol}\noejemplox}
  {\popQED\endnoejemplox}
 
\newtheorem{ejemploastx}[proposicion]{Ejemplo ($\clubsuit$)}
\newenvironment{ejemploast}
  {\pushQED{\qed}\renewcommand{\qedsymbol}{\examplesymbol}\ejemploastx}
  {\popQED\endejemploastx}

\ifdefined\exercisespersection
  \theoremstyle{sectionexercise}
  \newtheorem{ejercicio}{}[section]
  \theoremstyle{mydefinition}
\else
  \ifdefined\exercisesglobal
    \theoremstyle{sectionexercise}
    \newtheorem{ejercicio}{}
    \theoremstyle{mydefinition}
  \else
    \ifdefined\exercisespersection
      \newtheorem{ejercicio}[proposicion]{Ejercicio}
    \fi
  \fi
\fi

% % % % % % % % % % % % % % % % % % % % % % % % % % % % % %

\theoremstyle{myplain}
\newtheorem{proposition}{Proposition}[section]
\newtheorem*{fact*}{Fact}
\newtheorem*{proposition*}{Proposition}
\newtheorem{lemma}[proposition]{Lemma}
\newtheorem*{lemma*}{Lemma}

\newtheorem{exercise}{Exercise}
\newtheorem*{hint}{Hint}

\newtheorem{theorem}[proposition]{Theorem}
\newtheorem{conjecture}[proposition]{Conjecture}
\newtheorem*{theorem*}{Theorem}
\newtheorem{corollary}[proposition]{Corollary}
\newtheorem{fact}[proposition]{Fact}
\newtheorem*{claim}{Claim}
\newtheorem{definition-theorem}[proposition]{Definition-theorem}

\theoremstyle{mydefinition}
\newtheorem{examplex}[proposition]{Example}
\newenvironment{example}
  {\pushQED{\qed}\renewcommand{\qedsymbol}{\examplesymbol}\examplex}
  {\popQED\endexamplex}

\newtheorem*{examplexx}{Example}
\newenvironment{example*}
  {\pushQED{\qed}\renewcommand{\qedsymbol}{\examplesymbol}\examplexx}
  {\popQED\endexamplexx}

\newtheorem{definition}[proposition]{Definition}
\newtheorem*{definition*}{Definition}
\newtheorem{wrong-definition}[proposition]{Wrong definition}
\newtheorem{remark}[proposition]{Remark}

\makeatletter
\newcommand{\xRightarrow}[2][]{\ext@arrow 0359\Rightarrowfill@{#1}{#2}}
\makeatother

% % % % % % % % % % % % % % % % % % % % % % % % % % % % % %

\newcommand{\Et}{\mathop{\text{\rm Ét}}}

\newcommand{\categ}[1]{\text{\bf #1}}
\newcommand{\vcateg}{\mathcal}
\newcommand{\bone}{{\boldsymbol 1}}
\newcommand{\bDelta}{{\boldsymbol\Delta}}
\newcommand{\bR}{{\mathbf{R}}}

\newcommand{\univ}{\mathfrak}

\newcommand{\TODO}{\colorbox{red}{\textbf{*** TODO ***}}}
\newcommand{\proofreadme}{\colorbox{red}{\textbf{*** NEEDS PROOFREADING ***}}}

\makeatletter
\def\iddots{\mathinner{\mkern1mu\raise\p@
\vbox{\kern7\p@\hbox{.}}\mkern2mu
\raise4\p@\hbox{.}\mkern2mu\raise7\p@\hbox{.}\mkern1mu}}
\makeatother

\newcommand{\ssincl}{\reflectbox{\rotatebox[origin=c]{45}{$\subseteq$}}}
\newcommand{\vsupseteq}{\reflectbox{\rotatebox[origin=c]{-90}{$\supseteq$}}}
\newcommand{\vin}{\reflectbox{\rotatebox[origin=c]{90}{$\in$}}}

\newcommand{\Ga}{\mathbb{G}_\mathrm{a}}
\newcommand{\Gm}{\mathbb{G}_\mathrm{m}}

\renewcommand{\U}{\UUU}

\DeclareRobustCommand{\Stirling}{\genfrac\{\}{0pt}{}}
\DeclareRobustCommand{\stirling}{\genfrac[]{0pt}{}}

% % % % % % % % % % % % % % % % % % % % % % % % % % % % % %
% tikz

\usepackage{tikz-cd}
\usetikzlibrary{babel}
\usetikzlibrary{decorations.pathmorphing}
\usetikzlibrary{arrows}
\usetikzlibrary{calc}
\usetikzlibrary{fit}
\usetikzlibrary{hobby}

% % % % % % % % % % % % % % % % % % % % % % % % % % % % % %
% Banners

\newcommand\mybannerext[3]{{\normalfont\sffamily\bfseries\large\noindent #1

\noindent #2

\noindent #3

}\noindent\rule{\textwidth}{1.25pt}

\vspace{1em}}

\newcommand\mybanner[2]{{\normalfont\sffamily\bfseries\large\noindent #1

\noindent #2

}\noindent\rule{\textwidth}{1.25pt}

\vspace{1em}}

\renewcommand{\O}{\mathcal{O}}


\numberwithin{equation}{section}

\usepackage[numbers]{natbib}

\usepackage[
  top=2cm,
  bottom=2cm,
  left=3cm,
  right=2cm,
  marginparwidth=1.5cm,
  headheight=17pt,
  includehead,includefoot,
  heightrounded,
]{geometry}

\hypersetup{
  pdftitle = {Divisibilidad en Z},
  pdfauthor = {Alexey Beshenov (cadadr@gmail.com)},
  pdfdisplaydoctitle = true
}

\author{Alexey Beshenov (cadadr@gmail.com)}
\title{Divisibilidad en $\ZZ$}
\date{Universidad de El Salvador. 2018}

\begin{document}

{\normalfont\sffamily\bfseries \maketitle}

El propósito de esta nota es revisar algunas nociones básicas de la teoría de
números elemental.

% % % % % % % % % % % % % % % % % % % % % % % % % % % % % %

\section{Subgrupos de $\ZZ$}

Un \term{subgrupo} $A\subset \ZZ$ es un subconjunto de números enteros que
satisface las siguientes condiciones:

\begin{enumerate}
\item[1)] $0 \in A$,

\item[2)] para cualesquiera $a,b\in A$ tenemos $a+b \in A$,

\item[3)] para cualquier $a\in A$ tenemos $-a\in A$.
\end{enumerate}

\begin{observacion}
  Si $A$ y $B$ son dos subgrupos de $\ZZ$, entonces su intersección $A\cap B$ es
  también un subgrupo.
\end{observacion}

Para $a_1,\ldots,a_n \in \ZZ$ el \term{subgrupo generado} por $a_1,\ldots,a_n$
es el subconjunto $\langle a_1,\ldots,a_n\rangle \subseteq \ZZ$ que satisface
una de las siguientes condiciones equivalentes.

\begin{enumerate}
\item[1)] $\langle a_1,\ldots,a_n\rangle$ es el mínimo subgrupo de $\ZZ$ que
  contiene todos los números $a_1,\ldots,a_n$,

\item[2)] $\langle a_1,\ldots,a_n\rangle$ es el conjunto de las combinaciones
  $\ZZ$-lineales de $a_1,\ldots,a_n$:
$$\langle a_1,\ldots,a_n\rangle = \Bigl\{ \sum_i n_i\,a_i \Bigm| n_i \in \ZZ \Bigr\}.$$
\end{enumerate}

Nos van a interesar dos casos particulares: los subgrupos generados por un
número $d\in \ZZ$:
$$\langle d\rangle = \{ md \mid m\in\ZZ \}$$
y subgrupos generados por dos números:
$$\langle a,b\rangle = \{ ma + nb \mid m,n\in\ZZ \}.$$

% % % % % % % % % % % % % % % % % % % % % % % % % % % % % %

\section{División con resto}

\begin{teorema}[Euclides]
  Sean $a,b\in\ZZ$ dos números enteros, con $b \ne 0$. Entonces, existen
  $q,r\in\ZZ$ tales que
  $$a = qb + r, \quad 0 \le r < |b|.$$

  \begin{proof}
    Para el conjunto
    $$\{ a - xb \mid x\in \ZZ \}$$
    sea
    $$r = a - qb$$
    su mínimo elemento tal que $r \ge 0$ (este existe, puesto que
    $b\ne 0$). Supongamos que $r \ge |b|$. Si $b > 0$, tenemos
    $$0 \le r - b = a - qb - b = a - (q+1)\,b < r.$$
    De la misma manera, si $b < 0$, entonces
    $$0 \le r + b = a - qb + b = a - (q-1)\,b < r.$$
    En ambos casos se produce un elemento $a - (q\pm 1)\,b$, lo que contradice
    nuestra elección de $r$. Podemos concluir que $r < |b|$.
  \end{proof}
\end{teorema}

El resultado que acabamos de describir se llama
la \term{división con resto}\index{división con resto} de $a$ por $b$.
He aquí una de sus consecuencias importantes.

\begin{proposicion}
  \label{prop:Z-es-un-dominio-de-ideales-principales}
  Todo subgrupo de $\ZZ$ es de la forma $\langle d\rangle$ para algún
  $d\in \ZZ$. En particular, para cualesquiera $a,b\in\ZZ$ se tiene

  \begin{enumerate}
  \item[1)] $\langle a,b\rangle = \langle d\rangle$ para algún $d\in\ZZ$,

  \item[2)] $\langle a\rangle \cap \langle b\rangle = \langle d\rangle$ para
    algún $d\in\ZZ$.
  \end{enumerate}

  \begin{proof}
    Sea $A\subseteq \ZZ$ un subgrupo. Si $A = 0$, entonces
    $A = \langle 0\rangle$ y enunciado es trivial. Luego, si $A \ne 0$, entonces
    $A$ contiene números no nulos. Para cada $x\in A$ también $-x\in A$, así que
    $A$ contiene números positivos. Sea entonces
    $$d \dfn \min \{ x\in A \mid x > 0 \}.$$
    Está claro que $\langle d\rangle \subseteq A$. Para ver la otra inclusión,
    consideremos un elemento arbitrario $c \in A$. La división con resto por $d$
    nos da
    $$c = qd + r, \quad 0 \le r < d.$$
    Luego, puesto que $c, d \in A$, tenemos también $r = c-qd \in A$.
    Por nuestra elección de $d$, podemos descartar el caso
    $0 < r < d$. Entonces, $r = 0$ y $c = qd \in \langle d\rangle$.
  \end{proof}
\end{proposicion}

% % % % % % % % % % % % % % % % % % % % % % % % % % % % % %

\section{Divisibilidad y los números primos}

\begin{definicion}
  Para dos números enteros $d,n\in \ZZ$ se dice que
  \term{$d$ divide a $n$}\index{divisibilidad} y se escribe ``$d\mid n$'' si
  $n = mx$ para algún $m\in \ZZ$. En este caso también se dice que $d$ es un
  \term{divisor} de $n$ o que $n$ es \term{divisible por $d$}. Cuando $d$ no
  divide a $n$, se escribe ``$d\nmid n$''.
\end{definicion}

Notamos que en términos de subgrupos de $\ZZ$,
$$d\mid n \iff \langle n\rangle \subseteq \langle d\rangle.$$

El lector puede comprobar las siguientes propiedades de la relación de
divisibilidad.

\begin{enumerate}
\item[0)] $a\mid 0$ para todo\footnote{Algunas fuentes insisten que $0\nmid 0$,
    pero la relación $0\mid 0$ no tiene nada de malo. De hecho
    $0\in \langle d\rangle$ para cualquier $d\in\ZZ$, en particular para
    $d = 0$.} $a\in\ZZ$. Esto caracteriza a $0$ de modo único. Tenemos
  $0 \mid a$ solamente para $a = 0$.

\item[1)] $a\mid a$ y $\pm 1\mid a$ para todo $a\in\ZZ$.

\item[2)] Si $a\mid b$ y $b\mid a$, entonces $a = \pm b$.

\item[3)] Si $a\mid b$ y $b\mid c$, entonces $a\mid c$.

\item[4)] Si $a\mid b$, entonces $a\mid bc$ para cualquier $c\in \ZZ$.

\item[5)] Si $a\mid b$ y $a\mid c$, entonces $a\mid (b+c)$.
\end{enumerate}

\begin{definicion}
  Se dice que un número entero positivo $p > 0$ es
  \term{primo}\index{número!primos} si $p\ne 1$ y los únicos divisores de $p$
  son $\pm 1$ y $\pm p$.
\end{definicion}

En otras palabras, $p$ es primo si y solamente si para $m,n > 0$, si tenemos
$p = mn$, entonces o bien $m = p$, $n = 1$ o bien $m = 1$, $n = p$. Los primeros
números primos son
$$2, ~ 3, ~ 5, ~ 7, ~ 11, ~ 13, ~ 17, ~ 19, ~ 23, ~ 29, ~ 31, ~ 37, ~ 41, ~ 43, ~ 47, ~ 53, ~ 59, ~ 61, ~ 67, ~ 71, ~ 73, ~ 79, ~ 83, ~ 89, ~ 97, ~ \ldots$$
Por ejemplo, $57 = 3\cdot 19$ no es primo.

\begin{proposicion}
  \label{prop:todo-entero-es-producto-de-primos}
  Todo entero no nulo puede ser expresado como
  $$n = \pm p_1^{k_1}\,p_2^{k_2}\cdots p_\ell^{k_\ell}$$
  donde $p_i$ son primos diferentes.

  \begin{proof}
    Sin pérdida de generalidad, podemos considerar el caso de $n > 0$. Sería
    suficiente ver que $n$ es un producto de primos y juntando múltiplos
    iguales, se obtiene la expresión de arriba.

    Para $n = 1$ tenemos $n = p^0$ para cualquier primo $p$. Luego, se puede
    proceder por inducción. Supongamos que el resultado se cumple para todos los
    números positivos $< n$. Si $n$ es primo, no hay que demostrar nada. Si $n$
    no es primo, entonces $n = ab$ donde $a < n$ y $b < n$. Por la hipótesis de
    inducción, $a$ y $b$ son productos de números primos, y por lo tanto $n$ lo
    es.
  \end{proof}
\end{proposicion}

En este caso la palabra ``primo'' es un sinónimo de ``primero'' y refiere
precisamente al hecho de que todo número entero sea un producto de primos. No se
trata de ninguna relación de parentesco entre los números.

\begin{teorema}[Euclides]
  Hay un número infinito de primos.

  \begin{proof}
    Consideremos los primeros $n$ números primos
    $$p_1 = 2, ~ p_2 = 3, ~ p_3 = 5, ~ p_4 = 7, ~ p_5 = 11, ~ \ldots, ~ p_n.$$
    Luego, el número
    $$N \dfn p_1 p_2 \cdots p_n + 1$$
    no es divisible por ningún primo entre $p_1, \ldots, p_n$. Sin embargo, $N$
    tiene que ser un producto de primos, así que es necesariamente divisible por
    algún primo $p$ tal que $p_n < p \le N$.
  \end{proof}
\end{teorema}

% % % % % % % % % % % % % % % % % % % % % % % % % % % % % %

\section{El máximo común divisor}

\begin{definicion}
  \label{dfn:maximo-comun-divisor}
  Para dos números enteros $a,b\in \ZZ$ su \term{máximo común
    divisor}\index{máximo común divisor}\index[notacion]{mcd} (\term{mcd}) es un
  número $d \dfn \mcd (a,b)$ caracterizado por las siguientes propiedades:

  \begin{enumerate}
  \item[1)] $d \mid a$ y $d\mid b$,

  \item[2)] si $d'$ es otro número tal que $d' \mid a$ y $d' \mid b$, entonces
    $d' \mid d$.
  \end{enumerate}
\end{definicion}

Las condiciones de arriba pueden ser escritas como

\begin{enumerate}
\item[1)] $\langle a\rangle \subseteq \langle d\rangle$ y $\langle b\rangle \subseteq \langle d\rangle$,

\item[2)] si $\langle a\rangle \subseteq \langle d'\rangle$ y
  $\langle b\rangle \subseteq \langle d'\rangle$, entonces
  $\langle d\rangle \subseteq \langle d'\rangle$.
\end{enumerate}

El subgrupo mínimo de $\ZZ$ que contiene a $\langle a\rangle$ y
$\langle b\rangle$ es $\langle a,b\rangle$. Gracias a
\ref{prop:Z-es-un-dominio-de-ideales-principales}, sabemos que
$\langle a,b\rangle = \langle d\rangle$ para algún $d\in \ZZ$.

\[ \begin{tikzcd}[column sep=0.5em]
    & \langle d\rangle = \langle a,b\rangle\ar[-]{dl}\ar[-]{dr} \\
    \langle a\rangle & & \langle b\rangle
  \end{tikzcd} \]

Esto nos lleva al siguiente resultado.

\begin{proposicion}
  El mcd siempre existe: tenemos
  $$\langle a,b\rangle = \langle d\rangle \quad\text{donde } d = \mcd (a,b).$$
  En particular, se cumple
  $$ax + by = \mcd (a,b)\quad\text{para algunos }x,y\in\ZZ$$
  y $\mcd (a,b)$ es el mínimo número posible que puede ser representado como una
  combinación $\ZZ$-lineal de $a$ y $b$.
\end{proposicion}

La última expresión se conoce como la
\term{identidad de Bézout}\index{identidad!de Bézout}. Aquí los coeficientes $x$
e $y$ no son únicos. Por ejemplo,
$$2\cdot (-1) + 3\cdot 1 = 2\cdot (-4) + 3\cdot 3 = 2\cdot 2 + 3\cdot (-1) = \cdots = 1.$$

He aquí algunas observaciones respecto a $\mcd (a,b)$.

\begin{enumerate}
\item[1)] La definición de $d \dfn \mcd (a,b)$ caracteriza a $d$
  \emph{salvo signo}. De hecho, si $d$ y $d'$ satisfacen las condiciones de
  $\mcd (a,b)$, entonces $d \mid d'$ y $d' \mid d$ (o la condición equivalente
  $\langle d\rangle = \langle d'\rangle$) implica que $d' = \pm d$. Normalmente
  se escoge $d > 0$, pero estrictamente hablando, todas las identidades con
  $\mcd (a,b)$ pueden ser interpretadas salvo signo.

\item[2)] La definición de $\mcd (a,b)$ es visiblemente simétrica en $a$ y $b$,
  así que
  $$\mcd (a,b) = \mcd (b,a).$$

\item[3)] Para todo $a\in\ZZ$ se tiene
  $$\mcd (a,0) = a.$$
  En particular\footnote{Algunas fuentes insisten que $\mcd (0,0)$ no está
    definido, pero como vemos, es lógico poner $\mcd (0,0) = 0$.},
  $$\mcd (0,0) = 0.$$
  Esto nada más significa que cualquier número divide a $0$, o de manera
  equivalente, que $\langle 0\rangle \subseteq \langle a\rangle$ para todo
  $a\in\ZZ$, y también para $a = 0$.
\end{enumerate}

\begin{definicion}
  Si $\mcd (a,b) = 1$, se dice que $a$ y $b$ son
  \term{coprimos}\index{números!coprimos}.
\end{definicion}

Si $a$ y $b$ son coprimos, entonces
$\langle a,b\rangle = \langle 1\rangle = \ZZ$, y en particular tenemos
$$ax + by = 1\quad\text{para algunos }x,y\in\ZZ.$$

\begin{observacion}
  Si $a\mid bc$ donde $a$ y $b$ son coprimos, entonces $a\mid c$.

  \begin{proof}
    Tenemos
    $$ax + by = 1$$
    para algunos $x,y\in\ZZ$. Luego,
    $$axc + byc = c,$$
    y la expresión a la izquierda es divisible por $a$.
  \end{proof}
\end{observacion}

\begin{corolario}
  \label{corr:primo-divide-producto}
  Si $p$ es primo y $p\mid bc$, entonces $p\mid b$ o $p\mid c$.
\end{corolario}

Muy a menudo se usa el contrapuesto: si $p\nmid b$ y $p\nmid c$, entonces
$p\nmid bc$.

\begin{proof}
  Ya que los únicos divisores de $p$ son $\pm 1$ y $\pm p$, tenemos dos casos
  posibles. En el primer caso, $\mcd (p,b) = 1$ y luego $p\mid c$ por el
  resultado precedente. En el segundo caso, $\mcd (p,b) = p$, lo que significa
  que $p\mid b$.
\end{proof}

% % % % % % % % % % % % % % % % % % % % % % % % % % % % % %

\section{El mínimo común múltiplo}

\begin{definicion}
  \label{dfn:minimo-comun-multiplo}
  Para dos números enteros $a,b\in \ZZ$ su \term{mínimo común múltiplo
    (mcm)}\index{mínimo común múltiplo}\index[notacion]{mcm} es un número
  $m \dfn \mcm (a,b)$ caracterizado por las siguientes propiedades:

  \begin{enumerate}
  \item[1)] $a\mid m$ y $b\mid m$,

  \item[2)] si $m'$ es otro número tal que $a\mid m'$ y $b\mid m'$, entonces
    $m \mid m'$.
  \end{enumerate}
\end{definicion}

Las condiciones de arriba pueden ser escritas como

\begin{enumerate}
\item[1)] $\langle m\rangle \subseteq \langle a\rangle$ y
  $\langle m\rangle \subseteq \langle b\rangle$,

\item[2)] si $m'$ es otro número tal que
  $\langle m'\rangle \subseteq \langle a\rangle$ y
  $\langle m'\rangle \subseteq \langle b\rangle$, entonces
  $\langle m'\rangle \subseteq \langle m\rangle$.
\end{enumerate}

El subgrupo máximo de $\ZZ$ que contiene a $\langle a\rangle$ y
$\langle b\rangle$ es su intersección $\langle a\rangle \cap \langle
b\rangle$. Gracias a \ref{prop:Z-es-un-dominio-de-ideales-principales} sabemos
que es también de la forma $\langle m\rangle$ para algún $m\in \ZZ$.

\[ \begin{tikzcd}[column sep=0.5em]
    \langle a\rangle & & \langle b\rangle \\
    & \langle m\rangle = \langle a\rangle \cap \langle b\rangle\ar[-]{ul}\ar[-]{ur}
  \end{tikzcd} \]

\begin{proposicion}
  El mcm siempre existe: tenemos
  $$\langle a\rangle \cap \langle b\rangle = \langle m\rangle \quad \text{donde } m = \mcm (a,b).$$
\end{proposicion}

Tenemos las siguientes propiedades.

\begin{enumerate}
\item[1)] La definición caracteriza a $\mcm (a,b)$ de modo único salvo signo.

\item[2)] Para cualesquiera $a,b\in\ZZ$ se tiene
$$\mcm (a,b) = \mcm (b,a).$$

\item[3)] Para todo $a$ se cumple
  $$\mcm (a,0) = a.$$
  En particular,
  $$\mcm (0,0) = 0.$$
  (De hecho, $0 \mid m$ implica que $m = 0$.)
\end{enumerate}

\begin{proposicion}
  Para cualesquiera $a,b\in\ZZ$ tenemos
  $$\mcm (a,b)\cdot \mcd (a,b) = ab.$$
  En particular,
  $$\mcm (a,b) = ab\text{ si y solamente si }a\text{ y }b\text{ son coprimos}.$$
\end{proposicion}

\begin{proof}[Primera demostración]
  El caso de $a = b = 0$ es trivial y podemos descartarlo. Sea
  $d \dfn \mcd (a,b)$ y $m \dfn ab/d$. Vamos a ver que $m = \mcm (a,b)$.

  Primero, puesto que $d\mid a$ y $d\mid b$, podemos escribir
  $$a = d a',\quad b = d b'.$$
  Luego,
  $$m = d a' b' = a b' = b a',$$
  así que $a \mid m$ y $b \mid m$.

  Ahora notemos que
  $$d = \mcd (a,b) = \mcd (da', db') = d\cdot \mcd (a', b'),$$
  así que
  $$\mcd (a', b') = 1$$
  y los números $a'$ y $b'$ son coprimos.

  Sea $m'$ otro número tal que $a\mid m'$ y $b\mid m'$. Queremos ver que
  $m\mid m'$. Escribamos
  $$m' = ax = by.$$
  Luego,
  $$m' b' = a b' x = mx, \quad m' a' = b a' y = my,$$
  lo que nos da $m\mid m' b'$ y $m \mid m' a'$ y por lo tanto
  $$m \mid \mcd (m' b', m' a') = m'\cdot \mcd (a', b') = m'.$$
\end{proof}

\begin{proof}[Segunda demostración, usando la teoría de grupos]
  Consideremos los subgrupos $m\,\ZZ$ y $n\,\ZZ$ en $\ZZ$. Luego, el segundo
  teorema de isomorfía nos dice que
  $$a\ZZ/(a\ZZ\cap b\ZZ) \isom (a\ZZ + b\ZZ)/b\ZZ.$$
  Tenemos $a\ZZ\cap b\ZZ = m\ZZ$
  donde $m = \mcm (a,b)$ y $a\ZZ + b\ZZ = d\ZZ$ donde
  $d = \mcd (a,b)$. Entonces,
  $$a\ZZ/m\ZZ \isom d\ZZ/b\ZZ.$$
  Lo que nos da la identidad $a/m = d/b$.
\end{proof}

Note que la última proposición nos dice básicamente que la existencia de
$\mcd (a,b)$ es equivalente a la existencia de $\mcm (a,b)$.

También se pueden definir mcd y mcm de $n$ números. El lector puede generalizar
de manera evidente las definiciones \ref{dfn:maximo-comun-divisor} y
\ref{dfn:minimo-comun-multiplo} y ver que estas generalizaciones son
equivalentes a
\begin{enumerate}
\item[1)] $\langle a_1, \ldots, a_n\rangle = \langle d\rangle$ para
  $d = \mcd (a_1, \ldots, a_n)$,

\item[2)]
  $\langle a_1\rangle \cap \cdots \cap \langle a_n\rangle = \langle m\rangle$
  para $m = \mcm (a_1, \ldots, a_n)$.
\end{enumerate}

Por ejemplo, tenemos la siguiente generalización de la identidad de Bézout:
existen $x_1,\ldots,x_n \in \ZZ$ tales que
$$x_1 a_1 + \cdots + x_n a_n = \mcd (a_1, \ldots, a_n).$$

Además, se puede ver que las operaciones $\mcd (-,-)$ y $\mcm (-,-)$ son
asociativas y por lo tanto la definición generalizada se reduce al caso binario:
\begin{enumerate}
\item[1)] $\mcd (\mcd (a, b), c) = \mcd (a, \mcd (b, c)) = \mcd (a, b, c)$,

\item[2)] $\mcm (\mcm (a, b), c) = \mcm (a, \mcm (b, c)) = \mcm (a, b, c)$.
\end{enumerate}

% % % % % % % % % % % % % % % % % % % % % % % % % % % % % %

\section{El teorema fundamental de la aritmética}

\begin{definicion}
  Sea $p$ un número primo fijo. Para un número entero no nulo $n$ su
  \term{valuación $p$-ádica}\index{valuación $p$-ádica} es el número natural
  máximo $k$ tal que $p^k$ divide a $n$:
  $$v_p (n) \dfn \max \{ k ~ \mid ~ p^k \mid n \}.$$
\end{definicion}

\noindent (Para $n = 0$ normalmente se pone $v_p (0) \dfn +\infty$, pero no
vamos a necesitar esta convención.)

Notamos que $v_p (n) = 0$ si y solamente si $p\nmid n$. La valuación $p$-ádica
se caracteriza por
$$n = p^{v_p (n)}\,n',$$
donde $p\nmid n'$ (véase \ref{corr:primo-divide-producto}).

\begin{lema}
  Para cualesquiera $m,n\in\ZZ$ se cumple
  $$v_p (mn) = v_p (m) + v_p (n).$$

  \begin{proof}
    Tenemos
    $$m = p^{v_p (m)}\,m', \quad n = p^{v_p (n)}\,n',$$
    donde $p\nmid m'$ y $p\nmid n'$. Luego,
    $$mn = p^{v_p (m) + v_p (n)}\,m' n',$$
    donde $p\nmid (m'n')$, así que $v_p (mn) = v_p (m) + v_p (n)$.
  \end{proof}
\end{lema}

\begin{teorema}
  Todo número entero entero no nulo puede ser representado de modo único como
  $$n = \pm p_1^{k_1}\,p_2^{k_2}\cdots p_\ell^{k_\ell}$$
  donde $p_i$ son algunos primos diferentes. A saber, tenemos
  $k_i = v_{p_i} (n)$.
\end{teorema}

\noindent (La unicidad se entiende salvo permutaciones de los factores
$p_i^{k_i}$.)

\begin{proof}
  Ya hemos notado en \ref{prop:todo-entero-es-producto-de-primos} que todo
  entero no nulo es un producto de primos; la parte interesante es la
  unicidad. Dada una expresión
  $$n = \pm p_1^{k_1}\,p_2^{k_2}\cdots p_\ell^{k_\ell},$$
  para todo primo $p$ podemos calcular la valuación $p$-ádica correspondiente:
  $$v_p (n) = v_p (p_1^{k_1}) + v_p (p_2^{k_2}) + \cdots + v_p (p_\ell^{k_\ell}).$$
  Aquí
  $$v_p (p_i^{k_i}) = \begin{cases}
    k_i, & p = p_i,\\
    0, & p \ne p_i.
  \end{cases}$$
  Entonces, $k_i = v_{p_i} (n)$.
\end{proof}

Entonces, podemos escribir
$$n = \pm \prod_{p\text{ primo}} p^{v_p (n)},$$
donde el producto es sobre todos los números primos, pero
$v_p (n) \ne 0$ solamente para un número finito de $p$.

El último resultado se conoce como el
\term{teorema fundamental de la aritmética}\index{teorema!fundamental de la
  aritmética}.  Su primera demostración completa fue publicada por Gauss en
el tratado ``Disquisitiones Arithmeticae''.

Notamos que
$$\mcd (m,n) = \prod_{p\text{ primo}} p^{\min \{v_p (m), v_p (n)\}}$$
y
$$\mcm (m,n) = \prod_{p\text{ primo}} p^{\max \{v_p (m), v_p (n)\}}.$$
Estas fórmulas no ayudan mucho para grandes valores de $m$ y $n$. En práctica
se usa el \term{algoritmo de Euclides}\index{algoritmo!de Euclides} basado en
la división con resto repetida (es algo parecido a nuestra demostración de
\ref{prop:Z-es-un-dominio-de-ideales-principales}).

% % % % % % % % % % % % % % % % % % % % % % % % % % % % % %

\section{Generalizaciones}

Las definiciones \ref{dfn:maximo-comun-divisor} y
\ref{dfn:minimo-comun-multiplo} de mcd y mcm tienen sentido en cualquier dominio
de integridad $A$. En este caso $\mcm (a,b)$ y $\mcd (a,b)$ están definidos
salvo un múltiplo $u\in A^\times$. Para $A = \ZZ$ tenemos
$\ZZ^\times = \{ \pm 1 \}$. Sin embargo, la existencia de $\mcm (a,b)$ y
$\mcd (a,b)$ no está garantizada en general.

Un dominio de integridad donde se puede definir un análogo de la división con
resto se llama un \term{dominio euclidiano}; en este caso mcd y mcm siempre
existen gracias a los mismos argumentos que vimos arriba (solo hay que
reemplazar los subgrupos $A \subseteq \ZZ$ por \term{ideales}
$\mathfrak{a} \subseteq A$). Un ejemplo típico de dominios euclidianos, excepto
$\ZZ$, es el anillo de polinomios $k [X]$ sobre un cuerpo $k$: para
$f,g\in k [X]$, $g\ne 0$ existen $q,r\in k[X]$ tales que $f = qg + r$ donde
$-\infty \le \deg r < \deg g$.

Un dominio de integridad donde se cumple un análogo del teorema fundamental de
la aritmética se llama un \term{dominio factorización única}. Un típico ejemplo
es el anillo de polinomios $k [X_1,\ldots,X_n]$ en $n$ variables sobre un cuerpo
$k$. Todos los dominios euclidianos son dominios de factorización única.

\end{document}
