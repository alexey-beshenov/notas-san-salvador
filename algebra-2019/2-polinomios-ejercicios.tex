\begin{ejercicio}
  Para una serie de potencias $f = \sum_{i\ge 0} a_i\,X^i \in A [\![X]\!]$ la
  noción de grado no existe, pero se puede considerar el mínimo índice tal que
  el coeficiente correspondiente no es nulo:
  $$v (f) \dfn \min \{ i \mid a_i \ne 0 \};$$
  y si $f = 0$, pongamos
  $$v (0) \dfn +\infty.$$
  Demuestre que para cualesquiera $f, g\in A [\![X]\!]$ se cumplen las
  desigualdades
  \begin{align*}
    v (f \, g) & \ge v (f) + v (g), \\
    v (f + g) & \ge \min \{ v (f), v (g) \},
  \end{align*}
  y la igualdad
  $$v (fg) = v (f) + v (g)$$
  si $A$ es un dominio.
\end{ejercicio}

\begin{ejercicio}
  ~

  \begin{enumerate}
  \item[a)] Sea $f \in \RR [X]$ un polinomio real. Demuestre que si $z\in \CC$
    es una raíz compleja de $f$, entonces $\overline{z}$ es también una raíz.

  \item[b)] Deduzca que un polinomio real de grado impar debe tener por lo menos
    una raíz real.

  \item[c)] Demuéstrelo usando el análisis real, sin recurrir al teorema
    fundamental del álgebra.
  \end{enumerate}
\end{ejercicio}

\begin{ejercicio}
  \label{ejerc:polinomios-ciclotomicos-m-y-2m}
  Demuestre que si $m > 1$ es impar, entonces $\Phi_{2m} (X) = \Phi_m (-X)$.

  \noindent \emph{Sugerencia: compare las expresiones
    $$\prod_{d\mid 2m} \Phi_d (X) = X^{2m} - 1 = (X^m - 1)\,(X^m + 1) = -(X^m - 1)\,((-X)^m - 1) = -\prod_{d\mid m} \Phi_d (X)\,\Phi_d (-X)$$
    usando la inducción sobre $m$.}
\end{ejercicio}

\begin{ejercicio}
  Encuentre los coeficientes en la expansión de los polinomios ciclotómicos
  $$\Phi_{11}, \, \Phi_{12}, \, \Phi_{13}, \, \Phi_{14}, \, \Phi_{15}, \, \Phi_{16}, \, \Phi_{17}, \, \Phi_{18}, \, \Phi_{19}, \, \Phi_{20}.$$
\end{ejercicio}

\begin{ejercicio}[Determinante de Vandermonde]
  Sea $k$ un cuerpo y $x_0, \ldots, x_n \in k$. Demuestre que
  \[ V (x_0,x_1,\ldots,x_n) \dfn \det \begin{pmatrix}
      1 & x_0 & x_0^2 & \cdots & x_0^{n-1} & x_0^n \\
      1 & x_1 & x_1^2 & \cdots & x_1^{n-1} & x_1^n \\
      \vdots & \vdots & \vdots & \ddots & \vdots & \vdots \\
      1 & x_{n-1} & x_{n-1}^2 & \cdots & x_{n-1}^{n-1} & x_{n-1}^n \\
      1 & x_n & x_n^2 & \cdots & x_n^{n-1} & x_n^n
    \end{pmatrix} = \prod_{0 \le i < j \le n} (x_j - x_i). \]

  \ifsolutions
  \begin{solucion}
    Hay varios modos de hacer este cálculo. Podemos, por ejemplo, proceder por
    inducción sobre $n$. Si $n = 1$, tenemos
    $$\det \begin{pmatrix}
      1 & x_0 \\
      1 & x_1
    \end{pmatrix} = x_1 - x_0.$$
    Para el paso inductivo, podemos restar de la $i$-ésima columna la columna
    $(i-1)$ multiplicada por $x_0$. Esto no cambia el determinante, así que
    \begin{multline*}
      V (x_0,x_1,\ldots,x_n) = \det \begin{pmatrix}
        1 & 0 & 0 & \cdots & 0 & 0 \\
        1 & x_1-x_0 & x_1\,(x_1-x_0) & \cdots & x_1^{n-2}\,(x_1-x_0) & x_1^{n-1}\,(x_1-x_0) \\
        \vdots & \vdots & \vdots & \ddots & \vdots & \vdots \\
        1 & x_{n-1}-x_0 & x_{n-1}\,(x_{n-1}-x_0) & \cdots & x_{n-1}^{n-2}\,(x_{n-1}-x_0) & x_{n-1}^{n-1}\,(x_{n-1}-x_0) \\
        1 & x_n-x_0 & x_n\,(x_n-x_0) & \cdots & x_n^{n-2}\,(x_n-x_0) & x_n^{n-1}\,(x_n-x_0)
      \end{pmatrix} \\
      = \prod_{1 \le j \le n} (x_j - x_0) \cdot \det \begin{pmatrix}
        1 & x_1 & \cdots & x_1^{n-2} & x_1^{n-1} \\
        \vdots & \vdots & \ddots & \vdots & \vdots \\
        1 & x_{n-1} & \cdots & x_{n-1}^{n-2} & x_{n-1}^{n-1} \\
        1 & x_n & \cdots & x_n^{n-2} & x_n^{n-1}
      \end{pmatrix} = \prod_{1 \le j \le n} (x_j - x_0) \cdot V (x_1,\ldots,x_n) \\
      = \prod_{1 \le j \le n} (x_j - x_0) \cdot \prod_{1 \le i < j \le n} (x_j - x_i) = \prod_{0 \le i < j \le n} (x_j - x_i).
    \end{multline*}
  \end{solucion}
  \fi
\end{ejercicio}

\begin{ejercicio}[Interpolación polinomial]
  Sea $k$ un cuerpo. Consideremos $n$ puntos $(x_i,y_i) \in k^2$, donde
  $i = 0,\ldots,n$ y $x_i \ne x_j$ para $i \ne j$. Usando el ejercicio anterior,
  demuestre que existe un polinomio único
  $$f = a_n\,X^n + a_{n-1}\,X^{n-1} + \cdots + a_1\,X + a_0 \in k [X]$$
  de grado $\le n$ tal que $f (x_i) = y_i$ para todo $i$.

  \noindent \emph{Sugerencia: use el ejercicio anterior.}

  \ifsolutions
  \begin{solucion}
    Tenemos un sistema de ecuaciones polinomiales

    \begin{equation}
      \label{eqn:interpolacion}
      \begin{aligned}
        a_n\,x_0^n + a_{n-1}\,x_0^{n-1} + \cdots + a_1\,x_0 + a_0 & = y_0, \\
        a_n\,x_1^n + a_{n-1}\,x_1^{n-1} + \cdots + a_1\,x_1 + a_0 & = y_1, \\
        & \cdots \\
        a_n\,x_n^n + a_{n-1}\,x_n^{n-1} + \cdots + a_1\,x_n + a_0 & = y_n.
      \end{aligned}
    \end{equation}

    En la forma matricial, tenemos
    $$\begin{pmatrix}
      1 & x_0 & x_0^2 & \cdots & x_0^{n-1} & x_0^n \\
      1 & x_1 & x_1^2 & \cdots & x_1^{n-1} & x_1^n \\
      \vdots & \vdots & \vdots & \ddots & \vdots & \vdots \\
      1 & x_{n-1} & x_{n-1}^2 & \cdots & x_{n-1}^{n-1} & x_{n-1}^n \\
      1 & x_n & x_n^2 & \cdots & x_n^{n-1} & x_n^n
    \end{pmatrix} \, \begin{pmatrix} a_0 \\ a_1 \\ \vdots \\ a_{n-1} \\ a_n \end{pmatrix} = \begin{pmatrix} y_0 \\ y_1 \\ \vdots \\ y_{n-1} \\ y_n \end{pmatrix}.$$
    La primera matriz se llama la \term{matriz de Vandermonde} asociada a
    $x_0, x_1, \ldots, x_n$. Es fácil calcular su determinante.
  
    En particular, se ve que si $x_i \ne x_j$ para $i\ne j$, entonces la matriz
    de Vandermonde es invertible. Esto implica que el sistema de ecuaciones
    \eqnref{eqn:interpolacion} tiene una solución única. Es posible escribir
    explícitamente el polinomio $f$, pero no lo vamos a
    hacer\footnote{Las palabras clave son ``el polinomio de Lagrange''.}.
  \end{solucion}
  \fi
\end{ejercicio}

\begin{ejercicio}
  Consideremos el polinomio $f = X^n - 1 \in \FF_p [X]$. Demuestre que $f$ no
  tiene raíces múltiples si y solo si $p \nmid n$.
\end{ejercicio}

\begin{ejercicio}
  Sea $A$ un anillo conmutativo. Para una serie de potencias
  $f = \sum_{n \ge 0} a_n\,X^n \in A [\![X]\!]$ definamos su
  \term{derivada formal} como la serie
  $$f' \dfn \sum_{n \ge 1} n\,a_n\,X^{n-1}.$$

  \begin{enumerate}
  \item[a)] Demuestre que para cualesquiera $f, g\in A[\![X]\!]$ se cumple
    $$(f+g)' = f' + g', \quad (f \, g)' = f' \, g + f \, g'.$$

  \item[b)] Calcule las derivadas de las siguientes series formales en
    $\QQ [\![X]\!]$:
    \begin{gather*}
      \exp (X) \dfn \sum_{n \ge 0} \frac{X^n}{n!}, \quad \log (1+X) \dfn \sum_{n \ge 0} (-1)^{n+1} \frac{X^n}{n},\\
      \sen (X) \dfn \sum_{n\ge 0} (-1)^n\,\frac{X^{2n+1}}{(2n+1)!}, \quad \cos (X) \dfn \sum_{n\ge 0} (-1)^n\,\frac{X^{2n}}{(2n)!}.
    \end{gather*}
  \end{enumerate}
\end{ejercicio}

\begin{ejercicio}
  Demuestre la identidad $\sen (X)^2 + \cos(X)^2 = 1$ en el anillo de series
  formales $\QQ [\![X]\!]$
  \begin{enumerate}
  \item[a)] calculando la derivada formal de $\sen (X)^2 + \cos(X)^2$;

  \item[b)] directamente, analizando los coeficientes de
    $\sen (X)^2 + \cos(X)^2$.
  \end{enumerate}
\end{ejercicio}

\begin{ejercicio}[Serie de Taylor]
  Demuestre que si $\QQ \subseteq A$, entonces para $f\in A [\![X]\!]$ se cumple
  $$f = \sum_{n \ge 0} \frac{f^{(n)} (0)}{n!}\,X^n,$$
  donde $f^{(0)} \dfn f$ y $f^{(n)} \dfn (f^{(n-1)})'$ para $n \ge 1$.
\end{ejercicio}

\begin{ejercicio}
  Si $\QQ \subseteq A$, definamos las \term{integrales formales} por
  $$\int_0^X \left(\sum_{n\ge 0} a_n\,X^n\right) \, dX \dfn \sum_{n\ge 0} \frac{a_n}{n+1}\,X^{n+1}.$$

  \begin{enumerate}
  \item[a)] Demuestre que se cumple el \term{teorema fundamental del cálculo}:
    $$\int_0^X f' (X)\,dX = f (X) - f (0) \quad\text{y}\quad \left(\int_0^X f (X)\,dX\right)' = f (X),$$
    donde $f (0)$ denota el término constante de $f$.

  \item[b)] Demuestre que se cumple la \term{integración por partes}:
    $$f (X)\,g (X) - f (0)\,g (0) = \int_0^X f (X)\,g' (X)\,dX + \int_0^X f' (X)\,g (X)\,dX.$$

  \item[c)] Calcule las series
    $$\int_0^X \exp (X)\,dX, ~ \int_0^X \log (1+X)\,dX, ~ \int_0^X X\,\exp (X)\,dX.$$
  \end{enumerate}
\end{ejercicio}
