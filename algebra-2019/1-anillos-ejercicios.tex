\begin{ejercicio}
  Demuestre las identidades trigonométricas
  \begin{align*}
    \sen (\phi + \psi) & = \sen\phi \cos\psi + \cos\phi \sen\psi,\\
    \cos (\phi + \psi) & = \cos\phi \cos\psi - \sen\phi \sen\psi
  \end{align*}
  usando la identidad de Euler para los números complejos.
\end{ejercicio}

\begin{ejercicio}
  Sea $n = 2,3,4,\ldots$ un número fijo y $\zeta_n \dfn e^{2\pi i/n}$.

  \begin{enumerate}
  \item[a)] Para un polinomio complejo
    $f = a_{n-1}\,X^{n-1} + \cdots + a_1\,X + a_0$ de grado $< n$ demuestre que
    $$\frac{1}{n}\,\sum_{0\le k \le n-1} f (\zeta_n^k) = a_0.$$

  \item[b)] Demuestre que
    $\prod\limits_{1\le k\le n-1} (1-\zeta_n^k) = n$.
  \end{enumerate}
\end{ejercicio}

\begin{ejercicio}
  Sea $X$ un conjunto y $2^X$ el conjunto de los subconjuntos de $X$. Demuestre
  que $2^X$ es un anillo conmutativo de característica $2$ respecto a la suma
  $A\triangle B$ (diferencia simétrica) y producto $A\cap B$ (intersección).
\end{ejercicio}

\begin{ejercicio}[Los números duales]
  Inmitando la definición de los números complejos, consideremos las expresiones
  $x + y\epsilon$, donde $x,y$ son números reales, respecto a la suma y producto
  \begin{align*}
    (x_1 + y_1\epsilon) + (x_2 + y_2\epsilon) & \dfn (x_1+x_2) + (y_1 + y_2)\,\epsilon,\\
    (x_1 + y_1\epsilon) \cdot (x_2 + y_2\epsilon) & \dfn x_1 x_2 + (x_1 y_2 + x_2 y_1)\,\epsilon.
  \end{align*}

  \begin{enumerate}
  \item[a)] Demuestre que de esta manera se obtiene un anillo conmutativo.

  \item[b)] Demuestre que no es un dominio.

  \item[c)] Determine cuándo un elemento $x + y\epsilon$ es invertible y
    encuentre la fórmula para su inverso.
  \end{enumerate}
\end{ejercicio}

\begin{ejercicio}[Cuaterniones]
  \label{ejerc:cuaterniones}
  Denotemos por $u\cdot v$ y $u\times v$ el producto escalar y producto cruz
  sobre $\RR^3$ respectivamente.

  \begin{enumerate}
  \item[a)] Demuestre que en general,
    $(u\times v)\times w \ne u\times (v\times w)$, pero se cumple
    la \term{identidad de Jacobi}
    $$u \times (v \times w) + v \times (w \times u) + w \times (u \times  v) = 0.$$

  \item[b)] Identifiquemos los elementos de $\RR^4$ con pares $(a,u)$, donde
    $a \in \RR$ y $u \in \RR^3$. Demuestre que $\RR^4$ forma un anillo no
    conmutativo respecto a las operaciones
    $$(a,u) + (b,v) \dfn (a+b, u+v), \quad (a,u) \cdot (b,v) \dfn (ab - u\cdot v, \, av + bu + u\times v).$$
    Este se llama el \term{anillo de cuaterniones} y se denota por $\HH$.

  \item[c)] Demuestre que todo elemento no nulo en $\HH$ es invertible.

    \emph{Sugerencia: defina $\overline{(a,u)} \dfn (a,-u)$ y calcule
      $(a,u)\cdot \overline{(a,u)}$.}
  \end{enumerate}
\end{ejercicio}

\begin{ejercicio}[Enteros ciclotómicos]
  Para un número primo $p$ consideremos el conjunto
  $$\ZZ [\zeta_p] \dfn \{ a_0 + a_1\,\zeta_p + a_2\,\zeta_p^2 + \cdots + a_{p-2}\,\zeta_p^{p-2} \mid a_i\in\ZZ \} \subset \CC.$$

  \begin{enumerate}
  \item[a)] Demuestre que $\ZZ [\zeta_p]$ es un subanillo de $\CC$.

  \item[b)] Calcule $(1 + \zeta_5^3)^2$, $(1 + \zeta_5^3)^3$,
    $(1 + \zeta_5^3)^{-1}$ en $\ZZ [\zeta_5]$.
  \end{enumerate}
\end{ejercicio}

\begin{ejercicio}
  Para un número fijo $n = 1,2,3,\ldots$ consideremos el conjunto de fracciones
  con potencias de $n$ en el denominador:
  $$\ZZ \Bigl[\frac{1}{n}\Bigr] \dfn \Bigl\{ \frac{m}{n^k} \Bigm| m\in \ZZ, ~ k = 0, 1, 2, 3, \ldots \Bigr\} \subset \QQ.$$
  De modo similar, para un número primo fijo $p = 2,3,5,7,11,\ldots$
  consideremos las fracciones con denominador no divisible por $p$:
  $$\ZZ_{(p)} \dfn \Bigl\{ \frac{a}{b} \Bigm| a,b\in \ZZ, ~ b\ne 0, ~ p\nmid b \Bigr\} \subset \QQ.$$
  Verifique que $\ZZ \Bigl[\frac{1}{n}\Bigr]$ y $\ZZ_{(p)}$ son subanillos de
  $\QQ$.
\end{ejercicio}

\begin{ejercicio}
  Sea $A$ un anillo y $A_i \subseteq A$ una familia de subanilos. Demuestre que
  $\bigcap_i A_i$ es un subanillo de $A$.
\end{ejercicio}

\begin{ejercicio}[Series formales de potencias]
  \label{ejerc:series-formales}
  Sea $A$ un anillo conmutativo. Una \term{serie formal de potencias} con
  coeficientes en $A$ en una variable $X$ es una suma formal
  $$f = \sum_{i\ge 0} a_i\,X^i,$$
  donde $a_i \in A$. A diferencia de polinomios, se puede tener un número
  infinito de coeficientes no nulos. Las sumas y productos de series formales
  están definidos por
  $$\sum_{i\ge 0} a_i\,X^i + \sum_{i\ge 0} b_i\,X^i \dfn \sum_{i\ge 0} (a_i + b_i)\,X^i, \quad \left(\sum_{i\ge 0} a_i\,X^i\right) \cdot \left(\sum_{i\ge 0} b_i\,X^i\right) \dfn \sum_{k\ge 0} \left(\sum_{i+j = k} a_i b_j\right)\,X^k.$$

  \begin{enumerate}
  \item[a)] Demuestre que las series formales forman un anillo conmutativo. Este
    se denota por $A [\![X]\!]$\index[notacion]{AX@$A [[X]]$}.

  \item[b)] Demuestre que $A [X]$ es un subanillo de $A [\![X]\!]$.

  \item[c)] Demuestre que si $A$ es un dominio, entonces $A [\![X]\!]$ es
    también un dominio.

    \emph{Sugerencia: para dos series no nulas $f, g \in A [\![X]\!]$, sean
      $a_m$ y $b_n$ el primer coeficiente no nulo de $f$ y $g$ respectivamente:
      $$f = a_m\,X^m + a_{m+1}\,X^{m+1} + \cdots, \quad g = b_n\,X^n + b_{n+1}\,X^{n+1} + \cdots$$
      Analice los coeficientes del producto $fg$.}

  \item[d)] Verifique la identidad
    $$(1+X)\cdot (1 - X + X^2 - X^3 + X^4 - X^5 + \cdots) = 1$$
    en el anillo de series formales $A [\![X]\!]$.

  \item[e)] Verifique la identidad
    $\Bigl(\sum\limits_{i\ge 0} \frac{X^i}{i!}\Bigr)^n =
    \sum\limits_{i\ge 0} \frac{n^i}{i!}\,X^i$
    en el anillo de series formales $\QQ [\![X]\!]$.
  \end{enumerate}
\end{ejercicio}

% \begin{ejercicio}
%   Sea $p$ un número primo. Demuestre que los coeficientes binomiales
%   ${p\choose i}$ son divisibles por $p$ para $i = 1,\ldots,p-1$.
% \end{ejercicio}

\begin{ejercicio}
  \label{ejerc:binomio-matrices}
  En el anillo de matrices $M_2 (A)$ encuentre dos elementos $a,b$ tales que
  $$(ab)^2 \ne a^2\,b^2, \quad (a+b)^2 \ne a^2 + 2\,ab + b^2.$$
\end{ejercicio}

\begin{ejercicio}
  Sea $A$ un anillo conmutativo.

  \begin{enumerate}
  \item[a)] Si $x,y\in A$ son nilpotentes, demuestre que $x+y$ es también
    nilpotente.

    \emph{Sugerencia: calcule $(x+y)^n$ usando el teorema del binomio.}

  \item[b)] En el anillo de matrices $M_2 (A)$ encuentre $a,b\in M_2 (A)$ tales
    que $a$ y $b$ son nilpotentes, pero $a+b$ no es nilpotente.
  \end{enumerate}
\end{ejercicio}

\begin{ejercicio}
  Sea $A$ un anillo. Demuestre que si $x \in A$ es nilpotente, entonces
  $1 \pm x$ es invertible en $A$.

  \emph{Sugerencia: revise la fórmula para la serie geométrica
    $\sum_{k\ge 0} x^k$.}
\end{ejercicio}

\begin{ejercicio}
  Consideremos las matrices con coeficientes en cualquier anillo conmutativo
  $A$.

  \begin{enumerate}
  \item[a)] Demuestre que las matrices de la forma
    \[ \begin{pmatrix}
        0 & a_{12} & a_{13} \\
        0 & 0 & a_{23} \\
        0 & 0 & 0
      \end{pmatrix} \]
    son nilpotentes.

  \item[b)] En general, demuestre que toda
    \term{matriz triangular superior estricta} de $n\times n$; es decir
    $a \in M_n (A)$ con $a_{ij} = 0$ para $i \ge j$ (la diagonal es también
    nula) es nilpotente.
  \end{enumerate}
\end{ejercicio}

\begin{ejercicio}
  Sea $a \in M_n (A)$ una matriz triangular superior estricta. Demuestre que
  $$(1-a)^{-1} = 1 + a + a^2 + a^3 + \cdots + a^{n-1}.$$
\end{ejercicio}

\begin{ejercicio}
  Sea $A$ un dominio.

  \begin{enumerate}
  \item[a)] Demuestre que para todo $a\ne 0$ la aplicación
    $$\mu_a\colon A \to A, \quad x \mapsto ax$$
    es inyectiva.

  \item[b)] Demuestre que si $A$ es un dominio finito, entonces la aplicación
    $x \mapsto ax$ es biyectiva.

  \item[c)] Deduzca de lo anterior que todo dominio finito es un cuerpo.
  \end{enumerate}
\end{ejercicio}

\begin{ejercicio}
  He aquí una variación sobre el ejercicio anterior. Sean $A$ un dominio y $k$
  un cuerpo. Asumamos que $A$ es un $k$-espacio vectorial de dimensión finita de
  tal manera que la multiplicación
  $$A\times A \to A, \quad (a,b) \mapsto ab$$
  es una aplicación $k$-bilineal. Demuestre que $A$ es un dominio.

  \noindent (De nuevo, para $a\ne 0$ considere la aplicación lineal
  $\mu_a\colon A \to A$, $x \mapsto ax$.)
\end{ejercicio}

\begin{ejercicio}
  Sean $L$ un cuerpo y $K \subseteq L$ un subcuerpo. Demuestre que $L$ es un
  espacio vectorial sobre $K$.
\end{ejercicio}

\begin{ejercicio}
  Calcule la dimensión del espacio vectorial

  \begin{enumerate}
  \item[a)] $\CC$ sobre $\RR$,

  \item[b)] $\QQ (\sqrt{n})$ sobre $\QQ$, donde $n\ne 1$ es libre de cuadrados.
  \end{enumerate}
\end{ejercicio}

\begin{ejercicio}
  Demuestre que $\RR$ tiene dimensión infinita sobre $\QQ$.

  \emph{Sugerencia: recuerde que $\RR$ no es un conjunto numerable.}
\end{ejercicio}

\begin{ejercicio}
  Sean $A$ un dominio y $\Frac A$ su cuerpo de fracciones. Demuestre
  explícitamente todos los axiomas de anillos (anillos conmutativos, cuerpos)
  para $\Frac A$.
\end{ejercicio}

\begin{ejercicio}
  Volvamos al anillo de las series formales $A [\![X]\!]$ introducido en el
  ejercicio \ref{ejerc:series-formales}.

  \begin{enumerate}
  \item[a)] En el anillo $\ZZ [\![X]\!]$ demuestre que los siguientes elementos
    son invertibes y encuentre sus inversos:
    $$f = X^2-2X+1, \quad g = 1 - X - X^2.$$

  \item[b)] Generalizando estos cálculos, demuestre que una serie formal es
    invertible si y solo si su término constante es invertible:
    $$A [\![X]\!]^\times = \Bigl\{ \sum_{i\ge 0} a_i\,X^i \Bigm| a_0 \in A^\times \Bigr\}.$$
  \end{enumerate}
\end{ejercicio}

\begin{ejercicio}
  Sea $k$ un cuerpo. Una \term{serie de Laurent} es una serie formal que puede
  tener un número finito de términos $a_i\,X^i$ con $i < 0$:
  $$f = \sum_{i\ge -k} a_i X^i = a_{-k} X^{-k} + a_{-k+1} X^{-k+1} + \cdots + a_{-1} X^{-1} + a_0 + a_1 X + a_2 X^2 + a_3 X^3 + \cdots,$$
  donde $a_i \in k$. Demuestre que las series de Laurent forman un cuerpo. Este
  se denota por $k (\!(X)\!)$.
\end{ejercicio}
