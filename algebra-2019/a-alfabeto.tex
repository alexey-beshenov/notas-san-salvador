\chapter{Alfabeto griego y fraktur}

\ifdefined\separatechapter\bookbanner\fi

Como en todas las áreas de matemáticas, en álgebra muy a menudo se usa el
alfabeto griego.

\begin{center}
  \noindent\begin{tabular}{x{1.5cm}x{1.5cm}x{1.5cm}x{1.5cm}x{1.5cm}x{1.5cm}x{1.5cm}}
    \biglet{$\mathrm{A}\alpha$} & \biglet{$\mathrm{B}\beta$} & \biglet{$\Gamma\gamma$} & \biglet{$\Delta\delta$} & \biglet{$\mathrm{E}\epsilon$} & \biglet{$\mathrm{Z}\zeta$} & \biglet{$\mathrm{H}\eta$} \tabularnewline
alfa & beta & gamma & delta & épsilon & zeta & eta \tabularnewline
    & & (gama) & & & (dseta) & \tabularnewline
    \tabularnewline
    \biglet{$\Theta\theta$} & \biglet{$I\iota$} & \biglet{$\mathrm{K}\kappa$} & \biglet{$\Lambda\lambda$} & \biglet{$\mathrm{M}\mu$} & \biglet{$\mathrm{N}\nu$} & \biglet{$\Xi\xi$} \tabularnewline
    theta & iota & kappa & lambda & my & ny & xi \tabularnewline
    (teta) & & (kapa) & & (mi) & (ni) & (csi) \tabularnewline
    \tabularnewline
    \biglet{$\mathrm{Oo}$} & \biglet{$\Pi\pi$} & \biglet{$\mathrm{P}\rho$} & \biglet{$\Sigma\sigma$} & \biglet{$\mathrm{T}\tau$} & \biglet{$Y\upsilon$} & \biglet{$\Phi\phi$} \tabularnewline
    ómicron & pi & rho & sigma & tau & ípsilon & fi \tabularnewline
    & & (ro)  & & & (ypsilon) \tabularnewline
    \tabularnewline
    \biglet{$\mathrm{X}\chi$} & \biglet{$\Psi\psi$} & \biglet{$\Omega\omega$} & & & \tabularnewline
    ji & psi & omega & & & \tabularnewline
  \end{tabular}
\end{center}

Las letras $\theta$, $\sigma$, $\pi$ también tienen otras variantes:
$\vartheta$, $\varsigma$, $\varpi$, pero no las vamos a ocupar.

Del alfabeto hebreo vamos a necesitar en pocas ocasiones solamente su primera
letra:
\begin{center}
  \noindent\begin{tabular}{c}
    \biglet{$\aleph$} \tabularnewline
    álef \tabularnewline
  \end{tabular}
\end{center}

\pagebreak

Los textos en \emph{fraktur} (o simplemente ``letras góticas'') no son muy
legibles, pero este tipo de letra todavía se usa en algunas ocasiones en
matemáticas. En este libro se usan las letras
$\mathfrak{a}, \mathfrak{b}, \mathfrak{c}$ para denotar los ideales; las letras
$\mathfrak{p}$ y $\mathfrak{q}$ para los ideales primos y $\mathfrak{m}$ y
$\mathfrak{n}$ para los ideales maximales.

\begin{center}
  \noindent {\begin{tabular}{x{1.5cm}x{1.5cm}x{1.5cm}x{1.5cm}x{1.5cm}x{1.5cm}x{1.5cm}x{1.5cm}x{1.5cm}}
    \biglet{$\mathfrak{Aa}$} & \biglet{$\mathfrak{Bb}$} & \biglet{$\mathfrak{Cc}$} & \biglet{$\mathfrak{Dd}$} & \biglet{$\mathfrak{Ee}$} & \biglet{$\mathfrak{Ff}$} & \biglet{$\mathfrak{Gg}$} \tabularnewline
    \tabularnewline
    \biglet{$\mathfrak{Hh}$} & \biglet{$\mathfrak{Ii}$} & \biglet{$\mathfrak{Jj}$} & \biglet{$\mathfrak{Kk}$} & \biglet{$\mathfrak{Ll}$} & \biglet{$\mathfrak{Mm}$} & \biglet{$\mathfrak{Nn}$} \tabularnewline
    \tabularnewline
    \biglet{$\mathfrak{Oo}$} & \biglet{$\mathfrak{Pp}$} & \biglet{$\mathfrak{Qq}$} & \biglet{$\mathfrak{Rr}$} & \biglet{$\mathfrak{Ss}$} & \biglet{$\mathfrak{Tt}$} & \biglet{$\mathfrak{Uu}$} \tabularnewline
    \tabularnewline
    \biglet{$\mathfrak{Vv}$} & \biglet{$\mathfrak{Ww}$} & \biglet{$\mathfrak{Xx}$} & \biglet{$\mathfrak{Yy}$} & \biglet{$\mathfrak{Zz}$} \tabularnewline
  \end{tabular}}
\end{center}
