\chapter{Programa \gp}

\ifdefined\separatechapter\bookbanner\fi

% \emph{Nota: este texto será expandido durante el curso.}

\vspace{1em}

En nuestro curso aparecen varios ejemplos que, como una gran parte de las
matemáticas interesantes en general, provienen de la teoría de números. Uno de
los mejores programas para la teoría de números computacional es \gp{} que se
está desarrollando en la universidad de Burdeos (Francia). Invito al lector a
descargar el programa de su página oficial
\begin{center}
  \url{http://pari.math.u-bordeaux.fr/}
\end{center}
antes de leer este apéndice.
% Para probar \gp{} desde el navegador web, se puede usar la página
% \begin{center}
% \url{http://pari.math.u-bordeaux.fr/gp.html}
% \end{center}

\gp{} es un programa muy poderoso y entre otras cosas cuenta con su propio
lenguaje de programación. Para más detalles, consulte la documentación.
El lector interesado en los algoritmos empleados por \gp{} puede empezar por
el libro de texto \cite{Cohen-GTM-138}.

% % % % % % % % % % % % % % % % % % % % % % % % % % % % % %

\section{Sesión interactiva}

Para abrir la calculadora \gp, hay que ejecutar en el terminal el comando
\texttt{gp}. El símbolo \texttt{?} al inicio de la línea significa que
el programa nos invita a digitar un comando o expresión para evaluar y presionar
la tecla \boxed{\text{Enter}}. Luego \gp{} hace los cálculos necesarios e
imprime el resultado. Las salidas se etiquetan por \texttt{\%1}, \texttt{\%2},
\texttt{\%3}, etc. con el propósito de usar los resultados de cálculos más
adelante. Para usar el último resultado, se puede escribir simplemente
\texttt{\%}.

\begin{framed}\footnotesize
\begin{verbatim}
? 10!
%1 = 3628800
? 1/2 + 1/3 + 1/4
%2 = 13/12
? binomial(5,2)
%3 = 10
? %^2
%4 = 100
? %1/(5*6*7*8*9*10)
%5 = 24
\end{verbatim}
\end{framed}

Las teclas $\boxed{\uparrow}$ y $\boxed{\downarrow}$ pueden ser usadas para ver
los comandos digitados previamente, y la tecla \boxed{\text{Tab}} completa los
comandos. Por ejemplo, podemos digitar \texttt{sqr} y presionar
\boxed{\text{Tab}} para ver todos los comandos cuyo nombre empieza por
``\texttt{sqr}''.

Para asignar el valor $c$ a una variable $x$, hay que escribir
\texttt{$x$ = $c$}. Para olvidar la variable, hay que ejecutar el comando
\texttt{kill\,($x$)}. Cuando no nos interesa ver la salida, se puede terminar
la expresión por el punto y coma ``\texttt{;}''. En este caso el valor de
la expresión no aparecerá.

\begin{framed}\footnotesize
\begin{verbatim}
? x = 1;
? x^2 + x + 1
% = 3
? kill (x)
? x^2 + x + 1
% = x^2 + x + 1
\end{verbatim}
\end{framed}

A veces algo va mal en los cálculos: la expresión contiene un error de
digitación o la entrada es incorrecta. Por ejemplo, tratemos de calcular la raíz
cuadrada de $-1$ módulo $11$:

\begin{framed}\footnotesize
\begin{verbatim}
? sqrt (Mod (-1,11))
  ***   at top-level: sqrt(Mod(-1,11))
  ***                 ^----------------
  *** sqrt: not an n-th power residue in gsqrt: Mod(10, 11).
  ***   Break loop: type 'break' to go back to GP prompt
break> 
\end{verbatim}
\end{framed}

\gp{} nos señala que hay un error: el número $-1$ no es un cuadrado módulo
$11$. El programa entró en un ciclo de interrupción (\emph{break loop}) que
sirve principalmente para corregir errores en código del usuario. Para volver
al modo normal, hay que ejecutar el comando \texttt{break} o simplemente
presionar \boxed{\text{Ctrl}}+\boxed{\text{D}}.

Para salir del programa, use el comando \texttt{\textbackslash q}.

% % % % % % % % % % % % % % % % % % % % % % % % % % % % % %

\section{Ayuda}

Un comando muy útil es ``\texttt{?}''. Al ejecutarlo, \gp{} muestra una lista de
temas.

\begin{framed}\footnotesize
\begin{verbatim}
? ?
Help topics: for a list of relevant subtopics, type ?n for n in
   0: user-defined functions (aliases, installed and user functions)
   1: PROGRAMMING under GP
   2: Standard monadic or dyadic OPERATORS
   3: CONVERSIONS and similar elementary functions
   4: functions related to COMBINATORICS
   5: NUMBER THEORETICAL functions
   . . . . .
\end{verbatim}
\end{framed}

Esto quiere decir que, por ejemplo, si queremos consultar las funciones
relacionadas con la teoría de números, podemos escribir \texttt{?5}. Al hacerlo
sale un larga lista de funciones. Allí aparecen por ejemplo \texttt{divisors} y
\texttt{numdiv}. Si queremos saber a qué sirven estas funciones, hay que digitar
\texttt{?divisors} y \texttt{?numdiv} respectivamente.

\begin{framed}\footnotesize
\begin{verbatim}
? ?divisors
divisors(x,{flag=0}): gives a vector formed by the divisors of x in increasing order.
If flag = 1, return pairs [d, factor(d)].

? ?numdiv
numdiv(x): number of divisors of x.
\end{verbatim}
\end{framed}

Entonces, el programa nos dice que \texttt{divisors\,($x$)} calcula la lista de
divisores de $x$ y \texttt{numdiv\,($x$)} calcula el número de divisores de $x$.

\begin{framed}\footnotesize
\begin{verbatim}
? divisors (20)
% = [1, 2, 4, 5, 10, 20]
? numdiv (20)
% = 6
\end{verbatim}
\end{framed}

% % % % % % % % % % % % % % % % % % % % % % % % % % % % % %

\section{Números enteros y racionales}

Para los números enteros están definidas las operaciones aritméticas habituales,
como \texttt{$m$+$n$}, \texttt{$m$-$n$}, \texttt{$m$*$n$}, \texttt{$m$\^{}$n$}
(el número $m^n$), \texttt{$n$!} (el factorial). El número racional
$\frac{a}{b}$ se denota por \texttt{$a$/$b$}. He aquí algunas funciones útiles.

\begin{itemize}
\item \texttt{binomial\,($m$,$n$)} devuelve el coeficiente binomial
  ${m\choose n}$. Por ejemplo, \texttt{binomial(6,2)} devuelve \texttt{15}.

\item \texttt{isprime\,($n$)} verifica si $n$ es primo. Por ejemplo,
  \texttt{isprime\,(2\^{}521-1)} devuelve \texttt{1} porque el número
  $2^{521}-1$ es primo, mientras que \texttt{isprime\,(2\^{}{520}-1)} devuelve
  \texttt{0}: el número $2^{520}-1$ no es primo: este es divisible por $3$:
  $$2^{520}-1 \equiv (-1)^{520} + 2 \equiv 0 \pmod{3}.$$

\item \texttt{primes($n$)} devuelve la lista de los primeros $n$ primos. Por
  ejemplo, \texttt{primes(10)} devuelve
  \begin{center}
    \texttt{[2, 3, 5, 7, 11, 13, 17, 19, 23, 29]}
  \end{center}

\item \texttt{divisors($n$)} devuelve la lista de los divisores de $n$. Por
  ejemplo, \texttt{divisors(40)} devuelve
  \begin{center}
    \texttt{[1,\,2,\,4,\,5,\,8,\,10,\,20,\,40]}
  \end{center}

\item \texttt{factor($n$)} devuelve los factores primos de $n$. Por ejemplo,
\end{itemize}

\begin{framed}\footnotesize
\begin{verbatim}
? factor (-180)
% = 
[-1 1]

[ 2 2]

[ 3 2]

[ 5 1]
\end{verbatim}
\end{framed}

La salida es una matriz de $s\times 2$ donde en cada fila está un factor
irreducible y el exponente correspondiente. En la primera fila también puede
estar un factor invertible (en el caso de los números enteros, esto significa
que allí estará $-1$ si el número es negativo).

\begin{itemize}
\item \texttt{valuation\,($n$,$p$)} devuelve la valuación $p$-ádica de $n$.
\end{itemize}

\begin{framed}\footnotesize
\begin{verbatim}
? valuation (180,2)
% = 2
? valuation (180,3)
% = 2
? valuation (18/25,2)
% = 1
? valuation (18/25,3)
% = 2
? valuation (18/25,5)
% = -2
\end{verbatim}
\end{framed}

\begin{itemize}
\item \texttt{eulerphi($n$)} calcula la función de Euler $\phi (n)$. Por
  ejemplo,
\end{itemize}
\begin{framed}\footnotesize
\begin{verbatim}
? eulerphi(4)
% = 2
? eulerphi(25)
% = 20
? eulerphi(100)
% = 40
\end{verbatim}
\end{framed}

% % % % % % % % % % % % % % % % % % % % % % % % % % % % % %

\section{Números reales y complejos}

Para trabajar con aproximaciones a los números reales, se usan los números con
punto flotante. Para los números complejos, se usa el símbolo \texttt{I} que
denota la unidad imaginaria. Note que esta es la \texttt{I} mayúscula.

Están predefinidas varias funciones como \texttt{sqrt\,($z$)} (raíz cuadrada),
\texttt{sin\,($z$)} (seno), \texttt{cos\,($z$)} (coseno), \texttt{exp\,($z$)}
(exponencial), \texttt{log\,($z$)} (logaritmo natural), etc. La constante
\texttt{Pi} (con la P mayúscula) corresponde al valor aproximado del número
$\pi$.

\begin{framed}\footnotesize
\begin{verbatim}
? Pi
% = 3.1415926535897932384626433832795028842

? sqrt (-2)
% = 1.4142135623730950488016887242096980786*I

? cos (2.3)^2 + sin (2.3)^2
% = 1.0000000000000000000000000000000000000

? exp (2*Pi*I/3) + exp (4*Pi*I/3)
% = -1.0000000000000000000000000000000000000 + 5.877471754111437540 E-39*I
\end{verbatim}
\end{framed}

No vamos a enumerar todas las posibles funciones relacionadas con los números
reales y complejos; el lector puede revisar la documentación de \gp{}
(en particular la sección ``Transcendental functions'').

Para cambiar la precisión, se puede usar el comando \texttt{\textbackslash p}.

\begin{framed}\footnotesize
\begin{verbatim}
? \p 10
   realprecision = 19 significant digits (10 digits displayed)
? Pi
% = 3.141592654
? sin (1)
% = 0.8414709848
? \p 50
   realprecision = 57 significant digits (50 digits displayed)
? Pi
% = 3.1415926535897932384626433832795028841971693993751
? sin (1)
% = 0.84147098480789650665250232163029899962256306079837
\end{verbatim}
\end{framed}

Para convertir un número racional $x$ en un número con punto flotante, podemos
escribir \texttt{$x$*1.0}. Por ejemplo,
\begin{framed}\footnotesize
\begin{verbatim}
? x = 2/3
% = 2/3
? x*1.0
% = 0.66666666666666666666666666666666666667
? 2.0/3
% = 0.66666666666666666666666666666666666667
\end{verbatim}
\end{framed}

% % % % % % % % % % % % % % % % % % % % % % % % % % % % % %

\section{Anillos $\ZZ/n\ZZ$}

El resto de $a$ módulo $n$ se denota por \texttt{Mod\,($a$,$n$)}. Para estos
restos están definidas las operaciones aritméticas habituales.

\begin{framed}\footnotesize
\begin{verbatim}
? Mod (4,10) + Mod (6,10)
% = Mod(0, 10)
? Mod (6,10)^2
% = Mod(6, 10)
? Mod (3,10)^-1
% = Mod(7, 10)
\end{verbatim}
\end{framed}

\begin{itemize}
\item \texttt{issquare\,(Mod($a$,$n$))} verifica si $a$ es un cuadrado módulo
  $n$.
\end{itemize}

\begin{framed}\footnotesize
\begin{verbatim}
? issquare (Mod (-1,11))
% = 0
? issquare (Mod (-1,13))
% = 1
? sqrt (Mod (-1,13))
% = Mod(5, 13)
\end{verbatim}
\end{framed}

\begin{itemize}
\item \texttt{kronecker\,($a$,$p$)} calcula el símbolo de Legendre
  \[ \legendre{a}{p} = \begin{cases}
    +1, & p\nmid a\text{ y }a\text{ es un cuadrado módulo }p,\\
    -1, & p\nmid a\text{ y }a\text{ no es un cuadrado módulo }p,\\
    0, & p\mid a.
  \end{cases} \]
(De hecho, como sugiere su nombre, esta función calcula el símbolo de Kronecker
que es una generalización del símbolo de Legendre, donde $p$ no es
necesariamente primo.)
\end{itemize}

\begin{framed}\footnotesize
\begin{verbatim}
? kronecker (3,5)
% = -1
? kronecker (5,3)
% = -1
\end{verbatim}
\end{framed}

% % % % % % % % % % % % % % % % % % % % % % % % % % % % % %

\section{Enteros de Gauss $\ZZ [i]$}

Funcionan todas las operaciones aritméticas con los enteros de Gauss que son
los números de la forma \texttt{$a$ + $b$*I}, donde $a$ y $b$ son enteros.
Al dividir dos enteros de Gauss, se devuelve una expresión de la forma
\texttt{$x$ + $y$*I}, donde $x,y\in\QQ$. Por ejemplo,
\begin{framed}\footnotesize
\begin{verbatim}
? (2+3*I)/(3+2*I)
% = 12/13 + 5/13*I
? (5-5*I)/(1+I)
% = -5*I
\end{verbatim}
\end{framed}

Para factorizar un entero de Gauss $\alpha \in \ZZ [i]$, hay que escribir
\texttt{factor($\alpha$,I)}. Por ejemplo,

\begin{framed}\footnotesize
\begin{verbatim}
? factor (180,I)
% = 
[      I 1]

[      3 2]

[  1 + I 4]

[  2 + I 1]

[1 + 2*I 1]
\end{verbatim}
\end{framed}

% % % % % % % % % % % % % % % % % % % % % % % % % % % % % %

\section{Polinomios}

Un polinomio $a_n\,x^n + a_{n-1}\,x^{n-1} + \cdots + a_2\,x^2 + a_1\,x + a_0$ se
escribe como
\begin{center}
  \texttt{$a_n$*$x$\^{}$n$ + $a_{n-1}$*$x$\^{}$(n-1)$ + $\cdots$ + $a_2$*$x$\^{}$2$ + $a_1$*$x$ + $a_0$}
\end{center}
Los coeficientes $a_i$ pueden ser números enteros, racionales, reales,
complejos, números módulo $n$, etc. En lugar de $x$ puede estar cualquier otra
letra sin valor asignado\footnote{Recordemos que para olvidar el valor asignado
  a $x$, hay que escribir \texttt{kill\,($x$)}.} como $a,t$ etc. o de hecho
cualquier palabra. Es nada más una cuestión de estilo, pero \gp{} usa las letras
minúsculas para denotar las variables. Por esto vamos a romper con nuestra
tradición de usar la $X$ mayúscula para la variable de un polinomio.

\begin{framed}\footnotesize
\begin{verbatim}
? (x+1)^4
% = x^4 + 4*x^3 + 6*x^2 + 4*x + 1
? (x+y)^4
% = y^4 + 4*x*y^3 + 6*x^2*y^2 + 4*x^3*y + x^4
\end{verbatim}
\end{framed}

Notamos que \gp{} no está dirigido a cálculos con polinomios en diversas
variables. Para este propósito hay otros programas especializados, como por
ejemplo Macaulay2 (\url{http://macaulay2.com/}).

\gp{} también trabaja con el cuerpo de fracciones
$$k(x) \dfn \Frac k[x] \dfn \Bigl\{ \frac{f}{g} \Bigm| f,g\in k[x], \, g\ne 0 \Bigr\}.$$
Las fracciones de polinomios se simplifican según lo esperado:

\begin{framed}\footnotesize
\begin{verbatim}
? (1-x^2)/(1-x)^2
% = (-x - 1)/(x - 1)
\end{verbatim}
\end{framed}

La función \texttt{polcyclo\,($n$)} devuelve el polinomio ciclotómico $\Phi_n$
en la variable \texttt{x}. He aquí un par de ejemplos.

\begin{framed}\footnotesize
\begin{verbatim}
? polcyclo(105)
% = x^48 + x^47 + x^46 - x^43 - x^42 - 2*x^41 - x^40 - x^39 + x^36 + x^35
+ x^34 + x^33 + x^32 + x^31 - x^28 - x^26 - x^24 - x^22 - x^20 + x^17
+ x^16 + x^15 + x^14 + x^13 + x^12 - x^9 - x^8 - 2*x^7 - x^6 - x^5
+ x^2 + x + 1

? polcyclo(1)*polcyclo(2)*polcyclo(5)*polcyclo(10)
% = x^10 - 1
\end{verbatim}
\end{framed}

Si queremos evaluar un polinomio en variable $x$ en $x=c$, podemos escribir
\begin{center}
\texttt{subst\,($f$,$x$,$c$)}
\end{center}
Por ejemplo, si queremos calcular los coeficientes del polinomio $\Phi_9 (x+1)$,
podemos hacer lo siguiente.

\begin{framed}\footnotesize
\begin{verbatim}
? subst (polcyclo(9),x,x+1)
% = x^6 + 6*x^5 + 15*x^4 + 21*x^3 + 18*x^2 + 9*x + 3
\end{verbatim}
\end{framed}

% % % % % % % % % % % % % % % % % % % % % % % % % % % % % %

\section{Series formales}

Si queremos indicar que no se trata de un polinomio, sino de una serie formal o
serie de Laurent con términos hasta $a_{n-1}\,x^{n-1}$, hay que añadir a
la expresión ``\texttt{+ O ($x$\^{}$n$)}''. Por ejemplo, \texttt{1/(1-x)}
corresponde a la fracción $\frac{1}{1-x}$, mientras que
\texttt{1/(1-x) + O (x\^{}10)} nos dará los primeros $10$ coeficientes
(es decir, $a_0,a_1,\ldots,a_9$) de la serie inversa a $1-x$. He aquí algunos
cálculos de este tipo.

\begin{framed}\footnotesize
\begin{verbatim}
? 1/(1-x) + O (x^10)
% = 1 + x + x^2 + x^3 + x^4 + x^5 + x^6 + x^7 + x^8 + x^9 + O(x^10)

? 1/(1-x)^2 + O (x^10)
% = 1 + 2*x + 3*x^2 + 4*x^3 + 5*x^4 + 6*x^5 + 7*x^6 + 8*x^7 + 9*x^8 + 10*x^9 + O(x^10)

? 1/(x^4-2*x^3+x^2) + O (x^10)
% = x^-2 + 2*x^-1 + 3 + 4*x + 5*x^2 + 6*x^3 + 7*x^4 + 8*x^5 + 9*x^6 + 10*x^7
+ 11*x^8 + 12*x^9 + O(x^10)
\end{verbatim}
\end{framed}

Las series formales como \texttt{sin(x)}, \texttt{cos(x)}, \texttt{tan(x)},
\texttt{cotan(x)}, \texttt{exp(x)}, \texttt{log(1+x)}, etc. están
predefinidas. Para cambiar la precisión a $n$ términos, hay que escribir
\begin{center}
  \texttt{default\,(seriesprecision,$n$)}
\end{center}
He aquí un ejemplo.

\begin{framed}\footnotesize
\begin{verbatim}
? log (1+x)
% = x - 1/2*x^2 + 1/3*x^3 - 1/4*x^4 + 1/5*x^5 - 1/6*x^6 + 1/7*x^7 - 1/8*x^8 + 1/9*x^9
- 1/10*x^10 + 1/11*x^11 - 1/12*x^12 + 1/13*x^13 - 1/14*x^14 + 1/15*x^15 + O(x^16)
? default(seriesprecision,10)
? log (1+x)
% = x - 1/2*x^2 + 1/3*x^3 - 1/4*x^4 + 1/5*x^5 - 1/6*x^6 + 1/7*x^7 - 1/8*x^8
+ 1/9*x^9 + O(x^10)
? cos (x)
% = 1 - 1/2*x^2 + 1/24*x^4 - 1/720*x^6 + 1/40320*x^8 - 1/3628800*x^10 + O(x^12)
? sin (x)
% = x - 1/6*x^3 + 1/120*x^5 - 1/5040*x^7 + 1/362880*x^9 + O(x^11)
? cos (x)^2 + sin (x)^2
% = 1 + O(x^12)
\end{verbatim}
\end{framed}

La función \texttt{deriv\,($f$,$x$)} devuelve la derivada formal de $f$ respecto
a la variable $x$. Aquí $f$ debe ser un polinomio o una serie formal. El segundo
argumento se puede omitir si en $f$ aparece una sola variable. De manera
similar, \texttt{intformal\,($f$,$x$)} calcula la integral formal
$\int_0^x f (x)\,dx$.

\begin{framed}\footnotesize
\begin{verbatim}
? default(seriesprecision,8)
? log (1+x)
% = x - 1/2*x^2 + 1/3*x^3 - 1/4*x^4 + 1/5*x^5 - 1/6*x^6 + 1/7*x^7 + O(x^8)
? deriv (log (1+x),x)
% = 1 - x + x^2 - x^3 + x^4 - x^5 + x^6 + O(x^7)
? % - 1/(1+x)
% = O(x^7)
? intformal (log(1+x))
% = 1/2*x^2 - 1/6*x^3 + 1/12*x^4 - 1/20*x^5 + 1/30*x^6 - 1/42*x^7 + 1/56*x^8 + O(x^9)
? % - ((1+x)*log(1+x) - x)
% = O(x^8)
\end{verbatim}
\end{framed}

% % % % % % % % % % % % % % % % % % % % % % % % % % % % % %

\section{Factorización de polinomios}

La función \texttt{factor($f$)} factoriza el polinomio $f$ en polinomios
irreducibles. Aquí $f$ puede tener coeficientes en $\ZZ$, $\QQ$ o $\FF_p$.
Si queremos considerar un polinomio $f$ con coeficientes enteros como un
polinomio módulo $p$, hay que escribir \texttt{$f$*Mod(1,$p$)}. Por ejemplo,
factoricemos el octavo polinomio ciclotómico $\Phi_8 = x^4+1$ en $\FF_p [x]$
para $p = 2,3$:

\begin{framed}\footnotesize
\begin{verbatim}
? factor (polcyclo(8)*Mod(1,2))
% = 
[Mod(1, 2)*x + Mod(1, 2) 4]

? factor (polcyclo(8)*Mod(1,3))
% = 
[Mod(1, 3)*x^2 + Mod(1, 3)*x + Mod(2, 3) 1]

[Mod(1, 3)*x^2 + Mod(2, 3)*x + Mod(2, 3) 1]

\end{verbatim}
\end{framed}

Para dar otro ejemplo, factoricemos el polinomio $x^8-x$ en $\FF_2 [x]$.

\begin{framed}\footnotesize
\begin{verbatim}
? factor ((x^8-x)*Mod(1,2))
% = 
[                              Mod(1, 2)*x 1]

[                  Mod(1, 2)*x + Mod(1, 2) 1]

[  Mod(1, 2)*x^3 + Mod(1, 2)*x + Mod(1, 2) 1]

[Mod(1, 2)*x^3 + Mod(1, 2)*x^2 + Mod(1, 2) 1]
\end{verbatim}
\end{framed}

La función \texttt{polisirreducible($f$)} verifica si $f$ es un polinomio
irreducible.

\begin{framed}\footnotesize
\begin{verbatim}
? f = x^4+x^3+x^2+x+1;
? polisirreducible (f)
% = 1
? polisirreducible (f*Mod(1,2))
% = 1
? polisirreducible (f*Mod(1,11))
% = 0
? factor (f*Mod(1,11))
% = 
[Mod(1, 11)*x + Mod(2, 11) 1]

[Mod(1, 11)*x + Mod(6, 11) 1]

[Mod(1, 11)*x + Mod(7, 11) 1]

[Mod(1, 11)*x + Mod(8, 11) 1]
\end{verbatim}
\end{framed}

El contenido de un polinomio $f \in \QQ [x]$ puede ser calculado mediante
\texttt{content\,($f$)}. Por ejemplo, \texttt{content\,(3*x\^{}2 + 9/2*x + 3/4)}
devuelve \texttt{3/4}.

% % % % % % % % % % % % % % % % % % % % % % % % % % % % % %

\section{División con resto, MCD y MCM}

\begin{itemize}
\item \texttt{divrem\,($a$,$b$)} devuelve \texttt{[$q$,$r$]}, donde $q$ es
  el cociente y $r$ es el resto de división. Esto funciona de la misma manera
  para números enteros $a,b\in \ZZ$ y para polinomios $f,g \in k [x]$
  (recordemos que $\ZZ$ y $k [x]$ son dominios euclidianios).
\end{itemize}

\begin{framed}\footnotesize
\begin{verbatim}
? divrem (13, 4)
% = [3, 1]~

? divrem (x^2+x, x-1)
% = [x + 2, 2]~
\end{verbatim}
\end{framed}

\begin{itemize}
\item \texttt{gcd\,($a$,$b$)} y \texttt{lcm\,($a$,$b$)} calculan el mcd y mcm
  respectivamente. De nuevo, $a$ y $b$ pueden ser números enteros o polinomios.
\end{itemize}

\begin{framed}\footnotesize
\begin{verbatim}
? gcd (12,18)
% = 6
? lcm (12,18)
% = 36

? gcd (x^3-1, x^5-1)
% = x - 1
? lcm (x^3-1, x^5-1)
% = x^7 + x^6 + x^5 - x^2 - x - 1
\end{verbatim}
\end{framed}

\begin{itemize}
\item \texttt{gcdext\,($a$,$b$)} calcula la identidad de Bézout: la salida es
  [$u$,$v$,$d$], donde $d = \mcm (a,b)$ y $u,v$ son elementos tales que
  $ua + vb = d$.
\end{itemize}

\begin{framed}\footnotesize
\begin{verbatim}
? gcdext (13, 17)
% = [4, -3, 1]

? gcdext (x^3-1, x^5-1)
% = [x^3 + 1, -x, x - 1]
\end{verbatim}
\end{framed}

% % % % % % % % % % % % % % % % % % % % % % % % % % % % % %

\section{Anillos cociente $k [x]/(f)$}

Un caso muy importante de los anillos cociente son $k [x]/(f)$, donde $k$ es
un cuerpo y $f \in k [x]$ es un polinomio. Si $f$ es irreducible, entonces
el cociente $k [x]/(f)$ es un cuerpo. En todo caso diferentes elementos
del cociente se representan por los polinomios de grado $<n$:
$$a_{n-1}\,x^{n-1} + \cdots + a_1\,x + a_0, \quad a_i \in k, ~ n = \deg f.$$
Un polinomio $g$ módulo $f$ se denota en \gp{} por \texttt{Mod($g$,$f$)}.

Por ejemplo, si queremos calcular diferentes potencias $(1+\sqrt{2})^n$ en
el cuerpo
\begin{align*}
  \QQ (\sqrt{2}) & \xrightarrow{\isom} \QQ [x]/(x^2-2),\\
  a + b\sqrt{2} & \mapsto bx + a,
\end{align*}
podemos hacer lo siguiente:

\begin{framed}\footnotesize
\begin{verbatim}
? a = Mod (1+x, x^2 - 2);
? a^2
% = Mod(2*x + 3, x^2 - 2)
? a^3
% = Mod(5*x + 7, x^2 - 2)
? a^4
% = Mod(12*x + 17, x^2 - 2)
\end{verbatim}
\end{framed}

Esto significa que
\[ (1+\sqrt{2})^2 = 3 + 2\sqrt{2}, \quad
(1+\sqrt{2})^3 = 7 + 5\sqrt{2}, \quad
(1+\sqrt{2})^4 = 17 + 12\sqrt{2}. \]

Para dar otro ejemplo, el polinomio $x^3 + x + 1$ es irreducible en $\FF_2 [x]$
y el cociente $\FF_2 [x]/(x^3 + x + 1)$ es un cuerpo de $8$
elementos. Calculemos cuál es el inverso de $x+1$ en este cuerpo:

\begin{framed}\footnotesize
\begin{verbatim}
? Mod (x+1, (x^3+x+1)*Mod(1,2))^-1
% = Mod(Mod(1, 2)*x^2 + Mod(1, 2)*x, Mod(1, 2)*x^3 + Mod(1, 2)*x + Mod(1, 2))
\end{verbatim}
\end{framed}

Si queremos pasar de \texttt{Mod($g$,$f$)} al polinomio $g$, podemos usar
la función \texttt{liftpol}. Si además $g$ es un polinomio con coeficientes en
$\FF_p$ y nos gustaría pasar al polinomio correspondiente con coeficientes
enteros que representa a $g$ en $\FF_p [x] \isom \ZZ [x]/(p)$, podemos usar
\texttt{liftall}.

\begin{framed}\footnotesize
\begin{verbatim}
? f = Mod (x, (x^3+x+1)*Mod(1,2))^-1
% = Mod(Mod(1, 2)*x^2 + Mod(1, 2), Mod(1, 2)*x^3 + Mod(1, 2)*x + Mod(1, 2))
? liftpol(f)
% = Mod(1, 2)*x^2 + Mod(1, 2)
? liftall(f)
% = x^2 + 1
\end{verbatim}
\end{framed}

La función \texttt{liftall} hace los resultados más legibles. Calculemos por
ejemplo las potencias de $x$ en el cuerpo cociente $\FF_2 [x]/(x^3+x+1)$.

\begin{framed}\footnotesize
\begin{verbatim}
? f = Mod(x, (x^3+x+1)*Mod(1,2));
? liftall (vector (7,i,f^i))
% = [x, x^2, x + 1, x^2 + x, x^2 + x + 1, x^2 + 1, 1]
\end{verbatim}
\end{framed}

Aquí la función \texttt{vector\,($n$,$i$,$x_i$)} construye el vector
$[x_1,\ldots,x_n]$.

\vspace{1em}

En general, para cualquier primo $p$ y $n = 1,2,3,\ldots$ en el cuerpo
$\FF_p [x]$ existe un polinomio mónico irreducible de grado $n$. La función
\texttt{ffinit\,($p$,$n$)} devuelve uno de estos polinomios $f$. Esto sirve para
construir un cuerpo finito de $p^n$ elementos como el cociente $\FF_p [x]/(f)$.

\begin{framed}\footnotesize
\begin{verbatim}
? ffinit (2,2)
% = Mod(1, 2)*x^2 + Mod(1, 2)*x + Mod(1, 2)
? ffinit (2,3)
% = Mod(1, 2)*x^3 + Mod(1, 2)*x^2 + Mod(1, 2)
? ffinit (3,2)
% = Mod(1, 3)*x^2 + Mod(1, 3)*x + Mod(2, 3)
? ffinit (3,3)
% = Mod(1, 3)*x^3 + Mod(1, 3)*x^2 + Mod(1, 3)*x + Mod(2, 3)
\end{verbatim}
\end{framed}

% % % % % % % % % % % % % % % % % % % % % % % % % % % % % %

\section{Matrices}

La matriz
\[ \begin{pmatrix}
    a_{11} & \cdots & a_{1n} \\
    \vdots & \ddots & \vdots \\
    a_{m1} & \cdots & a_{mn} \\
  \end{pmatrix} \]
se denota por
\begin{center}
  \texttt{[$a_{11}$,$\ldots$,$a_{1n}$; $\ldots$; $a_{m1}$,$\ldots$,$a_{mn}$]}
\end{center}
Es decir, las entradas de la misma fila se separan por la coma ``\texttt{,}'' y
las diferentes filas se separan por el punto y coma ``\texttt{;}''. Las entradas
de matrices pueden ser números enteros, racionales, números módulo $n$,
polinomios, etc.

\begin{itemize}
\item La suma y producto de matrices $a$ y $b$ se calculan mediante
  \texttt{$a$+$b$} y \texttt{$a$*$b$}. La matriz $a^n$ se calcula mediante
  \texttt{$a$\^{}$n$}, y en particular la matriz inversa se calcula mediante
  \texttt{$a$\^{}-1}. He aquí un pequeño ejemplo:
\end{itemize}

\begin{framed}\footnotesize
\begin{verbatim}
? a = [0,1,1; 0,0,1; 0,0,0];
% = 
[0 1 1]

[0 0 1]

[0 0 0]

? a^2
% = 
[0 0 1]

[0 0 0]

[0 0 0]

? a^3
% = 
[0 0 0]

[0 0 0]

[0 0 0]

? [1,1,1; 0,1,1; 0,0,1]^-1
% = 
[1 -1  0]

[0  1 -1]

[0  0  1]
\end{verbatim}
\end{framed}

\begin{itemize}
\item \texttt{matdet\,($a$)} calcula el determinante de $a$.

\item \texttt{trace\,($a$)} calcula la traza de $a$.

\item \texttt{matsize\,($a$)} devuelve $[m,n]$ si $a$ es una matriz de
  $m\times n$.

\item \texttt{$a$[$i$,$j$]} devuelve el elemento $a_{ij}$ de la matriz
  (las filas y columnas se numeran a partir de $1$).
\end{itemize}

Hay muchas más funciones relacionadas con matrices; la mayoría tienen nombre que
empieza por ``\texttt{mat}''. El lector interesado puede consultar
la documentación de \gp{} (sección ``Vectors, matrices, linear algebra and
sets'').

% % % % % % % % % % % % % % % % % % % % % % % % % % % % % %

\section{Sumas y productos}

Para calcular sumas y productos, pueden servir las siguientes funciones.

\begin{itemize}
\item \texttt{sum\,($i$=$m$,$n$,$a_i$)} calcula la suma
  $\sum\limits_{m \le i \le n} a_i$, donde $a_i$ es una expresión que depende de
  $i$.

\item \texttt{sumdiv\,($n$,$d$,$a_d$)} calcula la suma
  $\sum\limits_{d \mid n} a_d$ indexada por los divisores $d \mid n$, donde
  $a_d$ es una expresión que depende de $d$.

\item \texttt{prod\,($i$=$m$,$n$,$a_i$)} calcula el producto
  $\prod\limits_{m \le i \le n} a_i$, donde $a_i$ es una expresión que depende
  de $i$.
\end{itemize}

He aquí algunos ejemplos.

\begin{framed}\footnotesize
\begin{verbatim}
? sum (i=1,4,i^2)
% = 30
? sum (i=1,16, kronecker(i,17))
% = 0
? sumdiv (100,d, eulerphi(d))
% = 100
? prod (i=1,10,i)
% = 3628800
\end{verbatim}
\end{framed}

% % % % % % % % % % % % % % % % % % % % % % % % % % % % % %

\section{Ciclos y expresiones condicionales}

Para escribir pequeños programas, pueden ser útiles ciclos y expresiones
condicionales. He aquí algunas de las funciones relevantes.

\begin{itemize}
\item \texttt{for\,($i$=$m$,$n$, $expr_1$; $expr_2$; $\ldots$)} ejecuta
  las expresiones $expr_1, expr_2, \ldots$ para $m \le i \le n$. Las diferentes
  expresiones que tienen que ejecutarse se separan por el punto y coma
  ``\texttt{;}''.

  Cuando el ciclo toma demasiado tiempo y perdimos la paciencia, podemos
  interrumpir la ejecución presionando \boxed{\text{Ctrl}}+\boxed{\text{D}}.

\item \texttt{forprime\,($p$=$m$,$n$, $expr_1$; $expr_2$; $\ldots$)} hace
  lo mismo, pero para los números primos $m \le p\le n$.

\item \texttt{if\,(condición, $expr$, $expr'$)} ejecuta la expresión $expr$ si
  la condición es verdadera y $expr'$ en el caso contrario. El argumento $expr'$
  puede ser omitido.

  Las condiciones lógicas normalmente se construyen usando los operadores
  \texttt{!} (negación), \texttt{\&\&} (conjunción), \texttt{||} (disyunción) y
  las comparaciones \texttt{==} (igual\footnote{No confunda \texttt{$x$==$y$}
    (verifica si $x$ es igual a $y$) con \texttt{$x$=$y$} (asigna a $x$ el valor
    de $y$).}), \texttt{!=} (no igual), \texttt{<}, \texttt{<=}, \texttt{>},
  \texttt{>=}.

\item En fin, la función \texttt{print\,($x$)} sirve para imprimir el valor de
  $x$. Véase también la documentación sobre la función \texttt{printf}.
\end{itemize}

\subsection*{Ejemplo elaborado}

Para dar un ejemplo interesante de experimentos numéricos en \gp{}, analicemos
cómo el polinomio
$$f \dfn x^3 + x^2 + 1$$
puede factorizarse en $\FF_p [x]$. Primero escribamos un programa para encontrar
los números primos $p \le 100$ tales que $f$ es irreducible en $\FF_p [x]$.

\begin{framed}\footnotesize
\begin{verbatim}
? f = x^3 + x^2 + 1;
? forprime (p=2,100, if (polisirreducible(f*Mod(1,p)), print (p)));
2
5
7
19
41
59
71
97
\end{verbatim}
\end{framed}

He aquí otro programa que calcula la proporción de los primos $p$ tales que $f$
es irreducible en $\FF_p [x]$. En este caso podemos considerar
$2 \le p \le 10^6$. De hecho, hay $78\,498$ primos $\le 10^6$, lo que será
suficiente para nuestros propósitos.

\begin{framed}\footnotesize
\begin{verbatim}
? n=0; m=0;
? forprime (p=2,10^6, n++; if (polisirreducible(f*Mod(1,p)), m++));
? m/n*1.0
% = 0.33352442100435679889933501490483833983
\end{verbatim}
\end{framed}

Entonces, aproximadamente $\frac{1}{3}$ de los primos tienen esta propiedad.

Siendo un polinomio cúbico, si $f$ se vuelve reducible módulo $p$, este tiene
por lo menos una raíz. ¿Cuándo hay raíces múltiples? Notamos que si
$\alpha \in \FF_p$ es una raíz múltiple, entonces $f' (\alpha) = 0$. Pero
$f' = 3\,x^2 + 2x$ tiene raíces $0$ y $-\frac{2}{3}$ para $p \ne 3$. Ahora $0$
no es una raíz de $f$, mientras que
\[ \left(-\frac{2}{3}\right)^3 + \left(-\frac{2}{3}\right)^2 + 1
  = \frac{31}{3^3} \equiv 0 \pmod{p}
  \quad\text{si y solamente si }p = 31. \]
Entonces, con la única excepción de $p = 31$, las raíces de $f$ son distintas, y
si $f$ es reducible, este puede factorizarse de dos maneras:
\[ f = (x-\alpha_1)\,(x-\alpha_2)\,(x-\alpha_3)
  \quad\text{o}\quad
  f = (x-\alpha)\,g
  \quad\text{en }\FF_p [x], ~ p\ne 31, \]
donde $g$ es un polinomio cuadrático irreducible. El caso cuando $f$ es un
producto de tres distintos polinomios lineales ocurre raramente: los primeros
primos con esta propiedad son $47$ y $67$.

\begin{framed}\footnotesize
\begin{verbatim}
? forprime (p=2,10^2, if (matsize (factor (f*Mod(1,p)))[1] == 3, print(p)));
47
67
\end{verbatim}
\end{framed}

Contemos cuál es la proporción de los primos con esta propiedad.

\begin{framed}\footnotesize
\begin{verbatim}
? n = 0; m = 0;
? forprime (p=2,10^6, n++; if (matsize (factor (f*Mod(1,p)))[1] == 3, m++));
? m/n*1.0
% = 0.16564753242120818364799103161863996535
\end{verbatim}
\end{framed}

Entonces, $f$ es un producto de tres polinomios lineales en $\FF_p [x]$ para
aproximadamente $\frac{1}{6}$ de los primos $p$. Nos quedan
$1 - \frac{1}{3} - \frac{1}{6} = \frac{1}{2}$ de los primos, y estos darán
factorizaciones de la forma $(x-\alpha)\,g$, donde $g$ es cuadrático
irreducible. Podemos concluir nuestras investigaciones de la siguiente manera.

\begin{center}
  \begin{tabular}{f{6cm}f{5cm}}
    \hline
    \textbf{tipo de factorización de $f$ mód $p$} & \textbf{proporción de primos $p$} \tabularnewline
    \hline
    $(x-\alpha_1)\,(x-\alpha_2)\,(x-\alpha_3)$ & $\frac{1}{6}$ \tabularnewline
    \hline
    $(x-\alpha)\,g$, donde $\deg g = 2$ & $\frac{1}{2}$ \tabularnewline
    \hline
    $f$ (irreducible) & $\frac{1}{3}$ \tabularnewline
    \hline
    $(x-\alpha_1)\,(x-\alpha_2)^2$ & solo $p = 31$ (excepcional) \tabularnewline
    \hline
  \end{tabular}
\end{center}

La aparición de los números $\frac{1}{3}$, $\frac{1}{6}$, $\frac{1}{2}$ se
explica\footnote{La explicación que podremos entender mucho más adelante: el
  grupo de Galois del polinomio $x^3 + x^2 + 1 \in \QQ [x]$ es el grupo
  simétrico $S_3$ que tiene $3! = 6$ elementos que se descomponen en tres clases
  de conjungación:
  $$\{ \idid \} \sqcup \{ (1~2), \, (1~3), \, (2~3) \} \sqcup \{ (1~2~3), \, (1~3~2) \}.$$
  De aquí salen los números $\frac{1}{6}$, $\frac{3}{6}$, $\frac{2}{6}$.} por el
\term{teorema de Frobenius} y el \term{teorema de densidad de Chebotarëv}. Un
buen artículo sobre el tema es \cite{Lenstra-Stevenhagen-1996}.

\iffalse
% FOR LATER: algdep, norm, trace, etc.

Una función curiosa es \texttt{algdep($z$,$n$)}. Para un número complejo $z$
esta devuelve un polinomio de grado $\le n$ con coeficientes enteros tal que una
de sus raíces está cerca de $z$. Por ejemplo, el comando
\texttt{algdep\,(sqrt(2)+sqrt(3),\,4)} devuelve el polinomio
$x^4 - 10\,x^2 + 1$, y de hecho sus cuatro raíces son precisamente
$\pm (\sqrt{2}+\sqrt{3}), \pm (\sqrt{2}-\sqrt{3})$. De la misma manera,
\texttt{algdep\,(exp(2*Pi*I/10),\,4)} devuelve, como uno puede esperar,
el décimo polinomio ciclotómico $x^4 - x^3 + x^2 - x + 1$.

\begin{framed}\footnotesize
\begin{verbatim}
? algdep (sqrt(2)+sqrt(3),4)
% = x^4 - 10*x^2 + 1
? algdep (exp(2*Pi*I/10),4)
% = x^4 - x^3 + x^2 - x + 1
\end{verbatim}
\end{framed}

Sin embargo, tenemos lo siguiente:

\begin{framed}\footnotesize
\begin{verbatim}
? algdep (sqrt(2)+sqrt(3),3)
% = 10948540*x^3 - 44677367*x^2 + 12077522*x + 63271103
? algdep (Pi,3)
% = 2130241*x^3 - 22294338*x^2 + 51516201*x - 7857464
\end{verbatim}
\end{framed}

Esto sucede porque $\sqrt{2} + \sqrt{3}$ no es una raíz de ningún polinomio
cúbico, y el resultado de \gp{} es una aproximación. El número $\pi$ es
\term{trascendente}; es decir, no es una raíz de ningún polinomio
$f \in \QQ [x]$. La función \texttt{algdep} es útil para detectar los
\term{números algebraicos}; es decir los números complejos que son raíces de
algún polinomio $f \in \QQ [x]$.  \fi
