\begin{ejercicio}
  Sean $A_i$, $i \in I$ y $B$ conjuntos.

  \begin{enumerate}
  \item[a)] Demuestre que
    \[ \Bigl(\bigcap_{i\in I} A_i\Bigr) \cup B = \bigcap_{i\in I} (A_i\cup B)
      \quad\text{y}\quad
      \Bigl(\bigcup_{i\in I} A_i\Bigr) \cap B = \bigcup_{i\in I} (A_i\cap B). \]

  \item[b)] Demuestre que si $A_i \subseteq X$ para todo $i \in I$, entonces
    \[ X \setminus \bigcup_{i\in I} A_i = \bigcap_{i\in I} (X\setminus A_i)
      \quad\text{y}\quad
      X \setminus \bigcap_{i\in I} A_i = \bigcup_{i\in I} (X\setminus A_i). \]
    \end{enumerate}
\end{ejercicio}

\begin{ejercicio}
  Sea $X$ un conjunto. Para dos subconjuntos $A, B \subseteq X$ su
  \term{diferencia simétrica} se define por
  $$A\triangle B \dfn ( A \cup B ) \setminus ( A \cap B ).$$

  \begin{enumerate}
  \item[a)] Demuestre que $A\triangle A = \emptyset$ y $A\triangle \emptyset = A$.

  \item[b)] Demuestre que $A\triangle B = (A \setminus B ) \cup (B \setminus A)$.

  \item[c)] Demuestre que $(A\triangle B)\triangle C = A\triangle (B\triangle C)$.

    Encuentre una fórmula simétrica en $A,B,C$ para este conjunto.

  \item[d)] Demuestre que $(A \triangle B) \cap C = (A \cap C) \triangle (B \cap C)$.
  \end{enumerate}
\end{ejercicio}

\begin{ejercicio}
  Sean $f\colon X\to Y$ una aplicación, $X_i \subseteq X$, $Y_j \subseteq Y$
  familias de subconjuntos.

  \begin{enumerate}
  \item[a)] Demuestre que
    \[ f \Bigl(\bigcup_{i\in I} X_i\Bigr) = \bigcup_{i\in I} f (X_i)
      \quad\text{y}\quad
      f \Bigl(\bigcap_{i\in I} X_i\Bigr) \subseteq \bigcap_{i\in I} f (X_i). \]
    Encuentre un ejemplo cuando $f (X_1\cap X_2) \subsetneq f (X_1) \cap f (X_2)$.

  \item[b)] Demuestre que
    \[ f^{-1} \Bigl(\bigcup_{j\in J} Y_j\Bigr) = \bigcup_{j\in J} f^{-1} (Y_j)
      \quad\text{y}\quad
      f^{-1} \Bigl(\bigcap_{j\in J} Y_j\Bigr) = \bigcap_{j\in J} f^{-1} (Y_j). \]
  \end{enumerate}
\end{ejercicio}

\begin{ejercicio}
  Sea $f\colon X\to Y$ una aplicación entre conjuntos. Demuestre las siguientes
  propiedades.

  \begin{enumerate}
  \item[a)] Para cualquier subconjunto $B \subseteq Y$ se tiene
    $f (f^{-1} (B)) \subseteq B$.

    Además, si $f$ es sobreyectiva, entonces $f (f^{-1} (B)) = B$.

  \item[b)] Para cualquier subconjunto $A \subseteq X$ se tiene
    $A \subseteq f^{-1} (f (A))$.

    Además, si $f$ es inyectiva, entonces $f^{-1} (f (A)) = A$.

  \item[c)] Si $A_1 \subseteq A_2 \subseteq X$, entonces
    $f (A_1) \subseteq f (A_2)$.

  \item[d)] Si $B_1 \subseteq B_2 \subseteq Y$, entonces
    $f^{-1} (B_1) \subseteq f^{-1} (B_2)$.
  \end{enumerate}
\end{ejercicio}

\begin{ejercicio}
  Sean $X$ e $Y$ conjuntos finitos.

  \begin{enumerate}
  \item[a)] ¿Cuántos elementos tiene $X\times Y$ e $X\sqcup Y$?

  \item[b)] ¿Cuántos subconjuntos tiene $X$?

  \item[c)] ¿Cuántas aplicaciones distintas $X\to Y$ hay?

  \item[d)] ¿Cuántas biyecciones distintas $X\to X$ hay?
  \end{enumerate}
\end{ejercicio}

\begin{ejercicio}
  Sea $X = \{ 1,2,3 \}$ un conjunto de tres elementos. Describa todas las
  biyecciones $X\to X$. Compile la tabla de composición de estas biyecciones:
  \begin{center}
    \begin{tabular}{x{1cm}|x{1cm}x{1cm}x{1cm}}
      & $\cdots$ & $f$ & $\cdots$ \tabularnewline
      \hline
      $\vdots$ & & $\vdots$ \tabularnewline
      $g$ & $\cdots$ & $g\circ f$ & $\cdots$ \tabularnewline
      $\vdots$ & & $\vdots$ \tabularnewline
    \end{tabular}
  \end{center}
\end{ejercicio}

\begin{ejercicio}
  Encuentre una biyección entre el conjunto de los números enteros $\ZZ$ y algún
  subconjunto propio $X \subsetneq \ZZ$.
\end{ejercicio}

\begin{ejercicio}
  Sean $f\colon X\to Y$ e $g\colon X\to Z$ dos aplicaciones
  biyectivas. Demuestre que la aplicación
  \begin{align*}
    X & \to Y\times Z,\\
    x & \mapsto (f (x), g (x))
  \end{align*}
  no es biyectiva si $X$ tiene más de un elemento.
\end{ejercicio}

\begin{ejercicio}
  Sean $X$ un conjunto finito y $f\colon X\to X$ una aplicación. Demuestre que
  las siguientes condiciones son equivalentes:
  \begin{enumerate}
  \item[a)] $f$ es inyectiva;
  \item[b)] $f$ es sobreyectiva;
  \item[c)] $f$ es biyectiva.
  \end{enumerate}
  Encuentre un contraejemplo para $X$ infinito.
\end{ejercicio}

\begin{ejercicio}
  Sea $f\colon X\to Y$ una aplicación entre conjuntos. Definamos la siguiente
  relación de equivalencia sobre los elementos de $X$:
  $$x \sim x' \iff f (x) = f (x')$$
  Demuestre que $f$ induce una biyección
  \begin{align*}
    X/\!\sim & \to f (X),\\
    {}[x] & \mapsto f (x).
  \end{align*}
\end{ejercicio}

\begin{ejercicio}
  Sea $f\colon X\to Y$ una aplicación.

  \begin{enumerate}
  \item[a)] Asumamos que $X \ne \emptyset$. Demuestre que $f$ es inyectiva si y
    solo si existe una aplicación $r\colon Y\to X$ tal que $r\circ f = \id{X}$.

  \item[b)] Demuestre que $f$ es sobreyectiva si y solo si existe una aplicación
    $s\colon Y\to X$ tal que $f\circ s = \id{Y}$\footnote{De hecho, este
      resultado es equvalente al \term{axioma de elección}.}.
  \end{enumerate}
\end{ejercicio}
