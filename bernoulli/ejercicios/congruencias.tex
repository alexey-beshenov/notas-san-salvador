\documentclass{article}

\usepackage[utf8]{inputenc}
\usepackage[sc]{mathpazo}
\linespread{1.05}
\usepackage[T1]{fontenc}

\usepackage[spanish]{babel}

\usepackage{amsmath,amsthm,amssymb,hyperref,graphicx,cancel,fullpage}

\newcommand{\dfn}{\mathrel{\mathop:}=}
\newcommand{\ZZ}{\mathbb{Z}}
\newcommand{\term}{\textbf}
\DeclareMathOperator{\ord}{ord}
\DeclareMathOperator{\mcd}{mcd}

\author{}
\title{Ejercicios sobre congruencias}
\date{6 de marzo de 2017}

\theoremstyle{plain}
\newtheorem{ejerc}{Ejercicio}
\renewcommand{\qedsymbol}{$\blacksquare$}

\newif\ifsol
\soltrue
%\solfalse

\begin{document}

\maketitle

\begin{ejerc}
Fijemos algún $n = 1,2,3,4,\ldots$ Consideremos la siguiente relación sobre los números enteros: se dice que $x$ es \term{congruente a $y$ módulo $n$} si $n$ divide a $x-y$:
$$x \equiv y \pmod{n} ~\Leftrightarrow~ n \mid (x-y).$$
En otras palabras, $x$ e $y$ tienen el mismo resto de la división por $n$.

Demuestre que la congruencia módulo $n$ es una relación de equivalencia sobre $\ZZ$; es decir, para todo $x,y,z\in \ZZ$
$$x\equiv x, \quad x\equiv y \Rightarrow y\equiv x, \quad x\equiv y \text{ e } y\equiv z \Rightarrow x\equiv z.$$
\end{ejerc}

\ifsol\begin{proof}[Solución (Martha)]
Obviamente, $n \mid x-x = 0$ para todo $x$. Luego, $n \mid (x-y)$ si y solamente si $n \mid (y-x)$. Por fin, si $n \mid (x-y)$ y $n\mid (y-z)$, entonces
$$n \mid (x-y) + (y-z) = x-z.$$
\end{proof}\fi

Las clases de equivalencia se llaman los \term{residuos módulo $n$}. La clase de equivalencia de $x$ se denota por $[x]$. Note que hay $n$ residuos diferentes: $[0], ~ [1], ~ [2], ~ \ldots, ~ [n-1]$.

\begin{ejerc}
Demuestre que si $x \equiv x'$, $y \equiv y'$, entonces

\begin{align*}
x+y & \equiv x'+y',\\
x\cdot y & \equiv x'\cdot y'.
\end{align*}

Esto quiere decir que la adición y multiplicación tiene sentido para residuos módulo $n$: podemos definir

\begin{align}
\label{suma-de-residuos} [x] + [y] & \dfn [x+y],\\
\label{producto-de-residuos} [x]\cdot [y] & \dfn [x\cdot y].
\end{align}
\end{ejerc}

\ifsol\begin{proof}[Solución (Javier)]
Si $n \mid x-x'$ y $n \mid y-y'$, entonces
$$n \mid (x-x') + (y-y') = (x+y) - (x'-y'),$$
y también
$$n \mid (x-x')\,y + x'\,(y-y') = xy - x'y'.$$

Para ver que las definiciones \eqref{suma-de-residuos} y \eqref{producto-de-residuos} tienen sentido (no dependen de los representantes particulares de las clases de equivalencia), tenemos que verificar que si $[x] = [x']$ y $[y] = [y']$, entonces

\begin{align*}
[x + y] & = [x' + y'],\\
[x\cdot y] & = [x'\cdot y'].
\end{align*}

\noindent Es lo que acabamos de demostrar.
\end{proof}\fi

\begin{ejerc}
\label{cancelacion}
Demuestre que tiene sentido la cancelación módulo $p$: si tenemos
$$x\cdot y \equiv x\cdot y' \pmod{p},$$
donde $p \nmid x$, entonces $y \equiv y' \pmod{p}$.
\end{ejerc}

\noindent\emph{Indicación: si $p$ es primo, entonces $p\mid xy$ implica $p\mid x$ o $p\mid y$.}

\ifsol\begin{proof}[Solución (Raúl)]
$x\cdot y \equiv x\cdot y' \pmod{p}$ significa que $p\mid (x\cdot y - x\cdot y') = x\cdot (y-y')$. Pero $p\nmid x$, y por lo tanto $p\mid (y-y')$.
\end{proof}\fi

\begin{ejerc}
\label{coeficientes-binomiales-divisibles-por-p}
Sea $p$ un número primo. Demuestre que los coeficientes binomiales ${p\choose 1}, {p \choose 2}, \ldots, {p\choose p-1}$ son divisibles por $p$:
$$p\mid {p \choose i} = \frac{p!}{i! \, (p-i)!}\quad\text{para }i = 1,2,\ldots,p-1.$$
Para $i = 0,p$ tenemos ${p\choose 0} = {p\choose p} = 1$.

\begin{center}
\includegraphics{../pic/pascal.mps}
\end{center}
\end{ejerc}

\ifsol\begin{proof}[Solución (Ariel)]
Para $i = 1,2,\ldots,p-1$ en el número
$${p\choose i} = \frac{p\,(p-1)\,(p-2)\cdots (p-i+1)}{i!}$$
$p$ aparece en el numerador, pero $p \nmid i!$, y por lo tanto $p \mid {p\choose i}$.
\end{proof}\fi

\begin{ejerc}
Demuestre que ${p-1\choose i} \equiv (-1)^i \pmod{p}$.
\end{ejerc}

\noindent Por ejemplo, ${6\choose 3} = 20 \equiv 6 \equiv -1 \pmod{7}$ y ${6\choose 2} = 15 \equiv 1 \pmod{7}$.

\noindent\emph{Indicación: del ejercicio \ref{coeficientes-binomiales-divisibles-por-p} sabemos que ${p \choose i} \equiv 0 \pmod{p}$ para $i = 1,2,\ldots,p-1$; luego, use la relación de Pascal ${p \choose i} = {p-1\choose i} + {p-1 \choose i-1}$.}

\ifsol\begin{proof}[Primera solución (Dennis)]
Vamos a usar la existencia de elementos inversos módulo $p$ (véase abajo). 
Para $i = 0$ la fórmula es evidente: ${p-1\choose 0} = 1$. Para $i\ne 0$, tenemos
$${p-1\choose i} = \frac{(p-1)\,(p-2)\cdots (p-i)}{i!},$$
donde módulo $p$,
$$i! = 1\cdot 2\cdot 3\cdots i \equiv (1-p)\,(2-p)\cdots (i-p) = (-1)^i\,(p-1)\,(p-2)\cdots (p-i),$$
así que
$${p-1\choose i} = \frac{(p-1)\,(p-2)\cdots (p-i)}{i!} \equiv \frac{\cancel{(p-1)}\,\cancel{(p-2)}\cdots \cancel{(p-i)}}{(-1)^i\,\cancel{(p-1)}\,\cancel{(p-2)}\cdots \cancel{(p-i)}} = (-1)^i.$$
\end{proof}\fi

\ifsol\begin{proof}[Segunda solución]
El valor inicial es
$${p-1\choose 0} = 1.$$
Luego, para $i = 1,\ldots,p-1$ tenemos
$${p-1 \choose i} + {p-1 \choose i-1} = {p\choose i} \equiv 0 \pmod{p},$$
es decir,

\begin{align*}
{p-1 \choose 1} & \equiv -{p-1\choose 0} \equiv -1 \pmod{p},\\
{p-1 \choose 2} & \equiv -{p-1\choose 1} \equiv +1 \pmod{p},\\
 & \cdots \\
{p-1 \choose i} & \equiv (-1)^i \pmod{p}.
\end{align*}

\vspace{1em}
\end{proof}\fi

\begin{ejerc}
Demuestre \term{el teorema del binomio módulo $p$}: para $p$ primo se tiene
$$(x+y)^p \equiv x^p + y^p \pmod{p}.$$
\end{ejerc}

\noindent Por ejemplo, $(2+2)^3 = 64 \equiv 1 \pmod{3}$ y $2^3 + 2^3 = 16 \equiv 1 \pmod{3}$.

\noindent\emph{Indicación: use el ejercicio \ref{coeficientes-binomiales-divisibles-por-p}.}

\ifsol\begin{proof}[Solución (Alejandra)]
El teorema del binomio nos dice que
$$(x+y)^p = \sum_{0 \le i \le p} {p \choose i} x^{p-i} \, y^i,$$
pero módulo $p$ los coeficientes ${p \choose i}$ son nulos, excepto ${p \choose 0} = {p\choose p} = 1$.
\end{proof}\fi

\begin{ejerc}
\label{pequeno-fermat}
Demuestre el \term{pequeño teorema de Fermat}: para todo $x \in \ZZ$ se tiene
$$x^p \equiv x \pmod{p};$$
y si $p \nmid x$, entonces $x^{p-1} \equiv 1 \pmod{p}$.
\end{ejerc}

\noindent Por ejemplo, $2^3 = 8 \equiv 2 \pmod{3}$, $2^2 = 4 \equiv 1 \pmod{3}$.

\noindent\emph{Indicación: podemos suponer que la clase de equivalencia $[x]$ representada por algún número $x = 0,1,2,\ldots, p-1$. Si $x = 0$, el resultado está claro. Demuestre el paso de inducción: si $[x]^p = [x]$, entonces $[x+1]^p = [x+1]$.}

\ifsol\begin{proof}[Primera solución (Martha)]
Vamos a usar el siguiente

\noindent\textbf{Lema}. Si $p \nmid x$, entonces los múltiplos de $[x]$ nos dan todos los residuos módulo $p$:
$$\{ [0\cdot x], ~ [1\cdot x], ~ [2\,x], ~ \ldots, ~ [(p-1)\,x] \} = \{ [0], ~ [1], ~ \ldots, ~ [p-1] \}.$$
Efectivamente, tenemos que ver que los residuos $[n\,x]$ son diferentes para $n = 0,1,\ldots, p-1$. Y de hecho, si $[m\,x] = [n\,x]$, entonces $[m] = [n]$ por la propiedad de cancelación (ejercicio \ref{cancelacion}).

\vspace{1em}

Ahora por este lema, si $p\nmid x$, entonces
$$x\cdot (2\,x)\cdots (p-1)\,x = (p-1)! \, x^{p-1} \equiv 1\cdot 2\cdots (p-1) = (p-1)! \pmod{p}.$$
Luego, $p\nmid (p-1)!$, así que podemos cancelar $(p-1)!$ y concluir que
$$x^{p-1} \equiv 1 \pmod{p}.$$
Multiplicando esta identidad por $x$, tenemos
$$x^p \equiv x \pmod{p}.$$

Si $p\mid x$, es decir $x \equiv 0 \pmod{p}$, entonces la identidad
$$x^p \equiv x \pmod{p}$$
es evidente ($0^p = 0$).
\end{proof}\fi

\ifsol\begin{proof}[Segunda solución]
En efecto, $0^p = 0$. Luego, el paso de inducción es (usando el ejercicio anterior)
$$[x+1]^p = [(x+1)^p] = [x^p + 1^p] = [x^p] + [1] = [x]^p + [1] = [x] + [1] = [x+1].$$

Si $p\nmid x$, entonces $x^p \equiv x \pmod{p}$ implica $x^{p-1} \equiv 1$ gracias al ejercicio \ref{cancelacion}.
\end{proof}\fi

\begin{ejerc}
Demuestre que si $p\nmid x$, entonces existe $y\in \ZZ$ (definido de modo único módulo $p$) tal que $x\,y \equiv 1 \pmod{p}$. En este caso escribimos $[x]^{-1} = [y]$.
\end{ejerc}

\noindent\emph{Indicación: use el ejercicio \ref{pequeno-fermat}.}

\ifsol\begin{proof}[Solución]
Si $p\nmid x$, entonces $x^{p-1} = x\cdot x^{p-2} \equiv 1 \pmod{p}$, así que $x^{p-2}$ es inverso a $x$ módulo $p$.

Para ver que el inverso es único módulo $p$, notamos que $x\,y \equiv 1$, $x\,y' \equiv 1$ implica $y\equiv y'$ por la propiedad de cancelación.
\end{proof}\fi

\ifsol\begin{proof}[Otra solución]
Por el \textbf{Lema} de arriba, si $p\nmid x$, entonces la multiplicación por $[x]$ nos da una biyección

\begin{align*}
\text{residuos módulo }p & \to \text{residuos módulo }p,\\
[y] & \mapsto [y]\cdot [x].
\end{align*}

En particular, para el residuo $[1]$ existe único $[y]$ tal que $[y]\cdot [x] = 1$.
\end{proof}\fi

\begin{ejerc}
\begin{enumerate}
\item[1)] Demuestre que $1 + 2 + 3 + \cdots + (p-1) \equiv 0 \pmod{p}$ para $p \ne 2$.

Por ejemplo, $1+2+3+4 = 10 \equiv 0 \pmod{5}$.

\item[2)] Demuestre que $1^2 + 2^2 + 3^2 + \cdots + (p-1)^2 \equiv 0 \pmod{p}$ para $p \ne 2,3$.

Por ejemplo, $1^2+2^2+3^2+4^2 = 30 \equiv 0 \pmod{5}$.

\item[3)] Demuestre que $1^3 + 2^3 + 3^3 + \cdots + (p-1)^3 \equiv 0 \pmod{p}$ para $p \ne 2$.

Por ejemplo, $1^3 + 2^3 + 3^3 + 4^3 = 100 \equiv 0 \pmod{5}$.

\item[4)] En general, dado $k$ fijo, ¿para cuáles $p$ se va a cumplir $1^k + 2^k + 3^k + \cdots + (p-1)^k \equiv 0 \pmod{p}$?
\end{enumerate}
\end{ejerc}

\ifsol\begin{proof}[Solución]
En 1), como notó Gauss cuando estudiaba en la primaria, tenemos
$$1 + 2 + 3 + \cdots + (p-1) = (1 + (p-1)) + (2 + (p-2)) + (3 + (p-3)) + \cdots,$$
que es visiblemente divisible por $p$.

En 2), Dennis nos recordó la fórmula
$$1^2 + 2^2 + 3^2 + \cdots + p^2 = \frac{p\,(p+1)\,(2p+1)}{6}.$$
Si $p\nmid 6$, entonces esta fórmula implica que la suma de cuadrados es divisible por $p$.

En general, para $S_k (n) \dfn \sum_{0 \le i \le n} i^k$ y $k = 1,2,3,4, \ldots$ tenemos fórmulas

\begin{align*}
S_1 (n) & = \frac{1}{2}\,n^2\,+\frac{1}{2}\,n,\\
S_2 (n) & = \frac{1}{3}\,n^3+\frac{1}{2}\,n^2+\frac{1}{6}\,n,\\
S_3 (n) & = \frac{1}{4}\,n^4+\frac{1}{2}\,n^3+\frac{1}{4}\,n^2,\\
 & \cdots
\end{align*}

\noindent(véase mi primera lección). Cuando $p$ no aparece en los denominadores, las fórmulas de arriba nos dicen que la suma es divisible por $p$ (el término constante de $S_k (n)$ es siempre nulo). En general, se puede analizar la expresión
$$S_k (n) = \frac{1}{k+1}\,\sum_{0 \le i \le k} {k+1\choose i}\,B_i\,n^{k+1-i}$$
y ver cuáles números primos aparecen en los denominadores.
\end{proof}\fi

\vspace{1em}

Se dice que un número $x$ es una \term{raíz primitiva de la unidad módulo $p$} si las potencias de $x$ nos dan todos los residuos no nulos módulo $p$:
$$\{ [x], ~ [x]^2, ~ [x]^3, ~ [x]^4, \ldots \} = \{ [1], ~ [2], ~ [3], ~ \ldots, ~ [p-1] \}.$$
Por ejemplo, $2$ es una raíz primitiva de la unidad módulo $5$:
$$\{ [2], ~ [2]^2, ~ [2]^3, ~ [2]^4 \} = \{ [2], ~ [4], ~ [8], ~ [16] \} = \{ [2], ~ [4], ~ [3], ~ [1] \}$$
Módulo todo número primo $p$ existen raíces primitivas de la unidad, pero no es algo obvio y por el momento podemos aceptar este resultado (esto se demuestra en cursos de álgebra).

\begin{ejerc}
Si $x$ es un número entero tal que $p\nmid x$, entonces el \term{orden de $x$ módulo $p$} es el mínimo número natural positivo $k = 1,2,3,4,\ldots$ tal que $x^k \equiv 1 \pmod{p}$. En este caso escribimos $\ord_p (x) = k$.

\begin{enumerate}
\item[1)] Verifique que $\ord_p (x) \le p-1$ y que la existencia de raíces primitivas módulo $p$ quiere decir que existe algún $x$ de orden $p-1$.

\item[2)] Demuestre que $x^k \equiv 1 \pmod{p}$ si y solamente si $\ord_p (x) \mid k$. En particular, $\ord_p (x) \mid p-1$.

Indicación: si $x^k \equiv 1$, la división con resto nos da $k = n\,\ord_p (x) + r$, donde $r < \ord_p (x)$.

\item[3)] Demuestre que $\ord_p (x^k) = \frac{\ord_p (x)}{\mcd (k, \ord_p (x))}$.

\item[4)] Demuestre que
$$1^k + 2^k + \cdots + (p-1)^k \equiv 0 \pmod{p}$$
si $p-1 \nmid k$.
Por ejemplo,
$$1^3 + 2^3 + 3^3 + 4^3 = 100 \equiv 0 \pmod{5}.$$

Indicación: use la existencia de una raíz primitiva de la unidad módulo $p$.
\end{enumerate}
\end{ejerc}

\ifsol\begin{proof}[Solución (Rodrigo, Dennis)]
$x^{p-1} \equiv 1 \pmod{p}$ por el pequeño teorema de Fermat, así que $\ord_p (x) \le p-1$. Luego, si $\ord_p (x) = p-1$, las potencias de $x$ módulo $p$
$$[x], ~ [x]^2, ~ [x]^3, ~ \ldots, ~ [x]^{p-1}$$
son diferentes y nos dan todos los restos módulo $p$ (en efecto, si $[x]^m = [x]^n$ para $m > n$, entonces $[x]^{m-n} = [1]$).

En 2), para ver que $x^k \equiv 1 \pmod{p}$ implica $\ord_p (x) \mid k$, podemos usar la división con resto: $k = \ord_p (x) + r$ para algún $0 \le r < \ord_p (x)$. Luego, $x^k = x^{\ord_p (x)} \, x^r \equiv x^r \equiv 1 \pmod{p}$, pero $\ord_p (x)$ es el mínimo número positivo tal que $x^{\ord_p (x)} \equiv 1 \pmod{p}$ y por lo tanto $r = 0$.

En 3) tenemos por la parte 2)

\begin{multline*}
(x^k)^m \equiv 1\pmod{p} \iff \ord_p (x) \mid km \\
\iff \frac{\ord_p (x)}{\mcd (k,\ord_p (x))} \mid \frac{km}{\mcd (k,\ord_p (x))} \iff \frac{\ord_p (x)}{\mcd (k,\ord_p (x))} \mid m.
\end{multline*}

En 4), como sugirió Rodrigo, podemos usar la fórmula para las sumas parciales de la serie geométrica:
$$1 + x^k + x^{2k} + \cdots + x^{(p-1)\,k} = \frac{1 - (x^k)^p}{1-x^k}.$$
Módulo $p$ tenemos, gracias al pequeño teorema de Fermat,
$$x^k + x^{2k} + \cdots + x^{(p-1)\,k} = \frac{1 - (x^k)^p}{1-x^k} - 1 \equiv \frac{1-x^k}{1-x^k} - 1 \equiv 0.$$
Aquí $x^k \not\equiv 1 \pmod{p}$ por nuestra hipótesis $p-1 \nmid k$.
\end{proof}\fi

\end{document}
