\documentclass{article}

% TODO : CLEAN UP THIS MESS
% (AND MAKE SURE ALL TEXTS STILL COMPILE)
\usepackage[leqno]{amsmath}
\usepackage{amssymb}
\usepackage{graphicx}

\usepackage{diagbox} % table heads with diagonal lines
\usepackage{relsize}

\usepackage{wasysym}
\usepackage{scrextend}
\usepackage{epigraph}
\setlength\epigraphwidth{.6\textwidth}

\usepackage[utf8]{inputenc}

\usepackage{titlesec}
\titleformat{\chapter}[display]
  {\normalfont\sffamily\huge\bfseries}
  {\chaptertitlename\ \thechapter}{5pt}{\Huge}
\titleformat{\section}
  {\normalfont\sffamily\Large\bfseries}
  {\thesection}{1em}{}
\titleformat{\subsection}
  {\normalfont\sffamily\large\bfseries}
  {\thesubsection}{1em}{}
\titleformat{\part}[display]
  {\normalfont\sffamily\huge\bfseries}
  {\partname\ \thepart}{0pt}{\Huge}

\usepackage[T1]{fontenc}
\usepackage{fourier}
\usepackage{paratype}

\usepackage[symbol,perpage]{footmisc}

\usepackage{perpage}
\MakePerPage{footnote}

\usepackage{array}
\newcolumntype{x}[1]{>{\centering\hspace{0pt}}p{#1}}

% TODO: the following line causes conflict with new texlive (!)
% \usepackage[english,russian,polutonikogreek,spanish]{babel}
% \newcommand{\russian}[1]{{\selectlanguage{russian}#1}}

% Remove conflicting options for the moment:
\usepackage[english,polutonikogreek,spanish]{babel}

\AtBeginDocument{\shorthandoff{"}}
\newcommand{\greek}[1]{{\selectlanguage{polutonikogreek}#1}}

% % % % % % % % % % % % % % % % % % % % % % % % % % % % % %
% Limit/colimit symbols (with accented i: lím / colím)

\usepackage{etoolbox} % \patchcmd

\makeatletter
\patchcmd{\varlim@}{lim}{\lim}{}{}
\makeatother
\DeclareMathOperator*{\colim}{co{\lim}}
\newcommand{\dirlim}{\varinjlim}
\newcommand{\invlim}{\varprojlim}

% % % % % % % % % % % % % % % % % % % % % % % % % % % % % %

\usepackage[all,color]{xy}

\usepackage{pigpen}
\newcommand{\po}{\ar@{}[dr]|(.4){\text{\pigpenfont I}}}
\newcommand{\pb}{\ar@{}[dr]|(.3){\text{\pigpenfont A}}}
\newcommand{\polr}{\ar@{}[dr]|(.65){\text{\pigpenfont A}}}
\newcommand{\pour}{\ar@{}[ur]|(.65){\text{\pigpenfont G}}}
\newcommand{\hstar}{\mathop{\bigstar}}

\newcommand{\bigast}{\mathop{\Huge \mathlarger{\mathlarger{\ast}}}}

\newcommand{\term}{\textbf}

\usepackage{stmaryrd}

\usepackage{cancel}

\usepackage{tikzsymbols}

\newcommand{\open}{\underset{\mathrm{open}}{\hookrightarrow}}
\newcommand{\closed}{\underset{\mathrm{closed}}{\hookrightarrow}}

\newcommand{\tcol}[2]{{#1 \choose #2}}

\newcommand{\homot}{\simeq}
\newcommand{\isom}{\cong}
\newcommand{\cH}{\mathcal{H}}
\renewcommand{\hom}{\mathrm{hom}}
\renewcommand{\div}{\mathop{\mathrm{div}}}
\renewcommand{\Im}{\mathop{\mathrm{Im}}}
\renewcommand{\Re}{\mathop{\mathrm{Re}}}
\newcommand{\id}[1]{\mathrm{id}_{#1}}
\newcommand{\idid}{\mathrm{id}}

\newcommand{\ZG}{{\ZZ G}}
\newcommand{\ZH}{{\ZZ H}}

\newcommand{\quiso}{\simeq}

\newcommand{\personality}[1]{{\sc #1}}

\newcommand{\mono}{\rightarrowtail}
\newcommand{\epi}{\twoheadrightarrow}
\newcommand{\xepi}[1]{\xrightarrow{#1}\mathrel{\mkern-14mu}\rightarrow}

% % % % % % % % % % % % % % % % % % % % % % % % % % % % % %

\DeclareMathOperator{\Ad}{Ad}
\DeclareMathOperator{\Aff}{Aff}
\DeclareMathOperator{\Ann}{Ann}
\DeclareMathOperator{\Aut}{Aut}
\DeclareMathOperator{\Br}{Br}
\DeclareMathOperator{\CH}{CH}
\DeclareMathOperator{\Cl}{Cl}
\DeclareMathOperator{\Coeq}{Coeq}
\DeclareMathOperator{\Coind}{Coind}
\DeclareMathOperator{\Cop}{Cop}
\DeclareMathOperator{\Corr}{Corr}
\DeclareMathOperator{\Cor}{Cor}
\DeclareMathOperator{\Cov}{Cov}
\DeclareMathOperator{\Der}{Der}
\DeclareMathOperator{\Div}{Div}
\DeclareMathOperator{\D}{D}
\DeclareMathOperator{\Ehr}{Ehr}
\DeclareMathOperator{\End}{End}
\DeclareMathOperator{\Eq}{Eq}
\DeclareMathOperator{\Ext}{Ext}
\DeclareMathOperator{\Frac}{Frac}
\DeclareMathOperator{\Frob}{Frob}
\DeclareMathOperator{\Funct}{Funct}
\DeclareMathOperator{\Fun}{Fun}
\DeclareMathOperator{\GL}{GL}
\DeclareMathOperator{\Gal}{Gal}
\DeclareMathOperator{\Gr}{Gr}
\DeclareMathOperator{\Hol}{Hol}
\DeclareMathOperator{\Hom}{Hom}
\DeclareMathOperator{\Ho}{Ho}
\DeclareMathOperator{\Id}{Id}
\DeclareMathOperator{\Ind}{Ind}
\DeclareMathOperator{\Inn}{Inn}
\DeclareMathOperator{\Isom}{Isom}
\DeclareMathOperator{\Ker}{Ker}
\DeclareMathOperator{\Lan}{Lan}
\DeclareMathOperator{\Lie}{Lie}
\DeclareMathOperator{\Map}{Map}
\DeclareMathOperator{\Mat}{Mat}
\DeclareMathOperator{\Max}{Max}
\DeclareMathOperator{\Mor}{Mor}
\DeclareMathOperator{\Nat}{Nat}
\DeclareMathOperator{\Nrd}{Nrd}
\DeclareMathOperator{\Ob}{Ob}
\DeclareMathOperator{\Out}{Out}
\DeclareMathOperator{\PGL}{PGL}
\DeclareMathOperator{\PSL}{PSL}
\DeclareMathOperator{\PSU}{PSU}
\DeclareMathOperator{\Pic}{Pic}
\DeclareMathOperator{\RHom}{RHom}
\DeclareMathOperator{\Rad}{Rad}
\DeclareMathOperator{\Ran}{Ran}
\DeclareMathOperator{\Rep}{Rep}
\DeclareMathOperator{\Res}{Res}
\DeclareMathOperator{\SL}{SL}
\DeclareMathOperator{\SO}{SO}
\DeclareMathOperator{\SU}{SU}
\DeclareMathOperator{\Sh}{Sh}
\DeclareMathOperator{\Sing}{Sing}
\DeclareMathOperator{\Specm}{Specm}
\DeclareMathOperator{\Spec}{Spec}
\DeclareMathOperator{\Sp}{Sp}
\DeclareMathOperator{\Stab}{Stab}
\DeclareMathOperator{\Sym}{Sym}
\DeclareMathOperator{\Tors}{Tors}
\DeclareMathOperator{\Tor}{Tor}
\DeclareMathOperator{\Tot}{Tot}
\DeclareMathOperator{\UUU}{U}

\DeclareMathOperator{\adj}{adj}
\DeclareMathOperator{\ad}{ad}
\DeclareMathOperator{\af}{af}
\DeclareMathOperator{\card}{card}
\DeclareMathOperator{\cm}{cm}
\DeclareMathOperator{\codim}{codim}
\DeclareMathOperator{\cod}{cod}
\DeclareMathOperator{\coeq}{coeq}
\DeclareMathOperator{\coim}{coim}
\DeclareMathOperator{\coker}{coker}
\DeclareMathOperator{\cont}{cont}
\DeclareMathOperator{\conv}{conv}
\DeclareMathOperator{\cor}{cor}
\DeclareMathOperator{\depth}{depth}
\DeclareMathOperator{\diag}{diag}
\DeclareMathOperator{\diam}{diam}
\DeclareMathOperator{\dist}{dist}
\DeclareMathOperator{\dom}{dom}
\DeclareMathOperator{\eq}{eq}
\DeclareMathOperator{\ev}{ev}
\DeclareMathOperator{\ex}{ex}
\DeclareMathOperator{\fchar}{char}
\DeclareMathOperator{\fr}{fr}
\DeclareMathOperator{\gr}{gr}
\DeclareMathOperator{\im}{im}
\DeclareMathOperator{\infl}{inf}
\DeclareMathOperator{\interior}{int}
\DeclareMathOperator{\intrel}{intrel}
\DeclareMathOperator{\inv}{inv}
\DeclareMathOperator{\length}{length}
\DeclareMathOperator{\mcd}{mcd}
\DeclareMathOperator{\mcm}{mcm}
\DeclareMathOperator{\multideg}{multideg}
\DeclareMathOperator{\ord}{ord}
\DeclareMathOperator{\pr}{pr}
\DeclareMathOperator{\rel}{rel}
\DeclareMathOperator{\res}{res}
\DeclareMathOperator{\rkred}{rkred}
\DeclareMathOperator{\rkss}{rkss}
\DeclareMathOperator{\rk}{rk}
\DeclareMathOperator{\sgn}{sgn}
\DeclareMathOperator{\sk}{sk}
\DeclareMathOperator{\supp}{supp}
\DeclareMathOperator{\trdeg}{trdeg}
\DeclareMathOperator{\tr}{tr}
\DeclareMathOperator{\vol}{vol}

\newcommand{\iHom}{\underline{\Hom}}

\renewcommand{\AA}{\mathbb{A}}
\newcommand{\CC}{\mathbb{C}}
\renewcommand{\SS}{\mathbb{S}}
\newcommand{\TT}{\mathbb{T}}
\newcommand{\PP}{\mathbb{P}}
\newcommand{\BB}{\mathbb{B}}
\newcommand{\RR}{\mathbb{R}}
\newcommand{\ZZ}{\mathbb{Z}}
\newcommand{\FF}{\mathbb{F}}
\newcommand{\HH}{\mathbb{H}}
\newcommand{\NN}{\mathbb{N}}
\newcommand{\QQ}{\mathbb{Q}}
\newcommand{\KK}{\mathbb{K}}

% % % % % % % % % % % % % % % % % % % % % % % % % % % % % %

\usepackage{amsthm}

\newcommand{\legendre}[2]{\left(\frac{#1}{#2}\right)}

\newcommand{\examplesymbol}{$\blacktriangle$}
\renewcommand{\qedsymbol}{$\blacksquare$}

\newcommand{\dfn}{\mathrel{\mathop:}=}
\newcommand{\rdfn}{=\mathrel{\mathop:}}

\usepackage{xcolor}
\definecolor{mylinkcolor}{rgb}{0.0,0.4,1.0}
\definecolor{mycitecolor}{rgb}{0.0,0.4,1.0}
\definecolor{shadecolor}{rgb}{0.79,0.78,0.65}
\definecolor{gray}{rgb}{0.6,0.6,0.6}

\usepackage{colortbl}

\definecolor{myred}{rgb}{0.7,0.0,0.0}
\definecolor{mygreen}{rgb}{0.0,0.7,0.0}
\definecolor{myblue}{rgb}{0.0,0.0,0.7}

\definecolor{redshade}{rgb}{0.9,0.5,0.5}
\definecolor{greenshade}{rgb}{0.5,0.9,0.5}

\usepackage[unicode,colorlinks=true,linkcolor=mylinkcolor,citecolor=mycitecolor]{hyperref}
\newcommand{\refref}[2]{\hyperref[#2]{#1~\ref*{#2}}}
\newcommand{\eqnref}[1]{\hyperref[#1]{(\ref*{#1})}}

\newcommand{\tos}{\!\!\to\!\!}

\usepackage{framed}

\newcommand{\cequiv}{\simeq}

\makeatletter
\newcommand\xleftrightarrow[2][]{%
  \ext@arrow 9999{\longleftrightarrowfill@}{#1}{#2}}
\newcommand\longleftrightarrowfill@{%
  \arrowfill@\leftarrow\relbar\rightarrow}
\makeatother

\newcommand{\bsquare}{\textrm{\ding{114}}}

% % % % % % % % % % % % % % % % % % % % % % % % % % % % % %

\newtheoremstyle{myplain}
  {\topsep}   % ABOVESPACE
  {\topsep}   % BELOWSPACE
  {\itshape}  % BODYFONT
  {0pt}       % INDENT (empty value is the same as 0pt)
  {\bfseries} % HEADFONT
  {.}         % HEADPUNCT
  {5pt plus 1pt minus 1pt} % HEADSPACE
  {\thmnumber{#2}. \thmname{#1}\thmnote{ (#3)}}   % CUSTOM-HEAD-SPEC

\newtheoremstyle{myplainnameless}
  {\topsep}   % ABOVESPACE
  {\topsep}   % BELOWSPACE
  {\normalfont}  % BODYFONT
  {0pt}       % INDENT (empty value is the same as 0pt)
  {\bfseries} % HEADFONT
  {.}         % HEADPUNCT
  {5pt plus 1pt minus 1pt} % HEADSPACE
  {\thmnumber{#2}}   % CUSTOM-HEAD-SPEC 

\newtheoremstyle{sectionexercise}
  {\topsep}   % ABOVESPACE
  {\topsep}   % BELOWSPACE
  {\normalfont}  % BODYFONT
  {0pt}       % INDENT (empty value is the same as 0pt)
  {\bfseries} % HEADFONT
  {.}         % HEADPUNCT
  {5pt plus 1pt minus 1pt} % HEADSPACE
  {Ejercicio \thmnumber{#2}\thmnote{ (#3)}}   % CUSTOM-HEAD-SPEC

\newtheoremstyle{mydefinition}
  {\topsep}   % ABOVESPACE
  {\topsep}   % BELOWSPACE
  {\normalfont}  % BODYFONT
  {0pt}       % INDENT (empty value is the same as 0pt)
  {\bfseries} % HEADFONT
  {.}         % HEADPUNCT
  {5pt plus 1pt minus 1pt} % HEADSPACE
  {\thmnumber{#2}. \thmname{#1}\thmnote{ (#3)}}   % CUSTOM-HEAD-SPEC

% EN ESPAÑOL

\newtheorem*{hecho*}{Hecho}
\newtheorem*{corolario*}{Corolario}
\newtheorem*{teorema*}{Teorema}
\newtheorem*{conjetura*}{Conjetura}
\newtheorem*{proyecto*}{Proyecto}
\newtheorem*{observacion*}{Observación}

\newtheorem*{lema*}{Lema}
\newtheorem*{resultado-clave*}{Resultado clave}
\newtheorem*{proposicion*}{Proposición}

\theoremstyle{definition}
\newtheorem*{ejercicio*}{Ejercicio}
\newtheorem*{definicion*}{Definición}
\newtheorem*{comentario*}{Comentario}
\newtheorem*{definicion-alternativa*}{Definición alternativa}
\newtheorem*{ejemploxs}{Ejemplo}
\newenvironment{ejemplo*}
  {\pushQED{\qed}\renewcommand{\qedsymbol}{\examplesymbol}\ejemploxs}
  {\popQED\endejemploxs}

\theoremstyle{myplain}
\newtheorem{proposicion}{Proposición}[section]

\newtheorem{proyecto}[proposicion]{Proyecto}
\newtheorem{teorema}[proposicion]{Teorema}
\newtheorem{corolario}[proposicion]{Corolario}
\newtheorem{hecho}[proposicion]{Hecho}
\newtheorem{lema}[proposicion]{Lema}

\newtheorem{observacion}[proposicion]{Observación}

\newenvironment{observacionejerc}
    {\pushQED{\qed}\renewcommand{\qedsymbol}{$\square$}\csname inner@observacionejerc\endcsname}
    {\popQED\csname endinner@observacionejerc\endcsname}
\newtheorem{inner@observacionejerc}[proposicion]{Observación}

\newenvironment{proposicionejerc}
    {\pushQED{\qed}\renewcommand{\qedsymbol}{$\square$}\csname inner@proposicionejerc\endcsname}
    {\popQED\csname endinner@proposicionejerc\endcsname}
\newtheorem{inner@proposicionejerc}[proposicion]{Proposicion}

\newenvironment{lemaejerc}
    {\pushQED{\qed}\renewcommand{\qedsymbol}{$\square$}\csname inner@lemaejerc\endcsname}
    {\popQED\csname endinner@lemaejerc\endcsname}
\newtheorem{inner@lemaejerc}[proposicion]{Lema}

\newtheorem{calculo}[proposicion]{Cálculo}

\theoremstyle{myplainnameless}
\newtheorem{nameless}[proposicion]{}

\theoremstyle{mydefinition}
\newtheorem{comentario}[proposicion]{Comentario}
\newtheorem{comentarioast}[proposicion]{Comentario ($\clubsuit$)}
\newtheorem{construccion}[proposicion]{Construcción}
\newtheorem{aplicacion}[proposicion]{Aplicación}
\newtheorem{definicion}[proposicion]{Definición}
\newtheorem{definicion-alternativa}[proposicion]{Definición alternativa}
\newtheorem{notacion}[proposicion]{Notación}
\newtheorem{advertencia}[proposicion]{Advertencia}
\newtheorem{digresion}[proposicion]{Digresión}
\newtheorem{ejemplox}[proposicion]{Ejemplo}
\newenvironment{ejemplo}
  {\pushQED{\qed}\renewcommand{\qedsymbol}{\examplesymbol}\ejemplox}
  {\popQED\endejemplox}
\newtheorem{contraejemplox}[proposicion]{Contraejemplo}
\newenvironment{contraejemplo}
  {\pushQED{\qed}\renewcommand{\qedsymbol}{\examplesymbol}\contraejemplox}
  {\popQED\endcontraejemplox}
\newtheorem{noejemplox}[proposicion]{No-ejemplo}
\newenvironment{noejemplo}
  {\pushQED{\qed}\renewcommand{\qedsymbol}{\examplesymbol}\noejemplox}
  {\popQED\endnoejemplox}
 
\newtheorem{ejemploastx}[proposicion]{Ejemplo ($\clubsuit$)}
\newenvironment{ejemploast}
  {\pushQED{\qed}\renewcommand{\qedsymbol}{\examplesymbol}\ejemploastx}
  {\popQED\endejemploastx}

\ifdefined\exercisespersection
  \theoremstyle{sectionexercise}
  \newtheorem{ejercicio}{}[section]
  \theoremstyle{mydefinition}
\else
  \ifdefined\exercisesglobal
    \theoremstyle{sectionexercise}
    \newtheorem{ejercicio}{}
    \theoremstyle{mydefinition}
  \else
    \ifdefined\exercisespersection
      \newtheorem{ejercicio}[proposicion]{Ejercicio}
    \fi
  \fi
\fi

% % % % % % % % % % % % % % % % % % % % % % % % % % % % % %

\theoremstyle{myplain}
\newtheorem{proposition}{Proposition}[section]
\newtheorem*{fact*}{Fact}
\newtheorem*{proposition*}{Proposition}
\newtheorem{lemma}[proposition]{Lemma}
\newtheorem*{lemma*}{Lemma}

\newtheorem{exercise}{Exercise}
\newtheorem*{hint}{Hint}

\newtheorem{theorem}[proposition]{Theorem}
\newtheorem{conjecture}[proposition]{Conjecture}
\newtheorem*{theorem*}{Theorem}
\newtheorem{corollary}[proposition]{Corollary}
\newtheorem{fact}[proposition]{Fact}
\newtheorem*{claim}{Claim}
\newtheorem{definition-theorem}[proposition]{Definition-theorem}

\theoremstyle{mydefinition}
\newtheorem{examplex}[proposition]{Example}
\newenvironment{example}
  {\pushQED{\qed}\renewcommand{\qedsymbol}{\examplesymbol}\examplex}
  {\popQED\endexamplex}

\newtheorem*{examplexx}{Example}
\newenvironment{example*}
  {\pushQED{\qed}\renewcommand{\qedsymbol}{\examplesymbol}\examplexx}
  {\popQED\endexamplexx}

\newtheorem{definition}[proposition]{Definition}
\newtheorem*{definition*}{Definition}
\newtheorem{wrong-definition}[proposition]{Wrong definition}
\newtheorem{remark}[proposition]{Remark}

\makeatletter
\newcommand{\xRightarrow}[2][]{\ext@arrow 0359\Rightarrowfill@{#1}{#2}}
\makeatother

% % % % % % % % % % % % % % % % % % % % % % % % % % % % % %

\newcommand{\Et}{\mathop{\text{\rm Ét}}}

\newcommand{\categ}[1]{\text{\bf #1}}
\newcommand{\vcateg}{\mathcal}
\newcommand{\bone}{{\boldsymbol 1}}
\newcommand{\bDelta}{{\boldsymbol\Delta}}
\newcommand{\bR}{{\mathbf{R}}}

\newcommand{\univ}{\mathfrak}

\newcommand{\TODO}{\colorbox{red}{\textbf{*** TODO ***}}}
\newcommand{\proofreadme}{\colorbox{red}{\textbf{*** NEEDS PROOFREADING ***}}}

\makeatletter
\def\iddots{\mathinner{\mkern1mu\raise\p@
\vbox{\kern7\p@\hbox{.}}\mkern2mu
\raise4\p@\hbox{.}\mkern2mu\raise7\p@\hbox{.}\mkern1mu}}
\makeatother

\newcommand{\ssincl}{\reflectbox{\rotatebox[origin=c]{45}{$\subseteq$}}}
\newcommand{\vsupseteq}{\reflectbox{\rotatebox[origin=c]{-90}{$\supseteq$}}}
\newcommand{\vin}{\reflectbox{\rotatebox[origin=c]{90}{$\in$}}}

\newcommand{\Ga}{\mathbb{G}_\mathrm{a}}
\newcommand{\Gm}{\mathbb{G}_\mathrm{m}}

\renewcommand{\U}{\UUU}

\DeclareRobustCommand{\Stirling}{\genfrac\{\}{0pt}{}}
\DeclareRobustCommand{\stirling}{\genfrac[]{0pt}{}}

% % % % % % % % % % % % % % % % % % % % % % % % % % % % % %
% tikz

\usepackage{tikz-cd}
\usetikzlibrary{babel}
\usetikzlibrary{decorations.pathmorphing}
\usetikzlibrary{arrows}
\usetikzlibrary{calc}
\usetikzlibrary{fit}
\usetikzlibrary{hobby}

% % % % % % % % % % % % % % % % % % % % % % % % % % % % % %
% Banners

\newcommand\mybannerext[3]{{\normalfont\sffamily\bfseries\large\noindent #1

\noindent #2

\noindent #3

}\noindent\rule{\textwidth}{1.25pt}

\vspace{1em}}

\newcommand\mybanner[2]{{\normalfont\sffamily\bfseries\large\noindent #1

\noindent #2

}\noindent\rule{\textwidth}{1.25pt}

\vspace{1em}}

\renewcommand{\O}{\mathcal{O}}

\usepackage{diagbox}

\usepackage[numbers]{natbib}

\usepackage{fullpage}

\author{Alexey Beshenov (cadadr@gmail.com)}
\title{Sumas de potencias de números naturales\\y los números de Bernoulli}
\date{24 de Febrero de 2017}

\usepackage{xcolor}
\newcommand{\highlight}[1]{\colorbox{shadecolor}{$\displaystyle #1$}}

\def\mystrut(#1,#2){\vrule height #1 depth #2 width 0pt}

\newcolumntype{f}[1]{%
>{\mystrut(3ex,2ex)\centering}%
p{#1}%
<{}}

\begin{document}

{\normalfont\sffamily\bfseries \maketitle}

La suma de $n$ números naturales consecutivos puede ser calculada mediante la fórmula
$$1+2+\cdots+n = \frac{(n+1)\,n}{2} = \frac{1}{2}\,n^2 + \frac{1}{2}\,n.$$
Probablemente el lector también conoce la fórmula para las sumas de cuadrados:
$$1^2 + 2^2 + \cdots + n^2 = \frac{n\,(n+1)\,(2n + 1)}{6} = \frac{1}{3}\,n^3 + \frac{1}{2}\,n^2 + \frac{1}{6}\,n.$$
---es fácil demostrarla por inducción. Muchos matemáticos trataron de encontrar la fórmula similar para las sumas de cubos y otras potencias superiores. Es un problema muy natural, y la solución fue descubierta al principio del siglo XVIII por el matemático suizo \personality{Jacob Bernoulli} (1654--1705) y independientemente por el matemático japonés \personality{Seki Takakazu} (1642--1708). Denotemos por $S_k (n)$ la suma de las $k$-ésimas potencias de los números naturales hasta $n$:
$$S_k (n) \dfn \sum_{1 \le i \le n} i^k = 1^k + 2^k + \cdots + n^k.$$
En particular,
$$S_0 (n) = n, \quad S_1 (n) = \frac{1}{2}\,n^2 + \frac{1}{2}\,n, \quad S_2 (n) = \frac{1}{3}\,n^3 + \frac{1}{2}\,n^2 + \frac{1}{6}\,n.$$
Para obtener las fórmulas para $S_3 (n), S_4 (n), S_5 (n)$, etcétera, recordemos primero el \term{teorema del binomio}:
$$(x+y)^k = \sum_{0 \le i \le k} {k \choose i} x^{k-i}\,y^i,$$
donde
$${k \choose i} = \frac{k!}{(k-i)!\,i!}$$
denota un \term{coeficiente binomial}, definido como el número de posibilidades de escoger $i$ objetos entre un total de $k$ objetos.

\begin{shaded}
\small\noindent En PARI/GP, \verb|binomial(k,i)| $= {k\choose i}$.

\begin{verbatim}
/* vector (n,i,expr) devuelve un vector con la expresión
   expr evaluada con i=1, i=2, ..., i=n: */
? vector (7,i,binomial (6,i-1))
% = [1, 6, 15, 20, 15, 6, 1]
\end{verbatim}
\end{shaded}

En particular, tenemos
$$(m+1)^{k+1} - m^{k+1} = \sum_{0 \le i \le k} {k+1 \choose i}\,m^i.$$
La suma de estas identidades para $m = 1, 2, \ldots, n$ nos da
$$(n+1)^{k+1} - 1 = \sum_{0 \le i \le k} {k+1 \choose i}\,S_i (n),$$
de donde tenemos una expresión de $S_k (n)$ en términos de $S_0 (n), S_1 (n), \ldots, S_{k-1} (n)$:

\begin{equation}
\label{formula-para-Skn}
S_k (n) = \frac{1}{k+1} \, \left((n+1)^{k+1} - 1 - \sum_{0 \le i \le k-1} {k+1\choose i} \, S_i (n)\right).
\end{equation}

Por inducción se ve que $S_k (n)$ es un polinomio en $n$ de grado $k+1$, con coeficiente principal $\frac{1}{k+1}$. Para evitar una posible confusión, denotemos la variable por $x$. El polinomio $S_k (x) \in \QQ [x]$ está determinado por sus valores en $x = n \in \NN$. Por la definición de $S_k (n)$, tenemos $S_k (n+1) - S_k (n) = (n+1)^k$ para $n = 1,2,3,\ldots$ Para los polinomios, esto nos da la relación

\begin{equation}
\label{eqn:Skn+1-y-Skn}
S_k (x+1) - S_k (x) = (x+1)^k.
\end{equation}

En particular, $S_k (1) - S_k (0) = 1$, y ya que $S_k (1) = 1$, esto significa que $S_k (0) = 0$; es decir, el término constante del polinomio $S_k (x)$ es nulo (también podemos verlo por inducción de la fórmula \eqnref{formula-para-Skn}). Usando \eqnref{formula-para-Skn}, podemos calcular algunos $S_k (x)$.

\begin{shaded}
\small

\noindent Implementemos nuestra fórmula para $S_k$ en PARI/GP:

\begin{verbatim}
S(k) = if (k == 0, x, 1/(k+1)*((x+1)^(k+1) - 1 - sum (i=0, k-1, binomial(k+1,i) * S(i))));

? S(3)
% = 1/4*x^4 + 1/2*x^3 + 1/4*x^2
\end{verbatim}

\noindent El lector que conoce un poco de programación puede notar que el código de arriba es muy ineficaz; por ejemplo, para calcular \verb|S(20)| ya se necesita mucho tiempo. He aquí otra versión mucho más rápida:

\begin{verbatim}
/* La tabla de S (k): */
s_table = [];

S (k) = {
  if (k == 0, return (x));

  /* Extender la tabla de valores, de ser necesario: */
  if (length(s_table) < k, s_table = concat(s_table, vector(k-length(s_table))));

  /* Devolver el valor, si está en la tabla;
     sino, calcularlo y poner en la tabla: */	
  if (s_table[k], s_table[k],
      s_table[k] = 1/(k+1)*((x+1)^(k+1) - 1 - sum (i=0, k-1, binomial(k+1,i) * S(i))))
}
\end{verbatim}

\noindent (Trate de calcular, por ejemplo, \verb|S(20)| usando ambas versiones.)
\end{shaded}

\begin{align*}
S_0 (x) & = x,\\
S_1 (x) & = \frac{1}{2}\,x^2\,+\highlight{\frac{1}{2}}\,x,\\
S_2 (x) & = \frac{1}{3}\,x^3+\frac{1}{2}\,x^2+\highlight{\frac{1}{6}}\,x,\\
S_3 (x) & = \frac{1}{4}\,x^4+\frac{1}{2}\,x^3+\frac{1}{4}\,x^2,\\
S_4 (x) & = \frac{1}{5}\,x^5+\frac{1}{2}\,x^4+\frac{1}{3}\,x^3\highlight{-\frac{1}{30}}\,x,\\
S_5 (x) & = \frac{1}{6}\,x^6+\frac{1}{2}\,x^5+\frac{5}{12}\,x^4-\frac{1}{12}\,x^2,\\
S_6 (x) & = \frac{1}{7}\,x^7+\frac{1}{2}\,x^6+\frac{1}{2}\,x^5-\frac{1}{6}\,x^3+\highlight{\frac{1}{42}}\,x,\\
S_7 (x) & = \frac{1}{8}\,x^8+\frac{1}{2}\,x^7+\frac{7}{12}\,x^6-\frac{7}{24}\,x^4+\frac{1}{12}\,x^2,\\
S_8 (x) & = \frac{1}{9}\,x^9+\frac{1}{2}\,x^8+\frac{2}{3}\,x^7-\frac{7}{15}\,x^5+\frac{2}{9}\,x^3\highlight{-\frac{1}{30}}\,x,\\
S_9 (x) & = \frac{1}{10}\,x^{10}+\frac{1}{2}\,x^9+\frac{3}{4}\,x^8-\frac{7}{10}\,x^6+\frac{1}{2}\,x^4-\frac{3}{20}\,x^2,\\
S_{10} (x) & = \frac{1}{11}\,x^{11}+\frac{1}{2}\,x^{10}+\frac{5}{6}\,x^9-x^7+x^5-\frac{1}{2}\,x^3+\highlight{\frac{5}{66}}\,x.
\end{align*}

Las expresiones de arriba, también hasta $S_{10} (n)$, aparecen en la página 97 del libro de Bernoulli \href{http://books.google.com/books?id=kD4PAAAAQAAJ}{``Ars conjectandi''}, publicado póstumamente en 1713. Luego Bernoulli escribe que, usando sus fórmulas, calculó en un ``semi-cuarto de hora'' la suma
$$1^{10} + 2^{10} + \cdots + 1000^{10} = S_{10} (1000) = 91\,409\,924\,241\,424\,243\,424\,241\,924\,242\,500.$$
Con ayuda de una computadora, se puede verificar que ¡el resultado es correcto!

\begin{shaded}
\small
\begin{verbatim}
? { local(x); x = 1000; eval (S(10)) }
% = 91409924241424243424241924242500

? sum (i=1,1000,i^10)
% = 91409924241424243424241924242500
\end{verbatim}
\end{shaded}

\begin{definicion*}
El \term{$k$-ésimo número de Bernoulli $B_k$} es el coeficiente de $x$ en el polinomio $S_k (x)$. En otras palabras,
$$B_k \dfn S_k' (0).$$
\end{definicion*}

Euler leyó ``Ars Conjectandi'' y estudió los números $B_k$, llamándolos los ``números de Bernoulli'', en el capítulo II.5 de su libro \href{http://eulerarchive.maa.org/pages/E212.html}{``Institutiones calculi differentialis cum eius usu in analysi finitorum ac doctrina serierum''}. Varias identidades para $B_k$ que aparecen en nuestro curso fueron descubiertas por Euler. Por ejemplo, la derivada de \eqnref{formula-para-Skn} nos da
$$S_k' (x) = \frac{1}{k+1} \, \left((k+1)\,(x+1)^k - \sum_{0 \le i \le k-1} {k+1\choose i} \, S_i' (x)\right),$$
y para $x = 0$ tenemos
$$B_k = S_k' (0) = 1 - \frac{1}{k+1} \, \sum_{0 \le i \le k-1} {k+1\choose i} \, B_i.$$

\begin{proposicion*}
Para todo $k \ge 0$ se tiene
$$\sum_{0 \le i \le k} {k+1 \choose i}\,B_i = k+1.$$
\end{proposicion*}

Esto nos da una definición recursiva de los $B_k$:

\begin{align*}
B_0 & = 1,\\
B_0 + 2\,B_1 & = 2,\\
B_0 + 3\,B_1 + 3\,B_2 & = 3,\\
B_0 + 4\,B_1 + 6\,B_2 + 4\,B_3 & = 4,\\
 & ~~ \vdots
\end{align*}

A partir de estas identidades se pueden calcular sucesivamente $B_1, B_2, B_3, B_4, \ldots$

\begin{shaded}
\small
\begin{verbatim}
/* La tabla de B (k): */
b_table = [];

B (k) = {
  if (k == 0, return (1));

  if (length(b_table) < k, b_table = concat(b_table, vector(k-length(b_table))));
  if (b_table[k], b_table[k],
      b_table[k] = 1 - 1/(k+1)*sum (i=0, k-1, binomial(k+1,i)*B (i)))
}

? polcoeff (S(10),1,n)
% = 5/66
? B(10)
% = 5/66
\end{verbatim}
\end{shaded}

Luego los primeros números de Bernoulli son

{\def\arraystretch{1.25}
\[ \begin{array}{rx{0.7cm}x{0.7cm}x{0.7cm}x{0.7cm}x{0.7cm}x{0.7cm}x{0.7cm}x{0.7cm}x{0.7cm}x{0.7cm}x{0.7cm}x{0.7cm}}
k\colon & $0$ & $1$ & $2$ & $3$ & $4$ & $5$ & $6$ & $7$ & $8$ & $9$ & $10$ & $\cdots$ \tabularnewline
\hline
B_k\colon & $1$ & $\frac{1}{2}$ & $\frac{1}{6}$ & $0$ & $-\frac{1}{30}$ & $0$ & $\frac{1}{42}$ & $0$ & $-\frac{1}{30}$ & $0$ & $\frac{5}{66}$ & $\cdots$
\end{array} \]
}

\noindent (Bernoulli y Euler no usaban la notación $B_k$, sino que escribían $A = \frac{1}{6}$, $B = -\frac{1}{30}$, $C = \frac{1}{42}$, $D = -\frac{1}{30}$, etcétera.)

La derivada de \eqnref{eqn:Skn+1-y-Skn} es
$$S_k' (x+1) - S_k' (x) = k\,(x+1)^{k-1},$$
y la suma de estas identidades para $x = 0,1,2,\ldots,n-1$ nos da
$$S_k' (n) - S_k' (0) = k\,S_{k-1} (n).$$
Entonces, los polinomios $S_k (x)$ satisfacen la identidad
$$S_k' (x) = k\,S_{k-1} (x) + B_k.$$
Esto, junto con $S_1 (x) = x$, define completamente a todos los $S_k (x)$ (el término constante es nulo).

\begin{ejercicio*}
\label{ejercicio:suma-de-potencias-y-Bk}
Demuestre la identidad
$$S_k (n) = \frac{1}{k+1}\,\sum_{0 \le i \le k} {k+1\choose i}\,B_i\,n^{k+1-i}.$$
\noindent Indicación: si $p (x) \in \QQ [x]$ es un polinomio de grado $d$, entonces
$$p (x) = \sum_{0 \le i \le d} \frac{p^{(i)} (0)}{i!}\,x^i,$$
donde $p^{(i)} (x)$ es la $i$-ésima derivada iterada de $p (x)$:
$$p^{(0)} (x) \dfn p (x), ~ p^{(1)} (x) \dfn p' (x), ~ p^{(2)} (x) \dfn p'' (x), ~ p^{(3)} (x) \dfn p''' (x), ~ \ldots$$
\end{ejercicio*}

Revisemos nuestra tabla de los primeros números de Bernoulli. Se observan dos patrones:

\begin{itemize}
\item $B_k = 0$ para $k \ge 3$ impar. Esto va a ser evidente más adelante a partir de otra representación de $B_k$ mediante una función generatriz par.

\item $B_k \ne 0$ para $k$ par, y los signos se alternan.
Es posible demostrar esto directamente de modo combinatorio o a partir de la fórmula con la \term{función zeta de Riemann}
$$(-1)^{k+1} \, B_{2k}\,\frac{2^{2k-1}}{(2k)!}\,\pi^{2k} = \zeta (2k) \dfn \sum_{n \ge 1} \frac{1}{n^{2k}} > 0.$$
\end{itemize}

En cuanto a los valores de $B_k$, no se nota ningún patrón evidente. Por ejemplo, en $B_{12} = -\frac{691}{2730}$ el numerador $691$ es un número primo que aparentemente no tiene nada que ver con $12$. Sin embargo, si factorizamos los \emph{denominadores} de $B_k$, se revelan números primos relacionados con $k$ de alguna manera:

{\def\arraystretch{1.25}
\[ \begin{array}{rccccccccccccccc}
k\colon & 0 & 1 & 2 & 4 & 6 & 8 & 10 & 12 & 14 & 16 & 18 & 20 & \cdots \\
\hline
B_k\colon & 1 & \frac{1}{2} & \frac{1}{2\cdot 3} & -\frac{1}{2\cdot 3\cdot 5} & \frac{1}{2\cdot 3\cdot 7} & -\frac{1}{2\cdot 3\cdot 5} & \frac{5}{2\cdot 3\cdot 11} & -\frac{691}{2\cdot 3\cdot 5\cdot 7\cdot 13} & \frac{7}{2\cdot 3} & -\frac{3617}{2\cdot 3\cdot 5\cdot 17} & \frac{43867}{2\cdot 3\cdot 7\cdot 19} & -\frac{174611}{2\cdot 3\cdot 5\cdot 11} & \cdots
\end{array} \]
}

El lector puede formular su propia conjetura sobre los denominadores; más adelante vamos a ver la respuesta en el \term{teorema de Clausen--von Staudt}.

\begin{shaded}
\small\noindent Por supuesto, PARI/GP ya sabe calcular los números de Bernoulli. La función \verb|bernfrac(k)| devuelve $B_k$:

\begin{verbatim}
? bernfrac(1)
% = -1/2
? bernfrac(10)
% = 5/66
\end{verbatim}

\noindent Atención: PARI/GP (y también muchos libros de texto) usa otra definición de $B_k$ según la cual $B_1 = -\frac{1}{2}$ (note el signo ``$-$''). Para $k > 1$, los $B_k$ de PARI/GP son los mismos que los nuestros. La función \verb|bernreal(k)| calcula el valor aproximado de $B_k$:

\begin{verbatim}
? bernreal(4)   
% = -0.033333333333333333333333333333333333333
\end{verbatim}
\end{shaded}

\end{document}
