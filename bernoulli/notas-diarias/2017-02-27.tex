\documentclass{article}

% TODO : CLEAN UP THIS MESS
% (AND MAKE SURE ALL TEXTS STILL COMPILE)
\usepackage[leqno]{amsmath}
\usepackage{amssymb}
\usepackage{graphicx}

\usepackage{diagbox} % table heads with diagonal lines
\usepackage{relsize}

\usepackage{wasysym}
\usepackage{scrextend}
\usepackage{epigraph}
\setlength\epigraphwidth{.6\textwidth}

\usepackage[utf8]{inputenc}

\usepackage{titlesec}
\titleformat{\chapter}[display]
  {\normalfont\sffamily\huge\bfseries}
  {\chaptertitlename\ \thechapter}{5pt}{\Huge}
\titleformat{\section}
  {\normalfont\sffamily\Large\bfseries}
  {\thesection}{1em}{}
\titleformat{\subsection}
  {\normalfont\sffamily\large\bfseries}
  {\thesubsection}{1em}{}
\titleformat{\part}[display]
  {\normalfont\sffamily\huge\bfseries}
  {\partname\ \thepart}{0pt}{\Huge}

\usepackage[T1]{fontenc}
\usepackage{fourier}
\usepackage{paratype}

\usepackage[symbol,perpage]{footmisc}

\usepackage{perpage}
\MakePerPage{footnote}

\usepackage{array}
\newcolumntype{x}[1]{>{\centering\hspace{0pt}}p{#1}}

% TODO: the following line causes conflict with new texlive (!)
% \usepackage[english,russian,polutonikogreek,spanish]{babel}
% \newcommand{\russian}[1]{{\selectlanguage{russian}#1}}

% Remove conflicting options for the moment:
\usepackage[english,polutonikogreek,spanish]{babel}

\AtBeginDocument{\shorthandoff{"}}
\newcommand{\greek}[1]{{\selectlanguage{polutonikogreek}#1}}

% % % % % % % % % % % % % % % % % % % % % % % % % % % % % %
% Limit/colimit symbols (with accented i: lím / colím)

\usepackage{etoolbox} % \patchcmd

\makeatletter
\patchcmd{\varlim@}{lim}{\lim}{}{}
\makeatother
\DeclareMathOperator*{\colim}{co{\lim}}
\newcommand{\dirlim}{\varinjlim}
\newcommand{\invlim}{\varprojlim}

% % % % % % % % % % % % % % % % % % % % % % % % % % % % % %

\usepackage[all,color]{xy}

\usepackage{pigpen}
\newcommand{\po}{\ar@{}[dr]|(.4){\text{\pigpenfont I}}}
\newcommand{\pb}{\ar@{}[dr]|(.3){\text{\pigpenfont A}}}
\newcommand{\polr}{\ar@{}[dr]|(.65){\text{\pigpenfont A}}}
\newcommand{\pour}{\ar@{}[ur]|(.65){\text{\pigpenfont G}}}
\newcommand{\hstar}{\mathop{\bigstar}}

\newcommand{\bigast}{\mathop{\Huge \mathlarger{\mathlarger{\ast}}}}

\newcommand{\term}{\textbf}

\usepackage{stmaryrd}

\usepackage{cancel}

\usepackage{tikzsymbols}

\newcommand{\open}{\underset{\mathrm{open}}{\hookrightarrow}}
\newcommand{\closed}{\underset{\mathrm{closed}}{\hookrightarrow}}

\newcommand{\tcol}[2]{{#1 \choose #2}}

\newcommand{\homot}{\simeq}
\newcommand{\isom}{\cong}
\newcommand{\cH}{\mathcal{H}}
\renewcommand{\hom}{\mathrm{hom}}
\renewcommand{\div}{\mathop{\mathrm{div}}}
\renewcommand{\Im}{\mathop{\mathrm{Im}}}
\renewcommand{\Re}{\mathop{\mathrm{Re}}}
\newcommand{\id}[1]{\mathrm{id}_{#1}}
\newcommand{\idid}{\mathrm{id}}

\newcommand{\ZG}{{\ZZ G}}
\newcommand{\ZH}{{\ZZ H}}

\newcommand{\quiso}{\simeq}

\newcommand{\personality}[1]{{\sc #1}}

\newcommand{\mono}{\rightarrowtail}
\newcommand{\epi}{\twoheadrightarrow}
\newcommand{\xepi}[1]{\xrightarrow{#1}\mathrel{\mkern-14mu}\rightarrow}

% % % % % % % % % % % % % % % % % % % % % % % % % % % % % %

\DeclareMathOperator{\Ad}{Ad}
\DeclareMathOperator{\Aff}{Aff}
\DeclareMathOperator{\Ann}{Ann}
\DeclareMathOperator{\Aut}{Aut}
\DeclareMathOperator{\Br}{Br}
\DeclareMathOperator{\CH}{CH}
\DeclareMathOperator{\Cl}{Cl}
\DeclareMathOperator{\Coeq}{Coeq}
\DeclareMathOperator{\Coind}{Coind}
\DeclareMathOperator{\Cop}{Cop}
\DeclareMathOperator{\Corr}{Corr}
\DeclareMathOperator{\Cor}{Cor}
\DeclareMathOperator{\Cov}{Cov}
\DeclareMathOperator{\Der}{Der}
\DeclareMathOperator{\Div}{Div}
\DeclareMathOperator{\D}{D}
\DeclareMathOperator{\Ehr}{Ehr}
\DeclareMathOperator{\End}{End}
\DeclareMathOperator{\Eq}{Eq}
\DeclareMathOperator{\Ext}{Ext}
\DeclareMathOperator{\Frac}{Frac}
\DeclareMathOperator{\Frob}{Frob}
\DeclareMathOperator{\Funct}{Funct}
\DeclareMathOperator{\Fun}{Fun}
\DeclareMathOperator{\GL}{GL}
\DeclareMathOperator{\Gal}{Gal}
\DeclareMathOperator{\Gr}{Gr}
\DeclareMathOperator{\Hol}{Hol}
\DeclareMathOperator{\Hom}{Hom}
\DeclareMathOperator{\Ho}{Ho}
\DeclareMathOperator{\Id}{Id}
\DeclareMathOperator{\Ind}{Ind}
\DeclareMathOperator{\Inn}{Inn}
\DeclareMathOperator{\Isom}{Isom}
\DeclareMathOperator{\Ker}{Ker}
\DeclareMathOperator{\Lan}{Lan}
\DeclareMathOperator{\Lie}{Lie}
\DeclareMathOperator{\Map}{Map}
\DeclareMathOperator{\Mat}{Mat}
\DeclareMathOperator{\Max}{Max}
\DeclareMathOperator{\Mor}{Mor}
\DeclareMathOperator{\Nat}{Nat}
\DeclareMathOperator{\Nrd}{Nrd}
\DeclareMathOperator{\Ob}{Ob}
\DeclareMathOperator{\Out}{Out}
\DeclareMathOperator{\PGL}{PGL}
\DeclareMathOperator{\PSL}{PSL}
\DeclareMathOperator{\PSU}{PSU}
\DeclareMathOperator{\Pic}{Pic}
\DeclareMathOperator{\RHom}{RHom}
\DeclareMathOperator{\Rad}{Rad}
\DeclareMathOperator{\Ran}{Ran}
\DeclareMathOperator{\Rep}{Rep}
\DeclareMathOperator{\Res}{Res}
\DeclareMathOperator{\SL}{SL}
\DeclareMathOperator{\SO}{SO}
\DeclareMathOperator{\SU}{SU}
\DeclareMathOperator{\Sh}{Sh}
\DeclareMathOperator{\Sing}{Sing}
\DeclareMathOperator{\Specm}{Specm}
\DeclareMathOperator{\Spec}{Spec}
\DeclareMathOperator{\Sp}{Sp}
\DeclareMathOperator{\Stab}{Stab}
\DeclareMathOperator{\Sym}{Sym}
\DeclareMathOperator{\Tors}{Tors}
\DeclareMathOperator{\Tor}{Tor}
\DeclareMathOperator{\Tot}{Tot}
\DeclareMathOperator{\UUU}{U}

\DeclareMathOperator{\adj}{adj}
\DeclareMathOperator{\ad}{ad}
\DeclareMathOperator{\af}{af}
\DeclareMathOperator{\card}{card}
\DeclareMathOperator{\cm}{cm}
\DeclareMathOperator{\codim}{codim}
\DeclareMathOperator{\cod}{cod}
\DeclareMathOperator{\coeq}{coeq}
\DeclareMathOperator{\coim}{coim}
\DeclareMathOperator{\coker}{coker}
\DeclareMathOperator{\cont}{cont}
\DeclareMathOperator{\conv}{conv}
\DeclareMathOperator{\cor}{cor}
\DeclareMathOperator{\depth}{depth}
\DeclareMathOperator{\diag}{diag}
\DeclareMathOperator{\diam}{diam}
\DeclareMathOperator{\dist}{dist}
\DeclareMathOperator{\dom}{dom}
\DeclareMathOperator{\eq}{eq}
\DeclareMathOperator{\ev}{ev}
\DeclareMathOperator{\ex}{ex}
\DeclareMathOperator{\fchar}{char}
\DeclareMathOperator{\fr}{fr}
\DeclareMathOperator{\gr}{gr}
\DeclareMathOperator{\im}{im}
\DeclareMathOperator{\infl}{inf}
\DeclareMathOperator{\interior}{int}
\DeclareMathOperator{\intrel}{intrel}
\DeclareMathOperator{\inv}{inv}
\DeclareMathOperator{\length}{length}
\DeclareMathOperator{\mcd}{mcd}
\DeclareMathOperator{\mcm}{mcm}
\DeclareMathOperator{\multideg}{multideg}
\DeclareMathOperator{\ord}{ord}
\DeclareMathOperator{\pr}{pr}
\DeclareMathOperator{\rel}{rel}
\DeclareMathOperator{\res}{res}
\DeclareMathOperator{\rkred}{rkred}
\DeclareMathOperator{\rkss}{rkss}
\DeclareMathOperator{\rk}{rk}
\DeclareMathOperator{\sgn}{sgn}
\DeclareMathOperator{\sk}{sk}
\DeclareMathOperator{\supp}{supp}
\DeclareMathOperator{\trdeg}{trdeg}
\DeclareMathOperator{\tr}{tr}
\DeclareMathOperator{\vol}{vol}

\newcommand{\iHom}{\underline{\Hom}}

\renewcommand{\AA}{\mathbb{A}}
\newcommand{\CC}{\mathbb{C}}
\renewcommand{\SS}{\mathbb{S}}
\newcommand{\TT}{\mathbb{T}}
\newcommand{\PP}{\mathbb{P}}
\newcommand{\BB}{\mathbb{B}}
\newcommand{\RR}{\mathbb{R}}
\newcommand{\ZZ}{\mathbb{Z}}
\newcommand{\FF}{\mathbb{F}}
\newcommand{\HH}{\mathbb{H}}
\newcommand{\NN}{\mathbb{N}}
\newcommand{\QQ}{\mathbb{Q}}
\newcommand{\KK}{\mathbb{K}}

% % % % % % % % % % % % % % % % % % % % % % % % % % % % % %

\usepackage{amsthm}

\newcommand{\legendre}[2]{\left(\frac{#1}{#2}\right)}

\newcommand{\examplesymbol}{$\blacktriangle$}
\renewcommand{\qedsymbol}{$\blacksquare$}

\newcommand{\dfn}{\mathrel{\mathop:}=}
\newcommand{\rdfn}{=\mathrel{\mathop:}}

\usepackage{xcolor}
\definecolor{mylinkcolor}{rgb}{0.0,0.4,1.0}
\definecolor{mycitecolor}{rgb}{0.0,0.4,1.0}
\definecolor{shadecolor}{rgb}{0.79,0.78,0.65}
\definecolor{gray}{rgb}{0.6,0.6,0.6}

\usepackage{colortbl}

\definecolor{myred}{rgb}{0.7,0.0,0.0}
\definecolor{mygreen}{rgb}{0.0,0.7,0.0}
\definecolor{myblue}{rgb}{0.0,0.0,0.7}

\definecolor{redshade}{rgb}{0.9,0.5,0.5}
\definecolor{greenshade}{rgb}{0.5,0.9,0.5}

\usepackage[unicode,colorlinks=true,linkcolor=mylinkcolor,citecolor=mycitecolor]{hyperref}
\newcommand{\refref}[2]{\hyperref[#2]{#1~\ref*{#2}}}
\newcommand{\eqnref}[1]{\hyperref[#1]{(\ref*{#1})}}

\newcommand{\tos}{\!\!\to\!\!}

\usepackage{framed}

\newcommand{\cequiv}{\simeq}

\makeatletter
\newcommand\xleftrightarrow[2][]{%
  \ext@arrow 9999{\longleftrightarrowfill@}{#1}{#2}}
\newcommand\longleftrightarrowfill@{%
  \arrowfill@\leftarrow\relbar\rightarrow}
\makeatother

\newcommand{\bsquare}{\textrm{\ding{114}}}

% % % % % % % % % % % % % % % % % % % % % % % % % % % % % %

\newtheoremstyle{myplain}
  {\topsep}   % ABOVESPACE
  {\topsep}   % BELOWSPACE
  {\itshape}  % BODYFONT
  {0pt}       % INDENT (empty value is the same as 0pt)
  {\bfseries} % HEADFONT
  {.}         % HEADPUNCT
  {5pt plus 1pt minus 1pt} % HEADSPACE
  {\thmnumber{#2}. \thmname{#1}\thmnote{ (#3)}}   % CUSTOM-HEAD-SPEC

\newtheoremstyle{myplainnameless}
  {\topsep}   % ABOVESPACE
  {\topsep}   % BELOWSPACE
  {\normalfont}  % BODYFONT
  {0pt}       % INDENT (empty value is the same as 0pt)
  {\bfseries} % HEADFONT
  {.}         % HEADPUNCT
  {5pt plus 1pt minus 1pt} % HEADSPACE
  {\thmnumber{#2}}   % CUSTOM-HEAD-SPEC 

\newtheoremstyle{sectionexercise}
  {\topsep}   % ABOVESPACE
  {\topsep}   % BELOWSPACE
  {\normalfont}  % BODYFONT
  {0pt}       % INDENT (empty value is the same as 0pt)
  {\bfseries} % HEADFONT
  {.}         % HEADPUNCT
  {5pt plus 1pt minus 1pt} % HEADSPACE
  {Ejercicio \thmnumber{#2}\thmnote{ (#3)}}   % CUSTOM-HEAD-SPEC

\newtheoremstyle{mydefinition}
  {\topsep}   % ABOVESPACE
  {\topsep}   % BELOWSPACE
  {\normalfont}  % BODYFONT
  {0pt}       % INDENT (empty value is the same as 0pt)
  {\bfseries} % HEADFONT
  {.}         % HEADPUNCT
  {5pt plus 1pt minus 1pt} % HEADSPACE
  {\thmnumber{#2}. \thmname{#1}\thmnote{ (#3)}}   % CUSTOM-HEAD-SPEC

% EN ESPAÑOL

\newtheorem*{hecho*}{Hecho}
\newtheorem*{corolario*}{Corolario}
\newtheorem*{teorema*}{Teorema}
\newtheorem*{conjetura*}{Conjetura}
\newtheorem*{proyecto*}{Proyecto}
\newtheorem*{observacion*}{Observación}

\newtheorem*{lema*}{Lema}
\newtheorem*{resultado-clave*}{Resultado clave}
\newtheorem*{proposicion*}{Proposición}

\theoremstyle{definition}
\newtheorem*{ejercicio*}{Ejercicio}
\newtheorem*{definicion*}{Definición}
\newtheorem*{comentario*}{Comentario}
\newtheorem*{definicion-alternativa*}{Definición alternativa}
\newtheorem*{ejemploxs}{Ejemplo}
\newenvironment{ejemplo*}
  {\pushQED{\qed}\renewcommand{\qedsymbol}{\examplesymbol}\ejemploxs}
  {\popQED\endejemploxs}

\theoremstyle{myplain}
\newtheorem{proposicion}{Proposición}[section]

\newtheorem{proyecto}[proposicion]{Proyecto}
\newtheorem{teorema}[proposicion]{Teorema}
\newtheorem{corolario}[proposicion]{Corolario}
\newtheorem{hecho}[proposicion]{Hecho}
\newtheorem{lema}[proposicion]{Lema}

\newtheorem{observacion}[proposicion]{Observación}

\newenvironment{observacionejerc}
    {\pushQED{\qed}\renewcommand{\qedsymbol}{$\square$}\csname inner@observacionejerc\endcsname}
    {\popQED\csname endinner@observacionejerc\endcsname}
\newtheorem{inner@observacionejerc}[proposicion]{Observación}

\newenvironment{proposicionejerc}
    {\pushQED{\qed}\renewcommand{\qedsymbol}{$\square$}\csname inner@proposicionejerc\endcsname}
    {\popQED\csname endinner@proposicionejerc\endcsname}
\newtheorem{inner@proposicionejerc}[proposicion]{Proposicion}

\newenvironment{lemaejerc}
    {\pushQED{\qed}\renewcommand{\qedsymbol}{$\square$}\csname inner@lemaejerc\endcsname}
    {\popQED\csname endinner@lemaejerc\endcsname}
\newtheorem{inner@lemaejerc}[proposicion]{Lema}

\newtheorem{calculo}[proposicion]{Cálculo}

\theoremstyle{myplainnameless}
\newtheorem{nameless}[proposicion]{}

\theoremstyle{mydefinition}
\newtheorem{comentario}[proposicion]{Comentario}
\newtheorem{comentarioast}[proposicion]{Comentario ($\clubsuit$)}
\newtheorem{construccion}[proposicion]{Construcción}
\newtheorem{aplicacion}[proposicion]{Aplicación}
\newtheorem{definicion}[proposicion]{Definición}
\newtheorem{definicion-alternativa}[proposicion]{Definición alternativa}
\newtheorem{notacion}[proposicion]{Notación}
\newtheorem{advertencia}[proposicion]{Advertencia}
\newtheorem{digresion}[proposicion]{Digresión}
\newtheorem{ejemplox}[proposicion]{Ejemplo}
\newenvironment{ejemplo}
  {\pushQED{\qed}\renewcommand{\qedsymbol}{\examplesymbol}\ejemplox}
  {\popQED\endejemplox}
\newtheorem{contraejemplox}[proposicion]{Contraejemplo}
\newenvironment{contraejemplo}
  {\pushQED{\qed}\renewcommand{\qedsymbol}{\examplesymbol}\contraejemplox}
  {\popQED\endcontraejemplox}
\newtheorem{noejemplox}[proposicion]{No-ejemplo}
\newenvironment{noejemplo}
  {\pushQED{\qed}\renewcommand{\qedsymbol}{\examplesymbol}\noejemplox}
  {\popQED\endnoejemplox}
 
\newtheorem{ejemploastx}[proposicion]{Ejemplo ($\clubsuit$)}
\newenvironment{ejemploast}
  {\pushQED{\qed}\renewcommand{\qedsymbol}{\examplesymbol}\ejemploastx}
  {\popQED\endejemploastx}

\ifdefined\exercisespersection
  \theoremstyle{sectionexercise}
  \newtheorem{ejercicio}{}[section]
  \theoremstyle{mydefinition}
\else
  \ifdefined\exercisesglobal
    \theoremstyle{sectionexercise}
    \newtheorem{ejercicio}{}
    \theoremstyle{mydefinition}
  \else
    \ifdefined\exercisespersection
      \newtheorem{ejercicio}[proposicion]{Ejercicio}
    \fi
  \fi
\fi

% % % % % % % % % % % % % % % % % % % % % % % % % % % % % %

\theoremstyle{myplain}
\newtheorem{proposition}{Proposition}[section]
\newtheorem*{fact*}{Fact}
\newtheorem*{proposition*}{Proposition}
\newtheorem{lemma}[proposition]{Lemma}
\newtheorem*{lemma*}{Lemma}

\newtheorem{exercise}{Exercise}
\newtheorem*{hint}{Hint}

\newtheorem{theorem}[proposition]{Theorem}
\newtheorem{conjecture}[proposition]{Conjecture}
\newtheorem*{theorem*}{Theorem}
\newtheorem{corollary}[proposition]{Corollary}
\newtheorem{fact}[proposition]{Fact}
\newtheorem*{claim}{Claim}
\newtheorem{definition-theorem}[proposition]{Definition-theorem}

\theoremstyle{mydefinition}
\newtheorem{examplex}[proposition]{Example}
\newenvironment{example}
  {\pushQED{\qed}\renewcommand{\qedsymbol}{\examplesymbol}\examplex}
  {\popQED\endexamplex}

\newtheorem*{examplexx}{Example}
\newenvironment{example*}
  {\pushQED{\qed}\renewcommand{\qedsymbol}{\examplesymbol}\examplexx}
  {\popQED\endexamplexx}

\newtheorem{definition}[proposition]{Definition}
\newtheorem*{definition*}{Definition}
\newtheorem{wrong-definition}[proposition]{Wrong definition}
\newtheorem{remark}[proposition]{Remark}

\makeatletter
\newcommand{\xRightarrow}[2][]{\ext@arrow 0359\Rightarrowfill@{#1}{#2}}
\makeatother

% % % % % % % % % % % % % % % % % % % % % % % % % % % % % %

\newcommand{\Et}{\mathop{\text{\rm Ét}}}

\newcommand{\categ}[1]{\text{\bf #1}}
\newcommand{\vcateg}{\mathcal}
\newcommand{\bone}{{\boldsymbol 1}}
\newcommand{\bDelta}{{\boldsymbol\Delta}}
\newcommand{\bR}{{\mathbf{R}}}

\newcommand{\univ}{\mathfrak}

\newcommand{\TODO}{\colorbox{red}{\textbf{*** TODO ***}}}
\newcommand{\proofreadme}{\colorbox{red}{\textbf{*** NEEDS PROOFREADING ***}}}

\makeatletter
\def\iddots{\mathinner{\mkern1mu\raise\p@
\vbox{\kern7\p@\hbox{.}}\mkern2mu
\raise4\p@\hbox{.}\mkern2mu\raise7\p@\hbox{.}\mkern1mu}}
\makeatother

\newcommand{\ssincl}{\reflectbox{\rotatebox[origin=c]{45}{$\subseteq$}}}
\newcommand{\vsupseteq}{\reflectbox{\rotatebox[origin=c]{-90}{$\supseteq$}}}
\newcommand{\vin}{\reflectbox{\rotatebox[origin=c]{90}{$\in$}}}

\newcommand{\Ga}{\mathbb{G}_\mathrm{a}}
\newcommand{\Gm}{\mathbb{G}_\mathrm{m}}

\renewcommand{\U}{\UUU}

\DeclareRobustCommand{\Stirling}{\genfrac\{\}{0pt}{}}
\DeclareRobustCommand{\stirling}{\genfrac[]{0pt}{}}

% % % % % % % % % % % % % % % % % % % % % % % % % % % % % %
% tikz

\usepackage{tikz-cd}
\usetikzlibrary{babel}
\usetikzlibrary{decorations.pathmorphing}
\usetikzlibrary{arrows}
\usetikzlibrary{calc}
\usetikzlibrary{fit}
\usetikzlibrary{hobby}

% % % % % % % % % % % % % % % % % % % % % % % % % % % % % %
% Banners

\newcommand\mybannerext[3]{{\normalfont\sffamily\bfseries\large\noindent #1

\noindent #2

\noindent #3

}\noindent\rule{\textwidth}{1.25pt}

\vspace{1em}}

\newcommand\mybanner[2]{{\normalfont\sffamily\bfseries\large\noindent #1

\noindent #2

}\noindent\rule{\textwidth}{1.25pt}

\vspace{1em}}

\renewcommand{\O}{\mathcal{O}}

\usepackage{diagbox}

\usepackage[numbers]{natbib}

\usepackage{fullpage}

\author{Alexey Beshenov (cadadr@gmail.com)}
\title{Series formales de potencias}
\date{27 de Febrero de 2017}

\begin{document}

{\normalfont\sffamily\bfseries \maketitle}

Toda sucesión de números $a_k$ puede ser vista como los coeficientes de una serie de potencias $\sum_k a_k\,t^k$. A veces esta serie surge como la \term{serie de Taylor} de una función real o compleja $f$:
$$\sum_{k\ge 0} \frac{f^{(k)} (t_0)}{k!} \, (t-t_0)^k$$
(cuando las derivadas de $f$ en $t_0$ existen). Las funciones que pueden ser representadas de tal manera se llaman \term{analíticas}. He aquí algunos ejemplos de series de Taylor:

\begin{gather*}
\text{\term{la serie geométrica} }\frac{1}{1-t} = \sum_{k \ge 0} t^k\quad\text{para }|t| < 1,\\
e^t = \sum_{k\ge 0} \frac{t^k}{k!}, \quad\quad \ln (1+t) = \sum_{k \ge 1} (-1)^{k+1}\,\frac{t^k}{k}\quad\text{para }|t| < 1,\\
\sen t = \sum_{k \ge 0} \frac{(-1)^k}{(2k+1)!}\,t^{2k+1}, \quad\quad \cos t = \sum_{k \ge 0} \frac{(-1)^k}{(2k)!}\,t^{2k}.
\end{gather*}

En general, la serie $\sum_k a_k\,t^k$ que corresponde a una sucesión arbitraria $(a_k)$ no tiene por qué ser convergente, aunque sería útil manipular con expresiones como ``$\sum_k a_k\,t^k$'' de manera puramente formal, como en efecto hacían los matemáticos de la época de Euler, cuando todavía no había una base rigurosa de análisis.

\begin{definicion*}
Una \term{serie formal de potencias} en variable $t$ con coeficientes racionales es una expresión
$$f (t) = \sum_{k \ge 0} a_k\,t^k = a_0 + a_1\,t + a_2\,t^2 + a_3\,t^3 + \cdots$$
donde $a_k \in \QQ$.
\end{definicion*}

Vamos a denotar el conjunto de tales series formales por $\QQ [\![t]\!]$. Las series formales se pueden manipular de la misma manera que los polinomios. A saber, la suma de dos series se calcula término por término:

\begin{equation}
\label{adicion-de-series-formales}
\left(\sum_k a_k\,t^k\right) + \left(\sum_k b_k\,t^k\right) \dfn \sum_k (a_k+b_k)\,t^k.
\end{equation}

\pagebreak

El producto de dos series se calcula mediante la distributividad formal:

\begin{align*}
(a_0 + a_1\,t + a_2\,t^2 + a_3\,t^3 + \cdots)\cdot (b_0 + b_1\,t + b_2\,t^2 + b_3\,t^3 + \cdots) & = a_0\,b_0 + ~\\
 & \quad (a_0\,b_1 + a_1\,b_0)\,t + ~ \\
 & \quad (a_0\,b_2 + a_1\,b_1 + a_2\,b_0)\,t^2 + ~ \\
 & \quad (a_0\,b_3 + a_1\,b_2 + a_2\,b_1 + a_3\,b_0)\,t^3 + \cdots
\end{align*}

\noindent Es decir,

\begin{equation}
\label{multiplicacion-de-series-formales}
\left(\sum_k a_k\,t^k\right) \cdot \left(\sum_k b_k\,t^k\right) \dfn \sum_k \left(\sum_{i+j=k} a_i\,b_j\right)\,t^k.
\end{equation}

Note que la adición y multiplicación de polinomios están definidos mediante las mismas fórmulas \eqnref{adicion-de-series-formales} y \eqnref{multiplicacion-de-series-formales}, y todo polinomio $a_n\,t^n + a_{n-1}\,t^{n-1} + \cdots a_1\,t + a_0$ puede ser visto como una serie formal de potencias
$$a_0 + a_1\,t + \cdots + a_{n-1}\,t^{n-1} + a_n\,t^n + 0\,t^{n+1} + 0\,t^{n+2} + \cdots$$
En otras palabras, un polinomio es una serie formal donde casi todos los coeficientes son nulos.

La adición y multiplicación de series satisfacen las propiedades habituales:

\begin{enumerate}
\item[1)] La suma es asociativa: $f(t)+(g(t)+h(t)) = (f(t)+g(t))+h(t)$ para todo $f(t),g(t),h(t)\in \QQ[\![t]\!]$.

\item[2)] La suma es conmutativa: $f(t)+g(t) = g(t)+f(t)$ para todo $f(t),g(t)\in \QQ[\![t]\!]$.

\item[3)] La \term{serie nula}
$$0 \dfn 0 + 0\,t + 0\,t^2 + 0\,t^3 + \cdots$$
es el cero respecto a la adición: $f (t) + 0 = f (t)$ para todo $f (t) \in \QQ [\![t]\!]$.

\item[4)] Para toda serie $f(t)\in \QQ[\![t]\!]$ tenemos la serie opuesta $-f(t)\in R$ tal que $f(t) + (-f(t)) = 0$.

(Si $f (t) = \sum_{k\ge 0} a_k\,t^k$, entonces $-f (t) = \sum_{k \ge 0} (-a_k)\,t^k$.)

Para $f (t) + (-g(t))$ normalmente se escribe $f(t)-g(t)$.

\item[5)] El producto es asociativo: $f(t)\cdot (g(t)\cdot h(t)) = (f(t)\cdot g(t))\cdot h(t)$ para todo $f(t),g(t),h(t)\in \QQ[\![t]\!]$.

\item[6)] El producto es conmutativo: $f(t)\cdot g(t) = g(t)\cdot f(t)$ para todo $f(t),g(t)\in \QQ[\![t]\!]$.

\item[7)] La \term{serie identidad}
$$1 \dfn 1 + 0\,t + 0\,t^2 + 0\,t^3 + \cdots$$
es la identidad respecto a la multiplicación: $f (t) \cdot 1 = f (t)$ para todo $f (t) \in \QQ [\![t]\!]$.

\item[8)] La multiplicación es distributiva respecto a la adición:
$$f(t)\cdot (g(t)+h(t)) = f(t)\cdot g(t) + f(t)\cdot h(t)$$
para todo $f(t), g(t), h(t) \in \QQ [\![t]\!]$. Ya que el producto es conmutativo, también tenemos
$$(f(t)+g(t))\cdot h(t) = f(t)\cdot h(t) + g(t)\cdot h(t).$$
\end{enumerate}

\pagebreak

Un ejemplo muy importante de las series formales de potencias que nos va a servir mucho es el siguiente.

\begin{definicion*}
La \term{función exponencial formal} es la serie en $\QQ [\![t]\!]$ definida como
$$e^t \dfn \sum_{k \ge 0} \frac{t^k}{k!}.$$
\end{definicion*}

\vspace{1em}

\begin{observacion*}
Si $f (t) \ne 0$ y $g (t) \ne 0$, entonces $f (t) \cdot g (t) \ne 0$.

\begin{proof}
Sean $f (t) = \sum_{k \ge 0} a_k\,t^k$ y $g (t) = \sum_{k \ge 0} b_k\,t^k$ dos series de potencias no nulas. Sea $a_i$ el primer coeficiente no nulo de $f (t)$ y sea $b_j$ el primer coeficiente no nulo en $g (t)$. El coeficiente de $t^{i+j}$ en $f (t) \cdot g (t)$ es
$$a_0\,b_{i+j} + a_1\,b_{i+j-1} + \cdots + a_i\,b_j + a_{i+1}\,b_{j-1} + \cdots + a_{i+j}\,b_0,$$
donde por nuestra elección de $a_i$ y $b_j$ todos los términos son nulos excepto $a_i\,b_j$, que no es nulo porque $a_i \ne 0$, $b_j \ne 0$.
\end{proof}
\end{observacion*}

\begin{definicion*}
Se dice que una serie de potencias $f (t) \in \QQ [\![t]\!]$ es \term{invertible} si existe otra serie $g (t) \in \QQ [\![t]\!]$ que es inversa a $f (t)$ respecto a la multiplicación; es decir, $f (t) \cdot g (t) = 1$. En este caso escribimos $g (t) = \frac{1}{f (t)}$. El producto $h (t) \cdot \frac{1}{f (t)}$ de una serie $h (t)$ con la serie inversa para $f (t)$ se escribe como una fracción $\frac{h (t)}{f (t)}$.
\end{definicion*}

Notemos que si $g (t)$ existe, es necesariamente única. En efecto, si hay dos series $g_1 (t)$ y $g_2 (t)$ tales que $f (t)\cdot g_1 (t) = f (t)\cdot g_2 (t) = 1$, entonces
$$g_2 (t) = \underbrace{f (t)\cdot g_1 (t)}_{= 1}\cdot g_2 (t) = \underbrace{f (t)\cdot g_2 (t)}_{=1}\cdot g_1 (t) = g_1 (t).$$

\begin{observacion*}
\label{obs:series-formales-invertibles}
Una serie $f (t) = \sum_{k \ge 0} a_k\,t^k \in \QQ [\![t]\!]$ es invertible si y solamente si $a_0 = \text{``}f (0)\text{''}$ no es nulo.
\end{observacion*}

\noindent Note que, en general, sumas infinitas de números racionales no están definidas, así que no se puede evaluar $f (t)$ en un número racional; es posible solo en análisis, donde hay nociones de convergencia. Sin embargo, $f (0)$ sí tiene sentido, y es el término constante de $f (t)$.

\begin{proof}
Estamos buscando otra serie $g (t) = \sum_{k \ge 0} b_k\,t^k \in \QQ [\![t]\!]$ tal que $f (t) \cdot g (t) = 1$, es decir,

\begin{align*}
a_0\,b_0 & = 1,\\
a_0\,b_1 + a_1\,b_0 & = 0,\\
a_0\,b_2 + a_1\,b_1 + a_2\,b_0 & = 0,\\
 & ~~\vdots \\
\sum_{0 \le i \le k} a_i\,b_{k-i} & = 0 \quad (k \ge 1)
\end{align*}

\pagebreak

De la primera ecuación se ve que $a_0$ tiene que ser no nulo. En este caso, podemos calcular $b_k$ sucesivamente:

\begin{align*}
b_0 & = a_0^{-1},\\
b_1 & = -a_0^{-1}\,(a_1\,b_0),\\
b_2 & = -a_0^{-1}\,(a_1\,b_1 + a_2\,b_0),\\
 & ~~\vdots \\
b_k & = -a_0^{-1}\,\sum_{1 \le i \le k} a_i\,b_{k-i}.
\end{align*}
\end{proof}

\begin{ejemplo*}
\label{serie-geometrica-formal}
Tenemos

\begin{align*}
(1-t)\cdot (1 + t + t^2 + t^3 + t^4 + \cdots) & = (1 + t + t^2 + t^3 + t^4 + \cdots) - (t + t^2 + t^3 + t^4 + t^5 + \cdots) = 1,\\
(1+t)\cdot (1 - t + t^2 - t^3 + t^4 - \cdots) & = (1 - t + t^2 - t^3 + t^4 - \cdots) + (t - t^2 + t^3 - t^4 + t^5 - \cdots) = 1.
\end{align*}

Es un análogo de la serie geométrica $\frac{1}{1-t} = \sum_{k \ge 0} t^k$, que en análisis tiene sentido para $|t| < 1$. En nuestro caso, $t$ es una variable formal.
\end{ejemplo*}

Como hemos visto, si tenemos una serie
$$f (t) = a_0 + a_1\,t + a_2\,t^2 + a_3\,t^3 + \cdots$$
tal que $a_0 = 0$, entonces $f (t)$ no es invertible en $\QQ [\![t]\!]$. Para resolver este problema, podemos introducir potencias negativas de $t$ y escribir
$$f (t) = t^{-n} \, (a_n + a_{n+1}\,t + a_{n+2}\,t^2 + a_{n+3}\,t^3 + \cdots),$$
donde $a_n$ es el primer coeficiente no nulo en $f (t)$. Aquí la serie entre paréntesis es invertible en $\QQ [\![t]\!]$. Para que tenga sentido el término ``$t^{-n}$'', podemos introducir la siguiente generalización.

\begin{definicion*}
Una \term{serie formal de Laurent}\footnote{\href{http://www-history.mcs.st-andrews.ac.uk/Biographies/Laurent_Pierre.html}{\personality{Pierre Alphonse Laurent}} (1813--1854), un matemático y oficial militar francés.} es una serie formal con un número finito de potencias negativas:
$$f(t) = \sum_{k \ge -N} a_k\,t^k\quad \text{para algún }N\in \NN.$$
\end{definicion*}

Para las series de Laurent también tienen sentido adición y multiplicación, definidas mediante las mismas fórmulas \eqnref{adicion-de-series-formales} y \eqnref{multiplicacion-de-series-formales}, y toda serie puede ser vista como una serie de Laurent con coeficientes negativos nulos. El conjunto de las series de Laurent se denota por $\QQ (\!(t)\!)$.

\vspace{1em}

Tenemos las siguientes generalizaciones de los resultados de arriba:

\begin{enumerate}
\item[1)] Si $f (t) \ne 0$ y $g (t) \ne 0$ son dos series de Laurent no nulas, entonces $f (t) \cdot g (t) \ne 0$ (la demostración es la misma).

\item[2)] Todas las series de Laurent no nulas son invertibles. En particular, toda serie no nula $f (t) \in \QQ [\![t]\!]$ es invertible en $\QQ (\!(t)\!)$.
\end{enumerate}

\begin{ejemplo*}
La serie $t + t^2 + t^3 + \cdots$ es invertible como serie de Laurent:
$$(t^{-1} - 1)\,(t + t^2 + t^3 + \cdots) = (1 + t + t^2 + \cdots) - (t + t^2 + t^3 + \cdots) = 1.$$
\end{ejemplo*}

\begin{shaded}
\small\noindent PARI/GP puede trabajar con series de potencias. Para indicar que los términos de grado $\ge n$ están omitidos, se escribe ``\verb|+ O(t^n)|'':

\begin{verbatim}
? 1/(1-t + O(t^10))
% = 1 + t + t^2 + t^3 + t^4 + t^5 + t^6 + t^7 + t^8 + t^9 + O(t^10)

? (t + 2*t^2 + 3*t^3 + 4*t^4 + 5*t^5 + O (t^6))^2
% = t^2 + 4*t^3 + 10*t^4 + 20*t^5 + 35*t^6 + O(t^7)
\end{verbatim}

\noindent Series de Laurent:

\begin{verbatim}
? 1/(t + t^2 + t^3 + t^4 + t^5 + t^6 + O (t^7))
% = t^-1 - 1 + O(t^5)
\end{verbatim}

\noindent PARI/GP conoce la exponencial formal:

\begin{verbatim}
? exp (t)
% = 1 + t + 1/2*t^2 + 1/6*t^3 + 1/24*t^4 + 1/120*t^5 + 1/720*t^6 +
     1/5040*t^7 + 1/40320*t^8 + 1/362880*t^9 + 1/3628800*t^10 +
     1/39916800*t^11 + 1/479001600*t^12 + 1/6227020800*t^13 +
     1/87178291200*t^14 + 1/1307674368000*t^15 + 1/20922789888000*t^16 +
     O(t^17)
\end{verbatim}

\noindent El número de términos se puede cambiar con el parámetro \verb|seriesprecision|:

\begin{verbatim}
? default (seriesprecision, 6)
? exp (t)
% = 1 + t + 1/2*t^2 + 1/6*t^3 + 1/24*t^4 + 1/120*t^5 + 1/720*t^6 + O(t^7)
\end{verbatim}

%\noindent Los números enteros $p$-ádicos $\ZZ_p$ pueden ser vistos como las series formales $\sum_{k \ge 0} a_k\,p^k$ con $a_k \in \{ 0, \ldots, p-1 \}$. He aquí un par de ejemplos:
%
%\begin{verbatim}
%? -1 + O(3^10)
%% = 2 + 2*3 + 2*3^2 + 2*3^3 + 2*3^4 + 2*3^5 + 2*3^6 + 2*3^7 + 2*3^8 + 2*3^9 + O(3^10)
%
%? (2 + O (3^10))^-1
%% = 2 + 3 + 3^2 + 3^3 + 3^4 + 3^5 + 3^6 + 3^7 + 3^8 + 3^9 + O(3^10)
%
%? sqrt (2+O(7^10))
%% = 3 + 7 + 2*7^2 + 6*7^3 + 7^4 + 2*7^5 + 7^6 + 2*7^7 + 4*7^8 + 6*7^9 + O(7^10)
%? %^2
%% = 2 + O(7^10)
%\end{verbatim}
%
%\noindent El primer ejemplo nos da la expresión canónica para $-1$ en el anillo $\ZZ_3$; en el segundo ejemplo PARI/GP calcula el número inverso a $2$ en $\ZZ_3$. El último ejemplo calcula la raíz cuadrada $7$-ádica de $2$.
\end{shaded}

\begin{definicion*}
Dadas dos series de potencias $f (t) = \sum_{k \ge 0} a_k \, t^k$ y $g (t) = \sum_{k \ge 0} b_k \, t^k$, si $g (0) = b_0 = 0$, entonces la \term{composición $(f \circ g) (t)$} (\term{sustitución de $g$ en $f$}) es la serie
$$(f \circ g) (t) \dfn f (g (t)) \dfn \sum_{k \ge 0} a_k \, g (t)^k.$$
Ya que $b_0 = 0$, toda potencia $g (t)^k$ no tiene términos de grado $< k$, así que la suma infinita tiene sentido.
\end{definicion*}

\begin{ejemplo*}
Si $f (t)$ es una serie formal tal que $f (0) = 0$, entonces

\begin{align*}
\frac{1}{1 - f (t)} & = 1 + f(t) + f (t)^2 + f(t)^3 + f (t)^4 + \cdots,\\
\frac{1}{1 + f (t)} & = 1 - f(t) + f (t)^2 - f(t)^3 + f (t)^4 - \cdots
\end{align*}

\noindent ---es una generalización de la serie geométrica.
\end{ejemplo*}

\begin{ejemplo*}
Podemos ``evaluar'' $e^t$ en $-t$. El resultado de la sustitución es la serie formal
$$e^{-t} = \sum_{k \ge 0} (-1)^k\,\frac{t^k}{k!}.$$

En general, podemos componer $e^t$ con toda $f (t)$ tal que $f (0) = 0$. Tenemos la identidad habitual
$$e^{f(t) + g(t)} = e^{f(t)}\cdot e^{g(t)}.$$
En efecto,

\begin{align*}
e^{f(t)} \cdot e^{g(t)} & = \left(\sum_{i \ge 0} \frac{f(t)^i}{i!}\right) \cdot \left(\sum_{j \ge 0} \frac{g(t)^j}{j!}\right)\\
 & = \sum_{k \ge 0} \sum_{i+j = k} \frac{k!}{k!} \frac{f(t)^i}{i!} \frac{g(t)^j}{j!} \\
 & = \sum_{k \ge 0} \frac{1}{k!} \sum_{i \ge 0} {k \choose i} \, f(t)^i \, g(t)^{k-i} \\
 & = \sum_{k \ge 0} \frac{(f(t)+g(t))^k}{k!} = e^{f(t)+g(t)}.
\end{align*}
\end{ejemplo*}

% % % % % % % % % % % % % % % % % % % % % % % % % % % % % %

\section*{Derivadas formales}

\begin{definicion*}
La \term{derivada formal} de una serie formal de potencias $f (t) = \sum_{k \ge 0} a_k\,t^k \in \QQ [\![t]\!]$ está definida por
$$f' (t) \dfn \sum_{k \ge 1} k\,a_k\,t^{k-1}.$$
\end{definicion*}

\begin{ejemplo*}
$$(e^t)' = \left(\sum_{k \ge 0} \frac{t^k}{k!}\right)' = \sum_{k \ge 1} k\,\frac{t^{k-1}}{k!} = \sum_{k \ge 1} \frac{t^{k-1}}{(k-1)!} = e^t.$$
\end{ejemplo*}

\begin{observacion*}[Serie de Taylor formal]
\label{serie-de-Taylor-formal}
Para las derivadas iteradas de $f (t) = \sum_{k \ge 0} a_k \, t^k \in \QQ [\![t]\!]$ se tiene $f^{(k)} (0) = k! \, a_k$, lo que nos da
$$f (t) = \sum_{k \ge 0} \frac{f^{(k)} (0)}{k!}\,t^k.$$

\begin{proof}
Se ve inmediatamente de las definiciones.
\end{proof}
\end{observacion*}

\begin{observacion*}
Para $f (t), g (t) \in \QQ [\![t]\!]$ se tiene
$$(f (t) + g (t))' = f' (t) + g' (t).$$

\begin{proof}
Evidente de la definición.
\end{proof}
\end{observacion*}

\begin{observacion*}[Regla de Leibniz]
Para $f (t), g (t) \in \QQ [\![t]\!]$ se tiene
$$(f (t)\cdot g (t))' = f' (t)\cdot g (t) + f (t)\cdot g' (t).$$

\pagebreak

\begin{proof}
Para $f (t) = \sum_{k \ge 0} a_k\,t^k$ y $g (t) = \sum_{k \ge 0} b_k\,t^k$

\begin{align*}
\left( \left(\sum_{k\ge 0} a_k\,t^k\right)\cdot \left(\sum_{k\ge 0} b_k\,t^k\right) \right)' & = \left(\sum_{k\ge 0} \left(\sum_{0 \le i \le k} a_i\,b_{k-i}\right)\,t^k\right)' \\
 & = \sum_{k\ge 1} k\,\left(\sum_{0 \le i \le k} a_i\,b_{k-i}\right)\,t^{k-1} \\
 & = \sum_{k\ge 1} \left(\sum_{0 \le i \le k} i \, a_i\,b_{k-i} + \sum_{0 \le i \le k} (k-i) \, a_i\,b_{k-i}\right)\,t^{k-1} \\
 & = \sum_{k \ge 1} \left(\sum_{0 \le i \le k} i \, a_i\,b_{k-i}\right)\,t^{k-1} + \sum_{k \ge 1} \left(\sum_{0 \le i \le k-1} a_i\,(k-i)\,b_{k-i}\right)\,t^{k-1} \\
 & = f' (t)\cdot g (t) + f (t)\cdot g' (t).
 \end{align*}
\end{proof}
\end{observacion*}

\begin{ejercicio*}
Demuestre que para $f(t), g(t) \in \QQ (\!(t)\!)$ se tiene
$$\left(\frac{f(t)}{g(t)}\right)' = \frac{f' (t) \cdot g(t) - f (t)\cdot g' (t)}{g (t)^2}.$$
\end{ejercicio*}

\begin{corolario*}
Para $f (t) \in \QQ [\![t]\!]$ se tiene
$$(f (t)^k)' = k\,f(t)'\,f(t)^{k-1}.$$

\begin{proof}
Por inducción, usando la regla de Leibniz.
\end{proof}
\end{corolario*}

\begin{observacion*}[Regla de la cadena]
\label{regla-de-la-cadena-formal}
Sean $f (t), g (t) \in \QQ [\![t]\!]$ dos series de potencias formales tales que $g (0) = 0$. Entonces para la composición se tiene
$$(f (g (t)))' = f' (g (t)) \cdot g' (t).$$

\begin{proof}
Si $f (t) = \sum_{k\ge 0} a_k\,t^k$, entonces
$$(f (g (t)))' = \sum_{k\ge 1} k\,a_k\,g' (t)\, (g (t))^{k-1} = \left( \sum_{k \ge 1} k\,a_k\,(g (t))^{k-1} \right) \, g' (t) = f' (g (t))\cdot g' (t).$$
\end{proof}
\end{observacion*}

\begin{shaded}
\noindent\small En PARI/GP:

\begin{verbatim}
? default (seriesprecision, 6)

? deriv (t*exp(t), t)         
% = 1 + 2*t + 3/2*t^2 + 2/3*t^3 + 5/24*t^4 + 1/20*t^5 + 7/720*t^6 + O(t^7)
\end{verbatim}
\end{shaded}

Para resumir, las derivadas formales se comportan como las derivadas habituales: son lineales, cumplen la regla de Leibniz y la regla de la cadena.

% % % % % % % % % % % % % % % % % % % % % % % % % % % % % %

\section*{Logaritmo formal}

\begin{definicion*}
El \term{logaritmo formal} es la serie en $\QQ [\![t]\!]$ definida por
$$\ln (1 + t) \dfn \sum_{k\ge 1} (-1)^{k+1}\,\frac{t^k}{k}.$$
\end{definicion*}

Observamos que la derivada formal de $\ln (1+t)$ es precisamente lo que se espera del logaritmo:
$$(\ln (1+t))' = \frac{1}{1+t} = 1 - t + t^2 - t^3 + t^4 - t^5 + \cdots.$$

\begin{shaded}
\noindent\small En PARI/GP:

\begin{verbatim}
? log (1+t)
% = t - 1/2*t^2 + 1/3*t^3 - 1/4*t^4 + 1/5*t^5 + O(t^6)
\end{verbatim}
\end{shaded}

\begin{teorema*}
Tenemos
$$\ln (1 + (e^t - 1)) = t, \quad e^{\ln (1 + t)} = 1 + t,$$
en el sentido de sustitución de una serie formal en otra.
\end{teorema*}

%\begin{ejercicio}
%Calcule los primeros términos de las series $\ln (e^t)$ y $e^{\ln (1 + t)}$ con ayuda de PARI/GP.
%\end{ejercicio}

Las identidades del teorema nos dan un ejemplo de series inversas respecto a la composición:

\begin{proposicion*}
Para una serie de potencias formal $f (t) = \sum_{k \ge 0} a_k \, t^k$ existe otra serie $g (t)$ tal que $g (0) = 0$ y $f (g (t)) = t$ si y solamente si $a_0 = 0$ y $a_1 \ne 0$. En este caso la serie $g (t)$ es única, y además se tiene $g (f (t)) = t$. Es decir, $f$ y $g$ son mutuamente inversas respecto a la composición.

\begin{proof}
La condición sobre $a_0$ y $a_1$ es necesaria: si existe $g (t) = \sum_{k\ge 0} b_k\,t^k$ con $b_0 = 0$ tal que $f (g (t)) = \sum_{k\ge 0} a_k\,g(t)^k = t$, entonces $a_0 = 0$ y $a_1\,b_1 = 1$.

Ahora sea $f (t)$ una serie con $a_0 = 0$ y $a_1 \ne 0$. Tenemos que encontrar una serie $g (t) = \sum_{k \ge 0} b_k\,t^k$ con $b_0 = 0$ tal que $f (g (t)) = t$. La última identidad implica que necesitamos poner $b_1 \dfn a_1^{-1}$. Luego, para $k \ge 2$, el coeficiente de $t^k$ en $f (g (t))$ es igual al coeficiente de $t^k$ en la suma
$$a_1 \, g (t) + a_2 \, g (t)^2 + \cdots + a_k \, g(t)^k$$
(ya que $g (0) = 0$, en las potencias $g (t)^{k+1}, g (t)^{k+2}, \ldots$ ya no hay términos de grado $k$). Pero este coeficiente tiene que ser nulo, lo que nos da las ecuaciones
$$a_1\,b_k + (\text{algún polinomio en }a_2, a_3, \ldots, a_k, b_1, b_2, \ldots, b_{k-1}) = 0.$$
Puesto que $a_1 \ne 0$, estas ecuaciones por inducción definen \emph{de modo único} todos los coeficientes $b_2, b_3, b_4, \ldots$ Esto demuestra que $g (t)$ existe y es único.

Para ver que también se tiene $g (f (t)) = t$, notamos que en $g (t)$ también $b_0 = 0$ y $b_1 \ne 0$, entonces existe $h (t)$ tal que $g (h (t)) = t$. Luego,

\begin{gather*}
t = f (g (t)),\\
h (t) = f (g (h (t))) = f (t),\\
g (h (t)) = g (f (t)) = t.
\end{gather*}
\end{proof}
\end{proposicion*}

\begin{shaded}
\noindent\small En PARI/GP, la serie inversa respecto a la composición puede ser calculada por la función \verb|serreverse|:

\begin{verbatim}
? serreverse (exp (t) - 1)
% = t - 1/2*t^2 + 1/3*t^3 - 1/4*t^4 + 1/5*t^5 - 1/6*t^6 + O(t^7)
\end{verbatim}
\end{shaded}

\begin{proof}[Demostración del teorema]
La primera tentación es calcular directamente los coeficientes de las series
$$\ln (1 + (e^t - 1)) \quad\text{y}\quad e^{\ln (1 + t)},$$
pero esto no es tan fácil. Por ejemplo, las potencias de la serie $e^t - 1$ tienen como coeficientes los números de Stirling:
$$\frac{(e^t - 1)^\ell}{\ell!} = \sum_{k \ge \ell} \Stirling{k}{\ell}\,\frac{t^k}{k!}.$$
Lo vamos a necesitar de todas maneras más adelante y ver las definiciones y las propiedades básicas de $\Stirling{k}{\ell}$ en otra lección. Para el logaritmo también hay una fórmula parecida con otros números de Stirling:
$$\frac{\ln (1+t)^\ell}{\ell!} = (-1)^\ell \, \sum_{k \ge \ell} (-1)^k\,\stirling{k}{\ell}\,\frac{t^k}{k!}.$$
Afortunadamente, por el momento se puede evitar esta pesadilla combinatoria. Primero notemos que gracias a la proposición de arriba, será suficiente demostrar que por ejemplo,
$$e^{\ln (1+t)} = 1+t,$$
y $\ln (1 + (e^t - 1)) = t$ se sigue automáticamente. Gracias a la serie de Taylor $f (t) = \sum_{k\ge 0} \frac{f^{(k)} (0)}{k!}\,t^k$, podemos simplemente verificar que

\begin{align*}
e^{\ln (1+0)} & = 1,\\
(e^{\ln (1+t)})' (0) & = 1,\\
(e^{\ln (1+t)})'' (0) & = 0,\\
(e^{\ln (1+t)})''' (0) & = 0,\\
& \cdots
\end{align*}

En efecto, $\ln (1 + 0) = 0$ y $e^0 = 1$. Luego, por la regla de la cadena,
$$(e^{\ln (1+t)})' = e^{\ln (1+t)} \, \frac{1}{1+t},$$
y así $(e^{\ln (1+t)})' (0) = 1$. La segunda derivada nos da

\begin{align*}
(e^{\ln (1+t)})'' & = \left(e^{\ln (1+t)} \, \frac{1}{1+t}\right)' \\
 & = (e^{\ln (1+t)})'\,\frac{1}{1+t} - e^{\ln (1+t)}\,\frac{1}{(1+t)^2} \\
  & = e^{\ln (1+t)} \, \frac{1}{1+t}\, \frac{1}{1+t} - e^{\ln (1+t)}\,\frac{1}{(1+t)^2} = 0.
\end{align*}
\end{proof}

\end{document}
