\documentclass{article}

% TODO : CLEAN UP THIS MESS
% (AND MAKE SURE ALL TEXTS STILL COMPILE)
\usepackage[leqno]{amsmath}
\usepackage{amssymb}
\usepackage{graphicx}

\usepackage{diagbox} % table heads with diagonal lines
\usepackage{relsize}

\usepackage{wasysym}
\usepackage{scrextend}
\usepackage{epigraph}
\setlength\epigraphwidth{.6\textwidth}

\usepackage[utf8]{inputenc}

\usepackage{titlesec}
\titleformat{\chapter}[display]
  {\normalfont\sffamily\huge\bfseries}
  {\chaptertitlename\ \thechapter}{5pt}{\Huge}
\titleformat{\section}
  {\normalfont\sffamily\Large\bfseries}
  {\thesection}{1em}{}
\titleformat{\subsection}
  {\normalfont\sffamily\large\bfseries}
  {\thesubsection}{1em}{}
\titleformat{\part}[display]
  {\normalfont\sffamily\huge\bfseries}
  {\partname\ \thepart}{0pt}{\Huge}

\usepackage[T1]{fontenc}
\usepackage{fourier}
\usepackage{paratype}

\usepackage[symbol,perpage]{footmisc}

\usepackage{perpage}
\MakePerPage{footnote}

\usepackage{array}
\newcolumntype{x}[1]{>{\centering\hspace{0pt}}p{#1}}

% TODO: the following line causes conflict with new texlive (!)
% \usepackage[english,russian,polutonikogreek,spanish]{babel}
% \newcommand{\russian}[1]{{\selectlanguage{russian}#1}}

% Remove conflicting options for the moment:
\usepackage[english,polutonikogreek,spanish]{babel}

\AtBeginDocument{\shorthandoff{"}}
\newcommand{\greek}[1]{{\selectlanguage{polutonikogreek}#1}}

% % % % % % % % % % % % % % % % % % % % % % % % % % % % % %
% Limit/colimit symbols (with accented i: lím / colím)

\usepackage{etoolbox} % \patchcmd

\makeatletter
\patchcmd{\varlim@}{lim}{\lim}{}{}
\makeatother
\DeclareMathOperator*{\colim}{co{\lim}}
\newcommand{\dirlim}{\varinjlim}
\newcommand{\invlim}{\varprojlim}

% % % % % % % % % % % % % % % % % % % % % % % % % % % % % %

\usepackage[all,color]{xy}

\usepackage{pigpen}
\newcommand{\po}{\ar@{}[dr]|(.4){\text{\pigpenfont I}}}
\newcommand{\pb}{\ar@{}[dr]|(.3){\text{\pigpenfont A}}}
\newcommand{\polr}{\ar@{}[dr]|(.65){\text{\pigpenfont A}}}
\newcommand{\pour}{\ar@{}[ur]|(.65){\text{\pigpenfont G}}}
\newcommand{\hstar}{\mathop{\bigstar}}

\newcommand{\bigast}{\mathop{\Huge \mathlarger{\mathlarger{\ast}}}}

\newcommand{\term}{\textbf}

\usepackage{stmaryrd}

\usepackage{cancel}

\usepackage{tikzsymbols}

\newcommand{\open}{\underset{\mathrm{open}}{\hookrightarrow}}
\newcommand{\closed}{\underset{\mathrm{closed}}{\hookrightarrow}}

\newcommand{\tcol}[2]{{#1 \choose #2}}

\newcommand{\homot}{\simeq}
\newcommand{\isom}{\cong}
\newcommand{\cH}{\mathcal{H}}
\renewcommand{\hom}{\mathrm{hom}}
\renewcommand{\div}{\mathop{\mathrm{div}}}
\renewcommand{\Im}{\mathop{\mathrm{Im}}}
\renewcommand{\Re}{\mathop{\mathrm{Re}}}
\newcommand{\id}[1]{\mathrm{id}_{#1}}
\newcommand{\idid}{\mathrm{id}}

\newcommand{\ZG}{{\ZZ G}}
\newcommand{\ZH}{{\ZZ H}}

\newcommand{\quiso}{\simeq}

\newcommand{\personality}[1]{{\sc #1}}

\newcommand{\mono}{\rightarrowtail}
\newcommand{\epi}{\twoheadrightarrow}
\newcommand{\xepi}[1]{\xrightarrow{#1}\mathrel{\mkern-14mu}\rightarrow}

% % % % % % % % % % % % % % % % % % % % % % % % % % % % % %

\DeclareMathOperator{\Ad}{Ad}
\DeclareMathOperator{\Aff}{Aff}
\DeclareMathOperator{\Ann}{Ann}
\DeclareMathOperator{\Aut}{Aut}
\DeclareMathOperator{\Br}{Br}
\DeclareMathOperator{\CH}{CH}
\DeclareMathOperator{\Cl}{Cl}
\DeclareMathOperator{\Coeq}{Coeq}
\DeclareMathOperator{\Coind}{Coind}
\DeclareMathOperator{\Cop}{Cop}
\DeclareMathOperator{\Corr}{Corr}
\DeclareMathOperator{\Cor}{Cor}
\DeclareMathOperator{\Cov}{Cov}
\DeclareMathOperator{\Der}{Der}
\DeclareMathOperator{\Div}{Div}
\DeclareMathOperator{\D}{D}
\DeclareMathOperator{\Ehr}{Ehr}
\DeclareMathOperator{\End}{End}
\DeclareMathOperator{\Eq}{Eq}
\DeclareMathOperator{\Ext}{Ext}
\DeclareMathOperator{\Frac}{Frac}
\DeclareMathOperator{\Frob}{Frob}
\DeclareMathOperator{\Funct}{Funct}
\DeclareMathOperator{\Fun}{Fun}
\DeclareMathOperator{\GL}{GL}
\DeclareMathOperator{\Gal}{Gal}
\DeclareMathOperator{\Gr}{Gr}
\DeclareMathOperator{\Hol}{Hol}
\DeclareMathOperator{\Hom}{Hom}
\DeclareMathOperator{\Ho}{Ho}
\DeclareMathOperator{\Id}{Id}
\DeclareMathOperator{\Ind}{Ind}
\DeclareMathOperator{\Inn}{Inn}
\DeclareMathOperator{\Isom}{Isom}
\DeclareMathOperator{\Ker}{Ker}
\DeclareMathOperator{\Lan}{Lan}
\DeclareMathOperator{\Lie}{Lie}
\DeclareMathOperator{\Map}{Map}
\DeclareMathOperator{\Mat}{Mat}
\DeclareMathOperator{\Max}{Max}
\DeclareMathOperator{\Mor}{Mor}
\DeclareMathOperator{\Nat}{Nat}
\DeclareMathOperator{\Nrd}{Nrd}
\DeclareMathOperator{\Ob}{Ob}
\DeclareMathOperator{\Out}{Out}
\DeclareMathOperator{\PGL}{PGL}
\DeclareMathOperator{\PSL}{PSL}
\DeclareMathOperator{\PSU}{PSU}
\DeclareMathOperator{\Pic}{Pic}
\DeclareMathOperator{\RHom}{RHom}
\DeclareMathOperator{\Rad}{Rad}
\DeclareMathOperator{\Ran}{Ran}
\DeclareMathOperator{\Rep}{Rep}
\DeclareMathOperator{\Res}{Res}
\DeclareMathOperator{\SL}{SL}
\DeclareMathOperator{\SO}{SO}
\DeclareMathOperator{\SU}{SU}
\DeclareMathOperator{\Sh}{Sh}
\DeclareMathOperator{\Sing}{Sing}
\DeclareMathOperator{\Specm}{Specm}
\DeclareMathOperator{\Spec}{Spec}
\DeclareMathOperator{\Sp}{Sp}
\DeclareMathOperator{\Stab}{Stab}
\DeclareMathOperator{\Sym}{Sym}
\DeclareMathOperator{\Tors}{Tors}
\DeclareMathOperator{\Tor}{Tor}
\DeclareMathOperator{\Tot}{Tot}
\DeclareMathOperator{\UUU}{U}

\DeclareMathOperator{\adj}{adj}
\DeclareMathOperator{\ad}{ad}
\DeclareMathOperator{\af}{af}
\DeclareMathOperator{\card}{card}
\DeclareMathOperator{\cm}{cm}
\DeclareMathOperator{\codim}{codim}
\DeclareMathOperator{\cod}{cod}
\DeclareMathOperator{\coeq}{coeq}
\DeclareMathOperator{\coim}{coim}
\DeclareMathOperator{\coker}{coker}
\DeclareMathOperator{\cont}{cont}
\DeclareMathOperator{\conv}{conv}
\DeclareMathOperator{\cor}{cor}
\DeclareMathOperator{\depth}{depth}
\DeclareMathOperator{\diag}{diag}
\DeclareMathOperator{\diam}{diam}
\DeclareMathOperator{\dist}{dist}
\DeclareMathOperator{\dom}{dom}
\DeclareMathOperator{\eq}{eq}
\DeclareMathOperator{\ev}{ev}
\DeclareMathOperator{\ex}{ex}
\DeclareMathOperator{\fchar}{char}
\DeclareMathOperator{\fr}{fr}
\DeclareMathOperator{\gr}{gr}
\DeclareMathOperator{\im}{im}
\DeclareMathOperator{\infl}{inf}
\DeclareMathOperator{\interior}{int}
\DeclareMathOperator{\intrel}{intrel}
\DeclareMathOperator{\inv}{inv}
\DeclareMathOperator{\length}{length}
\DeclareMathOperator{\mcd}{mcd}
\DeclareMathOperator{\mcm}{mcm}
\DeclareMathOperator{\multideg}{multideg}
\DeclareMathOperator{\ord}{ord}
\DeclareMathOperator{\pr}{pr}
\DeclareMathOperator{\rel}{rel}
\DeclareMathOperator{\res}{res}
\DeclareMathOperator{\rkred}{rkred}
\DeclareMathOperator{\rkss}{rkss}
\DeclareMathOperator{\rk}{rk}
\DeclareMathOperator{\sgn}{sgn}
\DeclareMathOperator{\sk}{sk}
\DeclareMathOperator{\supp}{supp}
\DeclareMathOperator{\trdeg}{trdeg}
\DeclareMathOperator{\tr}{tr}
\DeclareMathOperator{\vol}{vol}

\newcommand{\iHom}{\underline{\Hom}}

\renewcommand{\AA}{\mathbb{A}}
\newcommand{\CC}{\mathbb{C}}
\renewcommand{\SS}{\mathbb{S}}
\newcommand{\TT}{\mathbb{T}}
\newcommand{\PP}{\mathbb{P}}
\newcommand{\BB}{\mathbb{B}}
\newcommand{\RR}{\mathbb{R}}
\newcommand{\ZZ}{\mathbb{Z}}
\newcommand{\FF}{\mathbb{F}}
\newcommand{\HH}{\mathbb{H}}
\newcommand{\NN}{\mathbb{N}}
\newcommand{\QQ}{\mathbb{Q}}
\newcommand{\KK}{\mathbb{K}}

% % % % % % % % % % % % % % % % % % % % % % % % % % % % % %

\usepackage{amsthm}

\newcommand{\legendre}[2]{\left(\frac{#1}{#2}\right)}

\newcommand{\examplesymbol}{$\blacktriangle$}
\renewcommand{\qedsymbol}{$\blacksquare$}

\newcommand{\dfn}{\mathrel{\mathop:}=}
\newcommand{\rdfn}{=\mathrel{\mathop:}}

\usepackage{xcolor}
\definecolor{mylinkcolor}{rgb}{0.0,0.4,1.0}
\definecolor{mycitecolor}{rgb}{0.0,0.4,1.0}
\definecolor{shadecolor}{rgb}{0.79,0.78,0.65}
\definecolor{gray}{rgb}{0.6,0.6,0.6}

\usepackage{colortbl}

\definecolor{myred}{rgb}{0.7,0.0,0.0}
\definecolor{mygreen}{rgb}{0.0,0.7,0.0}
\definecolor{myblue}{rgb}{0.0,0.0,0.7}

\definecolor{redshade}{rgb}{0.9,0.5,0.5}
\definecolor{greenshade}{rgb}{0.5,0.9,0.5}

\usepackage[unicode,colorlinks=true,linkcolor=mylinkcolor,citecolor=mycitecolor]{hyperref}
\newcommand{\refref}[2]{\hyperref[#2]{#1~\ref*{#2}}}
\newcommand{\eqnref}[1]{\hyperref[#1]{(\ref*{#1})}}

\newcommand{\tos}{\!\!\to\!\!}

\usepackage{framed}

\newcommand{\cequiv}{\simeq}

\makeatletter
\newcommand\xleftrightarrow[2][]{%
  \ext@arrow 9999{\longleftrightarrowfill@}{#1}{#2}}
\newcommand\longleftrightarrowfill@{%
  \arrowfill@\leftarrow\relbar\rightarrow}
\makeatother

\newcommand{\bsquare}{\textrm{\ding{114}}}

% % % % % % % % % % % % % % % % % % % % % % % % % % % % % %

\newtheoremstyle{myplain}
  {\topsep}   % ABOVESPACE
  {\topsep}   % BELOWSPACE
  {\itshape}  % BODYFONT
  {0pt}       % INDENT (empty value is the same as 0pt)
  {\bfseries} % HEADFONT
  {.}         % HEADPUNCT
  {5pt plus 1pt minus 1pt} % HEADSPACE
  {\thmnumber{#2}. \thmname{#1}\thmnote{ (#3)}}   % CUSTOM-HEAD-SPEC

\newtheoremstyle{myplainnameless}
  {\topsep}   % ABOVESPACE
  {\topsep}   % BELOWSPACE
  {\normalfont}  % BODYFONT
  {0pt}       % INDENT (empty value is the same as 0pt)
  {\bfseries} % HEADFONT
  {.}         % HEADPUNCT
  {5pt plus 1pt minus 1pt} % HEADSPACE
  {\thmnumber{#2}}   % CUSTOM-HEAD-SPEC 

\newtheoremstyle{sectionexercise}
  {\topsep}   % ABOVESPACE
  {\topsep}   % BELOWSPACE
  {\normalfont}  % BODYFONT
  {0pt}       % INDENT (empty value is the same as 0pt)
  {\bfseries} % HEADFONT
  {.}         % HEADPUNCT
  {5pt plus 1pt minus 1pt} % HEADSPACE
  {Ejercicio \thmnumber{#2}\thmnote{ (#3)}}   % CUSTOM-HEAD-SPEC

\newtheoremstyle{mydefinition}
  {\topsep}   % ABOVESPACE
  {\topsep}   % BELOWSPACE
  {\normalfont}  % BODYFONT
  {0pt}       % INDENT (empty value is the same as 0pt)
  {\bfseries} % HEADFONT
  {.}         % HEADPUNCT
  {5pt plus 1pt minus 1pt} % HEADSPACE
  {\thmnumber{#2}. \thmname{#1}\thmnote{ (#3)}}   % CUSTOM-HEAD-SPEC

% EN ESPAÑOL

\newtheorem*{hecho*}{Hecho}
\newtheorem*{corolario*}{Corolario}
\newtheorem*{teorema*}{Teorema}
\newtheorem*{conjetura*}{Conjetura}
\newtheorem*{proyecto*}{Proyecto}
\newtheorem*{observacion*}{Observación}

\newtheorem*{lema*}{Lema}
\newtheorem*{resultado-clave*}{Resultado clave}
\newtheorem*{proposicion*}{Proposición}

\theoremstyle{definition}
\newtheorem*{ejercicio*}{Ejercicio}
\newtheorem*{definicion*}{Definición}
\newtheorem*{comentario*}{Comentario}
\newtheorem*{definicion-alternativa*}{Definición alternativa}
\newtheorem*{ejemploxs}{Ejemplo}
\newenvironment{ejemplo*}
  {\pushQED{\qed}\renewcommand{\qedsymbol}{\examplesymbol}\ejemploxs}
  {\popQED\endejemploxs}

\theoremstyle{myplain}
\newtheorem{proposicion}{Proposición}[section]

\newtheorem{proyecto}[proposicion]{Proyecto}
\newtheorem{teorema}[proposicion]{Teorema}
\newtheorem{corolario}[proposicion]{Corolario}
\newtheorem{hecho}[proposicion]{Hecho}
\newtheorem{lema}[proposicion]{Lema}

\newtheorem{observacion}[proposicion]{Observación}

\newenvironment{observacionejerc}
    {\pushQED{\qed}\renewcommand{\qedsymbol}{$\square$}\csname inner@observacionejerc\endcsname}
    {\popQED\csname endinner@observacionejerc\endcsname}
\newtheorem{inner@observacionejerc}[proposicion]{Observación}

\newenvironment{proposicionejerc}
    {\pushQED{\qed}\renewcommand{\qedsymbol}{$\square$}\csname inner@proposicionejerc\endcsname}
    {\popQED\csname endinner@proposicionejerc\endcsname}
\newtheorem{inner@proposicionejerc}[proposicion]{Proposicion}

\newenvironment{lemaejerc}
    {\pushQED{\qed}\renewcommand{\qedsymbol}{$\square$}\csname inner@lemaejerc\endcsname}
    {\popQED\csname endinner@lemaejerc\endcsname}
\newtheorem{inner@lemaejerc}[proposicion]{Lema}

\newtheorem{calculo}[proposicion]{Cálculo}

\theoremstyle{myplainnameless}
\newtheorem{nameless}[proposicion]{}

\theoremstyle{mydefinition}
\newtheorem{comentario}[proposicion]{Comentario}
\newtheorem{comentarioast}[proposicion]{Comentario ($\clubsuit$)}
\newtheorem{construccion}[proposicion]{Construcción}
\newtheorem{aplicacion}[proposicion]{Aplicación}
\newtheorem{definicion}[proposicion]{Definición}
\newtheorem{definicion-alternativa}[proposicion]{Definición alternativa}
\newtheorem{notacion}[proposicion]{Notación}
\newtheorem{advertencia}[proposicion]{Advertencia}
\newtheorem{digresion}[proposicion]{Digresión}
\newtheorem{ejemplox}[proposicion]{Ejemplo}
\newenvironment{ejemplo}
  {\pushQED{\qed}\renewcommand{\qedsymbol}{\examplesymbol}\ejemplox}
  {\popQED\endejemplox}
\newtheorem{contraejemplox}[proposicion]{Contraejemplo}
\newenvironment{contraejemplo}
  {\pushQED{\qed}\renewcommand{\qedsymbol}{\examplesymbol}\contraejemplox}
  {\popQED\endcontraejemplox}
\newtheorem{noejemplox}[proposicion]{No-ejemplo}
\newenvironment{noejemplo}
  {\pushQED{\qed}\renewcommand{\qedsymbol}{\examplesymbol}\noejemplox}
  {\popQED\endnoejemplox}
 
\newtheorem{ejemploastx}[proposicion]{Ejemplo ($\clubsuit$)}
\newenvironment{ejemploast}
  {\pushQED{\qed}\renewcommand{\qedsymbol}{\examplesymbol}\ejemploastx}
  {\popQED\endejemploastx}

\ifdefined\exercisespersection
  \theoremstyle{sectionexercise}
  \newtheorem{ejercicio}{}[section]
  \theoremstyle{mydefinition}
\else
  \ifdefined\exercisesglobal
    \theoremstyle{sectionexercise}
    \newtheorem{ejercicio}{}
    \theoremstyle{mydefinition}
  \else
    \ifdefined\exercisespersection
      \newtheorem{ejercicio}[proposicion]{Ejercicio}
    \fi
  \fi
\fi

% % % % % % % % % % % % % % % % % % % % % % % % % % % % % %

\theoremstyle{myplain}
\newtheorem{proposition}{Proposition}[section]
\newtheorem*{fact*}{Fact}
\newtheorem*{proposition*}{Proposition}
\newtheorem{lemma}[proposition]{Lemma}
\newtheorem*{lemma*}{Lemma}

\newtheorem{exercise}{Exercise}
\newtheorem*{hint}{Hint}

\newtheorem{theorem}[proposition]{Theorem}
\newtheorem{conjecture}[proposition]{Conjecture}
\newtheorem*{theorem*}{Theorem}
\newtheorem{corollary}[proposition]{Corollary}
\newtheorem{fact}[proposition]{Fact}
\newtheorem*{claim}{Claim}
\newtheorem{definition-theorem}[proposition]{Definition-theorem}

\theoremstyle{mydefinition}
\newtheorem{examplex}[proposition]{Example}
\newenvironment{example}
  {\pushQED{\qed}\renewcommand{\qedsymbol}{\examplesymbol}\examplex}
  {\popQED\endexamplex}

\newtheorem*{examplexx}{Example}
\newenvironment{example*}
  {\pushQED{\qed}\renewcommand{\qedsymbol}{\examplesymbol}\examplexx}
  {\popQED\endexamplexx}

\newtheorem{definition}[proposition]{Definition}
\newtheorem*{definition*}{Definition}
\newtheorem{wrong-definition}[proposition]{Wrong definition}
\newtheorem{remark}[proposition]{Remark}

\makeatletter
\newcommand{\xRightarrow}[2][]{\ext@arrow 0359\Rightarrowfill@{#1}{#2}}
\makeatother

% % % % % % % % % % % % % % % % % % % % % % % % % % % % % %

\newcommand{\Et}{\mathop{\text{\rm Ét}}}

\newcommand{\categ}[1]{\text{\bf #1}}
\newcommand{\vcateg}{\mathcal}
\newcommand{\bone}{{\boldsymbol 1}}
\newcommand{\bDelta}{{\boldsymbol\Delta}}
\newcommand{\bR}{{\mathbf{R}}}

\newcommand{\univ}{\mathfrak}

\newcommand{\TODO}{\colorbox{red}{\textbf{*** TODO ***}}}
\newcommand{\proofreadme}{\colorbox{red}{\textbf{*** NEEDS PROOFREADING ***}}}

\makeatletter
\def\iddots{\mathinner{\mkern1mu\raise\p@
\vbox{\kern7\p@\hbox{.}}\mkern2mu
\raise4\p@\hbox{.}\mkern2mu\raise7\p@\hbox{.}\mkern1mu}}
\makeatother

\newcommand{\ssincl}{\reflectbox{\rotatebox[origin=c]{45}{$\subseteq$}}}
\newcommand{\vsupseteq}{\reflectbox{\rotatebox[origin=c]{-90}{$\supseteq$}}}
\newcommand{\vin}{\reflectbox{\rotatebox[origin=c]{90}{$\in$}}}

\newcommand{\Ga}{\mathbb{G}_\mathrm{a}}
\newcommand{\Gm}{\mathbb{G}_\mathrm{m}}

\renewcommand{\U}{\UUU}

\DeclareRobustCommand{\Stirling}{\genfrac\{\}{0pt}{}}
\DeclareRobustCommand{\stirling}{\genfrac[]{0pt}{}}

% % % % % % % % % % % % % % % % % % % % % % % % % % % % % %
% tikz

\usepackage{tikz-cd}
\usetikzlibrary{babel}
\usetikzlibrary{decorations.pathmorphing}
\usetikzlibrary{arrows}
\usetikzlibrary{calc}
\usetikzlibrary{fit}
\usetikzlibrary{hobby}

% % % % % % % % % % % % % % % % % % % % % % % % % % % % % %
% Banners

\newcommand\mybannerext[3]{{\normalfont\sffamily\bfseries\large\noindent #1

\noindent #2

\noindent #3

}\noindent\rule{\textwidth}{1.25pt}

\vspace{1em}}

\newcommand\mybanner[2]{{\normalfont\sffamily\bfseries\large\noindent #1

\noindent #2

}\noindent\rule{\textwidth}{1.25pt}

\vspace{1em}}

\renewcommand{\O}{\mathcal{O}}

\usepackage{diagbox}

\usepackage[numbers]{natbib}

\usepackage{fullpage}

\author{Alexey Beshenov (cadadr@gmail.com)}
\title{La función generatriz para $B_k$. Polinomios de Bernoulli}
\date{28 de Febrero de 2017}

\usepackage{xcolor}
\newcommand{\highlight}[1]{\colorbox{shadecolor}{$\displaystyle #1$}}

\begin{document}

{\normalfont\sffamily\bfseries \maketitle}

\section*{La función generatriz para $B_k$}

\begin{teorema*}
Los números de Bernoulli pueden ser definidos por
$$\frac{t\,e^t}{e^t - 1} = \sum_{k \ge 0} B_k \, \frac{t^k}{k!}.$$
\end{teorema*}

Aunque se puede pensar en esta identidad como en la serie de Taylor para $\frac{t\,e^t}{e^t - 1}$ en un entorno de $0$, para nosotros esto significa nada más que el cociente de series formales $\frac{t\,e^t}{e^t - 1}$ en $\QQ (\!(t)\!)$ es igual a la serie formal $\sum_{k \ge 0} B_k \, \frac{t^k}{k!}$.

\begin{proof}
Tenemos que ver que la identidad
$$\left(\sum_{k \ge 0} B_k \, \frac{t^k}{k!}\right)\,(e^t-1) = t\,e^t.$$
define los números de Bernoulli. Calculemos el producto al lado izquierdo:

\begin{align*}
\left(\sum_{k \ge 0} B_k \, \frac{t^k}{k!}\right)\,(e^t-1) & = 
\left(\sum_{k \ge 0} B_k \, \frac{t^k}{k!}\right)\,\left(\sum_{k\ge 1} \frac{t^k}{k!}\right) \\
 & = \sum_{k \ge 1} \left(\sum_{0 \le i \le k-1} \frac{B_i}{i!}\,\frac{1}{(k-i)!}\right)\,t^k \\
 & = \sum_{k \ge 1} \left(\sum_{0 \le i \le k-1} \frac{B_i}{i!}\,\frac{k!}{(k-i)!}\right)\,\frac{t^k}{k!} \\
 & = \sum_{k \ge 1} \left(\sum_{0 \le i \le k-1} {k \choose i} B_i\right)\,\frac{t^k}{k!} \\
 & \stackrel{???}{=} \sum_{k\ge 1} \frac{t^k}{(k-1)!} = t\,e^t.
\end{align*}

La última igualdad se cumple si y solamente si
$$\sum_{0 \le i \le k-1} {k \choose i} B_i = k.$$
Como hemos visto, esta identidad define los números de Bernoulli.
\end{proof}

\begin{ejemplo*}
Calculemos algunos términos de la serie formal $\frac{t\,e^t}{e^t - 1}$. Tenemos
$$e^t - 1 = t + \frac{t^2}{2} + \frac{t^3}{6} + \frac{t^4}{24} + \frac{t^5}{120} + \cdots = t\,\left(1 + \frac{t}{2!} + \frac{t^2}{3!} + \frac{t^3}{4!} + \frac{t^4}{5!} + \cdots\right).$$
Luego,
$$\frac{t}{e^t - 1} = \cfrac{1}{1 + \cfrac{t}{2!} + \cfrac{t^2}{3!} + \cfrac{t^3}{4!} + \cfrac{t^4}{5!} + \cdots}$$
Podemos calcular la última serie usando la fórmula
$$\frac{1}{1 + f (t)} = 1 - f(t) + f (t)^2 - f(t)^3 + f (t)^4 - \cdots$$ Tenemos

\begin{multline*}
1 -
\left(\frac{t}{2!} + \frac{t^2}{3!} + \frac{t^3}{4!} + \frac{t^4}{5!} + \cdots\right) +
\left(\frac{t}{2!} + \frac{t^2}{3!} + \frac{t^3}{4!} + \frac{t^4}{5!} + \cdots\right)^2\\
- \left(\frac{t}{2!} + \frac{t^2}{3!} + \frac{t^3}{4!} + \frac{t^4}{5!} + \cdots\right)^3 +
\left(\frac{t}{2!} + \frac{t^2}{3!} + \frac{t^3}{4!} + \frac{t^4}{5!} + \cdots\right)^4 - \cdots\\
= 1 -
\left(\frac{t}{2} + \frac{t^2}{6} + \frac{t^3}{24} + \frac{t^4}{120} + \cdots\right) +
\left(\frac{t^2}{4} + \frac{t^3}{6} + \frac{5\,t^4}{72} + \cdots\right) - \left(\frac{t^3}{8} + \frac{t^4}{8} + \cdots\right) + \left(\frac{t^4}{16} + \cdots\right) - \cdots \\
= 1 - \frac{t}{2} + \frac{t^2}{12} + 0\cdot t^3 - \frac{t^4}{720} + \cdots
\end{multline*}

Multiplicando las seres, se obtiene
$$\frac{t\,e^t}{e^t - 1} = \left(1 + t + \frac{t^2}{2!} + \frac{t^3}{3!} + \frac{t^4}{4!} + \cdots\right) \cdot \left(1 - \frac{t}{2} + \frac{t^2}{12} + 0\cdot t^3 - \frac{t^4}{720} + \cdots\right) = 1 + \frac{t}{2} + \frac{t^2}{12} - \frac{t^4}{720} + \cdots$$
y entonces
$$B_0 = 1, \quad B_1 = \frac{1}{2}, \quad B_2 = 2! \cdot \frac{1}{12} = \frac{1}{6}, \quad B_3 = 0, \quad B_4 = -4!\cdot\frac{1}{720} = -\frac{1}{30}.$$
\end{ejemplo*}

\begin{shaded}
\small\noindent Por supuesto, el último ejemplo es un poco masoquista: todo esto se puede hacer en PARI/GP.

\begin{verbatim}
? ser = (t*exp(t))/(exp(t)-1)
% = 1 + 1/2*t + 1/12*t^2 - 1/720*t^4 + 1/30240*t^6 - 1/1209600*t^8 + 1/47900160*t^10
- 691/1307674368000*t^12 + 1/74724249600*t^14 + O(t^16)

? vector (11,k, polcoeff(ser,(k-1),t)*(k-1)!)
% = [1, 1/2, 1/6, 0, -1/30, 0, 1/42, 0, -1/30, 0, 5/66]
\end{verbatim}
\end{shaded}

En muchos libros (y también en PARI/GP) se usa otra convención para los números de Bernoulli según la cual $B_1 = -\frac{1}{2}$. En este caso la función generatriz es $\frac{t\,e^t}{e^t - 1} - t = \frac{t}{e^t - 1}$.

\begin{ejemplo*}
La fórmula $\frac{t\,e^t}{e^t - 1} = \sum_{k \ge 0} \frac{B_k}{k!}\,t^k$ nos permite demostrar que $B_k = 0$ para $k \ge 3$ impar. En efecto, para ignorar el caso excepcional $B_1 = \frac{1}{2}$, examinemos la función
$$f (t) \dfn \frac{t\,e^t}{e^t - 1} - \frac{t}{2} = B_0 + \frac{B_2}{2!}\,t^2 + \frac{B_3}{3!}\,t^3 + \frac{B_4}{4!}\,t^4 + \frac{B_5}{5!}\,t^5 + \cdots$$
Tenemos
$$f (t) = \frac{t\,e^t}{e^t - 1} - \frac{t}{2} = \frac{t\,(e^t - 1 + 1)}{e^t - 1} - \frac{t}{2} = \frac{t}{e^t - 1} + \frac{t}{2}.$$
Luego,
$$f (-t) = \frac{(-t)\,e^{-t}}{e^{-t}-1} - \frac{(-t)}{2} = \frac{t}{e^t - 1} + \frac{t}{2}.$$
Entonces, $f (t) = f (-t)$, lo que implica que los coeficientes impares de $f (t)$ son nulos.
\end{ejemplo*}

Los números de Bernoulli también surgen en otras series. Por ejemplo, tenemos la siguiente

\begin{proposicion*}
$$t\,\cot (t) = 1 + \sum_{k \ge 1} (-1)^k \, 2^{2k}\,\frac{B_{2k}}{(2k)!}\,t^{2k}.$$
\end{proposicion*}

\noindent Esto ya tiene que ser interpretado analíticamente. Las series con números de Bernoulli para varias funciones como $\tan (t)$, $\cot (t)$, $\tanh (t)$, $\coth (t)$ fueron descubiertas por Euler.

\begin{proof}
Se tiene
$$\cos (t) = \frac{e^{it} + e^{-it}}{2}, \quad \sen (t) = \frac{e^{it} - e^{-it}}{2i}.$$
Luego,

\begin{align*}
t\,\cot (t) & = t\,\frac{\cos (t)}{\sen (t)} = it\,\frac{e^{it} + e^{-it}}{e^{it} - e^{-it}} = it\,\frac{e^{2it} + 1}{e^{2it} - 1} = -it + \frac{2it \, e^{2it}}{e^{2it} - 1}\\
 & = -it + \sum_{k \ge 0} \frac{B_k\cdot (2it)^k}{k!} = 1 + \sum_{k \ge 1} (-1)^k \, 2^{2k} \, \frac{B_{2k}}{(2k)!}\,t^{2k}.
\end{align*}
\end{proof}

\begin{ejercicio*}[Euler]
% "Bernoulli Numbers and Zeta Functions", 1.15
Demuestre la identidad
$$(2k + 1)\,B_{2k} = -\sum_{1 \le \ell \le k-1} {2k \choose 2\ell}\,B_{2\ell}\,B_{2\,(k-\ell)} \quad \text{para }k\ge 2.$$
Por ejemplo, para $k = 3$ tenemos
$$-7\,\underbrace{B_6}_{= \frac{1}{42}} = \underbrace{{6 \choose 2}\,B_2\,B_4}_{= 15\cdot \frac{1}{6}\cdot\left(-\frac{1}{30}\right)} + \underbrace{{6 \choose 4}\,B_4\,B_2}_{= 15\cdot\left(-\frac{1}{30}\right)\cdot \frac{1}{6}}.$$

\vspace{1em}

\noindent Indicación: considere la función generatriz para los números pares $f (t) \dfn \frac{t\,e^t}{e^t - 1} - \frac{t}{2} = \sum_{k \ge 0} \frac{B_{2k}}{(2k)!}\,t^{2k}$. Demuestre la identidad con la derivada formal $f (t) - t\,f(t)' = f(t)^2 - \frac{t^2}{4}$; sustituya $f (t)$ por $\sum_{k \ge 0} \frac{B_{2k}}{(2k)!}\,t^{2k}$ y compare los coeficientes de $t^{2k}$.
\end{ejercicio*}

\begin{ejercicio*}
% "Bernoulli Numbers and Zeta Functions", 1.16
Demuestre por inducción que $(-1)^{k+1}\,B_{2k} > 0$ para todo $k \ge 1$.

\noindent Indicación: use el ejercicio anterior.
\end{ejercicio*}

% % % % % % % % % % % % % % % % % % % % % % % % % % % % % %

\section*{Polinomios de Bernoulli}

Hay varios modos de definir los polinomios de Bernoulli; el más común es por una función generatriz. Vamos a necesitar las series de potencias formales en dos variables:
$$\sum_{k, \ell \ge 0} a_{k,\ell} \, t^k \, x^\ell,$$
respecto a la suma término por término y multiplicación que extiende la multiplicación de polinomios en dos variables. Tenemos la serie formal $\frac{t}{e^t - 1} \in \QQ [\![t]\!] \subset \QQ [\![t,x]\!]$ y podemos multiplicarla por la serie
$$e^{tx} \dfn \sum_{k \ge 0} \frac{t^k\,x^k}{k!} \in \QQ [\![t,x]\!].$$
Un momento de reflexión demuestra que el resultado es de la forma

\begin{equation}
\label{polinomios-de-Bernoulli-por-una-funcion-generatriz}
\frac{t\,e^{tx}}{e^t-1} \dfn \sum_{k \ge 0} B_k (x)\,\frac{t^k}{k!},
\end{equation}

\noindent donde $B_k (x)$ son algunos \emph{polinomios} en $x$.

\begin{definicion*}
El \term{polinomio de Bernoulli} $B_k (x)$ es el polinomio definido por \eqnref{polinomios-de-Bernoulli-por-una-funcion-generatriz}.
\end{definicion*}

\begin{ejemplo*}
Vamos a ver un poco más adelante cómo calcular los polinomios $B_k (x)$; por el momento podemos obtener algunos de los primeros. Como hemos calculado arriba,
$$\frac{t}{e^t - 1} = 1 - \frac{t}{2} + \frac{t^2}{12} - \frac{t^4}{720} + \cdots$$
Luego,
$$\frac{t}{e^t - 1} \, e^{tx} = \left(1 - \frac{t}{2} + \frac{t^2}{12} - \cdots\right) \, \left(1 + t\,x + \frac{t^2\,x^2}{2} + \cdots\right) = 1 + \left(x - \frac{1}{2}\right)\,t + \left(\frac{x^2}{2} - \frac{x}{2} + \frac{1}{12}\right)\,t^2 + \cdots$$
de donde
$$B_0 (x) = 1, \quad B_1 (x) = x - \frac{1}{2}, \quad B_2 (x) = x^2 - x + \frac{1}{6}.$$
\end{ejemplo*}

\begin{observacion*}
Para todo $k\ge 0$,
$$B_k (1) = B_k$$
es el $k$-ésimo número de Bernoulli.

\begin{proof}
Comparando \eqnref{polinomios-de-Bernoulli-por-una-funcion-generatriz} con la función generatriz $\frac{t\,e^t}{e^t-1} = \sum_{k \ge 0} B_k\,\frac{t^k}{k!}$.
\end{proof}
\end{observacion*}

Resulta que el término constante de $B_k (x)$ es también igual a $B_k$:

\begin{observacion*}
Para todo $k \ge 0$

\begin{equation}
\label{Bk(x+1)-y-Bk(x)}
B_k (x+1) - B_k (x) = k\,x^{k-1}.
\end{equation}

En particular, para $x = 0$ y $k \ne 1$ tenemos
$$B_k (1) = B_k (0) = B_k.$$

\begin{proof}
Tenemos la identidad
$$\frac{t\,e^{(x+1)\,t}}{e^t-1} - \frac{t\,e^{t\,x}}{e^t-1} = t\,e^{t\,x},$$
de donde
$$\sum_{k\ge 0} (B_k (x+1) - B_k (x))\,\frac{t^k}{k!} = \sum_{k \ge 0} \frac{x^k}{k!}\,t^{k+1}.$$
Comparando los coeficientes de $t^k$, se obtiene \eqnref{Bk(x+1)-y-Bk(x)}.
\end{proof}
\end{observacion*}

Note que para $k = 1$ tenemos $B_1 (0) = -\frac{1}{2}$ y $B_1 (1) = +\frac{1}{2}$.

\begin{observacion*}
Para todo $k \ge 0$
$$B_k (1-x) = (-1)^k\,B_k (x).$$
\end{observacion*}

\noindent (En particular, para $x = 0$ tenemos $B_k = (-1)^k\,B_k$ para $k \ge 3$, lo que implica que $B_k = 0$ para $k \ge 3$ impar, como ya hemos visto.)

\begin{proof}
Usando funciones generatrices,
$$\sum_{k\ge 0} B_k (1-x)\,\frac{t^k}{k!} = \frac{t\,e^{(1-x)\,t}}{e^t - 1} = \frac{(-t)\,e^{x\,(-t)}}{e^{-t} - 1} = \sum_{k\ge 0} (-1)^k\,B_k (x)\,\frac{t^k}{k!}.$$
\end{proof}

Los polinomios de Bernoulli pueden ser expresados en términos de los números de Bernoulli:

\begin{proposicion*}
$$B_k (x) = \sum_{0 \le i \le k} (-1)^i \, {k \choose i}\,B_i\,x^{k-i}.$$

\begin{proof}
Calculemos el producto de series de potencias
$$\frac{t}{e^t-1}\cdot e^{tx}.$$
Tenemos
$$\frac{t}{e^t-1} = \frac{(-t)\,e^{-t}}{e^{-t} - 1} = \sum_{k \ge 0} (-1)^k \, B_k \, \frac{t^k}{k!}, \quad e^{tx} = \sum_{k \ge 0} \frac{(tx)^k}{k!}.$$
Luego,

\begin{align*}
\left(\sum_{k \ge 0} (-1)^k \, B_k \, \frac{t^k}{k!}\right) \cdot \left(\sum_{k \ge 0} \frac{(tx)^k}{k!}\right) & = \sum_{k \ge 0} \left(\sum_{0 \le i \le k} (-1)^i \, \frac{1}{i! \, (k-i)!} \, B_i\,x^{k-i}\right)\,t^k \\
 & = \sum_{k \ge 0} \left(\sum_{0 \le i \le k} (-1)^i \, {k \choose i} \, B_i\,x^{k-i}\right)\,\frac{t^k}{k!}.
\end{align*}
\end{proof}
\end{proposicion*}

\begin{proposicion*}
Para todo $k \ge 1$ se tiene
$$B_k' (x) = k\,B_{k-1} (x), \quad \int_0^1 B_k (x)\,dx = 0.$$

\begin{proof}
Hay varios modos de verificar esto. Se puede usar la expresión $B_k (x) = \sum_{0 \le i \le k} (-1)^i \, {k \choose i}\,B_i\,x^{k-i}$. También podemos tomar las derivadas formales de la identidad \eqnref{funcion-generatriz-para-los-polinomios-de-bernoulli}:
$$\frac{\partial}{\partial x} \left(\frac{t\,e^{tx}}{e^t - 1}\right) = \frac{t\cdot t\,e^{tx}}{e^t - 1} = t\,\sum_{k \ge 0} B_k (x) \, \frac{t^k}{k!} = \sum_{k \ge 1} B_{k-1} (x) \, \frac{t^k}{(k-1)!} = \sum_{k \ge 0} B_k' (x) \, \frac{t^k}{k!}.$$
Luego, para ver que $\int_0^1 B_k (x)\,dx = 0$, es suficiente observar que $\int B_k (x)\,dx = \frac{1}{k+1}\,B_{k+1} (x) + C$, donde $B_{k+1} (0) = B_{k+1} (1)$.
\end{proof}
\end{proposicion*}

Esto nos da otra definición de los polinomios de Bernoulli:

\begin{definicion-alternativa*}
Los polinomios $B_k (x)$ están definidos por
$$B_0 (x) \dfn 1$$
y
$$B_k' (x) = k\,B_{k-1} (x), \quad \int_0^1 B_k (x) \, dx = 0 \quad \text{para }k \ge 1.$$
\end{definicion-alternativa*}

\noindent (En efecto, la identidad $B_k' (x) = k\,B_{k-1} (x)$ define $B_k (x)$ salvo el término constante, pero el último se recupera de la condición $\int_0^1 B_k (x) \, dx = 0$.) Recordemos que los polinomios $S_k (x)$ que hemos estudiado en la primera lección satisfacen la identidad
$$S_k' (x) = k\,S_{k-1} (x) + B_k.$$
Esto significa que las derivadas $S_k' (x)$ satisfacen la misma identidad que $B_k (x)$:
$$S_k'' (x) = k\,S_{k-1}' (x).$$
Además, para $k\ne 1$ tenemos $B_k (0) = S_k' (0) \rdfn B_k$, y se ve que los polinomios de Bernoulli son simplemente las derivadas de los polinomios $S_k (x)$:
$$B_k (x) = S_k' (x), \quad \text{para }k\ne 1.$$
(El caso $k = 1$ es excepcional: $S_1 (x) = \frac{1}{2}\,x^2\,+\frac{1}{2}\,x$, $B_1 (x) = x - \frac{1}{2}$.)

\pagebreak

Ahora podemos compilar fácilmente una lista de los primeros polinomios de Bernoulli:

\begin{align*}
B_0 (x) & = \highlight{1}, \\
B_1 (x) & = x - \frac{1}{2}, \\
B_2 (x) & = x^2 - x + \highlight{\frac{1}{6}}, \\
B_3 (x) & = x^3 - \frac{3}{2}\,x^2 + \frac{1}{2}\,x,\\
B_4 (x) & = x^4 - 2\,x^3 + x^2 \highlight{- \frac{1}{30}}, \\
B_5 (x) & = x^5 - \frac{5}{2}\,x^4 + \frac{5}{3}\,x^3 - \frac{1}{6}\,x, \\
B_6 (x) & = x^6 - 3\,x^5 + \frac{5}{2}\,x^4 - \frac{1}{2}\,x^2 + \highlight{\frac{1}{42}},\\
B_7 (x) & = x^7 - \frac{7}{2}\,x^6 + \frac{7}{2}\,x^5 - \frac{7}{6}\,x^3 + \frac{1}{6}\,x, \\
B_8 (x) & = x^8 - 4\,x^7 + \frac{14}{3}\,x^6 - \frac{7}{3}\,x^4 + \frac{2}{3}\,x^2 \highlight{- \frac{1}{30}},\\
B_9 (x) & = x^9 - \frac{9}{2}\,x^8 + 6\,x^7 - \frac{21}{5}\,x^5 + 2\,x^3 - \frac{3}{10}\,x, \\
B_{10} (x) & = x^{10} - 5\,x^9 + \frac{15}{2}\,x^8 - 7\,x^6 + 5\,x^4 - \frac{3}{2}\,x^2 \highlight{+ \frac{5}{66}}.
\end{align*}

\vspace{\fill}

Podemos dibujar algunas gráficas para visualizar la relación $B_k (1-x) = (-1)^k \, B_k (x)$:

\begin{center}
\includegraphics{../pic/bernpol.mps}
\end{center}

\pagebreak

\begin{shaded}
\small

\begin{verbatim}
? Bpoly (k) = sum (i=0,k, (-1)^i * binomial(k,i) * B(i) * x^(k-i));

? vector (10,k,Bpoly(k))  
% = [x - 1/2,
     x^2 - x + 1/6,
     x^3 - 3/2*x^2 + 1/2*x,
     x^4 - 2*x^3 + x^2 - 1/30,
     x^5 - 5/2*x^4 + 5/3*x^3 - 1/6*x,
     x^6 - 3*x^5 + 5/2*x^4 - 1/2*x^2 + 1/42,
     x^7 - 7/2*x^6 + 7/2*x^5 - 7/6*x^3 + 1/6*x,
     x^8 - 4*x^7 + 14/3*x^6 - 7/3*x^4 + 2/3*x^2 - 1/30,
     x^9 - 9/2*x^8 + 6*x^7 - 21/5*x^5 + 2*x^3 - 3/10*x,
     x^10 - 5*x^9 + 15/2*x^8 - 7*x^6 + 5*x^4 - 3/2*x^2 + 5/66]
\end{verbatim}

\begin{verbatim}
? deriv (Bpoly(10),x)
% = 10*x^9 - 45*x^8 + 60*x^7 - 42*x^5 + 20*x^3 - 3*x
? 10 * Bpoly(9)
% = 10*x^9 - 45*x^8 + 60*x^7 - 42*x^5 + 20*x^3 - 3*x
\end{verbatim}

\noindent Para comprobar los resultados, podemos directamente calcular la serie $\frac{t\,e^{tx}}{e^t - 1}$:

\begin{verbatim}
? ser = t*exp (t*x) / (exp (t) - 1);
? polcoeff(ser,10,t)*10!
% = x^10 - 5*x^9 + 15/2*x^8 - 7*x^6 + 5*x^4 - 3/2*x^2 + 5/66
\end{verbatim}

\noindent También podemos calcular las derivadas de $S_k (x)$:

\begin{verbatim}
? deriv (S(10),x)
% = x^10 + 5*x^9 + 15/2*x^8 - 7*x^6 + 5*x^4 - 3/2*x^2 + 5/66
\end{verbatim}

\noindent En PARI/GP, la función predefinida \verb|bernpol(k)| devuelve el polinomio de Bernoulli $B_k (x)$:

\begin{verbatim}
? bernpol(1)
% = x - 1/2
? bernpol(2)
% = x^2 - x + 1/6
? bernpol(3)
% = x^3 - 3/2*x^2 + 1/2*x
\end{verbatim}
\end{shaded}

\end{document}
