\documentclass{article}

% TODO : CLEAN UP THIS MESS
% (AND MAKE SURE ALL TEXTS STILL COMPILE)
\usepackage[leqno]{amsmath}
\usepackage{amssymb}
\usepackage{graphicx}

\usepackage{diagbox} % table heads with diagonal lines
\usepackage{relsize}

\usepackage{wasysym}
\usepackage{scrextend}
\usepackage{epigraph}
\setlength\epigraphwidth{.6\textwidth}

\usepackage[utf8]{inputenc}

\usepackage{titlesec}
\titleformat{\chapter}[display]
  {\normalfont\sffamily\huge\bfseries}
  {\chaptertitlename\ \thechapter}{5pt}{\Huge}
\titleformat{\section}
  {\normalfont\sffamily\Large\bfseries}
  {\thesection}{1em}{}
\titleformat{\subsection}
  {\normalfont\sffamily\large\bfseries}
  {\thesubsection}{1em}{}
\titleformat{\part}[display]
  {\normalfont\sffamily\huge\bfseries}
  {\partname\ \thepart}{0pt}{\Huge}

\usepackage[T1]{fontenc}
\usepackage{fourier}
\usepackage{paratype}

\usepackage[symbol,perpage]{footmisc}

\usepackage{perpage}
\MakePerPage{footnote}

\usepackage{array}
\newcolumntype{x}[1]{>{\centering\hspace{0pt}}p{#1}}

% TODO: the following line causes conflict with new texlive (!)
% \usepackage[english,russian,polutonikogreek,spanish]{babel}
% \newcommand{\russian}[1]{{\selectlanguage{russian}#1}}

% Remove conflicting options for the moment:
\usepackage[english,polutonikogreek,spanish]{babel}

\AtBeginDocument{\shorthandoff{"}}
\newcommand{\greek}[1]{{\selectlanguage{polutonikogreek}#1}}

% % % % % % % % % % % % % % % % % % % % % % % % % % % % % %
% Limit/colimit symbols (with accented i: lím / colím)

\usepackage{etoolbox} % \patchcmd

\makeatletter
\patchcmd{\varlim@}{lim}{\lim}{}{}
\makeatother
\DeclareMathOperator*{\colim}{co{\lim}}
\newcommand{\dirlim}{\varinjlim}
\newcommand{\invlim}{\varprojlim}

% % % % % % % % % % % % % % % % % % % % % % % % % % % % % %

\usepackage[all,color]{xy}

\usepackage{pigpen}
\newcommand{\po}{\ar@{}[dr]|(.4){\text{\pigpenfont I}}}
\newcommand{\pb}{\ar@{}[dr]|(.3){\text{\pigpenfont A}}}
\newcommand{\polr}{\ar@{}[dr]|(.65){\text{\pigpenfont A}}}
\newcommand{\pour}{\ar@{}[ur]|(.65){\text{\pigpenfont G}}}
\newcommand{\hstar}{\mathop{\bigstar}}

\newcommand{\bigast}{\mathop{\Huge \mathlarger{\mathlarger{\ast}}}}

\newcommand{\term}{\textbf}

\usepackage{stmaryrd}

\usepackage{cancel}

\usepackage{tikzsymbols}

\newcommand{\open}{\underset{\mathrm{open}}{\hookrightarrow}}
\newcommand{\closed}{\underset{\mathrm{closed}}{\hookrightarrow}}

\newcommand{\tcol}[2]{{#1 \choose #2}}

\newcommand{\homot}{\simeq}
\newcommand{\isom}{\cong}
\newcommand{\cH}{\mathcal{H}}
\renewcommand{\hom}{\mathrm{hom}}
\renewcommand{\div}{\mathop{\mathrm{div}}}
\renewcommand{\Im}{\mathop{\mathrm{Im}}}
\renewcommand{\Re}{\mathop{\mathrm{Re}}}
\newcommand{\id}[1]{\mathrm{id}_{#1}}
\newcommand{\idid}{\mathrm{id}}

\newcommand{\ZG}{{\ZZ G}}
\newcommand{\ZH}{{\ZZ H}}

\newcommand{\quiso}{\simeq}

\newcommand{\personality}[1]{{\sc #1}}

\newcommand{\mono}{\rightarrowtail}
\newcommand{\epi}{\twoheadrightarrow}
\newcommand{\xepi}[1]{\xrightarrow{#1}\mathrel{\mkern-14mu}\rightarrow}

% % % % % % % % % % % % % % % % % % % % % % % % % % % % % %

\DeclareMathOperator{\Ad}{Ad}
\DeclareMathOperator{\Aff}{Aff}
\DeclareMathOperator{\Ann}{Ann}
\DeclareMathOperator{\Aut}{Aut}
\DeclareMathOperator{\Br}{Br}
\DeclareMathOperator{\CH}{CH}
\DeclareMathOperator{\Cl}{Cl}
\DeclareMathOperator{\Coeq}{Coeq}
\DeclareMathOperator{\Coind}{Coind}
\DeclareMathOperator{\Cop}{Cop}
\DeclareMathOperator{\Corr}{Corr}
\DeclareMathOperator{\Cor}{Cor}
\DeclareMathOperator{\Cov}{Cov}
\DeclareMathOperator{\Der}{Der}
\DeclareMathOperator{\Div}{Div}
\DeclareMathOperator{\D}{D}
\DeclareMathOperator{\Ehr}{Ehr}
\DeclareMathOperator{\End}{End}
\DeclareMathOperator{\Eq}{Eq}
\DeclareMathOperator{\Ext}{Ext}
\DeclareMathOperator{\Frac}{Frac}
\DeclareMathOperator{\Frob}{Frob}
\DeclareMathOperator{\Funct}{Funct}
\DeclareMathOperator{\Fun}{Fun}
\DeclareMathOperator{\GL}{GL}
\DeclareMathOperator{\Gal}{Gal}
\DeclareMathOperator{\Gr}{Gr}
\DeclareMathOperator{\Hol}{Hol}
\DeclareMathOperator{\Hom}{Hom}
\DeclareMathOperator{\Ho}{Ho}
\DeclareMathOperator{\Id}{Id}
\DeclareMathOperator{\Ind}{Ind}
\DeclareMathOperator{\Inn}{Inn}
\DeclareMathOperator{\Isom}{Isom}
\DeclareMathOperator{\Ker}{Ker}
\DeclareMathOperator{\Lan}{Lan}
\DeclareMathOperator{\Lie}{Lie}
\DeclareMathOperator{\Map}{Map}
\DeclareMathOperator{\Mat}{Mat}
\DeclareMathOperator{\Max}{Max}
\DeclareMathOperator{\Mor}{Mor}
\DeclareMathOperator{\Nat}{Nat}
\DeclareMathOperator{\Nrd}{Nrd}
\DeclareMathOperator{\Ob}{Ob}
\DeclareMathOperator{\Out}{Out}
\DeclareMathOperator{\PGL}{PGL}
\DeclareMathOperator{\PSL}{PSL}
\DeclareMathOperator{\PSU}{PSU}
\DeclareMathOperator{\Pic}{Pic}
\DeclareMathOperator{\RHom}{RHom}
\DeclareMathOperator{\Rad}{Rad}
\DeclareMathOperator{\Ran}{Ran}
\DeclareMathOperator{\Rep}{Rep}
\DeclareMathOperator{\Res}{Res}
\DeclareMathOperator{\SL}{SL}
\DeclareMathOperator{\SO}{SO}
\DeclareMathOperator{\SU}{SU}
\DeclareMathOperator{\Sh}{Sh}
\DeclareMathOperator{\Sing}{Sing}
\DeclareMathOperator{\Specm}{Specm}
\DeclareMathOperator{\Spec}{Spec}
\DeclareMathOperator{\Sp}{Sp}
\DeclareMathOperator{\Stab}{Stab}
\DeclareMathOperator{\Sym}{Sym}
\DeclareMathOperator{\Tors}{Tors}
\DeclareMathOperator{\Tor}{Tor}
\DeclareMathOperator{\Tot}{Tot}
\DeclareMathOperator{\UUU}{U}

\DeclareMathOperator{\adj}{adj}
\DeclareMathOperator{\ad}{ad}
\DeclareMathOperator{\af}{af}
\DeclareMathOperator{\card}{card}
\DeclareMathOperator{\cm}{cm}
\DeclareMathOperator{\codim}{codim}
\DeclareMathOperator{\cod}{cod}
\DeclareMathOperator{\coeq}{coeq}
\DeclareMathOperator{\coim}{coim}
\DeclareMathOperator{\coker}{coker}
\DeclareMathOperator{\cont}{cont}
\DeclareMathOperator{\conv}{conv}
\DeclareMathOperator{\cor}{cor}
\DeclareMathOperator{\depth}{depth}
\DeclareMathOperator{\diag}{diag}
\DeclareMathOperator{\diam}{diam}
\DeclareMathOperator{\dist}{dist}
\DeclareMathOperator{\dom}{dom}
\DeclareMathOperator{\eq}{eq}
\DeclareMathOperator{\ev}{ev}
\DeclareMathOperator{\ex}{ex}
\DeclareMathOperator{\fchar}{char}
\DeclareMathOperator{\fr}{fr}
\DeclareMathOperator{\gr}{gr}
\DeclareMathOperator{\im}{im}
\DeclareMathOperator{\infl}{inf}
\DeclareMathOperator{\interior}{int}
\DeclareMathOperator{\intrel}{intrel}
\DeclareMathOperator{\inv}{inv}
\DeclareMathOperator{\length}{length}
\DeclareMathOperator{\mcd}{mcd}
\DeclareMathOperator{\mcm}{mcm}
\DeclareMathOperator{\multideg}{multideg}
\DeclareMathOperator{\ord}{ord}
\DeclareMathOperator{\pr}{pr}
\DeclareMathOperator{\rel}{rel}
\DeclareMathOperator{\res}{res}
\DeclareMathOperator{\rkred}{rkred}
\DeclareMathOperator{\rkss}{rkss}
\DeclareMathOperator{\rk}{rk}
\DeclareMathOperator{\sgn}{sgn}
\DeclareMathOperator{\sk}{sk}
\DeclareMathOperator{\supp}{supp}
\DeclareMathOperator{\trdeg}{trdeg}
\DeclareMathOperator{\tr}{tr}
\DeclareMathOperator{\vol}{vol}

\newcommand{\iHom}{\underline{\Hom}}

\renewcommand{\AA}{\mathbb{A}}
\newcommand{\CC}{\mathbb{C}}
\renewcommand{\SS}{\mathbb{S}}
\newcommand{\TT}{\mathbb{T}}
\newcommand{\PP}{\mathbb{P}}
\newcommand{\BB}{\mathbb{B}}
\newcommand{\RR}{\mathbb{R}}
\newcommand{\ZZ}{\mathbb{Z}}
\newcommand{\FF}{\mathbb{F}}
\newcommand{\HH}{\mathbb{H}}
\newcommand{\NN}{\mathbb{N}}
\newcommand{\QQ}{\mathbb{Q}}
\newcommand{\KK}{\mathbb{K}}

% % % % % % % % % % % % % % % % % % % % % % % % % % % % % %

\usepackage{amsthm}

\newcommand{\legendre}[2]{\left(\frac{#1}{#2}\right)}

\newcommand{\examplesymbol}{$\blacktriangle$}
\renewcommand{\qedsymbol}{$\blacksquare$}

\newcommand{\dfn}{\mathrel{\mathop:}=}
\newcommand{\rdfn}{=\mathrel{\mathop:}}

\usepackage{xcolor}
\definecolor{mylinkcolor}{rgb}{0.0,0.4,1.0}
\definecolor{mycitecolor}{rgb}{0.0,0.4,1.0}
\definecolor{shadecolor}{rgb}{0.79,0.78,0.65}
\definecolor{gray}{rgb}{0.6,0.6,0.6}

\usepackage{colortbl}

\definecolor{myred}{rgb}{0.7,0.0,0.0}
\definecolor{mygreen}{rgb}{0.0,0.7,0.0}
\definecolor{myblue}{rgb}{0.0,0.0,0.7}

\definecolor{redshade}{rgb}{0.9,0.5,0.5}
\definecolor{greenshade}{rgb}{0.5,0.9,0.5}

\usepackage[unicode,colorlinks=true,linkcolor=mylinkcolor,citecolor=mycitecolor]{hyperref}
\newcommand{\refref}[2]{\hyperref[#2]{#1~\ref*{#2}}}
\newcommand{\eqnref}[1]{\hyperref[#1]{(\ref*{#1})}}

\newcommand{\tos}{\!\!\to\!\!}

\usepackage{framed}

\newcommand{\cequiv}{\simeq}

\makeatletter
\newcommand\xleftrightarrow[2][]{%
  \ext@arrow 9999{\longleftrightarrowfill@}{#1}{#2}}
\newcommand\longleftrightarrowfill@{%
  \arrowfill@\leftarrow\relbar\rightarrow}
\makeatother

\newcommand{\bsquare}{\textrm{\ding{114}}}

% % % % % % % % % % % % % % % % % % % % % % % % % % % % % %

\newtheoremstyle{myplain}
  {\topsep}   % ABOVESPACE
  {\topsep}   % BELOWSPACE
  {\itshape}  % BODYFONT
  {0pt}       % INDENT (empty value is the same as 0pt)
  {\bfseries} % HEADFONT
  {.}         % HEADPUNCT
  {5pt plus 1pt minus 1pt} % HEADSPACE
  {\thmnumber{#2}. \thmname{#1}\thmnote{ (#3)}}   % CUSTOM-HEAD-SPEC

\newtheoremstyle{myplainnameless}
  {\topsep}   % ABOVESPACE
  {\topsep}   % BELOWSPACE
  {\normalfont}  % BODYFONT
  {0pt}       % INDENT (empty value is the same as 0pt)
  {\bfseries} % HEADFONT
  {.}         % HEADPUNCT
  {5pt plus 1pt minus 1pt} % HEADSPACE
  {\thmnumber{#2}}   % CUSTOM-HEAD-SPEC 

\newtheoremstyle{sectionexercise}
  {\topsep}   % ABOVESPACE
  {\topsep}   % BELOWSPACE
  {\normalfont}  % BODYFONT
  {0pt}       % INDENT (empty value is the same as 0pt)
  {\bfseries} % HEADFONT
  {.}         % HEADPUNCT
  {5pt plus 1pt minus 1pt} % HEADSPACE
  {Ejercicio \thmnumber{#2}\thmnote{ (#3)}}   % CUSTOM-HEAD-SPEC

\newtheoremstyle{mydefinition}
  {\topsep}   % ABOVESPACE
  {\topsep}   % BELOWSPACE
  {\normalfont}  % BODYFONT
  {0pt}       % INDENT (empty value is the same as 0pt)
  {\bfseries} % HEADFONT
  {.}         % HEADPUNCT
  {5pt plus 1pt minus 1pt} % HEADSPACE
  {\thmnumber{#2}. \thmname{#1}\thmnote{ (#3)}}   % CUSTOM-HEAD-SPEC

% EN ESPAÑOL

\newtheorem*{hecho*}{Hecho}
\newtheorem*{corolario*}{Corolario}
\newtheorem*{teorema*}{Teorema}
\newtheorem*{conjetura*}{Conjetura}
\newtheorem*{proyecto*}{Proyecto}
\newtheorem*{observacion*}{Observación}

\newtheorem*{lema*}{Lema}
\newtheorem*{resultado-clave*}{Resultado clave}
\newtheorem*{proposicion*}{Proposición}

\theoremstyle{definition}
\newtheorem*{ejercicio*}{Ejercicio}
\newtheorem*{definicion*}{Definición}
\newtheorem*{comentario*}{Comentario}
\newtheorem*{definicion-alternativa*}{Definición alternativa}
\newtheorem*{ejemploxs}{Ejemplo}
\newenvironment{ejemplo*}
  {\pushQED{\qed}\renewcommand{\qedsymbol}{\examplesymbol}\ejemploxs}
  {\popQED\endejemploxs}

\theoremstyle{myplain}
\newtheorem{proposicion}{Proposición}[section]

\newtheorem{proyecto}[proposicion]{Proyecto}
\newtheorem{teorema}[proposicion]{Teorema}
\newtheorem{corolario}[proposicion]{Corolario}
\newtheorem{hecho}[proposicion]{Hecho}
\newtheorem{lema}[proposicion]{Lema}

\newtheorem{observacion}[proposicion]{Observación}

\newenvironment{observacionejerc}
    {\pushQED{\qed}\renewcommand{\qedsymbol}{$\square$}\csname inner@observacionejerc\endcsname}
    {\popQED\csname endinner@observacionejerc\endcsname}
\newtheorem{inner@observacionejerc}[proposicion]{Observación}

\newenvironment{proposicionejerc}
    {\pushQED{\qed}\renewcommand{\qedsymbol}{$\square$}\csname inner@proposicionejerc\endcsname}
    {\popQED\csname endinner@proposicionejerc\endcsname}
\newtheorem{inner@proposicionejerc}[proposicion]{Proposicion}

\newenvironment{lemaejerc}
    {\pushQED{\qed}\renewcommand{\qedsymbol}{$\square$}\csname inner@lemaejerc\endcsname}
    {\popQED\csname endinner@lemaejerc\endcsname}
\newtheorem{inner@lemaejerc}[proposicion]{Lema}

\newtheorem{calculo}[proposicion]{Cálculo}

\theoremstyle{myplainnameless}
\newtheorem{nameless}[proposicion]{}

\theoremstyle{mydefinition}
\newtheorem{comentario}[proposicion]{Comentario}
\newtheorem{comentarioast}[proposicion]{Comentario ($\clubsuit$)}
\newtheorem{construccion}[proposicion]{Construcción}
\newtheorem{aplicacion}[proposicion]{Aplicación}
\newtheorem{definicion}[proposicion]{Definición}
\newtheorem{definicion-alternativa}[proposicion]{Definición alternativa}
\newtheorem{notacion}[proposicion]{Notación}
\newtheorem{advertencia}[proposicion]{Advertencia}
\newtheorem{digresion}[proposicion]{Digresión}
\newtheorem{ejemplox}[proposicion]{Ejemplo}
\newenvironment{ejemplo}
  {\pushQED{\qed}\renewcommand{\qedsymbol}{\examplesymbol}\ejemplox}
  {\popQED\endejemplox}
\newtheorem{contraejemplox}[proposicion]{Contraejemplo}
\newenvironment{contraejemplo}
  {\pushQED{\qed}\renewcommand{\qedsymbol}{\examplesymbol}\contraejemplox}
  {\popQED\endcontraejemplox}
\newtheorem{noejemplox}[proposicion]{No-ejemplo}
\newenvironment{noejemplo}
  {\pushQED{\qed}\renewcommand{\qedsymbol}{\examplesymbol}\noejemplox}
  {\popQED\endnoejemplox}
 
\newtheorem{ejemploastx}[proposicion]{Ejemplo ($\clubsuit$)}
\newenvironment{ejemploast}
  {\pushQED{\qed}\renewcommand{\qedsymbol}{\examplesymbol}\ejemploastx}
  {\popQED\endejemploastx}

\ifdefined\exercisespersection
  \theoremstyle{sectionexercise}
  \newtheorem{ejercicio}{}[section]
  \theoremstyle{mydefinition}
\else
  \ifdefined\exercisesglobal
    \theoremstyle{sectionexercise}
    \newtheorem{ejercicio}{}
    \theoremstyle{mydefinition}
  \else
    \ifdefined\exercisespersection
      \newtheorem{ejercicio}[proposicion]{Ejercicio}
    \fi
  \fi
\fi

% % % % % % % % % % % % % % % % % % % % % % % % % % % % % %

\theoremstyle{myplain}
\newtheorem{proposition}{Proposition}[section]
\newtheorem*{fact*}{Fact}
\newtheorem*{proposition*}{Proposition}
\newtheorem{lemma}[proposition]{Lemma}
\newtheorem*{lemma*}{Lemma}

\newtheorem{exercise}{Exercise}
\newtheorem*{hint}{Hint}

\newtheorem{theorem}[proposition]{Theorem}
\newtheorem{conjecture}[proposition]{Conjecture}
\newtheorem*{theorem*}{Theorem}
\newtheorem{corollary}[proposition]{Corollary}
\newtheorem{fact}[proposition]{Fact}
\newtheorem*{claim}{Claim}
\newtheorem{definition-theorem}[proposition]{Definition-theorem}

\theoremstyle{mydefinition}
\newtheorem{examplex}[proposition]{Example}
\newenvironment{example}
  {\pushQED{\qed}\renewcommand{\qedsymbol}{\examplesymbol}\examplex}
  {\popQED\endexamplex}

\newtheorem*{examplexx}{Example}
\newenvironment{example*}
  {\pushQED{\qed}\renewcommand{\qedsymbol}{\examplesymbol}\examplexx}
  {\popQED\endexamplexx}

\newtheorem{definition}[proposition]{Definition}
\newtheorem*{definition*}{Definition}
\newtheorem{wrong-definition}[proposition]{Wrong definition}
\newtheorem{remark}[proposition]{Remark}

\makeatletter
\newcommand{\xRightarrow}[2][]{\ext@arrow 0359\Rightarrowfill@{#1}{#2}}
\makeatother

% % % % % % % % % % % % % % % % % % % % % % % % % % % % % %

\newcommand{\Et}{\mathop{\text{\rm Ét}}}

\newcommand{\categ}[1]{\text{\bf #1}}
\newcommand{\vcateg}{\mathcal}
\newcommand{\bone}{{\boldsymbol 1}}
\newcommand{\bDelta}{{\boldsymbol\Delta}}
\newcommand{\bR}{{\mathbf{R}}}

\newcommand{\univ}{\mathfrak}

\newcommand{\TODO}{\colorbox{red}{\textbf{*** TODO ***}}}
\newcommand{\proofreadme}{\colorbox{red}{\textbf{*** NEEDS PROOFREADING ***}}}

\makeatletter
\def\iddots{\mathinner{\mkern1mu\raise\p@
\vbox{\kern7\p@\hbox{.}}\mkern2mu
\raise4\p@\hbox{.}\mkern2mu\raise7\p@\hbox{.}\mkern1mu}}
\makeatother

\newcommand{\ssincl}{\reflectbox{\rotatebox[origin=c]{45}{$\subseteq$}}}
\newcommand{\vsupseteq}{\reflectbox{\rotatebox[origin=c]{-90}{$\supseteq$}}}
\newcommand{\vin}{\reflectbox{\rotatebox[origin=c]{90}{$\in$}}}

\newcommand{\Ga}{\mathbb{G}_\mathrm{a}}
\newcommand{\Gm}{\mathbb{G}_\mathrm{m}}

\renewcommand{\U}{\UUU}

\DeclareRobustCommand{\Stirling}{\genfrac\{\}{0pt}{}}
\DeclareRobustCommand{\stirling}{\genfrac[]{0pt}{}}

% % % % % % % % % % % % % % % % % % % % % % % % % % % % % %
% tikz

\usepackage{tikz-cd}
\usetikzlibrary{babel}
\usetikzlibrary{decorations.pathmorphing}
\usetikzlibrary{arrows}
\usetikzlibrary{calc}
\usetikzlibrary{fit}
\usetikzlibrary{hobby}

% % % % % % % % % % % % % % % % % % % % % % % % % % % % % %
% Banners

\newcommand\mybannerext[3]{{\normalfont\sffamily\bfseries\large\noindent #1

\noindent #2

\noindent #3

}\noindent\rule{\textwidth}{1.25pt}

\vspace{1em}}

\newcommand\mybanner[2]{{\normalfont\sffamily\bfseries\large\noindent #1

\noindent #2

}\noindent\rule{\textwidth}{1.25pt}

\vspace{1em}}

\renewcommand{\O}{\mathcal{O}}

\usepackage{diagbox}

\usepackage[numbers]{natbib}

\usepackage{fullpage}

\author{Alexey Beshenov (cadadr@gmail.com)}
\title{Números de Bernoulli y números de Stirling}
\date{2 de Marzo de 2017}

\usepackage{diagbox}
\usepackage{array}
\def\mystrut(#1,#2){\vrule height #1 depth #2 width 0pt}

\newcolumntype{f}[1]{%
>{\mystrut(3ex,2ex)\centering}%
p{#1}%
<{}}

\begin{document}

{\normalfont\sffamily\bfseries \maketitle}

\section*{Digresión combinatoria: los números de Stirling}

Nuestro próximo objetivo es obtener algunas expresiones para los números de Bernoulli que permitan estudiar sus propiedades aritméticas, específicamente sus numeradores y denominadores. En el camino surgen ciertos números combinatorios, conocidos como los \term{números de Stirling}.

\begin{definicion*}
Sean $k$ y $\ell$ dos números naturales positivos.

El \term{número de Stirling de primera clase} $\stirling{k}{\ell}$ es el número de permutaciones en el grupo simétrico $S_k$ que consisten en $\ell$ ciclos disjuntos.

El \term{número de Stirling de segunda clase} $\Stirling{k}{\ell}$ es el número de posibilidades de escribir un conjunto de $k$ elementos como una unión disjunta de $\ell$ conjuntos no vacíos.
\end{definicion*}

\begin{ejemplo*}
$\stirling{4}{2} = 11$. Las permutaciones correspondientes en $S_4$ son

\begin{gather*}
(1)\,(2~3~4), ~ (1)\,(2~4~3), ~ (2)\,(1~3~4), ~ (2)\,(1~4~3),\\
(3)\,(1~2~4), ~ (3)\,(1~4~2), ~ (4)\,(1~2~3), ~ (4)\,(1~3~2),\\
(1~2)\,(3~4), ~ (1~3)\,(2~4), ~ (1~4)\,(2~3).
\end{gather*}
\end{ejemplo*}

\begin{ejemplo*}
$\Stirling{4}{2} = 7$. Las descomposiciones de conjuntos correspondientes son

\begin{align*}
\{ 1,2,3,4 \} & = \{ 1 \} \cup \{ 2,3,4 \} = \{ 2 \} \cup \{ 1,3,4 \} = \{ 3 \} \cup \{ 1,2,4 \} = \{ 4 \} \cup \{ 1,2,3 \} \\
 & = \{ 1,2 \} \cup \{ 3,4 \} = \{ 1,3 \} \cup \{ 2,4 \} = \{ 1,4 \} \cup \{ 2,3 \}.
\end{align*}
\end{ejemplo*}

\pagebreak

De la definición se siguen las identidades

\begin{align}
\label{eqn:stirling-I-0} \stirling{k}{\ell} & = 0 \quad\text{para }\ell > k,\\
\label{eqn:stirling-I-1} \stirling{k}{k} & = 1,\\
\label{eqn:stirling-I-2} \stirling{k}{1} & = (k-1)!,\\
\label{eqn:stirling-I-3} \sum_{1 \le \ell \le k} \stirling{k}{\ell} & = k!,\\
\label{eqn:stirling-I-recur} \stirling{k+1}{\ell} & = \stirling{k}{\ell-1} + k\,\stirling{k}{\ell},
\end{align}

\begin{align}
\label{eqn:stirling-II-0} \Stirling{k}{\ell} & = 0 \quad\text{para }\ell > k,\\
\label{eqn:stirling-II-1} \Stirling{k}{k} & = 1,\\
\label{eqn:stirling-II-2} \Stirling{k}{1} & = 1,\\
\label{eqn:stirling-II-3} \sum_{1 \le \ell \le k} \Stirling{k}{\ell} & = b (k),\\
\label{eqn:stirling-II-recur} \Stirling{k+1}{\ell} & = \Stirling{k}{\ell-1} + \ell\,\Stirling{k}{\ell}.
\end{align}

\eqnref{eqn:stirling-I-1} significa que la única permutación en $S_k$ que se descompone en el producto de $k$ ciclos disjuntos es la permutación identidad. \eqnref{eqn:stirling-I-2} significa que en $S_k$ hay $(k-1)!$ diferentes $k$-ciclos (¿por qué?).

\eqnref{eqn:stirling-I-3} es el hecho de que toda permutación puede descomponerse en un producto de ciclos disjuntos. \eqnref{eqn:stirling-II-3} es el análogo de esta identidad: el número total de particiones se conoce como el \term{número de Bell} $b (k)$. Los primeros números de Bell son $b (1) = 1$, $b (2) = 2$, $b (3) = 5$, $b (4) = 15$, $b (5) = 52$, $b (6) = 203$, $\ldots$; véase \url{http://oeis.org/A000110} En este curso, no vamos estudiar estos números (también porque la notación parece mucho a los números de Bernoulli :-)

Las recurrencias \eqnref{eqn:stirling-I-recur} y \eqnref{eqn:stirling-II-recur} se siguen de la definición combinatoria. Por ejemplo, en \eqnref{eqn:stirling-I-recur}, consideremos las permutaciones de elementos $\{ 1, \ldots, k, k+1 \}$. Sea $\sigma \in S_{k+1}$ una permutación que se descompone en el producto de $\ell$ ciclos disjuntos. Si $\sigma (k+1) = k+1$, entonces $(k+1)$ forma un ciclo por sí mismo, y para el resto de los elementos hay $\stirling{k}{\ell-1}$ posibles descomposiciones. Si $\sigma (k+1) \ne k+1$, entonces $k+1$ pertenece a algún ciclo. Para enumerar todas las posibilidades, podemos primero considerar $\stirling{k}{\ell}$ descomposiciones de las permutaciones de $\{ 1,\ldots,k \}$ en $\ell$ ciclos disjuntos, y luego para cada descomposición hay $k$ posibilidades de poner $k+1$ en uno de los ciclos. La fórmula \eqnref{eqn:stirling-II-recur} se explica de la misma manera: si tenemos un conjunto $X$ de $k+1$ elementos, podemos considerar un elemento $x\in X$. Para las descomposiciones de $X$ en la unión de $\ell$ subconjuntos hay dos casos: o bien $\{ x \}$ forma un conjunto en la descomposición, y quedan $\Stirling{k}{\ell-1}$ posibilidades para descomponer $X\setminus \{ x \}$; o bien $x$ pertenece a algún conjunto. En el segundo caso, hay $\Stirling{k}{\ell}$ posibilidades de descomponer $X\setminus \{ x \}$ en $\ell$ subconjuntos, y luego en cada caso hay $\ell$ posibilidades de poner $x$ en uno de los conjuntos.

\pagebreak

También será útil definir $\stirling{k}{\ell}$ y $\Stirling{k}{\ell}$ para $k,\ell = 0$:

\begin{definicion*}
\label{numeros-de-stirling-para-0}
\begin{gather*}
\stirling{0}{0} = 1, \quad \stirling{k}{0} = \stirling{0}{\ell} = 0 \text{ para }k,\ell\ne 0,\\
\Stirling{0}{0} = 1, \quad \Stirling{k}{0} = \Stirling{0}{\ell} = 0 \text{ para }k,\ell\ne 0.
\end{gather*}
\end{definicion*}

Podemos definir $\stirling{k}{\ell}$ y $\Stirling{k}{\ell}$ por los valores iniciales de arriba y las relaciones de recurrencia \eqnref{eqn:stirling-I-recur} y \eqnref{eqn:stirling-II-recur}. Esta definición es compatible con la primera. Por ejemplo, en el caso de $\stirling{k}{\ell}$, podemos ver que las identidades
$$\stirling{0}{0} = 1, \quad \stirling{k}{0} = \stirling{0}{\ell} = 0 \text{ para }k,\ell\ne 0$$
implican
$$\stirling{k}{1} = (k-1)! \text{ para }k \ge 1, \quad \stirling{1}{\ell} = 0 \text{ para }\ell \ge 1.$$
En efecto,

\begin{align*}
\stirling{k}{1} & = \underbrace{\stirling{k}{0}}_0 + (k-1)\,\stirling{k-1}{1} = (k-1)\,(k-2)\,\stirling{k-2}{1} = \cdots \\
 & = (k-1)\,(k-2)\cdots 2\cdot 1 \, \Bigg(\underbrace{\stirling{0}{0}}_1 + \underbrace{\stirling{0}{1}}_0 \Bigg) = (k-1)!
\end{align*}

\noindent y para $\ell > 1$
$$\stirling{1}{\ell} = \stirling{0}{\ell-1} + 0\cdot \stirling{0}{\ell} = 0.$$

\begin{shaded}
\small\noindent En PARI/GP, \verb|stirling(k,l,2)| $= \Stirling{k}{\ell}$ (el parametro ``$2$'' significa ``de segunda clase''):

\begin{verbatim}
? stirling (4,2,2)
% = 7
\end{verbatim}

\noindent PARI/GP usa otra definición de los números de Stirling de primera clase. La única diferencia es el signo: \verb|stirling(k,l)| $= (-1)^{k-\ell} \stirling{k}{\ell}$:

\begin{verbatim}
? stirling (4,2)  
% = 11
? stirling(4,3)  
% = -6
\end{verbatim}
\end{shaded}

\begin{ejercicio*}
Demuestre que $\stirling{k}{k-1} = {k \choose 2}$.
\end{ejercicio*}

\begin{ejercicio*}
Note que las recurrencias de arriba con los valores iniciales para $k, \ell = 0$ nos permiten definir $\stirling{k}{\ell}$ y $\Stirling{k}{\ell}$ para todo $k,\ell \in \ZZ$. Demuestre que
$$\stirling{k}{\ell} = \Stirling{-\ell}{-k}.$$
Esto significa que los números de Stirling de primera y de segunda clase son esencialmente el mismo objeto.
\end{ejercicio*}

\begin{ejercicio*}
Demuestre que $\Stirling{k}{\ell} = 0$ para $k\ell < 0$.
\end{ejercicio*}

\noindent (Los últimos dos ejercicios sirven solo para acostumbrarse a las recurrencias con $\stirling{k}{\ell}$ y $\Stirling{k}{\ell}$; no vamos a usar los números de Stirling para $k$ y $\ell$ negativos.)

\vspace{1em}

\begin{center}
{\small
\noindent\begin{tabular}{|f{0.9cm}|f{0.9cm}|f{0.9cm}|f{0.9cm}|f{0.9cm}|f{0.9cm}|f{0.9cm}|f{0.9cm}|f{0.9cm}|f{0.9cm}|f{0.9cm}|}
\hline
\backslashbox{$k$}{$\ell$} & $0$ & $1$ & $2$ & $3$ & $4$ & $5$ & $6$ & $7$ & $8$ & $9$ \tabularnewline\hline
$0$ & $1$ &  &  &  &  &  &  &  &  & \tabularnewline\hline
$1$ & $0$ & $1$ &  &  &  &  &  &  &  & \tabularnewline\hline
$2$ & $0$ & $1$ & $1$ &  &  &  &  &  &  & \tabularnewline\hline
$3$ & $0$ & $2$ & $3$ & $1$ &  &  &  &  &  & \tabularnewline\hline
$4$ & $0$ & $6$ & $11$ & $6$ & $1$ &  &  &  &  & \tabularnewline\hline
$5$ & $0$ & $24$ & $50$ & $35$ & $10$ & $1$ &  &  &  & \tabularnewline\hline
$6$ & $0$ & $120$ & $274$ & $225$ & $85$ & $15$ & $1$ &  &  & \tabularnewline\hline
$7$ & $0$ & $720$ & $1764$ & $1624$ & $735$ & $175$ & $21$ & $1$ &  & \tabularnewline\hline
$8$ & $0$ & $5040$ & $13068$ & $13132$ & $6769$ & $1960$ & $322$ & $28$ & $1$ & \tabularnewline\hline
$9$ & $0$ & $40320$ & $109584$ & $118124$ & $67284$ & $22449$ & $4536$ & $546$ & $36$ & $1$ \tabularnewline\hline
\end{tabular}}

\vspace{1em}

\noindent Valores de $\stirling{k}{\ell}$

\vspace{\fill}

{\small
\noindent\begin{tabular}{|f{0.9cm}|f{0.9cm}|f{0.9cm}|f{0.9cm}|f{0.9cm}|f{0.9cm}|f{0.9cm}|f{0.9cm}|f{0.9cm}|f{0.9cm}|f{0.9cm}|}
\hline
\backslashbox{$k$}{$\ell$} & $0$ & $1$ & $2$ & $3$ & $4$ & $5$ & $6$ & $7$ & $8$ & $9$ \tabularnewline\hline
$0$ & $1$ &  &  &  &  &  &  &  &  & \tabularnewline\hline
$1$ & $0$ & $1$ &  &  &  &  &  &  &  & \tabularnewline\hline
$2$ & $0$ & $1$ & $1$ &  &  &  &  &  &  & \tabularnewline\hline
$3$ & $0$ & $1$ & $3$ & $1$ &  &  &  &  &  & \tabularnewline\hline
$4$ & $0$ & $1$ & $7$ & $6$ & $1$ &  &  &  &  & \tabularnewline\hline
$5$ & $0$ & $1$ & $15$ & $25$ & $10$ & $1$ &  &  &  & \tabularnewline\hline
$6$ & $0$ & $1$ & $31$ & $90$ & $65$ & $15$ & $1$ &  &  & \tabularnewline\hline
$7$ & $0$ & $1$ & $63$ & $301$ & $350$ & $140$ & $21$ & $1$ &  & \tabularnewline\hline
$8$ & $0$ & $1$ & $127$ & $966$ & $1701$ & $1050$ & $266$ & $28$ & $1$ & \tabularnewline\hline
$9$ & $0$ & $1$ & $255$ & $3025$ & $7770$ & $6951$ & $2646$ & $462$ & $36$ & $1$ \tabularnewline\hline
\end{tabular}}

\vspace{1em}

\noindent Valores de $\Stirling{k}{\ell}$
\end{center}

\pagebreak

% % % % % % % % % % % % % % % % % % % % % % % % % % % % % %

\section*{Relación entre $B_k$ y los números de Stirling}

\begin{lema*}
\label{exp-stirling}
Para todo $\ell \ge 0$
$$\frac{(e^t - 1)^\ell}{\ell!} = \sum_{k \ge \ell} \Stirling{k}{\ell}\,\frac{t^k}{k!}.$$

\begin{proof}
Tenemos que verificar que
$$\frac{d^k}{dt^k} \left(\frac{(e^t - 1)^\ell}{\ell!}\right) (0) = \Stirling{k}{\ell}.$$
Los valores iniciales coinciden, y va a ser suficiente demostrar que la recurrencia
$$\Stirling{k+1}{\ell} = \Stirling{k}{\ell-1} + \ell \, \Stirling{k}{\ell}$$
se cumple en nuestro caso:
$$\frac{d^{k+1}}{dt^{k+1}} \left(\frac{(e^t - 1)^\ell}{\ell!}\right) (0) = \frac{d^k}{dt^k} \left(\frac{(e^t - 1)^{\ell-1}}{(\ell-1)!}\right) (0) + \ell\,\frac{d^k}{dt^k} \left(\frac{(e^t - 1)^\ell}{\ell!}\right) (0).$$
En efecto,

\begin{align*}
\frac{d^{k+1}}{dt^{k+1}} \left(\frac{(e^t - 1)^\ell}{\ell!}\right) & = \frac{d^k}{dt^k} \left(\frac{(e^t - 1)^{\ell-1}}{(\ell-1)!}\,e^t\right) = \frac{d^k}{dt^k} \left(\frac{(e^t - 1)^{\ell-1} \, (1 + e^t - 1)}{(\ell - 1)!}\right) \\
 & = \frac{d^k}{dt^k} \left(\frac{(e^t - 1)^{\ell-1}}{(\ell-1)!} + \frac{(e^t-1)^\ell}{(\ell-1)!}\right) \\
 & = \frac{d^k}{dt^k} \left(\frac{(e^t - 1)^{\ell-1}}{(\ell-1)!}\right) + \ell\,\frac{d^k}{dt^k} \left(\frac{(e^t-1)^\ell}{\ell!}\right).
\end{align*}
\end{proof}
\end{lema*}

\begin{ejercicio*}
\label{log-stirling}
Demuestre la identidad
$$\frac{(-\ln (1-t))^\ell}{\ell!} = \sum_{k \ge \ell} \stirling{k}{\ell}\,\frac{t^k}{k!}.$$
(De nuevo, es suficiente considerar las derivadas formales y verificar que se cumple la misma recurrencia que define los números de Stirling correspondientes: $\stirling{k+1}{\ell} = \stirling{k}{\ell-1} + k\,\stirling{k}{\ell}$.)
\end{ejercicio*}

Lo que acabamos de ver son las funciones generatrices para los números de Stirling, pero no soy tan sádico para dar esto como la definición de $\Stirling{k}{\ell}$ y $\stirling{k}{\ell}$.

\begin{lema*}
\label{stirling-via-binomial}
Para $k,\ell\ge 0$
$$\Stirling{k}{\ell} = \frac{(-1)^\ell}{\ell!}\,\sum_{0 \le i \le \ell} (-1)^i \, {\ell \choose i} \, i^k.$$

\begin{proof}
De nuevo, podemos verificar que los valores iniciales coinciden y la suma satisface la misma recurrencia que $\Stirling{k}{\ell}$:
$$\Stirling{k+1}{\ell} = \Stirling{k}{\ell-1} + \ell \, \Stirling{k}{\ell}.$$
Para los valores iniciales, si $k = \ell = 0$, la suma nos da $\Stirling{0}{0} = 1$ (como siempre en el contexto algebraico/combinatorio, $0^0 = 1$); si $k > 0$, $\ell = 0$, la suma nos da $0$; si $k = 0$, $\ell > 0$, la suma también nos da $\sum_{0 \le i \le \ell} (-1)^i \, {\ell\choose i} = 0$. Para la recurrencia,

\begin{align*}
\frac{(-1)^\ell}{\ell!}\,\sum_{0 \le i \le \ell} (-1)^i \, {\ell \choose i} \, i^{k+1} & = \frac{(-1)^\ell}{(\ell-1)!}\,\sum_{0 \le i \le \ell} (-1)^i \, \frac{i}{\ell}\, {\ell \choose i} \, i^k \\
 & = \frac{(-1)^\ell}{(\ell-1)!}\,\sum_{0 \le i \le \ell} (-1)^i \, \left({\ell \choose i} - {\ell - 1\choose i}\right) \, i^k \\
 & = \frac{(-1)^{\ell-1}}{(\ell-1)!}\,\sum_{0 \le i \le \ell-1} (-1)^i \, {\ell - 1 \choose i} \, i^k + \ell\,\frac{(-1)^\ell}{\ell!}\,\sum_{0 \le i \le \ell} (-1)^i \, {\ell \choose i} \, i^k.
\end{align*}

Aquí hemos usado la identidad
$${\ell \choose i} - {\ell - 1 \choose i} = \frac{i}{\ell}\,{\ell \choose i}.$$
\end{proof}
\end{lema*}

\begin{teorema*}
$$B_k = (-1)^k \, \sum_{0 \le \ell \le k} \frac{(-1)^\ell \, \ell! \, \Stirling{k}{\ell}}{\ell+1} = (-1)^k \, \sum_{0 \le \ell \le k} \frac{1}{\ell+1}\,\sum_{0 \le i \le \ell} (-1)^i \, {\ell \choose i} \, i^k.$$
\end{teorema*}

La segunda igualdad viene de la expresión de los números de Stirling en términos de los coeficientes binomiales y (¡por fin!) nos da una expresión para $B_k$ sin recurrencias.

\begin{proof}
La función generatriz para $B_k$ es $\frac{t\,e^t}{e^t - 1}$. Ya que la exponencial y el logaritmo formales son inversos, podemos escribir
$$\frac{t\,e^t}{e^t - 1} = \frac{t}{1 - e^{-t}} = \frac{-\ln (1 - (1-e^{-t}))}{1 - e^{-t}}.$$
Luego,

\begin{align*}
\frac{-\ln (1 - (1-e^{-t}))}{1 - e^{-t}} & = \frac{1}{1 - e^{-t}}\,\sum_{\ell \ge 1} \frac{(1 - e^{-t})^\ell}{\ell} \\
 & = \sum_{\ell \ge 1} \frac{(1 - e^{-t})^{\ell-1}}{\ell} \\
 & = \sum_{\ell \ge 0} \frac{(-1)^\ell\,\ell!}{\ell+1} \sum_{k \ge \ell} \Stirling{k}{\ell}\,\frac{(-t)^k}{k!} \quad\text{por \ref{exp-stirling}} \\
 & = \sum_{k \ge 0} (-1)^k \left(\sum_{0\le \ell \le k} \frac{(-1)^\ell\,\ell! \, \Stirling{k}{\ell}}{\ell+1}\right)\,\frac{t^k}{k!}.
\end{align*}
\end{proof}

\begin{shaded}
\small
\begin{verbatim}
? bernbin (k) = (-1)^k * sum (l=0,k, 1/(l+1)*sum(i=0,l, (-1)^i*binomial(l,i)*i^k));
? vector (10,k,bernbin(k))                                                         
% = [1/2, 1/6, 0, -1/30, 0, 1/42, 0, -1/30, 0, 5/66]
\end{verbatim}
\end{shaded}

\end{document}
