\documentclass{article}

% TODO : CLEAN UP THIS MESS
% (AND MAKE SURE ALL TEXTS STILL COMPILE)
\usepackage[leqno]{amsmath}
\usepackage{amssymb}
\usepackage{graphicx}

\usepackage{diagbox} % table heads with diagonal lines
\usepackage{relsize}

\usepackage{wasysym}
\usepackage{scrextend}
\usepackage{epigraph}
\setlength\epigraphwidth{.6\textwidth}

\usepackage[utf8]{inputenc}

\usepackage{titlesec}
\titleformat{\chapter}[display]
  {\normalfont\sffamily\huge\bfseries}
  {\chaptertitlename\ \thechapter}{5pt}{\Huge}
\titleformat{\section}
  {\normalfont\sffamily\Large\bfseries}
  {\thesection}{1em}{}
\titleformat{\subsection}
  {\normalfont\sffamily\large\bfseries}
  {\thesubsection}{1em}{}
\titleformat{\part}[display]
  {\normalfont\sffamily\huge\bfseries}
  {\partname\ \thepart}{0pt}{\Huge}

\usepackage[T1]{fontenc}
\usepackage{fourier}
\usepackage{paratype}

\usepackage[symbol,perpage]{footmisc}

\usepackage{perpage}
\MakePerPage{footnote}

\usepackage{array}
\newcolumntype{x}[1]{>{\centering\hspace{0pt}}p{#1}}

% TODO: the following line causes conflict with new texlive (!)
% \usepackage[english,russian,polutonikogreek,spanish]{babel}
% \newcommand{\russian}[1]{{\selectlanguage{russian}#1}}

% Remove conflicting options for the moment:
\usepackage[english,polutonikogreek,spanish]{babel}

\AtBeginDocument{\shorthandoff{"}}
\newcommand{\greek}[1]{{\selectlanguage{polutonikogreek}#1}}

% % % % % % % % % % % % % % % % % % % % % % % % % % % % % %
% Limit/colimit symbols (with accented i: lím / colím)

\usepackage{etoolbox} % \patchcmd

\makeatletter
\patchcmd{\varlim@}{lim}{\lim}{}{}
\makeatother
\DeclareMathOperator*{\colim}{co{\lim}}
\newcommand{\dirlim}{\varinjlim}
\newcommand{\invlim}{\varprojlim}

% % % % % % % % % % % % % % % % % % % % % % % % % % % % % %

\usepackage[all,color]{xy}

\usepackage{pigpen}
\newcommand{\po}{\ar@{}[dr]|(.4){\text{\pigpenfont I}}}
\newcommand{\pb}{\ar@{}[dr]|(.3){\text{\pigpenfont A}}}
\newcommand{\polr}{\ar@{}[dr]|(.65){\text{\pigpenfont A}}}
\newcommand{\pour}{\ar@{}[ur]|(.65){\text{\pigpenfont G}}}
\newcommand{\hstar}{\mathop{\bigstar}}

\newcommand{\bigast}{\mathop{\Huge \mathlarger{\mathlarger{\ast}}}}

\newcommand{\term}{\textbf}

\usepackage{stmaryrd}

\usepackage{cancel}

\usepackage{tikzsymbols}

\newcommand{\open}{\underset{\mathrm{open}}{\hookrightarrow}}
\newcommand{\closed}{\underset{\mathrm{closed}}{\hookrightarrow}}

\newcommand{\tcol}[2]{{#1 \choose #2}}

\newcommand{\homot}{\simeq}
\newcommand{\isom}{\cong}
\newcommand{\cH}{\mathcal{H}}
\renewcommand{\hom}{\mathrm{hom}}
\renewcommand{\div}{\mathop{\mathrm{div}}}
\renewcommand{\Im}{\mathop{\mathrm{Im}}}
\renewcommand{\Re}{\mathop{\mathrm{Re}}}
\newcommand{\id}[1]{\mathrm{id}_{#1}}
\newcommand{\idid}{\mathrm{id}}

\newcommand{\ZG}{{\ZZ G}}
\newcommand{\ZH}{{\ZZ H}}

\newcommand{\quiso}{\simeq}

\newcommand{\personality}[1]{{\sc #1}}

\newcommand{\mono}{\rightarrowtail}
\newcommand{\epi}{\twoheadrightarrow}
\newcommand{\xepi}[1]{\xrightarrow{#1}\mathrel{\mkern-14mu}\rightarrow}

% % % % % % % % % % % % % % % % % % % % % % % % % % % % % %

\DeclareMathOperator{\Ad}{Ad}
\DeclareMathOperator{\Aff}{Aff}
\DeclareMathOperator{\Ann}{Ann}
\DeclareMathOperator{\Aut}{Aut}
\DeclareMathOperator{\Br}{Br}
\DeclareMathOperator{\CH}{CH}
\DeclareMathOperator{\Cl}{Cl}
\DeclareMathOperator{\Coeq}{Coeq}
\DeclareMathOperator{\Coind}{Coind}
\DeclareMathOperator{\Cop}{Cop}
\DeclareMathOperator{\Corr}{Corr}
\DeclareMathOperator{\Cor}{Cor}
\DeclareMathOperator{\Cov}{Cov}
\DeclareMathOperator{\Der}{Der}
\DeclareMathOperator{\Div}{Div}
\DeclareMathOperator{\D}{D}
\DeclareMathOperator{\Ehr}{Ehr}
\DeclareMathOperator{\End}{End}
\DeclareMathOperator{\Eq}{Eq}
\DeclareMathOperator{\Ext}{Ext}
\DeclareMathOperator{\Frac}{Frac}
\DeclareMathOperator{\Frob}{Frob}
\DeclareMathOperator{\Funct}{Funct}
\DeclareMathOperator{\Fun}{Fun}
\DeclareMathOperator{\GL}{GL}
\DeclareMathOperator{\Gal}{Gal}
\DeclareMathOperator{\Gr}{Gr}
\DeclareMathOperator{\Hol}{Hol}
\DeclareMathOperator{\Hom}{Hom}
\DeclareMathOperator{\Ho}{Ho}
\DeclareMathOperator{\Id}{Id}
\DeclareMathOperator{\Ind}{Ind}
\DeclareMathOperator{\Inn}{Inn}
\DeclareMathOperator{\Isom}{Isom}
\DeclareMathOperator{\Ker}{Ker}
\DeclareMathOperator{\Lan}{Lan}
\DeclareMathOperator{\Lie}{Lie}
\DeclareMathOperator{\Map}{Map}
\DeclareMathOperator{\Mat}{Mat}
\DeclareMathOperator{\Max}{Max}
\DeclareMathOperator{\Mor}{Mor}
\DeclareMathOperator{\Nat}{Nat}
\DeclareMathOperator{\Nrd}{Nrd}
\DeclareMathOperator{\Ob}{Ob}
\DeclareMathOperator{\Out}{Out}
\DeclareMathOperator{\PGL}{PGL}
\DeclareMathOperator{\PSL}{PSL}
\DeclareMathOperator{\PSU}{PSU}
\DeclareMathOperator{\Pic}{Pic}
\DeclareMathOperator{\RHom}{RHom}
\DeclareMathOperator{\Rad}{Rad}
\DeclareMathOperator{\Ran}{Ran}
\DeclareMathOperator{\Rep}{Rep}
\DeclareMathOperator{\Res}{Res}
\DeclareMathOperator{\SL}{SL}
\DeclareMathOperator{\SO}{SO}
\DeclareMathOperator{\SU}{SU}
\DeclareMathOperator{\Sh}{Sh}
\DeclareMathOperator{\Sing}{Sing}
\DeclareMathOperator{\Specm}{Specm}
\DeclareMathOperator{\Spec}{Spec}
\DeclareMathOperator{\Sp}{Sp}
\DeclareMathOperator{\Stab}{Stab}
\DeclareMathOperator{\Sym}{Sym}
\DeclareMathOperator{\Tors}{Tors}
\DeclareMathOperator{\Tor}{Tor}
\DeclareMathOperator{\Tot}{Tot}
\DeclareMathOperator{\UUU}{U}

\DeclareMathOperator{\adj}{adj}
\DeclareMathOperator{\ad}{ad}
\DeclareMathOperator{\af}{af}
\DeclareMathOperator{\card}{card}
\DeclareMathOperator{\cm}{cm}
\DeclareMathOperator{\codim}{codim}
\DeclareMathOperator{\cod}{cod}
\DeclareMathOperator{\coeq}{coeq}
\DeclareMathOperator{\coim}{coim}
\DeclareMathOperator{\coker}{coker}
\DeclareMathOperator{\cont}{cont}
\DeclareMathOperator{\conv}{conv}
\DeclareMathOperator{\cor}{cor}
\DeclareMathOperator{\depth}{depth}
\DeclareMathOperator{\diag}{diag}
\DeclareMathOperator{\diam}{diam}
\DeclareMathOperator{\dist}{dist}
\DeclareMathOperator{\dom}{dom}
\DeclareMathOperator{\eq}{eq}
\DeclareMathOperator{\ev}{ev}
\DeclareMathOperator{\ex}{ex}
\DeclareMathOperator{\fchar}{char}
\DeclareMathOperator{\fr}{fr}
\DeclareMathOperator{\gr}{gr}
\DeclareMathOperator{\im}{im}
\DeclareMathOperator{\infl}{inf}
\DeclareMathOperator{\interior}{int}
\DeclareMathOperator{\intrel}{intrel}
\DeclareMathOperator{\inv}{inv}
\DeclareMathOperator{\length}{length}
\DeclareMathOperator{\mcd}{mcd}
\DeclareMathOperator{\mcm}{mcm}
\DeclareMathOperator{\multideg}{multideg}
\DeclareMathOperator{\ord}{ord}
\DeclareMathOperator{\pr}{pr}
\DeclareMathOperator{\rel}{rel}
\DeclareMathOperator{\res}{res}
\DeclareMathOperator{\rkred}{rkred}
\DeclareMathOperator{\rkss}{rkss}
\DeclareMathOperator{\rk}{rk}
\DeclareMathOperator{\sgn}{sgn}
\DeclareMathOperator{\sk}{sk}
\DeclareMathOperator{\supp}{supp}
\DeclareMathOperator{\trdeg}{trdeg}
\DeclareMathOperator{\tr}{tr}
\DeclareMathOperator{\vol}{vol}

\newcommand{\iHom}{\underline{\Hom}}

\renewcommand{\AA}{\mathbb{A}}
\newcommand{\CC}{\mathbb{C}}
\renewcommand{\SS}{\mathbb{S}}
\newcommand{\TT}{\mathbb{T}}
\newcommand{\PP}{\mathbb{P}}
\newcommand{\BB}{\mathbb{B}}
\newcommand{\RR}{\mathbb{R}}
\newcommand{\ZZ}{\mathbb{Z}}
\newcommand{\FF}{\mathbb{F}}
\newcommand{\HH}{\mathbb{H}}
\newcommand{\NN}{\mathbb{N}}
\newcommand{\QQ}{\mathbb{Q}}
\newcommand{\KK}{\mathbb{K}}

% % % % % % % % % % % % % % % % % % % % % % % % % % % % % %

\usepackage{amsthm}

\newcommand{\legendre}[2]{\left(\frac{#1}{#2}\right)}

\newcommand{\examplesymbol}{$\blacktriangle$}
\renewcommand{\qedsymbol}{$\blacksquare$}

\newcommand{\dfn}{\mathrel{\mathop:}=}
\newcommand{\rdfn}{=\mathrel{\mathop:}}

\usepackage{xcolor}
\definecolor{mylinkcolor}{rgb}{0.0,0.4,1.0}
\definecolor{mycitecolor}{rgb}{0.0,0.4,1.0}
\definecolor{shadecolor}{rgb}{0.79,0.78,0.65}
\definecolor{gray}{rgb}{0.6,0.6,0.6}

\usepackage{colortbl}

\definecolor{myred}{rgb}{0.7,0.0,0.0}
\definecolor{mygreen}{rgb}{0.0,0.7,0.0}
\definecolor{myblue}{rgb}{0.0,0.0,0.7}

\definecolor{redshade}{rgb}{0.9,0.5,0.5}
\definecolor{greenshade}{rgb}{0.5,0.9,0.5}

\usepackage[unicode,colorlinks=true,linkcolor=mylinkcolor,citecolor=mycitecolor]{hyperref}
\newcommand{\refref}[2]{\hyperref[#2]{#1~\ref*{#2}}}
\newcommand{\eqnref}[1]{\hyperref[#1]{(\ref*{#1})}}

\newcommand{\tos}{\!\!\to\!\!}

\usepackage{framed}

\newcommand{\cequiv}{\simeq}

\makeatletter
\newcommand\xleftrightarrow[2][]{%
  \ext@arrow 9999{\longleftrightarrowfill@}{#1}{#2}}
\newcommand\longleftrightarrowfill@{%
  \arrowfill@\leftarrow\relbar\rightarrow}
\makeatother

\newcommand{\bsquare}{\textrm{\ding{114}}}

% % % % % % % % % % % % % % % % % % % % % % % % % % % % % %

\newtheoremstyle{myplain}
  {\topsep}   % ABOVESPACE
  {\topsep}   % BELOWSPACE
  {\itshape}  % BODYFONT
  {0pt}       % INDENT (empty value is the same as 0pt)
  {\bfseries} % HEADFONT
  {.}         % HEADPUNCT
  {5pt plus 1pt minus 1pt} % HEADSPACE
  {\thmnumber{#2}. \thmname{#1}\thmnote{ (#3)}}   % CUSTOM-HEAD-SPEC

\newtheoremstyle{myplainnameless}
  {\topsep}   % ABOVESPACE
  {\topsep}   % BELOWSPACE
  {\normalfont}  % BODYFONT
  {0pt}       % INDENT (empty value is the same as 0pt)
  {\bfseries} % HEADFONT
  {.}         % HEADPUNCT
  {5pt plus 1pt minus 1pt} % HEADSPACE
  {\thmnumber{#2}}   % CUSTOM-HEAD-SPEC 

\newtheoremstyle{sectionexercise}
  {\topsep}   % ABOVESPACE
  {\topsep}   % BELOWSPACE
  {\normalfont}  % BODYFONT
  {0pt}       % INDENT (empty value is the same as 0pt)
  {\bfseries} % HEADFONT
  {.}         % HEADPUNCT
  {5pt plus 1pt minus 1pt} % HEADSPACE
  {Ejercicio \thmnumber{#2}\thmnote{ (#3)}}   % CUSTOM-HEAD-SPEC

\newtheoremstyle{mydefinition}
  {\topsep}   % ABOVESPACE
  {\topsep}   % BELOWSPACE
  {\normalfont}  % BODYFONT
  {0pt}       % INDENT (empty value is the same as 0pt)
  {\bfseries} % HEADFONT
  {.}         % HEADPUNCT
  {5pt plus 1pt minus 1pt} % HEADSPACE
  {\thmnumber{#2}. \thmname{#1}\thmnote{ (#3)}}   % CUSTOM-HEAD-SPEC

% EN ESPAÑOL

\newtheorem*{hecho*}{Hecho}
\newtheorem*{corolario*}{Corolario}
\newtheorem*{teorema*}{Teorema}
\newtheorem*{conjetura*}{Conjetura}
\newtheorem*{proyecto*}{Proyecto}
\newtheorem*{observacion*}{Observación}

\newtheorem*{lema*}{Lema}
\newtheorem*{resultado-clave*}{Resultado clave}
\newtheorem*{proposicion*}{Proposición}

\theoremstyle{definition}
\newtheorem*{ejercicio*}{Ejercicio}
\newtheorem*{definicion*}{Definición}
\newtheorem*{comentario*}{Comentario}
\newtheorem*{definicion-alternativa*}{Definición alternativa}
\newtheorem*{ejemploxs}{Ejemplo}
\newenvironment{ejemplo*}
  {\pushQED{\qed}\renewcommand{\qedsymbol}{\examplesymbol}\ejemploxs}
  {\popQED\endejemploxs}

\theoremstyle{myplain}
\newtheorem{proposicion}{Proposición}[section]

\newtheorem{proyecto}[proposicion]{Proyecto}
\newtheorem{teorema}[proposicion]{Teorema}
\newtheorem{corolario}[proposicion]{Corolario}
\newtheorem{hecho}[proposicion]{Hecho}
\newtheorem{lema}[proposicion]{Lema}

\newtheorem{observacion}[proposicion]{Observación}

\newenvironment{observacionejerc}
    {\pushQED{\qed}\renewcommand{\qedsymbol}{$\square$}\csname inner@observacionejerc\endcsname}
    {\popQED\csname endinner@observacionejerc\endcsname}
\newtheorem{inner@observacionejerc}[proposicion]{Observación}

\newenvironment{proposicionejerc}
    {\pushQED{\qed}\renewcommand{\qedsymbol}{$\square$}\csname inner@proposicionejerc\endcsname}
    {\popQED\csname endinner@proposicionejerc\endcsname}
\newtheorem{inner@proposicionejerc}[proposicion]{Proposicion}

\newenvironment{lemaejerc}
    {\pushQED{\qed}\renewcommand{\qedsymbol}{$\square$}\csname inner@lemaejerc\endcsname}
    {\popQED\csname endinner@lemaejerc\endcsname}
\newtheorem{inner@lemaejerc}[proposicion]{Lema}

\newtheorem{calculo}[proposicion]{Cálculo}

\theoremstyle{myplainnameless}
\newtheorem{nameless}[proposicion]{}

\theoremstyle{mydefinition}
\newtheorem{comentario}[proposicion]{Comentario}
\newtheorem{comentarioast}[proposicion]{Comentario ($\clubsuit$)}
\newtheorem{construccion}[proposicion]{Construcción}
\newtheorem{aplicacion}[proposicion]{Aplicación}
\newtheorem{definicion}[proposicion]{Definición}
\newtheorem{definicion-alternativa}[proposicion]{Definición alternativa}
\newtheorem{notacion}[proposicion]{Notación}
\newtheorem{advertencia}[proposicion]{Advertencia}
\newtheorem{digresion}[proposicion]{Digresión}
\newtheorem{ejemplox}[proposicion]{Ejemplo}
\newenvironment{ejemplo}
  {\pushQED{\qed}\renewcommand{\qedsymbol}{\examplesymbol}\ejemplox}
  {\popQED\endejemplox}
\newtheorem{contraejemplox}[proposicion]{Contraejemplo}
\newenvironment{contraejemplo}
  {\pushQED{\qed}\renewcommand{\qedsymbol}{\examplesymbol}\contraejemplox}
  {\popQED\endcontraejemplox}
\newtheorem{noejemplox}[proposicion]{No-ejemplo}
\newenvironment{noejemplo}
  {\pushQED{\qed}\renewcommand{\qedsymbol}{\examplesymbol}\noejemplox}
  {\popQED\endnoejemplox}
 
\newtheorem{ejemploastx}[proposicion]{Ejemplo ($\clubsuit$)}
\newenvironment{ejemploast}
  {\pushQED{\qed}\renewcommand{\qedsymbol}{\examplesymbol}\ejemploastx}
  {\popQED\endejemploastx}

\ifdefined\exercisespersection
  \theoremstyle{sectionexercise}
  \newtheorem{ejercicio}{}[section]
  \theoremstyle{mydefinition}
\else
  \ifdefined\exercisesglobal
    \theoremstyle{sectionexercise}
    \newtheorem{ejercicio}{}
    \theoremstyle{mydefinition}
  \else
    \ifdefined\exercisespersection
      \newtheorem{ejercicio}[proposicion]{Ejercicio}
    \fi
  \fi
\fi

% % % % % % % % % % % % % % % % % % % % % % % % % % % % % %

\theoremstyle{myplain}
\newtheorem{proposition}{Proposition}[section]
\newtheorem*{fact*}{Fact}
\newtheorem*{proposition*}{Proposition}
\newtheorem{lemma}[proposition]{Lemma}
\newtheorem*{lemma*}{Lemma}

\newtheorem{exercise}{Exercise}
\newtheorem*{hint}{Hint}

\newtheorem{theorem}[proposition]{Theorem}
\newtheorem{conjecture}[proposition]{Conjecture}
\newtheorem*{theorem*}{Theorem}
\newtheorem{corollary}[proposition]{Corollary}
\newtheorem{fact}[proposition]{Fact}
\newtheorem*{claim}{Claim}
\newtheorem{definition-theorem}[proposition]{Definition-theorem}

\theoremstyle{mydefinition}
\newtheorem{examplex}[proposition]{Example}
\newenvironment{example}
  {\pushQED{\qed}\renewcommand{\qedsymbol}{\examplesymbol}\examplex}
  {\popQED\endexamplex}

\newtheorem*{examplexx}{Example}
\newenvironment{example*}
  {\pushQED{\qed}\renewcommand{\qedsymbol}{\examplesymbol}\examplexx}
  {\popQED\endexamplexx}

\newtheorem{definition}[proposition]{Definition}
\newtheorem*{definition*}{Definition}
\newtheorem{wrong-definition}[proposition]{Wrong definition}
\newtheorem{remark}[proposition]{Remark}

\makeatletter
\newcommand{\xRightarrow}[2][]{\ext@arrow 0359\Rightarrowfill@{#1}{#2}}
\makeatother

% % % % % % % % % % % % % % % % % % % % % % % % % % % % % %

\newcommand{\Et}{\mathop{\text{\rm Ét}}}

\newcommand{\categ}[1]{\text{\bf #1}}
\newcommand{\vcateg}{\mathcal}
\newcommand{\bone}{{\boldsymbol 1}}
\newcommand{\bDelta}{{\boldsymbol\Delta}}
\newcommand{\bR}{{\mathbf{R}}}

\newcommand{\univ}{\mathfrak}

\newcommand{\TODO}{\colorbox{red}{\textbf{*** TODO ***}}}
\newcommand{\proofreadme}{\colorbox{red}{\textbf{*** NEEDS PROOFREADING ***}}}

\makeatletter
\def\iddots{\mathinner{\mkern1mu\raise\p@
\vbox{\kern7\p@\hbox{.}}\mkern2mu
\raise4\p@\hbox{.}\mkern2mu\raise7\p@\hbox{.}\mkern1mu}}
\makeatother

\newcommand{\ssincl}{\reflectbox{\rotatebox[origin=c]{45}{$\subseteq$}}}
\newcommand{\vsupseteq}{\reflectbox{\rotatebox[origin=c]{-90}{$\supseteq$}}}
\newcommand{\vin}{\reflectbox{\rotatebox[origin=c]{90}{$\in$}}}

\newcommand{\Ga}{\mathbb{G}_\mathrm{a}}
\newcommand{\Gm}{\mathbb{G}_\mathrm{m}}

\renewcommand{\U}{\UUU}

\DeclareRobustCommand{\Stirling}{\genfrac\{\}{0pt}{}}
\DeclareRobustCommand{\stirling}{\genfrac[]{0pt}{}}

% % % % % % % % % % % % % % % % % % % % % % % % % % % % % %
% tikz

\usepackage{tikz-cd}
\usetikzlibrary{babel}
\usetikzlibrary{decorations.pathmorphing}
\usetikzlibrary{arrows}
\usetikzlibrary{calc}
\usetikzlibrary{fit}
\usetikzlibrary{hobby}

% % % % % % % % % % % % % % % % % % % % % % % % % % % % % %
% Banners

\newcommand\mybannerext[3]{{\normalfont\sffamily\bfseries\large\noindent #1

\noindent #2

\noindent #3

}\noindent\rule{\textwidth}{1.25pt}

\vspace{1em}}

\newcommand\mybanner[2]{{\normalfont\sffamily\bfseries\large\noindent #1

\noindent #2

}\noindent\rule{\textwidth}{1.25pt}

\vspace{1em}}

\renewcommand{\O}{\mathcal{O}}

\usepackage{diagbox}

\usepackage[numbers]{natbib}

\usepackage{fullpage}

\author{Alexey Beshenov (cadadr@gmail.com)}
\title{Valores especiales de la función zeta}
\date{1 de Marzo de 2017}

\usepackage{xcolor}
\newcommand{\highlight}[1]{\colorbox{shadecolor}{$\displaystyle #1$}}

\begin{document}

{\normalfont\sffamily\bfseries \maketitle}

\section*{La función zeta de Riemann}

\begin{definicion*}
La \term{función zeta de Riemann} está definida por la serie infinita
$$\zeta (s) \dfn \sum_{n \ge 1} \frac{1}{n^s} = 1 + \frac{1}{2^s} + \frac{1}{3^s} + \frac{1}{4^s} + \frac{1}{5^s} + \frac{1}{6^s} + \cdots$$
\end{definicion*}

\begin{observacion*}
La serie de arriba es absolutamente convergente para todo $s \in \CC$ tal que $\Re s > 1$.

\begin{proof}
Si $s = a + i\,b$, tenemos
$$\left|\frac{1}{n^s}\right| = \frac{1}{n^a}.$$
Podemos usar el \term{criterio integral de convergencia}: $\sum_{n \ge 1} \frac{1}{n^a}$ es convergente si y solamente si
$$\int_1^\infty \frac{1}{x^a}\,dx < \infty.$$
En efecto, tenemos
$$\int_1^\infty \frac{1}{x^a}\,dx = \lim_{n \to \infty} \left[\frac{x^{1-a}}{1-a}\right]^n_1 = \lim_{n \to \infty} \left(\frac{n^{1-a}}{1-a} - \frac{1}{1-a}\right).$$
Este límite existe precisamente cuando $a > 1$.
\end{proof}
\end{observacion*}

Note que para $s = 1$ se obtiene la \term{serie armónica}
$$\zeta (1) = 1 + \frac{1}{2} + \frac{1}{3} + \frac{1}{4} + \cdots$$
que es divergente.

\begin{proof}[Demostración (Nicolás Oresme, siglo XIV)]
En la serie
$$1 + \frac{1}{2} + \frac{1}{3} + \frac{1}{4} + \frac{1}{5} + \frac{1}{6} + \frac{1}{7} + \frac{1}{8} + \frac{1}{9} + \frac{1}{10} + \frac{1}{11} + \frac{1}{12} + \frac{1}{13} + \frac{1}{14} + \frac{1}{15} + \frac{1}{16} + \cdots$$
reemplacemos cada término $\frac{1}{n}$ por el número máximo $\frac{1}{2^k} \le \frac{1}{n}$. Se obtiene una serie
$$1 + \frac{1}{2} + \underbrace{\left(\frac{1}{4} + \frac{1}{4}\right)}_{= \frac{1}{2}} + \underbrace{\left(\frac{1}{8} + \frac{1}{8} + \frac{1}{8} + \frac{1}{8}\right)}_{= \frac{1}{2}} + \underbrace{\left(\frac{1}{16} + \frac{1}{16} + \frac{1}{16} + \frac{1}{16} + \frac{1}{16} + \frac{1}{16} + \frac{1}{16} + \frac{1}{16}\right)}_{= \frac{1}{2}} + \cdots$$
que es obviamente divergente. Por tanto la serie armónica es también divergente.
\end{proof}

\begin{shaded}
\noindent\small En PARI/GP:

\begin{verbatim}
? zeta (2)
% = 1.6449340668482264364724151666460251892
\end{verbatim}
\end{shaded}

Para $s > 1$ la función $\zeta (s)$ es monótonamente decreciente, y se tiene $\lim_{s\to \infty} \zeta (s) = 1$:

\begin{center}
\includegraphics{../pic/zeta-pos.mps}
\end{center}

\begin{teorema*}[Fórmula del producto de Euler]
$$\sum_{n \ge 1} \frac{1}{n^s} = \prod_{p\text{ primo}} \frac{1}{1 - p^{-s}}.$$
\end{teorema*}

La fórmula de arriba tiene una gran importancia en la teoría de números y fue descubierta por Euler. He aquí la demostración original, reproducida de su artículo \href{http://eulerarchive.maa.org/pages/E072.html}{``Variae observationes circa series infinitas''}:

\begin{quote}
Si

\begin{equation}
\label{euler-1}
x = 1 + \frac{1}{2^s} + \frac{1}{3^s} + \frac{1}{4^s} + \frac{1}{5^s} + \frac{1}{6^s} + \cdots,
\end{equation}

\noindent entonces

\begin{equation}
\label{euler-2}
\frac{1}{2^s}\,x = \frac{1}{2^s} + \frac{1}{4^s} + \frac{1}{6^s} + \frac{1}{8^s} + \frac{1}{10^s} + \frac{1}{12^s} + \cdots,
\end{equation}

\noindent y subtraendo $\text{\eqnref{euler-1}}-\text{\eqnref{euler-2}}$ se obtiene

\begin{equation}
\label{euler-3}
\frac{2^s - 1}{2^s}\,x = 1 + \frac{1}{3^s} + \frac{1}{5^s} + \frac{1}{7^s} + \frac{1}{9^s} + \frac{1}{11^s} + \cdots
\end{equation}

Luego,

\begin{equation}
\label{euler-4}
\left(\frac{2^s - 1}{2^s}\right)\,\frac{1}{3^s}\,x = \frac{1}{3^s} + \frac{1}{9^s} + \frac{1}{15^s} + \frac{1}{21^s} + \frac{1}{27^s} + \cdots
\end{equation}

\noindent y $\text{\eqnref{euler-3}} - \text{\eqnref{euler-4}}$ nos da

\[ \left(\frac{2^s - 1}{2^s}\right)\,\left(\frac{3^s - 1}{3^s}\right)\,x = 1 + \frac{1}{5^s} + \frac{1}{7^s} + \frac{1}{11^s} + \cdots \]

Después de aplicar operaciones similares para todos los números primos, todos los términos excepto el primero se eliminan:
$$1 = \left(\frac{2^s - 1}{2^s}\right)\,\left(\frac{3^s - 1}{3^s}\right)\,\left(\frac{5^s - 1}{5^s}\right)\,\left(\frac{7^s - 1}{7^s}\right)\,\left(\frac{11^s - 1}{11^s}\right)\cdots x,$$
de donde se encuentra la serie $x$:
$$\left(\frac{2^s}{2^s - 1}\right)\,\left(\frac{3^s}{3^s - 1}\right)\,\left(\frac{5^s}{5^s - 1}\right)\,\left(\frac{7^s}{7^s - 1}\right)\,\left(\frac{11^s}{11^s - 1}\right)\cdots = x = 1 + \frac{1}{2^s} + \frac{1}{3^s} + \frac{1}{4^s} + \frac{1}{5^s} + \frac{1}{6^s} + \cdots$$
Q.E.D.
\end{quote}

Dejo al lector pensar por qué esta demostración es esencialmente correcta.

% % % % % % % % % % % % % % % % % % % % % % % % % % % % % %

\section*{Los valores $\zeta (2k)$}

El siguiente resultado fue descubierto por Euler:

\begin{teorema*}
Para todo $k\ge 1$
$$\zeta (2k) \dfn 1 + \frac{1}{2^{2k}} + \frac{1}{3^{2k}} + \frac{1}{4^{2k}} + \cdots = (-1)^{k+1} \, B_{2k}\,\frac{2^{2k-1}}{(2k)!}\,\pi^{2k}.$$
\end{teorema*}

Es algo sorprendente: ¡los números de Bernoulli surgen del estudio de las sumas de potencias $\sum_{1 \le i \le n} i^k$, y ahora los mismos números aparecen en sumas de potencias infinitas! Los primeros valores de $\zeta (2k)$ son entonces

\begin{align*}
\zeta (2) & = \frac{\pi^2}{6} \approx 1.644934\ldots, \\
\zeta (4) & = \frac{\pi^4}{90} \approx 1.082323\ldots, \\
\zeta (6) & = \frac{\pi^6}{945}  \approx 1.017343\ldots, \\
\zeta (8) & = \frac{\pi^8}{9450} \approx 1.004077\ldots, \\
\zeta (10) & = \frac{\pi^{10}}{93\,555} \approx 1.000994\ldots, \\
\zeta (12) & = \frac{691\,\pi^{12}}{638\,512\,875} \approx 1.000246\ldots
\end{align*}

En particular, el cálculo de $\zeta (2) = 1 + \frac{1}{4} + \frac{1}{9} + \frac{1}{16} + \cdots$ se conoce como el \term{problema de Basilea} que fue formulado por el matemático italiano \personality{Pietro Mengoli} (1626--1686) en 1644. La primera solución fue encontrada por Euler en 1735.

\begin{ejercicio*}
Calcule las sumas parciales $\sum_{1 \le n \le N} \frac{1}{n^2}$ en PARI/GP. Note que su convergencia a $\zeta (2)$ es bastante lenta. Esto explica un siglo de sufrimiento de los matemáticos que trataban de obtener un valor aproximado de $\zeta (2)$\dots hasta la llegada de Euler.
\end{ejercicio*}

\begin{corolario*}
$(-1)^{k+1}\,B_{2k} > 0$ para todo $k \ge 1$. Es decir, $B_{2k} \ne 0$ y los signos de los números de Bernoulli pares se alternan.

\begin{proof}
$$(-1)^{k+1}\, B_{2k} = \frac{(2k)! \, \zeta (2k)}{2^{2k-1}\,\pi^{2k}}.$$
\end{proof}
\end{corolario*}

También se ve que $|B_{2k}| \xrightarrow{k \to \infty} \infty$, y que $|B_{2k}|$ crece muy rápido con $k$:

\begin{align*}
B_2 & \approx +0.166667,\\
B_4 & \approx -0.033333,\\
B_6 & \approx +0.023810,\\
B_8 & \approx -0.033333,\\
B_{10} & \approx +0.075758,\\
B_{12} & \approx -0.253114,\\
B_{14} & \approx +1.166667,\\
B_{16} & \approx -7.092157,\\
B_{18} & \approx +54.971178,\\
B_{20} & \approx -529.124242.
\end{align*}

\begin{proof}[Primera demostración de la fórmula para $\zeta (2k)$]
Hemos visto en la lección de ayer la serie

\begin{equation}
\label{eqn:serie-para-t-cot-t}
t\,\cot (t) = 1 + \sum_{k \ge 1} (-1)^k \, 2^{2k}\,\frac{B_{2k}}{(2k)!}\,t^{2k}.
\end{equation}

En el análisis complejo se deduce otra serie [Ahlfors, ``Complex analysis'', Chapter 5, \S 2]
$$\cot (t) = \sum_{n\in \ZZ} \frac{1}{t - \pi n},$$
que corresponde a la ``descomposición en fracciones simples'' de una función meromorfa: $\cot (t)$ tiene polos simples en $t = \pi n$ para todo $n \in \ZZ$ con residuo

$$\lim_{t \to \pi n} (t - \pi n)\,\cot (t) = \lim_{t \to 0} \cos (t+\pi n) \, \frac{t}{\sen (t + \pi n)} = \lim_{t \to 0} (-1)^n\,\cos (t) \, \frac{t}{(-1)^n\,\sen (t)} = 1.$$
Por ``$\sum_{n\in \ZZ} \frac{1}{t - \pi n}$'' se entiende $\lim_{N\to \infty} \sum_{-N \le n \le N} \frac{1}{t - \pi n}$.
Luego,

\begin{align*}
t\,\cot (t) & = t\,\left(\frac{1}{t} + \sum_{n \ge 1} \left(\frac{1}{t - \pi n} + \frac{1}{t + \pi n}\right)\right) = 1 - 2\,\sum_{n \ge 1} \left(\frac{t^2}{(\pi n)^2 - t^2}\right) = 1 - 2\,\sum_{n \ge 1} \frac{t^2}{(\pi n)^2} \, \frac{1}{1-\left(\frac{t}{\pi n}\right)^2} \\
 & = 1 - 2\,\sum_{n \ge 1} \frac{t^2}{(\pi n)^2} \, \sum_{k \ge 0} \left(\frac{t}{\pi n}\right)^{2k} = 1 - 2\,\sum_{n \ge 1} \, \sum_{k \ge 1} \left(\frac{t}{\pi n}\right)^{2k} \quad \text{(la serie geométrica)}\\
 & = 1 - 2\,\sum_{k \ge 1} \left(\sum_{n\ge 1} \frac{1}{n^{2k}}\right)\,\frac{t^{2k}}{\pi^{2k}} = 1 - 2\,\sum_{k\ge 1} \frac{\zeta (2k)\,t^{2k}}{\pi^{2k}}. \quad \text{(cambiando el orden de sumación)}
\end{align*}

Comparando coeficientes con \eqnref{eqn:serie-para-t-cot-t}, tenemos
$$(-1)^k \, 2^{2k}\,\frac{B_{2k}}{(2k)!} = -2\,\frac{\zeta (2k)}{\pi^{2k}}.$$
\end{proof}

% % % % % % % % % % % % % % % % % % % % % % % % % % % % % %

\section*{Series de Fourier para $B_k (x)$}

Vamos a necesitar el siguiente resultado del análisis armónico, que es un caso especial de las \term{series de Fourier}:

\begin{hecho*}
Sea $f\colon \RR \to \RR$ una función continua por trozos y periódica:
$$f (x+1) = f (x).$$
Entonces para todo $x_0\in \RR$ donde $f$ es continua y la derivada izquierda y derecha de $f$ existen (pero no necesariamente coinciden) se tiene
$$f (x_0) = \sum_{n\in\ZZ} \widehat{f} (n) \, e^{2\pi i n x_0}, \quad \text{donde }\widehat{f} (n) \dfn \int_0^1 e^{-2\pi i n x} \, f(x) \, dx.$$
\end{hecho*}

\vspace{1em}

En nuestro caso, nos interesan las funciones
$$f (x) \dfn B_k (x - \lfloor x\rfloor),$$
donde $B_k (x)$ es el $k$-ésimo polinomio de Bernoulli. Para $k > 1$ la función $B_k (x - \lfloor x\rfloor)$ es continua y para $k = 1$ es discontinua en los puntos $x = n \in \ZZ$. También $B_k (x - \lfloor x\rfloor)$ es lisa para $k > 2$, pero $B_2 (x)$ no es lisa en los puntos $x = n \in \ZZ$, donde existen la derivada izquierda y derecha, pero son diferentes.

\begin{center}
\includegraphics{../pic/bernpol-per.mps}
\end{center}

Los coeficientes de la serie de Fourier para $f (x)$ se calculan fácilmente. Para $n = 0$ tenemos
$$\widehat{f} (0) = \int_0^1 B_k (x)\,dx = 0.$$
Luego, para $n \ne 0$ y $k = 1$ podemos usar integración por partes ($\int_a^b f'(x)\,g(x)\,dx = \left[f(x)\,g(x)\right]_a^b - \int_a^b f(x) \, g'(x)\,dx$):

\begin{align*}
\int_0^1 e^{-2\pi i n x}\,\left(x - \frac{1}{2}\right)\,dx & = -\frac{1}{2\pi i n} \int_0^1 \left(e^{-2\pi i n x}\right)' \, \left(x - \frac{1}{2}\right)\,dx \\
 & = -\frac{1}{2\pi i n} \, \left( \left[e^{-2\pi i n x}\,\left(x - \frac{1}{2}\right)\right]^1_0 - \underbrace{\int_0^1 e^{-2\pi i n x}\,dx}_{=0} \right) = -\frac{1}{2\pi i n}.
\end{align*}

Para $k > 1$ integración por partes y la relación $B_k' (x) = k\,B_{k-1} (x)$ nos dan

\begin{align*}
\widehat{f} (n) = \int_0^1 e^{-2\pi i n x} \, B_k (x)\,dx & = -\frac{1}{2\pi i n}\,\int_0^1 (e^{-2\pi i n x})' \, B_k (x)\,dx \\
 & = -\frac{1}{2\pi i n}\,\left(\left[e^{-2\pi i n x}\,B_k (x)\right]_0^1 - k\,\int_0^1 e^{-2\pi i n x} \, B_{k-1} (x)\,dx\right) \\
 & = \frac{k}{2\pi i n} \, \int_0^1 e^{-2\pi i n x}\,B_{k-1} (x)\,dx \\
 & = \frac{k\,(k-1)}{(2\pi i n)^2} \int_0^1 e^{-2\pi i n x}\,B_{k-2} (x)\,dx \\
 & = \cdots \\
 & = \frac{k!}{(2\pi i n)^{k-1}} \int_0^1 e^{-2\pi i n x}\,\left(x - \frac{1}{2}\right)\,dx \\
 & = \frac{k!}{(2\pi i n)^{k-1}} \cdot \left(-\frac{1}{2\pi i n}\right) = -\frac{k!}{(2\pi i n)^k}.
\end{align*}

Entonces, la serie de Fourier es

\begin{equation}
\label{serie-de-forier-para-Bk}
B_k (x - \lfloor x\rfloor) = -\frac{k!}{(2\pi i)^k}\sum_{\substack{n\in \ZZ \\ n \ne 0}} \frac{e^{2\pi i n x}}{n^k}.
\end{equation}

Como un caso especial, se obtiene la fórmula para $\zeta (2k)$:

\begin{proof}[Segunda demostración de la fórmula para $\zeta (2k)$]
Para $x = 0$ la identidad \eqnref{serie-de-forier-para-Bk} nos da
$$B_{2k} = B_{2k} (0) = -\frac{(2k)!}{(-1)^k\,(2\pi)^{2k}} \, 2\,\sum_{n \ge 1} \frac{1}{n^{2k}} = (-1)^{k+1}\frac{(2k)!}{2^{2k-1}\,\pi^{2k}}\,\zeta (2k).$$
\end{proof}

Note que los valores en los enteros impares $\zeta (2k+1)$ no se obtienen con este método.

% % % % % % % % % % % % % % % % % % % % % % % % % % % % % %

\section*{Los valores $\zeta (2k+1)$}

Los valores en los enteros positivos impares
$$\zeta (3), ~ \zeta (5), ~ \zeta (7), ~ \zeta (9), ~ \zeta (11), ~ \ldots$$
son más misteriosos. \emph{Al parecer}, son números trascendentes.

\begin{shaded}
Recordemos que un número $z\in \CC$ es \term{irracional} si $z \notin \QQ$.

Por otro lado, un número $z$ es \term{algebraico} si $z$ es una raíz de algún polinomio con coeficientes en $\ZZ$. Por ejemplo, $\sqrt{2}$ es un número algebraico irracional.

Si $z$ no es algebraico, se dice que es \term{trascendente}. Puede demostrarse que los números algebraicos forman un conjunto numerable, y entonces ¡casi todos los números son trascendentes! Lamentablemente, es muy difícil demostrar que un número específico es trascendente. Por ejemplo, $\pi$ y $e$ son trascendentes---es un corolario del célebre \term{teorema de Lindemann--Weierstrass}.
\end{shaded}

Por supuesto, los números
$$\zeta (2k) = (-1)^{k+1} \, B_{2k}\,\frac{2^{2k-1}}{(2k)!}\,\pi^{2k}$$
son también transcendentes, ya que $\pi$ es trascendente (¡de hecho, es uno de los pocos números específicos cuya trascendencia se puede demostrar!). Los valores $\zeta (2k+1)$ deberían de ser trascendentes por alguna razón más sofisticada, y se supone que entre $\zeta (2k+1)$ distintos no hay ninguna relación algebraica.

Sin embargo, todavía no hay demostraciones ni siquiera de que los $\zeta (2k+1)$ sean irracionales. En 1977 el matemático francés \personality{Roger Apéry} demostró que el número
$$\zeta (3) \approx 1.20205690315959428539973816\ldots$$
es irracional. La tumba de Apéry en París lleva la inscripción

\begin{center}
\noindent\textsc{Roger APÉRY}

\noindent\textsc{1916--1994}

\vspace{0.7em}

\noindent\rule{1cm}{0.4mm}

\vspace{1em}

\noindent$1 + \frac{1}{8} + \frac{1}{27} + \frac{1}{64} + \cdots \ne \frac{p}{q}$
\end{center}

Los métodos de Apéry no se generalizan para demostrar que $\zeta (5)$ es irracional; hay pocos resultados en esta dirección. El matemático francés \personality{Tanguy Rivoal} demostró en 2000 que entre los números $\zeta (3), \zeta (7), \zeta (9), \ldots$ hay una infinidad de irracionales, mientras que el matemático ruso \personality{Wadim Zudilin} demostró en 2001 que por lo menos un número entre $\zeta(5)$, $\zeta(7)$, $\zeta(9)$ y $\zeta(11)$ es irracional.

% % % % % % % % % % % % % % % % % % % % % % % % % % % % % %

\section*{Valores $\zeta (-1), \zeta (-2), \zeta (-3), \ldots$}

La función zeta puede ser definida en todo plano complejo (véase [Ahlfors, ``Complex analysis'', Chapter 5, \S 4] o cualquier libro de la teoría de números):

\begin{hecho*}
La función $\zeta (s)$ puede prolongarse analíticamente al plano complejo como una función meromorfa con un polo simple de residuo $1$ en $s = 1$. Esta prolongación, que también se denota por $\zeta (s)$, satisface la siguiente \term{ecuación funcional}:

\begin{equation}
\label{ecuacion-funcional}
\zeta (s) = 2^s \, \pi^{s-1} \, \sen \left(\frac{\pi s}{2}\right)\,\Gamma (1-s)\,\zeta (1-s).
\end{equation}
\end{hecho*}

Aquí $\Gamma (z) \dfn \int_0^\infty x^{z-1}\,e^{-x}\,dx$ denota la \term{función Gamma}. En particular, $\Gamma (n) = (n-1)!$ para $n = 1,2,3,4,\ldots$

\vspace{1em}

Gracias a la ecuación funcional y la fórmula de Euler
$$\zeta (2k) = (-1)^{k+1} \, B_{2k}\,\frac{2^{2k-1}}{(2k)!}\,\pi^{2k},$$
podemos obtener los valores de la función en los enteros negativos. En efecto, para los enteros negativos pares $s = -2k$ tenemos
$$\zeta (-2k) = 2^{-2k} \, \pi^{-2k-1} \, \underbrace{\sen \left(-\frac{\pi k}{2}\right)}_{= 0}\,\Gamma (2k+1)\,\zeta (2k+1) = 0.$$
Y para los $s = -(2k+1)$ impares,

\begin{align*}
\zeta (-(2k+1)) & = 2^{-(2k+1)} \, \pi^{-(2k+2)} \, \sen \left(-\frac{\pi\,(2k+1)}{2}\right)\,(2k+1)!\,\zeta (2k+2) \\
 & = \cancel{2^{-(2k+1)}} \, \cancel{\pi^{-(2k+2)}}\,(-1)^{k+1}\,(2k+1)!\,(-1)^k \, B_{2k+2}\,\frac{\cancel{2^{2k+1}}}{(2k+2)!}\,\cancel{\pi^{2k+2}} \\
 & = -\frac{B_{2k+2}}{2k+2}.
\end{align*}

Ya que $B_n = 0$ para $n$ impar, en ambos casos se tiene
$$\zeta (-n) = -\frac{B_{n+1}}{n+1}.$$
Además, para $n = 0$ la prolongación analítica nos da $\zeta (0) = -\frac{1}{2} = -B_1$, así que esta fórmula es válida también para $n = 0$.

\[ \begin{array}{rcccccccccccc}
n\colon & 0 & -1 & -2 & -3 & -4 & -5 & -6 & -7 & -8 & -9 & -10 & \cdots \\
\hline
\zeta(n)\colon & -\frac{1}{2} & -\frac{1}{12} & 0 & \frac{1}{120} & 0 & -\frac{1}{252} & 0 & \frac{1}{240} & 0 & -\frac{1}{132} & 0 & \cdots
\end{array} \]

\begin{center}
\noindent\includegraphics{../pic/zeta.mps}
\end{center}

\noindent (Después $\zeta (s)$ es decreciente hasta su polo en $s = 1$.)

\pagebreak

Terminamos por el cálculo de $\zeta (-1) = -\frac{1}{12}$ encontrado por Euler:

\begin{quote}
Para la serie geométrica
$$1 + x + x^2 + x^3 + x^4 + x^5 + x^6 + x^7 + x^8 + \cdots = \frac{1}{1-x}$$
la derivada formal nos da
$$1 + 2\,x + 3\,x^2 + 4\,x^3 + 5\,x^4 + 6\,x^5 + 7\,x^6 + 8\,x^7 + \cdots = \frac{1}{(1-x)^2},$$
de donde para $x = -1$ (¡sic!)
$$1 - 2 + 3 - 4 + 5 - 6 + 7 - 8 + \cdots = \frac{1}{4}.$$

Luego,

\begin{align*}
-3\,\zeta (-1) & = \zeta (-1) - 4\,\zeta (-1) \\
 & = (1 + 2 + 3 + 4 + \cdots) - (4 + 8 + 12 + 16 + \cdots) \\
  & = 1 - 2 + 3 - 4 + 5 - 6 + 7 - 8 + \cdots = \frac{1}{4},
\end{align*}

\noindent lo que implica $\zeta (-1) = -\frac{1}{12}$, Q.E.D.
\end{quote}

El lector no debería tomar en serio el argumento de arriba ni usar métodos similares en sus demostraciones.

\end{document}