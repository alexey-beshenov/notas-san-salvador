\documentclass{article}

% TODO : CLEAN UP THIS MESS
% (AND MAKE SURE ALL TEXTS STILL COMPILE)
\usepackage[leqno]{amsmath}
\usepackage{amssymb}
\usepackage{graphicx}

\usepackage{diagbox} % table heads with diagonal lines
\usepackage{relsize}

\usepackage{wasysym}
\usepackage{scrextend}
\usepackage{epigraph}
\setlength\epigraphwidth{.6\textwidth}

\usepackage[utf8]{inputenc}

\usepackage{titlesec}
\titleformat{\chapter}[display]
  {\normalfont\sffamily\huge\bfseries}
  {\chaptertitlename\ \thechapter}{5pt}{\Huge}
\titleformat{\section}
  {\normalfont\sffamily\Large\bfseries}
  {\thesection}{1em}{}
\titleformat{\subsection}
  {\normalfont\sffamily\large\bfseries}
  {\thesubsection}{1em}{}
\titleformat{\part}[display]
  {\normalfont\sffamily\huge\bfseries}
  {\partname\ \thepart}{0pt}{\Huge}

\usepackage[T1]{fontenc}
\usepackage{fourier}
\usepackage{paratype}

\usepackage[symbol,perpage]{footmisc}

\usepackage{perpage}
\MakePerPage{footnote}

\usepackage{array}
\newcolumntype{x}[1]{>{\centering\hspace{0pt}}p{#1}}

% TODO: the following line causes conflict with new texlive (!)
% \usepackage[english,russian,polutonikogreek,spanish]{babel}
% \newcommand{\russian}[1]{{\selectlanguage{russian}#1}}

% Remove conflicting options for the moment:
\usepackage[english,polutonikogreek,spanish]{babel}

\AtBeginDocument{\shorthandoff{"}}
\newcommand{\greek}[1]{{\selectlanguage{polutonikogreek}#1}}

% % % % % % % % % % % % % % % % % % % % % % % % % % % % % %
% Limit/colimit symbols (with accented i: lím / colím)

\usepackage{etoolbox} % \patchcmd

\makeatletter
\patchcmd{\varlim@}{lim}{\lim}{}{}
\makeatother
\DeclareMathOperator*{\colim}{co{\lim}}
\newcommand{\dirlim}{\varinjlim}
\newcommand{\invlim}{\varprojlim}

% % % % % % % % % % % % % % % % % % % % % % % % % % % % % %

\usepackage[all,color]{xy}

\usepackage{pigpen}
\newcommand{\po}{\ar@{}[dr]|(.4){\text{\pigpenfont I}}}
\newcommand{\pb}{\ar@{}[dr]|(.3){\text{\pigpenfont A}}}
\newcommand{\polr}{\ar@{}[dr]|(.65){\text{\pigpenfont A}}}
\newcommand{\pour}{\ar@{}[ur]|(.65){\text{\pigpenfont G}}}
\newcommand{\hstar}{\mathop{\bigstar}}

\newcommand{\bigast}{\mathop{\Huge \mathlarger{\mathlarger{\ast}}}}

\newcommand{\term}{\textbf}

\usepackage{stmaryrd}

\usepackage{cancel}

\usepackage{tikzsymbols}

\newcommand{\open}{\underset{\mathrm{open}}{\hookrightarrow}}
\newcommand{\closed}{\underset{\mathrm{closed}}{\hookrightarrow}}

\newcommand{\tcol}[2]{{#1 \choose #2}}

\newcommand{\homot}{\simeq}
\newcommand{\isom}{\cong}
\newcommand{\cH}{\mathcal{H}}
\renewcommand{\hom}{\mathrm{hom}}
\renewcommand{\div}{\mathop{\mathrm{div}}}
\renewcommand{\Im}{\mathop{\mathrm{Im}}}
\renewcommand{\Re}{\mathop{\mathrm{Re}}}
\newcommand{\id}[1]{\mathrm{id}_{#1}}
\newcommand{\idid}{\mathrm{id}}

\newcommand{\ZG}{{\ZZ G}}
\newcommand{\ZH}{{\ZZ H}}

\newcommand{\quiso}{\simeq}

\newcommand{\personality}[1]{{\sc #1}}

\newcommand{\mono}{\rightarrowtail}
\newcommand{\epi}{\twoheadrightarrow}
\newcommand{\xepi}[1]{\xrightarrow{#1}\mathrel{\mkern-14mu}\rightarrow}

% % % % % % % % % % % % % % % % % % % % % % % % % % % % % %

\DeclareMathOperator{\Ad}{Ad}
\DeclareMathOperator{\Aff}{Aff}
\DeclareMathOperator{\Ann}{Ann}
\DeclareMathOperator{\Aut}{Aut}
\DeclareMathOperator{\Br}{Br}
\DeclareMathOperator{\CH}{CH}
\DeclareMathOperator{\Cl}{Cl}
\DeclareMathOperator{\Coeq}{Coeq}
\DeclareMathOperator{\Coind}{Coind}
\DeclareMathOperator{\Cop}{Cop}
\DeclareMathOperator{\Corr}{Corr}
\DeclareMathOperator{\Cor}{Cor}
\DeclareMathOperator{\Cov}{Cov}
\DeclareMathOperator{\Der}{Der}
\DeclareMathOperator{\Div}{Div}
\DeclareMathOperator{\D}{D}
\DeclareMathOperator{\Ehr}{Ehr}
\DeclareMathOperator{\End}{End}
\DeclareMathOperator{\Eq}{Eq}
\DeclareMathOperator{\Ext}{Ext}
\DeclareMathOperator{\Frac}{Frac}
\DeclareMathOperator{\Frob}{Frob}
\DeclareMathOperator{\Funct}{Funct}
\DeclareMathOperator{\Fun}{Fun}
\DeclareMathOperator{\GL}{GL}
\DeclareMathOperator{\Gal}{Gal}
\DeclareMathOperator{\Gr}{Gr}
\DeclareMathOperator{\Hol}{Hol}
\DeclareMathOperator{\Hom}{Hom}
\DeclareMathOperator{\Ho}{Ho}
\DeclareMathOperator{\Id}{Id}
\DeclareMathOperator{\Ind}{Ind}
\DeclareMathOperator{\Inn}{Inn}
\DeclareMathOperator{\Isom}{Isom}
\DeclareMathOperator{\Ker}{Ker}
\DeclareMathOperator{\Lan}{Lan}
\DeclareMathOperator{\Lie}{Lie}
\DeclareMathOperator{\Map}{Map}
\DeclareMathOperator{\Mat}{Mat}
\DeclareMathOperator{\Max}{Max}
\DeclareMathOperator{\Mor}{Mor}
\DeclareMathOperator{\Nat}{Nat}
\DeclareMathOperator{\Nrd}{Nrd}
\DeclareMathOperator{\Ob}{Ob}
\DeclareMathOperator{\Out}{Out}
\DeclareMathOperator{\PGL}{PGL}
\DeclareMathOperator{\PSL}{PSL}
\DeclareMathOperator{\PSU}{PSU}
\DeclareMathOperator{\Pic}{Pic}
\DeclareMathOperator{\RHom}{RHom}
\DeclareMathOperator{\Rad}{Rad}
\DeclareMathOperator{\Ran}{Ran}
\DeclareMathOperator{\Rep}{Rep}
\DeclareMathOperator{\Res}{Res}
\DeclareMathOperator{\SL}{SL}
\DeclareMathOperator{\SO}{SO}
\DeclareMathOperator{\SU}{SU}
\DeclareMathOperator{\Sh}{Sh}
\DeclareMathOperator{\Sing}{Sing}
\DeclareMathOperator{\Specm}{Specm}
\DeclareMathOperator{\Spec}{Spec}
\DeclareMathOperator{\Sp}{Sp}
\DeclareMathOperator{\Stab}{Stab}
\DeclareMathOperator{\Sym}{Sym}
\DeclareMathOperator{\Tors}{Tors}
\DeclareMathOperator{\Tor}{Tor}
\DeclareMathOperator{\Tot}{Tot}
\DeclareMathOperator{\UUU}{U}

\DeclareMathOperator{\adj}{adj}
\DeclareMathOperator{\ad}{ad}
\DeclareMathOperator{\af}{af}
\DeclareMathOperator{\card}{card}
\DeclareMathOperator{\cm}{cm}
\DeclareMathOperator{\codim}{codim}
\DeclareMathOperator{\cod}{cod}
\DeclareMathOperator{\coeq}{coeq}
\DeclareMathOperator{\coim}{coim}
\DeclareMathOperator{\coker}{coker}
\DeclareMathOperator{\cont}{cont}
\DeclareMathOperator{\conv}{conv}
\DeclareMathOperator{\cor}{cor}
\DeclareMathOperator{\depth}{depth}
\DeclareMathOperator{\diag}{diag}
\DeclareMathOperator{\diam}{diam}
\DeclareMathOperator{\dist}{dist}
\DeclareMathOperator{\dom}{dom}
\DeclareMathOperator{\eq}{eq}
\DeclareMathOperator{\ev}{ev}
\DeclareMathOperator{\ex}{ex}
\DeclareMathOperator{\fchar}{char}
\DeclareMathOperator{\fr}{fr}
\DeclareMathOperator{\gr}{gr}
\DeclareMathOperator{\im}{im}
\DeclareMathOperator{\infl}{inf}
\DeclareMathOperator{\interior}{int}
\DeclareMathOperator{\intrel}{intrel}
\DeclareMathOperator{\inv}{inv}
\DeclareMathOperator{\length}{length}
\DeclareMathOperator{\mcd}{mcd}
\DeclareMathOperator{\mcm}{mcm}
\DeclareMathOperator{\multideg}{multideg}
\DeclareMathOperator{\ord}{ord}
\DeclareMathOperator{\pr}{pr}
\DeclareMathOperator{\rel}{rel}
\DeclareMathOperator{\res}{res}
\DeclareMathOperator{\rkred}{rkred}
\DeclareMathOperator{\rkss}{rkss}
\DeclareMathOperator{\rk}{rk}
\DeclareMathOperator{\sgn}{sgn}
\DeclareMathOperator{\sk}{sk}
\DeclareMathOperator{\supp}{supp}
\DeclareMathOperator{\trdeg}{trdeg}
\DeclareMathOperator{\tr}{tr}
\DeclareMathOperator{\vol}{vol}

\newcommand{\iHom}{\underline{\Hom}}

\renewcommand{\AA}{\mathbb{A}}
\newcommand{\CC}{\mathbb{C}}
\renewcommand{\SS}{\mathbb{S}}
\newcommand{\TT}{\mathbb{T}}
\newcommand{\PP}{\mathbb{P}}
\newcommand{\BB}{\mathbb{B}}
\newcommand{\RR}{\mathbb{R}}
\newcommand{\ZZ}{\mathbb{Z}}
\newcommand{\FF}{\mathbb{F}}
\newcommand{\HH}{\mathbb{H}}
\newcommand{\NN}{\mathbb{N}}
\newcommand{\QQ}{\mathbb{Q}}
\newcommand{\KK}{\mathbb{K}}

% % % % % % % % % % % % % % % % % % % % % % % % % % % % % %

\usepackage{amsthm}

\newcommand{\legendre}[2]{\left(\frac{#1}{#2}\right)}

\newcommand{\examplesymbol}{$\blacktriangle$}
\renewcommand{\qedsymbol}{$\blacksquare$}

\newcommand{\dfn}{\mathrel{\mathop:}=}
\newcommand{\rdfn}{=\mathrel{\mathop:}}

\usepackage{xcolor}
\definecolor{mylinkcolor}{rgb}{0.0,0.4,1.0}
\definecolor{mycitecolor}{rgb}{0.0,0.4,1.0}
\definecolor{shadecolor}{rgb}{0.79,0.78,0.65}
\definecolor{gray}{rgb}{0.6,0.6,0.6}

\usepackage{colortbl}

\definecolor{myred}{rgb}{0.7,0.0,0.0}
\definecolor{mygreen}{rgb}{0.0,0.7,0.0}
\definecolor{myblue}{rgb}{0.0,0.0,0.7}

\definecolor{redshade}{rgb}{0.9,0.5,0.5}
\definecolor{greenshade}{rgb}{0.5,0.9,0.5}

\usepackage[unicode,colorlinks=true,linkcolor=mylinkcolor,citecolor=mycitecolor]{hyperref}
\newcommand{\refref}[2]{\hyperref[#2]{#1~\ref*{#2}}}
\newcommand{\eqnref}[1]{\hyperref[#1]{(\ref*{#1})}}

\newcommand{\tos}{\!\!\to\!\!}

\usepackage{framed}

\newcommand{\cequiv}{\simeq}

\makeatletter
\newcommand\xleftrightarrow[2][]{%
  \ext@arrow 9999{\longleftrightarrowfill@}{#1}{#2}}
\newcommand\longleftrightarrowfill@{%
  \arrowfill@\leftarrow\relbar\rightarrow}
\makeatother

\newcommand{\bsquare}{\textrm{\ding{114}}}

% % % % % % % % % % % % % % % % % % % % % % % % % % % % % %

\newtheoremstyle{myplain}
  {\topsep}   % ABOVESPACE
  {\topsep}   % BELOWSPACE
  {\itshape}  % BODYFONT
  {0pt}       % INDENT (empty value is the same as 0pt)
  {\bfseries} % HEADFONT
  {.}         % HEADPUNCT
  {5pt plus 1pt minus 1pt} % HEADSPACE
  {\thmnumber{#2}. \thmname{#1}\thmnote{ (#3)}}   % CUSTOM-HEAD-SPEC

\newtheoremstyle{myplainnameless}
  {\topsep}   % ABOVESPACE
  {\topsep}   % BELOWSPACE
  {\normalfont}  % BODYFONT
  {0pt}       % INDENT (empty value is the same as 0pt)
  {\bfseries} % HEADFONT
  {.}         % HEADPUNCT
  {5pt plus 1pt minus 1pt} % HEADSPACE
  {\thmnumber{#2}}   % CUSTOM-HEAD-SPEC 

\newtheoremstyle{sectionexercise}
  {\topsep}   % ABOVESPACE
  {\topsep}   % BELOWSPACE
  {\normalfont}  % BODYFONT
  {0pt}       % INDENT (empty value is the same as 0pt)
  {\bfseries} % HEADFONT
  {.}         % HEADPUNCT
  {5pt plus 1pt minus 1pt} % HEADSPACE
  {Ejercicio \thmnumber{#2}\thmnote{ (#3)}}   % CUSTOM-HEAD-SPEC

\newtheoremstyle{mydefinition}
  {\topsep}   % ABOVESPACE
  {\topsep}   % BELOWSPACE
  {\normalfont}  % BODYFONT
  {0pt}       % INDENT (empty value is the same as 0pt)
  {\bfseries} % HEADFONT
  {.}         % HEADPUNCT
  {5pt plus 1pt minus 1pt} % HEADSPACE
  {\thmnumber{#2}. \thmname{#1}\thmnote{ (#3)}}   % CUSTOM-HEAD-SPEC

% EN ESPAÑOL

\newtheorem*{hecho*}{Hecho}
\newtheorem*{corolario*}{Corolario}
\newtheorem*{teorema*}{Teorema}
\newtheorem*{conjetura*}{Conjetura}
\newtheorem*{proyecto*}{Proyecto}
\newtheorem*{observacion*}{Observación}

\newtheorem*{lema*}{Lema}
\newtheorem*{resultado-clave*}{Resultado clave}
\newtheorem*{proposicion*}{Proposición}

\theoremstyle{definition}
\newtheorem*{ejercicio*}{Ejercicio}
\newtheorem*{definicion*}{Definición}
\newtheorem*{comentario*}{Comentario}
\newtheorem*{definicion-alternativa*}{Definición alternativa}
\newtheorem*{ejemploxs}{Ejemplo}
\newenvironment{ejemplo*}
  {\pushQED{\qed}\renewcommand{\qedsymbol}{\examplesymbol}\ejemploxs}
  {\popQED\endejemploxs}

\theoremstyle{myplain}
\newtheorem{proposicion}{Proposición}[section]

\newtheorem{proyecto}[proposicion]{Proyecto}
\newtheorem{teorema}[proposicion]{Teorema}
\newtheorem{corolario}[proposicion]{Corolario}
\newtheorem{hecho}[proposicion]{Hecho}
\newtheorem{lema}[proposicion]{Lema}

\newtheorem{observacion}[proposicion]{Observación}

\newenvironment{observacionejerc}
    {\pushQED{\qed}\renewcommand{\qedsymbol}{$\square$}\csname inner@observacionejerc\endcsname}
    {\popQED\csname endinner@observacionejerc\endcsname}
\newtheorem{inner@observacionejerc}[proposicion]{Observación}

\newenvironment{proposicionejerc}
    {\pushQED{\qed}\renewcommand{\qedsymbol}{$\square$}\csname inner@proposicionejerc\endcsname}
    {\popQED\csname endinner@proposicionejerc\endcsname}
\newtheorem{inner@proposicionejerc}[proposicion]{Proposicion}

\newenvironment{lemaejerc}
    {\pushQED{\qed}\renewcommand{\qedsymbol}{$\square$}\csname inner@lemaejerc\endcsname}
    {\popQED\csname endinner@lemaejerc\endcsname}
\newtheorem{inner@lemaejerc}[proposicion]{Lema}

\newtheorem{calculo}[proposicion]{Cálculo}

\theoremstyle{myplainnameless}
\newtheorem{nameless}[proposicion]{}

\theoremstyle{mydefinition}
\newtheorem{comentario}[proposicion]{Comentario}
\newtheorem{comentarioast}[proposicion]{Comentario ($\clubsuit$)}
\newtheorem{construccion}[proposicion]{Construcción}
\newtheorem{aplicacion}[proposicion]{Aplicación}
\newtheorem{definicion}[proposicion]{Definición}
\newtheorem{definicion-alternativa}[proposicion]{Definición alternativa}
\newtheorem{notacion}[proposicion]{Notación}
\newtheorem{advertencia}[proposicion]{Advertencia}
\newtheorem{digresion}[proposicion]{Digresión}
\newtheorem{ejemplox}[proposicion]{Ejemplo}
\newenvironment{ejemplo}
  {\pushQED{\qed}\renewcommand{\qedsymbol}{\examplesymbol}\ejemplox}
  {\popQED\endejemplox}
\newtheorem{contraejemplox}[proposicion]{Contraejemplo}
\newenvironment{contraejemplo}
  {\pushQED{\qed}\renewcommand{\qedsymbol}{\examplesymbol}\contraejemplox}
  {\popQED\endcontraejemplox}
\newtheorem{noejemplox}[proposicion]{No-ejemplo}
\newenvironment{noejemplo}
  {\pushQED{\qed}\renewcommand{\qedsymbol}{\examplesymbol}\noejemplox}
  {\popQED\endnoejemplox}
 
\newtheorem{ejemploastx}[proposicion]{Ejemplo ($\clubsuit$)}
\newenvironment{ejemploast}
  {\pushQED{\qed}\renewcommand{\qedsymbol}{\examplesymbol}\ejemploastx}
  {\popQED\endejemploastx}

\ifdefined\exercisespersection
  \theoremstyle{sectionexercise}
  \newtheorem{ejercicio}{}[section]
  \theoremstyle{mydefinition}
\else
  \ifdefined\exercisesglobal
    \theoremstyle{sectionexercise}
    \newtheorem{ejercicio}{}
    \theoremstyle{mydefinition}
  \else
    \ifdefined\exercisespersection
      \newtheorem{ejercicio}[proposicion]{Ejercicio}
    \fi
  \fi
\fi

% % % % % % % % % % % % % % % % % % % % % % % % % % % % % %

\theoremstyle{myplain}
\newtheorem{proposition}{Proposition}[section]
\newtheorem*{fact*}{Fact}
\newtheorem*{proposition*}{Proposition}
\newtheorem{lemma}[proposition]{Lemma}
\newtheorem*{lemma*}{Lemma}

\newtheorem{exercise}{Exercise}
\newtheorem*{hint}{Hint}

\newtheorem{theorem}[proposition]{Theorem}
\newtheorem{conjecture}[proposition]{Conjecture}
\newtheorem*{theorem*}{Theorem}
\newtheorem{corollary}[proposition]{Corollary}
\newtheorem{fact}[proposition]{Fact}
\newtheorem*{claim}{Claim}
\newtheorem{definition-theorem}[proposition]{Definition-theorem}

\theoremstyle{mydefinition}
\newtheorem{examplex}[proposition]{Example}
\newenvironment{example}
  {\pushQED{\qed}\renewcommand{\qedsymbol}{\examplesymbol}\examplex}
  {\popQED\endexamplex}

\newtheorem*{examplexx}{Example}
\newenvironment{example*}
  {\pushQED{\qed}\renewcommand{\qedsymbol}{\examplesymbol}\examplexx}
  {\popQED\endexamplexx}

\newtheorem{definition}[proposition]{Definition}
\newtheorem*{definition*}{Definition}
\newtheorem{wrong-definition}[proposition]{Wrong definition}
\newtheorem{remark}[proposition]{Remark}

\makeatletter
\newcommand{\xRightarrow}[2][]{\ext@arrow 0359\Rightarrowfill@{#1}{#2}}
\makeatother

% % % % % % % % % % % % % % % % % % % % % % % % % % % % % %

\newcommand{\Et}{\mathop{\text{\rm Ét}}}

\newcommand{\categ}[1]{\text{\bf #1}}
\newcommand{\vcateg}{\mathcal}
\newcommand{\bone}{{\boldsymbol 1}}
\newcommand{\bDelta}{{\boldsymbol\Delta}}
\newcommand{\bR}{{\mathbf{R}}}

\newcommand{\univ}{\mathfrak}

\newcommand{\TODO}{\colorbox{red}{\textbf{*** TODO ***}}}
\newcommand{\proofreadme}{\colorbox{red}{\textbf{*** NEEDS PROOFREADING ***}}}

\makeatletter
\def\iddots{\mathinner{\mkern1mu\raise\p@
\vbox{\kern7\p@\hbox{.}}\mkern2mu
\raise4\p@\hbox{.}\mkern2mu\raise7\p@\hbox{.}\mkern1mu}}
\makeatother

\newcommand{\ssincl}{\reflectbox{\rotatebox[origin=c]{45}{$\subseteq$}}}
\newcommand{\vsupseteq}{\reflectbox{\rotatebox[origin=c]{-90}{$\supseteq$}}}
\newcommand{\vin}{\reflectbox{\rotatebox[origin=c]{90}{$\in$}}}

\newcommand{\Ga}{\mathbb{G}_\mathrm{a}}
\newcommand{\Gm}{\mathbb{G}_\mathrm{m}}

\renewcommand{\U}{\UUU}

\DeclareRobustCommand{\Stirling}{\genfrac\{\}{0pt}{}}
\DeclareRobustCommand{\stirling}{\genfrac[]{0pt}{}}

% % % % % % % % % % % % % % % % % % % % % % % % % % % % % %
% tikz

\usepackage{tikz-cd}
\usetikzlibrary{babel}
\usetikzlibrary{decorations.pathmorphing}
\usetikzlibrary{arrows}
\usetikzlibrary{calc}
\usetikzlibrary{fit}
\usetikzlibrary{hobby}

% % % % % % % % % % % % % % % % % % % % % % % % % % % % % %
% Banners

\newcommand\mybannerext[3]{{\normalfont\sffamily\bfseries\large\noindent #1

\noindent #2

\noindent #3

}\noindent\rule{\textwidth}{1.25pt}

\vspace{1em}}

\newcommand\mybanner[2]{{\normalfont\sffamily\bfseries\large\noindent #1

\noindent #2

}\noindent\rule{\textwidth}{1.25pt}

\vspace{1em}}

\renewcommand{\O}{\mathcal{O}}

\usepackage{diagbox}

\usepackage[numbers]{natbib}

\numberwithin{equation}{section}

\usepackage{tocloft}

\renewcommand\cftsecfont{\normalfont}
\renewcommand\cfttoctitlefont{\sffamily\large\bfseries}
\renewcommand\cftsecpagefont{\normalfont}
\renewcommand{\cftsecleader}{\cftdotfill{\cftsecdotsep}}
\renewcommand\cftsecdotsep{\cftdot}
\renewcommand\cftsubsecdotsep{\cftdot}

\def\mystrut(#1,#2){\vrule height #1 depth #2 width 0pt}

\newcolumntype{f}[1]{%
>{\mystrut(3ex,2ex)\centering}%
p{#1}%
<{}}

\usepackage{fancyhdr}

\usepackage[
  top=2cm,
  bottom=2cm,
  left=3cm,
  right=2cm,
  headheight=17pt,
  includehead,includefoot,
  heightrounded,
]{geometry}

\fancyfoot{}
\fancyhead[RO,LE]{\thepage}
\fancyhead[LO]{\nouppercase\leftmark}
\fancyhead[RE]{\nouppercase\rightmark}

\pagestyle{fancy}

\author{Alexey Beshenov (cadadr@gmail.com)}
\title{Números de Bernoulli}
\date{Febrero de 2017}

\usepackage{xcolor}
\newcommand{\highlight}[1]{\colorbox{shadecolor}{$\displaystyle #1$}}

\theoremstyle{definition}
\newtheorem{ejercicio}{Ejercicio}

\begin{document}

{\normalfont\sffamily\bfseries \maketitle}

Los números de Bernoulli son ciertos números racionales
$$B_0 = 1, ~
B_1 = \frac{1}{2}, ~
B_2 = \frac{1}{6}, ~
B_3 = 0, ~
B_4 = -\frac{1}{30}, ~
B_5 = 0, ~
B_6 = \frac{1}{42}, ~
B_7 = 0, ~
B_8 = -\frac{1}{30}, ~
\ldots$$
que aparecen en varios contextos de la teoría de números, combinatoria y
análisis. En estos apuntes vamos a revisar sus diferentes definiciones:
una que surge del estudio de sumas de potencias $1^k + 2^k + \cdots + n^k$,
y otra definición por una función generatriz:
$$\frac{t\,e^t}{e^t - 1} = \sum_{k\ge 0} \frac{B_k}{k!}\,t^k.$$
Luego vamos a ver la demostración de la famosa fórmula de Euler para los valores
especiales de la función zeta:
$$\zeta (2k) \dfn 1 + \frac{1}{2^{2k}} + \frac{1}{3^{2k}} + \frac{1}{4^{2k}} + \cdots =
(-1)^{k+1} \, B_{2k}\,\frac{2^{2k-1}}{(2k)!}\,\pi^{2k}.$$
Finalmente, vamos a demostrar el \term{teorema de Clausen--von Staudt} sobre los
denominadores de $B_k$ y las \term{congruencias de Kummer}.

\vspace{1em}

Un aspecto atractivo de este tema es que todos los resultados se tratan de
números concretos y el lector puede experimentar y hacer cálculos por sí
mismo. Afortunadamente, ya no estamos en los tiempos de Euler y Bernoulli y
podemos usar la computadora. Mi programa preferido para la teoría de números es
PARI/GP, y voy a ilustrar todo con su código. El lector puede descargar PARI/GP
de la página

\begin{center}
  \noindent\url{http://pari.math.u-bordeaux.fr/}
\end{center}

\vspace{1em}

Para preparar estos apuntes, he usado principalmente los primeros capítulos
del libro \cite{Arakawa-Ibukiyama-Kaneko-2014}, y el lector interesado puede
consultarlo para más información sobre los números de Bernoulli y las funciones
zeta.

\vspace{1em}

Le doy gracias a \personality{Gabriel Chicas Reyes} de la Universidad de
El Salvador por su ayuda con la redacción de estos apuntes.

\pagebreak
\tableofcontents

% % % % % % % % % % % % % % % % % % % % % % % % % % % % % %

\pagebreak
\section{Sumas de potencias de números naturales}
\label{sumas-de-potencias}

La suma de $n$ números naturales consecutivos puede ser calculada mediante la
fórmula
$$1+2+\cdots+n = \frac{(n+1)\,n}{2} = \frac{1}{2}\,n^2 + \frac{1}{2}\,n.$$

Probablemente el lector también conoce la fórmula para las sumas de cuadrados:
$$1^2 + 2^2 + \cdots + n^2 = \frac{n\,(n+1)\,(2n + 1)}{6} = \frac{n^3}{3} + \frac{n^2}{2} + \frac{n}{6}.$$
---es fácil demostrarla por inducción. Muchos matemáticos trataron de encontrar
la fórmula similar para las sumas de cubos y otras potencias superiores.
Es un problema muy natural, y la solución fue descubierta al principio del siglo
XVIII por el matemático suizo \personality{Jacob Bernoulli} (1654--1705) y
independientemente por el matemático japonés \personality{Seki Takakazu}
(1642--1708). Denotemos por $S_k (n)$ la suma de las $k$-ésimas potencias de los
números naturales hasta $n$:
$$S_k (n) \dfn \sum_{1 \le i \le n} i^k = 1^k + 2^k + \cdots + n^k.$$
En particular,
\[ S_0 (n) = n, \quad
  S_1 (n) = \frac{1}{2}\,n^2 + \frac{1}{2}\,n, \quad
  S_2 (n) = \frac{n^3}{3} + \frac{n^2}{2} + \frac{n}{6}. \]
Para obtener las fórmulas para $S_3 (n), S_4 (n), S_5 (n)$, etcétera, recordemos
primero el \term{teorema del binomio}:
$$(x+y)^k = \sum_{0 \le i \le k} {k \choose i} x^{k-i}\,y^i,$$
donde
$${k \choose i} = \frac{k!}{(k-i)!\,i!}$$
denota un \term{coeficiente binomial}, definido como el número de posibilidades
de escoger $i$ objetos entre un total de $k$ objetos.

\begin{shaded}
  \small\noindent En PARI/GP, \verb|binomial(k,i)| $= {k\choose i}$.

\begin{verbatim}
/* vector (n,i,expr) devuelve un vector con la expresión
   expr evaluada con i=1, i=2, ..., i=n: */
? vector (7,i,binomial (6,i-1))
% = [1, 6, 15, 20, 15, 6, 1]
\end{verbatim}
\end{shaded}

En particular, tenemos
$$(m+1)^{k+1} - m^{k+1} = \sum_{0 \le i \le k} {k+1 \choose i}\,m^i.$$
La suma de estas identidades para $m = 1, 2, \ldots, n$ nos da
$$(n+1)^{k+1} - 1 = \sum_{0 \le i \le k} {k+1 \choose i}\,S_i (n),$$
de donde tenemos una expresión de $S_k (n)$ en términos de
$S_0 (n), S_1 (n), \ldots, S_{k-1} (n)$:

\begin{equation}
  \label{formula-para-Skn}
  S_k (n) = \frac{1}{k+1} \, \left((n+1)^{k+1} - 1 - \sum_{0 \le i \le k-1} {k+1\choose i} \, S_i (n)\right).
\end{equation}

Por inducción se ve que $S_k (n)$ es un polinomio en $n$ de grado $k+1$,
con coeficiente principal $\frac{1}{k+1}$. Para evitar una posible confusión,
denotemos la variable por $x$. El polinomio $S_k (x) \in \QQ [x]$ está
determinado por sus valores en $x = n \in \NN$ (¿por qué?). Por la definición
de $S_k (n)$, tenemos $S_k (n+1) - S_k (n) = (n+1)^k$ para $n = 1,2,3,\ldots$
Para los polinomios, esto nos da la relación

\begin{equation}
  \label{eqn:Skn+1-y-Skn}
  S_k (x+1) - S_k (x) = (x+1)^k.
\end{equation}

En particular, $S_k (1) - S_k (0) = 1$, y ya que $S_k (1) = 1$, esto significa
que $S_k (0) = 0$; es decir, el término constante del polinomio $S_k (x)$ es
nulo (también podemos verlo por inducción de la fórmula
\eqnref{formula-para-Skn}). Usando \eqnref{formula-para-Skn}, podemos calcular
algunos $S_k (x)$.

\begin{shaded}
  \small

  \noindent Implementemos nuestra fórmula para $S_k$ en PARI/GP:

\begin{verbatim}
S(k) = if (k == 0, x, 1/(k+1)*((x+1)^(k+1) - 1 - sum (i=0, k-1, binomial(k+1,i) * S(i))));

? S(3)
% = 1/4*x^4 + 1/2*x^3 + 1/4*x^2
\end{verbatim}

  \noindent El lector que conoce un poco de programación puede notar que
  el código de arriba es muy ineficaz; por ejemplo, para calcular \verb|S(20)|
  ya se necesita mucho tiempo. He aquí otra versión mucho más rápida:

\begin{verbatim}
/* La tabla de S (k): */
s_table = [];

S (k) = {
  if (k == 0, return (x));

  /* Extender la tabla de valores, de ser necesario: */
  if (length(s_table) < k, s_table = concat(s_table, vector(k-length(s_table))));

  /* Devolver el valor, si está en la tabla;
     sino, calcularlo y poner en la tabla: */	
  if (s_table[k], s_table[k],
      s_table[k] = 1/(k+1)*((x+1)^(k+1) - 1 - sum (i=0, k-1, binomial(k+1,i) * S(i))))
}
\end{verbatim}

  \noindent (Trate de calcular, por ejemplo, \verb|S(20)| usando ambas
  versiones.)
\end{shaded}

\begin{align*}
  S_0 (x) & = x,\\
  S_1 (x) & = \frac{1}{2}\,x^2\,+\highlight{\frac{1}{2}}\,x,\\
  S_2 (x) & = \frac{1}{3}\,x^3+\frac{1}{2}\,x^2+\highlight{\frac{1}{6}}\,x,\\
  S_3 (x) & = \frac{1}{4}\,x^4+\frac{1}{2}\,x^3+\frac{1}{4}\,x^2,\\
  S_4 (x) & = \frac{1}{5}\,x^5+\frac{1}{2}\,x^4+\frac{1}{3}\,x^3\highlight{-\frac{1}{30}}\,x,\\
  S_5 (x) & = \frac{1}{6}\,x^6+\frac{1}{2}\,x^5+\frac{5}{12}\,x^4-\frac{1}{12}\,x^2,\\
  S_6 (x) & = \frac{1}{7}\,x^7+\frac{1}{2}\,x^6+\frac{1}{2}\,x^5-\frac{1}{6}\,x^3+\highlight{\frac{1}{42}}\,x,\\
  S_7 (x) & = \frac{1}{8}\,x^8+\frac{1}{2}\,x^7+\frac{7}{12}\,x^6-\frac{7}{24}\,x^4+\frac{1}{12}\,x^2,\\
  S_8 (x) & = \frac{1}{9}\,x^9+\frac{1}{2}\,x^8+\frac{2}{3}\,x^7-\frac{7}{15}\,x^5+\frac{2}{9}\,x^3\highlight{-\frac{1}{30}}\,x,\\
  S_9 (x) & = \frac{1}{10}\,x^{10}+\frac{1}{2}\,x^9+\frac{3}{4}\,x^8-\frac{7}{10}\,x^6+\frac{1}{2}\,x^4-\frac{3}{20}\,x^2,\\
  S_{10} (x) & = \frac{1}{11}\,x^{11}+\frac{1}{2}\,x^{10}+\frac{5}{6}\,x^9-x^7+x^5-\frac{1}{2}\,x^3+\highlight{\frac{5}{66}}\,x.
\end{align*}

Las expresiones de arriba, también hasta $S_{10} (n)$, aparecen en la página 97
del libro de Bernoulli \cite{Bernoulli-1713}, publicado póstumamente en
1713. Luego Bernoulli escribe que, usando sus fórmulas, calculó en un
«semi-cuarto de hora» la suma
$$1^{10} + 2^{10} + \cdots + 1000^{10} =
S_{10} (1000) =
91\,409\,924\,241\,424\,243\,424\,241\,924\,242\,500.$$
Con ayuda de una computadora, se puede verificar que ¡el resultado es correcto!

\begin{shaded}
  \small
\begin{verbatim}
? { local(x); x = 1000; eval (S(10)) }
% = 91409924241424243424241924242500

? sum (i=1,1000,i^10)
% = 91409924241424243424241924242500
\end{verbatim}
\end{shaded}

\begin{definicion}
  El \term{$k$-ésimo número de Bernoulli $B_k$} es el coeficiente de $x$ en
  el polinomio $S_k (x)$. En otras palabras,
  $$B_k \dfn S_k' (0).$$
\end{definicion}

Euler leyó «Ars Conjectandi» y estudió los números $B_k$, llamándolos los
«números de Bernoulli», en el capítulo II.5 de su libro \cite{E212}.
Varias identidades para $B_k$ que aparecen en nuestro curso fueron descubiertas
por Euler. Por ejemplo, la derivada de \eqnref{formula-para-Skn} nos da
$$S_k' (x) = \frac{1}{k+1} \, \left((k+1)\,(x+1)^k - \sum_{0 \le i \le k-1} {k+1\choose i} \, S_i' (x)\right),$$
y para $x = 0$ tenemos
$$B_k = S_k' (0) = 1 - \frac{1}{k+1} \, \sum_{0 \le i \le k-1} {k+1\choose i} \, B_i.$$

\begin{proposicion}
  \label{definicion-recursiva-de-Bk}
  Para todo $k \ge 0$ se tiene
  $$\sum_{0 \le i \le k} {k+1 \choose i}\,B_i = k+1.$$
\end{proposicion}

Esto nos da una definición recursiva de los $B_k$:

\begin{align*}
  B_0 & = 1,\\
  B_0 + 2\,B_1 & = 2,\\
  B_0 + 3\,B_1 + 3\,B_2 & = 3,\\
  B_0 + 4\,B_1 + 6\,B_2 + 4\,B_3 & = 4,\\
      & ~~ \vdots
\end{align*}

A partir de estas identidades se pueden calcular sucesivamente
$B_1, B_2, B_3, B_4, \ldots$

\begin{shaded}
  \small
\begin{verbatim}
/* La tabla de B (k): */
b_table = [];

B (k) = {
  if (k == 0, return (1));

  if (length(b_table) < k, b_table = concat(b_table, vector(k-length(b_table))));
  if (b_table[k], b_table[k],
      b_table[k] = 1 - 1/(k+1)*sum (i=0, k-1, binomial(k+1,i)*B (i)))
}

? polcoeff (S(10),1,n)
% = 5/66
? B(10)
% = 5/66
\end{verbatim}
\end{shaded}

Luego los primeros números de Bernoulli son

{\def\arraystretch{1.25}
  \[ \begin{array}{rx{0.7cm}x{0.7cm}x{0.7cm}x{0.7cm}x{0.7cm}x{0.7cm}x{0.7cm}x{0.7cm}x{0.7cm}x{0.7cm}x{0.7cm}x{0.7cm}}
       k\colon & $0$ & $1$ & $2$ & $3$ & $4$ & $5$ & $6$ & $7$ & $8$ & $9$ & $10$ & $\cdots$ \tabularnewline
       \hline
       B_k\colon & $1$ & $\frac{1}{2}$ & $\frac{1}{6}$ & $0$ & $-\frac{1}{30}$ & $0$ & $\frac{1}{42}$ & $0$ & $-\frac{1}{30}$ & $0$ & $\frac{5}{66}$ & $\cdots$
     \end{array} \]}

\noindent (Bernoulli y Euler no usaban la notación $B_k$, sino que escribían
$A = \frac{1}{6}$, $B = -\frac{1}{30}$, $C = \frac{1}{42}$, $D = -\frac{1}{30}$,
etcétera.)

La derivada de \eqnref{eqn:Skn+1-y-Skn} es
$$S_k' (x+1) - S_k' (x) = k\,(x+1)^{k-1},$$
y la suma de estas identidades para $x = 0,1,2,\ldots,n-1$ nos da
$$S_k' (n) - S_k' (0) = k\,S_{k-1} (n).$$
Entonces, los polinomios $S_k (x)$ satisfacen la identidad
$$S_k' (x) = k\,S_{k-1} (x) + B_k.$$
Esto, junto con $S_1 (x) = x$, define completamente a todos los $S_k (x)$
(el término constante es nulo).

\begin{ejercicio}[La fórmula de Faulhaber]
  \label{ejercicio:suma-de-potencias-y-Bk}
  Demuestre la identidad
  $$S_k (n) = \frac{1}{k+1}\,\sum_{0 \le i \le k} {k+1\choose i}\,B_i\,n^{k+1-i}.$$
  \noindent Indicación: si $p (x) \in \QQ [x]$ es un polinomio de grado $d$,
  entonces
  $$p (x) = \sum_{0 \le i \le d} \frac{p^{(i)} (0)}{i!}\,x^i,$$
  donde $p^{(i)} (x)$ es la $i$-ésima derivada iterada de $p (x)$:
  $$p^{(0)} (x) \dfn p (x), ~
  p^{(1)} (x) \dfn p' (x), ~
  p^{(2)} (x) \dfn p'' (x), ~
  p^{(3)} (x) \dfn p''' (x), ~
  \ldots$$
\end{ejercicio}

Revisemos nuestra tabla de los primeros números de Bernoulli. Se observan dos patrones:

\begin{itemize}
\item $B_k = 0$ para $k \ge 3$ impar. Esto va a ser evidente en
  \S\ref{section:funcion-generatriz-para-Bk} a partir de otra representación de
  $B_k$ mediante una función generatriz par.

\item $B_k \ne 0$ para $k$ par, y los signos se alternan. Esto puede ser probado
  directamente de modo combinatorio (véase ejerciciosb
  \ref{ejercicio:identidad-de-Euler} y
  \ref{ejercicio:identidad-de-Euler-corolario}) o a partir de la fórmula
  $$(-1)^{k+1} \, B_{2k}\,\frac{2^{2k-1}}{(2k)!}\,\pi^{2k} =
  \zeta (2k) \dfn \sum_{n \ge 1} \frac{1}{n^{2k}} > 0$$
  que será explicada en \S\ref{section:valores-de-zeta2k}.
\end{itemize}

En cuanto a los valores de $B_k$, no se nota ningún patrón evidente.
Por ejemplo, en $B_{12} = -\frac{691}{2730}$ el numerador $691$ es un número
primo que aparentemente no tiene nada que ver con $12$. Sin embargo, si
factorizamos los \emph{denominadores} de $B_k$, se revelan números primos
relacionados con $k$ de alguna manera:

{\def\arraystretch{1.25}
\[ \begin{array}{rccccccccccccccc}
k\colon & 0 & 1 & 2 & 4 & 6 & 8 & 10 & 12 & 14 & 16 & 18 & 20 & \cdots \\
\hline
B_k\colon & 1 & \frac{1}{2} & \frac{1}{2\cdot 3} & -\frac{1}{2\cdot 3\cdot 5} & \frac{1}{2\cdot 3\cdot 7} & -\frac{1}{2\cdot 3\cdot 5} & \frac{5}{2\cdot 3\cdot 11} & -\frac{691}{2\cdot 3\cdot 5\cdot 7\cdot 13} & \frac{7}{2\cdot 3} & -\frac{3617}{2\cdot 3\cdot 5\cdot 17} & \frac{43867}{2\cdot 3\cdot 7\cdot 19} & -\frac{174611}{2\cdot 3\cdot 5\cdot 11} & \cdots
\end{array} \]
}

El lector puede formular su propia conjetura sobre los denominadores; más
adelante vamos a ver la respuesta en el \term{teorema de Clausen--von Staudt}
(\S\ref{section:teorema-de-von-Staudt}).

\begin{shaded}
  \small\noindent Por supuesto, PARI/GP ya sabe calcular los números de
  Bernoulli. La función \verb|bernfrac(k)| devuelve $B_k$:

\begin{verbatim}
? bernfrac(1)
% = -1/2
? bernfrac(10)
% = 5/66
\end{verbatim}

  \noindent Atención: PARI/GP (y también muchos libros de texto) usa otra
  definición de $B_k$ según la cual $B_1 = -\frac{1}{2}$
  (note el signo «$-$»). Para $k > 1$, los $B_k$ de PARI/GP son los mismos que
  los nuestros. La función \verb|bernreal(k)| calcula el valor aproximado de
  $B_k$:

\begin{verbatim}
? bernreal(4)   
% = -0.033333333333333333333333333333333333333
\end{verbatim}
\end{shaded}

% % % % % % % % % % % % % % % % % % % % % % % % % % % % % %

\section{Series formales de potencias}

Toda sucesión de números $a_k$ puede ser vista como los coeficientes de una
serie de potencias $\sum_k a_k\,t^k$. A veces esta serie surge como
la \term{serie de Taylor} de una función real o compleja $f$:
$$\sum_{k\ge 0} \frac{f^{(k)} (t_0)}{k!} \, (t-t_0)^k$$
(cuando las derivadas de $f$ en $t_0$ existen). Las funciones que pueden ser
representadas de tal manera se llaman \term{analíticas}. He aquí algunos
ejemplos de series de Taylor:

\begin{gather*}
  \text{\term{la serie geométrica} }\frac{1}{1-t} = \sum_{k \ge 0} t^k\quad\text{para }|t| < 1,\\
  e^t = \sum_{k\ge 0} \frac{t^k}{k!}, \quad\quad
  \ln (1+t) = \sum_{k \ge 1} (-1)^{k+1}\,\frac{t^k}{k}\quad\text{para }|t| < 1,\\
  \sen t = \sum_{k \ge 0} \frac{(-1)^k}{(2k+1)!}\,t^{2k+1}, \quad\quad
  \cos t = \sum_{k \ge 0} \frac{(-1)^k}{(2k)!}\,t^{2k}.
\end{gather*}

En general, la serie $\sum_k a_k\,t^k$ que corresponde a una sucesión arbitraria
$(a_k)$ no tiene por qué ser convergente, aunque sería útil manipular con
expresiones como «$\sum_k a_k\,t^k$» de manera puramente formal, como en
efecto hacían los matemáticos de la época de Euler, cuando todavía no había
una base rigurosa de análisis.

\begin{definicion}
  Sea $R$ un anillo conmutativo. Una \term{serie formal de potencias} en
  variable $t$ con coeficientes en $R$ es una expresión
  $$f (t) = \sum_{k \ge 0} a_k\,t^k = a_0 + a_1\,t + a_2\,t^2 + a_3\,t^3 + \cdots$$
  donde $a_k \in R$.
\end{definicion}

Las series formales se pueden manipular de la misma manera que los polinomios.
A saber, la suma de dos series se calcula término por término:

\begin{equation}
  \label{adicion-de-series-formales}
  \left(\sum_k a_k\,t^k\right) + \left(\sum_k b_k\,t^k\right) \dfn \sum_k (a_k+b_k)\,t^k.
\end{equation}

El producto de dos series se calcula mediante la distributividad formal:

\begin{align*}
  (a_0 + a_1\,t + a_2\,t^2 + a_3\,t^3 + \cdots)\cdot
  (b_0 + b_1\,t + b_2\,t^2 + b_3\,t^3 + \cdots) & = a_0\,b_0 + ~\\
                                                & \quad (a_0\,b_1 + a_1\,b_0)\,t + ~ \\
                                                & \quad (a_0\,b_2 + a_1\,b_1 + a_2\,b_0)\,t^2 + ~ \\
                                                & \quad (a_0\,b_3 + a_1\,b_2 + a_2\,b_1 + a_3\,b_0)\,t^3 + \cdots
\end{align*}
Es decir,
\begin{equation}
  \label{multiplicacion-de-series-formales}
  \left(\sum_k a_k\,t^k\right) \cdot \left(\sum_k b_k\,t^k\right) \dfn
  \sum_k \left(\sum_{i+j=k} a_i\,b_j\right)\,t^k.
\end{equation}

Se ve que las series formales respecto las operaciones de arriba forman un
anillo conmutativo, que vamos a denotar por $R [\![t]\!]$. Los polinomios
$R [t]$ forman un subanillo de $R [\![t]\!]$. En efecto, la adición y
multiplicación de polinomios están definidos mediante las mismas fórmulas
\eqnref{adicion-de-series-formales} y
\eqnref{multiplicacion-de-series-formales}, y todo polinomio
$a_n\,t^n + a_{n-1}\,t^{n-1} + \cdots a_1\,t + a_0$ puede ser visto como una
serie formal de potencias
$$a_0 + a_1\,t + \cdots + a_{n-1}\,t^{n-1} + a_n\,t^n + 0\,t^{n+1} + 0\,t^{n+2} + \cdots$$
En otras palabras, un polinomio es una serie formal donde casi todos los
coeficientes son nulos.

Un ejemplo muy importante de las series formales de potencias que nos va a
servir mucho es el siguiente.

\begin{definicion}
  Si $\QQ \subset R$, entonces la \term{función exponencial formal} es la serie
  en $R [\![t]\!]$ definida como
  $$e^t \dfn \sum_{k \ge 0} \frac{t^k}{k!}.$$
\end{definicion}

\vspace{1em}

\begin{observacion}
  Si $R$ es un dominio de integridad ($a\cdot b \ne 0$ para $a,b\ne 0$),
  entonces $R [\![t]\!]$ es también un dominio de integridad
  ($f (t) \cdot g (t) \ne 0$ para $f (t), g(t) \ne 0$).

  \begin{proof}
    Sean $f (t) = \sum_{k \ge 0} a_k\,t^k$ y $g (t) = \sum_{k \ge 0} b_k\,t^k$
    dos series de potencias no nulas. Sea $a_i$ el primer coeficiente no nulo de
    $f (t)$ y sea $b_j$ el primer coeficiente no nulo en $g (t)$. El coeficiente
    de $t^{i+j}$ en $f (t) \cdot g (t)$ es
    $$a_0\,b_{i+j} + a_1\,b_{i+j-1} + \cdots + a_i\,b_j + a_{i+1}\,b_{j-1} + \cdots + a_{i+j}\,b_0,$$
    donde por nuestra elección de $a_i$ y $b_j$ todos los términos son nulos
    excepto $a_i\,b_j$, que no es nulo porque $a_i \ne 0$, $b_j \ne 0$.
  \end{proof}
\end{observacion}

A partir de ahora vamos a suponer que $R$ es un dominio de integridad.

\begin{observacion}
  \label{obs:series-formales-invertibles}
  Una serie $f (t) = \sum_{k \ge 0} a_k\,t^k \in R [\![t]\!]$ es invertible
  si y solamente si $a_0 = \text{«}f (0)\text{»} \ne 0$ es invertible.
\end{observacion}

\noindent Note que, en general, sumas infinitas en $R$ no están definidas,
así que no se puede evaluar $f (t)$ en un elemento de $R$; es posible solo
en análisis, donde hay nociones de convergencia. Sin embargo, $f (0)$ sí tiene
sentido, y es el término constante de $f (t)$.

\begin{proof}
  Estamos buscando otra serie $g (t) = \sum_{k \ge 0} b_k\,t^k \in R [\![t]\!]$
  tal que $f (t) \cdot g (t) = 1$, es decir,
  \begin{align*}
    a_0\,b_0 & = 1,\\
    a_0\,b_1 + a_1\,b_0 & = 0,\\
    a_0\,b_2 + a_1\,b_1 + a_2\,b_0 & = 0,\\
             & ~~\vdots \\
    \sum_{0 \le i \le k} a_i\,b_{k-i} & = 0 \quad (k \ge 1)
  \end{align*}

  De la primera ecuación se ve que $a_0$ tiene que ser invertible. En este caso,
  podemos calcular $b_k$ sucesivamente:
  \begin{align*}
    b_0 & = a_0^{-1},\\
    b_1 & = -a_0^{-1}\,(a_1\,b_0),\\
    b_2 & = -a_0^{-1}\,(a_1\,b_1 + a_2\,b_0),\\
        & ~~\vdots \\
    b_k & = -a_0^{-1}\,\sum_{1 \le i \le k} a_i\,b_{k-i}. \qedhere
  \end{align*}
\end{proof}

\begin{ejemplo}
  Tenemos
  \begin{align*}
    (1-t)\cdot (1 + t + t^2 + t^3 + t^4 + \cdots) & =
                                                    (1 + t + t^2 + t^3 + t^4 + \cdots) -
                                                    (t + t^2 + t^3 + t^4 + t^5 + \cdots) = 1,\\
    (1+t)\cdot (1 - t + t^2 - t^3 + t^4 - \cdots) & =
                                                    (1 - t + t^2 - t^3 + t^4 - \cdots) +
                                                    (t - t^2 + t^3 - t^4 + t^5 - \cdots) = 1.
  \end{align*}

  Este es un análogo de la serie geométrica
  $\frac{1}{1-t} = \sum_{k \ge 0} t^k$, que en análisis tiene sentido para
  $|t| < 1$. En nuestro caso, $t$ es una variable formal.
\end{ejemplo}

Si $R = k$ es un cuerpo, una serie
$$f (t) = a_0 + a_1\,t + a_2\,t^2 + a_3\,t^3 + \cdots$$
tal que $a_0 = 0$ no es invertible en $k [\![t]\!]$. Para resolver este
problema, podemos introducir potencias negativas de $t$ y escribir
$$f (t) = t^{-n} \, (a_n + a_{n+1}\,t + a_{n+2}\,t^2 + a_{n+3}\,t^3 + \cdots),$$
donde $a_n$ es el primer coeficiente no nulo en $f (t)$. Aquí la serie entre
paréntesis es invertible en $k [\![t]\!]$. Para que tenga sentido el término
«$t^{-n}$», podemos introducir la siguiente generalización.

\begin{definicion}
  Una \term{serie formal de Laurent}\footnote{\href{http://www-history.mcs.st-andrews.ac.uk/Biographies/Laurent_Pierre.html}{\personality{Pierre Alphonse Laurent}} (1813--1854), un matemático y oficial militar francés.}
  es una serie formal con un número finito de potencias negativas:
  $$f(t) = \sum_{k \ge -N} a_k\,t^k\quad \text{para algún }N\in \NN.$$
\end{definicion}

Para las series de Laurent también tienen sentido adición y multiplicación,
definidas mediante las mismas fórmulas \eqnref{adicion-de-series-formales}
y \eqnref{multiplicacion-de-series-formales}, y toda serie puede ser vista como
una serie de Laurent con coeficientes negativos nulos. El anillo de las series
de Laurent se denota por $R (\!(t)\!)$.

\vspace{1em}

Tenemos las siguientes generalizaciones de los resultados de arriba:

\begin{enumerate}
\item[1)] Si $R$ es un dominio de integridad, entonces $R (\!(t)\!)$ es también
  un dominio de integridad (la demostración es la misma).

\item[2)] Una serie de Laurent $f (t) \in R (\!(t)\!)$ es invertible si y
  solamente si su primer coeficiente no nulo es invertible en $R$. En
  particular, si $R = k$ es un cuerpo, todas las series de Laurent no nulas son
  invertibles, y se ve que $k (\!(t)\!)$ es el cuerpo de fracciones de
  $k [\![t]\!]$.
\end{enumerate}

\begin{ejemplo}
  La serie $t + t^2 + t^3 + \cdots$ es invertible como serie de Laurent:
  \[ (t^{-1} - 1)\,(t + t^2 + t^3 + \cdots) =
    (1 + t + t^2 + \cdots) - (t + t^2 + t^3 + \cdots) = 1. \qedhere \]
\end{ejemplo}

\begin{shaded}
  \small\noindent PARI/GP puede trabajar con series de potencias. Para indicar
  que los términos de grado $\ge n$ están omitidos, se escribe
  «\verb|+ O(t^n)|»:

\begin{verbatim}
? 1/(1-t + O(t^10))
% = 1 + t + t^2 + t^3 + t^4 + t^5 + t^6 + t^7 + t^8 + t^9 + O(t^10)

? (t + 2*t^2 + 3*t^3 + 4*t^4 + 5*t^5 + O (t^6))^2
% = t^2 + 4*t^3 + 10*t^4 + 20*t^5 + 35*t^6 + O(t^7)
\end{verbatim}

  \noindent Series de Laurent:

\begin{verbatim}
? 1/(t + t^2 + t^3 + t^4 + t^5 + t^6 + O (t^7))
% = t^-1 - 1 + O(t^5)
\end{verbatim}

  \noindent PARI/GP conoce la exponencial formal:

\begin{verbatim}
? exp (t)
% = 1 + t + 1/2*t^2 + 1/6*t^3 + 1/24*t^4 + 1/120*t^5 + 1/720*t^6 +
     1/5040*t^7 + 1/40320*t^8 + 1/362880*t^9 + 1/3628800*t^10 +
     1/39916800*t^11 + 1/479001600*t^12 + 1/6227020800*t^13 +
     1/87178291200*t^14 + 1/1307674368000*t^15 + 1/20922789888000*t^16 +
     O(t^17)
\end{verbatim}

  \noindent El número de términos se puede cambiar con el parámetro \verb|seriesprecision|:

\begin{verbatim}
? default (seriesprecision, 6)
? exp (t)
% = 1 + t + 1/2*t^2 + 1/6*t^3 + 1/24*t^4 + 1/120*t^5 + 1/720*t^6 + O(t^7)
\end{verbatim}

  % \noindent Los números enteros $p$-ádicos $\ZZ_p$ pueden ser vistos como las
  % series formales $\sum_{k \ge 0} a_k\,p^k$ con
  % $a_k \in \{ 0, \ldots, p-1 \}$. He aquí un par de ejemplos:
  % 
%\begin{verbatim}
%? -1 + O(3^10)
%% = 2 + 2*3 + 2*3^2 + 2*3^3 + 2*3^4 + 2*3^5 + 2*3^6 + 2*3^7 + 2*3^8 + 2*3^9 + O(3^10)
%
%? (2 + O (3^10))^-1
%% = 2 + 3 + 3^2 + 3^3 + 3^4 + 3^5 + 3^6 + 3^7 + 3^8 + 3^9 + O(3^10)
%
%? sqrt (2+O(7^10))
%% = 3 + 7 + 2*7^2 + 6*7^3 + 7^4 + 2*7^5 + 7^6 + 2*7^7 + 4*7^8 + 6*7^9 + O(7^10)
%? %^2
%% = 2 + O(7^10)
%\end{verbatim}
%
% \noindent El primer ejemplo nos da la expresión canónica para $-1$ en el
% anillo $\ZZ_3$; en el segundo ejemplo PARI/GP calcula el número inverso a $2$
% en $\ZZ_3$. El último ejemplo calcula la raíz cuadrada $7$-ádica de $2$.
\end{shaded}

\begin{definicion}
  Dadas dos series de potencias $f (t) = \sum_{k \ge 0} a_k \, t^k$ y
  $g (t) = \sum_{k \ge 0} b_k \, t^k$, si $g (0) = b_0 = 0$, entonces
  la \term{composición $(f \circ g) (t)$} (\term{sustitución de $g$ en $f$}) es
  la serie
  $$(f \circ g) (t) \dfn f (g (t)) \dfn \sum_{k \ge 0} a_k \, g (t)^k.$$
  Ya que $b_0 = 0$, toda potencia $g (t)^k$ no tiene términos de grado $< k$,
  así que la suma infinita tiene sentido.
\end{definicion}

\begin{ejemplo}
  \label{serie-geometrica-formal}
  Si $f (t)$ es una serie formal tal que $f (0) = 0$, entonces

  \begin{align*}
    \frac{1}{1 - f (t)} & = 1 + f(t) + f (t)^2 + f(t)^3 + f (t)^4 + \cdots,\\
    \frac{1}{1 + f (t)} & = 1 - f(t) + f (t)^2 - f(t)^3 + f (t)^4 - \cdots
  \end{align*}
  ---es una generalización de la serie geométrica.
\end{ejemplo}

\begin{ejemplo}
  Podemos «evaluar» $e^t$ en $-t$. El resultado de la sustitución es la serie formal
  $$e^{-t} = \sum_{k \ge 0} (-1)^k\,\frac{t^k}{k!}.$$

  En general, podemos componer $e^t$ con toda $f (t)$ tal que $f (0) =
  0$. Tenemos la identidad habitual
  $$e^{f(t) + g(t)} = e^{f(t)}\cdot e^{g(t)}.$$
  En efecto,
  \begin{align*}
    e^{f(t)} \cdot e^{g(t)} & = \left(\sum_{i \ge 0} \frac{f(t)^i}{i!}\right) \cdot \left(\sum_{j \ge 0} \frac{g(t)^j}{j!}\right)\\
                            & = \sum_{k \ge 0} \sum_{i+j = k} \frac{k!}{k!} \frac{f(t)^i}{i!} \frac{g(t)^j}{j!} \\
                            & = \sum_{k \ge 0} \frac{1}{k!} \sum_{i \ge 0} {k \choose i} \, f(t)^i \, g(t)^{k-i} \\
                            & = \sum_{k \ge 0} \frac{(f(t)+g(t))^k}{k!} = e^{f(t)+g(t)}.
                              \qedhere
  \end{align*}
\end{ejemplo}

% % % % % % % % % % % % % % % % % % % % % % % % % % % % % %

\section{Derivadas formales}

\begin{definicion}
  La \term{derivada formal} de una serie formal de potencias
  $f (t) = \sum_{k \ge 0} a_k\,t^k \in R [\![t]\!]$ está definida por
  $$f' (t) \dfn \sum_{k \ge 1} k\,a_k\,t^{k-1}.$$
\end{definicion}

\begin{ejemplo}
  \[ (e^t)' = \left(\sum_{k \ge 0} \frac{t^k}{k!}\right)' =
    \sum_{k \ge 1} k\,\frac{t^{k-1}}{k!} =
    \sum_{k \ge 1} \frac{t^{k-1}}{(k-1)!} = e^t. \qedhere \]
\end{ejemplo}

\begin{observacion}[Serie de Taylor formal]
  \label{serie-de-Taylor-formal}
  Para las derivadas iteradas de
  $f (t) = \sum_{k \ge 0} a_k \, t^k \in R [\![t]\!]$ se tiene
  $f^{(k)} (0) = k! \, a_k$, lo que nos da
  $$f (t) = \sum_{k \ge 0} \frac{f^{(k)} (0)}{k!}\,t^k,$$
  cuando $\QQ \subset R$.

  \begin{proof}
    Se ve inmediatamente de las definiciones.
  \end{proof}
\end{observacion}

\begin{observacion}
  Para $f (t), g (t) \in R [\![t]\!]$ se tiene
  $$(f (t) + g (t))' = f' (t) + g' (t).$$

  \begin{proof}
    Evidente de la definición.
  \end{proof}
\end{observacion}

\begin{observacion}[Regla de Leibniz]
  Para $f (t), g (t) \in R [\![t]\!]$ se tiene
  $$(f (t)\cdot g (t))' = f' (t)\cdot g (t) + f (t)\cdot g' (t).$$

  \begin{proof}
    Para $f (t) = \sum_{k \ge 0} a_k\,t^k$ y $g (t) = \sum_{k \ge 0} b_k\,t^k$

    \begin{align*}
      \left( \left(\sum_{k\ge 0} a_k\,t^k\right)\cdot
      \left(\sum_{k\ge 0} b_k\,t^k\right) \right)' & = \left(\sum_{k\ge 0} \left(\sum_{0 \le i \le k} a_i\,b_{k-i}\right)\,t^k\right)' \\
                                                   & = \sum_{k\ge 1} k\,\left(\sum_{0 \le i \le k} a_i\,b_{k-i}\right)\,t^{k-1} \\
                                                   & = \sum_{k\ge 1} \left(\sum_{0 \le i \le k} i \, a_i\,b_{k-i} + \sum_{0 \le i \le k} (k-i) \, a_i\,b_{k-i}\right)\,t^{k-1} \\
                                                   & = \sum_{k \ge 1} \left(\sum_{0 \le i \le k} i \, a_i\,b_{k-i}\right)\,t^{k-1} + \sum_{k \ge 1} \left(\sum_{0 \le i \le k-1} a_i\,(k-i)\,b_{k-i}\right)\,t^{k-1} \\
                                                   & = f' (t)\cdot g (t) + f (t)\cdot g' (t).
                                                     \qedhere
    \end{align*}
  \end{proof}
\end{observacion}

\begin{ejercicio}
  Demuestre que para $f(t), g(t) \in R (\!(t)\!)$ se tiene
  $$\left(\frac{f(t)}{g(t)}\right)' = \frac{f' (t) \cdot g(t) - f (t)\cdot g' (t)}{g (t)^2}.$$
\end{ejercicio}

\begin{corolario}
  Para $f (t) \in R [\![t]\!]$ se tiene
  $$(f (t)^k)' = k\,f(t)'\,f(t)^{k-1}.$$

  \begin{proof}
    Por inducción, usando la regla de Leibniz.
  \end{proof}
\end{corolario}

\begin{observacion}[Regla de la cadena]
  \label{regla-de-la-cadena-formal}
  Sean $f (t), g (t) \in R [\![t]\!]$ dos series de potencias formales tales que
  $g (0) = 0$. Entonces para la composición se tiene
  $$(f (g (t)))' = f' (g (t)) \cdot g' (t).$$

  \begin{proof}
    Si $f (t) = \sum_{k\ge 0} a_k\,t^k$, entonces
    \[ (f (g (t)))' =
      \sum_{k\ge 1} k\,a_k\,g' (t)\, (g (t))^{k-1} =
      \left( \sum_{k \ge 1} k\,a_k\,(g (t))^{k-1} \right) \, g' (t) =
      f' (g (t))\cdot g' (t). \qedhere \]
  \end{proof}
\end{observacion}

\begin{shaded}
  \noindent\small En PARI/GP:

\begin{verbatim}
? default (seriesprecision, 6)

? deriv (t*exp(t), t)         
% = 1 + 2*t + 3/2*t^2 + 2/3*t^3 + 5/24*t^4 + 1/20*t^5 + 7/720*t^6 + O(t^7)
\end{verbatim}
\end{shaded}

Para resumir, las derivadas formales se comportan como las derivadas habituales:
son lineales, cumplen la regla de Leibniz y la regla de la cadena.

% % % % % % % % % % % % % % % % % % % % % % % % % % % % % %

\section{Logaritmo formal}

\begin{definicion}
  Si $\QQ \subset R$, el \term{logaritmo formal} es la serie en $R [\![t]\!]$
  definida por
  $$\ln (1 + t) \dfn \sum_{k\ge 1} (-1)^{k+1}\,\frac{t^k}{k}.$$
\end{definicion}

Observamos que la derivada formal de $\ln (1+t)$ es precisamente lo que se
espera del logaritmo:
$$(\ln (1+t))' = \frac{1}{1+t} = 1 - t + t^2 - t^3 + t^4 - t^5 + \cdots.$$

\begin{shaded}
  \noindent\small En PARI/GP:

\begin{verbatim}
? log (1+t)
% = t - 1/2*t^2 + 1/3*t^3 - 1/4*t^4 + 1/5*t^5 + O(t^6)
\end{verbatim}
\end{shaded}

\begin{teorema}
  \label{thm:exp-y-log-son-inversos}
  Tenemos
  $$\ln (1 + (e^t - 1)) = t, \quad e^{\ln (1 + t)} = 1 + t,$$
  en el sentido de sustitución de una serie formal en otra.
\end{teorema}

% \begin{ejercicio}
%   Calcule los primeros términos de las series $\ln (e^t)$ y $e^{\ln (1 + t)}$
%   con ayuda de PARI/GP.
% \end{ejercicio}

Las identidades de \ref{thm:exp-y-log-son-inversos} nos dan un ejemplo de series
inversas respecto a la composición:

\begin{proposicion}
  \label{prop:inversos-respecto-la-composicion}
  Para una serie de potencias formal $f (t) = \sum_{k \ge 0} a_k \, t^k$ existe
  otra serie $g (t)$ tal que $g (0) = 0$ y $f (g (t)) = t$ si y solamente si
  $a_0 = 0$ y $a_1$ es invertible. En este caso la serie $g (t)$ es única, y
  además se tiene $g (f (t)) = t$. Es decir, $f$ y $g$ son mutuamente inversas
  respecto a la composición.

  \begin{proof}
    La condición sobre $a_0$ y $a_1$ es necesaria: si existe
    $g (t) = \sum_{k\ge 0} b_k\,t^k$ con $b_0 = 0$ tal que
    $f (g (t)) = \sum_{k\ge 0} a_k\,g(t)^k = t$, entonces $a_0 = 0$ y
    $a_1\,b_1 = 1$.

    Ahora sea $f (t)$ una serie con $a_0 = 0$ y $a_1$ invertible. Tenemos que
    encontrar una serie $g (t) = \sum_{k \ge 0} b_k\,t^k$ con $b_0 = 0$ tal que
    $f (g (t)) = t$. La última identidad implica que necesitamos poner
    $b_1 \dfn a_1^{-1}$. Luego, para $k \ge 2$, el coeficiente de $t^k$ en
    $f (g (t))$ es igual al coeficiente de $t^k$ en la suma
    $$a_1 \, g (t) + a_2 \, g (t)^2 + \cdots + a_k \, g(t)^k$$
    (ya que $g (0) = 0$, en las potencias $g (t)^{k+1}, g (t)^{k+2}, \ldots$ ya
    no hay términos de grado $k$). Pero este coeficiente tiene que ser nulo, lo
    que nos da las ecuaciones
    $$a_1\,b_k + (\text{algún polinomio en }a_2, a_3, \ldots, a_k, b_1, b_2, \ldots, b_{k-1}) = 0.$$
    Puesto que $a_1 \ne 0$, estas ecuaciones por inducción definen
    \emph{de modo único} todos los coeficientes $b_2, b_3, b_4, \ldots$ Esto
    demuestra que $g (t)$ existe y es único.

    Para ver que también se tiene $g (f (t)) = t$, notamos que en $g (t)$
    también $b_0 = 0$ y $b_1$ es invertible, entonces existe $h (t)$ tal que
    $g (h (t)) = t$. Luego,
    \begin{gather*}
      t = f (g (t)),\\
      h (t) = f (g (h (t))) = f (t),\\
      g (h (t)) = g (f (t)) = t. \qedhere
    \end{gather*}
  \end{proof}
\end{proposicion}

\begin{shaded}
  \noindent\small En PARI/GP, la serie inversa respecto a la composición puede
  ser calculada por la función \verb|serreverse|:

\begin{verbatim}
? serreverse (exp (t) - 1)
% = t - 1/2*t^2 + 1/3*t^3 - 1/4*t^4 + 1/5*t^5 - 1/6*t^6 + O(t^7)
\end{verbatim}
\end{shaded}

\begin{proof}[Demostración de \ref{thm:exp-y-log-son-inversos}]
  La primera tentación es calcular directamente los coeficientes de las series
  $$\ln (1 + (e^t - 1)) \quad\text{y}\quad e^{\ln (1 + t)},$$
  pero esto no es tan fácil. Por ejemplo, las potencias de la serie $e^t - 1$
  tienen como coeficientes los números de Stirling:
  $$\frac{(e^t - 1)^\ell}{\ell!} = \sum_{k \ge \ell} \Stirling{k}{\ell}\,\frac{t^k}{k!};$$
  los vamos a necesitar de todas maneras más adelante y
  \S\ref{section:numeros-de-stirling} está dedicado a las definiciones y
  las propiedades básicas de $\Stirling{k}{\ell}$. Para el logaritmo también hay
  una fórmula parecida con otros números de Stirling:
  \[ \frac{\ln (1+t)^\ell}{\ell!} =
    (-1)^\ell \, \sum_{k \ge \ell} (-1)^k\,\stirling{k}{\ell}\,\frac{t^k}{k!}. \]
  Vamos a establecer estas identidades en \ref{exp-stirling} y
  \ref{log-stirling}. Afortunadamente, por el momento se puede evitar esta
  pesadilla combinatoria. Primero notemos que gracias a
  \ref{prop:inversos-respecto-la-composicion}, será suficiente demostrar que por
  ejemplo,
  $$e^{\ln (1+t)} = 1+t,$$
  y $\ln (1 + (e^t - 1)) = t$ se sigue automáticamente. Gracias a la serie de
  Taylor \ref{serie-de-Taylor-formal}, podemos simplemente verificar que
  \begin{align*}
    e^{\ln (1+0)} & = 1,\\
    (e^{\ln (1+t)})' (0) & = 1,\\
    (e^{\ln (1+t)})'' (0) & = 0,\\
    (e^{\ln (1+t)})''' (0) & = 0,\\
                  & \cdots
  \end{align*}

  En efecto, $\ln (1 + 0) = 0$ y $e^0 = 1$. Luego, por la regla de la cadena
  \ref{regla-de-la-cadena-formal},
  $$(e^{\ln (1+t)})' = e^{\ln (1+t)} \, \frac{1}{1+t},$$
  y así $(e^{\ln (1+t)})' (0) = 1$. La segunda derivada nos da
  \begin{align*}
    (e^{\ln (1+t)})'' & = \left(e^{\ln (1+t)} \, \frac{1}{1+t}\right)' \\
                      & = (e^{\ln (1+t)})'\,\frac{1}{1+t} - e^{\ln (1+t)}\,\frac{1}{(1+t)^2} \\
                      & = e^{\ln (1+t)} \, \frac{1}{1+t}\, \frac{1}{1+t} - e^{\ln (1+t)}\,\frac{1}{(1+t)^2} = 0.
                        \qedhere
  \end{align*}
\end{proof}

% % % % % % % % % % % % % % % % % % % % % % % % % % % % % %

\section{La función generatriz para $B_k$}
\label{section:funcion-generatriz-para-Bk}

\begin{teorema}
  Los números de Bernoulli pueden ser definidos por
  $$\frac{t\,e^t}{e^t - 1} = \sum_{k \ge 0} B_k \, \frac{t^k}{k!}.$$
\end{teorema}

Aunque se puede pensar en esta identidad como en la serie de Taylor para
$\frac{t\,e^t}{e^t - 1}$ en un entorno de $0$, para nosotros esto significa nada
más que el cociente de series formales $\frac{t\,e^t}{e^t - 1}$ en
$\QQ (\!(t)\!)$ es igual a la serie formal
$\sum_{k \ge 0} B_k \, \frac{t^k}{k!}$.

\begin{proof}
  Tenemos que ver que la identidad
  $$\left(\sum_{k \ge 0} B_k \, \frac{t^k}{k!}\right)\,(e^t-1) = t\,e^t.$$
  define los números de Bernoulli. Calculemos el producto al lado izquierdo:

  \begin{align*}
    \left(\sum_{k \ge 0} B_k \, \frac{t^k}{k!}\right)\,(e^t-1) & = 
                                                                 \left(\sum_{k \ge 0} B_k \, \frac{t^k}{k!}\right)\,\left(\sum_{k\ge 1} \frac{t^k}{k!}\right) \\
                                                               & = \sum_{k \ge 1} \left(\sum_{0 \le i \le k-1} \frac{B_i}{i!}\,\frac{1}{(k-i)!}\right)\,t^k \\
                                                               & = \sum_{k \ge 1} \left(\sum_{0 \le i \le k-1} \frac{B_i}{i!}\,\frac{k!}{(k-i)!}\right)\,\frac{t^k}{k!} \\
                                                               & = \sum_{k \ge 1} \left(\sum_{0 \le i \le k-1} {k \choose i} B_i\right)\,\frac{t^k}{k!} \\
                                                               & \stackrel{???}{=} \sum_{k\ge 1} \frac{t^k}{(k-1)!} = t\,e^t.
  \end{align*}

  La última igualdad se cumple si y solamente si
  $$\sum_{0 \le i \le k-1} {k \choose i} B_i = k.$$
  Como hemos visto en \ref{definicion-recursiva-de-Bk}, esta identidad define
  los números de Bernoulli.
\end{proof}

\begin{ejemplo}
  \label{calculo-masoquista}
  Calculemos algunos términos de la serie formal
  $\frac{t\,e^t}{e^t - 1}$. Tenemos
  $$e^t - 1 = t + \frac{t^2}{2} + \frac{t^3}{6} + \frac{t^4}{24} + \frac{t^5}{120} + \cdots =
  t\,\left(1 + \frac{t}{2!} + \frac{t^2}{3!} + \frac{t^3}{4!} + \frac{t^4}{5!} + \cdots\right).$$
  Luego,
  $$\frac{t}{e^t - 1} = \cfrac{1}{1 + \cfrac{t}{2!} + \cfrac{t^2}{3!} + \cfrac{t^3}{4!} + \cfrac{t^4}{5!} + \cdots}$$
  Podemos calcular la última serie usando nuestra observación en
  \ref{serie-geometrica-formal}. Tenemos

  \begin{multline*}
    1 -
    \left(\frac{t}{2!} + \frac{t^2}{3!} + \frac{t^3}{4!} + \frac{t^4}{5!} + \cdots\right) +
    \left(\frac{t}{2!} + \frac{t^2}{3!} + \frac{t^3}{4!} + \frac{t^4}{5!} + \cdots\right)^2\\
    - \left(\frac{t}{2!} + \frac{t^2}{3!} + \frac{t^3}{4!} + \frac{t^4}{5!} + \cdots\right)^3 +
    \left(\frac{t}{2!} + \frac{t^2}{3!} + \frac{t^3}{4!} + \frac{t^4}{5!} + \cdots\right)^4 - \cdots\\
    = 1 -
    \left(\frac{t}{2} + \frac{t^2}{6} + \frac{t^3}{24} + \frac{t^4}{120} + \cdots\right) +
    \left(\frac{t^2}{4} + \frac{t^3}{6} + \frac{5\,t^4}{72} + \cdots\right) - \left(\frac{t^3}{8} + \frac{t^4}{8} + \cdots\right) + \left(\frac{t^4}{16} + \cdots\right) - \cdots \\
    = 1 - \frac{t}{2} + \frac{t^2}{12} + 0\cdot t^3 - \frac{t^4}{720} + \cdots
  \end{multline*}

  Multiplicando las seres, se obtiene
  $$\frac{t\,e^t}{e^t - 1} =
  \left(1 + t + \frac{t^2}{2!} + \frac{t^3}{3!} + \frac{t^4}{4!} + \cdots\right) \cdot \left(1 - \frac{t}{2} + \frac{t^2}{12} + 0\cdot t^3 - \frac{t^4}{720} + \cdots\right) =
  1 + \frac{t}{2} + \frac{t^2}{12} - \frac{t^4}{720} + \cdots$$
  y entonces
  \[ B_0 = 1, \quad
    B_1 = \frac{1}{2}, \quad
    B_2 = 2! \cdot \frac{1}{12} = \frac{1}{6}, \quad
    B_3 = 0, \quad
    B_4 = -4!\cdot\frac{1}{720} = -\frac{1}{30}. \qedhere \]
\end{ejemplo}

\begin{shaded}
  \small\noindent Por supuesto, el último ejemplo es un poco masoquista: todo
  esto se puede hacer en PARI/GP.

\begin{verbatim}
? ser = (t*exp(t))/(exp(t)-1)
% = 1 + 1/2*t + 1/12*t^2 - 1/720*t^4 + 1/30240*t^6 - 1/1209600*t^8 + 1/47900160*t^10
- 691/1307674368000*t^12 + 1/74724249600*t^14 + O(t^16)

? vector (11,k, polcoeff(ser,(k-1),t)*(k-1)!)
% = [1, 1/2, 1/6, 0, -1/30, 0, 1/42, 0, -1/30, 0, 5/66]
\end{verbatim}
\end{shaded}

En muchos libros (y también en PARI/GP) se usa otra convención para los números
de Bernoulli según la cual $B_1 = -\frac{1}{2}$. En este caso la función
generatriz es $\frac{t\,e^t}{e^t - 1} - t = \frac{t}{e^t - 1}$.

\begin{ejemplo}
  La fórmula $\frac{t\,e^t}{e^t - 1} = \sum_{k \ge 0} \frac{B_k}{k!}\,t^k$ nos
  permite demostrar que $B_k = 0$ para $k \ge 3$ impar. En efecto, para ignorar
  el caso excepcional $B_1 = \frac{1}{2}$, examinemos la función
  $$f (t) \dfn \frac{t\,e^t}{e^t - 1} - \frac{t}{2} =
  B_0 + \frac{B_2}{2!}\,t^2 + \frac{B_3}{3!}\,t^3 + \frac{B_4}{4!}\,t^4 + \frac{B_5}{5!}\,t^5 + \cdots$$
  Tenemos
  $$f (t) = \frac{t\,e^t}{e^t - 1} - \frac{t}{2} =
  \frac{t\,(e^t - 1 + 1)}{e^t - 1} - \frac{t}{2} =
  \frac{t}{e^t - 1} + \frac{t}{2}.$$
  Luego,
  $$f (-t) = \frac{(-t)\,e^{-t}}{e^{-t}-1} - \frac{(-t)}{2} =
  \frac{t}{e^t - 1} + \frac{t}{2}.$$
  Entonces, $f (t) = f (-t)$, lo que implica que los coeficientes impares de
  $f (t)$ son nulos.
\end{ejemplo}

Los números de Bernoulli también surgen en otras series. Por ejemplo, tenemos la
siguiente

\begin{proposicion}
  \label{serie-para-t-cot-t}
  $$t\,\cot (t) = 1 + \sum_{k \ge 1} (-1)^k \, 2^{2k}\,\frac{B_{2k}}{(2k)!}\,t^{2k}.$$
\end{proposicion}

\noindent Esto ya tiene que ser interpretado analíticamente. Las series con
números de Bernoulli para varias funciones como $\tan (t)$, $\cot (t)$,
$\tanh (t)$, $\coth (t)$ fueron descubiertas por Euler.

\begin{proof}
  Se tiene
  $$\cos (t) = \frac{e^{it} + e^{-it}}{2}, \quad
  \sen (t) = \frac{e^{it} - e^{-it}}{2i}.$$
  Luego,
  \begin{align*}
    t\,\cot (t) & = t\,\frac{\cos (t)}{\sen (t)} =
                  it\,\frac{e^{it} + e^{-it}}{e^{it} - e^{-it}} =
                  it\,\frac{e^{2it} + 1}{e^{2it} - 1} =
                  -it + \frac{2it \, e^{2it}}{e^{2it} - 1}\\
                & = -it + \sum_{k \ge 0} \frac{B_k\cdot (2it)^k}{k!} =
                  1 + \sum_{k \ge 1} (-1)^k \, 2^{2k} \, \frac{B_{2k}}{(2k)!}\,t^{2k}.
                  \qedhere
  \end{align*}
\end{proof}

\begin{ejercicio}[Euler]
  \label{ejercicio:identidad-de-Euler}
  % "Bernoulli Numbers and Zeta Functions", 1.15
  Demuestre la identidad
  $$(2k + 1)\,B_{2k} = -\sum_{1 \le \ell \le k-1} {2k \choose 2\ell}\,B_{2\ell}\,B_{2\,(k-\ell)} \quad \text{para }k\ge 2.$$
  Por ejemplo, para $k = 3$ tenemos
  $$-7\,\underbrace{B_6}_{= \frac{1}{42}} =
  \underbrace{{6 \choose 2}\,B_2\,B_4}_{= 15\cdot \frac{1}{6}\cdot\left(-\frac{1}{30}\right)} + \underbrace{{6 \choose 4}\,B_4\,B_2}_{= 15\cdot\left(-\frac{1}{30}\right)\cdot \frac{1}{6}}.$$

  \vspace{1em}

  \noindent Indicación: considere la función generatriz para los números pares
  $f (t) \dfn \frac{t\,e^t}{e^t - 1} - \frac{t}{2} = \sum_{k \ge 0}
  \frac{B_{2k}}{(2k)!}\,t^{2k}$. Demuestre la identidad con la derivada formal
  $f (t) - t\,f(t)' = f(t)^2 - \frac{t^2}{4}$; sustituya $f (t)$ por
  $\sum_{k \ge 0} \frac{B_{2k}}{(2k)!}\,t^{2k}$ y compare los coeficientes de
  $t^{2k}$.
\end{ejercicio}

\begin{ejercicio}
  \label{ejercicio:identidad-de-Euler-corolario}
  % "Bernoulli Numbers and Zeta Functions", 1.16
  Demuestre por inducción que $(-1)^{k+1}\,B_{2k} > 0$ para todo
  $k \ge 1$. Véase \S\ref{section:valores-de-zeta2k} para la explicación.

  \noindent Indicación: use el ejercicio anterior.
\end{ejercicio}

% % % % % % % % % % % % % % % % % % % % % % % % % % % % % %

\section{Polinomios de Bernoulli}

Hay varios modos de definir los polinomios de Bernoulli; el más común es por una
función generatriz. Vamos a necesitar las series de potencias formales en dos
variables:
$$\sum_{k, \ell \ge 0} a_{k,\ell} \, t^k \, x^\ell,$$
respecto a la suma término por término y multiplicación que extiende
la multiplicación de polinomios en dos variables. Tenemos la serie formal
$\frac{t}{e^t - 1} \in \QQ [\![t]\!] \subset \QQ [\![t,x]\!]$ y podemos
multiplicarla por la serie
$$e^{tx} \dfn \sum_{k \ge 0} \frac{t^k\,x^k}{k!} \in \QQ [\![t,x]\!].$$
Un momento de reflexión demuestra que el resultado es de la forma

\begin{equation}
  \label{polinomios-de-Bernoulli-por-una-funcion-generatriz}
  \frac{t\,e^{tx}}{e^t-1} \dfn \sum_{k \ge 0} B_k (x)\,\frac{t^k}{k!},
\end{equation}

\noindent donde $B_k (x)$ son algunos \emph{polinomios} en $x$.

\begin{definicion}
  El \term{polinomio de Bernoulli} $B_k (x)$ es el polinomio definido por
  \eqnref{polinomios-de-Bernoulli-por-una-funcion-generatriz}.
\end{definicion}

\begin{ejemplo}
  Vamos a ver un poco más adelante cómo calcular los polinomios $B_k (x)$; por
  el momento podemos obtener algunos de los primeros. Como hemos calculado en
  \ref{calculo-masoquista},
  $$\frac{t}{e^t - 1} = 1 - \frac{t}{2} + \frac{t^2}{12} - \frac{t^4}{720} + \cdots$$
  Luego,
  $$\frac{t}{e^t - 1} \, e^{tx} =
  \left(1 - \frac{t}{2} + \frac{t^2}{12} - \cdots\right) \, \left(1 + t\,x + \frac{t^2\,x^2}{2} + \cdots\right) =
  1 + \left(x - \frac{1}{2}\right)\,t + \left(\frac{x^2}{2} - \frac{x}{2} + \frac{1}{12}\right)\,t^2 + \cdots$$
  de donde
  \[ B_0 (x) = 1, \quad
    B_1 (x) = x - \frac{1}{2}, \quad
    B_2 (x) = x^2 - x + \frac{1}{6}. \qedhere \]
\end{ejemplo}

\begin{observacion}
  Para todo $k\ge 0$,
  $$B_k (1) = B_k$$
  es el $k$-ésimo número de Bernoulli.

  \begin{proof}
    Comparando \eqnref{polinomios-de-Bernoulli-por-una-funcion-generatriz} con la
    función generatriz
    $\frac{t\,e^t}{e^t-1} = \sum_{k \ge 0} B_k\,\frac{t^k}{k!}$.
  \end{proof}
\end{observacion}

Resulta que el término constante de $B_k (x)$ es también igual a $B_k$:

\begin{observacion}
  Para todo $k \ge 0$

  \begin{equation}
    \label{Bk(x+1)-y-Bk(x)}
    B_k (x+1) - B_k (x) = k\,x^{k-1}.
  \end{equation}

  En particular, para $x = 0$ y $k \ne 1$ tenemos
  $$B_k (1) = B_k (0) = B_k.$$

  \begin{proof}
    Tenemos la identidad
    $$\frac{t\,e^{(x+1)\,t}}{e^t-1} - \frac{t\,e^{t\,x}}{e^t-1} = t\,e^{t\,x},$$
    de donde
    $$\sum_{k\ge 0} (B_k (x+1) - B_k (x))\,\frac{t^k}{k!} =
    \sum_{k \ge 0} \frac{x^k}{k!}\,t^{k+1}.$$
    Comparando los coeficientes de $t^k$, se obtiene \eqnref{Bk(x+1)-y-Bk(x)}.
  \end{proof}
\end{observacion}

Note que para $k = 1$ tenemos $B_1 (0) = -\frac{1}{2}$ y
$B_1 (1) = +\frac{1}{2}$.

\begin{observacion}
  Para todo $k \ge 0$
  $$B_k (1-x) = (-1)^k\,B_k (x).$$
\end{observacion}

\noindent (En particular, para $x = 0$ tenemos $B_k = (-1)^k\,B_k$ para
$k \ge 3$, lo que implica que $B_k = 0$ para $k \ge 3$ impar, como ya hemos
visto.)

\begin{proof}
  Usando funciones generatrices,
  \[ \sum_{k\ge 0} B_k (1-x)\,\frac{t^k}{k!} =
    \frac{t\,e^{(1-x)\,t}}{e^t - 1} =
    \frac{(-t)\,e^{x\,(-t)}}{e^{-t} - 1} =
    \sum_{k\ge 0} (-1)^k\,B_k (x)\,\frac{t^k}{k!}. \qedhere \]
\end{proof}

Los polinomios de Bernoulli pueden ser expresados en términos de los números de
Bernoulli:

\begin{proposicion}
  \label{proposicion:expresion-para-los-polinomios-de-Bernoulli}
  $$B_k (x) = \sum_{0 \le i \le k} (-1)^i \, {k \choose i}\,B_i\,x^{k-i}.$$

  \begin{proof}
    Calculemos el producto de series de potencias
    $$\frac{t}{e^t-1}\cdot e^{tx}.$$
    Tenemos
    $$\frac{t}{e^t-1} = \frac{(-t)\,e^{-t}}{e^{-t} - 1} =
    \sum_{k \ge 0} (-1)^k \, B_k \, \frac{t^k}{k!}, \quad
    e^{tx} = \sum_{k \ge 0} \frac{(tx)^k}{k!}.$$
    Luego,
    \begin{align*}
      \left(\sum_{k \ge 0} (-1)^k \, B_k \, \frac{t^k}{k!}\right) \cdot \left(\sum_{k \ge 0} \frac{(tx)^k}{k!}\right) & = \sum_{k \ge 0} \left(\sum_{0 \le i \le k} (-1)^i \, \frac{1}{i! \, (k-i)!} \, B_i\,x^{k-i}\right)\,t^k \\
                                                                                                                      & = \sum_{k \ge 0} \left(\sum_{0 \le i \le k} (-1)^i \, {k \choose i} \, B_i\,x^{k-i}\right)\,\frac{t^k}{k!}.
                                                                                                                        \qedhere
    \end{align*}
  \end{proof}
\end{proposicion}

\begin{proposicion}
  Para todo $k \ge 1$ se tiene
  $$B_k' (x) = k\,B_{k-1} (x), \quad \int_0^1 B_k (x)\,dx = 0.$$

  \begin{proof}
    Hay varios modos de verificar esto. Se puede usar la expresión de
    \ref{proposicion:expresion-para-los-polinomios-de-Bernoulli}. También
    podemos tomar las derivadas formales de la identidad
    $$\frac{t\,e^{tx}}{e^t - 1} = \sum_{k \ge 0} B_k (x) \, \frac{t^k}{k!}.$$
    Se obtiene
    $$\frac{\partial}{\partial x} \left(\frac{t\,e^{tx}}{e^t - 1}\right) =
    \frac{t\cdot t\,e^{tx}}{e^t - 1} =
    t\,\sum_{k \ge 0} B_k (x) \, \frac{t^k}{k!} =
    \sum_{k \ge 1} B_{k-1} (x) \, \frac{t^k}{(k-1)!} =
    \sum_{k \ge 0} B_k' (x) \, \frac{t^k}{k!}.$$

    Luego, para ver que $\int_0^1 B_k (x)\,dx = 0$, es suficiente observar que
    $\int B_k (x)\,dx = \frac{1}{k+1}\,B_{k+1} (x) + C$, donde
    $B_{k+1} (0) = B_{k+1} (1)$.
  \end{proof}
\end{proposicion}

Esto nos da otra definición de los polinomios de Bernoulli:

\begin{definicion-alternativa}
  Los polinomios $B_k (x)$ están definidos por
  $$B_0 (x) \dfn 1$$
  y
  $$B_k' (x) = k\,B_{k-1} (x), \quad
  \int_0^1 B_k (x) \, dx = 0 \quad
  \text{para }k \ge 1.$$
\end{definicion-alternativa}

\noindent (En efecto, la identidad $B_k' (x) = k\,B_{k-1} (x)$ define $B_k (x)$
salvo el término constante, pero el último se recupera de la condición
$\int_0^1 B_k (x) \, dx = 0$.) Recordemos que los polinomios $S_k (x)$ que hemos
estudiado en \S\ref{sumas-de-potencias} satisfacen la identidad
$$S_k' (x) = k\,S_{k-1} (x) + B_k.$$
Esto significa que las derivadas $S_k' (x)$ satisfacen la misma identidad que
$B_k (x)$:
$$S_k'' (x) = k\,S_{k-1}' (x).$$
Además, para $k\ne 1$ tenemos $B_k (0) = S_k' (0) \rdfn B_k$, y se ve que los
polinomios de Bernoulli son simplemente las derivadas de los polinomios
$S_k (x)$:
$$B_k (x) = S_k' (x), \quad \text{para }k\ne 1.$$
(El caso $k = 1$ es excepcional: $S_1 (x) = \frac{1}{2}\,x^2\,+\frac{1}{2}\,x$,
$B_1 (x) = x - \frac{1}{2}$.)

Ahora podemos compilar fácilmente una lista de los primeros polinomios de
Bernoulli:

\begin{align*}
  B_0 (x) & = \highlight{1}, \\
  B_1 (x) & = x - \frac{1}{2}, \\
  B_2 (x) & = x^2 - x + \highlight{\frac{1}{6}}, \\
  B_3 (x) & = x^3 - \frac{3}{2}\,x^2 + \frac{1}{2}\,x,\\
  B_4 (x) & = x^4 - 2\,x^3 + x^2 \highlight{- \frac{1}{30}}, \\
  B_5 (x) & = x^5 - \frac{5}{2}\,x^4 + \frac{5}{3}\,x^3 - \frac{1}{6}\,x, \\
  B_6 (x) & = x^6 - 3\,x^5 + \frac{5}{2}\,x^4 - \frac{1}{2}\,x^2 + \highlight{\frac{1}{42}},\\
  B_7 (x) & = x^7 - \frac{7}{2}\,x^6 + \frac{7}{2}\,x^5 - \frac{7}{6}\,x^3 + \frac{1}{6}\,x, \\
  B_8 (x) & = x^8 - 4\,x^7 + \frac{14}{3}\,x^6 - \frac{7}{3}\,x^4 + \frac{2}{3}\,x^2 \highlight{- \frac{1}{30}},\\
  B_9 (x) & = x^9 - \frac{9}{2}\,x^8 + 6\,x^7 - \frac{21}{5}\,x^5 + 2\,x^3 - \frac{3}{10}\,x, \\
  B_{10} (x) & = x^{10} - 5\,x^9 + \frac{15}{2}\,x^8 - 7\,x^6 + 5\,x^4 - \frac{3}{2}\,x^2 \highlight{+ \frac{5}{66}}.
\end{align*}

\vspace{\fill}

Podemos dibujar algunas gráficas para visualizar la relación
$B_k (1-x) = (-1)^k \, B_k (x)$:

\begin{center}
  \includegraphics{pic/bernpol.mps}
\end{center}

\pagebreak

\begin{shaded}
\small

\begin{verbatim}
? Bpoly (k) = sum (i=0,k, (-1)^i * binomial(k,i) * B(i) * x^(k-i));

? vector (10,k,Bpoly(k))  
% = [x - 1/2,
     x^2 - x + 1/6,
     x^3 - 3/2*x^2 + 1/2*x,
     x^4 - 2*x^3 + x^2 - 1/30,
     x^5 - 5/2*x^4 + 5/3*x^3 - 1/6*x,
     x^6 - 3*x^5 + 5/2*x^4 - 1/2*x^2 + 1/42,
     x^7 - 7/2*x^6 + 7/2*x^5 - 7/6*x^3 + 1/6*x,
     x^8 - 4*x^7 + 14/3*x^6 - 7/3*x^4 + 2/3*x^2 - 1/30,
     x^9 - 9/2*x^8 + 6*x^7 - 21/5*x^5 + 2*x^3 - 3/10*x,
     x^10 - 5*x^9 + 15/2*x^8 - 7*x^6 + 5*x^4 - 3/2*x^2 + 5/66]
\end{verbatim}

\begin{verbatim}
? deriv (Bpoly(10),x)
% = 10*x^9 - 45*x^8 + 60*x^7 - 42*x^5 + 20*x^3 - 3*x
? 10 * Bpoly(9)
% = 10*x^9 - 45*x^8 + 60*x^7 - 42*x^5 + 20*x^3 - 3*x
\end{verbatim}

\noindent Para comprobar los resultados, podemos directamente calcular la serie
$\frac{t\,e^{tx}}{e^t - 1}$:

\begin{verbatim}
? ser = t*exp (t*x) / (exp (t) - 1);
? polcoeff(ser,10,t)*10!
% = x^10 - 5*x^9 + 15/2*x^8 - 7*x^6 + 5*x^4 - 3/2*x^2 + 5/66
\end{verbatim}

\noindent También podemos calcular las derivadas de $S_k (x)$:

\begin{verbatim}
? deriv (S(10),x)
% = x^10 + 5*x^9 + 15/2*x^8 - 7*x^6 + 5*x^4 - 3/2*x^2 + 5/66
\end{verbatim}

\noindent En PARI/GP, la función predefinida \verb|bernpol(k)| devuelve el
polinomio de Bernoulli $B_k (x)$:

\begin{verbatim}
? bernpol(1)
% = x - 1/2
? bernpol(2)
% = x^2 - x + 1/6
? bernpol(3)
% = x^3 - 3/2*x^2 + 1/2*x
\end{verbatim}
\end{shaded}

% % % % % % % % % % % % % % % % % % % % % % % % % % % % % %

\pagebreak
\section{La función zeta de Riemann}

\begin{definicion}
  La \term{función zeta de Riemann} está definida por la serie infinita
  $$\zeta (s) \dfn \sum_{n \ge 1} \frac{1}{n^s} =
  1 + \frac{1}{2^s} + \frac{1}{3^s} + \frac{1}{4^s} + \frac{1}{5^s} + \frac{1}{6^s} + \cdots$$
\end{definicion}

\begin{observacion}
  La serie de arriba es absolutamente convergente para todo $s \in \CC$ tal que
  $\Re s > 1$.

  \begin{proof}
    Si $s = a + i\,b$, tenemos
    $$\left|\frac{1}{n^s}\right| = \frac{1}{n^a}.$$
    Podemos usar el \term{criterio integral de convergencia}:
    $\sum_{n \ge 1} \frac{1}{n^a}$ es convergente si y solamente si
    $$\int_1^\infty \frac{1}{x^a}\,dx < \infty.$$
    En efecto, tenemos
    $$\int_1^\infty \frac{1}{x^a}\,dx =
    \lim_{n \to \infty} \left[\frac{x^{1-a}}{1-a}\right]^n_1 =
    \lim_{n \to \infty} \left(\frac{n^{1-a}}{1-a} - \frac{1}{1-a}\right).$$
    Este límite existe precisamente cuando $a > 1$.
  \end{proof}
\end{observacion}

Note que para $s = 1$ se obtiene la \term{serie armónica}
$$\zeta (1) = 1 + \frac{1}{2} + \frac{1}{3} + \frac{1}{4} + \cdots$$
que es divergente.

\begin{proof}[Demostración (Nicolás Oresme, siglo XIV)]
  En la serie
  $$1 + \frac{1}{2} + \frac{1}{3} + \frac{1}{4} +
  \frac{1}{5} + \frac{1}{6} + \frac{1}{7} + \frac{1}{8} +
  \frac{1}{9} + \frac{1}{10} + \frac{1}{11} + \frac{1}{12} +
  \frac{1}{13} + \frac{1}{14} + \frac{1}{15} + \frac{1}{16} + \cdots$$
  reemplacemos cada término $\frac{1}{n}$ por el número máximo
  $\frac{1}{2^k} \le \frac{1}{n}$. Se obtiene una serie
  $$1 + \frac{1}{2} + \underbrace{\left(\frac{1}{4} + \frac{1}{4}\right)}_{= \frac{1}{2}} +
  \underbrace{\left(\frac{1}{8} + \frac{1}{8} + \frac{1}{8} + \frac{1}{8}\right)}_{= \frac{1}{2}} +
  \underbrace{\left(\frac{1}{16} + \frac{1}{16} + \frac{1}{16} + \frac{1}{16} + \frac{1}{16} + \frac{1}{16} + \frac{1}{16} + \frac{1}{16}\right)}_{= \frac{1}{2}} + \cdots$$
  que es obviamente divergente. Por tanto la serie armónica es también
  divergente.
\end{proof}

\pagebreak

Para $s > 1$ la función $\zeta (s)$ es monótonamente decreciente, y se tiene
$\lim_{s\to \infty} \zeta (s) = 1$:

\begin{center}
  \includegraphics{pic/zeta-pos.mps}
\end{center}

\begin{teorema}[Fórmula del producto de Euler]
  $$\sum_{n \ge 1} \frac{1}{n^s} = \prod_{p\text{ primo}} \frac{1}{1 - p^{-s}}.$$
\end{teorema}

La fórmula de arriba tiene una gran importancia en la teoría de números y fue
descubierta por Euler. He aquí la demostración original, reproducida de su
artículo \cite{E72}:

\begin{quote}
  Si

  \begin{equation}
    \label{euler-1}
    x = 1 + \frac{1}{2^s} + \frac{1}{3^s} + \frac{1}{4^s} + \frac{1}{5^s} + \frac{1}{6^s} + \cdots,
  \end{equation}
  entonces
  \begin{equation}
    \label{euler-2}
    \frac{1}{2^s}\,x = \frac{1}{2^s} + \frac{1}{4^s} + \frac{1}{6^s} + \frac{1}{8^s} + \frac{1}{10^s} + \frac{1}{12^s} + \cdots,
  \end{equation}
  y subtraendo $\text{\eqnref{euler-1}}-\text{\eqnref{euler-2}}$ se obtiene
  \begin{equation}
    \label{euler-3}
    \frac{2^s - 1}{2^s}\,x = 1 + \frac{1}{3^s} + \frac{1}{5^s} + \frac{1}{7^s} + \frac{1}{9^s} + \frac{1}{11^s} + \cdots
  \end{equation}
  Luego,
  \begin{equation}
    \label{euler-4}
    \left(\frac{2^s - 1}{2^s}\right)\,\frac{1}{3^s}\,x = \frac{1}{3^s} + \frac{1}{9^s} + \frac{1}{15^s} + \frac{1}{21^s} + \frac{1}{27^s} + \cdots
  \end{equation}
  y $\text{\eqnref{euler-3}} - \text{\eqnref{euler-4}}$ nos da
  \[ \left(\frac{2^s - 1}{2^s}\right)\,\left(\frac{3^s - 1}{3^s}\right)\,x = 1 + \frac{1}{5^s} + \frac{1}{7^s} + \frac{1}{11^s} + \cdots \]

  Después de aplicar operaciones similares para todos los números primos, todos
  los términos excepto el primero se eliminan:
  $$1 = \left(\frac{2^s - 1}{2^s}\right)\,\left(\frac{3^s - 1}{3^s}\right)\,\left(\frac{5^s - 1}{5^s}\right)\,\left(\frac{7^s - 1}{7^s}\right)\,\left(\frac{11^s - 1}{11^s}\right)\cdots x,$$
  de donde se encuentra la serie $x$:
  $$\left(\frac{2^s}{2^s - 1}\right)\,\left(\frac{3^s}{3^s - 1}\right)\,\left(\frac{5^s}{5^s - 1}\right)\,\left(\frac{7^s}{7^s - 1}\right)\,\left(\frac{11^s}{11^s - 1}\right)\cdots = x = 1 + \frac{1}{2^s} + \frac{1}{3^s} + \frac{1}{4^s} + \frac{1}{5^s} + \frac{1}{6^s} + \cdots$$
  Q.E.D.
\end{quote}

Dejo al lector pensar por qué esta demostración es esencialmente correcta.

El siguiente resultado necesita más análisis de lo que conocía Euler y no
lo vamos a demostrar (véase \cite[Chapter 5, \S 4]{Ahlfors-1978} o cualquier
libro de la teoría de números).

\begin{hecho}
  La función $\zeta (s)$ tiene prolongación analítica al plano complejo como una
  función meromorfa que tiene un polo simple en $s = 1$ de residuo $1$. Esta
  prolongación analítica, que también se denota por $\zeta (s)$, satisface
  la \term{ecuación funcional}

  \begin{equation}
    \label{ecuacion-funcional}
    \zeta (s) = 2^s \, \pi^{s-1} \, \sen \left(\frac{\pi s}{2}\right)\,\Gamma (1-s)\,\zeta (1-s).
  \end{equation}
\end{hecho}

Aquí $\Gamma (z) \dfn \int_0^\infty x^{z-1}\,e^{-x}\,dx$ denota la
\term{función gamma}. En particular, $\Gamma (n) = (n-1)!$ para
$n = 1,2,3,4,\ldots$

% % % % % % % % % % % % % % % % % % % % % % % % % % % % % %

\section{Los valores $\zeta (2k)$}
\label{section:valores-de-zeta2k}

El siguiente resultado fue descubierto por Euler y aparece en sus artículos
\cite{E41}, \cite{E352}, etc.:

\begin{teorema}
  \label{thm:valores-de-zeta-2k}
  Para todo $k\ge 1$
  $$\zeta (2k) \dfn 1 + \frac{1}{2^{2k}} + \frac{1}{3^{2k}} + \frac{1}{4^{2k}} + \cdots =
  (-1)^{k+1} \, B_{2k}\,\frac{2^{2k-1}}{(2k)!}\,\pi^{2k}.$$
\end{teorema}

Es algo sorprendente: ¡los números de Bernoulli surgen del estudio de las sumas
de potencias $\sum_{1 \le i \le n} i^k$, y ahora los mismos números aparecen en
sumas de potencias infinitas! Los primeros valores de $\zeta (2k)$ son entonces

\begin{align*}
  \zeta (2) & = \frac{\pi^2}{6} \approx 1.644934\ldots, \\
  \zeta (4) & = \frac{\pi^4}{90} \approx 1.082323\ldots, \\
  \zeta (6) & = \frac{\pi^6}{945}  \approx 1.017343\ldots, \\
  \zeta (8) & = \frac{\pi^8}{9450} \approx 1.004077\ldots, \\
  \zeta (10) & = \frac{\pi^{10}}{93\,555} \approx 1.000994\ldots, \\
  \zeta (12) & = \frac{691\,\pi^{12}}{638\,512\,875} \approx 1.000246\ldots
\end{align*}

En particular, el cálculo de
$\zeta (2) = 1 + \frac{1}{4} + \frac{1}{9} + \frac{1}{16} + \cdots$ se conoce
como el \term{problema de Basilea} que fue formulado por el matemático italiano
\personality{Pietro Mengoli} (1626--1686) en 1644. La primera solución fue
encontrada por Euler en 1735.

\begin{ejercicio}
  Calcule las sumas parciales $\sum_{1 \le n \le N} \frac{1}{n^2}$ en
  PARI/GP. Note que su convergencia a $\zeta (2)$ es bastante lenta. Esto
  explica un siglo de sufrimiento de los matemáticos que trataban de obtener un
  valor aproximado de $\zeta (2)$... hasta la llegada de Euler.
\end{ejercicio}

\begin{corolario}
  $(-1)^{k+1}\,B_{2k} > 0$ para todo $k \ge 1$. Es decir, $B_{2k} \ne 0$ y los
  signos de los números de Bernoulli pares se alternan.

\begin{proof}
  \[ (-1)^{k+1}\, B_{2k} = \frac{(2k)! \, \zeta (2k)}{2^{2k-1}\,\pi^{2k}}. \qedhere \]
\end{proof}
\end{corolario}

También se ve que $|B_{2k}| \xrightarrow{k \to \infty} \infty$, y que $|B_{2k}|$
crece muy rápido con $k$:

\begin{align*}
  B_2 & \approx +0.166667,\\
  B_4 & \approx -0.033333,\\
  B_6 & \approx +0.023810,\\
  B_8 & \approx -0.033333,\\
  B_{10} & \approx +0.075758,\\
  B_{12} & \approx -0.253114,\\
  B_{14} & \approx +1.166667,\\
  B_{16} & \approx -7.092157,\\
  B_{18} & \approx +54.971178,\\
  B_{20} & \approx -529.124242.
\end{align*}

Hay muchas demostraciones del teorema \ref{thm:valores-de-zeta-2k}; se puede
encontrar una colección en la página

\begin{center}
  \noindent\url{http://empslocal.ex.ac.uk/people/staff/rjchapma/etc/zeta2.pdf}
\end{center}

\begin{proof}[Primera demostración de \ref{thm:valores-de-zeta-2k}]
  Hemos visto en \ref{serie-para-t-cot-t} la serie
  \begin{equation}
    \label{eqn:serie-para-t-cot-t}
    t\,\cot (t) = 1 + \sum_{k \ge 1} (-1)^k \, 2^{2k}\,\frac{B_{2k}}{(2k)!}\,t^{2k}.
  \end{equation}

  En el análisis complejo se deduce otra serie
  \cite[Chapter 5, \S 2]{Ahlfors-1978}
  $$\cot (t) = \sum_{n\in \ZZ} \frac{1}{t - \pi n},$$
  que corresponde a la «descomposición en fracciones simples» de una función
  meromorfa: $\cot (t)$ tiene polos simples en $t = \pi n$ para todo $n \in \ZZ$
  con residuo
  $$\lim_{t \to \pi n} (t - \pi n)\,\cot (t) =
  \lim_{t \to 0} \cos (t+\pi n) \, \frac{t}{\sen (t + \pi n)} =
  \lim_{t \to 0} (-1)^n\,\cos (t) \, \frac{t}{(-1)^n\,\sen (t)} = 1.$$
  Por «$\sum_{n\in \ZZ} \frac{1}{t - \pi n}$» se entiende
  $\lim_{N\to \infty} \sum_{-N \le n \le N} \frac{1}{t - \pi n}$.
  Luego,
  \begin{align*}
    t\,\cot (t) & = t\,\left(\frac{1}{t} + \sum_{n \ge 1} \left(\frac{1}{t - \pi n} + \frac{1}{t + \pi n}\right)\right) = 1 - 2\,\sum_{n \ge 1} \left(\frac{t^2}{(\pi n)^2 - t^2}\right) = 1 - 2\,\sum_{n \ge 1} \frac{t^2}{(\pi n)^2} \, \frac{1}{1-\left(\frac{t}{\pi n}\right)^2} \\
                & = 1 - 2\,\sum_{n \ge 1} \frac{t^2}{(\pi n)^2} \, \sum_{k \ge 0} \left(\frac{t}{\pi n}\right)^{2k} = 1 - 2\,\sum_{n \ge 1} \, \sum_{k \ge 1} \left(\frac{t}{\pi n}\right)^{2k} \quad \text{(la serie geométrica)}\\
                & = 1 - 2\,\sum_{k \ge 1} \left(\sum_{n\ge 1} \frac{1}{n^{2k}}\right)\,\frac{t^{2k}}{\pi^{2k}} = 1 - 2\,\sum_{k\ge 1} \frac{\zeta (2k)\,t^{2k}}{\pi^{2k}}. \quad \text{(cambiando el orden de sumación)}
  \end{align*}
  Comparando coeficientes con \eqnref{eqn:serie-para-t-cot-t}, tenemos
  \[ (-1)^k \, 2^{2k}\,\frac{B_{2k}}{(2k)!} = -2\,\frac{\zeta (2k)}{\pi^{2k}}. \qedhere \]
\end{proof}

% % % % % % % % % % % % % % % % % % % % % % % % % % % % % %

\section{Series de Fourier para $B_k (x)$}

Vamos a necesitar el siguiente resultado del análisis armónico:

\begin{hecho}
  Sea $f\colon \RR \to \RR$ una función continua por trozos y periódica:
  $$f (x+1) = f (x).$$
  Entonces para todo $x_0\in \RR$ donde $f$ es continua y la derivada izquierda
  y derecha de $f$ existen (pero no necesariamente coinciden) se tiene
  \[ f (x_0) = \sum_{n\in\ZZ} \widehat{f} (n) \, e^{2\pi i n x_0}, \quad
    \text{donde }\widehat{f} (n) \dfn \int_0^1 e^{-2\pi i n x} \, f(x) \, dx. \]
\end{hecho}

Es un caso especial de las \term{series de Fourier} que va a ser suficiente para
nosotros; para la teoría general véase por ejemplo \cite{Katznelson-2004}.

\vspace{1em}

En nuestro caso, nos interesan las funciones
$$f (x) \dfn B_k (x - \lfloor x\rfloor),$$
donde $B_k (x)$ es el $k$-ésimo polinomio de Bernoulli. Para $k > 1$ la función
$B_k (x - \lfloor x\rfloor)$ es continua y para $k = 1$ es discontinua en
los puntos $x = n \in \ZZ$. También $B_k (x - \lfloor x\rfloor)$ es lisa para
$k > 2$, pero $B_2 (x)$ no es lisa en los puntos $x = n \in \ZZ$, donde existen
la derivada izquierda y derecha, pero son diferentes.

\begin{center}
  \includegraphics{pic/bernpol-per.mps}
\end{center}

Los coeficientes de la serie de Fourier para $f (x)$ se calculan
fácilmente. Para $n = 0$ tenemos
$$\widehat{f} (0) = \int_0^1 B_k (x)\,dx = 0.$$
Luego, para $n \ne 0$ y $k = 1$ podemos usar integración por partes
($\int_a^b f'(x)\,g(x)\,dx = \left[f(x)\,g(x)\right]_a^b - \int_a^b f(x) \, g'(x)\,dx$):

\begin{align*}
  \int_0^1 e^{-2\pi i n x}\,\left(x - \frac{1}{2}\right)\,dx & = -\frac{1}{2\pi i n} \int_0^1 \left(e^{-2\pi i n x}\right)' \, \left(x - \frac{1}{2}\right)\,dx \\
                                                             & = -\frac{1}{2\pi i n} \, \left( \left[e^{-2\pi i n x}\,\left(x - \frac{1}{2}\right)\right]^1_0 - \underbrace{\int_0^1 e^{-2\pi i n x}\,dx}_{=0} \right) = -\frac{1}{2\pi i n}.
\end{align*}

Para $k > 1$ integración por partes y la relación $B_k' (x) = k\,B_{k-1} (x)$
nos dan

\begin{align*}
  \widehat{f} (n) = \int_0^1 e^{-2\pi i n x} \, B_k (x)\,dx & = -\frac{1}{2\pi i n}\,\int_0^1 (e^{-2\pi i n x})' \, B_k (x)\,dx \\
                                                            & = -\frac{1}{2\pi i n}\,\left(\left[e^{-2\pi i n x}\,B_k (x)\right]_0^1 - k\,\int_0^1 e^{-2\pi i n x} \, B_{k-1} (x)\,dx\right) \\
                                                            & = \frac{k}{2\pi i n} \, \int_0^1 e^{-2\pi i n x}\,B_{k-1} (x)\,dx \\
                                                            & = \frac{k\,(k-1)}{(2\pi i n)^2} \int_0^1 e^{-2\pi i n x}\,B_{k-2} (x)\,dx \\
                                                            & = \cdots \\
                                                            & = \frac{k!}{(2\pi i n)^{k-1}} \int_0^1 e^{-2\pi i n x}\,\left(x - \frac{1}{2}\right)\,dx \\
                                                            & = \frac{k!}{(2\pi i n)^{k-1}} \cdot \left(-\frac{1}{2\pi i n}\right)
                                                              = -\frac{k!}{(2\pi i n)^k}.
\end{align*}

Entonces, la serie de Fourier es

\begin{equation}
  \label{serie-de-forier-para-Bk}
  B_k (x - \lfloor x\rfloor) = -\frac{k!}{(2\pi i)^k}\sum_{\substack{n\in \ZZ \\ n \ne 0}} \frac{e^{2\pi i n x}}{n^k}.
\end{equation}

Como un caso especial, se obtiene la fórmula para $\zeta (2k)$:

\begin{proof}[Segunda demostración de \ref{thm:valores-de-zeta-2k}]
  Para $x = 0$ la identidad \eqnref{serie-de-forier-para-Bk} nos da
  \[ B_{2k} = B_{2k} (0) =
    -\frac{(2k)!}{(-1)^k\,(2\pi)^{2k}} \, 2\,\sum_{n \ge 1} \frac{1}{n^{2k}} =
    (-1)^{k+1}\frac{(2k)!}{2^{2k-1}\,\pi^{2k}}\,\zeta (2k). \qedhere \]
\end{proof}

Note que los valores en los enteros impares $\zeta (2k+1)$ no se obtienen con este método.

% % % % % % % % % % % % % % % % % % % % % % % % % % % % % %

\section{Los valores de $\zeta (-1)$, $\zeta (-2)$, $\zeta (-3)$, $\ldots$}

Gracias a la ecuación funcional
$$\zeta (s) = 2^s \, \pi^{s-1} \, \sen \left(\frac{\pi s}{2}\right)\,\Gamma (1-s)\,\zeta (1-s)$$
y la fórmula de Euler
$$\zeta (2k) = (-1)^{k+1} \, B_{2k}\,\frac{2^{2k-1}}{(2k)!}\,\pi^{2k},$$
podemos obtener los valores en los enteros negativos. En efecto, para los
enteros negativos pares $s = -2k$ tenemos
$$\zeta (-2k) = 2^{-2k} \, \pi^{-2k-1} \, \underbrace{\sen \left(-\frac{\pi k}{2}\right)}_{= 0}\,\Gamma (2k+1)\,\zeta (2k+1) = 0.$$
Y para $s = -(2k+1)$ impares,

\begin{align*}
  \zeta (-(2k+1)) & = 2^{-(2k+1)} \, \pi^{-(2k+2)} \, \sen \left(-\frac{\pi\,(2k+1)}{2}\right)\,(2k+1)!\,\zeta (2k+2) \\
                  & = 2^{-(2k+1)} \, \pi^{-(2k+2)}\,(-1)^{k+1}\,(2k+1)!\,(-1)^k \, B_{2k+2}\,\frac{2^{2k+1}}{(2k+2)!}\,\pi^{2k+2} \\
                  & = -\frac{B_{2k+2}}{2k+2}.
\end{align*}

Ya que $B_n = 0$ para $n$ impar, en ambos casos se tiene
$$\zeta (-n) = -\frac{B_{n+1}}{n+1}.$$
Además, para $n = 0$ la prolongación analítica nos da
$\zeta (0) = -\frac{1}{2} = -B_1$, así que esta fórmula es valida también para
$n = 0$.

\[ \begin{array}{rcccccccccccc}
     n\colon & 0 & -1 & -2 & -3 & -4 & -5 & -6 & -7 & -8 & -9 & -10 & \cdots \\
     \hline
     \zeta(n)\colon & -\frac{1}{2} & -\frac{1}{12} & 0 & \frac{1}{120} & 0 & -\frac{1}{252} & 0 & \frac{1}{240} & 0 & -\frac{1}{132} & 0 & \cdots
   \end{array} \]

\begin{center}
  \noindent\includegraphics{pic/zeta.mps}
\end{center}

\noindent (Después $\zeta (s)$ es decreciente hasta su polo en $s = 1$.)

\vspace{1em}

Terminamos por el cálculo de $\zeta (-1) = -\frac{1}{12}$ encontrado por Euler
en su artículo \cite{E247}:

\begin{quote}
  Para la serie geométrica
  $$1 + x + x^2 + x^3 + x^4 + x^5 + x^6 + x^7 + x^8 + \cdots = \frac{1}{1-x}$$
  la derivada formal nos da
  $$1 + 2\,x + 3\,x^2 + 4\,x^3 + 5\,x^4 + 6\,x^5 + 7\,x^6 + 8\,x^7 + \cdots = \frac{1}{(1-x)^2},$$
  de donde para $x = -1$ (¡sic!)
  $$1 - 2 + 3 - 4 + 5 - 6 + 7 - 8 + \cdots = \frac{1}{4}.$$
  Luego,
  \begin{align*}
    -3\,\zeta (-1) & = \zeta (-1) - 4\,\zeta (-1) \\
                   & = (1 + 2 + 3 + 4 + \cdots) - (4 + 8 + 12 + 16 + \cdots) \\
                   & = 1 - 2 + 3 - 4 + 5 - 6 + 7 - 8 + \cdots = \frac{1}{4},
  \end{align*}
  lo que implica $\zeta (-1) = -\frac{1}{12}$, Q.E.D.
\end{quote}

El lector no debería tomar en serio el argumento de arriba y usar métodos
similares en sus demostraciones.

\begin{shaded}
  \small\noindent En PARI/GP, la función \verb|zeta(s)| calcula $\zeta (s)$:

\begin{verbatim}
? zeta (0.0)
% = -0.50000000000000000000000000000000000000

? zeta (-1)
% = -0.083333333333333333333333333333333333333
? -bernreal(2)/2
% = -0.083333333333333333333333333333333333333

? zeta (-2)
% = 0

? zeta (-3)
% = 0.0083333333333333333333333333333333333334
? -bernreal(4)/4
% = 0.0083333333333333333333333333333333333334
\end{verbatim}
\end{shaded}

% % % % % % % % % % % % % % % % % % % % % % % % % % % % % %

\section{* Conjetura de Lichtenbaum}

El matemático estadounidense \personality{Stephen Lichtenbaum} formuló en 1973
la hipótesis que los valores $\zeta (-n)$ están relacionados con ciertos grupos
abelianos $K_i (\ZZ)$ asociados al anillo $\ZZ$ (los \term{grupos K}, definidos
por el matemático estadounidense \personality{Daniel Quillen} (1940--2011)):
$$\zeta (-n) = \pm 2^? \, \frac{\# K_{2n} (\ZZ)}{\# K_{2n+1} (\ZZ)_{tors}},$$
donde $K_{2n} (\ZZ)$ son grupos abelianos finitos, y $\# K_{2n+1} (\ZZ)_{tors}$
denota la parte de torsión de grupos $K_{2n+1} (\ZZ)$ que tienen rango $1$.
El factor $2^?$ denota alguna potencia de $2$.

La definición de los $K_i (\ZZ)$ requiere otro curso de nivel de posgrado, así
que no voy a entrar en detalles... Estos grupos son muy difíciles de calcular,
y solo hasta en los años 2000 fueron determinados en gran parte. Como sugiere
la conjetura de Lichtenbaum, estos en efecto están relacionados con los números
de Bernoulli. Por ejemplo, $K_{22} (\ZZ) \isom \ZZ/691 \ZZ$ y
$K_{23} (\ZZ) \isom \ZZ / 65\,520 \ZZ$,
\[ \zeta (-11) = -\frac{B_{12}}{12} = \frac{691}{12\cdot 2730}, \quad
  \frac{\# K_{22} (\ZZ)}{\# K_{23} (\ZZ)} = \frac{691}{2\cdot 12\cdot 2730} \]
---los valores coinciden salvo una potencia de $2$.

\pagebreak

He aquí la tabla de algunos de los grupos $K_i (\ZZ)$:

\begin{center}
  \begin{tabular}{rx{1.8cm}x{1.8cm}x{1.8cm}x{1.8cm}}
    \hline
      $i\colon$ & 2 & 3 & 4 & 5 \tabularnewline
      \rule{0pt}{1.1em}$K_i (\ZZ)\colon$ & $\ZZ/2\ZZ$ & $\ZZ/48\ZZ$ & $0$ & $\ZZ$ \tabularnewline
      \hline
      $i\colon$ & 6 & 7 & 8 & 9 \tabularnewline
      \rule{0pt}{1.1em}$K_i (\ZZ)\colon$ & $0$ & $\ZZ/240\ZZ$ & $(0?)$ & $\ZZ\oplus\ZZ/2\ZZ$ \tabularnewline
      \hline
      $i\colon$ & 10 & 11 & 12 & 13 \tabularnewline
      \rule{0pt}{1.1em}$K_i (\ZZ)\colon$ & $\ZZ/2\ZZ$ & $\ZZ/1008\ZZ$ & $(0?)$ & $\ZZ$ \tabularnewline
      \hline
      $i\colon$ & 14 & 15 & 16 & 17 \tabularnewline
      \rule{0pt}{1.1em}$K_i (\ZZ)\colon$ & $0$ & $\ZZ/480\ZZ$ & $(0?)$ & $\ZZ\oplus\ZZ/2\ZZ$ \tabularnewline
      \hline
      $i\colon$ & 18 & 19 & 20 & 21 \tabularnewline
      \rule{0pt}{1.1em}$K_i (\ZZ)\colon$ & $\ZZ/2\ZZ$ & $\ZZ/528\ZZ$ & $(0?)$ & $\ZZ$ \tabularnewline
      \hline
      $i\colon$ & 22 & 23 & 24 & 25 \tabularnewline
      \rule{0pt}{1.1em}$K_i (\ZZ)\colon$ & $\ZZ/691\ZZ$ & $\ZZ/65\,520\ZZ$ & $(0?)$ & $\ZZ\oplus\ZZ/2\ZZ$ \tabularnewline
      \hline\tabularnewline
  \end{tabular}
\end{center}

Los grupos $K_{4i} (\ZZ)$ son \emph{conjeturalmente} nulos, pero este hecho
es equivalente a la \term{conjetura de Vandiver} de la teoría de números,
que tampoco ha sido resuelta. La rama de matemáticas que estudia los grupos
$K_i (\ZZ)$ se llama la \term{teoría K algebraica}. Estos grupos están
relacionados con otros objetos, más adecuados para el estudio de los valores

de las funciones zeta: los grupos de \term{cohomología motívica}. Para más
información, véase \cite{Kahn-2005} y \cite{Weibel-2005}.

% % % % % % % % % % % % % % % % % % % % % % % % % % % % % %

\section{* Los valores $\zeta (2k+1)$}

Hemos visto que los valores de la función zeta en los enteros positivos pares
$\zeta (2k)$ y los enteros negativos $\zeta (-n)$ se expresan por los números
de Bernoulli. Los valores en los enteros positivos impares
$$\zeta (3), ~ \zeta (5), ~ \zeta (7), ~ \zeta (9), ~ \zeta (11), ~ \ldots$$
son más misteriosos. Note que la ecuación funcional \eqnref{ecuacion-funcional}
no dice nada sobre ellos. \emph{Al parecer}, son números trascendentes. Por
supuesto, los números
$$\zeta (2k) = (-1)^{k+1} \, B_{2k}\,\frac{2^{2k-1}}{(2k)!}\,\pi^{2k}$$
son también transcendentes, ya que $\pi$ es trascendente (¡de hecho, uno de los
pocos números específicos cuya trascendencia se puede demostrar!). Los valores
$\zeta (2k+1)$ deberían de ser trascendentes por alguna razón más sofisticada,
y se supone que entre $\zeta (2k+1)$ distintos no hay ninguna relación
algebraica.

Sin embargo, todavía no hay demostraciones ni siquiera de que los $\zeta (2k+1)$
sean irracionales. En 1977 el matemático francés \personality{Roger Apéry}
demostró que el número
$$\zeta (3) \approx 1.20205690315959428539973816\ldots$$
es irracional. La tumba de Apéry en París lleva la inscripción

\begin{center}
\noindent\textsc{Roger APÉRY}

\noindent\textsc{1916--1994}

\vspace{0.7em}

\noindent\rule{1cm}{0.4mm}

\vspace{1em}

\noindent$1 + \frac{1}{8} + \frac{1}{27} + \frac{1}{64} + \cdots \ne \frac{p}{q}$
\end{center}

\noindent Para más información sobre el teorema de Apéry, véase el artículo
\cite{van-der-Poorten-1978}.

Los métodos de Apéry no se generalizan para demostrar que $\zeta (5)$ es
irracional. Hay pocos resultados en esta dirección. El matemático francés
\personality{Tanguy Rivoal} demostró en 2000 que entre los números
$\zeta (3), \zeta (7), \zeta (9), \ldots$ hay una infinidad de irracionales,
mientras que el matemático ruso \personality{Wadim Zudilin} demostró en 2001 que
por lo menos un número entre $\zeta(5)$, $\zeta(7)$, $\zeta(9)$ y $\zeta(11)$ es
irracional \cite{Zudilin-2004}.

% % % % % % % % % % % % % % % % % % % % % % % % % % % % % %

\section{Digresión combinatoria: los números de Stirling}
\label{section:numeros-de-stirling}

Nuestro próximo objetivo es obtener algunas expresiones para los números de
Bernoulli que permitan estudiar sus propiedades aritméticas, específicamente sus
numeradores y denominadores. En el camino surgen ciertos números combinatorios,
conocidos como los \term{números de Stirling}.

\begin{definicion}
  \label{def:numeros-de-stirling-N}
  Sean $k$ y $\ell$ dos números naturales positivos.

  El \term{número de Stirling de primera clase} $\stirling{k}{\ell}$ es el número
  de permutaciones en el grupo simétrico $S_k$ que consisten en $\ell$ ciclos
  disjuntos.

  El \term{número de Stirling de segunda clase} $\Stirling{k}{\ell}$ es
  el número de posibilidades de escribir un conjunto de $k$ elementos como una
  unión disjunta de $\ell$ conjuntos no vacíos.
\end{definicion}

\begin{ejemplo}
  $\stirling{4}{2} = 11$. Las permutaciones correspondientes en $S_4$ son
  \begin{gather*}
    (1)\,(2~3~4), ~ (1)\,(2~4~3), ~ (2)\,(1~3~4), ~ (2)\,(1~4~3),\\
    (3)\,(1~2~4), ~ (3)\,(1~4~2), ~ (4)\,(1~2~3), ~ (4)\,(1~3~2),\\
    (1~2)\,(3~4), ~ (1~3)\,(2~4), ~ (1~4)\,(2~3). \qedhere
  \end{gather*}
\end{ejemplo}

\begin{ejemplo}
  $\Stirling{4}{2} = 7$. Las descomposiciones de conjuntos correspondientes son
  \begin{align*}
    \{ 1,2,3,4 \} & = \{ 1 \} \cup \{ 2,3,4 \} = \{ 2 \} \cup \{ 1,3,4 \} = \{ 3 \} \cup \{ 1,2,4 \} = \{ 4 \} \cup \{ 1,2,3 \} \\
                  & = \{ 1,2 \} \cup \{ 3,4 \} = \{ 1,3 \} \cup \{ 2,4 \} = \{ 1,4 \} \cup \{ 2,3 \}. \qedhere
  \end{align*}
\end{ejemplo}

De la definición se siguen las identidades

\begin{align}
  \label{eqn:stirling-I-0} \stirling{k}{\ell} & = 0 \quad\text{para }\ell > k,\\
  \label{eqn:stirling-I-1} \stirling{k}{k} & = 1,\\
  \label{eqn:stirling-I-2} \stirling{k}{1} & = (k-1)!,\\
  \label{eqn:stirling-I-3} \sum_{1 \le \ell \le k} \stirling{k}{\ell} & = k!,\\
  \label{eqn:stirling-I-recur} \stirling{k+1}{\ell} & = \stirling{k}{\ell-1} + k\,\stirling{k}{\ell},
\end{align}

\begin{align}
  \label{eqn:stirling-II-0} \Stirling{k}{\ell} & = 0 \quad\text{para }\ell > k,\\
  \label{eqn:stirling-II-1} \Stirling{k}{k} & = 1,\\
  \label{eqn:stirling-II-2} \Stirling{k}{1} & = 1,\\
  \label{eqn:stirling-II-3} \sum_{1 \le \ell \le k} \Stirling{k}{\ell} & = b (k),\\
  \label{eqn:stirling-II-recur} \Stirling{k+1}{\ell} & = \Stirling{k}{\ell-1} + \ell\,\Stirling{k}{\ell}.
\end{align}

\eqnref{eqn:stirling-I-1} significa que la única permutación en $S_k$ que
se descompone en el producto de $k$ ciclos disjuntos es la permutación
identidad. \eqnref{eqn:stirling-I-2} significa que en $S_k$ hay $(k-1)!$
diferentes $k$-ciclos (¿por qué?).

\eqnref{eqn:stirling-I-3} es el hecho de que el tipo cíclico es una relación
de equivalencia sobre los elementos de $S_k$ (en efecto, los tipos cíclicos
corresponden a las clases de conjugación). \eqnref{eqn:stirling-II-3} es
el análogo de esta identidad: el número total de particiones se conoce como
el \term{número de Bell} $b (k)$. Los primeros números de Bell son $b (1) = 1$,
$b (2) = 2$, $b (3) = 5$, $b (4) = 15$, $b (5) = 52$, $b (6) = 203$, $\ldots$;
véase \url{http://oeis.org/A000110} En este curso, no vamos estudiar estos
números (también porque la notación parece mucho a los números de Bernoulli :-)

Las recurrencias \eqnref{eqn:stirling-I-recur} y \eqnref{eqn:stirling-II-recur}
se siguen de la definición combinatoria.
Por ejemplo, en \eqnref{eqn:stirling-I-recur}, consideremos las permutaciones
de elementos $\{ 1, \ldots, k, k+1 \}$. Sea $\sigma \in S_{k+1}$ una permutación
que se descompone en el producto de $\ell$ ciclos disjuntos. Si
$\sigma (k+1) = k+1$, entonces $(k+1)$ forma un ciclo sí mismo, y para el resto
de los elementos hay $\stirling{k}{\ell-1}$ posibles descomposiciones.
Si $\sigma (k+1) \ne k+1$, entonces $k+1$ pertenece a algún ciclo. Para enumerar
todas las posibilidades, podemos primero considerar $\stirling{k}{\ell}$
descomposiciones de las permutaciones de $\{ 1,\ldots,k \}$ en $\ell$ ciclos
disjuntos, y luego para cada descomposición hay $k$ posibilidades de poner $k+1$
en uno de los ciclos. La fórmula \eqnref{eqn:stirling-II-recur} se explica de
la misma manera: si tenemos un conjunto $X$ de $k+1$ elemento, podemos
considerar un elemento $x\in X$. Para las descomposiciones de $X$ en la unión
de $\ell$ subconjuntos hay dos casos: o bien $\{ x \}$ forma un conjunto en
la descomposición, y quedan $\Stirling{k}{\ell-1}$ posibilidades para
descomponer $X\setminus \{ x \}$; o bien $x$ pertenece a algún conjunto.
En el segundo caso, hay $\Stirling{k}{\ell}$ posibilidades de descomponer
$X\setminus \{ x \}$ en $\ell$ subconjuntos, y luego en cada caso hay $\ell$
posibilidades de poner $x$ en uno de los conjuntos.

También será útil definir $\stirling{k}{\ell}$ y $\Stirling{k}{\ell}$ para
$k,\ell = 0$:

\begin{definicion}
  \label{numeros-de-stirling-para-0}
  \begin{gather*}
    \stirling{0}{0} = 1, \quad \stirling{k}{0} = \stirling{0}{\ell} = 0 \text{ para }k,\ell\ne 0,\\
    \Stirling{0}{0} = 1, \quad \Stirling{k}{0} = \Stirling{0}{\ell} = 0 \text{ para }k,\ell\ne 0.
  \end{gather*}
\end{definicion}

Podemos definir $\stirling{k}{\ell}$ y $\Stirling{k}{\ell}$ por los valores
iniciales \ref{numeros-de-stirling-para-0} y las relaciones de recurrencia
\eqnref{eqn:stirling-I-recur} y \eqnref{eqn:stirling-II-recur}. Esta definición
está compatible con \ref{def:numeros-de-stirling-N}. Por ejemplo, en el caso de
$\stirling{k}{\ell}$, podemos ver que las identidades
$$\stirling{0}{0} = 1, \quad \stirling{k}{0} = \stirling{0}{\ell} = 0 \text{ para }k,\ell\ne 0$$
implican
$$\stirling{k}{1} = (k-1)! \text{ para }k \ge 1, \quad \stirling{1}{\ell} = 0 \text{ para }\ell \ge 1.$$
En efecto,
\begin{align*}
  \stirling{k}{1} & = \underbrace{\stirling{k}{0}}_0 + (k-1)\,\stirling{k-1}{1}
                    = (k-1)\,(k-2)\,\stirling{k-2}{1} = \cdots \\
                  & = (k-1)\,(k-2)\cdots 2\cdot 1 \, \Bigg(\underbrace{\stirling{0}{0}}_1 + \underbrace{\stirling{0}{1}}_0 \Bigg)
                    = (k-1)!
\end{align*}
y para $\ell > 1$
$$\stirling{1}{\ell} = \stirling{0}{\ell-1} + 0\cdot \stirling{0}{\ell} = 0.$$

\begin{shaded}
  \small\noindent En PARI/GP, \verb|stirling(k,l,2)| $= \Stirling{k}{\ell}$
  (el parametro «$2$» significa «de segunda clase»):

\begin{verbatim}
? stirling (4,2,2)
% = 7
\end{verbatim}

  \noindent PARI/GP usa otra definición de los números de Stirling de primera
  clase. La única diferencia es el signo: \verb|stirling(k,l)|
  $= (-1)^{k-\ell} \stirling{k}{\ell}$:

\begin{verbatim}
? stirling (4,2)
% = 11
? stirling(4,3) 
% = -6
\end{verbatim}
\end{shaded}

\begin{ejercicio}
  Demuestre que $\stirling{k}{k-1} = {k \choose 2}$.
\end{ejercicio}

\begin{ejercicio}
  Note que las recurrencias de arriba con los valores iniciales para
  $k, \ell = 0$ nos permiten definir $\stirling{k}{\ell}$ y $\Stirling{k}{\ell}$
  para todo $k,\ell \in \ZZ$. Demuestre que
  $$\stirling{k}{\ell} = \Stirling{-\ell}{-k}.$$
  Esto significa que los números de Stirling de primera y de segunda clase son
  esencialmente el mismo objeto.
\end{ejercicio}

\begin{ejercicio}
  Demuestre que $\Stirling{k}{\ell} = 0$ para $k\ell < 0$.
\end{ejercicio}

\noindent (Los últimos dos ejercicios sirven solo para acostumbrarse a las
recurrencias con $\stirling{k}{\ell}$ y $\Stirling{k}{\ell}$; no vamos a usar
los números de Stirling para $k$ y $\ell$ negativos.)

\pagebreak

\begin{center}
{\small
\noindent\begin{tabular}{|f{0.9cm}|f{0.9cm}|f{0.9cm}|f{0.9cm}|f{0.9cm}|f{0.9cm}|f{0.9cm}|f{0.9cm}|f{0.9cm}|f{0.9cm}|f{0.9cm}|}
\hline
\backslashbox{$k$}{$\ell$} & $0$ & $1$ & $2$ & $3$ & $4$ & $5$ & $6$ & $7$ & $8$ & $9$ \tabularnewline\hline
$0$ & $1$ &  &  &  &  &  &  &  &  & \tabularnewline\hline
$1$ & $0$ & $1$ &  &  &  &  &  &  &  & \tabularnewline\hline
$2$ & $0$ & $1$ & $1$ &  &  &  &  &  &  & \tabularnewline\hline
$3$ & $0$ & $2$ & $3$ & $1$ &  &  &  &  &  & \tabularnewline\hline
$4$ & $0$ & $6$ & $11$ & $6$ & $1$ &  &  &  &  & \tabularnewline\hline
$5$ & $0$ & $24$ & $50$ & $35$ & $10$ & $1$ &  &  &  & \tabularnewline\hline
$6$ & $0$ & $120$ & $274$ & $225$ & $85$ & $15$ & $1$ &  &  & \tabularnewline\hline
$7$ & $0$ & $720$ & $1764$ & $1624$ & $735$ & $175$ & $21$ & $1$ &  & \tabularnewline\hline
$8$ & $0$ & $5040$ & $13068$ & $13132$ & $6769$ & $1960$ & $322$ & $28$ & $1$ & \tabularnewline\hline
$9$ & $0$ & $40320$ & $109584$ & $118124$ & $67284$ & $22449$ & $4536$ & $546$ & $36$ & $1$ \tabularnewline\hline
\end{tabular}}

\vspace{1em}

\noindent Valores de $\stirling{k}{\ell}$

\vspace{\fill}

{\small
\noindent\begin{tabular}{|f{0.9cm}|f{0.9cm}|f{0.9cm}|f{0.9cm}|f{0.9cm}|f{0.9cm}|f{0.9cm}|f{0.9cm}|f{0.9cm}|f{0.9cm}|f{0.9cm}|}
\hline
\backslashbox{$k$}{$\ell$} & $0$ & $1$ & $2$ & $3$ & $4$ & $5$ & $6$ & $7$ & $8$ & $9$ \tabularnewline\hline
$0$ & $1$ &  &  &  &  &  &  &  &  & \tabularnewline\hline
$1$ & $0$ & $1$ &  &  &  &  &  &  &  & \tabularnewline\hline
$2$ & $0$ & $1$ & $1$ &  &  &  &  &  &  & \tabularnewline\hline
$3$ & $0$ & $1$ & $3$ & $1$ &  &  &  &  &  & \tabularnewline\hline
$4$ & $0$ & $1$ & $7$ & $6$ & $1$ &  &  &  &  & \tabularnewline\hline
$5$ & $0$ & $1$ & $15$ & $25$ & $10$ & $1$ &  &  &  & \tabularnewline\hline
$6$ & $0$ & $1$ & $31$ & $90$ & $65$ & $15$ & $1$ &  &  & \tabularnewline\hline
$7$ & $0$ & $1$ & $63$ & $301$ & $350$ & $140$ & $21$ & $1$ &  & \tabularnewline\hline
$8$ & $0$ & $1$ & $127$ & $966$ & $1701$ & $1050$ & $266$ & $28$ & $1$ & \tabularnewline\hline
$9$ & $0$ & $1$ & $255$ & $3025$ & $7770$ & $6951$ & $2646$ & $462$ & $36$ & $1$ \tabularnewline\hline
\end{tabular}}

\vspace{1em}

\noindent Valores de $\Stirling{k}{\ell}$
\end{center}

\pagebreak

% % % % % % % % % % % % % % % % % % % % % % % % % % % % % %

\section{Relación entre $B_k$ y los números de Stirling}

\begin{lema}
  \label{exp-stirling}
  Para todo $\ell \ge 0$
  $$\frac{(e^t - 1)^\ell}{\ell!} = \sum_{k \ge \ell} \Stirling{k}{\ell}\,\frac{t^k}{k!}.$$

  \begin{proof}
    Tenemos que ver que
    $$\frac{d^k}{dt^k} \left(\frac{(e^t - 1)^\ell}{\ell!}\right) (0) = \Stirling{k}{\ell}.$$
    Los valores iniciales coinciden, y va a ser suficiente demostrar que la
    recurrencia
    $$\Stirling{k+1}{\ell} = \Stirling{k}{\ell-1} + \ell \, \Stirling{k}{\ell}$$
    se cumple en nuestro caso:
    \[ \frac{d^{k+1}}{dt^{k+1}} \left(\frac{(e^t - 1)^\ell}{\ell!}\right) (0) =
      \frac{d^k}{dt^k} \left(\frac{(e^t - 1)^{\ell-1}}{(\ell-1)!}\right) (0) +
      \ell\,\frac{d^k}{dt^k} \left(\frac{(e^t - 1)^\ell}{\ell!}\right) (0). \]
    En efecto,
    \begin{align*}
      \frac{d^{k+1}}{dt^{k+1}} \left(\frac{(e^t - 1)^\ell}{\ell!}\right) & =
                                                                           \frac{d^k}{dt^k} \left(\frac{(e^t - 1)^{\ell-1}}{(\ell-1)!}\,e^t\right) =
                                                                           \frac{d^k}{dt^k} \left(\frac{(e^t - 1)^{\ell-1} \, (1 + e^t - 1)}{(\ell - 1)!}\right) \\
                                                                         & =
                                                                           \frac{d^k}{dt^k} \left(\frac{(e^t - 1)^{\ell-1}}{(\ell-1)!} +
                                                                           \frac{(e^t-1)^\ell}{(\ell-1)!}\right) \\
                                                                         & =
                                                                           \frac{d^k}{dt^k} \left(\frac{(e^t - 1)^{\ell-1}}{(\ell-1)!}\right) +
                                                                           \ell\,\frac{d^k}{dt^k} \left(\frac{(e^t-1)^\ell}{\ell!}\right).
                                                                           \qedhere
    \end{align*}
  \end{proof}
\end{lema}

\begin{ejercicio}
  \label{log-stirling}
  Demuestre la identidad
  $$\frac{(-\ln (1-t))^\ell}{\ell!} = \sum_{k \ge \ell} \stirling{k}{\ell}\,\frac{t^k}{k!}.$$
  (De nuevo, es suficiente considerar las derivadas formales y verificar que se
  cumple la misma recurrencia que define los números de Stirling
  correspondientes:
  $\stirling{k+1}{\ell} = \stirling{k}{\ell-1} + k\,\stirling{k}{\ell}$.)
\end{ejercicio}

Lo que acabamos de ver en \ref{exp-stirling} y \ref{log-stirling} son las
funciones generatrices para los números de Stirling, pero no soy tan sádico para
dar esto como la definición de $\Stirling{k}{\ell}$ y $\stirling{k}{\ell}$.

\begin{lema}
  \label{stirling-via-binomial}
  Para $k,\ell\ge 0$
  $$\Stirling{k}{\ell} = \frac{(-1)^\ell}{\ell!}\,\sum_{0 \le i \le \ell} (-1)^i \, {\ell \choose i} \, i^k.$$

  \begin{proof}
    De nuevo, podemos verificar que los valores iniciales coinciden y la suma
    satisface la misma recurrencia que $\Stirling{k}{\ell}$:
    $$\Stirling{k+1}{\ell} = \Stirling{k}{\ell-1} + \ell \, \Stirling{k}{\ell}.$$
    Para los valores iniciales, si $k = \ell = 0$, la suma nos da
    $\Stirling{0}{0} = 1$ (como siempre en el contexto algebraico/combinatorio,
    $0^0 = 1$); si $k > 0$, $\ell = 0$, la suma nos da $0$; si $k = 0$,
    $\ell > 0$, la suma también nos da
    $\sum_{0 \le i \le \ell} (-1)^i \, {\ell\choose i} = 0$. Para la
    recurrencia,
    \begin{align*}
      \frac{(-1)^\ell}{\ell!}\,\sum_{0 \le i \le \ell} (-1)^i \, {\ell \choose i} \, i^{k+1} & = \frac{(-1)^\ell}{(\ell-1)!}\,\sum_{0 \le i \le \ell} (-1)^i \, \frac{i}{\ell}\, {\ell \choose i} \, i^k \\
                                                                                             & = \frac{(-1)^\ell}{(\ell-1)!}\,\sum_{0 \le i \le \ell} (-1)^i \, \left({\ell \choose i} - {\ell - 1\choose i}\right) \, i^k \\
                                                                                             & = \frac{(-1)^{\ell-1}}{(\ell-1)!}\,\sum_{0 \le i \le \ell-1} (-1)^i \, {\ell - 1 \choose i} \, i^k + \ell\,\frac{(-1)^\ell}{\ell!}\,\sum_{0 \le i \le \ell} (-1)^i \, {\ell \choose i} \, i^k.
    \end{align*}
    Aquí hemos usado la identidad
    \[ {\ell \choose i} - {\ell - 1 \choose i} = \frac{i}{\ell}\,{\ell \choose i}. \qedhere \]
  \end{proof}
\end{lema}

\begin{teorema}
  \[ B_k = (-1)^k \, \sum_{0 \le \ell \le k} \frac{(-1)^\ell \, \ell! \, \Stirling{k}{\ell}}{\ell+1} =
    (-1)^k \, \sum_{0 \le \ell \le k} \frac{1}{\ell+1}\,\sum_{0 \le i \le \ell} (-1)^i \, {\ell \choose i} \, i^k. \]
\end{teorema}

La segunda igualdad viene de la expresión \ref{stirling-via-binomial}
y (¡por fin!) nos da una expresión para $B_k$ sin recurrencias.

\begin{proof}
  La función generatriz para $B_k$ es $\frac{t\,e^t}{e^t - 1}$. Gracias a
  \ref{thm:exp-y-log-son-inversos}, podemos escribir
  $$\frac{t\,e^t}{e^t - 1} = \frac{t}{1 - e^{-t}} = \frac{-\ln (1 - (1-e^{-t}))}{1 - e^{-t}}.$$
  Luego,
  \begin{align*}
    \frac{-\ln (1 - (1-e^{-t}))}{1 - e^{-t}} & = \frac{1}{1 - e^{-t}}\,\sum_{\ell \ge 1} \frac{(1 - e^{-t})^\ell}{\ell} \\
                                             & = \sum_{\ell \ge 1} \frac{(1 - e^{-t})^{\ell-1}}{\ell} \\
                                             & = \sum_{\ell \ge 0} \frac{(-1)^\ell\,\ell!}{\ell+1} \sum_{k \ge \ell} \Stirling{k}{\ell}\,\frac{(-t)^k}{k!} \quad\text{por \ref{exp-stirling}} \\
                                             & = \sum_{k \ge 0} (-1)^k \left(\sum_{0\le \ell \le k} \frac{(-1)^\ell\,\ell! \, \Stirling{k}{\ell}}{\ell+1}\right)\,\frac{t^k}{k!}.
                                               \qedhere
  \end{align*}
\end{proof}

\begin{shaded}
\small
\begin{verbatim}
? bernbin (k) = (-1)^k * sum (l=0,k, 1/(l+1)*sum(i=0,l, (-1)^i*binomial(l,i)*i^k));
? vector (10,k,bernbin(k))                                                         
% = [1/2, 1/6, 0, -1/30, 0, 1/42, 0, -1/30, 0, 5/66]
\end{verbatim}
\end{shaded}

% % % % % % % % % % % % % % % % % % % % % % % % % % % % % %

\section{Denominadores de $B_k$ (el teorema de Clausen--von Staudt)}
\label{section:teorema-de-von-Staudt}

\begin{teorema}
  Para todo $k \ge 2$ par se tiene
  $$B_k = -\sum_{\substack{p\text{ primo} \\ p-1 \, \mid \, k}} \frac{1}{p} + C_k,$$
  donde $C_k \in \ZZ$ y la suma es sobre todos los $p$ tales que $p-1$ divide a
  $k$.
\end{teorema}

Este resultado fue descubierto de manera independiente por el astrónomo y
matemático danés \personality{Thomas Clausen} (1801--1885) y el matemático
alemán \personality{Karl Georg Christian von Staudt} (1798--1867).

\begin{ejemplo}
  \begin{align*}
    B_2 & = \frac{1}{6} = - \left(\frac{1}{2} + \frac{1}{3}\right) + 1,\\
    B_4 & = -\frac{1}{30} = - \left(\frac{1}{2} + \frac{1}{3} + \frac{1}{5}\right) + 1,\\
    B_6 & = \frac{1}{42} = - \left(\frac{1}{2} + \frac{1}{3} + \frac{1}{7}\right) + 1,\\
        & \cdots \\
    B_{14} & = \frac{7}{6} = - \left(\frac{1}{2} + \frac{1}{3}\right) + 2,\\
        & \cdots \qedhere
  \end{align*}
\end{ejemplo}

En particular, el denominador de $B_k$ es precisamente el producto de todos
los primos $p$ tales que $p-1 \mid k$. Esto explica por qué los denominadores
de $B_k$ son libres de cuadrados y divisibles por $6$. No tenemos mucho control
sobre el número $C_k$ (véase \S\ref{sec:primos-irregulares}); solo podemos notar
que el valor de $C_k$ va a estar cerca de $B_k$, así que
$|C_{2k}| \xrightarrow{k \to \infty} \infty$.

\begin{proof}
  Gracias a la fórmula
  $$B_k = (-1)^k \, \sum_{0 \le \ell \le k} \frac{(-1)^\ell \, \ell! \, \Stirling{k}{\ell}}{\ell+1},$$
  sabemos que en el denominador aparecen solamente los primos que dividen
  a $\ell + 1$; los primos $p > k+1$ no aparecen en el denominador. Vamos
  a analizar las contribuciones del término
  $\frac{(-1)^\ell \, \ell! \, \Stirling{k}{\ell}}{\ell+1}$ para diferentes
  $\ell$.

  \begin{itemize}
  \item[(1)] Supongamos que $\ell+1$ es compuesto, es decir $\ell+1 = ab$ para
    algunos $1 < a,b < \ell$.

    \begin{itemize}
    \item[(1.1)] Si $a\ne b$, entonces $ab \mid \ell!$, y el término
      $\frac{(-1)^\ell \, \ell! \, \Stirling{k}{\ell}}{\ell+1}$ es entero.

    \item[(1.2.1)] Si $a = b$ y $2a \le \ell$, entonces $a \mid \ell!$ y
      $2a \mid \ell!$, entonces $a^2 = \ell + 1$ divide a $\ell!$ y el término
      $\frac{(-1)^\ell \, \ell! \, \Stirling{k}{\ell}}{\ell+1}$ es entero.

    \item[(1.2.2)] Si $a = b$ y $2a > \ell$, entonces
      $\ell+1 = a^2 \ge 2a \ge \ell + 1$, y por lo tanto $a^2 = 2a$ y $a = 2$,
      $\ell = 3$. Usando \ref{stirling-via-binomial}, podemos escribir
      \[ \frac{(-1)^\ell \, \ell! \, \Stirling{k}{\ell}}{\ell+1} =
        \frac{1}{4}\,\sum_{0 \le i \le 3} (-1)^i {3\choose i} i^k =
        \frac{1}{4} \left(0 - 3 + 3\cdot 2^k - 3^k\right). \]
      Este término es nulo para $\ell > k$, así que $k > 3$, y es un número par
      según la hipótesis del teorema. Tenemos
      $$- 3 + \cancel{3\cdot 2^k} - 3^k \equiv 1 - (-3)^k \equiv 1 - 1^k \equiv 0 \pmod{4}.$$
      Esto demuestra que el término
      $\frac{(-1)^\ell \, \ell! \, \Stirling{k}{\ell}}{\ell+1}$ es entero.
    \end{itemize}

    Hemos demostrado que si $\ell+1$ es compuesto, el término
    $\frac{(-1)^\ell \, \ell! \, \Stirling{k}{\ell}}{\ell+1}$ es entero.

  \item[(2)] Supongamos que $\ell+1 = p$ es primo. Tenemos por
    \ref{stirling-via-binomial}
    $$\frac{(-1)^\ell \, \ell! \, \Stirling{k}{\ell}}{\ell+1} =
    \frac{(-1)^{p-1} \, (p-1)! \, \Stirling{k}{p-1}}{p} =
    \frac{1}{p}\,\sum_{0 \le i \le p-1} (-1)^i \, {p-1 \choose i} \, i^k.$$
    Tenemos ${p - 1 \choose i} \equiv (-1)^i \pmod{p}$, entonces
    \[ \sum_{0 \le i \le p-1} (-1)^i \, {p-1 \choose i} \, i^k \equiv \sum_{0 \le i \le p-1} i^k \equiv \begin{cases}
        p-1 \equiv -1, & p-1 \mid k,\\
        0, & p-1 \nmid k.
      \end{cases} \pmod{p} \]
    Para ver la última congruencia, notamos que si $p-1\mid k$, entonces
    $i^{p-1} \equiv 1 \pmod{p}$ por el \term{pequeño teorema de Fermat}
    ($p\nmid i$). Si $p-1 \nmid k$, podemos escribir la suma
    $\sum_{0 \le i \le p-1} i^k$ como
    $$\sum_{1 \le i \le p-1} x^{ik} = \frac{1-x^{pk}}{1-x^k} - 1 \equiv 0 \pmod{p},$$
    donde $x$ es una raíz primitiva de la unidad módulo $p$. Aquí
    $x^{pk} \equiv x^k \not\equiv 1 \pmod{p}$ por el pequeño teorema de Fermat.

    Entonces, ci $\ell+1 = p$ es primo, el término
    $\frac{(-1)^\ell \, \ell! \, \Stirling{k}{\ell}}{\ell+1}$ va a contribuir
    $-\frac{1}{p}$ en el denominador si $p-1 \mid k$, y va a ser entero si
    $p-1 \nmid k$. \qedhere
  \end{itemize}
\end{proof}

% % % % % % % % % % % % % % % % % % % % % % % % % % % % % %

\section{Congruencias de Kummer}

Para un primo $p$ denotemos por $\ZZ_{(p)}$ el anillo de los números racionales
donde $p$ no aparece en el denominador:
$$\ZZ_{(p)} \dfn \Bigl\{ \frac{a}{b} \in \QQ \Bigm| p\nmid b \Bigr\}.$$
(Es un caso particular de \term{localización} de un anillo afuera de un ideal
primo.) Los elementos invertibles en $\ZZ_{(p)}$ son las fracciones no nulas
donde $p$ no aparece ni en el numerador ni en el denominador:
$$\ZZ_{(p)}^\times = \Bigl\{ \frac{a}{b} \in \QQ \Bigm| p\nmid a, ~ p\nmid b \Bigr\}.$$

El siguiente resultado es un caso particular de las
\term{congruencias de Kummer}
(véase \cite[\S 11.3]{Arakawa-Ibukiyama-Kaneko-2014}):

\begin{teorema}[Kummer, 1851]
  \label{thm:congruencias-de-kummer}
  Sea $p$ un número primo y $k$ un entero positivo tal que $p-1 \nmid k$.

  \begin{enumerate}
  \item[1)] $p$ no aparece en el denominador del número $B_k/k$:
    $$\frac{B_k}{k} \in \ZZ_{(p)}.$$

  \item[2)] Para todo $k'$ tal que $k' \equiv k \pmod{p-1}$ se cumple
    $$\frac{B_{k'}}{k'} \equiv \frac{B_k}{k} \pmod{p}.$$
  \end{enumerate}
\end{teorema}

Aquí la última relación puede ser interpretada como
$B_{k'}\cdot k \equiv B_k\cdot k' \pmod{p}$. También podemos interpretar una
fracción $\frac{B_k}{k}$ como un residuo módulo $p$ dado por $B_k\cdot k^{-1}$,
donde $k^{-1}$ es el residuo inverso a $k$ módulo $p$, que existe porque en este
caso $p \nmid k$.

\begin{ejemplo}
  Sea $p = 7$, $k = 10$, $k' = 4$. En este caso $(p-1) \nmid k, k'$ y
  $k \equiv k' \pmod{p-1}$. Luego,
  \[ \frac{B_4}{4} = -\frac{1}{30}\,\frac{1}{4} = -\frac{1}{120} \equiv -1 \equiv 6 \pmod{7}
    \quad\text{y}\quad
    \frac{B_{10}}{10} = \frac{5}{66}\,\frac{1}{10} = \frac{1}{132} \equiv \frac{1}{6} \equiv 6 \pmod{7}.
  \qedhere \]
\end{ejemplo}

Para demostrar el teorema, nos va a servir el siguiente

\begin{lema}
  Sea $p$ un primo impar y sea $f (t) \in \ZZ_{(p)} [\![t]\!]$ una serie formal
  de potencias con coeficientes en $\ZZ_{(p)}$. Entonces para los coeficientes
  de Taylor de la serie
  $$f (e^t - 1) = \sum_{k \ge 0} a_k \, \frac{t^k}{k!}$$
  se cumple
  $$a_k \in \ZZ_{(p)} \quad \text{y}\quad a_{k + (p-1)} \equiv a_k \pmod{p}.$$

  \begin{proof}
    Si $f (t) = \sum_{\ell \ge 0} b_\ell\,t^\ell$, tenemos
    \begin{align*}
      f (e^t - 1) & = \sum_{\ell \ge 0} b_\ell \, (e^t - 1)^\ell \\
                  & = \sum_{\ell \ge 0} b_\ell \, \sum_{0 \le i \le \ell} (-1)^i \, {\ell \choose i} \, e^{t \, (\ell-i)}.
    \end{align*}

    Los coeficientes de la serie de Taylor son
    \[ a_k = \frac{d^k}{dt^k} \left(f (e^t - 1)\right)_{t = 0} =
      \sum_{\ell \ge 0} b_\ell \, \sum_{0 \le i \le \ell} (-1)^i \, {\ell \choose i} \, (\ell-i)^k. \]
    Notemos que en $(e^t - 1)^\ell$ no hay términos de grado $k < \ell$, así que
    la suma de arriba es finita: en efecto es sobre $0 \le \ell \le k$. Ya que
    $b_\ell \in \ZZ_{(p)}$, de esta fórmula se deduce que $a_k \in
    \ZZ_{(p)}$. Luego,

    $$a_{k + (p-1)} - a_k =
    \sum_{\ell \ge 0} b_\ell \, \sum_{0 \le i \le \ell} (-1)^i \, {\ell \choose i} \, (\ell-i)^k\,\left((\ell-i)^{p-1} - 1\right).$$
    Ahora si $\ell - i$ es divisible por $p$, la fórmula demuestra que
    $a_{k + (p-1)} - a_k$ es también divisible por $p$. Si $\ell - i$ no es
    divisible por $p$, entonces por el pequeño teorema de Fermat
    $(\ell-i)^{p-1} - 1 \equiv 0 \pmod{p}$, y $a_{k + (p-1)} - a_k$ es también
    divisible por $p$.
  \end{proof}
\end{lema}

\begin{proof}[Demostración del teorema \ref{thm:congruencias-de-kummer}]
  Sea $c \ne 1$ algún número natural tal que $p \nmid c$. Consideramos la serie
  de potencias
  $$f (t) \dfn \frac{1}{t} - \frac{c}{(1+t)^c - 1}.$$
  Ya que $p \nmid c$, el polinomio $\frac{(1+t)^c - 1}{c}$ tiene coeficientes en
  $\ZZ_{(p)}$ y es invertible en $\ZZ_{(p)} (\!(t)\!)$:
  $$\frac{c}{(1+t)^c - 1} = \frac{1}{t} - \frac{c-1}{2} + \frac{c^2 - 1}{12}\,t + \cdots$$
  Se sigue que
  $$f (t) = \frac{1}{t} - \frac{c}{(1+t)^c - 1} = \frac{c-1}{2} - \frac{c^2 - 1}{12}\,t + \cdots$$
  tiene coeficientes en $\ZZ_{(p)}$. Podemos aplicar el lema de arriba a la
  serie
  \begin{align*}
    f (e^t - 1) & = \frac{1}{e^t-1} - \frac{c}{e^{ct} - 1}
                  = \frac{1}{t}\left(\frac{t}{e^t-1} - \frac{ct}{e^{ct} - 1}\right) \\
                & = -\frac{1-c}{2} + \sum_{k \ge 2} \left( (1-c^k)\,\frac{B_k}{k} \, \frac{t^{k-1}}{(k-1)!}\right).
  \end{align*}

  Aquí hemos usado la función generatriz
  $\frac{t}{e^t - 1} = \frac{t\,e^t}{e^t-1} - t = 1 - \frac{t}{2} \sum_{k \ge 0} B_k \, \frac{t^k}{k!}$.
  El lema precedente implica que $(1 - c^k)\,\frac{B_k}{k} \in \ZZ_{(p)}$ y que
  para todo $k' \equiv k \pmod{p-1}$ se tiene
  $$(1-c^k)\,\frac{B_k}{k} \equiv (1-c^{k'})\,\frac{B_{k'}}{k'} \pmod{p}.$$
  Sea $c$ una raíz primitiva módulo $p$ (un generador del grupo
  $(\ZZ/p\ZZ)^\times$). Si $p-1 \nmid k$, como en la hipótesis del teorema,
  entonces $p-1 \nmid k'$, y se tiene
  $(1-c^k), (1-c^{k'}) \in (\ZZ/p\ZZ)^\times$. Esto implica que
  $\frac{B_k}{k} \in \ZZ_{(p)}$ y
  $\frac{B_k}{k} \equiv \frac{B_{k'}}{k'} \pmod{p}$.
\end{proof}

% % % % % % % % % % % % % % % % % % % % % % % % % % % % % %

\section{* Numeradores de $B_k$ (primos irregulares)}
\label{sec:primos-irregulares}

Sea $p$ un número primo. Una \term{raíz $p$-ésima de la unidad} es un número
complejo $\zeta$ tal que $\zeta^p = 1$. Las raíces $p$-ésimas de la unidad son
precisamente
$$1, ~ \zeta_p \dfn e^{2\pi i /p}, ~ \zeta_p^2 = e^{2\pi i \, 2/p}, ~ \ldots, ~ \zeta_p^{p-1} = e^{2\pi i \, (p-1)/p}.$$
Estas son las raíces del \term{polinomio ciclotómico}
$$\Phi_p (x) = 1 + x + x^2 + \cdots + x^{p-1}$$
(en efecto, $\sum_{0 \le k \le p} e^{2\pi i k/p} = 1$). Si añadimos al cuerpo
$\QQ$ las raíces $p$-ésimas de la unidad, se obtiene el cuerpo $\QQ (\zeta_p)$
que recibe el nombre de \term{cuerpo ciclotómico}. Este cuerpo contiene el
anillo
$$\O_{\QQ (\zeta_p)} = \ZZ [\zeta_p] \dfn \{ a_0 + a_1\,\zeta_p + a_2\,\zeta_p^2 + \cdots + a_{p-2}\,\zeta_p^{p-2} \},$$
llamado el \term{anillo de enteros} de $\QQ (\zeta_p)$; es precisamente
el anillo formado por todos los elementos $\alpha \in \QQ (\zeta_p)$ tales que
$f (\alpha) = 0$ para algún polinomio \term{mónico} $f (x) \in \ZZ [x]$ (es
decir, el coeficiente mayor de $f (x)$ es $1$).

A semejanza de $\ZZ$, muchos anillos son \term{dominios de factorización única}:
todo elemento $x\ne 0$ puede ser escrito como un producto de primos
$$x = p_1\cdot p_2\cdots p_n,$$
y esta expresión es única, en el sentido de que si existe otra factorización
$x = q_1\cdot q_2\cdots q_m$, entonces $n = m$ y, salvo alguna permutación de
los factores, $q_i = u_i\,p_i$, donde $u_i$ es invertible ($u_i = \pm 1$ en el
caso de $\ZZ$).

No voy a entrar en detalles, pero resulta que el anillo $\ZZ [\zeta_p]$ es
un dominio de factorización única para $p < 23$ y a partir de $p = 23$
la factorización única no se cumple. Este fenómeno se estudia en
la \term{teoría algebraica de números}. A saber, a $\QQ (\zeta_p)$ se asocia un
grupo abeliano $\Cl (\QQ (\zeta_p))$, llamado el \term{grupo de clases} de
$\QQ (\zeta_p)$. Es un grupo finito, y su orden se llama el
\term{número de clase}:
$$h_{\QQ (\zeta_p)} \dfn \# \Cl (\QQ (\zeta_p)).$$
Resulta que $\ZZ [\zeta_p]$ es un dominio de factorización única si y solamente
si $h_{\QQ (\zeta_p)} = 1$ (es decir, si el grupo de clases
$\Cl (\QQ (\zeta_p))$ es trivial). La definición de $\Cl (\QQ (\zeta_p))$ y
la demostración de su finitud formaría parte de un curso de la teoría de números
algebraica (véase por ejemplo \cite{Neukirch-1999} y \cite{Marcus-1977}), pero
intuitivamente, $\Cl (\QQ (\zeta_p))$ mide qué tan lejos $\ZZ [\zeta_p]$ está
de ser un dominio de factorización única.

He aquí una tabla de los grupos de clases de algunos cuerpos ciclotómicos
$\QQ (\zeta_p)$, calculados con ayuda de PARI/GP:

\begin{minipage}[t]{0.5\textwidth}
  \begin{align*}
    \Cl (\QQ (\zeta_2)) & = 0,\\
    \Cl (\QQ (\zeta_3)) & = 0,\\
    \Cl (\QQ (\zeta_5)) & = 0,\\
    \Cl (\QQ (\zeta_7)) & = 0,\\
    \Cl (\QQ (\zeta_{11})) & = 0,\\
    \Cl (\QQ (\zeta_{13})) & = 0,\\
    \Cl (\QQ (\zeta_{17})) & = 0,\\
    \Cl (\QQ (\zeta_{19})) & = 0,
  \end{align*}
\end{minipage}
\begin{minipage}[t]{0.5\textwidth}
  \begin{align*}
    \Cl (\QQ (\zeta_{23})) & \isom \ZZ/3\ZZ, \\
    \Cl (\QQ (\zeta_{29})) & \isom \ZZ/2\ZZ \oplus \ZZ/2\ZZ \oplus \ZZ/2\ZZ, \\
    \Cl (\QQ (\zeta_{31})) & \isom \ZZ/9\ZZ, \\
    \Cl (\QQ (\zeta_{37})) & \isom \ZZ/37\ZZ, \\
    \Cl (\QQ (\zeta_{41})) & \isom \ZZ/11\ZZ \oplus \ZZ/11\ZZ, \\
    \Cl (\QQ (\zeta_{43})) & \isom \ZZ/211\ZZ, \\
    \Cl (\QQ (\zeta_{47})) & \isom \ZZ/695\ZZ, \\
    \Cl (\QQ (\zeta_{53})) & \isom \ZZ/4889\ZZ.
  \end{align*}
\end{minipage}

\vspace{1em}

Todo esto está relacionado con una historia embarazosa que pasó al famoso
matemático francés \personality{Gabriel Lamé} (1795--1870). En 1847 Lamé
presentó a la Academia de Ciencias de Francia una supuesta demostración
del \term{último teorema de Fermat} \cite{Lame-1847}:

\begin{quote}
  \noindent Para $n > 2$ la ecuación $x^n + y^n = z^n$ no tiene soluciones para
  $x,y,z$ enteros positivos.
\end{quote}

\noindent \personality{Joseph Liouville} (1809--1882) observó que su
demostración usaba la hipótesis sospechosa de que $\ZZ [\zeta_p]$ era un dominio
de factorización única. En efecto, el matemático alemán
\personality{Ernst Kummer} (1810--1893) había demostrado tres años antes en 1844
que para $\ZZ [\zeta_{23}]$ esto era falso, pero su resultado fue publicado en
una revista poco conocida. Kummer comunicó su ejemplo a Liouville, y este en fin
fue publicado en la revista de la Academia de Ciencias. Kummer logró demostrar
el último teorema de Fermat para ciertos números primos $n = p$ que él llamó
regulares:

\begin{definicion}
  Un primo $p$ es \term{irregular} si $p$ divide al número de clase
  $h_{\QQ (\zeta_p)}$ del cuerpo ciclotómico $\QQ (\zeta_p)$.
\end{definicion}

Por ejemplo, los primeros primos irregulares son $p = 37, 59, 67$, porque los
números de clases correspondientes son
\[ h_{\QQ (\zeta_{37})} = 37, \quad
  h_{\QQ (\zeta_{59})} = 3\cdot 59\cdot 233, \quad
  h_{\QQ (\zeta_{67})} = 67\cdot 12739. \]

Otro criterio, más fácil para nosotros, es el siguiente

\begin{hecho}[Kummer, 1850]
  $p$ es irregular si y solamente si $p$ divide al numerador de algún número de
  Bernoulli $B_{2k}$ para $2k \le p-3$.
\end{hecho}

\noindent Para la demostración, véase por ejemplo
\cite[Chapter 5]{Washington-1997}.

\begin{shaded}
  \small\noindent No es tan fácil calcular $h_{\QQ (\zeta_p)}$. Por ejemplo,
  en PARI/GP tenemos

\begin{verbatim}
? bnfinit(polcyclo(37)).clgp.no
% = 37
\end{verbatim}

  \noindent Para compilar una lista de los primos irregulares, tenemos que usar
  el criterio con los números de Bernoulli:

\begin{verbatim}
irregular_primes (n) = {
  local (p);
  for(i=1,n,
      p = prime (i);
      for (k=1, (p-3)/2,
          if (numerator(bernfrac(2*k))%p == 0, printf ("p = %d, B_%d\n", p,2*k))))
}
\end{verbatim}
\end{shaded}

\pagebreak

He aquí los primeros primos irregulares con los números de Bernoulli
correspondientes. Note que $157$ aparece en el numerador de $B_{62}$
y $B_{110}$:

\begin{center}
  \begin{tabular}{ll}
    $p = 37$: & $\displaystyle B_{32} = -\frac{\highlight{37}\cdot 683\cdot 305065927}{2\cdot 3 \cdot 5\cdot 17}$; \\
    \\
    $p = 59$: & $\displaystyle B_{44} = -\frac{11\cdot \highlight{59}\cdot 8089\cdot 2947939\cdot 1798482437}{2\cdot 3\cdot 5\cdot 23}$; \\
    \\
    $p = 67$: & $\displaystyle B_{58} = \frac{29\cdot \highlight{67}\cdot 186707\cdot 6235242049\cdot 37349583369104129}{2\cdot 3\cdot 59}$; \\
    \\
    $p = 101$: & $\displaystyle B_{68} = -\frac{17\cdot 37\cdot \highlight{101}\cdot 123143\cdot 1822329343\cdot 5525473366510930028227481}{2\cdot 3\cdot 5}$; \\
    \\
    $p = 103$: & $\displaystyle B_{24} = -\frac{\highlight{103}\cdot 2294797}{2\cdot 3\cdot 5\cdot 7\cdot 13}$; \\
    \\
    $p = 131$: & $\displaystyle B_{22} = \frac{11\cdot \highlight{131}\cdot 593}{2\cdot 3\cdot 23}$; \\
    \\
    $p = 149$: & $\displaystyle B_{130} = \frac{5\cdot 13\cdot \highlight{149}\cdot 463\cdot 2264267\cdot 3581984682522167 \cdots}{2\cdot 3\cdot 11\cdot 131}$; \\
    \\
    $p = 157$: & $\displaystyle B_{62} = \frac{31\cdot \highlight{157}\cdot 266689\cdot 329447317\cdot 28765594733083851481}{6}$, \\
    \\
              & $\displaystyle B_{110} = \frac{5\cdot \highlight{157}\cdot 76493\cdot 150235116317549231\cdot 36944818874116823428357691\cdots}{2\cdot 3\cdot 11\cdot 23}$. \\
  \end{tabular}
\end{center}

\vspace{1em}

Otro ejemplo curioso: $2017$ divide el numerador de $B_{1204}$ y es un primo
irregular.

\pagebreak

\begin{center}
  {\footnotesize
    \begin{tabular}{x{1cm}x{1cm}x{1cm}x{1cm}x{1cm}x{1cm}x{1cm}x{1cm}x{1cm}x{1cm}}
      $2$ & $\highlight{233}$ & $\highlight{547}$ & $\highlight{877}$ & $\highlight{1229}$ & $\highlight{1597}$ & $\highlight{1993}$ & $\highlight{2371}$ & $2749$ & $3187$ \tabularnewline
      $3$ & $239$ & $\highlight{557}$ & $\highlight{881}$ & $1231$ & $1601$ & $\highlight{1997}$ & $\highlight{2377}$ & $\highlight{2753}$ & $3191$ \tabularnewline
      $5$ & $241$ & $563$ & $883$ & $\highlight{1237}$ & $1607$ & $1999$ & $\highlight{2381}$ & $\highlight{2767}$ & $\highlight{3203}$ \tabularnewline
$7$ & $251$ & $569$ & $\highlight{887}$ & $1249$ & $\highlight{1609}$ & $\highlight{2003}$ & $\highlight{2383}$ & $\highlight{2777}$ & $3209$ \tabularnewline
      $11$ & $\highlight{257}$ & $571$ & $907$ & $1259$ & $\highlight{1613}$ & $2011$ & $\highlight{2389}$ & $\highlight{2789}$ & $3217$ \tabularnewline
      $13$ & $\highlight{263}$ & $\highlight{577}$ & $911$ & $1277$ & $\highlight{1619}$ & $\highlight{2017}$ & $2393$ & $\highlight{2791}$ & $\highlight{3221}$ \tabularnewline
      $17$ & $269$ & $\highlight{587}$ & $919$ & $\highlight{1279}$ & $\highlight{1621}$ & $2027$ & $2399$ & $2797$ & $\highlight{3229}$ \tabularnewline
      $19$ & $\highlight{271}$ & $\highlight{593}$ & $\highlight{929}$ & $\highlight{1283}$ & $1627$ & $2029$ & $\highlight{2411}$ & $2801$ & $3251$ \tabularnewline
      $23$ & $277$ & $599$ & $937$ & $1289$ & $\highlight{1637}$ & $\highlight{2039}$ & $2417$ & $2803$ & $3253$ \tabularnewline
$29$ & $281$ & $601$ & $941$ & $\highlight{1291}$ & $1657$ & $\highlight{2053}$ & $\highlight{2423}$ & $2819$ & $\highlight{3257}$ \tabularnewline
      $31$ & $\highlight{283}$ & $\highlight{607}$ & $947$ & $\highlight{1297}$ & $\highlight{1663}$ & $2063$ & $2437$ & $\highlight{2833}$ & $3259$ \tabularnewline
      $\highlight{37}$ & $\highlight{293}$ & $\highlight{613}$ & $\highlight{953}$ & $\highlight{1301}$ & $1667$ & $2069$ & $\highlight{2441}$ & $2837$ & $3271$ \tabularnewline
      $41$ & $\highlight{307}$ & $\highlight{617}$ & $967$ & $1303$ & $\highlight{1669}$ & $2081$ & $2447$ & $2843$ & $3299$ \tabularnewline
      $43$ & $\highlight{311}$ & $\highlight{619}$ & $\highlight{971}$ & $\highlight{1307}$ & $1693$ & $2083$ & $2459$ & $2851$ & $3301$ \tabularnewline
      $47$ & $313$ & $\highlight{631}$ & $977$ & $\highlight{1319}$ & $1697$ & $\highlight{2087}$ & $2467$ & $\highlight{2857}$ & $3307$ \tabularnewline
      $53$ & $317$ & $641$ & $983$ & $1321$ & $1699$ & $2089$ & $2473$ & $\highlight{2861}$ & $\highlight{3313}$ \tabularnewline
      $\highlight{59}$ & $331$ & $643$ & $991$ & $\highlight{1327}$ & $1709$ & $\highlight{2099}$ & $2477$ & $2879$ & $3319$ \tabularnewline
      $61$ & $337$ & $\highlight{647}$ & $997$ & $1361$ & $\highlight{1721}$ & $\highlight{2111}$ & $\highlight{2503}$ & $2887$ & $\highlight{3323}$ \tabularnewline
      $\highlight{67}$ & $\highlight{347}$ & $\highlight{653}$ & $1009$ & $\highlight{1367}$ & $1723$ & $2113$ & $2521$ & $2897$ & $\highlight{3329}$ \tabularnewline
      $71$ & $349$ & $\highlight{659}$ & $1013$ & $1373$ & $\highlight{1733}$ & $2129$ & $2531$ & $2903$ & $3331$ \tabularnewline
      $73$ & $\highlight{353}$ & $661$ & $1019$ & $\highlight{1381}$ & $1741$ & $2131$ & $2539$ & $\highlight{2909}$ & $3343$ \tabularnewline
      $79$ & $359$ & $\highlight{673}$ & $1021$ & $1399$ & $1747$ & $\highlight{2137}$ & $\highlight{2543}$ & $2917$ & $3347$ \tabularnewline
      $83$ & $367$ & $\highlight{677}$ & $1031$ & $\highlight{1409}$ & $\highlight{1753}$ & $2141$ & $2549$ & $\highlight{2927}$ & $3359$ \tabularnewline
      $89$ & $373$ & $\highlight{683}$ & $1033$ & $1423$ & $\highlight{1759}$ & $\highlight{2143}$ & $2551$ & $\highlight{2939}$ & $3361$ \tabularnewline
      $97$ & $\highlight{379}$ & $\highlight{691}$ & $1039$ & $1427$ & $\highlight{1777}$ & $\highlight{2153}$ & $\highlight{2557}$ & $2953$ & $3371$ \tabularnewline
      $\highlight{101}$ & $383$ & $701$ & $1049$ & $\highlight{1429}$ & $1783$ & $2161$ & $\highlight{2579}$ & $\highlight{2957}$ & $3373$ \tabularnewline
      $\highlight{103}$ & $\highlight{389}$ & $709$ & $1051$ & $1433$ & $\highlight{1787}$ & $2179$ & $\highlight{2591}$ & $2963$ & $3389$ \tabularnewline
      $107$ & $397$ & $719$ & $\highlight{1061}$ & $\highlight{1439}$ & $\highlight{1789}$ & $2203$ & $2593$ & $2969$ & $\highlight{3391}$ \tabularnewline
      $109$ & $\highlight{401}$ & $\highlight{727}$ & $1063$ & $1447$ & $1801$ & $2207$ & $2609$ & $2971$ & $\highlight{3407}$ \tabularnewline
      $113$ & $\highlight{409}$ & $733$ & $1069$ & $1451$ & $\highlight{1811}$ & $\highlight{2213}$ & $2617$ & $\highlight{2999}$ & $3413$ \tabularnewline
      $127$ & $419$ & $739$ & $1087$ & $1453$ & $1823$ & $2221$ & $\highlight{2621}$ & $3001$ & $\highlight{3433}$ \tabularnewline
      $\highlight{131}$ & $\highlight{421}$ & $743$ & $\highlight{1091}$ & $1459$ & $\highlight{1831}$ & $2237$ & $\highlight{2633}$ & $\highlight{3011}$ & $3449$ \tabularnewline
      $137$ & $431$ & $\highlight{751}$ & $1093$ & $1471$ & $\highlight{1847}$ & $\highlight{2239}$ & $\highlight{2647}$ & $3019$ & $3457$ \tabularnewline
      $139$ & $\highlight{433}$ & $\highlight{757}$ & $1097$ & $1481$ & $1861$ & $2243$ & $\highlight{2657}$ & $\highlight{3023}$ & $3461$ \tabularnewline
      $\highlight{149}$ & $439$ & $\highlight{761}$ & $1103$ & $\highlight{1483}$ & $1867$ & $2251$ & $2659$ & $3037$ & $3463$ \tabularnewline
      $151$ & $443$ & $769$ & $1109$ & $1487$ & $\highlight{1871}$ & $\highlight{2267}$ & $\highlight{2663}$ & $3041$ & $3467$ \tabularnewline
      $\highlight{157}$ & $449$ & $\highlight{773}$ & $\highlight{1117}$ & $1489$ & $1873$ & $2269$ & $\highlight{2671}$ & $\highlight{3049}$ & $\highlight{3469}$ \tabularnewline
      $163$ & $457$ & $787$ & $1123$ & $1493$ & $\highlight{1877}$ & $\highlight{2273}$ & $2677$ & $\highlight{3061}$ & $\highlight{3491}$ \tabularnewline
      $167$ & $\highlight{461}$ & $\highlight{797}$ & $\highlight{1129}$ & $\highlight{1499}$ & $\highlight{1879}$ & $2281$ & $2683$ & $3067$ & $3499$ \tabularnewline
      $173$ & $\highlight{463}$ & $\highlight{809}$ & $\highlight{1151}$ & $1511$ & $\highlight{1889}$ & $2287$ & $2687$ & $3079$ & $\highlight{3511}$ \tabularnewline
      $179$ & $\highlight{467}$ & $\highlight{811}$ & $\highlight{1153}$ & $\highlight{1523}$ & $\highlight{1901}$ & $\highlight{2293}$ & $\highlight{2689}$ & $\highlight{3083}$ & $\highlight{3517}$ \tabularnewline
      $181$ & $479$ & $\highlight{821}$ & $1163$ & $1531$ & $1907$ & $2297$ & $2693$ & $\highlight{3089}$ & $3527$ \tabularnewline
      $191$ & $487$ & $823$ & $1171$ & $1543$ & $1913$ & $\highlight{2309}$ & $2699$ & $3109$ & $\highlight{3529}$ \tabularnewline
      $193$ & $\highlight{491}$ & $\highlight{827}$ & $1181$ & $1549$ & $1931$ & $2311$ & $2707$ & $\highlight{3119}$ & $\highlight{3533}$ \tabularnewline
      $197$ & $499$ & $829$ & $1187$ & $1553$ & $\highlight{1933}$ & $2333$ & $2711$ & $3121$ & $\highlight{3539}$ \tabularnewline
      $199$ & $503$ & $\highlight{839}$ & $\highlight{1193}$ & $\highlight{1559}$ & $1949$ & $2339$ & $2713$ & $3137$ & $3541$ \tabularnewline
      $211$ & $509$ & $853$ & $\highlight{1201}$ & $1567$ & $\highlight{1951}$ & $2341$ & $2719$ & $3163$ & $3547$ \tabularnewline
      $223$ & $521$ & $857$ & $1213$ & $1571$ & $1973$ & $2347$ & $2729$ & $3167$ & $3557$ \tabularnewline
      $227$ & $\highlight{523}$ & $859$ & $\highlight{1217}$ & $1579$ & $\highlight{1979}$ & $2351$ & $2731$ & $3169$ & $\highlight{3559}$ \tabularnewline
      $229$ & $\highlight{541}$ & $863$ & $1223$ & $1583$ & $\highlight{1987}$ & $\highlight{2357}$ & $2741$ & $\highlight{3181}$ & $3571$ \tabularnewline
    \end{tabular}}

  \vspace{1em}

  Los primeros primos irregulares
\end{center}

\pagebreak

Desafortunadamente, hay un número infinito de primos irregulares. Esto fue
demostrado por el matemático danés \personality{K. L. Jensen} en 1915.
En efecto, su resultado era más fuerte: hay un número infinito de primos
irregulares de la forma $4k + 3$. Nosotros nos contentaremos con el siguiente

\begin{teorema}
  \label{infinitos-irregulares}
  Hay un número infinito de primos irregulares; es decir, $p$ que dividen
  el numerador de algún número de Bernoulli entre $B_2, B_4, \ldots, B_{p-3}$.
\end{teorema}

\begin{proof}
  El argumento es un poco similar a la demostración clásica del teorema de
  Euclides sobre la infinitud de los números primos: podemos suponer que
  $p_1,\ldots,p_r$ son todos los primos irregulares. Nuestro objetivo es
  encontrar otro primo irregular.

  Sea
  $$k \dfn N\cdot (p_1 - 1)\cdots (p_r - 1),$$
  donde $N$ es algún número tal que $|B_k/k| > 1$. Tal $N$ existe porque para
  $k = 2n$ par,
  $$|B_{2n}/2n| = \frac{(2n-1)!}{2^{2n-1}\,\pi^{2n}}\,\zeta (2n) \xrightarrow{n \to \infty} \infty.$$
  Entonces existe algún primo $p$ tal que $p$ divide el numerador de
  $B_k/k$. Por el teorema de Clausen--von Staudt, los $p_1, \ldots, p_r$ están
  en el denominador de $B_k$, de donde
  $p \notin \{ p_1, \ldots, p_r \}$. También tenemos $p-1 \nmid k$, porque en
  el caso $p-1 \mid k$ el primo $p$ estaría en el denominador.

  Sea $0 < k' < p-1$ el número tal que $k' \equiv k \pmod{p-1}$. Por las
  congruencias de Kummer
  $$\frac{B_{k'}}{k'} \equiv \frac{B_k}{k} \pmod{p},$$
  y entonces $p \mid B_{k'}$ y $p$ es irregular.
\end{proof}

Todavía no se sabe si el número de primos regulares es también infinito, pero
conjeturalmente, solo $1 - e^{-1/2} \approx 39\%$ de los primos son irregulares.

\begin{ejercicio}
  Calcule en PARI/GP el porcentaje de los primos irregulares entre los primeros
  $N$ primos para algún $N$ razonable (por ejemplo, $N = 300$).

% count_irregular_primes (n) = {
%  local (p, irreg);
%  irreg = 0;
%  for(i=1,n,
%      p = prime (i);
%      for (k=1, (p-3)/2,
%          if (numerator(bernfrac(2*k))%p == 0, irreg++; break)
%      )
%  );
%  irreg
%}
\end{ejercicio}

Para más información sobre el último teorema de Fermat para los primos
regulares, véase el libro \cite{Ribenboim-1979} (escrito mucho antes de
la demostración definitiva del teorema por \personality{Andrew Wiles} en 1995,
pero con buenas explicaciones de los resultados de Kummer).

\pagebreak

\bibliographystyle{../amsalpha-cust}
{\small\bibliography{bernoulli}}

\end{document}
