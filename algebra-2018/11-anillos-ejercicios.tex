\subsection*{Subanillos}

\begin{ejercicio}
  Verifique que hay una cadena de subanillos
  $$\ZZ [\sqrt{5}] \subset \ZZ \Bigl[\frac{1 + \sqrt{5}}{2}\Bigr] \subset \RR$$
  donde
  \[ \ZZ [\sqrt{5}] \dfn \{ a + b\sqrt{5} \mid a,b\in \ZZ \}, \quad
     \ZZ \Bigl[\frac{1 + \sqrt{5}}{2}\Bigr] \dfn
     \Bigl\{ a + b\,\frac{1 + \sqrt{5}}{2} \Bigm| a,b\in \ZZ \Bigr\}. \]

  \ifdefined\solutions
  \begin{solucion}
    Notamos que $\ZZ [\sqrt{5}]$ es un subconjunto de
    $\ZZ \Bigl[\frac{1 + \sqrt{5}}{2}\Bigr]$:
    $$a + b\sqrt{5} = (a - b) + 2b\,\frac{1 + \sqrt{5}}{2}.$$
    Los elementos $0$ y $1$ evidentemente pertenecen a estos conjuntos. Tenemos
    obviamente
    \[ \left(a + b\,\frac{1 + \sqrt{5}}{2}\right) \pm
       \left(c + d\,\frac{1 + \sqrt{5}}{2}\right) =
       (a\pm c) + (b\pm d)\,\frac{1 + \sqrt{5}}{2} \]
    y
    \[ \left(a + b\sqrt{5}\right) \pm \left(c + d\sqrt{5}\right) =
       (a\pm c) + (b\pm d)\sqrt{5}. \]
    Para los productos, calculamos que
    \[ \left(a + b\,\frac{1 + \sqrt{5}}{2}\right) \cdot
       \left(c + d\,\frac{1 + \sqrt{5}}{2}\right) =
       ac + (bc + ad)\,\frac{1 + \sqrt{5}}{2} + bd\,\frac{3 + \sqrt{5}}{2} =
       (ac + bd) + (bc + ad + bd)\,\frac{1 + \sqrt{5}}{2} \]
    y
    \[ \left(a + b\sqrt{5}\right)\cdot \left(c + d\sqrt{5}\right) =
       (ac + 5bd) + (ad + bc)\sqrt{5}. \]
  \end{solucion}
  \fi
\end{ejercicio}

\begin{ejercicio}
  Consideremos el anillo de las funciones $f\colon \RR\to\RR$ respecto a
  las operaciones \term{punto por punto}
  \[ (f + g) (x) \dfn f (x) + g (x), \quad
     (f\cdot g) (x) \dfn f (x)\cdot g (x). \]
  Demuestre que hay una cadena de subanillos
  \begin{multline*}
    \{ \text{funciones constantes } \RR \to \RR \} \subset
    \{ \text{funciones polinomiales } \RR \to \RR \} \\
    \subset \{ \text{funciones continuas } \RR \to \RR \} \subset
    \{ \text{funciones } \RR \to \RR \}.
  \end{multline*}

  \ifdefined\solutions
  \begin{solucion}
    Basta hacer las siguientes observaciones.

    \begin{itemize}
    \item Las funciones constantes $x\mapsto 0$ y $x\mapsto 1$ están en todos
      los conjuntos.

    \item De análisis sabemos que si $f,g\colon \RR \to \RR$ son funciones
      continuas, entonces $f\pm g$ y $f\cdot g$ son también continuas.

    \item Toda función polinomial es continua, y si $f$ y $g$ son polinomiales,
      entonces $f\pm g$ y $f\cdot g$ son también polinomiales.

    \item Toda función constante es un caso muy especial de funciones
      polinomiales. Si $f$ y $g$ son constantes, entonces $f\pm g$ y $f\cdot g$
      son constantes.
    \end{itemize}
  \end{solucion}
  \fi
\end{ejercicio}

\subsection*{Homomorfismos de anillos}

\begin{ejercicio}
  Sea $R$ un anillo conmutativo y $M_n (R)$ el anillo de las matrices
  de $n\times n$ con coeficientes en $R$. ¿Cuáles aplicaciones de abajo
  son homomorfismos?

  \begin{enumerate}
  \item[1)] La proyección
    $$\begin{pmatrix}
      x_{11} & x_{12} & \cdots & x_{1n} \\
      x_{21} & x_{22} & \cdots & x_{2n} \\
      \vdots & \vdots & \ddots & \vdots \\
      x_{n1} & x_{n2} & \cdots & x_{nn}
    \end{pmatrix} \mapsto x_{11}.$$

  \item[2)] La traza
    $$\begin{pmatrix}
      x_{11} & x_{12} & \cdots & x_{1n} \\
      x_{21} & x_{22} & \cdots & x_{2n} \\
      \vdots & \vdots & \ddots & \vdots \\
      x_{n1} & x_{n2} & \cdots & x_{nn}
    \end{pmatrix} \mapsto x_{11} + x_{22} + \cdots + x_{nn}.$$

  \item[3)] El determinante $A \mapsto \det A$.
  \end{enumerate}

  \ifdefined\solutions
  \begin{solucion}
    Notamos que para el producto de dos matrices
    \[ (x_{ij})_{1 \le i,j\le n}\cdot (y_{ij})_{1 \le i,j\le n} =
       (z_{ij})_{1 \le i,j\le n}\]
    tenemos
    $$z_{11} = \sum_{1\le k\le n} x_{1k}\,y_{k1}.$$
    Este número no tiene por qué ser igual a $x_{11}\,y_{11}$. Por ejemplo,
    $$\begin{pmatrix}
      1 & 1 \\
      1 & 1
    \end{pmatrix}\cdot \begin{pmatrix}
      1 & 1 \\
      1 & 1
    \end{pmatrix} = \begin{pmatrix}
      2 & 2 \\
      2 & 2
    \end{pmatrix}.$$
    Entonces, la primera aplicación no preserva productos, aunque visiblemente
    preserva las sumas y la identidad.

    Para la traza tenemos $\tr (A + B) = \tr A + \tr B$. Sin embargo, la traza
    de la matriz identidad $I$ de $n\times n$ es igual a $n$, así que
    normalmente la traza no preserva la identidad. Para los productos
    de matrices, $\tr (AB) \ne \tr (A)\,\tr (B)$. Por ejemplo,
    $\tr (I^2) = \tr (I) = n \ne \tr (I)^2$.

    El determinante preserva los productos: $\det (AB) = \det (A\,B)$ y también
    preserva la identidad, pero evidentemente no preserva las sumas (toda matriz
    se expresa como una suma de matrices de determinante $0$).

    Con esto podemos concluir que \emph{ninguna} de las aplicaciones de la lista
    es un homomorfismo de anillos.
  \end{solucion}
  \fi
\end{ejercicio}

% \begin{ejercicio}
%   Encuentre un ejemplo de aplicación entre anillos $f\colon R\to S$ que
%   satisface $f (x+y) = f(x) + f (y)$ y $f (xy) = f(x)\,f(y)$ para cualesquiera
%   $x,y\in R$, pero no satisface $f (1_R) = 1_S$.
% \end{ejercicio}

\begin{ejercicio}
  Sea $f\colon R\to S$ un homomorfismo de anillos conmutativos y sea
  $n = 1,2,3,\ldots$

  \begin{enumerate}
  \item[1)] Demuestre que $f$ induce un homomorfismo de los anillos de matrices
    correspondientes $f_*\colon M_n (R) \to M_n (S)$ dado por
    \[ \begin{pmatrix}
        x_{11} & x_{12} & \cdots & x_{1n} \\
        x_{21} & x_{22} & \cdots & x_{2n} \\
        \vdots & \vdots & \ddots & \vdots \\
        x_{n1} & x_{n2} & \cdots & x_{nn}
      \end{pmatrix}
      \mapsto
      \begin{pmatrix}
        f (x_{11}) & f (x_{12}) & \cdots & f (x_{1n}) \\
        f (x_{21}) & f (x_{22}) & \cdots & f (x_{2n}) \\
        \vdots & \vdots & \ddots & \vdots \\
        f (x_{n1}) & f (x_{n2}) & \cdots & f (x_{nn})
      \end{pmatrix}. \]

  \item[2)] Demuestre que $f$ induce un homomorfismo de grupos
    $\GL_n (f)\colon \GL_n (R) \to \GL_n (S)$.

  \item[3)] Demuestre que el diagrama de homomorfismos de grupos
    \[ \begin{tikzcd}
        \GL_n (R) \ar{r}{\det}\ar{d}[swap]{\GL_n (f)} &
        R^\times\ar{d}{f^\times} \\
        \GL_n (S) \ar{r}[swap]{\det} & S^\times
      \end{tikzcd} \]
    conmuta.

    (Sugerencia: use la fórmula
    $\det (x_{ij}) = \sum_{\sigma\in S_n}\sgn\sigma\cdot x_{1\sigma (1)}\cdots x_{n\sigma (n)}$.)
  \end{enumerate}

  \ifdefined\solutions
  \begin{solucion}
    La suma y producto de matrices vienen dados por la fórmula
    \[ (x_{ij})_{1 \le i,j\le n} + (y_{ij})_{1 \le i,j\le n} =
       (x_{ij} + y_{ij})_{1 \le i,j\le n}, \quad
       (x_{ij})_{1 \le i,j\le n}\cdot (y_{ij})_{1 \le i,j\le n} =
       (z_{ij})_{1 \le i,j\le n}, ~ z_{ij} =
       \sum_{1\le k\le n} x_{ik}\,y_{kj}. \]
    Si $f\colon R\to S$ es un homomorfismo, entonces $f$ preserva las sumas y
    por lo tanto
    \begin{multline*}
      f_* ((x_{ij})_{1 \le i,j\le n} + (y_{ij})_{1 \le i,j\le n}) =
      f_* ((x_{ij} + y_{ij})_{1 \le i,j\le n}) \dfn (f (x_{ij} + y_{ij}))_{1 \le i,j\le n} =
      (f (x_{ij}) + f (y_{ij}))_{1 \le i,j\le n} \\
      = (f (x_{ij}))_{1 \le i,j\le n} + (f (y_{ij}))_{1 \le i,j\le n} \rdfn
      f_* ((x_{ij})_{1 \le i,j\le n}) + f_* ((y_{ij})_{1 \le i,j\le n}).
    \end{multline*}
    Además, $f (0_R) = 0_S$ y $f (1_R) = 1_S$, así que $f_*$ preserva la matriz
    identidad. Para los productos de matrices, usando el hecho de que $f$
    preserve las sumas y productos, se obtiene
    \[ f (z_{ij}) = f \Bigl(\sum_{1\le k\le n} x_{ik}\,y_{kj}\Bigr) =
       \sum_{1\le k\le n} f (x_{ik})\,f (y_{kj}). \]
    Entonces,
    \[ f_* ((x_{ij})_{1 \le i,j\le n} \cdot (y_{ij})_{1 \le i,j\le n}) =
       f_* ((x_{ij})_{1 \le i,j\le n}) \cdot f_* ((y_{ij})_{1 \le i,j\le n}). \]
    Esto termina la verificación de la parte 1).

    En la parte 2), basta recordar que todo homomorfismo de anillos
    $\phi\colon A\to B$ se restringe a un homomorfismo de grupos
    $\phi^\times\colon A^\times \to B^\times$. En nuestro caso, $f\colon R\to S$
    induce un homomorfismo de anillos $f_*\colon M_n (R) \to M_n (S)$, y luego
    un homomorfismo de grupos
    $f_*^\times\colon M_n (R)^\times \to M_n (S)^\times$, donde
    $M_n (R)^\times = \GL_n (R)$ y $M_n (S)^\times = \GL_n (S)$.

    En la parte 3), si tenemos una matriz invertible $(x_{ij}) \in \GL_n (R)$,
    entonces la conmutatividad del diagrama quiere decir nada más que
    \[ f (\det (x_{ij}) ) =
       f \Bigl(\sum_{\sigma\in S_n} \sgn\sigma \cdot x_{1\sigma (1)}\cdots x_{n\sigma (n)}\Bigr) =
       \sum_{\sigma\in S_n} \sgn\sigma \cdot f (x_{1\sigma (1)})\cdots f (x_{n\sigma (n)}) =
       \det (f (x_{ij}))_{1\le i,j\le n} \]
    (usando que $f$ preserva sumas y productos).
  \end{solucion}
  \fi
\end{ejercicio}

\begin{ejercicio}
  Sea $R$ un anillo conmutativo. Calcule $Z (M_n (R))$, el centro del anillo
  de las matrices de $n\times n$ con coeficientes en $R$.

  \noindent (Véanse los ejercicios donde calculamos el centro del grupo lineal
  general $\GL_n (R)$.)

  \ifdefined\solutions
  \begin{solucion}
    Denotemos por $e_{ij}$ la matriz que tiene ceros en todas las entradas y $1$
    en la entrada $(i,j)$. Luego, para $A = (x_{ij})$ todas las entradas
    de $e_{ij} A$ son nulas, salvo la $i$-ésima fila donde está la $j$-ésima
    fila de $A$. La matriz $A e_{ij}$ tiene como su $j$-ésima columna
    la $i$-ésima columna de $A$. Las identidades $e_{ij} A = A e_{ij}$ para todo
    $1 \le i,j \le n$ implican que $A$ tiene elementos nulos afuera de
    la diagonal y los elementos de la diagonal de $A$ son iguales. Entonces,
    las matrices que conmutan con $e_{ij}$ son las matrices escalares.
    Además, está claro que las matrices escalares conmutan con todas
    las matrices. Podemos concluir que $Z (M_n (R)) \isom R$ consiste en
    las matrices escalares.
  \end{solucion}
  \fi
\end{ejercicio}

\begin{ejercicio}
  ~

  \begin{enumerate}
  \item[1)] Demuestre que un isomorfismo de anillos $R\to S$ se restringe a
    un isomorfismo de grupos $R^\times \to S^\times$.

  \item[2)] Demuestre que los anillos de polinomios $\ZZ [X]$ y $\QQ [X]$ no son
    isomorfos.
  \end{enumerate}

  \ifdefined\solutions
  \begin{solucion}
    Por la definición, si $f\colon R\to S$ es un isomorfismo de anillos,
    entonces existe un homomorfismo $g\colon S\to R$ tal que
    $$g\circ f = \id{R} \quad\text{y}\quad f\circ g = \id{S}.$$
    Hemos visto que todo homomorfismo de anillos $f\colon R\to S$ se restringe
    a un homomorfismo de grupos $f^\times\colon R^\times\to S^\times$.
    De la misma manera, $g$ se restringe a un homomorfismo de grupos
    $g^\times\colon S^\times \to R^\times$. Esta restricción preserva
    las composiciones y las aplicaciones identidad, así que las identidades
    de arriba nos dan
    \[ g^\times \circ f^\times = (g\circ f)^\times =
       \id{R}^\times = \id{R^\times}
       \quad\text{y}\quad
       f^\times\circ g^\times = (f\circ g)^\times =
       \id{S}^\times = \id{S^\times}. \]
    Esto demuestra la parte 1). Para la parte 2), recordemos que si $R$ es
    un dominio de integridad, entonces $R [X]^\times \isom R^\times$:
    los polinomios invertibles son precisamente los polinomios constantes
    invertibles en $R$. Los grupos $\ZZ^\times$ y $\QQ^\times$ no son isomorfos,
    y por ende $\ZZ [X]^\times \not\isom \QQ [X]^\times$. Gracias a la parte 1),
    esto implica que $\ZZ [X] \not\isom \QQ [X]$.
  \end{solucion}
  \fi
\end{ejercicio}

% \begin{ejercicio}
%   Demuestre que el anillo $M_2 (\RR)$ contiene un subanillo que es isomorfo a
%   $\CC$.
% \end{ejercicio}

\begin{ejercicio}
  Sea $f\colon R\to S$ un homomorfismo sobreyectivo de anillos. Demuestre que
  $f (Z (R)) \subseteq Z (S)$.

  \ifdefined\solutions
  \begin{solucion}
    Si $f (x) \in f (Z (R))$, entonces $xx' = x'x$ para todo
    $x'\in R$. Aplicando $f$ a la última identidad, se obtiene
    $f (x)\,f (x') = f (x')\,f(x)$ para todo $x'\in R$. Si $f$ es sobreyectivo,
    entonces todo $y\in S$ es de la forma $f (x')$ para algún $x'\in R$. Tenemos
    $f (x)\,y = y\,f(x)$ para todo $y\in S$.
  \end{solucion}
  \fi
\end{ejercicio}

\subsection*{Álgebra de grupo}

\begin{ejercicio}
  Sea $R$ un anillo conmutativo y $G$ un grupo. Demuestre que
  $$\epsilon\colon R [G] \epi R, \quad \sum_{g\in G} a_g\,g \mapsto \sum_{g\in G} a_g$$
  es un homomorfismo sobreyectivo de anillos.

  \ifdefined\solutions
  \begin{solucion}
    La aplicación preserva las sumas:
    \[ \epsilon \Bigl(\sum_{g\in G} a_g\,g + \sum_{g\in G} a_g\,g\Bigr) =
       \epsilon \Bigl(\sum_{g\in G} (a_g + b_g)\,g\Bigr) =
       \sum_{g\in G} (a_g + b_g) =
       \sum_{g\in G} a_g + \sum_{g\in G} b_g =
       \epsilon \Bigl(\sum_{g\in G} a_g\,g\Bigr) +
       \epsilon \Bigl(\sum_{g\in G} b_g\,g\Bigr). \]
    La identidad en $R [G]$ es el elemento $\sum_{g\in G} a_g\,g$ donde
    $$a_g = \begin{cases}
      1, & g = e,\\
      0, & g \ne e,
    \end{cases}$$
    y la suma de sus coeficientes es $1$, así que $\epsilon$ preserva
    la identidad. Para los productos, tenemos
    \begin{equation}
      \tag{*}
      \epsilon \left(\Bigl(\sum_{h\in G} a_h\,h\Bigr) \cdot
        \Bigl(\sum_{k\in G} b_k\,k\Bigr)\right) =
      \epsilon \left(\sum_{g\in G} \Bigl(\sum_{hk = g} a_h b_k\Bigr)\,g\right) =
      \sum_{g\in G} \sum_{hk = g} a_h b_k
    \end{equation}
    y
    \begin{equation}
      \tag{**}
      \epsilon \Bigl(\sum_{h\in G} a_h\,h\Bigr) \cdot
      \epsilon \Bigl(\sum_{k\in G} b_k\,k\Bigr) =
      \Bigr(\sum_{h\in G} a_h\Bigr)\,\Bigr(\sum_{k\in G} b_k\Bigr) =
      \sum_{h\in G}\sum_{k\in G} a_h b_k.
    \end{equation}
    Las sumas (*) y (**) visiblemente coinciden, y por lo tanto $\epsilon$
    preserva productos.
  \end{solucion}
  \fi
\end{ejercicio}

\begin{ejercicio}
  \label{ejerc:cuadrado-del-elemento-de-la-traza}
  Sea $R$ un anillo conmutativo y $G$ un grupo finito. Consideremos
  $t \dfn \sum_{g\in G} 1\cdot g \in R [G]$. Demuestre que $t^2 = |G|\,t$.

  \ifdefined\solutions
  \begin{solucion}
    Tenemos
    $$\Bigr(\sum_{g\in G} g\Bigl)^2 = \sum_{g\in G} g\,\sum_{h\in G} h = \sum_{g\in G} \sum_{h\in G} gh.$$
    Para $g\in G$ fijo el conjunto $\{ gh \mid h\in G \}$ está en biyección con
    $G$, entonces la suma es igual a
    $$\sum_{g\in G} \sum_{h\in G} h = \sum_{g\in G} t = |G|\cdot t.$$
  \end{solucion}
  \fi
\end{ejercicio}

\begin{ejercicio}
  \label{ejerc:centro-de-RG}
  En este ejercicio vamos a calcular el centro del álgebra de grupo
  $R [G]$. Consideremos
  $$x = \sum_{g\in G} a_g\,g \in R [G].$$

  \begin{enumerate}
  \item[1)] Demuestre que $x \in Z (R [G])$ si y solamente si $h\,x = x\,h$ para
    todo $h\in G$.

  \item[2)] Deduzca que $x\in Z (R [G])$ si y solamente si $a_g = a_{hgh^{-1}}$
    para cualesquiera $g,h\in G$.
  \end{enumerate}

  Entonces, el centro de $R [G]$ consiste en los elementos
  $\sum_{g\in G} a_g\,g$ cuyos coeficientes $a_g$ son constantes sobre
  las clases de conjugación de $G$.

  \ifdefined\solutions
  \begin{solucion}
    Si $x$ está en el centro, en particular $x$ tiene que conmutar con
    los elementos $h \in G$. Viceversa, si $x$ conmuta con cada $h\in G$,
    entonces $x$ conmuta con $a_h\,h$ para todo $a_h\in R$ y con todas las sumas
    $\sum_{h\in G} a_h\,h \in R [G]$. Entonces,
    $$Z (R [G]) = \{ x \in R [G] \mid xh = hx \text{ para todo }h\in G \}.$$
    Luego,
    \[ xh = \sum_{g\in G} a_g\,gh = \sum_{g\in G} a_{gh^{-1}}\,g
       \quad\text{y}\quad
       hx = \sum_{g\in G} a_g\,hg = \sum_{g\in G} a_{h^{-1}g}\,g. \]
    Entonces, $xh = hx$ si y solo si $a_{gh^{-1}} = a_{h^{-1}g}$. Reemplazando
    $g$ por $hg$, se obtiene la condición $a_{hgh^{-1}} = a_g$ para todo $g$.
  \end{solucion}
  \fi
\end{ejercicio}

\begin{ejercicio}
  Sea $R$ un anillo conmutativo y $G$ un grupo. Consideremos el homomorfismo
  \[ \epsilon\colon R [G] \epi R,
     \quad
     \sum_{g\in G} a_g\,g \mapsto \sum_{g\in G} a_g. \]

  \begin{enumerate}
  \item[1)] Demuestre que el ideal $I_G \dfn \ker \epsilon$ está generado por
    los elementos $g-e$ para $g\in G$. Este se llama el \term{ideal de aumento}.

  \item[2)] En particular, si $G = C_n = \{ e, g, \ldots, g^{n-1} \}$ es
    el grupo cíclico de orden $n$ generado por $g$, demuestre que
    $\ker \epsilon$ está generado por el elemento $g-e$.
  \end{enumerate}
\end{ejercicio}

\subsection*{Ideales}

\begin{ejercicio}
  Sea $R$ un anillo y $S \subseteq R$ un subanillo.

  \begin{enumerate}
  \item[1)] Demuestre que para todo ideal $I \subseteq R$ (izquierdo, derecho,
    bilateral) la intersección $I\cap S$ es un ideal en $S$ (izquierdo, derecho,
    bilateral).

  \item[2)] Encuentre un ejemplo de $S \subseteq R$ donde no todos los ideales
    de $S$ son de la forma $I\cap S$.
  \end{enumerate}
\end{ejercicio}

\begin{ejercicio}
  \label{ejerc:producto-de-anillos-distributivo}
  Sea $R$ un anillo y sean $I,J,K$ ideales bilaterales. Demuestre que
  $I\,(J+K) = IJ + IK$ y $(I+J)\,K = IK + JK$.
\end{ejercicio}

\begin{ejercicio}
  Sea $R$ un anillo. Para un ideal izquierdo $I\subseteq R$ definamos
  el \term{aniquilador} por
  $$\Ann I \dfn \{ r\in R \mid rx = 0\text{ para todo }x\in I \}.$$
  Demuestre que esto es un ideal bilateral en $R$.
\end{ejercicio}

\begin{ejercicio}
  \label{ejerc:ideales-en-MnR}
  Sea $R$ un anillo conmutativo y $M_n (R)$ el anillo de matrices
  correspondiente.

  \begin{enumerate}
  \item[1)] Sea $I \subseteq R$ un ideal. Denotemos por $M_n (I)$ el conjunto
    de las matrices que tienen como sus entradas elementos de $I$. Verifique que
    $M_n (I)$ es un ideal bilateral en $M_n (R)$.

  \item[2)] Sea $J \subseteq M_n (R)$ un ideal bilateral. Sea $I$ el conjunto
    de los coeficientes que aparecen en la entrada $(1,1)$ de las matrices
    que pertenecen a $J$. Demuestre que $I$ es un ideal en $R$.

  \item[3)] Demuestre que $J = M_n (I)$. (Use las mismas ideas
    de \ref{obs:ideales-en-Mnk}.)
  \end{enumerate}

  Entonces, todo ideal en el anillo de matrices $M_n (R)$ es de la forma
  $M_n (I)$ para algún ideal $I \subseteq R$. Esto generaliza el resultado
  de \ref{obs:ideales-en-Mnk}.
\end{ejercicio}

\subsection*{Nilradical y radical}

\begin{ejercicio}
  Sea $R$ un anillo conmutativo.

  \begin{enumerate}
  \item[1)] Demuestre que el conjunto de nilpotentes
    $$N (R) \dfn \{ x\in R \mid x^n = 0\text{ para algún }n = 1,2,3,\ldots \}$$
    es un ideal en $R$. Este se llama el \term{nilradical} de $R$.

  \item[2)] Demuestre que en el anillo no conmutativo $M_n (R)$ los nilpotentes
    no forman un ideal.
  \end{enumerate}
\end{ejercicio}

\begin{ejercicio}
  Sea $R$ un anillo conmutativo. Supongamos que el nilradical de $R$
  es finitamente generado; es decir, $N (R) = (x_1,\ldots,x_n)$ donde $x_i$
  son algunos nilpotentes. Demuestre que en este caso $N (R)$ es
  un \term{ideal nilpotente}:
  $$N (R)^m \dfn \underbrace{N (R)\cdots N (R)}_m = 0$$
  para algún $m = 1,2,3,\ldots$
\end{ejercicio}

\begin{ejercicio}
  Sea $R$ un anillo conmutativo y sea $I \subseteq R$ un ideal. Demuestre que
  \[ \sqrt{I} \dfn
     \{ x\in R \mid x^n \in I \text{ para algún }n = 1,2,3,\ldots \} \]
  es también un ideal en $R$, llamado el \term{radical} de $I$.
  (Note que el nilradical $N (R) = \sqrt{(0)}$ es un caso particular.)
\end{ejercicio}

\subsection*{Operaciones I y V}

\begin{ejercicio}[*]
  \label{ejerc:relaciones-para-I-y-V}
  Sea $k$ un cuerpo. Sean $J, J_1, J_2$ ideales en $k [X_1,\ldots,X_n]$ y sean
  $X,Y$ subconjuntos de $\AA^n (k)$. Demuestre las siguientes relaciones.

  \begin{enumerate}
  \item[0)] $I (\emptyset) = k [X_1,\ldots,X_n]$, $V (0) = \AA^n (k)$,
    $V (1) = V (k [X_1,\ldots,X_n]) = \emptyset$.

  \item[1)] Si $J_1 \subseteq J_2$, entonces $V (J_2) \subseteq V (J_1)$.

  \item[2)] Si $X\subseteq Y$, entonces $I (Y) \subseteq I (X)$.

  \item[3)] $V (J) = V (\sqrt{J})$.

  \item[4)] $J \subseteq \sqrt{J} \subseteq I V (J)$. Demuestre que la inclusión
    es estricta para $J = (X^2 + 1) \subset \RR [X]$.

  \item[5)] $X \subseteq V I (X)$.

  \item[6)] $VIV (J) = V (J)$ y $IVI (X) = I (X)$.
  \end{enumerate}
\end{ejercicio}

\begin{ejercicio}[**]
  \label{ejerc:IAn}
  ~

  \begin{enumerate}
  \item[1)] Demuestre que $I (\AA^n (k)) = (0)$ si $k$ es un cuerpo infinito.

  \item[2)] Note que $X^p - X \in I (\AA^1 (\FF_p))$, así que esto es falso para
    cuerpos finitos.
  \end{enumerate}
\end{ejercicio}

\subsection*{Anillos cociente}

\begin{ejercicio}
  Sea $R$ un anillo conmutativo y sea $N (R)$ su nilradical. Demuestre que
  el anillo cociente $R/N(R)$ no tiene nilpotentes; es decir,
  $N (R / N (R)) = 0$.
\end{ejercicio}

\begin{ejercicio}
  Sea $R$ un anillo conmutativo y sea $I \subseteq R$ un ideal. Demuestre que
  $M_n (R) / M_n (I) \isom M_n (R/I)$.
\end{ejercicio}

\begin{ejercicio}
  Sea $k$ un cuerpo y $c\in k$. Consideremos el homomorfismo de evaluación
  $$ev_c\colon k [X] \to k, \quad f \mapsto f (c).$$

  \begin{enumerate}
  \item[1)] Demuestre que $\ker ev_c = (X - c)$ es el ideal generado por
    el polinomio lineal $X - c$.

  \item[2)] Deduzca del primer teorema de isomorfía que $k [X] / (X-c) \isom k$.

  \item[3*)] De modo similar, demuestre que para $c_1,\ldots,c_n \in k$ se tiene
    $k [X_1,\ldots,X_n] / (X_1 - c_1, \ldots, X_n - c_n) \isom k$.

    Sugerencia: considere el automorfismo de $k [X_1,\ldots,X_n]$ dado por
    $X_i \mapsto X_i + c_i$.
  \end{enumerate}
\end{ejercicio}

\begin{ejercicio}
  Demuestre que el cociente $\QQ [X] / (X^2 + 5)$ es isomorfo al cuerpo
  $$\QQ (\sqrt{-5}) \dfn \{ x + y\sqrt{-5} \mid x,y \in \QQ \}$$
  (en particular, verifique que $\QQ (\sqrt{-5})$ es un cuerpo).
\end{ejercicio}

\begin{ejercicio}
  Para el anillo de los enteros de Gauss $\ZZ [\sqrt{-1}]$ demuestre que
  \[ \ZZ [\sqrt{-1}]/(1 + \sqrt{-1}) \isom \ZZ/2\ZZ,
     \quad
     \ZZ [\sqrt{-1}]/(1 + 2\sqrt{-1}) \isom \ZZ/5\ZZ. \]
\end{ejercicio}

\begin{ejercicio}
  Consideremos el anillo de los enteros de Gauss $\ZZ [\sqrt{-1}]$ y los ideales
  $$I = (1 + \sqrt{-1}), \quad J = (1 + 2\sqrt{-1}).$$

  \begin{enumerate}
  \item[1)] Demuestre que $I + J = \ZZ [\sqrt{-1}]$.

  \item[2)] Demuestre que $IJ = (1 - 3\sqrt{-1})$.

    Sugerencia: note que en cualquier anillo conmutativo, se tiene
    $(x)\cdot (y) = (xy)$ para cualesquiera $x,y\in R$.

  \item[3)] Demuestre que
    $\ZZ [\sqrt{-1}]/(1 - 3\sqrt{-1}) \isom \ZZ/2\ZZ \times \ZZ/5\ZZ$ usando
    el teorema chino del resto.
  \end{enumerate}
\end{ejercicio}

\begin{ejercicio}
  Demuestre el segundo teorema de isomorfía para anillos.
\end{ejercicio}

\begin{ejercicio}
  Demuestre el tercer teorema de isomorfía para anillos.
\end{ejercicio}

\subsection*{Productos de anillos}

\begin{ejercicio}
  Sean $R$ y $S$ anillos y sean $I \subseteq R$, $J \subseteq S$ ideales
  bilaterales.

  \begin{enumerate}
  \item[1)] Demuestre que
    $$I\times J \dfn \{ (x,y) \mid x\in I, ~ y\in J \}$$
    es un ideal bilateral en el producto $R\times S$.

  \item[2)] Demuestre que todos los ideales bilaterales en $R\times S$ son
    de esta forma.

    Sugerencia: para un ideal bilateral $A \subseteq R\times S$ considere
    $I = p_1 (A)$ y $J = p_2 (A)$ donde
    $$\begin{tikzcd}[row sep=0pt]
      R & R\times S\ar{l}[swap]{p_1}\ar{r}{p_2} & S \\
      r & (r,s) \ar[|->]{l}\ar[|->]{r} & s
    \end{tikzcd}$$
    son las proyecciones canónicas.
  \end{enumerate}
\end{ejercicio}

\begin{ejercicio}
  ~

  \begin{enumerate}
  \item[1)] Sean $R$ y $S$ dos anillos no nulos. Demuestre que el producto
    $R\times S$ tiene divisores de cero.

  \item[2)] Demuestre que el producto de dos anillos no nulos nunca es
    un cuerpo.

    Sugerencia: para un ideal bilateral $A \subseteq R\times S$ considere
    $I = p_1 (A)$ y $J = p_2 (A)$ donde
    $$\begin{tikzcd}[row sep=0pt]
      R & R\times S\ar{l}[swap]{p_1}\ar{r}{p_2} & S \\
      r & (r,s) \ar[|->]{l}\ar[|->]{r} & s
    \end{tikzcd}$$
    son las proyecciones canónicas.
  \end{enumerate}
\end{ejercicio}

% \begin{ejercicio}
%   Sea $R \dfn \prod_{i\in \NN} \QQ$ el producto de copias de $\QQ$ indexado
%   por los números naturales. Este anillo consiste en las sucesiones
%   $(x_n)_{n\in \NN}$ donde $x_n \in \QQ$ y las operaciones son término por
%   término.
%
%   \begin{enumerate}
%   \item[1)] Recordemos que $(x_n)_{n\in \NN}$ es una \term{sucesión de Cauchy}
%     si para todo $\epsilon > 0$ existe $N$ tal que
%     $$m,n > N \Rightarrow |x_m - x_n| < \epsilon.$$
%     Demuestre que las sucesiones de Cauchy forman un subanillo $S \subset R$.
%
%   \item[2)] Sea $N$ el subconjunto formado por las \term{sucesiones nulas}; es
%     decir, las sucesiones tales que $\lim\limits_{n\to \infty} x_n = 0$; en
%     otras palabras, tales que para todo $\epsilon > 0$ existe $N$ que
%     satisface
%     $$n > N \Rightarrow |x_n| < \epsilon.$$
%     Demuestre que $N$ es un ideal en $S$.
%   \end{enumerate}
%
%   (El anillo cociente $S/N$ es isomorfo al cuerpo de los números reales
%   $\RR$.)
% \end{ejercicio}
