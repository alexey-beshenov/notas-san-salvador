\subsection*{Ideales primos y maximales}

\begin{ejercicio}
  Sea $\mathcal{C} (\RR)$ el anillo de las funciones continuas
  $f\colon \RR\to \RR$ con operaciones punto por punto. Demuestre que para
  cualquier $x\in \RR$
  \[ \mathfrak{m}_x \dfn
     \{ \text{funciones continuas } f\colon \RR\to \RR \mid f (x) = 0 \} \]
  es un ideal maximal en $\mathcal{C} (\RR)$.
\end{ejercicio}

\begin{ejercicio}
  Determine si el ideal generado por el polinomio $X^2 + 1$ es primo o maximal
  en el anillo
  $$\RR [X], \quad \CC [X], \quad \ZZ [X], \quad \FF_2 [X].$$
\end{ejercicio}

\begin{ejercicio}
  Sea $R$ un anillo conmutativo y sea $\mathfrak{p} \subset R$ un ideal
  primo. Demuestre que si $x^n \in \mathfrak{p}$ para algún $x\in R$
  y $n = 1,2,3,\ldots$, entonces $x\in \mathfrak{p}$.
\end{ejercicio}

\begin{ejercicio}
  Sea $f\colon R\to S$ un homomorfismo de anillos conmutativos. Para un ideal
  primo $\mathfrak{p}\subset S$ verifique directamente que
  $f^{-1} (\mathfrak{p})$ es un ideal primo en $R$.
\end{ejercicio}

\begin{ejercicio}
  Sea $R$ un anillo conmutativo y sea $\mathfrak{p} \subset R$ un ideal primo.

  \begin{enumerate}
  \item[1)] Demuestre que para dos ideales $I, J \subseteq R$,
    si $IJ \subseteq \mathfrak{p}$, entonces $I \subseteq \mathfrak{p}$ o
    $J \subseteq \mathfrak{p}$.

  \item[2)] Demuestre que si para un ideal $I \subseteq R$ se tiene
    $I^n \subseteq \mathfrak{p}$ para algún $n = 1,2,3,\ldots$, entonces
    $I \subseteq \mathfrak{p}$.

  \item[3)] Demuestre que si $I \cap J \subseteq \mathfrak{p}$, entonces
    $I \subseteq \mathfrak{p}$ o $J \subseteq \mathfrak{p}$.
  \end{enumerate}
\end{ejercicio}

\begin{ejercicio}
  Sean $R$ un anillo conmutativo e $I \subseteq R$ un ideal.

  \begin{enumerate}
  \item[1)] Demuestre que
    $$I [X] \dfn \Bigl\{ \sum_i a_i\,X^i \in R [X] \Bigm| a_i \in I \}$$
    es un ideal en el anillo de polinomios $R [X]$ y
    $$R [X] / I [X] \isom (R/I) [X].$$

  \item[2)] Demuestre que si $\mathfrak{p} \subset R$ es un ideal primo,
    entonces el ideal $\mathfrak{p} [X]$ es primo en $R [X]$.

  \item[3)] Demuestre que si $\mathfrak{m} \subset R$ es un ideal maximal,
    entonces el ideal $\mathfrak{m} [X]$ no es maximal en $R [X]$.
  \end{enumerate}
\end{ejercicio}

\begin{ejercicio}
  Sea $R$ un anillo conmutativo. Para un subconjunto $S \subseteq R$ sea $V (S)$
  el conjunto de los ideales primos que contienen a $S$:
  $$V (S) \dfn \{ \mathfrak{p} \in \Spec R \mid \mathfrak{p} \supseteq S \}.$$

  \begin{enumerate}
  \item[1)] Demuestre que para $S_1 \subseteq S_2 \subseteq R$ se tiene
    $V (S_2) \subseteq V (S_1)$.

  \item[2)] Demuestre que $V (S) = V (I)$ donde $I = (S)$ es el ideal generado
    por $S$.

  \item[3)] Demuestre que $V (0) = \Spec R$ y $V (1) = \emptyset$.

  \item[4)] Demuestre que $V (I) \cup V (J) = V (I \cap J) = V (IJ)$ para
    ideales $I,J \subseteq R$.

  \item[5)] Demuestre que $\bigcap_k V (I_k) = V (\sum_k I_k)$ para ideales
    $I_k \subseteq R$.
  \end{enumerate}
\end{ejercicio}

\begin{ejercicio}
  Sean $R$ y $S$ anillos conmutativos. Consideremos el producto $R\times S$ con
  las proyecciones canónicas
  $$\begin{tikzcd}[row sep=0pt]
    R & R\times S\ar{l}[swap]{\pi_1}\ar{r}{\pi_2} & S \\
    r & (r,s) \ar[|->]{l}\ar[|->]{r} & s
  \end{tikzcd}$$

  \begin{enumerate}
  \item[1)] Si $\mathfrak{p} \subset R$ y $\mathfrak{q} \subset S$ son ideales
    primos, demuestre que
    \[ \mathfrak{p}\times S \dfn \pi_1^{-1} (\mathfrak{p}) =
       \{ (x,s) \mid x \in \mathfrak{p}, ~ s\in S \}, \quad
       R\times \mathfrak{q} =
       \pi_2^{-1} (\mathfrak{q}) \dfn
       \{ (r,y) \mid r\in R, ~ y\in \mathfrak{q} \} \]
    son ideales primos en el producto $R\times S$.

  \item[2)] Demuestre que si $\mathfrak{P} \subset R\times S$ es un ideal primo,
    entonces $\mathfrak{P}$ es de la forma $\mathfrak{p}\times S$ o
    $R\times \mathfrak{q}$ como en 1).

    Indicación: para $e_1 \dfn (1_R,0_S)$ y $e_2 \dfn (0_R,1_S)$ note que
    $e_1\,e_2 \in \mathfrak{P}$, así que $e_1 \in \mathfrak{P}$ o
    $e_2 \in \mathfrak{P}$.
  \end{enumerate}

  Ensto nos da una biyección natural
  $\Spec (R\times S) \isom \Spec R \sqcup \Spec S$.
\end{ejercicio}

\subsection*{Lema de Zorn}

\begin{ejercicio}
  \label{ejerc:primos-sin-interseccion-con-conj-mult}
  Sea $R$ un anillo conmutativo. Sea $U \subset R$ un subconjunto no vacío tal
  que $0 \notin U$ y si $x,y \in U$, entonces $xy \in U$.

  \begin{enumerate}
  \item[1)] Deduzca del lema de Zorn que existe un ideal
    $\mathfrak{p} \subset R$ que satisface las siguientes propiedades:

    \begin{itemize}
    \item $U \cap \mathfrak{p} = \emptyset$,

    \item Si $\mathfrak{p} \subseteq I$ para otro ideal $I$ que satisface
      $U \cap I = \emptyset$, entonces $I = \mathfrak{p}$.
    \end{itemize}

  \item[2)] Demuestre que $\mathfrak{p}$ es un ideal primo.
  \end{enumerate}

  Indicación: basta revisar y entender nuestra prueba de que
  $N (R) = \bigcap_{\mathfrak{p} \subset R\text{ primo}} \mathfrak{p}$.
\end{ejercicio}

\begin{ejercicio}
  Sean $R$ un anillo conmutativo no nulo e $I \subset R$ un ideal propio.

  \begin{enumerate}
  \item[1)] Sea
    $R \supset \mathfrak{p}_0 \supseteq \mathfrak{p}_1 \supseteq
    \mathfrak{p}_2 \supseteq \cdots \supseteq I$ una cadena descendente de
    ideales primos que contienen a $I$. Demuestre que
    $\mathfrak{p} \dfn \bigcap_i \mathfrak{p}_i$ es un ideal primo que contiene
    a $I$.

  \item[2)] Deduzca del lema de Zorn que en $R$ existen \term{ideales primos
      minimales sobre $I$}; es decir, ideales primos
    $I \subseteq \mathfrak{p} \subset R$ tales que si
    $\mathfrak{q} \subseteq \mathfrak{p}$ para otro ideal primo
    $I \subseteq \mathfrak{q} \subset R$, entonces
    $\mathfrak{q} = \mathfrak{p}$.
  \end{enumerate}
\end{ejercicio}

\begin{ejercicio}
  Sea $R$ un anillo conmutativo noetheriano. En este ejercicio vamos a probar
  que para todo ideal propio no nulo $I \subset R$ (es decir, $I \ne R$,
  $I \ne 0$)
  \begin{equation}
    \tag{*}\text{existen ideales primos }\mathfrak{p}_1,\ldots,\mathfrak{p}_n\ne 0
    \text{ tales que }\mathfrak{p}_1\cdots \mathfrak{p}_n \subseteq I.
  \end{equation}

  Para llegar a una contradicción, asumamos que esto es falso y existen ideales
  propios no nulos que no cumplen la propiedad (*).

  \begin{enumerate}
  \item[1)] Demuestre usando el lema de Zorn que en este caso existe un ideal
    propio no nulo $I$ que es maximal entre los ideales que no cumplen
    la propiedad (*). Demuestre que $I$ no es primo, así que existen $x,y \in R$
    tales que $xy \in I$, pero $x \notin I$ e $y \notin I$.

  \item[2)] Demuestre que para los ideales $A \dfn I + (x)$ y $B \dfn I + (y)$
    se tiene $AB \subseteq I$ y son ideales propios no nulos.

  \item[3)] Demuestre que $A$ y $B$ cumplen la propiedad (*): se tiene
    \[ \mathfrak{p}_1\cdots \mathfrak{p}_m \subseteq A, \quad
       \mathfrak{q}_1\cdots \mathfrak{q}_n \subseteq B \]
    para algunos ideales primos
    $\mathfrak{p}_1,\ldots, \mathfrak{p}_m, \mathfrak{q}_1, \ldots,
    \mathfrak{q}_n \subset R$. Deduzca que
    $\mathfrak{p}_1\cdots \mathfrak{p}_m\,\mathfrak{q}_1\cdots\mathfrak{q}_n\subseteq I$.
  \end{enumerate}

  Concluya que hemos obtenido una contradicción.

  \ifdefined\solutions
  \begin{solucion}
    Consideremos el conjunto parcialmente ordenado $\mathcal{P}$ formado por
    los ideales propios no nulos que no cumplen la propiedad (*). Por nuestra
    hipótesis en 1), no es vacío. Gracias a la hipótesis noetheriana (!), toda
    cadena en $\mathcal{P}$ es finita:
    $$I_1 \subseteq I_2 \subseteq \cdots \subseteq I_n,$$
    e $I_n$ es una cota superior obvia. El lema de Zorn implica que existe un
    ideal $I$ que es maximal en $\mathcal{P}$. Notamos que los ideales primos no
    nulos trivialmente cumplen la propiedad (*), así que $I$ no es primo, lo que
    significa que existen $x,y\in R$ tales que $xy\in I$ pero $x,y\notin I$.

    Puesto que $x,y\notin I$, se tiene
    $$A \dfn I + (x) \supsetneq I, \quad B \dfn I + (y) \supsetneq I.$$
    Ahora $I \ne 0$ implica $A,B\ne 0$. Además,
    \[ AB = (I + (x))\,(I + (y)) =
       I\cdot (x) + I\cdot (y) + I^2 + (xy) \subseteq I. \]
     Notamos que si $A = R$, entonces $AB = A \subseteq I$, lo que es falso.
     De la misma manera, podemos descartar el caso de $B = R$. Esto significa
     que $A$ y $B$ son ideales propios no nulos estrictamente más grandes que
     $I$, así que por la maximalidad de $I$, estos tienen que cumplir la
     propiedad (*): para algunos ideales primos no nulos
     \[ \mathfrak{p}_1\cdots \mathfrak{p}_m \subseteq A, \quad
        \mathfrak{q}_1\cdots \mathfrak{q}_n \subseteq B. \]
     Pero en este caso
     $$\mathfrak{p}_1\cdots \mathfrak{p}_m\,\mathfrak{q}_1\cdots\mathfrak{q}_n \subseteq AB \subseteq I,$$
     y esto contradice nuestra elección de $I$.
   \end{solucion}
   \fi
\end{ejercicio}

\subsection*{Localización}

\begin{ejercicio}
  En el cuerpo de las series de Laurent $\QQ (\!(X)\!)$, encuentre el elemento
  inverso de $X - X^2$.
\end{ejercicio}

\begin{ejercicio}
  Describa los cuerpos de fracciones $K (R)$ para los anillos
  \[ R = \ZZ [\sqrt{-1}], ~
         \ZZ [\sqrt{5}], ~
         \ZZ \Bigl[\frac{1 + \sqrt{5}}{2}\Bigr]. \]
\end{ejercicio}

\begin{ejercicio}
  Sea $R\times S$ un producto de anillos conmutativos no nulos. Consideremos
  $e \dfn (1,0)$. Demuestre que $(R\times S) [e^{-1}] \isom R$.

  \noindent Sugerencia: nota que
  $\frac{(r,s)}{(1,0)} = \frac{(r,s)}{(1,1)} = \frac{(r,0)}{(1,1)}$ para
  cualesquiera $r\in R$ y $s\in S$.
\end{ejercicio}

% \begin{ejercicio}
%   \label{ejerc:propiedad-universal-de-localizacion}
%   Supongamos que $\psi\colon R\to S$ es un homomorfismo de anillos que
%   satisface la misma propiedad universal que el homomorfismo canónico de
%   localización $\phi\colon R \to R [U^{-1}]$:
%
%   \begin{enumerate}
%   \item[1)] para todo $u \in U$ el elemento $\psi (u)$ es invertible en $S$;
%
%   \item[2)] si $S'$ es otro anillo junto con un homomorfismo
%     $f\colon R \to S'$ tal que $f (u)$ es invertible en $S'$ para todo
%     $u\in U$, entonces $f$ se factoriza de modo único por $\psi$:
%     \[ \begin{tikzcd}
%         R\ar{r}{f}\ar{d}[swap]{\psi} & S' \\
%         S\ar[dashed]{ur}[swap]{\exists ! \widetilde{f}}
%       \end{tikzcd} \]
%   \end{enumerate}
%
%   Demuestre que existe un isomorfismo único $R [U^{-1}] \to S$ que hace
%   conmutar el diagrama
%   \[ \begin{tikzcd}
%       R\ar{r}{\psi}\ar{d}[swap]{\phi} & S \\
%       R [U^{-1}]\ar[dashed]{ur}{\isom}[swap]{\exists !}
%     \end{tikzcd} \]
%   (Aplique la propiedad universal de $\phi$ a $\psi$ y luego la propiedad
%   universal de $\psi$ a $\phi$.)
% \end{ejercicio}

\begin{ejercicio}
  Consideremos el anillo finito $R = \ZZ/n\ZZ$ donde
  $n = p_1^{k_1}\cdots p_s^{k_s}$.

  \begin{enumerate}
  \item[1)] Demuestre que los ideales maximales en $R$ son
    $\mathfrak{m}_i = p_i\,\ZZ/n\ZZ$ para $i = 1, \ldots, s$.

  \item[2)] Demuestre que
    $R \isom R_{\mathfrak{m}_1} \times \cdots \times R_{\mathfrak{m}_s}$.
  \end{enumerate}

  \noindent Sugerencia: demuestre que la aplicación canónica
  $\ZZ/n\ZZ \epi \ZZ/p_i^{k_i}\ZZ$ satisface la propiedad universal de
  la localización $\ZZ/n\ZZ \to (\ZZ/n\ZZ)_{\mathfrak{m}_i}$.
\end{ejercicio}

\begin{ejercicio}
  Sean $R$ un anillo conmutativo, $U \subseteq R$ un subconjunto multiplicativo
  y $\phi\colon R \to R [U^{-1}]$ el homomorfismo canónico de localización.

  \begin{enumerate}
  \item[1)] Para un ideal primo $\mathfrak{p} \subset R$ tal que
    $\mathfrak{p} \cap U = \emptyset$ compruebe directamente que el ideal
    $\mathfrak{p} R [U^{-1}] \subset R [U^{-1}]$ (es decir, que
    $\mathfrak{p} R [U^{-1}] \ne R [U^{-1}]$ y
    $\frac{r}{u}\cdot\frac{s}{v} \in \mathfrak{p} R [U^{-1}]$ implica
    $\frac{r}{u} \in \mathfrak{p} R [U^{-1}]$ o
    $\frac{s}{v} \in \mathfrak{p} R [U^{-1}]$).

  \item[2)] Para un ideal primo $\mathfrak{q} \subset R [U^{-1}]$ compruebe
    directamente que $\phi^{-1} (\mathfrak{q}) \cap U = \emptyset$ (use
    la definición original de ideales primos).
  \end{enumerate}
\end{ejercicio}

\begin{ejercicio}
  He aquí una generalización de las ideas que hemos ocupado para caracterizar
  los ideales en $R [U^{-1}]$. Para un homomorfismo de anillos $f\colon R\to S$
  e ideales $I \subseteq R$, $J \subseteq S$ definamos
  $$I^e \dfn f (I)\,S \subseteq S, \quad J^c \dfn f^{-1} (J) \subseteq R$$
  (el ideal $I^e$ se llama la \term{extensión} de $I$ y el ideal $J^c$ se llama
  la \term{contracción} de $J$). Verifique las siguientes propiedades de estas
  operaciones:

  \begin{enumerate}
  \item[1)] Si $I_1 \subseteq I_2$, entonces $I_1^e \subseteq I_2^e$.

  \item[2)] Si $J_1 \subseteq J_2$, entonces $J_1^c \subseteq J_2^c$.

  \item[3)] $(J_1 \cap J_2)^c = J_1^c \cap J_2^c$.

  \item[4)] $I \subseteq I^{ec}$, $J \supseteq J^{ce}$. Encuentre ejemplos
    cuando las inclusiones son estrictas.

  \item[5)] $J^c = J^{cec}$, $I^e = I^{ece}$.
  \end{enumerate}
\end{ejercicio}

\begin{ejercicio}
  Sea $n = p_1^{k_1}\cdots p_s^{k_s}$. Describa los ideales primos en el anillo
  \[ \ZZ \Bigl[\frac{1}{n}\Bigr] \dfn
     \Bigl\{ \frac{a}{n^k} \Bigm| a\in \ZZ, ~ k = 0,1,2,3,\ldots \Bigr\}. \]
\end{ejercicio}

\begin{ejercicio}
  Sea $R$ un anillo conmutativo y $U \subseteq R$ un subconjunto multiplicativo.

  \begin{enumerate}
  \item[1)] Para un ideal $I \subseteq R$ y un elemento $x\in R$ verifique que
    $(I : x) \dfn \{ r\in R \mid xr \in I \}$ es un ideal en $R$.

  \item[2)] Demuestre que hay una biyección entre los ideales en la localización
    $R [U^{-1}]$ y los ideales en $R$ tales que $(I : u) = I$ para todo
    $u\in U$.
  \end{enumerate}
\end{ejercicio}

\begin{ejercicio}
  Sea $R$ un anillo conmutativo. Denotemos por
  $$N (R) \dfn \{ x\in R \mid x^n = 0 \text{ para algún }n=1,2,3,\ldots \}$$
  el nilradical. Demuestre que para todo subconjunto multiplicativo
  $U \subseteq R$ se tiene
  $$N (R [U^{-1}]) = N (R) R [U^{-1}].$$

  \ifdefined\solutions
  \begin{solucion}
    Si $\frac{x}{u}\in N (R) R [U^{-1}]$, esto significa que $x^n = 0$ para
    algún $n$. Luego
    $\left(\frac{x}{u}\right)^n = \frac{x^n}{u^n} = \frac{0}{u^n} =
    \frac{0}{1}$, así que $\frac{x}{u} \in N (R [U^{-1}])$. Viceversa, si
    $\frac{x}{u} \in N (R [U^{-1}])$, entonces
    $\left(\frac{x}{u}\right)^n = \frac{x^n}{u^n} = \frac{0}{1} = 0$, lo que
    significa que $v\,x^n = 0$ para algún $v \in U$. Luego,
    $(vx)^n = v^n x^n = 0$. Escribiendo $\frac{x}{u} = \frac{vx}{uv}$, podemos
    concluir que $\frac{x}{u} \in N (R)\,R [U^{-1}]$.
  \end{solucion}
  \fi
\end{ejercicio}

\begin{ejercicio}
  Sean $R$ un anillo conmutativo y $x \in R$ algún elemento no nulo.

  \begin{enumerate}
  \item[1)] Demuestre que $\Ann (x) \dfn \{ r \in R \mid rx = 0 \}$ es un ideal
    propio en $R$.

  \item[2)] Demuestre que existe un ideal maximal $\mathfrak{m} \subset R$ tal
    que $\frac{x}{1} \ne \frac{0}{1}$ en la localización $R_\mathfrak{m}$.
  \end{enumerate}

  \ifdefined\solutions
  \begin{solucion}
    Es fácil comprobar que $\Ann (x)$ es un ideal y está claro que
    $1 \notin \Ann (x)$, dado que $x \ne 0$. Luego, el lema de Zorn nos
    garantiza que $\Ann (x) \subseteq \mathfrak{m}$ para algún ideal maximal
    $\mathfrak{m} \subset R$. Supongamos que $\frac{x}{1} = \frac{0}{1}$ en
    $R_\mathfrak{m}$. Esto significa que $ux = 0$ para algún
    $u \notin \mathfrak{m}$, lo que no es posible, puesto que todos los
    elementos que aniquilan a $x$ pertenecen al ideal $\mathfrak{m}$.
  \end{solucion}
  \fi
\end{ejercicio}

\subsection*{Anillos locales}

\begin{ejercicio}
  Sea $R$ un anillo local y sea $\mathfrak{m}$ su único ideal maximal. Demuestre
  que para cualquier $x\in R$ se cumple $x\in R^\times$ o $1-x\in R^\times$.
\end{ejercicio}

\begin{ejercicio}
  Demuestre que un anillo es local si y solo si todos los elementos no
  invertibles en $R$ forman un ideal.
\end{ejercicio}

\begin{ejercicio}
  Sea $k$ un cuerpo.

  \begin{enumerate}
  \item[1)] Demuestre que el anillo de series formales $k [\![X]\!]$ es local y
    su ideal maximal es $(X)$.

    \noindent Indicación: véase el ejercicio anterior.

  \item[2)] Demuestre que si $R$ es un anillo local con ideal maximal
    $\mathfrak{m}$, entonces $R [\![X]\!]$ es también local con ideal maximal
    $\mathfrak{m} + (X)$.

  \item[3)] Use la parte anterior para probar que $k [\![X_1,\ldots,X_n]\!]$ es
    local y su ideal maximal es $(X_1,\ldots,X_n)$.
  \end{enumerate}
\end{ejercicio}

\begin{ejercicio}
  Demuestre que si $R$ es un anillo local, entonces el cociente $R/I$ por
  cualquier ideal $I\subsetneq R$ es también un anillo local.
\end{ejercicio}

\begin{ejercicio}
  ~

  \begin{enumerate}
  \item[1)] Demuestre que para cualquier cuerpo $k$ el anillo de polinomios
    $k [X]$ no es local.

  \item[2)] Demuestre que el anillo de series de potencias $\ZZ [\![X]\!]$ no es
    local.
  \end{enumerate}
\end{ejercicio}

\subsection*{Anillos noetherianos}

\begin{ejercicio}
  Sea $R$ un anillo conmutativo noetheriano y $U \subseteq R$ un subconjunto
  multiplicativo. Demuestre que la localización $R [U^{-1}]$ es también
  un anillo noetheriano.
\end{ejercicio}

\begin{ejercicio}
  Se dice que un anillo es \term{artiniano}\footnote{\personality{Emil Artin}
    (1898--1962), algebrista y teórico de números alemán.} si toda cadena
  \emph{descendente} de ideales
  $$R \supseteq I_0 \supseteq I_1 \supseteq I_2 \supseteq \cdots$$
  se estabiliza. Note que $\ZZ$ es un anillo noetheriano, pero no es artiniano.
\end{ejercicio}

\begin{ejercicio}
  \label{ejerc:subconjunto-infinito-de-NxN}
  Sea $X$ un subconjunto infinito de $\NN\times\NN$. Demuestre que en $X$ hay
  un subconjunto infinito de pares $(k_\ell,d_\ell)$ para
  $\ell = 0,1,2,3,\ldots$ tal que
  $$k_0 \le k_1 \le k_2 \le \cdots, \quad d_0 \le d_1 \le d_2 \le \cdots$$
\end{ejercicio}
