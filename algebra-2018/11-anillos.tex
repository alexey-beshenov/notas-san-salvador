\chapter{Anillos}

Ya introducimos anillos y cuerpos en el capítulo 3. Ahora vamos a estudiar otros
conceptos relacionados y ver más detalles. Recordemos del capítulo 3 que
un \term{anillo} $R$ es un conjunto dotado de dos operaciones $+$ (adición)
y $\cdot$ (multiplicación) que satisfacen los siguientes axiomas.

\begin{enumerate}
\item[R1)] $R$ es un grupo abeliano respecto a $+$; es decir,

\begin{itemize}
\item[R1a)] la adición es \term{asociativa}: para cualesquiera $x,y,z\in R$
  tenemos
  $$(x+y)+z = x+(y+z);$$

\item[R1b)] existe un elemento neutro aditivo $0\in R$ (cero) tal que para todo
  $x\in R$ se cumple
  $$0+x = x = x+0;$$

\item[R1c)] para todo $x\in R$ existe un elemento \term{opuesto} $-x\in R$ que
  satisface
  $$(-x) + x = x + (-x) = 0;$$

\item[R1d)] la adición es \term{conmutativa}: para cualesquiera $x,y\in R$ se
  cumple
  $$x+y = y+x;$$
\end{itemize}

\item[R2)] la multiplicación es \term{distributiva}\index{distributividad}
  respecto a la adición: para cualesquiera $x,y,z\in R$ se cumple
  $$x\cdot (y+z) = xy + xz, \quad (x+y)\cdot z = xz + yz;$$

\item[R3)] la multiplicación es asociativa: para cualesquiera $x,y,z\in R$
  tenemos
  $$(x\cdot y)\cdot z = x\cdot (y\cdot z);$$

\item[R4)] existe un elemento neutro multiplicativo $1\in R$ (identidad) tal que
  para todo $x\in R$ se cumple
  $$1\cdot x = x = x\cdot 1.$$
\end{enumerate}

Además, si se cumple el axioma
\begin{enumerate}
\item[R5)] la multiplicación es conmutativa: para cualesquiera $x,y\in R$
  se cumple
  $$xy = yx.$$
\end{enumerate}
se dice que $R$ es un \term{anillo conmutativo}. Al estudio de algunas
propiedades especiales de anillos conmutativos estará dedicado el siguiente
capítulo.

Advertencia para el lector: algunos libros de texto consideran anillos
sin identidad (anillos que no satisfacen el axioma R4)), pero en este curso
la palabra ``anillo'' siempre significa ``anillo con identidad''.

\vspace{1em}

Recordemos algunos ejemplos de anillos que hemos visto.

\begin{enumerate}
\item[1)] Los números enteros $\ZZ$, racionales $\QQ$, reales $\RR$, complejos
  $\CC$. Los últimos tres son \term{cuerpos}.

\item[2)] Para $n = 1,2,3,\ldots$ y para $p$ un número primo los conjuntos
  \[ \ZZ \Bigl[\frac{1}{n}\Bigr] \dfn
     \Bigl\{ \frac{a}{n^k}\in\QQ \Bigm| a\in\ZZ, ~ k = 0,1,2,\ldots \Bigr\}, \quad
  \ZZ_{(p)} \dfn \Bigl\{ \frac{a}{b} \in \QQ \Bigm| p\nmid b \Bigr\} \]
  son anillos. Esto es un ejemplo de \term{localización} que vamos a estudiar
  más adelante en el curso.

\item[3)] El anillo $\ZZ/n\ZZ$ de los restos módulo $n$. Cuando $n = p$
  es primo, $\FF_p \dfn \ZZ/p\ZZ$ es un \term{cuerpo}.

\item[4)] Los anillos aritméticos como los \term{enteros de Gauss}
  $$\ZZ [\sqrt{-1}] \dfn \{ a + b\sqrt{-1} \mid a,b\in \ZZ \}$$
  los \term{enteros de Eisenstein}
  $$\ZZ [\zeta_3] \dfn \{ a + b\,\zeta_3 \mid a,b\in \ZZ \}$$
  (donde $\zeta_3 \dfn e^{2\pi\sqrt{-1}/3}$) y el anillo
  $$\ZZ [\sqrt{2}] \dfn \{ a + b\sqrt{2} \mid a,b\in \ZZ \}.$$

\item[5)] El anillo de polinomios $R [X]$, donde $R$ es un anillo conmutativo.

  Esta construcción puede ser generalizada al \term{anillo de polinomios en $n$
    variables $R [X_1,\ldots,X_n]$}\index{anillo!de polinomios!en $n$
    variables}\index[notacion]{RX1Xn@$R [X_1,\ldots,X_n]$}. En este caso
  los elementos son las expresiones formales de la forma
  $$f = \sum_{i_1,\ldots,i_n \ge 0} a_{i_1,\ldots,i_n}\,X_1^{i_1}\cdots X_n^{i_n},$$
  donde $a_{i_1,\ldots,i_n} = 0$, salvo un número finito de
  $(i_1,\ldots,i_n)$. Las sumas y productos están definidos por
  \[ \left(\sum_{i_1,\ldots,i_n \ge 0} a_{i_1,\ldots,i_n}\,X_1^{i_1}\cdots X_n^{i_n}\right) +
     \left(\sum_{i_1,\ldots,i_n \ge 0} b_{i_1,\ldots,i_n}\,X_1^{i_1}\cdots X_n^{i_n}\right) \dfn
     \sum_{i_1,\ldots,i_n \ge 0} (a_{i_1,\ldots,i_n} + b_{i_1,\ldots,i_n})\,X_1^{i_1}\cdots X_n^{i_n} \]
  y
  \begin{multline*}
    \left(\sum_{i_1,\ldots,i_n \ge 0} a_{i_1,\ldots,i_n}\,X_1^{i_1}\cdots X_n^{i_n}\right) \cdot \left(\sum_{j_1,\ldots,j_n \ge 0} b_{j_1,\ldots,j_n}\,X_1^{j_1}\cdots X_n^{j_n}\right) \\
    \dfn \sum_{k_1,\ldots,k_n \ge 0} \left(\sum_{\substack{(k_1,\ldots,k_n) = \\ (i_1,\ldots,i_n) + (j_1,\ldots,j_n)}} a_{i_1,\ldots,i_n}\,b_{j_1,\ldots,j_n}\right)\,X_1^{k_1}\cdots X_n^{k_n}.
  \end{multline*}

\item[6)] Si quitamos la condición que $a_{i_1,\ldots,i_n} = 0$, salvo un número
  finito de $(i_1,\ldots,i_n)$, se obtiene el \term{anillo de las series
    formales de potencias en $n$ variables
    $R [\![X_1,\ldots,X_n]\!]$}\index{anillo!de series formales!en $n$
    variables}\index[notacion]{RX1Xn@$R [[X_1,\ldots,X_n]]$}.

\item[7)] Los anillos de matrices $M_n (R)$, donde $R$ es un anillo conmutativo.
\end{enumerate}

Todos los anillos de arriba son conmutativos, salvo el anillo de matrices
$M_n (R)$ para $n > 1$.

% % % % % % % % % % % % % % % % % % % % % % % % % % % % % %

\section{Subanillos}

\begin{definicion}
  Sea $R$ un anillo. Se dice que un subconjunto $S \subseteq R$ es
  un \term{subanillo}\index{subanillo} de $R$ si

  \begin{enumerate}
  \item[1)] $S$ es un subgrupo abeliano de $R$ respecto a la adición,

  \item[2)] $1 \in S$,

  \item[3)] $S$ es cerrado respecto a la multiplicación: $xy\in S$ para
    cualesquiera $x,y\in S$.
  \end{enumerate}
\end{definicion}

El lector puede comprobar que en este caso $S$ es también un anillo respecto
a las mismas operaciones que $R$.

\begin{ejemplo}
  Sea $R$ un anillo conmutativo. Identificándolo con los polinomios constantes
  \[ \sum_{i_1,\ldots,i_n\ge 0} a_{i_1,\ldots,i_n}\,X_1^{i_1}\cdots X_n^{i_n},
     \quad
     a_{i_1,\ldots,i_n} = 0 \text{ para } (i_1,\ldots,i_n) \ne (0,\ldots,0), \]
  podemos decir que $R$ es un subanillo de $R [X_1,\ldots,X_n]$. De la misma
  manera, por la definición, los polinomios forman un subanillo
  de $R [\![X_1,\ldots,X_n]\!]$.
  \[ R \subset R [X_1,\ldots,X_n] \subset R [\![X_1,\ldots,X_n]\!]. \qedhere \]
\end{ejemplo}

\begin{ejemplo}
  Tenemos una cadena de subanillos
  \[ \ZZ \subset \ZZ [\sqrt{5}] \subset
     \ZZ \Bigl[\frac{1 + \sqrt{5}}{2}\Bigr] \subset \RR \subset \CC, \]
    donde
    \[ \ZZ [\sqrt{5}] \dfn \{ a + b\sqrt{5} \mid a,b\in \ZZ \},
       \quad
       \ZZ \Bigl[\frac{1 + \sqrt{5}}{2}\Bigr] \dfn
       \Bigl\{ a + b\,\frac{1 + \sqrt{5}}{2} \Bigm| a,b\in \ZZ \Bigr\}. \qedhere \]
\end{ejemplo}

\begin{ejemplo}
  Para $n = 1,2,3,\ldots$ y para $p$ un número primo los conjuntos
  \[ \ZZ \Bigl[\frac{1}{n}\Bigr] \dfn
     \Bigl\{ \frac{a}{n^k} \in \QQ \Bigm| a\in \ZZ, ~ k = 0,1,2,\ldots \Bigr\},
     \quad
     \ZZ_{(p)} \dfn
     \Bigl\{ \frac{a}{b} \in \QQ \Bigm| a,b\in \ZZ, ~ p\nmid b \Bigr\} \]
  son subanillos de $\QQ$.
\end{ejemplo}

\begin{ejemplo}
  Consideremos el anillo de las aplicaciones $f\colon \RR\to\RR$ respecto a las
  operaciones \term{punto por punto}
  \[ (f + g) (x) \dfn f (x) + g (x), \quad
     (f\cdot g) (x) \dfn f (x)\cdot g (x). \]
  Las aplicaciones continuas $\RR \to \RR$ forman un subanillo.
\end{ejemplo}

\begin{ejemplo}
  Para un anillo $R$ consideremos el subconjunto de los elementos que conmutan
  con todos los elementos:
  $$Z (R) \dfn \{ x\in R \mid xy = yx \text{ para todo }y\in R \}.$$
  Es un subanillo de $R$, llamado el
  \term{centro}\index{centro!de anillo}. Notamos que $R$ es conmutativo si y
  solamente si $R = Z (R)$.
\end{ejemplo}

\begin{ejemplo}
  $\ZZ$ y $\ZZ/n\ZZ$ no tienen subanillos propios. En efecto,
  si $R \subseteq \ZZ$ es un subanillo, entonces $1 \in R$, y el mínimo subgrupo
  abeliano de $\ZZ$ que contiene a $1$ es todo $\ZZ$. De la misma manera, para
  un subanillo $R \subseteq \ZZ/n\ZZ$ tenemos necesariamente $[1]_n \in R$, pero
  para todo $a = 1,2,3,4,\ldots$ se cumple
  \[ [a]_n = \underbrace{[1]_n + \cdots + [1]_n}_n. \qedhere \]
\end{ejemplo}

\begin{observacion}
  Sea $R$ un anillo. Si $R_i \subseteq R$ son subanillos, entonces
  $\bigcap_i R_i$ es un subanillo.
\end{observacion}

% % % % % % % % % % % % % % % % % % % % % % % % % % % % % %

\section{Homomorfismos de anillos}

Un homomorfismo de anillos es una aplicación que preserva las operaciones
de adición y multiplicación.  Ya que no todos los elementos de $R$ son
invertibles, de la identidad $f (xy) = f(x)\,f(y)$ en general no se puede
deducir que $f (1_R) = 1_S$. La última condición hace parte de la definición
de homomorfismo de anillos.

\begin{definicion}
  Sean $R$ y $S$ anillos. Se dice que una aplicación $f\colon R\to S$ es
  un \term{homomorfismo}\index{homomorfismo!de anillos} si se cumplen
  las siguientes condiciones:
  \begin{enumerate}
  \item[1)] $f$ es un homomorfismo de grupos abelianos respecto a la adición;
    es decir, $f (x+y) = f (x) + f (y)$ para cualesquiera $x,y\in R$;

  \item[2)] $f$ preserva la identidad: $f (1_R) = 1_S$;

  \item[3)] $f$ preserva la multiplicación: $f (xy) = f(x)\,f(y)$ para
    cualesquiera $x,y\in R$.
  \end{enumerate}

  Un homomorfismo $f\colon R\to R$ se llama
  un \term{endomorfismo}\index{endomorfismo!de anillo} de $R$.
\end{definicion}

\begin{ejemplo}
  La proyección canónica
  $$\ZZ \to \ZZ/n\ZZ, \quad a \mapsto [a]_n$$
  es un homomorfismo de anillos. De hecho, $\ZZ/n\ZZ$ es un ejemplo
  de \term{anillo cociente} que vamos a introducir más adelante.
\end{ejemplo}

\begin{ejemplo}
  Para todo anillo $R$ existe un homomorfismo único $R \to 0$ al anillo nulo.
\end{ejemplo}

\begin{ejemplo}
  Para todo anillo $R$ existe un homomorfismo único $f\colon \ZZ \to R$ desde
  el anillo de los enteros. En efecto, por la definición, $f (1) = 1_R$, y luego
  para todo $n\in \ZZ$ se tiene
  \[ f (n) = \begin{cases}
      \underbrace{1_R + \cdots + 1_R}_n, & n > 0,\\
      -(\underbrace{1_R + \cdots + 1_R}_{-n}), & n < 0,\\
      0, & f = 0.
    \end{cases} \]
  El elemento $f (n) \in R$ por abuso de notación también se denota por $n$.
\end{ejemplo}

\begin{ejemplo}
  Sea $R$ un anillo conmutativo. Para $\underline{c} = (c_1,\ldots,c_n)$ donde
  $c_i \in R$ tenemos el
  \term{homomorfismo de evaluación}\index{homomorfismo!de evaluación}
  \begin{align*}
    \ev_{\underline{c}}\colon R [X_1,\ldots,X_n] & \to R,\\
    f & \mapsto f (c_1,\ldots,c_n).
  \end{align*}
  Aquí si
  $f = \sum_{i_1,\ldots,i_n \ge 0} a_{i_1,\ldots,i_n}\,X_1^{i_1}\cdots X_n^{i_n}$,
  entonces
  \[ f (c_1,\ldots,c_n) \dfn
     \sum_{i_1,\ldots,i_n \ge 0} a_{i_1,\ldots,i_n}\,c_1^{i_1}\cdots c_n^{i_n}. \qedhere \]
\end{ejemplo}

\begin{observacion}
  Sea $f\colon R\to S$ un homomorfismo de anillos. Se cumplen las siguientes
  propiedades.

  \begin{enumerate}
  \item[1)] $f (0_R) = 0_S$,

  \item[2)] $f (-x) = -f (x)$ para todo $x\in R$,

  \item[3)] se cumple $f (x^{-1}) = f (x)^{-1}$ para todo $x\in R^\times$,
    y de este modo $f$ se restringe a un homomorfismo de grupos
    $f^\times\colon R^\times \to S^\times$:
    \[ \begin{tikzcd}
        R^\times\ar[hookrightarrow]{d}\ar[dashed]{r}{f^\times} &
        S^\times \ar[hookrightarrow]{d} \\
        R \ar{r}{f} & S
      \end{tikzcd} \]
  \end{enumerate}

  \begin{proof}
    Las partes 1) y 2) ya las probamos para homomorfismos de grupos. La parte 3)
    es un análogo multiplicativo de 2) y se demuestra de la misma manera:
    \[ f (x^{-1})\,f(x) = f (x^{-1}\,x) = f (1_R) = 1_S, \quad
       f (x)\,f(x^{-1}) = f (x\,x^{-1}) = f (1_R) = 1_S. \]
  \end{proof}
\end{observacion}

\begin{definicion}
  Se dice que un homomorfismo de anillos $f\colon R\to S$ es
  un \term{isomorfismo}\index{isomorfismo!de anillos} si existe un homomorfismo
  de anillos $f^{-1}\colon S\to R$ tal que $f^{-1}\circ f = \id{R}$
  y $f\circ f^{-1} = \id{S}$.

  Un isomorfismo $f\colon R\to R$ se llama un
  \term{automorfismo}\index{automorfismo!de anillo} de $R$.
\end{definicion}

\begin{ejemplo}
  La conjugación compleja
  $$z = x + y\sqrt{-1} \mapsto \overline{z} \dfn x - y\sqrt{-1}$$
  es un automorfismo de $\CC$.
\end{ejemplo}

\begin{observacion}
  Todo homomorfismo de anillos $f\colon R\to S$ es un isomorfismo si y solamente
  si es biyectivo.

  \begin{proof}
    Si $f$ es un isomorfismo, entonces $f$ admite un homomorfismo inverso
    $f^{-1}\colon S\to R$, así que es una biyección.

    Viceversa, supongamos que $f$ es un homomorfismo biyectivo. En este caso
    existe una aplicación inversa $f^{-1}\colon S\to R$ y hay que comprobar que
    es un homomorfismo de anillos. Dado que $f (1_R) = 1_S$, tenemos
    $f^{-1} (1_S) = 1_R$. Luego, para $x,y\in S$
    \[ f^{-1} (xy) =
       f^{-1} (f\circ f^{-1} (x)\cdot f\circ f^{-1} (y)) =
       f^{-1} (f (f^{-1} (x)\cdot f^{-1} (y))) =
       f^{-1} (x)\cdot f^{-1} (y). \]
    Con el mismo truco se demuestra que
    $f^{-1} (x + y) = f^{-1} (x) + f^{-1} (y)$.
  \end{proof}
\end{observacion}

\begin{observacion}[Imagen y preimagen]
  Sea $f\colon R\to S$ un homomorfismo de anillos.

  \begin{enumerate}
  \item[1)] La \term{imagen}\index{imagen!de homomorfismo!de anillos}
    $\im f \dfn \{ f (x) \mid x\in R \}$ es un subanillo de $S$.

  \item[2)] Si $S' \subseteq S$ es un subanillo, entonces su preimagen
    $$f^{-1} (S') \dfn \{ x\in R \mid f (x) \in S' \}$$
    es un subanillo de $R$.
  \end{enumerate}

  \begin{proof}
    La parte 1) se sigue de las identidades
    \[ f (x+y) = f(x) + f (y), \quad
       f (1_R) = 1_S, \quad
       f (xy) = f(x)\,f(y). \]
    En la parte 2), si $x \pm y \in f^{-1} (S')$, entonces
    $f (x), f (y) \in S'$. Luego, $x \pm y \in f^{-1} (S')$, dado que
    $f (x\pm y) = f (x) \pm f (y) \in S'$. De la misma manera,
    $xy \in f^{-1} (S')$, dado que $f (xy) = f (x)\,f (y) \in S'$. Tenemos
    $f (0_R) = 0_S \in S'$ y $f (1_R) = 1_S \in S'$, y por lo tanto
    $0_R, 1_R \in f^{-1} (S')$.
  \end{proof}
\end{observacion}

\begin{proposicion}
  Sea $R$ un anillo. Consideremos el homomorfismo $f\colon \ZZ \to R$. Entonces,
  $\im f$ es el mínimo subanillo de $R$. Hay dos posibilidades.

  \begin{enumerate}
  \item[1)] $\im f \isom \ZZ$. En este caso se dice que $R$ es un anillo
    de \term{característica $0$}.

  \item[2)] $\im f \isom \ZZ/n\ZZ$ para algún $n = 1,2,3,\ldots$ En este caso
    se dice que $R$ es un anillo de
    \term{característica $n$}\index{característica!de anillo}.
  \end{enumerate}

  \begin{proof}
    Tenemos
    $$\im f = \{ \underbrace{1_R + \cdots + 1_R}_m \mid m = 0,1,2,3,\ldots \}.$$
    Notamos que todo subanillo $S \subseteq R$ necesariamente contiene $0_R$ y
    $1_R$, y siendo cerrado respecto a la suma, también contiene todos los
    elementos $\pm\underbrace{1 + \cdots + 1}_m$. Entonces, $\im f \subseteq S$
    para cualquier subanillo $S \subseteq R$. Hay dos posibilidades.

    \begin{enumerate}
    \item[1)] El orden de $1_R$ en el grupo aditivo de $R$ es infinito. En este
      caso $\im f \isom \ZZ$ y el isomorfismo viene dado por
      $$\ZZ \to \im f, \quad 1 \mapsto 1_R.$$

    \item[2)] El orden de $1_R$ en el grupo aditivo de $R$ es finito y es igual
      a algún número $n = 1,2,3,\ldots$ En este caso $\im f \isom \ZZ/n\ZZ$ y
      el isomorfismo viene dado por
      $$\ZZ/n\ZZ \to \im f, \quad [1]_n \mapsto 1_R.$$
    \end{enumerate}
  \end{proof}
\end{proposicion}

\begin{ejemplo}
  Los anillos $\QQ$, $\QQ [X]$, $M_m (\ZZ)$ y $\ZZ [X_1,\ldots,X_m]$ son de
  característica $0$. Los anillos $M_m (\ZZ/n\ZZ)$ y $\ZZ/n\ZZ [X_1,\ldots,X_m]$
  son de característica $n$. El cuerpo finito $\FF_p$ tiene característica $p$.
\end{ejemplo}

\begin{observacion}
  Si $R$ es un anillo no nulo sin divisores de cero ($xy = 0$ implica que
  $x = 0$ o $y = 0$), entonces la característica de $R$ es igual a $0$ o es
  un número primo $p$.

  \begin{proof}
    El anillo $\ZZ/n\ZZ$ tiene divisores de cero si y solamente si $n$ es
    un número compuesto.
  \end{proof}
\end{observacion}

% % % % % % % % % % % % % % % % % % % % % % % % % % % % % %

\section{Álgebras sobre anillos}

\begin{definicion}
  Sea $R$ un anillo. Una \term{$R$-álgebra}\index{álgebra} es un anillo $A$
  junto con un homomorfismo de anillos $\alpha\colon R\to A$. En este caso por
  abuso de notación para $r\in R$ y $x\in A$
  \begin{equation}
    \label{eqn:abuso-de-notacion-para-algebras}
    \text{en lugar de ``}\alpha (r)\cdot x\text{'' se escribe simplemente ``}r\cdot x\text{''}.
  \end{equation}

  Para dos $R$-álgebras $\alpha\colon R\to A$ y $\beta\colon R\to B$
  un \term{homomorfismo}\index{homomorfismo!de álgebras} es un homomorfismo
  de anillos $f\colon R\to B$ que hace conmutar el diagrama
  \[ \begin{tikzcd}
      A\ar{rr}{f} & & B \\
      & R\ar{ul}{\alpha}\ar{ur}[swap]{\beta}
    \end{tikzcd} \]
\end{definicion}

Notamos que la última condición $f\circ \alpha = \beta$ implica que para todo
$r\in R$ y $x\in A$ se cumple
$$f \circ \alpha (r)\cdot f (x) = \beta (r)\cdot f (x).$$
Puesto que $f$ es un homomorfismo, esto es equivalente
a $f (\alpha (r)\cdot x) = \beta (r)\cdot f (x)$ o, usando la notación
\eqnref{eqn:abuso-de-notacion-para-algebras},
$$f (r\cdot x) = r\cdot f (x).$$

\begin{ejemplo}
  Todo anillo tiene una estructura única de $\ZZ$-álgebra: existe
  un homomorfismo único $\ZZ \to R$. Un homomorfismo de $\ZZ$-álgebras es
  la misma cosa que homomorfismo de anillos:
  \[ \begin{tikzcd}
      R\ar{rr}{f} & & S \\
      & \ZZ\ar[dashed]{ul}{\exists !}\ar[dashed]{ur}[swap]{\exists !}
    \end{tikzcd} \]
  De nuevo, usando la notación \eqnref{eqn:abuso-de-notacion-para-algebras},
  para $n\in \ZZ$ y $r\in R$ tenemos
  \[ n\cdot r = \begin{cases}
      \underbrace{r + \cdots + r}_n, & \text{si }n > 0,\\
      -(\underbrace{r + \cdots + r}_{-n}), & \text{si }n < 0,\\
      0, & \text{si }n = 0.
    \end{cases} \]
\end{ejemplo}

\begin{ejemplo}
  Los números complejos forman una $\RR$-álgebra: tenemos un homomorfismo
  \begin{align*}
    \alpha\colon \RR & \mono \CC,\\
    x & \mapsto x + 0\,\sqrt{-1}.
  \end{align*}
  Notamos que para $x\in \RR$ se tiene
  $$x\cdot (u + v\,\sqrt{-1}) = xu + xv\,\sqrt{-1}.$$
\end{ejemplo}

\begin{ejemplo}
  Sea $R$ un anillo conmutativo. Los anillos de polinomios $R [X_1,\ldots,X_n]$
  y series formales de potencias $R [\![X_1,\ldots,X_n]\!]$ son $R$-álgebras.
  En el caso de polinomios, el homomorfismo
  $$\alpha\colon R \to R [X_1,\ldots,X_n]$$
  asocia a los elementos de $R$ los polinomios constantes correspondientes.
  En este caso
  \[ r\cdot \sum_{i_1,\ldots,i_n} a_{i_1,\ldots,i_n}\,X_1^{i_1}\cdots X_n^{i_n} =
     \sum_{i_1,\ldots,i_n} (r\cdot a_{i_1,\ldots,i_n})\,X_1^{i_1}\cdots X_n^{i_n}. \]
  De modo similar, tenemos para las series de potencias
  $$\alpha\colon R \to R [\![X_1,\ldots,X_n]\!].$$
\end{ejemplo}

\begin{ejemplo}
  Sea $R$ un anillo conmutativo. El homomorfismo
  \[ \alpha\colon R \to M_n (R), \quad
    r \mapsto \begin{pmatrix}
      r \\
      & r \\
      & & \ddots \\
      & & & r \\
    \end{pmatrix} \]
  que asocia a los elementos de $R$ las matrices escalares correspondientes
  define estructura de $R$-álgebra sobre $M_n (R)$. En este caso
  $$r\cdot (x_{ij}) = (r x_{ij}).$$
\end{ejemplo}

Ahora podemos finalmente aclarar qué es el anillo de polinomios
$R [X_1,\ldots,X_n]$.

\begin{proposicion}[Propiedad universal del álgebra de polinomios]
  \index{propiedad universal!del álgebra de polinomios}
  Sea $R$ un anillo conmutativo y sea $A$ una $R$-álgebra
  conmutativa. Consideremos elementos $x_1,\ldots,x_n \in A$. Existe
  un homomorfismo único de $R$-álgebras $f\colon R [X_1,\ldots,X_n] \to A$
  tal que $f (X_i) = x_i$ para $i = 1,\ldots,n$.

  \[ \begin{tikzcd}[row sep=0pt]
      X_i \ar[|->]{rr} && x_i \\
      R [X_1,\ldots,X_n] \ar[dashed]{rr}{\exists !} && A \\[1em]
      & R\ar{ul}\ar{ur}[swap]{\alpha}
    \end{tikzcd} \]

  \begin{proof}
    Si $f\colon R [X_1,\ldots,X_n] \to A$ es un homomorfismo de $R$-álgebras,
    entonces para todo polinomio tenemos
    \begin{multline*}
      f \left(\sum_{i_1,\ldots,i_n\ge 0} a_{i_1,\ldots,i_n} \, X_1^{i_1} \cdots X_n^{i_n}\right) =
      \sum_{i_1,\ldots,i_n\ge 0} f \left(a_{i_1,\ldots,i_n} \, X_1^{i_1} \cdots X_n^{i_n}\right) \\
      = \sum_{i_1,\ldots,i_n\ge 0} a_{i_1,\ldots,i_n} \cdot f (X_1^{i_1} \cdots X_n^{i_n}) =
      \sum_{i_1,\ldots,i_n\ge 0} a_{i_1,\ldots,i_n} \cdot f (X_1)^{i_1} \cdots f (X_n)^{i_n}.
    \end{multline*}
    Esto significa que $f$ está definido de modo único por las imágenes
    $f (X_i) \in A$. Además, se ve que especificando $f (X_i) = x_i$ para
    elementos arbitrarios $x_1,\ldots,x_n \in A$ se obtiene un homomorfismo
    de $R$-álgebras $f\colon R [X_1,\ldots,X_n] \to A$.
  \end{proof}
\end{proposicion}

\begin{corolario}
  Sea $R$ un anillo conmutativo. Consideremos elementos
  $x_1,\ldots,x_n \in R$. Existe un homomorfismo único de anillos
  $f\colon \ZZ [X_1,\ldots,X_n] \to R$ tal que $f (X_i) = x_i$ para
  $i = 1,\ldots,n$.

  \[ \begin{tikzcd}[row sep=0pt]
      X_i \ar[|->]{r} & x_i \\
      \ZZ [X_1,\ldots,X_n] \ar[dashed]{r}{\exists !} & R
    \end{tikzcd} \]

  \begin{proof}
    Recordemos que anillos son $\ZZ$-álgebras.
  \end{proof}
\end{corolario}

Como siempre, las palabras ``propiedad universal'' significan
que $R [X_1,\ldots,X_n]$ está definido de modo único salvo isomorfismo único
por esta propiedad. En efecto, supongamos que $A$ es una $R$-álgebra con algunos
elementos $x_1,\ldots,x_n$ que satisface la misma propiedad universal. Entonces,
existe un único homomorfismo de $R$-álgebras $f\colon R [X_1,\ldots,X_n] \to A$
tal que $X_i \mapsto x_i$ y un único homomorfismo de $R$-álgebras
$g\colon A \to R [X_1,\ldots,X_n]$ tal que $x_i \mapsto X_i$. Luego,
necesariamente $g\circ f = \id{R [X_1,\ldots,X_n]}$ y $f\circ g = \id{A}$:
\[ \begin{tikzcd}
    X_i \ar[|->]{rr} && x_i \ar[|->]{rr} && X_i \\
    R [X_1,\ldots,X_n] \ar[dashed]{rr}{\exists !}[swap]{f} \ar[dashed,bend left=15]{rrrr}{\exists ! = \idid} && A \ar[dashed]{rr}{\exists !}[swap]{g} && R [X_1,\ldots,X_n] \\
    & & R\ar{ull}\ar{u}[swap]{\alpha}\ar{urr}
\end{tikzcd} \]

\[ \begin{tikzcd}
    x_i \ar[|->]{rr} && X_i \ar[|->]{rr} && x_i \\
    A \ar[dashed]{rr}{\exists !}[swap]{g} \ar[dashed,bend left=18]{rrrr}{\exists ! = \idid} && R [X_1,\ldots,X_n] \ar[dashed]{rr}{\exists !}[swap]{f} && A \\
    & & R\ar{ull}{\alpha}\ar{u}\ar{urr}[swap]{\alpha}
\end{tikzcd} \]

\begin{comentario}
  Es importante que $A$ sea conmutativa. En el caso contrario, los elementos
  $f (X_i)$ no necesariamente conmutan entre sí, mientras que $X_i$ conmutan
  en $R [X_1,\ldots,X_n]$. La propiedad universal similar respecto a álgebras
  no conmutativas caracteriza a los ``polinomios en variables no conmutativas''
  (aunque suena exótico, es un objeto natural e importante).

  Sin embargo, para polinomios \emph{en una variable} tenemos la siguiente
  propiedad universal: si $A$ es una $R$-álgebra, \emph{no necesariamente
    conmutativa} y $x\in A$, entonces existe un homomorfismo único
  de $R$-álgebras $f\colon R[X] \to A$ tal que $f (X) = x$:
  $$f \Bigl(\sum_{i\ge 0} a_i\,X^i\Bigr) = \sum_{i\ge 0} a_i\cdot f (X)^i.$$
\end{comentario}

\begin{comentario}
  \label{comentario:prop-univ-de-R[X]}
  El anillo de series formales $R [\![X_1,\ldots,X_n]\!]$ también se caracteriza
  por cierta propiedad universal, pero es un poco más complicada y por esto
  la omitimos.
\end{comentario}

\begin{proposicion}
  Sea $R$ un anillo conmutativo y sea $n = 2,3,4,\ldots$ Tenemos isomorfismos
  \begin{align*}
    R [X_1,\ldots,X_{n-1}] [X_n] & \isom R [X_1,\ldots,X_n],\\
    R [\![X_1,\ldots,X_{n-1}]\!] [\![X_n]\!] & \isom R [\![X_1,\ldots,X_n]\!].
  \end{align*}

  \begin{proof}[Idea de la demostración]
    Todo elemento
    $\sum_{i_1,\ldots,i_n \ge 0} a_{i_1,\ldots,i_n}\,X_1^{i_1}\cdots X_n^{i_n}$
    puede ser escrito como $\sum_{i \ge 0} f_i\,X_n^i$, donde en $f_i$ aparecen
    las variables $X_1,\ldots,X_{n-1}$. Dejo los detalles al lector.
  \end{proof}
\end{proposicion}

% % % % % % % % % % % % % % % % % % % % % % % % % % % % % %

\section{El álgebra de grupo}

\begin{definicion}
  Sea $G$ un grupo y sea $R$ un anillo
  conmutativo. Definamos\index[notacion]{RG@$R [G]$}
  \[ R [G] \dfn
     \Bigl\{ \text{sumas formales }\sum_{g\in G} a_g\,g \Bigm|
             a_g\in R, ~ a_g = 0
             \text{ salvo un número finito de }g\in G \Bigr\}. \]
  Definamos la suma mediante
  \[ \Bigl(\sum_{g\in G} a_g\,g\Bigr) + \Bigl(\sum_{g\in G} b_g\,g\Bigr) \dfn
     \sum_{g\in G} (a_g+b_g)\,g \]
  y el producto mediante la multiplicación en $G$ y la distributividad formal:
  \begin{multline*}
    \Bigl(\sum_{h\in G} a_h\,h\Bigr) \cdot \Bigl(\sum_{k\in G} b_k\,k\Bigr) \dfn \sum_{h\in G} a_h \Bigl( \sum_{k\in G} b_k\,hk\Bigr) =
    \sum_{h\in G} a_h \, \Bigl( \sum_{g\in G} b_{h^{-1}g}\,h (h^{-1} g)\Bigr) \\
    = \sum_{h\in G} \Bigl(\sum_{h\in G} a_h\,b_{h^{-1}g}\Bigr)\,g =
    \sum_{g\in G} \Bigl(\sum_{hk = g} a_h b_k\Bigr)\,g.
  \end{multline*}
  Aquí la segunda igualdad sigue del hecho de que el conjunto
  $\{ h^{-1} g \mid g\in G \}$ está en biyección con los elementos
  de $G$. Entonces, podemos tomar como la definición la
  identidad\footnote{Note que es parecida a la fórmula
    \[ \Bigl(\sum_{i\ge 0} a_i\,X^i\Bigr) \cdot \Bigl(\sum_{j \ge 0} b_j\,X^j\Bigr) \dfn
      \sum_{k\ge 0} \Bigl(\sum_{i + j = k} a_i b_j\Bigr)\,X^k. \]
    (Esta no es una coincidencia.)}
  \[ \Bigl(\sum_{h\in G} a_h\,h\Bigr) \cdot \Bigl(\sum_{k\in G} b_k\,k\Bigr) \dfn
     \sum_{g\in G} \Bigl(\sum_{hk = g} a_h b_k\Bigr)\,g. \]

  Se puede comprobar que $R [G]$ es un anillo. El cero es la suma
  $\sum_{g\in G} a_g\,g$ donde $a_g = 0$ para todo $g\in G$ y la identidad
  es la suma donde $a_e = 1$ (donde $e\in G$ es el elemento neutro de $G$)
  y $a_g = 0$ para $g\ne e$. Notamos que el anillo $R [G]$ es conmutativo
  si y solamente si $G$ es un grupo abeliano. El homomorfismo
  \begin{align*}
    R & \to R [G],\\
    r & \mapsto \sum_{g\in G} a_g\,g, \quad a_g \dfn \begin{cases}
      r, & g = e,\\
      0, & g \ne e
    \end{cases}
  \end{align*}
  define una estructura de $R$-álgebra sobre $R [G]$. Tenemos
  $$r\cdot \sum_{g\in G} a_g\,g = \sum_{g\in G} (r\,a_g)\,g.$$
  El álgebra $R [G]$ se llama el \term{álgebra de grupo}\index{álgebra!de grupo}
  asociada a $G$.
\end{definicion}

Notamos que cada elemento $g\in G$ corresponde a un elemento
\[ \sum_{h\in G} a_h\,g \in R [G], \quad
  a_h \dfn \begin{cases}
    1, & h = g,\\
    0, & h \ne g,
  \end{cases} \]
y esto nos da una aplicación inyectiva $G \hookrightarrow R [G]$. Respecto
a esta inclusión, $G \subseteq R [G]^\times$.

Tenemos
\[ h\,\sum_{g\in G} a_g\,g =
   \sum_{g\in G} a_g\,hg =
   \sum_{g\in G} a_{h^{-1}g}\,g, \quad
   \Bigr(\sum_{g\in G} a_g\,g\Bigr)\,h =
   \sum_{g\in G} a_g\,gh =
   \sum_{g\in G} a_{gh^{-1}}\,g. \]
Comparando estas dos expresiones, se puede calcular el centro de $R [G]$
(haga el ejercicio \ref{ejerc:centro-de-RG}).

\begin{ejemplo}
  En el álgebra $\ZZ [S_3]$ calculamos
  \begin{multline*}
    (1\cdot (1~2) + 2\cdot (2~3))^2 =
    1\cdot \underbrace{(1~2)^2}_{= \idid} +
    2\cdot \underbrace{(1~2)\,(2~3)}_{= (1~2~3)} +
    2\cdot \underbrace{(2~3)\,(1~2)}_{= (1~3~2)} +
    4\cdot \underbrace{(2~3)^2}_{= \idid} \\
    = 5\cdot \idid + 2\cdot (1~2~3) + 2\cdot (1~3~2).
  \end{multline*}
  Si $C_3 = \{ e, g, g^2 \}$ es el grupo cíclico de orden $3$, entonces tenemos
  en $\ZZ [C_3]$
  \[ (e + g + g^2)^2 =
     e + g + g^2 + g + g^2 + \underbrace{g^3}_{= e} + g^2 +
     \underbrace{g^3}_{= e} + \underbrace{g^4}_{= g} = 3\cdot (e + g + g^2). \]
  (Para una generalización, de este cálculo, haga el ejercicio
  \ref{ejerc:cuadrado-del-elemento-de-la-traza}.)
\end{ejemplo}

\begin{proposicion}[Propiedad universal del álgebra de grupo o adjunción con
  $A \rightsquigarrow A^\times$]
  \index{propiedad universal!del álgebra de grupo}
  \label{prop:propiedad-universal-de-RG}
  Sea $R$ un anillo conmutativo, $G$ un grupo y $A$ una $R$-álgebra. Todo
  homomorfismo de grupos $f\colon G\to A^\times$ se extiende de modo único
  a un homomorfismo de $R$-álgebras $\widetilde{f}\colon R [G] \to A$:
  \[ \begin{tikzcd}
      G \ar[hookrightarrow]{r}\ar{d}[swap]{f} &
      R [G]\ar[dashed]{d}{\exists ! \widetilde{f}} \\
      A^\times \ar[hookrightarrow]{r} & A
    \end{tikzcd} \]
  En otras palabras, hay una biyección natural
  \[ \{ \text{homomorfismos de }R\text{-álgebras }R [G]\to A \} \isom
     \{ \text{homomorfismos de grupos }G\to A^\times \}. \]
  En particular, para todo anillo $R$ hay una biyección natural
  \[ \{ \text{homomorfismos de anillos }\ZZ [G]\to R \} \isom
     \{ \text{homomorfismos de grupos }G\to R^\times \}. \]

  \begin{proof}
    Sea $\alpha\colon R\to A$ el homomorfismo que define la estructura
    de $R$-álgebra. Sea $\widetilde{f}\colon R [G] \to A$ un homomorfismo
    de $R$-álgebras. Puesto que $G \subseteq R [G]^\times$, este homomorfismo
    se restinge a un homomorfismo de grupos $f\colon G\to A^\times$. Luego,
    \[ \widetilde{f} \Bigl( \sum_{g\in G} a_g\,g \Bigr) =
       \sum_{g\in G} \widetilde{f} (a_g\,g) =
       \sum_{g\in G} \alpha (a_g)\,f (g). \]
  \end{proof}
\end{proposicion}

\begin{corolario}
  Sea $R$ un anillo conmutativo. Todo homomorfismo de grupos $f\colon G\to H$
  se extiende de manera canónica a un homomorfismo de $R$-álgebras
  $\widetilde{f}\colon R [G] \to R [H]$.

  \begin{proof}
    El homomorfismo de $R$-álgebras en cuestión viene dado por
    \[ \widetilde{f} \Bigl( \sum_{g\in G} a_g\,g \Bigr) =
       \sum_{g\in G} a_g\,f (g), \]
    y es un caso particular del resultado anterior:
    \[ \begin{tikzcd}
        G \ar[hookrightarrow]{r}\ar{d}[swap]{f} &
        R [G]\ar[dashed]{dd}{\exists ! \widetilde{f}} \\
        H\ar[hookrightarrow]{d} \\
        R [H]^\times \ar[hookrightarrow]{r} & R [H]
      \end{tikzcd} \]
  \end{proof}
\end{corolario}

El álgebra $R [G]$ juega papel importante en la teoría de representación
de grupos finitos.

% % % % % % % % % % % % % % % % % % % % % % % % % % % % % %

\section{Monomorfismos y epimorfismos de anillos}

\begin{proposicion}
  Sea $f\colon R\to S$ un homomorfismo de anillos. Las siguientes condiciones
  son equivalentes.

  \begin{enumerate}
  \item[1)] $f$ es inyectivo.

  \item[2)] Si $S'$ es otro anillo y hay homomorfismos $g, g'\colon R'\to R$
    tales que $f\circ g = f\circ g'$, entonces $g = g'$.
  \end{enumerate}

  En este caso se dice que $f$ es un
  \term{monomorfismo}\index{monomorfismo!de anillos}.

  \begin{proof}
    La implicación $1) \Rightarrow 2)$ se cumple para cualquier aplicación
    inyectiva $f$. Para ver que $2) \Rightarrow 1)$, supongamos que $f$
    no es inyectiva y existen diferentes $x,x'\in R$ tales que
    $f (x) = f (x')$. Primero recordemos que para todo anillo $R$
    un homomorfismo $f\colon \ZZ [X] \to R$ está definido de modo único
    por $f (X) \in R$ (véase el comentario
    \ref{comentario:prop-univ-de-R[X]}). Consideremos los homomorfismos
    $$g\colon \ZZ [X] \to R, \quad X \mapsto x$$
    y
    $$g\colon \ZZ [X] \to R, \quad X \mapsto x'.$$
    Ahora $f\circ g = f\circ g'$, aunque $g\ne g'$.
  \end{proof}
\end{proposicion}

\begin{ejemplo}
  Consideremos la propiedad dual para un homomorfismo $f\colon R\to S$:
  \emph{si $S'$ es otro anillo y hay homomorfismos $g, g'\colon S\to S'$ tales
    que $g\circ f = g'\circ f$, entonces $g = g'$}. Esto se cumple si $f$
  es sobreyectivo. Sin embargo, esta propiedad no necesariamente implica que $f$
  es sobreyectivo. Por ejemplo, consideremos la inclusión
  $i\colon \ZZ\to \QQ$. Supongamos que $g\circ i = g'\circ i$. Luego, para todo
  $\frac{a}{b} \in \QQ$ se tiene
  \begin{multline*}
    g \left(\frac{a}{b}\right) =
    g (a)\cdot g \left(\frac{1}{b}\right) =
    g' (a)\cdot g \left(\frac{1}{b}\right) =
    g' \left(\frac{a}{b}\cdot b\right)\cdot g \left(\frac{1}{b}\right) =
    g' \left(\frac{a}{b}\right)\cdot g' (b)\cdot g \left(\frac{1}{b}\right) \\
    = g' \left(\frac{a}{b}\right)\cdot g (b)\cdot g \left(\frac{1}{b}\right) =
    g' \left(\frac{a}{b}\right)\cdot g \left(b\cdot \frac{1}{b}\right) =
    g' \left(\frac{a}{b}\right)\cdot g (1) =
    g' \left(\frac{a}{b}\right).
  \end{multline*}
  Entonces, $g = g'$.

  Los epimorfismos de anillos \emph{no son} necesariamente homomorfismos
  sobreyectivos; véase \cite{Roby-1968} para una discusión.
\end{ejemplo}

\begin{proposicion}[Propiedad universal de la imagen]
  \index{propiedad universal!de la imagen}
  Sea $f\colon R\to S$ un homomorfismo de anillos.

  \begin{enumerate}
  \item[1)] Existe una factorización de $f$ por el monomorfismo canónico
    $i\colon \im f \mono S$ (inclusión de subanillo):

    \[ \begin{tikzcd}
        R\ar{rr}{f}\ar[dashed]{dr}[swap]{\overline{f}} & & S\\
        & \im f\ar[>->]{ur}[swap]{i}
      \end{tikzcd} \]

    \[ f = i\circ \overline{f}. \]

  \item[2)] Supongamos que hay otro anillo $I$ junto con un monomorfismo
    $j\colon I \mono S$ y una factorización de $f$ por $I$:

    \[ \begin{tikzcd}
        R\ar{rr}{f}\ar[dashed]{dr}[swap]{f'} & & S\\
        & I\ar[>->]{ur}[swap]{j}
      \end{tikzcd} \]

    \[ f = j\circ f'. \]

    Luego existe un único homomorfismo $\phi\colon \im f \to I$
    que hace conmutar el siguiente diagrama:

    \[ \begin{tikzcd}[row sep=3em, column sep=3em]
        R\ar{rr}{f}\ar{dr}{\overline{f}}\ar[bend right]{ddr}[swap]{f'} & & S\\
        & \im f\ar[>->]{ur}{i}\ar[dashed]{d}{\phi}[swap]{\exists!} \\
        & I\ar[>->,bend right]{uur}[swap]{j}
      \end{tikzcd} \]
  \end{enumerate}

  \[ \phi\circ \overline{f} = f', \quad j\circ \phi = i. \]
  ($\phi$ es mono, puesto que $i = j\circ \phi$ lo es).

  \begin{proof}
    La parte 1) está clara de la definición de la imagen: ya que $f$ toma sus
    valores en $\im f \subset S$, en realidad $f$ puede ser vista como una
    aplicación $\overline{f}\colon R\to \im f$. Es un homomorfismo, puesto que
    $f$ es un homomorfismo. Su composición con la inclusión del subanillo
    $i\colon \im f\mono S$ coincide con $f$.

    En 2), la única opción para $\phi$ para que se cumpla
    $\phi\circ \overline{f} = f'$ es definir

    \begin{align*}
      \phi\colon \im f & \to I,\\
      f (x) & \mapsto f' (x).
    \end{align*}

    Esta aplicación está bien definida: si tenemos $f (x_1) = f (x_2)$,
    entonces
    \[ j (f' (x_1)) = f (x_1) = f (x_2) = j (f' (x_2)) \Rightarrow
       f' (x_1) = f' (x_2). \]
    También se cumple $i = j\circ \phi$. En efecto, para
    $h = f (x) \in \im f$ tenemos
    $$j (\phi (h)) = j (f' (x)) = f (x).$$
  \end{proof}
\end{proposicion}

% % % % % % % % % % % % % % % % % % % % % % % % % % % % % %

\section{Ideales}

El la teoría de grupos, el grupo cociente se construye a partir de un subgrupo
\emph{normal}. Para anillos, los cocientes se definen a partir de
un \emph{ideal}.

\begin{definicion}
  Sea $R$ un anillo y sea $I \subseteq R$ un subgrupo abeliano de $R$ respecto
  a la adición.

  \begin{enumerate}
  \item[1)] Si $rx \in I$ para cualesquiera $r\in R$ y $x\in I$, se dice que $I$
    es un \term{ideal izquierdo}\index{ideal!izquierdo} en $R$.

  \item[2)] Si $xr \in I$ para cualesquiera $r\in R$ y $x\in I$, se dice que $I$
    es un \term{ideal derecho}\index{ideal!derecho} en $R$.

  \item[3)] Si se cumplen las condiciones 2) y 3), entonces se dice que $I$
    es un \term{ideal bilateral}\index{ideal!bilateral} en $R$.
    Esto es equivalente a asumir que $r x r' \in I$ para cualesquiera
    $r,r'\in R$ y $x \in I$.
  \end{enumerate}
\end{definicion}

\begin{comentario}
  Tenemos $-x = (-1)\cdot x = x\cdot (-1)$, así que para comprobar que
  un subconjunto $I \subseteq R$ es un ideal, es suficiente comprobar que $I$
  no es vacío, cerrado respecto a la adición, y cumple una de las propiedades
  1)--3) de la definición de arriba.
\end{comentario}

\begin{comentario}
  Si $R$ es un anillo \emph{conmutativo}, entonces las condiciones 1)--3)
  son equivalentes. En este caso se dice simplemente que $I$ es un \term{ideal}
  en $R$.
\end{comentario}

\begin{observacion}
  Sea $R$ un anillo.

  \begin{enumerate}
  \item[1)] Si $I_k \subseteq R$ es una familia de ideales izquierdos
    (resp. derechos, bilaterales), entonces $\bigcap_k I_k$ es un ideal
    izquierdo (resp. derecho, resp. bilateral).

  \item[2)] Si
    $$I_1 \subseteq I_2 \subseteq I_3 \subseteq \cdots \subseteq R$$
    es una cadena de ideales izquierdos (resp. derechos, bilaterales), entonces
    $\bigcup_k I_k$ es un ideal izquierdo (resp. derecho, resp. bilateral).
  \end{enumerate}

  \begin{proof}
    Ejercicio para el lector.
  \end{proof}
\end{observacion}

\begin{ejemplo}
  $0$ y $R$ son ideales bilaterales para cualquier anillo $R$.
\end{ejemplo}

\begin{ejemplo}
  Consideremos el anillo de los números enteros $\ZZ$. Como sabemos,
  sus subgrupos abelianos son de la forma
  $$n\ZZ \dfn \{ na \mid a\in \ZZ \}$$
  para $n = 0,1,2,3,\ldots$ Se puede comprobar que $n\ZZ$ son ideales
  (al multiplicar un múltiplo de $n$ por cualquier número entero se obtiene
  un múltiplo de $n$).
\end{ejemplo}

\begin{observacion}
  \label{obs:ideales-y-unidades}
  Sea $R$ un anillo.

  \begin{enumerate}
  \item[1)] Para un ideal izquierdo (resp. derecho, resp. bilateral)
    $I \subseteq R$ se tiene $I = R$ si y solo si $u\in I$ para algún elemento
    invertible $u \in R^\times$.

  \item[2)] Si $R$ es un anillo conmutativo, entonces $R$ es un cuerpo si y solo
    si $0$ y $R$ son los únicos ideales en $R$.
  \end{enumerate}

  \begin{proof}
    En 1), notamos que si $I = R$, entonces $1 \in I$ y
    $1 \in R^\times$. Viceversa, si $u\in R^\times$ es un elemento tal que
    $u \in I$, entonces para todo $r\in R$
    $$r = r\cdot 1 = r\,(u^{-1}\,u) = (r\,u^{-1})\,u \in I.$$
    Este argumento funciona si $I$ es un ideal izquierdo. Para un ideal derecho,
    tenemos
    $$r = 1\cdot r = (u\,u^{-1})\,r = u\,(u^{-1}\,r) \in I.$$

    En 2), si $R$ es un cuerpo, entonces para todo ideal no nulo $I$ si $x\in I$
    y $x\ne 0$, entonces $x\in R^\times$ y por ende $I = R$ según la parte
    1). Viceversa, si $0$ y $R$ son los únicos ideales en $R$, para $x \ne 0$
    podemos considerar el ideal
    $$Rx \dfn \{ rx \mid r\in R \}.$$
    Tenemos $Rx \ne 0$, así que $Rx = R$. En particular, $rx = 1$ para algún
    $r\in R$, y este elemento $r$ es el inverso de $x$.
  \end{proof}
\end{observacion}

\begin{ejemplo}
  Sea $X$ un conjunto no vacío y sea $R$ un anillo. Entonces, las aplicaciones
  $f\colon X\to R$ forman un anillo $Fun (X,R)$ respecto a las operaciones punto
  por punto. Para un punto $x \in X$ sea $I_x$ el conjunto de las aplicaciones
  tales que $f (x) = 0$:
  $$I_x \dfn \{ f\colon X\to R \mid f (x) = 0 \}.$$
  Esto es un ideal en $Fun (X,R)$. En general, para un subconjunto
  $Y \subseteq X$, tenemos un ideal
  \[ I (Y) = \bigcap_{x\in Y} I_x =
     \{ f\colon X\to R \mid f (x) = 0\text{ para todo }x\in Y \} \subseteq
     Fun (X,R). \]
\end{ejemplo}

El último ejemplo tiene muchas variaciones. Por ejemplo, se puede tomar
$R = \RR$ y $X$ un subconjunto de $\RR$ y considerar las funciones continuas
$f\colon X\to \RR$. También se puede tomar un cuerpo $k$ y
el \term{espacio afín}\index{espacio!afín}
$$\AA^n (k) \dfn \{ (x_1,\ldots,x_n) \mid x_i\in k \}$$
y en lugar de todas las funciones $f\colon \AA^n (k) \to k$ considerar
los polinomios $f\in k [X_1,\ldots,X_n]$ que también pueden ser evaluados
en los puntos de $\AA^n (k)$.

\begin{ejemplo}
  \label{ejemplo:operacion-I}
  Sea $k$ un cuerpo. Para todo subconjunto $X \subseteq \AA^n (k)$ consideremos
  el conjunto de los polinomios en $n$ variables con coeficientes en $k$ que
  se anulan en todos los puntos de $X$:
  \[ I (X) \dfn
     \{ f \in k [X_1,\ldots,X_n] \mid f (x) = 0\text{ para todo }x\in X \} =
     \bigcap_{x\in X} I (\{ x \}). \]
  Esto es un ideal en el anillo de polinomios $k [X_1,\ldots,X_n]$. En efecto,
  si $f_i (x) = 0$ para todo $x\in X$, entonces todas las sumas finitas
  $\sum_i g_i f_i$ se anulan sobre $X$.
\end{ejemplo}

En este curso no vamos a ver muchos resultados sobre anillos no conmutativos,
pero es bueno conocer algunas definiciones básicas. El lector interesado puede
consultar el libro \cite{Lam-2001}.

\begin{ejemplo}
  Las matrices de la forma $\begin{pmatrix}
    \ast & 0 \\
    \ast & 0
  \end{pmatrix}$ forman un ideal izquierdo en $M_2 (R)$ que no es un ideal
  derecho. Viceversa, las matrices $\begin{pmatrix}
    \ast & \ast \\
    0 & 0
  \end{pmatrix}$ forman un ideal derecho que no es izquierdo.
\end{ejemplo}

\begin{observacion}
  \label{obs:ideales-en-Mnk}
  Sea $k$ un cuerpo. Entonces, los únicos ideales bilaterales en el anillo
  de matrices $R = M_n (k)$ son $0$ y $R$.

  \begin{proof}
    Denotemos por $e_{ij}$ la matriz que tiene ceros en todas las entradas y $1$
    en la entrada $(i,j)$. Notamos que
    $$e_{ij}\,A\,e_{k\ell} = a_{jk}\,e_{i\ell}.$$
    Supongamos que $I \subseteq R$ es un ideal bilateral no nulo. Sea $A \in I$
    donde $A$ es una matriz tal que $a_{jk} \ne 0$ para algunos
    $1 \le j,k \le n$. Luego, la fórmula de arriba nos dice que para todo
    $1 \le i,\ell \le n$ se tiene
    $$e_{i\ell} = a_{jk}^{-1}\,e_{ij}\,A\,e_{k\ell}.$$
    Puesto que $I$ es un ideal bilateral, podemos concluir que todas
    las matrices $e_{i\ell}$ pertenecen a $I$. Luego, para cualquier matriz
    $B = (b_{i\ell}) \in M_n (k)$ tenemos
    $$B = \sum_{1 \le i,\ell \le n} b_{i\ell}\,e_{i\ell},$$
    y esta matriz pertenece a $I$, siendo una suma
    de $b_{i\ell}\,e_{i\ell} \in I$. Entonces, acabamos de probar que un ideal
    bilateral no nulo en $M_n (k)$ necesariamente coincide con todo $M_n (k)$.
  \end{proof}
\end{observacion}

Para una generalización del último resultado, haga el ejercicio
\ref{ejerc:ideales-en-MnR}.

\begin{observacion}
  \label{obs:imagen-y-preimagen-de-un-ideal}
  Sea $f\colon R\to S$ un homomorfismo de anillos.

  \begin{enumerate}
  \item[1)] Si $I \subseteq S$ es un ideal izquierdo (resp. ideal derecho,
    resp. ideal bilateral), entonces $f^{-1} (I)$ es un ideal izquierdo
    (resp. ideal derecho, resp. ideal bilateral) en $R$.

  \item[2)] Si $f$ es sobreyectivo e $I \subseteq R$ es un ideal izquierdo
    (resp. ideal derecho, resp. ideal bilateral), entonces $f (I)$ es un ideal
    izquierdo (resp. ideal derecho, resp. ideal bilateral) en $S$.
  \end{enumerate}

  \begin{proof}
    Veamos el caso de ideales izquierdos; el caso de ideales derechos
    y bilaterales es similar.

    Tenemos $f (0_R) = 0_S \in I$, así que $0_R \in f^{-1} (I)$.
    Si $x,y \in f^{-1} (I)$, esto significa que $f (x), f (y) \in I$. Luego,
    $f (x+y) = f (x) + f (y) \in I$, así que $x+y \in f^{-1} (I)$. Ahora si
    $x\in f^{-1} (I)$, entonces $f (x) \in I$, y luego
    $f (r\,x) = f (r)\,f (x) \in I$ para cualesquiera $r \in R$, así que
    $r\,x \in f^{-1} (I)$.

    En la parte 2), tenemos $0_S = f (0_R)$ donde $0_R \in I$, así que
    $0_S \in f (I)$. Para $x,y \in I$ tenemos $x+y \in I$, así que
    $f (x), f (y) \in f (I)$ implica que $f (x) + f (y) = f (x+y) \in f
    (I)$. Para $x \in I$ y $s \in S$, dado que $f$ es una aplicación
    sobretectiva, se tiene $s = f (r)$ para algún $r\in R$. Luego,
    $s\,f (x) = f (r)\,f (x) = f (rx) \in f (I)$.
  \end{proof}
\end{observacion}

\begin{comentario}
  Si $f\colon R\to S$ es un homomorfismo que no es sobreyectivo e
  $I \subseteq R$ es un ideal, entonces $f (I)$ no tiene por qué ser un ideal
  en $S$. Considere por ejemplo la inclusión $f\colon \ZZ \hookrightarrow \QQ$.
\end{comentario}

% % % % % % % % % % % % % % % % % % % % % % % % % % % % % %

\section{Ideales generados}

\begin{definicion}
  Sea $R$ un anillo y $A \subset R$ un subconjunto.

  \begin{enumerate}
  \item[1)] El \term{ideal izquierdo generado por $A$}\index{ideal!generado}
    es el mínimo ideal izquierdo que contiene a $A$:
    $$RA \dfn \bigcap_{\substack{I\subseteq R \\ \text{izquierdo} \\ A\subseteq I}} I.$$

  \item[2)] El \term{ideal derecho generado por $A$} es el mínimo ideal derecho
    que contiene a $A$:
    $$AR \dfn \bigcap_{\substack{I\subseteq R \\ \text{derecho} \\ A\subseteq I}} I.$$

  \item[3)] El \term{ideal bilateral generado por $A$} es el mínimo ideal
    bilateral que contiene a $A$:
    $$RAR \dfn \bigcap_{\substack{I\subseteq R \\ \text{bilateral} \\ A\subseteq I}} I.$$
  \end{enumerate}
\end{definicion}

\begin{comentario}
  Si $A = \{ x_1,\ldots,x_n \}$ es un conjunto finito, se usa la notación
  \[ RA = R x_1 + \cdots + R x_n, \quad
     AR = x_1 R + \cdots + x_n R, \quad
     RAR = R x_1 R + \cdots + R x_n R. \]
\end{comentario}

\begin{comentario}
  Notamos que cuando $R$ es un anillo conmutativo, se tiene $RA = AR = RAR$,
  y normalmente este ideal se denota por $(A)$, o por $(x_1,\ldots,x_n)$ cuando
  $A = \{ x_1, \ldots, x_n \}$ es un conjunto finito.
\end{comentario}

\begin{definicion}
  Si $I \subseteq R$ es un ideal (izquierdo, derecho, bilateral) que puede ser
  generado por un número finito de elementos, se dice que $I$ es
  \term{finitamente generado}\index{ideal!finitamente generado}. Si $I$ puede
  ser generado por un elemento (es decir, $I = Rx$, $xR$, $RxR$
  respectivamente), se dice que $I$ es un
  \term{ideal principal}\index{ideal!principal}.
\end{definicion}

\begin{ejemplo}
  Todo ideal en $\ZZ$ es de la forma $n\ZZ$ para algún $n = 0,1,2,3,\ldots$
  El ideal $n\ZZ$ es el mínimo ideal que contiene a $n$, así que es exactamente
  el ideal generado por $n$. Entonces, todos los ideales en $\ZZ$
  son principales.
\end{ejemplo}

Más adelante vamos a estudiar los anillos conmutativos donde todos los ideales
son finitamente generados o donde todos los ideales son principales.

\begin{observacion}
  Sea $R$ un anillo y $A \subset R$ un subconjunto.

  \begin{enumerate}
  \item[1)] El ideal $RA$ consiste en todas las sumas finitas $\sum_i r_i\,a_i$
    donde $r_i\in R$ y $a_i \in A$.

  \item[2)] El ideal $AR$ consiste en todas las sumas finitas $\sum_i a_i\,r_i$
    donde $r_i\in R$ y $a_i \in A$.

  \item[3)] El ideal $RAR$ consiste en todas las sumas finitas
    $\sum_i r_i\,a_i\,r_i'$ donde $r_i,r_i'\in R$ y $a_i \in A$.
  \end{enumerate}

  \begin{proof}
    Verifiquemos, por ejemplo, la parte 1). Si $I$ es un ideal izquierdo tal que
    $A \subseteq I$, entonces $\sum_i r_i a_i \in I$ para cualesquiera
    $r_i \in R$, $a_i \in A$. Además, se ve que
    $$\Bigl\{ \sum_i r_i\,a_i \Bigm| r_i \in R, ~ a_i \in A \Bigr\}$$
    es un ideal izquierdo: es cerrado respecto a las sumas: si $\sum_i r_i\,a_i$
    y $\sum_j r_j'\,a_j'$ son sumas finitas con $r_i, r_j' \in R$ y
    $a_i, a_j' \in A$, entonces $\sum_i r_i\,a_i + \sum_j r_j'\,a_j'$ es una
    suma finita de la misma forma. Además, para todo $r\in R$
    $$r\,\sum_i r_i\,a_i = \sum_i (r\,r_i)\,a_i,$$
    así que el conjunto es cerrado respecto a la multiplicación por
    los elementos de $R$ por la izquierda.

    Las partes 2) y 3) se verifican de la misma manera.
  \end{proof}
\end{observacion}

\begin{corolario}[Sumas de ideales]
  Sea $R$ un anillo y sea $I_k \subseteq R$ una familia de ideales izquierdos
  (resp. derechos, resp. bilaterales). Entonces, el ideal izquierdo
  (resp. derecho, resp. bilateral) generado por los elementos de $I_k$ coincide
  con el conjunto
  $$\sum_k I_k \dfn \{ \text{sumas finitas }\sum_k x_k \mid x_k \in I_k \}$$
  y se llama la \term{suma}\index{suma!de ideales} de los ideales $I_k$.
  Este es el mínimo ideal izquierdo (resp. derecho, resp. bilateral) en $R$
  tal que $I_k \subseteq \sum_k I_k$ para todo $k$.

  \begin{proof}
    Por ejemplo, en el caso de ideales izquierdos, la observación anterior nos
    dice que hay que tomar las sumas finitas $\sum_i r_i a_i$ donde $r_i \in R$
    y $a_i \in I_k$ para algún $k$. Puesto que cada $I_k$ es un ideal izquierdo,
    en este caso se tiene $r_i a_i \in I_k$. Las sumas de elementos del mismo
    ideal $I_k$ también pertenecen a $I_k$. Entonces, el conjunto de las sumas
    finitas $\sum_i r_i a_i$ coincide con el conjunto de las sumas finitas
    $\sum_k x_k$ donde $x_k \in I_k$.
  \end{proof}
\end{corolario}

\begin{observacion}[Productos de ideales]
  Sea $R$ un anillo y sean $I_1,\ldots,I_n \subseteq R$ ideales izquierdos
  (resp. derechos, bilaterales).

  \begin{enumerate}
  \item[1)] El ideal izquierdo (resp. derecho, bilateral) generado por
    los productos $x_1\cdots x_n$ donde $x_k \in I_k$ coincide con el conjunto
    \[ I_1\cdots I_n \dfn
       \{ \text{sumas finitas } \sum_i x_{i_1}\cdots x_{i_n} \mid
          x_{i_k} \in I_k \} \]
    y se llama el \term{producto}\index{producto!de ideales} de los ideales
    $I_1,\ldots,I_n$.

  \item[2)] Si $I_1,\ldots,I_n$ son ideales bilaterales, entonces
    $$I_1\cdots I_n \subseteq I_1\cap \cdots \cap I_n.$$
  \end{enumerate}

  \begin{proof}
    Por ejemplo, en el caso de ideales izquierdos, hay que considerar las sumas
    $\sum_i r_i\,x_{i_1}\cdots x_{i_n}$ donde $r_i \in R$, pero $I_1$ es
    un ideal, así que $r_i\,x_{i_1} \in I_1$.

    Si todo $I_k$ es un ideal bilateral, tenemos
    $x_{i_1} \cdots x_{i_k} \cdots x_{i_n} \in I_k$ para todo $k = 1,\ldots,n$,
    así que $I_1\cdots I_n \subseteq I_k$ para todo $k$.
  \end{proof}
\end{observacion}

\begin{ejemplo}
  Para dos ideales $a\ZZ, b\ZZ \subseteq \ZZ$ tenemos
  \begin{align*}
    a\ZZ + b\ZZ & = d\ZZ, \quad d = \mcd (a,b),\\
    a\ZZ\cdot b\ZZ & = ab\ZZ,\\
    a\ZZ \cap b\ZZ & = m\ZZ, \quad m = \mcm (a,b).
  \end{align*}

  \[ \begin{tikzcd}
      & a\ZZ + b\ZZ\ar[-]{dr}\ar[-]{ddl} \\
      & & b\ZZ\ar[-]{ddl} \\
      a\ZZ\ar[-]{dr} \\
      & a\ZZ\cap b\ZZ\ar[-]{d} \\
      & a\ZZ\cdot b\ZZ
    \end{tikzcd} \]
\end{ejemplo}

\begin{definicion}
  Sea $R$ un anillo y sea $I \subseteq R$ un ideal bilateral. Para
  $n = 1,2,3,\ldots$ la $n$-ésima potencia de $I$ se define mediante
  \[ I^n \dfn \underbrace{I \cdots I}_n =
     \{ \text{sumas finitas }\sum_i x_{i_1}\cdots x_{i_n}\mid x_{i_k}\in I \} \]
  que es equivalente a la definición inductiva
  $$I^1 \dfn I, \quad I^n \dfn I\cdot I^{n-1}.$$ 
\end{definicion}

\begin{ejemplo}
  Sea $k$ un cuerpo y sea $k [X]$ el anillo de polinomios correspondiente.
  El ideal generado por $X$ en $k [X]$ viene dado por
  $$I \dfn (X) = \{ f\in k [X] \mid \deg f \ge 1 \} \cup \{ 0 \}.$$
  Luego, se ve que
  $$I^n = (X^n) = \{ f\in k [X] \mid \deg f \ge n \} \cup \{ 0 \}.$$
\end{ejemplo}

\begin{ejemplo}
  En el anillo $\ZZ [X]$ consideremos el ideal $I \dfn (2,X)$ generado por
  los elementos $2$ y $X$. Es el ideal de los polinomios con el término
  constante par:
  \[ (2,X) = \{ 2\,f + X\,g \mid f,g \in \ZZ [X] \} =
             \{ a_n\,X^n + a_{n-1}\,X + \cdots + a_1\,X + a_0 \mid
                n\ge 0, ~ a_i\in\ZZ, ~ a_0\text{ es par} \}. \]
  Luego,
  \[ I^2 = \Bigl\{ \text{sumas finitas }\sum_i f_i\,g_i \Bigm|
                   f_i,g_i \in I \Bigr\}. \]
  En particular, dado que $2\in I$ y $X \in I$, tenemos $4, X^2 \in I^2$, y por
  lo tanto $X^2 + 4 \in I^2$. Notamos que el polinomio $X^2 + 4$ no puede ser
  escrito como un producto $fg$ donde $f,g\in I$.
\end{ejemplo}

\begin{ejemplo}
  \label{ejemplo:operacion-V}
  Sea $k$ un cuerpo. Para una colección de polinomios
  $f_i \in k [X_1,\ldots,X_n]$ consideremos el conjunto de sus ceros comunes
  en $\AA^n (k)$:
  \[ V ( \{ f_i \}_{i\in I} ) \dfn
     \{ x\in \AA^n (k) \mid f_i (x) = 0\text{ para todo }i\in I \}. \]
  Diferentes colecciones $\{ f_i \}_{i\in I}$ pueden dar el mismo conjunto
  de los ceros. Para resolver este problema, podemos definir para todo ideal
  $J \subseteq k [X_1,\ldots,X_n]$
  $$V (J) \dfn \{ x\in \AA^n (k) \mid f (x) = 0\text{ para todo }f\in J \}.$$
  Ahora
  $$V ( \{ f_i \}_{i\in I} ) = V ((f_i)_{i\in I})$$
  donde $(f_i)_{i\in I}$ denota el ideal en $k [X_1,\ldots,X_n]$ generado por
  los polinomios $f_i$. En efecto, en general, la inclusión
  $\{ f_i \}_{i\in I} \subseteq (f_i)_{i\in I}$ implica que
  $V (\{ f_i \}_{i\in I}) \supseteq V ((f_i)_{i\in I})$. Viceversa,
  si $x\in V (\{ f_i \}_{i\in I})$, entonces $f_i (x) = 0$ para todo $i$,
  y por ende todas las sumas finitas $\sum_i g_i\,f_i$ se anulan en $x$.
\end{ejemplo}

Los ejemplos \ref{ejemplo:operacion-I} y \ref{ejemplo:operacion-V} nos dan dos
operaciones $I$ y $V$:
\[ \begin{tikzcd}
    \{ \text{ideales }J \subseteq k[X_1,\ldots,X_n] \} \ar[shift left]{r}{V} &
    \{ \text{subconjuntos }X\subseteq \AA^n (k) \} \ar[shift left]{l}{I}
  \end{tikzcd} \]
Vamos a ver algunas relaciones entre ellas en los ejercicios
\ref{ejerc:relaciones-para-I-y-V} y \ref{ejerc:IAn}. Su estudio pertenece
al terreno de la geometría algebraica. Para una introducción, el lector puede
consultar el libro \cite{Fulton-curves}.

% % % % % % % % % % % % % % % % % % % % % % % % % % % % % %

\section{El núcleo de un homomorfismo de anillos}

Un ejemplo importante de ideales bilaterales es el núcleo de un homomorfismo
de anillos.

\begin{observacion}
  Sea $f\colon R\to S$ un homomorfismo de anillos. Entonces, el conjunto
  $$\ker f \dfn \{ x\in R \mid f (x) = 0 \}$$
  es un ideal bilateral en $R$, llamado el
  \term{núcleo}\index{núcleo!de homomorfismo de anillos} de $f$.
\end{observacion}

Note que en la teoría de grupos, si $f\colon G\to H$ es un homomorfismo,
entonces $\ker f$ es un \emph{subgrupo normal} de $G$. Para un homomorfismo
de anillos $f\colon R\to S$, el núcleo $\ker f$ no es un \emph{subanillo} de
$R$, sino un \emph{ideal}.

\begin{proof}
  Un homomorfismo de anillos es en particular de los grupos abelianos
  correspondientes, y ya sabemos que el núcleo es un subgrupo abeliano.
  Falta comprobar que para cualesquiera $x\in \ker f$ y $r\cdot R$ se cumple
  $rx, xr \in \ker f$. En efecto, si $f (x) = 0$, entonces
  \[ f (rx) = f (r)\,f(x) = f (r)\cdot 0 = 0, \quad
     f (xr) = f (x)\,f(r) = 0\cdot f (r) = 0. \qedhere \]
\end{proof}

\begin{observacion}
  Un homomorfismo de anillos $f\colon R\to S$ es inyectivo (es decir,
  un monomorfismo) si y solo si $\ker f = 0$.

  \begin{proof}
    Ya lo verificamos para homomorfismos de grupos abelianos.
  \end{proof}
\end{observacion}

\begin{observacion}
  Sea $k$ un cuerpo y $R$ un anillo no nulo. Entonces, todo homomorfismo
  $f\colon k\to R$ es inyectivo.

  \begin{proof}
    Si $R \ne 0$, entonces $f (1_k) = 1_R \ne 0$ y $1_k \notin \ker f$. Pero las
    únicas opciones son $\ker f = 0$ y $\ker f = k$. Entonces, $\ker f = 0$.
  \end{proof}
\end{observacion}

% % % % % % % % % % % % % % % % % % % % % % % % % % % % % %

\section{Anillos cociente}

\begin{definicion}
  Sea $R$ un anillo y sea $I \subseteq R$ un ideal bilateral.
  El \term{anillo cociente}\index{anillo!cociente}\index[notacion]{RI@$R/I$}
  correspondiente $R/I$ es el grupo abeliano cociente $R/I$ con
  la multiplicación definida por
  $$(x + I)\cdot (y + I) \dfn (x y + I).$$
\end{definicion}

Hay que verificar que el producto está bien definido. Supongamos que
$x + I = x' + I$; es decir, $x - x' \in I$. Luego,
$x y - x'y = (x-x')\,y \in I$, dado que $I$ es un ideal derecho, y esto implica
que $xy + I = x'y + I$. De la misma manera, si $y + I = y' + I$, esto significa
que $y - y' \in I$. Esto implica que $x y - xy' = x\,(y-y') \in I$, puesto que
$I$ es un ideal izquierdo. De aquí se sigue que $xy + I = xy' + I$. Notamos que
en este argumento es importante que $I$ sea un ideal
\emph{bilateral}\footnote{De la misma manera, el producto sobre el grupo
  cociente $G/H$ está bien definido solo cuando $H$ es un subgrupo normal (véase
  el capítulo 7).}.

Dejo al lector verificar que los axiomas de anillo para el cociente $R/I$
se siguen de los axiomas correspondientes para $R$.

\begin{ejemplo}
  En todo anillo $R$ hay dos ideales evidentes: $I = 0$ e $I = R$.
  Al desarrollar las definiciones, se ve que $R/0 \isom R$ y $R/R = 0$.
\end{ejemplo}

\begin{ejemplo}
  El cociente del anillo $\ZZ$ por el ideal $n\ZZ$ es el anillo $\ZZ/n\ZZ$
  de los restos módulo $n$.
\end{ejemplo}

\begin{ejemplo}
  Tenemos $\RR [X]/(X^2 + 1) \isom \CC$. En efecto, puesto que
  $X^2 \equiv -1 \pmod{X^2 + 1}$ en el cociente, se ve que los elementos
  de $\RR [X]/(X^2 + 1)$ pueden ser representados por los polinomios $b\,X + a$,
  donde $a,b\in \RR$. Luego,
  \[ (b\,X + a)\,(d\,X + c) =
    bd\,X^2 + (bc + ad)\,X + ac \equiv (ac-bd) + (bc+ad)\,X \pmod{X^2 + 1}. \]
  Esta fórmula corresponde a la multiplicación compleja, y por ende se tiene
  un isomorfismo
  \begin{align*}
    \RR [X]/(X^2 + 1) & \xrightarrow{\isom} \CC\\
    b\,X + a & \mapsto a + b\sqrt{-1}.
  \end{align*}
\end{ejemplo}

\begin{ejemplo}[El cuerpo de cuatro elementos]
  Calculemos $\FF_2 [X] / (X^2 + X + 1)$. Puesto que
  $X^2 \equiv X + 1 \pmod{X^2 + X + 1}$, todos los elementos del cociente pueden
  ser representados por los polinomios de grado $\le 1$ en $\FF_2 [X]$:
  \[ \overline{0}, \quad \overline{1}, \quad
    \overline{X}, \quad \overline{X + 1}. \]

  La tabla de adición correspondiente viene dada por
    \[ \begin{array}{c|cccc}
         + & \overline{0} & \overline{1} & \overline{X} & \overline{X + 1} \\
         \hline
         \overline{0} & \overline{0} & \overline{1} & \overline{X} & \overline{X+1} \\
         \overline{1} & \overline{1} & \overline{0} & \overline{X + 1} & \overline{X} \\
         \overline{X} & \overline{X} & \overline{X + 1} & \overline{0} & \overline{1} \\
         \overline{X + 1} & \overline{X + 1} & \overline{X} & \overline{1} & \overline{0} \\
       \end{array} \]
 Notamos que este grupo es isomorfo al grupo de Klein $V \isom \ZZ/2\ZZ \times \ZZ/2\ZZ$. La tabla de multiplicación viene dada por
     \[ \begin{array}{c|cccc}
          \cdot & \overline{0} & \overline{1} & \overline{X} & \overline{X + 1} \\
          \hline
          \overline{0} & \overline{0} & \overline{0} & \overline{0} & \overline{0} \\
          \overline{1} & \overline{0} & \overline{1} & \overline{X} & \overline{X + 1} \\
          \overline{X} & \overline{0} & \overline{X } & \overline{X + 1} & \overline{1} \\
          \overline{X + 1} & \overline{0} & \overline{X + 1} & \overline{1} & \overline{X} \\
        \end{array} \]

  Se ve que todo elemento no nulo es invertible, así que
  $\FF_2 [X] / (X^2 + X + 1)$ es un cuerpo de cuatro elementos. Su grupo
  de elementos no nulos es de orden $3$ y en particular es cíclico. Esto
  coincide con el resultado del capítulo 7 que dice que si $k$ es un cuerpo,
  entonces todo subgrupo finito de $k^\times$ es necesariamente cíclico.
\end{ejemplo}

Más adelante en el curso vamos a construir todos los cuerpos finitos.

\begin{proposicion}[Propiedad universal del anillo cociente]
  \index{propiedad universal!del anillo cociente}
  \label{prop:propiedad-universal-del-anillo-cociente}
  Sea $I\subseteq R$ un ideal bilateral. Sea
  \begin{align*}
    p\colon R & \epi R/I,\\
    x & \mapsto x + I
  \end{align*}
  la proyección canónica sobre el anillo cociente. Si $f\colon R\to S$ es
  un homomorfismo de anillos tal que $I \subseteq \ker f$, entonces $f$
  se factoriza de modo único por $R/I$: existe un homomorfismo único
  $\overline{f}\colon R/I\to S$ tal que $f = \overline{f}\circ p$.

  \[ \begin{tikzcd}
      I \ar[>->]{d}\ar{dr}{= 0} \\
      R \ar{r}{f}\ar[->>]{d}[swap]{p} & S \\
      R/I\ar[dashed]{ur}{\exists !}[swap]{\overline{f}}
    \end{tikzcd} \]

  \begin{proof}
    La flecha punteada $\overline{f}$ es necesariamente
    $$x + I \mapsto f (x).$$
    Es una aplicación bien definida: si $x + I = x' + I$ para algunos
    $x,x'\in R$, entonces $x - x' \in I$, luego
    $x - x' \in \ker f$ y
    $$f (x - x') = 0 \iff f (x) = f (x').$$
    La aplicación $\overline{f}$ es un homomorfismo de anillos, puesto que
    $f$ lo es.
  \end{proof}
\end{proposicion}

\begin{corolario}[Funtorialidad del cociente]
  \index{funtorialidad!del cociente}
  \label{prop:funtorialidad-de-cocientes}
  ~

  \begin{enumerate}
  \item[1)] Sea $f\colon R\to S$ un homomorfismo de anillos.
    Sean $I \subseteq R$ y $J \subseteq S$ ideales bilaterales. Supongamos que
    $f (I) \subseteq J$. Entonces $f$ induce un homomorfismo canónico
    $\overline{f}\colon R/I \to S/J$ que conmuta con las proyecciones canónicas:

    \[ \begin{tikzcd}
        I\ar[dashed]{r}\ar[>->]{d} & J\ar[>->]{d} \\
        R\ar{r}{f}\ar[->>]{d} & S\ar[->>]{d} \\
        R/I\ar[dashed]{r}{\exists! \overline{f}} & S/J \\
      \end{tikzcd} \]

  \item[2)] La aplicación identidad $\idid\colon R\to R$ induce la aplicación
    identidad $\idid\colon R/I\to R/I$:

    \[ \begin{tikzcd}
        I\ar{r}{\idid}\ar[>->]{d} & I\ar[>->]{d} \\
        R\ar{r}{\idid}\ar[->>]{d} & R\ar[->>]{d} \\
        R/I\ar[dashed]{r}{\overline{\idid}=\idid} & R/I \\
      \end{tikzcd} \]

  \item[3)] Sean $f\colon R\to R'$ y $g\colon R'\to R''$ dos homomorfismos
    de anillos y sean $I\subseteq R$, $I'\subseteq R'$, $I''\subseteq R''$
    ideales bilaterales tales que $f (I) \subseteq I'$ y $g (I') \subseteq
    I''$. Entonces, $\overline{g\circ f} = \overline{g} \circ \overline{f}$:

    \[ \begin{tikzcd}
        I\ar[dashed]{r}\ar[>->]{d} & I'\ar[dashed]{r}\ar[>->]{d} & I''\ar[>->]{d} \\
        R\ar{r}{f}\ar[->>]{d} & R'\ar{r}{g}\ar[->>]{d} & R''\ar[->>]{d} \\
        R/I\ar[dashed]{r}{\overline{f}}\ar[dashed,bend right]{rr}[swap]{\overline{g\circ f} = \overline{g}\circ\overline{f}} & R'/I'\ar[dashed]{r}{\overline{g}} & R''/I'' \\
      \end{tikzcd} \]
  \end{enumerate}

  \begin{proof}
    En 1) la flecha $\overline{f}$ existe y es única gracias a la propiedad
    universal de $R/I$ aplicada a la composición $R\xrightarrow{f} S \epi
    S/J$. Los resultados de 2) y 3) siguen de la unicidad del homomorfismo
    inducido sobre los grupos cociente.
  \end{proof}
\end{corolario}

\begin{proposicion}[Primer teorema de isomorfía]
  \index{teorema!primer de isomorfía!para anillos}
  Sea $f\colon R\to S$ un homomorfismo de anillos. Entonces, existe
  un isomorfismo canónico
  $$\overline{f}\colon R/\ker f \xrightarrow{\isom} \im f$$
  que hace parte del diagrama conmutativo
  \[ \begin{tikzcd}
      R\ar{r}{f}\ar[->>]{d} & S \\
      R/\ker f\ar[dashed]{r}{\exists! \overline{f}}[swap]{\isom} & \im f\ar[>->]{u}
    \end{tikzcd} \]
\end{proposicion}

Descifremos el diagrama: la flecha $R\epi R/\ker f$ es la proyección canónica
$x\mapsto x + \ker f$ y la flecha $\im f \mono S$ es la inclusión de subanillo,
así que el isomorfismo $\overline{f}$ necesariamente viene dado por
$$\overline{f}\colon g + \ker f \mapsto f (g).$$

\begin{proof}
  La flecha $\overline{f}$ viene dada por la propiedad universal de $R/\ker f$:
  \[ \begin{tikzcd}
      \ker f\ar[>->]{d}\ar{dr}{=0}\\
      R\ar[->>]{r}{f}\ar[->>]{d} & \im f \\
      R/\ker f\ar[dashed]{ur}[swap]{\exists! \overline{f}}
    \end{tikzcd} \]
  Luego, el homomorfismo $\overline{f}$ es evidentemente sobreyectivo. Para ver
  que es inyectivo, notamos que
  \[ f (x) = f (y) \iff x-y \in \ker f
     \iff
     x + \ker f = y + \ker f. \qedhere \]
\end{proof}

\begin{ejemplo}
  Consideremos el homomorfismo de evaluación
  \begin{align*}
    f\colon \RR [X] & \hookrightarrow \CC [X] \to \CC,\\
    f & \mapsto f (\sqrt{-1}).
  \end{align*}
  Es visiblemente sobreyectivo. Su núcleo consiste en los polinomios en
  $\RR [X]$ que tienen $\sqrt{-1}$ como su raíz; es decir, los polinomios
  divisibles por $X^2 + 1$. Entonces, $\ker f = (X^2 + 1)$. Podemos concluir
  que $\RR [X]/(X^2 + 1) \isom \CC$.
\end{ejemplo}

\begin{ejemplo}
  Sea $p$ un número primo. Consideremos el anillo
  $$\ZZ_{(p)} \dfn \Bigl\{ \frac{a}{b} \in \QQ \Bigm| p\nmid b \Bigr\}.$$
  Este es un subanillo de $\QQ$. Consideremos la aplicación
  \begin{align*}
    f\colon \ZZ_{(p)} & \to \ZZ/p^k\ZZ,\\
    \frac{a}{b} & \mapsto [a]_{p^k}\,[b]_{p^k}^{-1}
  \end{align*}
  que a una fracción $\frac{a}{b} \in \ZZ_{(p)}$ asocia el producto del resto
  $[a]_{p^k}$ por el inverso multiplicativo $[b]_{p^k}^{-1}$ (que existe, dado
  que $p\nmid b$). Notamos que la aplicación está bien definida:
  si $\frac{a}{b} = \frac{a'}{b'}$, entonces
  $[a]_{p^k}\,[b]_{p^k}^{-1} = [a']_{p^k}\,[b']_{p^k}^{-1}$.
  Esto es un homomorfismo de anillos: tenemos
  $$f \left(\frac{1}{1}\right) = [1]_{p^k}\,[1]_{p^k}^{-1} = [1]_{p^k},$$
  \begin{multline*}
    f \left(\frac{a_1}{b_1} + \frac{a_2}{b_2}\right) =
    f \left(\frac{a_1 b_2 + a_2 b_1}{b_1 b_2}\right) =
    [a_1 b_2 + a_2 b_1]_{p^k}\,[b_1 b_2]_{p^k}^{-1} =
    ([a_1]_{p^k}\,[b_2]_{p^k} + [a_2]_{p^k}\,[b_1]_{p^k})\,[b_1]_{p^k}^{-1}\,[b_2]_{p^k}^{-1} \\
    = [a_1]_{p^k}\,[b_1]_{p^k}^{-1} + [a_2]_{p^k}\,[b_2]_{p^k}^{-1} =
    f \left(\frac{a_1}{b_1}\right) + f \left(\frac{a_2}{b_2}\right),
  \end{multline*}
  \[ f \left(\frac{a_1}{b_1}\cdot \frac{a_2}{b_2}\right) =
     f \left(\frac{a_1\,a_2}{b_1\,b_2}\right) =
     [a_1\,a_2]_{p^k}\,[b_1\,b_2]_{p^k}^{-1} =
     [a_1]_{p^k}\,[b_1]_{p^k}^{-1}\,[a_2]_{p^k}\,[b_2]_{p^k}^{-1} =
     f\left(\frac{a_1}{b_1}\right)\cdot f\left(\frac{a_2}{b_2}\right). \]
  Este homomorfismo es sobreyectivo: para $[a]_{p^k} \in \ZZ/p^k\ZZ$ tenemos
  $f \left(\frac{a}{1}\right) = [a]_{p^k}\,[1]_{p^k}^{-1} = [a]_{p^k}$.
  El núcleo viene dado por
  \[ \ker f =
     \Bigl\{\frac{a}{b}\in\ZZ_{(p)} \Bigm| [a]_{p^k}\,[b]_{p^k}^{-1} = 0\Bigr\} =
     \Bigl\{\frac{a}{b}\in\ZZ_{(p)} \Bigm| [a]_{p^k} = 0\Bigr\} =
     \Bigl\{\frac{a}{b}\in\ZZ_{(p)} \Bigm| p^k\mid a\Bigr\}. \]
  Este es precisamente el ideal $p^k\ZZ_{(p)}$ generado por $p^k$. El primer
  teorema de isomorfía nos dice que
  $$\ZZ_{(p)}/p^k\ZZ_{(p)} \isom \ZZ/p^k\ZZ.$$
\end{ejemplo}

Ahora vamos a formular el segundo y el tercer teorema de isomorfía, pero
los dejo como un ejercicio.

\begin{teorema}[Segundo teorema de isomorfía]
  \index{teorema!segundo de isomorfía!para anillos}
  Sean $R$ un anillo, $S \subseteq R$ un subanillo y $I \subseteq R$ un ideal
  bilateral. Entonces,

  \begin{enumerate}
  \item[1)] $S + I \dfn \{ x + y \mid x\in S, y\in I \}$ es un subanillo de $R$;

  \item[2)] $I$ es un ideal bilateral en $S + I$;

  \item[3)] la aplicación
    \begin{align*}
      S & \to (S + I)/I,\\
      x & \mapsto x + I
    \end{align*}
    es un homomorfismo de anillos sobreyectivo que tiene como núcleo
    a $S \cap I$.
  \end{enumerate}

  Luego, gracias al primer teorema de isomorfía,
  $$S/(S\cap I) \isom (S + I)/I.$$
\end{teorema}

\begin{teorema}[Tercer teorema de isomorfía]
  \index{teorema!tercer de isomorfía!para anillos}
  Sea $R$ un anillo y sean $I \subseteq J \subseteq R$ ideales
  bilaterales. Entonces, la aplicación
  \begin{align*}
    R/I & \to R/J,\\
    x + I & \mapsto x + J
  \end{align*}
  está bien definida y es un homomorfismo sobreyectivo que tiene como núcleo a
  $$J/I \dfn \{ x + I \mid x\in J \} \subseteq R/I.$$
  Luego, gracias al primer teorema de isomorfía,
  $$(R/I)/(J/I) \isom R/J.$$
\end{teorema}

En fin, vamos a describir los ideales en el anillo cociente.

\begin{teorema}
  Sea $R$ un anillo y sea $I \subseteq R$ un ideal bilateral. Denotemos por
  $p\colon R\to R/I$ la proyección sobre el anillo cociente dada por
  $x \mapsto x + I$. Hay una biyección
  \begin{align*}
    \{ \text{ideales bilaterales } J\subseteq R\text{ tales que }I\subseteq J \} & \leftrightarrow
                                                                                    \{ \text{ideales bilaterales }\overline{J} \subseteq R/I \},\\
    J & \mapsto p (J) = J/I,\\
    p^{-1} (\overline{J}) & \mapsfrom \overline{J}.
  \end{align*}
  Esta biyección preserva las inclusiones:
  \begin{enumerate}
  \item[1)] si $I \subseteq J_1 \subseteq J_2$, entonces
    $J_1/I \subseteq J_2/I$;

  \item[2)] si $\overline{J_1} \subseteq \overline{J_2} \subseteq R/I$, entonces
    $p^{-1} (\overline{I_1}) \subseteq p^{-1} (\overline{J_2})$.
  \end{enumerate}

  \begin{proof}
    El hecho de que las aplicaciones estén bien definidas se sigue
    de \ref{obs:imagen-y-preimagen-de-un-ideal}. El homomorfismo
    $p\colon R\to R/I$ es sobreyectivo, así que para todo ideal $J \subseteq R$
    su imagen $p (J)$ es un ideal en $R/I$. Para todo ideal
    $\overline{J} \subseteq R/I$ la preimagen $p^{-1} (\overline{J})$ es
    un ideal en $R$. Además, tenemos $0_{R/I} \in \overline{J}$ para todo ideal
    $\overline{J} \subseteq R/I$ y $p^{-1} (0_{R/I}) = I$, así que
    $I \subseteq p^{-1} (\overline{J})$.

    Hay que ver que las aplicaciones $J \mapsto J/I$ y
    $\overline{J} \mapsto p^{-1} (\overline{J})$ son mutualmente
    inversas. Tenemos claramente
    $p^{-1} (\overline{J})/I = p (p^{-1} (\overline{J})) = \overline{J}$ y
    \begin{multline*}
      p^{-1} (J/I) =
      \{ x\in R \mid p (x) \in J/I \} =
      \{ x\in R \mid p (x) = p (x') \text{ para algún }x' \in J \} \\
      = \{ x\in R \mid x - x' \in \ker p = I \text{ para algún }x' \in J \} = J
    \end{multline*}
    (usando que $I \subseteq J$). Está claro que las dos aplicaciones preservan
    las inclusiones (esto es la teoría de conjuntos elemental).
  \end{proof}
\end{teorema}

\begin{ejemplo}
  Describamos los ideales en el anillo cociente $\ZZ/12\ZZ$. Según el teorema,
  estos corresponden a los ideales en $\ZZ$ que contienen a $12\ZZ$:
  $$12\ZZ \subseteq n\ZZ \subseteq \ZZ.$$
  Dado que $12\ZZ \subseteq n\ZZ$ significa que $n \mid 12$, tenemos $n = 1,2,3,4,6,12$. Entonces, los ideales en el cociente son
  \begin{align*}
    \ZZ/12\ZZ & = \{ [0], [1], [2], [3], [4], [5], [6], [7], [8], [9], [10], [11] \},\\
    2\ZZ/12\ZZ & = \{ [0], [2], [4], [6], [8], [10] \},\\
    3\ZZ/12\ZZ & = \{ [0], [3], [6], [9] \},\\
    4\ZZ/12\ZZ & = \{ [0], [4], [8] \},\\
    6\ZZ/12\ZZ & = \{ [0], [6] \},\\
    12\ZZ/12\ZZ & = 0.
  \end{align*}
  Tenemos las siguientes inclusiones de ideales:
  \[ \begin{tikzcd}[row sep=1em, column sep=1em]
      & \ZZ\ar[-]{ddl}\ar[-]{ddddr} \\
      \\
      2\ZZ\ar[-]{ddd}\ar[-]{ddddr} \\
      \\
      & & 3\ZZ\ar[-]{ddl} \\
      4\ZZ\ar[-]{ddr} \\
      & 6\ZZ\ar[-]{d} \\
      & 0
    \end{tikzcd}
    \quad\quad\quad\quad
    \begin{tikzcd}[row sep=1em, column sep=1em]
      & \ZZ/12\ZZ\ar[-]{ddl}\ar[-]{ddddr} \\
      \\
      2\ZZ/12\ZZ\ar[-]{ddd}\ar[-]{ddddr} \\
      \\
      & & 3\ZZ/12\ZZ\ar[-]{ddl} \\
      4\ZZ/12\ZZ\ar[-]{ddr} \\
      & 6\ZZ/12\ZZ\ar[-]{d} \\
      & 0
    \end{tikzcd} \]
\end{ejemplo}

% % % % % % % % % % % % % % % % % % % % % % % % % % % % % %

\section{Productos de anillos}

\begin{definicion}
  Sea $R_i$, $i\in I$ una familia anillos.
  El \term{producto}\index{producto!de anillos} $\prod_{i\in I} R_i$
  es el conjunto
  $$\prod_{i\in I} R_i \dfn \{ (x_i)_{i\in I} \mid x_i \in R_i \}$$
  con la suma y producto definidos término por término:
  \begin{align*}
    (x_i)_{i\in I} + (y_i)_{i\in I} & \dfn (x_i + y_i)_{i\in I},\\
    (x_i)_{i\in I} \cdot (y_i)_{i\in I} & \dfn (x_i\,y_i)_{i\in I}.
  \end{align*}
\end{definicion}

Ya que las operaciones se definen término por término, los axiomas de anillos
para los $R_i$ implican los axiomas correspondientes para el producto
$\prod_{i\in I} R_i$. El cero es el elemento donde $x_i = 0$ para todo $i\in I$
y la identidad es el elemento donde $x_i = 1$ para todo $i\in I$. Notamos
que las proyecciones naturales sobre cada uno de los $R_i$:
\begin{align*}
  p_i\colon \prod_{i\in I} R_i & \epi R_i,\\
  (x_i)_{i\in I} & \mapsto x_i
\end{align*}
son homomorfismos de anillos\footnote{Note que las inclusiones
  $R_i \mono \prod_{i\in I} R_i$ no son homomorfismos de anillos: la identidad
  no se preserva.}.

Cuando $I = \{ 1, \ldots, n \}$ es un conjunto finito, se usa la notación
$R_1\times \cdots \times R_n$.

\begin{observacion}[Propiedad universal del producto]
  \index{propiedad universal!del producto!de anillos}
  Sea $R_i$, $i\in I$ una familia de anillos y $S$ cualquier otro anillo.
  Para toda familia de homomorfismos de anillos
  $\{ f_i\colon S\to R_i \}_{i\in I}$ existe un único homomorfismo de anillos
  $f\colon S\to \prod_{i\in I} R_i$ tal que $p_i\circ f = f_i$ para todo
  $i\in I$.

  \[ \begin{tikzcd}
      S \ar[dashed]{d}{f}[swap]{\exists !}\ar{dr}{f_i} \\
      \prod_{i\in I} R_i \ar{r}[swap]{p_i} & R_i
    \end{tikzcd} \]

  En otras palabras, hay una biyección natural entre conjuntos
  \begin{align*}
    \left\{ \text{homomorfismos }S \to \prod_{i\in I} R_i \right\} & \xrightarrow{\isom}
    \prod_{i\in I} \{ \text{homomorfismos }S \to R_i \},\\
    f & \mapsto p_i\circ f.
  \end{align*}

  \begin{proof}
    La condición $p_i\circ f = f_i$ implica que $f$ viene dado por
    $$s \mapsto (f_i (s))_{i\in I}.$$
    Puesto que cada uno de los $f_i$ es un homomorfismo de anillos, esta fórmula
    define un homomorfismo de anillos.
  \end{proof}
\end{observacion}

\begin{observacion}
  Sean $R_1, R_2, R_3$ anillos.

  \begin{enumerate}
  \item[1)] Hay un isomorfismo natural de anillos
    $R_1\times R_2 \isom R_2 \times R_1$.

  \item[2)] Hay isomorfismos naturales
    $(R_1\times R_2)\times R_3 \isom R_1 \times (R_2\times R_3) \isom R_1 \times R_2 \times R_3$.
  \end{enumerate}

  \begin{proof}
    Ejercicio para el lector. Esto se puede deducir de la propiedad universal
    del producto (véase las pruebas correspondientes para los productos
    de grupos en el capítulo 10).
  \end{proof}
\end{observacion}

\begin{observacion}[$R \rightsquigarrow R^\times$ preserva productos]
  \label{obs:grupo-de-unidades-preserva-productos}
  Sea $R_i$, $i\in I$ una familia de anillos. Hay un isomorfismo natural
  de grupos
  $$\Bigl(\prod_{i\in I} R_i\Bigr)^\times \isom \prod_{i\in I} R_i^\times.$$

  \begin{proof}
    Dado que el producto en el anillo $\prod_{i\in I} R_i$ está definido término
    por término, un elemento $(x_i)_{i\in I}$ es invertible
    en $\prod_{i\in I} R_i$ si y solo si cada $x_i$ es invertible en $R_i$.
  \end{proof}
\end{observacion}

\begin{digresion}[*]
  Otra prueba más general y abstracta puede ser obtenida de
  \ref{prop:propiedad-universal-de-RG}. Para cualquier grupo $G$ y anillo $R$
  hay una biyección natural
  \[ \{ \text{homomorfismos de anillos }\ZZ [G]\to R \} \isom
     \{ \text{homomorfismos de grupos }G\to R^\times \}. \]

  Junto con la propiedad universal del producto de grupos y de anillos, esto nos
  da biyecciones \emph{naturales} para cualquier grupo $G$
  \begin{multline*}
    \left\{ \text{homom. de grupos }G\to \prod_{i\in I} R_i^\times \right\} \isom
    \prod_{i\in I} \{ \text{homom. de grupos }G\to R_i^\times \} \\
    \isom \prod_{i\in I} \{ \text{homom. de anillos }\ZZ [G]\to R_i \} \isom
    \left\{ \text{homom. de anillos }\ZZ [G]\to \prod_{i\in I} R_i \right\} \\
    \isom \left\{ \text{homom. de grupos }G\to \Bigl(\prod_{i\in I} R_i\Bigr)^\times \right\}.
  \end{multline*}
  Esto es suficiente para concluir que
  $\prod_{i\in I} R_i^\times \isom \Bigl(\prod_{i\in I} R_i\Bigr)^\times$, pero
  omitiré los detalles. El punto es que el isomorfismo
  de \ref{obs:grupo-de-unidades-preserva-productos} puede ser obtenido como
  una consecuencia formal de \ref{prop:propiedad-universal-de-RG}.
\end{digresion}

En el capítulo 10 hemos probado que si $m$ y $n$ son coprimos, entonces hay
un isomorfismo de \emph{grupos abelianos}
$\ZZ/mn\ZZ \isom \ZZ/m\ZZ\times \ZZ/n\ZZ$. En realidad, esto es un isomorfismo
de \emph{anillos}, y ahora estamos listos para probar una generalización.

\begin{nameless}\textbf{Teorema chino del resto}.
  \index{teorema!chino del resto}
  \emph{Sea $R$ un anillo conmutativo y sean $I_1, \ldots, I_n \subseteq R$
    ideales tales que $I_k + I_\ell = R$ para $k \ne \ell$. Luego, hay
    un isomorfismo natural}
  $$R/(I_1\cdots I_n) \isom R/I_1 \times \cdots \times R/I_n.$$

\begin{proof}
  Denotemos por $p_k\colon R\to R/I_k$ las proyecciones canónicas
  $x \mapsto x + I_k$. Estas inducen un homomorfismo de anillos
  \begin{align*}
    R & \to R/I_1 \times \cdots \times R/I_n,\\
    x & \mapsto (x+I_1,\ldots,x+I_n).
  \end{align*}
  Vamos a probar que es sobreyectivo y su núcleo es igual al producto de ideales
  $I_1\cdots I_n$.

  Escribamos $x \equiv y \pmod{I}$ para $x + I = y + I$. Para ver
  la sobreyectividad, necesitamos probar que para cualesquiera
  $x_1,\ldots,x_n\in R$ existe $x\in I$ tal que $x \equiv x_k \pmod{I_k}$ para
  todo $k = 1, \ldots, n$. Tenemos
  \[ R = R\cdots R = (I_1 + I_2)\,(I_1 + I_3)\cdots (I_1 + I_n) =
     I + I_2\,I_3\cdots I_n, \]
   donde $I \subseteq I_1$. De hecho, al desarrollar el producto de sumas
   de ideales (vése el ejercicio \ref{ejerc:producto-de-anillos-distributivo}),
   se ve que todos los términos pertenecen a $I_1$, salvo el último término
   $I_2\,I_3\cdots I_n$. Podemos concluir que
   \begin{equation}
     \label{eqn:I1-I2I3In-coprimos}
     I_1 + I_2\,I_3\cdots I_n = R
   \end{equation}
   En particular, existen $z_1 \in I_1$ e $y_1 \in I_2\,I_3\cdots I_n$ tales
   que $z_1 + y_1 = 1$. Tenemos entonces $y_1 \equiv 1 \pmod{I_1}$ e
   $y_1 \equiv 0 \pmod{I_k}$ para $k \ne 1$. Usando el mismo argumento, podemos
   ver que existen $y_2, \ldots, y_n$ que satisfacen
   \[ y_k \equiv 1 \pmod{I_k}, \quad y_k \equiv 0 \pmod{I_\ell}
      \text{ si }\ell \ne k. \]
   El elemento
   $$x \dfn x_1\,y_1 + \cdots + x_n\,y_n$$
   cumple la condición deseada $x \equiv x_k \pmod{I_k}$ para todo
   $k = 1, \ldots, n$.

   Ahora necesitamos calcular el núcleo del homomorfismo
   $x \mapsto (x+I_1,\ldots,x+I_n)$. Está claro que
   $$\ker (x \mapsto (x+I_1,\ldots,x+I_n)) = I_1 \cap \cdots \cap I_n.$$
   Vamos a probar que la hipótesis de que $I_k + I_\ell = R$ para $k \ne \ell$
   implica que
   $$I_1 \cap \cdots \cap I_n = I_1\cdots I_n.$$
   La inclusión que se cumple en cualquier caso es
   $I_1\cdots I_n \subseteq I_1\cap\cdots\cap I_n$, y hay que probar
   la inclusión inversa.

   Procedamos por inducción sobre $n$. Supongamos que $n = 2$ e
   $I_1 + I_2 = R$. Luego, existen $y \in I_1$ y $z \in I_2$ tales que
   $y + z = 1$. Para todo $x \in I_1 \cap I_2$ se tiene
   $$x = x\,(y+z) = xy + xz \in I_1\,I_2,$$
   así que $I_1\cap I_2 \subseteq I_1\,I_2$.

   Para $n > 2$, supongamos que el resultado se cumple para $n-1$
   ideales. Entonces,
   \[ I_1 \cap \cdots \cap I_n = I_1 \cap (I_2\cap I_3\cap \cdots\cap I_n) =
      I_1 \cap I_2\,I_3\cdots I_n. \]
   Sin embargo, $I_1 + I_2\,I_3\cdots I_n = R$ (véase
   \eqnref{eqn:I1-I2I3In-coprimos}), así que
   $I_1 \cap I_2\,I_3\cdots I_n = I_1\,I_2\,I_3\cdots I_n$ por el caso de dos
   ideales.
 \end{proof}
\end{nameless}

\begin{corolario}
  Si $a_1, \ldots, a_n$ son números coprimos dos a dos, entonces hay
  un isomorfismo de anillos
  $$\ZZ/a_1\cdots a_n\ZZ \isom \ZZ/a_1\ZZ \times \cdots \times \ZZ/a_n\ZZ.$$

  \begin{proof}
    Para dos ideales $m\ZZ$ y $n\ZZ$ en $\ZZ$ tenemos $m\ZZ\cdot n\ZZ = mn\ZZ$
    y $m\ZZ + n\ZZ = d\ZZ$, donde $d = \mcd (m,n)$. En particular, si $m$ y $n$
    son coprimos, $m\ZZ + n\ZZ = \ZZ$. Se cumplen las condiciones del teorema
    anterior y se obtiene un isomorfismo
    \[ \ZZ/a_1\cdots a_n\ZZ \isom
       \ZZ/a_1\ZZ \times \cdots \times \ZZ/a_n\ZZ. \qedhere \]
  \end{proof}
\end{corolario}

\begin{corolario}
  ~

  \begin{enumerate}
  \item[1)] La función de Euler $\phi (n) = |(\ZZ/n\ZZ)^\times|$ es
    multiplicativa: si $m$ y $n$ son coprimos, entonces
    $$\phi (mn) = \phi (m)\,\phi (n).$$

  \item[2)] Si $n$ se factoriza en números primos como
    $p_1^{k_1}\cdots p_\ell^{k_\ell}$, entonces
    \[ \phi (n) = \phi (p_1^{k_1})\cdots \phi (p_\ell^{k_\ell}) =
        n\,\left(1 - \frac{1}{p_1}\right)\cdots\left(1 - \frac{1}{p_\ell}\right). \]
  \end{enumerate}

  \begin{proof}
    Si $m$ y $n$ son coprimos, el isomorfismo de anillos
    $$\ZZ/mn\ZZ \isom \ZZ/m\ZZ \times \ZZ/n\ZZ$$
    induce un isomorfismo de grupos
    $$(\ZZ/mn\ZZ)^\times \isom (\ZZ/m\ZZ)^\times \times (\ZZ/n\ZZ)^\times,$$
    como notamos en \ref{obs:grupo-de-unidades-preserva-productos}. De aquí
    sigue 1). La parte 2) se demuestra de la misma manera o por inducción usando
    la parte 1). La fórmula
    $$\phi (p^k) = p^k\,\left(1 - \frac{1}{p}\right)$$
    fue obtenida en el capítulo 4.
  \end{proof}
\end{corolario}

En el capítulo 10 hemos probado la multiplicatividad de la función $\phi$
de Euler usando que los elementos de $(\ZZ/n\ZZ)^\times$ son los generadores
del grupo cíclico $\ZZ/n\ZZ$. La prueba de arriba es más natural.
