\documentclass{article}

\def\exercisespersection{}
% TODO : CLEAN UP THIS MESS
% (AND MAKE SURE ALL TEXTS STILL COMPILE)
\usepackage[leqno]{amsmath}
\usepackage{amssymb}
\usepackage{graphicx}

\usepackage{diagbox} % table heads with diagonal lines
\usepackage{relsize}

\usepackage{wasysym}
\usepackage{scrextend}
\usepackage{epigraph}
\setlength\epigraphwidth{.6\textwidth}

\usepackage[utf8]{inputenc}

\usepackage{titlesec}
\titleformat{\chapter}[display]
  {\normalfont\sffamily\huge\bfseries}
  {\chaptertitlename\ \thechapter}{5pt}{\Huge}
\titleformat{\section}
  {\normalfont\sffamily\Large\bfseries}
  {\thesection}{1em}{}
\titleformat{\subsection}
  {\normalfont\sffamily\large\bfseries}
  {\thesubsection}{1em}{}
\titleformat{\part}[display]
  {\normalfont\sffamily\huge\bfseries}
  {\partname\ \thepart}{0pt}{\Huge}

\usepackage[T1]{fontenc}
\usepackage{fourier}
\usepackage{paratype}

\usepackage[symbol,perpage]{footmisc}

\usepackage{perpage}
\MakePerPage{footnote}

\usepackage{array}
\newcolumntype{x}[1]{>{\centering\hspace{0pt}}p{#1}}

% TODO: the following line causes conflict with new texlive (!)
% \usepackage[english,russian,polutonikogreek,spanish]{babel}
% \newcommand{\russian}[1]{{\selectlanguage{russian}#1}}

% Remove conflicting options for the moment:
\usepackage[english,polutonikogreek,spanish]{babel}

\AtBeginDocument{\shorthandoff{"}}
\newcommand{\greek}[1]{{\selectlanguage{polutonikogreek}#1}}

% % % % % % % % % % % % % % % % % % % % % % % % % % % % % %
% Limit/colimit symbols (with accented i: lím / colím)

\usepackage{etoolbox} % \patchcmd

\makeatletter
\patchcmd{\varlim@}{lim}{\lim}{}{}
\makeatother
\DeclareMathOperator*{\colim}{co{\lim}}
\newcommand{\dirlim}{\varinjlim}
\newcommand{\invlim}{\varprojlim}

% % % % % % % % % % % % % % % % % % % % % % % % % % % % % %

\usepackage[all,color]{xy}

\usepackage{pigpen}
\newcommand{\po}{\ar@{}[dr]|(.4){\text{\pigpenfont I}}}
\newcommand{\pb}{\ar@{}[dr]|(.3){\text{\pigpenfont A}}}
\newcommand{\polr}{\ar@{}[dr]|(.65){\text{\pigpenfont A}}}
\newcommand{\pour}{\ar@{}[ur]|(.65){\text{\pigpenfont G}}}
\newcommand{\hstar}{\mathop{\bigstar}}

\newcommand{\bigast}{\mathop{\Huge \mathlarger{\mathlarger{\ast}}}}

\newcommand{\term}{\textbf}

\usepackage{stmaryrd}

\usepackage{cancel}

\usepackage{tikzsymbols}

\newcommand{\open}{\underset{\mathrm{open}}{\hookrightarrow}}
\newcommand{\closed}{\underset{\mathrm{closed}}{\hookrightarrow}}

\newcommand{\tcol}[2]{{#1 \choose #2}}

\newcommand{\homot}{\simeq}
\newcommand{\isom}{\cong}
\newcommand{\cH}{\mathcal{H}}
\renewcommand{\hom}{\mathrm{hom}}
\renewcommand{\div}{\mathop{\mathrm{div}}}
\renewcommand{\Im}{\mathop{\mathrm{Im}}}
\renewcommand{\Re}{\mathop{\mathrm{Re}}}
\newcommand{\id}[1]{\mathrm{id}_{#1}}
\newcommand{\idid}{\mathrm{id}}

\newcommand{\ZG}{{\ZZ G}}
\newcommand{\ZH}{{\ZZ H}}

\newcommand{\quiso}{\simeq}

\newcommand{\personality}[1]{{\sc #1}}

\newcommand{\mono}{\rightarrowtail}
\newcommand{\epi}{\twoheadrightarrow}
\newcommand{\xepi}[1]{\xrightarrow{#1}\mathrel{\mkern-14mu}\rightarrow}

% % % % % % % % % % % % % % % % % % % % % % % % % % % % % %

\DeclareMathOperator{\Ad}{Ad}
\DeclareMathOperator{\Aff}{Aff}
\DeclareMathOperator{\Ann}{Ann}
\DeclareMathOperator{\Aut}{Aut}
\DeclareMathOperator{\Br}{Br}
\DeclareMathOperator{\CH}{CH}
\DeclareMathOperator{\Cl}{Cl}
\DeclareMathOperator{\Coeq}{Coeq}
\DeclareMathOperator{\Coind}{Coind}
\DeclareMathOperator{\Cop}{Cop}
\DeclareMathOperator{\Corr}{Corr}
\DeclareMathOperator{\Cor}{Cor}
\DeclareMathOperator{\Cov}{Cov}
\DeclareMathOperator{\Der}{Der}
\DeclareMathOperator{\Div}{Div}
\DeclareMathOperator{\D}{D}
\DeclareMathOperator{\Ehr}{Ehr}
\DeclareMathOperator{\End}{End}
\DeclareMathOperator{\Eq}{Eq}
\DeclareMathOperator{\Ext}{Ext}
\DeclareMathOperator{\Frac}{Frac}
\DeclareMathOperator{\Frob}{Frob}
\DeclareMathOperator{\Funct}{Funct}
\DeclareMathOperator{\Fun}{Fun}
\DeclareMathOperator{\GL}{GL}
\DeclareMathOperator{\Gal}{Gal}
\DeclareMathOperator{\Gr}{Gr}
\DeclareMathOperator{\Hol}{Hol}
\DeclareMathOperator{\Hom}{Hom}
\DeclareMathOperator{\Ho}{Ho}
\DeclareMathOperator{\Id}{Id}
\DeclareMathOperator{\Ind}{Ind}
\DeclareMathOperator{\Inn}{Inn}
\DeclareMathOperator{\Isom}{Isom}
\DeclareMathOperator{\Ker}{Ker}
\DeclareMathOperator{\Lan}{Lan}
\DeclareMathOperator{\Lie}{Lie}
\DeclareMathOperator{\Map}{Map}
\DeclareMathOperator{\Mat}{Mat}
\DeclareMathOperator{\Max}{Max}
\DeclareMathOperator{\Mor}{Mor}
\DeclareMathOperator{\Nat}{Nat}
\DeclareMathOperator{\Nrd}{Nrd}
\DeclareMathOperator{\Ob}{Ob}
\DeclareMathOperator{\Out}{Out}
\DeclareMathOperator{\PGL}{PGL}
\DeclareMathOperator{\PSL}{PSL}
\DeclareMathOperator{\PSU}{PSU}
\DeclareMathOperator{\Pic}{Pic}
\DeclareMathOperator{\RHom}{RHom}
\DeclareMathOperator{\Rad}{Rad}
\DeclareMathOperator{\Ran}{Ran}
\DeclareMathOperator{\Rep}{Rep}
\DeclareMathOperator{\Res}{Res}
\DeclareMathOperator{\SL}{SL}
\DeclareMathOperator{\SO}{SO}
\DeclareMathOperator{\SU}{SU}
\DeclareMathOperator{\Sh}{Sh}
\DeclareMathOperator{\Sing}{Sing}
\DeclareMathOperator{\Specm}{Specm}
\DeclareMathOperator{\Spec}{Spec}
\DeclareMathOperator{\Sp}{Sp}
\DeclareMathOperator{\Stab}{Stab}
\DeclareMathOperator{\Sym}{Sym}
\DeclareMathOperator{\Tors}{Tors}
\DeclareMathOperator{\Tor}{Tor}
\DeclareMathOperator{\Tot}{Tot}
\DeclareMathOperator{\UUU}{U}

\DeclareMathOperator{\adj}{adj}
\DeclareMathOperator{\ad}{ad}
\DeclareMathOperator{\af}{af}
\DeclareMathOperator{\card}{card}
\DeclareMathOperator{\cm}{cm}
\DeclareMathOperator{\codim}{codim}
\DeclareMathOperator{\cod}{cod}
\DeclareMathOperator{\coeq}{coeq}
\DeclareMathOperator{\coim}{coim}
\DeclareMathOperator{\coker}{coker}
\DeclareMathOperator{\cont}{cont}
\DeclareMathOperator{\conv}{conv}
\DeclareMathOperator{\cor}{cor}
\DeclareMathOperator{\depth}{depth}
\DeclareMathOperator{\diag}{diag}
\DeclareMathOperator{\diam}{diam}
\DeclareMathOperator{\dist}{dist}
\DeclareMathOperator{\dom}{dom}
\DeclareMathOperator{\eq}{eq}
\DeclareMathOperator{\ev}{ev}
\DeclareMathOperator{\ex}{ex}
\DeclareMathOperator{\fchar}{char}
\DeclareMathOperator{\fr}{fr}
\DeclareMathOperator{\gr}{gr}
\DeclareMathOperator{\im}{im}
\DeclareMathOperator{\infl}{inf}
\DeclareMathOperator{\interior}{int}
\DeclareMathOperator{\intrel}{intrel}
\DeclareMathOperator{\inv}{inv}
\DeclareMathOperator{\length}{length}
\DeclareMathOperator{\mcd}{mcd}
\DeclareMathOperator{\mcm}{mcm}
\DeclareMathOperator{\multideg}{multideg}
\DeclareMathOperator{\ord}{ord}
\DeclareMathOperator{\pr}{pr}
\DeclareMathOperator{\rel}{rel}
\DeclareMathOperator{\res}{res}
\DeclareMathOperator{\rkred}{rkred}
\DeclareMathOperator{\rkss}{rkss}
\DeclareMathOperator{\rk}{rk}
\DeclareMathOperator{\sgn}{sgn}
\DeclareMathOperator{\sk}{sk}
\DeclareMathOperator{\supp}{supp}
\DeclareMathOperator{\trdeg}{trdeg}
\DeclareMathOperator{\tr}{tr}
\DeclareMathOperator{\vol}{vol}

\newcommand{\iHom}{\underline{\Hom}}

\renewcommand{\AA}{\mathbb{A}}
\newcommand{\CC}{\mathbb{C}}
\renewcommand{\SS}{\mathbb{S}}
\newcommand{\TT}{\mathbb{T}}
\newcommand{\PP}{\mathbb{P}}
\newcommand{\BB}{\mathbb{B}}
\newcommand{\RR}{\mathbb{R}}
\newcommand{\ZZ}{\mathbb{Z}}
\newcommand{\FF}{\mathbb{F}}
\newcommand{\HH}{\mathbb{H}}
\newcommand{\NN}{\mathbb{N}}
\newcommand{\QQ}{\mathbb{Q}}
\newcommand{\KK}{\mathbb{K}}

% % % % % % % % % % % % % % % % % % % % % % % % % % % % % %

\usepackage{amsthm}

\newcommand{\legendre}[2]{\left(\frac{#1}{#2}\right)}

\newcommand{\examplesymbol}{$\blacktriangle$}
\renewcommand{\qedsymbol}{$\blacksquare$}

\newcommand{\dfn}{\mathrel{\mathop:}=}
\newcommand{\rdfn}{=\mathrel{\mathop:}}

\usepackage{xcolor}
\definecolor{mylinkcolor}{rgb}{0.0,0.4,1.0}
\definecolor{mycitecolor}{rgb}{0.0,0.4,1.0}
\definecolor{shadecolor}{rgb}{0.79,0.78,0.65}
\definecolor{gray}{rgb}{0.6,0.6,0.6}

\usepackage{colortbl}

\definecolor{myred}{rgb}{0.7,0.0,0.0}
\definecolor{mygreen}{rgb}{0.0,0.7,0.0}
\definecolor{myblue}{rgb}{0.0,0.0,0.7}

\definecolor{redshade}{rgb}{0.9,0.5,0.5}
\definecolor{greenshade}{rgb}{0.5,0.9,0.5}

\usepackage[unicode,colorlinks=true,linkcolor=mylinkcolor,citecolor=mycitecolor]{hyperref}
\newcommand{\refref}[2]{\hyperref[#2]{#1~\ref*{#2}}}
\newcommand{\eqnref}[1]{\hyperref[#1]{(\ref*{#1})}}

\newcommand{\tos}{\!\!\to\!\!}

\usepackage{framed}

\newcommand{\cequiv}{\simeq}

\makeatletter
\newcommand\xleftrightarrow[2][]{%
  \ext@arrow 9999{\longleftrightarrowfill@}{#1}{#2}}
\newcommand\longleftrightarrowfill@{%
  \arrowfill@\leftarrow\relbar\rightarrow}
\makeatother

\newcommand{\bsquare}{\textrm{\ding{114}}}

% % % % % % % % % % % % % % % % % % % % % % % % % % % % % %

\newtheoremstyle{myplain}
  {\topsep}   % ABOVESPACE
  {\topsep}   % BELOWSPACE
  {\itshape}  % BODYFONT
  {0pt}       % INDENT (empty value is the same as 0pt)
  {\bfseries} % HEADFONT
  {.}         % HEADPUNCT
  {5pt plus 1pt minus 1pt} % HEADSPACE
  {\thmnumber{#2}. \thmname{#1}\thmnote{ (#3)}}   % CUSTOM-HEAD-SPEC

\newtheoremstyle{myplainnameless}
  {\topsep}   % ABOVESPACE
  {\topsep}   % BELOWSPACE
  {\normalfont}  % BODYFONT
  {0pt}       % INDENT (empty value is the same as 0pt)
  {\bfseries} % HEADFONT
  {.}         % HEADPUNCT
  {5pt plus 1pt minus 1pt} % HEADSPACE
  {\thmnumber{#2}}   % CUSTOM-HEAD-SPEC 

\newtheoremstyle{sectionexercise}
  {\topsep}   % ABOVESPACE
  {\topsep}   % BELOWSPACE
  {\normalfont}  % BODYFONT
  {0pt}       % INDENT (empty value is the same as 0pt)
  {\bfseries} % HEADFONT
  {.}         % HEADPUNCT
  {5pt plus 1pt minus 1pt} % HEADSPACE
  {Ejercicio \thmnumber{#2}\thmnote{ (#3)}}   % CUSTOM-HEAD-SPEC

\newtheoremstyle{mydefinition}
  {\topsep}   % ABOVESPACE
  {\topsep}   % BELOWSPACE
  {\normalfont}  % BODYFONT
  {0pt}       % INDENT (empty value is the same as 0pt)
  {\bfseries} % HEADFONT
  {.}         % HEADPUNCT
  {5pt plus 1pt minus 1pt} % HEADSPACE
  {\thmnumber{#2}. \thmname{#1}\thmnote{ (#3)}}   % CUSTOM-HEAD-SPEC

% EN ESPAÑOL

\newtheorem*{hecho*}{Hecho}
\newtheorem*{corolario*}{Corolario}
\newtheorem*{teorema*}{Teorema}
\newtheorem*{conjetura*}{Conjetura}
\newtheorem*{proyecto*}{Proyecto}
\newtheorem*{observacion*}{Observación}

\newtheorem*{lema*}{Lema}
\newtheorem*{resultado-clave*}{Resultado clave}
\newtheorem*{proposicion*}{Proposición}

\theoremstyle{definition}
\newtheorem*{ejercicio*}{Ejercicio}
\newtheorem*{definicion*}{Definición}
\newtheorem*{comentario*}{Comentario}
\newtheorem*{definicion-alternativa*}{Definición alternativa}
\newtheorem*{ejemploxs}{Ejemplo}
\newenvironment{ejemplo*}
  {\pushQED{\qed}\renewcommand{\qedsymbol}{\examplesymbol}\ejemploxs}
  {\popQED\endejemploxs}

\theoremstyle{myplain}
\newtheorem{proposicion}{Proposición}[section]

\newtheorem{proyecto}[proposicion]{Proyecto}
\newtheorem{teorema}[proposicion]{Teorema}
\newtheorem{corolario}[proposicion]{Corolario}
\newtheorem{hecho}[proposicion]{Hecho}
\newtheorem{lema}[proposicion]{Lema}

\newtheorem{observacion}[proposicion]{Observación}

\newenvironment{observacionejerc}
    {\pushQED{\qed}\renewcommand{\qedsymbol}{$\square$}\csname inner@observacionejerc\endcsname}
    {\popQED\csname endinner@observacionejerc\endcsname}
\newtheorem{inner@observacionejerc}[proposicion]{Observación}

\newenvironment{proposicionejerc}
    {\pushQED{\qed}\renewcommand{\qedsymbol}{$\square$}\csname inner@proposicionejerc\endcsname}
    {\popQED\csname endinner@proposicionejerc\endcsname}
\newtheorem{inner@proposicionejerc}[proposicion]{Proposicion}

\newenvironment{lemaejerc}
    {\pushQED{\qed}\renewcommand{\qedsymbol}{$\square$}\csname inner@lemaejerc\endcsname}
    {\popQED\csname endinner@lemaejerc\endcsname}
\newtheorem{inner@lemaejerc}[proposicion]{Lema}

\newtheorem{calculo}[proposicion]{Cálculo}

\theoremstyle{myplainnameless}
\newtheorem{nameless}[proposicion]{}

\theoremstyle{mydefinition}
\newtheorem{comentario}[proposicion]{Comentario}
\newtheorem{comentarioast}[proposicion]{Comentario ($\clubsuit$)}
\newtheorem{construccion}[proposicion]{Construcción}
\newtheorem{aplicacion}[proposicion]{Aplicación}
\newtheorem{definicion}[proposicion]{Definición}
\newtheorem{definicion-alternativa}[proposicion]{Definición alternativa}
\newtheorem{notacion}[proposicion]{Notación}
\newtheorem{advertencia}[proposicion]{Advertencia}
\newtheorem{digresion}[proposicion]{Digresión}
\newtheorem{ejemplox}[proposicion]{Ejemplo}
\newenvironment{ejemplo}
  {\pushQED{\qed}\renewcommand{\qedsymbol}{\examplesymbol}\ejemplox}
  {\popQED\endejemplox}
\newtheorem{contraejemplox}[proposicion]{Contraejemplo}
\newenvironment{contraejemplo}
  {\pushQED{\qed}\renewcommand{\qedsymbol}{\examplesymbol}\contraejemplox}
  {\popQED\endcontraejemplox}
\newtheorem{noejemplox}[proposicion]{No-ejemplo}
\newenvironment{noejemplo}
  {\pushQED{\qed}\renewcommand{\qedsymbol}{\examplesymbol}\noejemplox}
  {\popQED\endnoejemplox}
 
\newtheorem{ejemploastx}[proposicion]{Ejemplo ($\clubsuit$)}
\newenvironment{ejemploast}
  {\pushQED{\qed}\renewcommand{\qedsymbol}{\examplesymbol}\ejemploastx}
  {\popQED\endejemploastx}

\ifdefined\exercisespersection
  \theoremstyle{sectionexercise}
  \newtheorem{ejercicio}{}[section]
  \theoremstyle{mydefinition}
\else
  \ifdefined\exercisesglobal
    \theoremstyle{sectionexercise}
    \newtheorem{ejercicio}{}
    \theoremstyle{mydefinition}
  \else
    \ifdefined\exercisespersection
      \newtheorem{ejercicio}[proposicion]{Ejercicio}
    \fi
  \fi
\fi

% % % % % % % % % % % % % % % % % % % % % % % % % % % % % %

\theoremstyle{myplain}
\newtheorem{proposition}{Proposition}[section]
\newtheorem*{fact*}{Fact}
\newtheorem*{proposition*}{Proposition}
\newtheorem{lemma}[proposition]{Lemma}
\newtheorem*{lemma*}{Lemma}

\newtheorem{exercise}{Exercise}
\newtheorem*{hint}{Hint}

\newtheorem{theorem}[proposition]{Theorem}
\newtheorem{conjecture}[proposition]{Conjecture}
\newtheorem*{theorem*}{Theorem}
\newtheorem{corollary}[proposition]{Corollary}
\newtheorem{fact}[proposition]{Fact}
\newtheorem*{claim}{Claim}
\newtheorem{definition-theorem}[proposition]{Definition-theorem}

\theoremstyle{mydefinition}
\newtheorem{examplex}[proposition]{Example}
\newenvironment{example}
  {\pushQED{\qed}\renewcommand{\qedsymbol}{\examplesymbol}\examplex}
  {\popQED\endexamplex}

\newtheorem*{examplexx}{Example}
\newenvironment{example*}
  {\pushQED{\qed}\renewcommand{\qedsymbol}{\examplesymbol}\examplexx}
  {\popQED\endexamplexx}

\newtheorem{definition}[proposition]{Definition}
\newtheorem*{definition*}{Definition}
\newtheorem{wrong-definition}[proposition]{Wrong definition}
\newtheorem{remark}[proposition]{Remark}

\makeatletter
\newcommand{\xRightarrow}[2][]{\ext@arrow 0359\Rightarrowfill@{#1}{#2}}
\makeatother

% % % % % % % % % % % % % % % % % % % % % % % % % % % % % %

\newcommand{\Et}{\mathop{\text{\rm Ét}}}

\newcommand{\categ}[1]{\text{\bf #1}}
\newcommand{\vcateg}{\mathcal}
\newcommand{\bone}{{\boldsymbol 1}}
\newcommand{\bDelta}{{\boldsymbol\Delta}}
\newcommand{\bR}{{\mathbf{R}}}

\newcommand{\univ}{\mathfrak}

\newcommand{\TODO}{\colorbox{red}{\textbf{*** TODO ***}}}
\newcommand{\proofreadme}{\colorbox{red}{\textbf{*** NEEDS PROOFREADING ***}}}

\makeatletter
\def\iddots{\mathinner{\mkern1mu\raise\p@
\vbox{\kern7\p@\hbox{.}}\mkern2mu
\raise4\p@\hbox{.}\mkern2mu\raise7\p@\hbox{.}\mkern1mu}}
\makeatother

\newcommand{\ssincl}{\reflectbox{\rotatebox[origin=c]{45}{$\subseteq$}}}
\newcommand{\vsupseteq}{\reflectbox{\rotatebox[origin=c]{-90}{$\supseteq$}}}
\newcommand{\vin}{\reflectbox{\rotatebox[origin=c]{90}{$\in$}}}

\newcommand{\Ga}{\mathbb{G}_\mathrm{a}}
\newcommand{\Gm}{\mathbb{G}_\mathrm{m}}

\renewcommand{\U}{\UUU}

\DeclareRobustCommand{\Stirling}{\genfrac\{\}{0pt}{}}
\DeclareRobustCommand{\stirling}{\genfrac[]{0pt}{}}

% % % % % % % % % % % % % % % % % % % % % % % % % % % % % %
% tikz

\usepackage{tikz-cd}
\usetikzlibrary{babel}
\usetikzlibrary{decorations.pathmorphing}
\usetikzlibrary{arrows}
\usetikzlibrary{calc}
\usetikzlibrary{fit}
\usetikzlibrary{hobby}

% % % % % % % % % % % % % % % % % % % % % % % % % % % % % %
% Banners

\newcommand\mybannerext[3]{{\normalfont\sffamily\bfseries\large\noindent #1

\noindent #2

\noindent #3

}\noindent\rule{\textwidth}{1.25pt}

\vspace{1em}}

\newcommand\mybanner[2]{{\normalfont\sffamily\bfseries\large\noindent #1

\noindent #2

}\noindent\rule{\textwidth}{1.25pt}

\vspace{1em}}

\renewcommand{\O}{\mathcal{O}}


\usepackage{fullpage}

\theoremstyle{definition}
\newtheorem{ejerc}{Ejercicio}
\newenvironment{solucion}{\begin{proof}[Solución]}{\end{proof}}

\begin{document}

\mybanner{Álgebra computacional. Tarea 4. Fecha límite: 21/05/2019}{Universidad de El Salvador, ciclo impar 2019}

Se puede usar Macaulay2 para \emph{comprobar} algunos cálculos, pero hay que
justificar todas las respuestas.

\section*{Descomposiciones primarias}

\begin{ejerc}
  Demuestre que en el anillo $A = \ZZ [x]$ el ideal $\mathfrak{m} \dfn (2,x)$ es
  maximal y el ideal $\mathfrak{q} = (4,x)$ es $\mathfrak{m}$-primario, pero no
  es una potencia de $\mathfrak{m}$.

  \ifdefined\solutions\begin{solucion}
    El ideal $\mathfrak{m} = (2,x)$ es maximal, puesto que
    $\ZZ [x]/(2,x) \isom \ZZ/2\ZZ$ es un cuerpo. Calculemos
    \[ \sqrt{\mathfrak{q}} =
       \sqrt{\sqrt{4} + \sqrt{x}} =
       \sqrt{(2,x)} =
       \mathfrak{m}. \]
    Recordemos que cuando el radical $\sqrt{\mathfrak{q}}$ es maximal, el ideal
    $\mathfrak{q}$ es necesariamente primario. (Sin embargo, no olvidemos que en
    general, para que $\mathfrak{q}$ sea primario, es necesario pero
    \emph{no es suficiente} que $\mathfrak{p} = \sqrt{\mathfrak{q}}$ sea primo.)

    Las potencias de $\mathfrak{m}$ forman una cadena descendente
    \[ \mathfrak{m} \supset \mathfrak{m}^2 \supset \mathfrak{m}^3 \supset
       \mathfrak{m}^4 \supset \cdots \]
    En particular, $\mathfrak{m}^2 = (4, 2x, x^2)$. Además,
    $$\mathfrak{m}^2 \subsetneq \mathfrak{q} \subsetneq \mathfrak{m}.$$
    Aquí las inclusiones son esctrictas: por ejemplo, $2 \in \mathfrak{m}$, pero
    $2 \notin \mathfrak{q}$. De la misma manera, $x \in \mathfrak{q}$, pero
    $x \notin \mathfrak{m}^2$. Esto es suficiente para concluir que
    $\mathfrak{q}$ no es una potencia de $\mathfrak{m}$.
  \end{solucion}\fi
\end{ejerc}

\begin{ejerc}
  Para dos ideales $I$ e $J = (f_1, \ldots, f_s)$, demuestre que
  $$(I : J) = \bigcap_{1 \le i \le s} (I : f_i),$$
  y para $f \ne 0$ se tiene
  $$(I : f) = \Bigl\{ \frac{g}{f} \Bigm| g \in I \cap (f) \Bigr\}.$$

  \ifdefined\solutions\begin{solucion}
    Recordemos la definición:
    $$(I : J) \dfn \{ g \in A \mid gJ \subseteq I \}.$$
    Ahora si $g \in (I : f_i)$ para todo $i$, esto quiere decir que $gf_i \in I$
    para todo $i$, y luego $gJ \subseteq I$. Esto establece la inclusión
    $$\bigcap_{1 \le i \le s} (I : f_i) \subseteq (I : J).$$
    Viceversa, tenemos $(f_i) \subseteq J$ para todo $i$, y luego
    $(I : J) \subseteq (I : f_i)$.

    Ahora para $f \ne 0$ se tiene
    \[ (I : f) = \{ g \in A \mid gf \in I \} =
       \Bigl\{ \frac{gf}{f} \Bigm| g \in A, \, gf \in I \cap (f) \Bigr\}. \]
    De aquí se ve la inclusión
    $$(I : f) \subseteq \Bigl\{ \frac{g}{f} \Bigm| g \in I \cap (f) \Bigr\}.$$
    Viceversa, si $g \in I \cap (f)$, entonces $g = f\cdot \frac{g}{f} \in I$,
    lo que nos da la otra inclusión.
  \end{solucion}\fi
\end{ejerc}

\begin{ejerc}
  Calcule el ideal cociente $(I : J)$ para $I = (xy^2, x^2z)$ y $J = (xy,z)$ en
  el anillo $k [x,y,z]$.

  \ifdefined\solutions\begin{solucion}
    Según el ejercicio anterior, tenemos
    $$(I : J) = (I : xy) \cap (I : z).$$
    Luego,
    \[ (I : xy) =
       \Bigl\{ \frac{g}{xy} \Bigm| g \in I \cap (xy) \Bigr\} =
       \Bigl\{ \frac{g}{xy} \Bigm| g \in (xy^2, \, x^2\,yz) \Bigr\} =
       (y, \, xz) \]
    y de manera similar,
    \[ (I : z) =
       \Bigl\{ \frac{g}{z} \Bigm| g \in I \cap (z) \Bigr\} =
       \Bigl\{ \frac{g}{z} \Bigm| g \in (xy^2z, \, x^2z) \Bigr\} =
       (xy^2, \, x^2). \]
    Entonces,
    \[ (I : J) = (y, \, xz) \cap (xy^2, \, x^2) = (x^2y, xy^2, x^2z). \qedhere \]
  \end{solucion}\fi
\end{ejerc}

\begin{ejerc}
  Demuestre que si $I$ es un ideal radical, entonces $I$ no tiene ideales
  asociados encajados.

  \ifdefined\solutions\begin{solucion}
    Sea
    $$I = \mathfrak{q}_1 \cap \cdots \cap \mathfrak{q}_s$$
    una descomposición primaria minimal para $I$. El primer teorema de unicidad
    implica que el número $s$ está definido de modo único. Ahora
    \[ I = \sqrt{I} =
       \sqrt{\mathfrak{q}_1} \cap \cdots \cap \sqrt{\mathfrak{q}_s} =
       \mathfrak{p}_1 \cap \cdots \cap \mathfrak{p}_s \]
    es también una descomposición primaria de $I$. Si
    $\mathfrak{p}_i \subseteq \mathfrak{p}_j$ para algunos $i\ne j$, entonces la
    descomposición de arriba nos da una descomposición primaria con $< s$
    ideales en la intersección, lo que contradice la minimalidad de $s$.
  \end{solucion}\fi
\end{ejerc}

\begin{ejerc}
  Consideremos el ideal monomial
  $I = (x^2, \, xy^2z, \, yz^2) \subset k [x,y,z]$.

  \begin{enumerate}
  \item[1)] Encuentre una descomposición primaria minimal para $I$.

  \item[2)] Encuentre los primos asociados.

  \item[3)] Exprese cada primo asociado como $\mathfrak{p} = (I : f)$ para algún
    polinomio $f \in k [x,y,z]$.

  \item[4)] Encuentre los primos minimales y encajados correspondientes.

  \item[5)] ¿Cómo se ve el conjunto algebraico
    $\mathbf{V} (I) \subset \AA^3 (k)$?
  \end{enumerate}

  \ifdefined\solutions\begin{solucion}
    Primero escribamos
    $$(x^2, \, xy^2z, \, yz^2) = (x^2, y) \cap (x^2, \, xy^2z, \, z^2).$$
    Luego,
    \[ (x^2, \, xy^2z, \, z^2) =
       (x^2, \, y^2, \, z^2) \cap (x^2, \, xz, \, z^2). \]
    Esto nos da una descomposición
    $$I = (x^2, y) \cap (x^2, \, y^2, \, z^2) \cap (x^2, \, xz, \, z^2).$$
    Es fácil verificar que los primeros dos ideales $(x^2,y)$ e
    $(x^2, \, y^2, \, z^2)$ son primarios, porque estos están generados por
    potencias de las variables. Notamos también que
    $\sqrt{(x^2, \, y^2, \, z^2)} = (x,y,z)$ es maximal, lo que también implica
    que $(x^2, \, y^2, \, z^2)$ es primario. El tercer ideal
    $(x^2, \, xz, \, z^2)$ es el cuadrado de $(x, z)^2$, así que es también
    primario por el ejercicio~7. Sin recurrir a este ejercicio, podemos escribir
    $$(x^2, \, xz, \, z^2) = (x, z^2) \cap (x^2, z),$$
    donde $(x,z^2)$ y $(x^2,z)$ son $(x,z)$-primarios, así que su intersección
    es también un ideal $(x,z)$-primario.

    Pongamos
    \[ \mathfrak{q}_1 \dfn (x^2, y), \quad
       \mathfrak{q}_2 \dfn (x^2, \, y^2, \, z^2), \quad
       \mathfrak{q}_3 \dfn (x^2, \, xz, \, z^2). \]
    Los radicales correspondientes son
    \[ \sqrt{\mathfrak{q}_1} = (x,y), \quad
       \sqrt{\mathfrak{q}_2} = (x,y,z), \quad
       \sqrt{\mathfrak{q}_3} = \sqrt{(x,z)^2} = (x,z). \]
    En particular, estos son ideales primos diferentes. Además, calculamos que
    \begin{align*}
      \mathfrak{q}_1 \cap \mathfrak{q}_2 & = (x^2, \, y^2, \, yz^2)
                                           \not\subseteq \mathfrak{q}_3,\\
      \mathfrak{q}_1 \cap \mathfrak{q}_3 & = (x^2, \, xyz, \, yz^2)
                                           \not\subseteq \mathfrak{q}_2,\\
      \mathfrak{q}_2 \cap \mathfrak{q}_3 & = (x^2, \, xy^2z, \, z^2)
                                           \not\subseteq \mathfrak{q}_1.
    \end{align*}
    Podemos concluir que la descomposición primaria que hemos encontrado es
    minimal.

    Los primos asociados son
    \[ \mathfrak{p}_1 = (x,y), \quad
       \mathfrak{p}_2 = (x,y,z), \quad
       \mathfrak{p}_3 = (x,z). \]
    Los primos $\mathfrak{p}_1$ y $\mathfrak{p}_3$ son minimales, mientras que
    $\mathfrak{p}_2$ es encajado.

    Calculamos que
    \[ \mathfrak{p}_1 = (I : xz^2), \quad
       \mathfrak{p}_2 = (I : xyz), \quad
       \mathfrak{p}_3 = (I : xy^2). \]

    En fin, notamos que
    \[ \mathbf{V} (I) = \mathbf{V} (\sqrt{I}) =
       \mathbf{V} (x, \, yz) = \mathbf{V} (x,y) \cap \mathbf{V} (x,z). \]
    Esta es la unión del eje $z$ con el eje $y$.
  \end{solucion}\fi
\end{ejerc}

\begin{ejerc}
  Para un anillo $A$ y un ideal $I \subseteq A$, denotemos por
  $I [x] \subseteq A [x]$ el ideal formado por los polinomios con coeficientes
  en $I$. Demuestre las siguientes propiedades.

  \begin{enumerate}
  \item[1)] Si $\mathfrak{p}$ es un ideal primo en $A$, entonces
    $\mathfrak{p}[x]$ es un ideal primo en $A [x]$.
  \item[2)] Si $\mathfrak{q}$ es un ideal $\mathfrak{p}$-primario en $A$,
    entonces $\mathfrak{q}[x]$ es un ideal $\mathfrak{p}[x]$-primario en
    $A [x]$.
  \item[3)] Si $I = \mathfrak{q}_1\cap\cdots\cap\mathfrak{q}_s$ es una
    descomposición primaria minimal en $A$, entonces
    $I [x] = \mathfrak{q}_1 [x]\cap\cdots\cap\mathfrak{q}_s [x]$ es una
    descomposición primaria minimal en $A [x]$.
  \item[4)] Si $\mathfrak{p}$ es un ideal primo minimal asociado a $I$, entonces
    $\mathfrak{p} [x]$ es un ideal primo minimal asociado a $I [x]$.
  \end{enumerate}

  \ifdefined\solutions\begin{solucion}
    Primero, tenemos
    $$A[x] / I [x] \isom (A/I) [x].$$
    Ahora si $\mathfrak{p} \subset A$ es primo, entonces $A/\mathfrak{p}$ es un
    dominio, y luego $(A/\mathfrak{p}) [x]$ es un dominio, así que el ideal
    $\mathfrak{p} [x] \subset A [x]$ es primo. Esto establece la parte 1).

    También sería útil observar que $J_1 [x] = J_2 [x]$ si y solamente si
    $J_1 = J_2$, y que $(J_1 \cap J_2) [x] = J_1 [x] \cap J_2 [x]$.

    Ahora si $\mathfrak{q} \subset A$ es un ideal $\mathfrak{p}$-primario,
    entonces
    \[ \sqrt{\mathfrak{q}[x]} =
       \sqrt{\mathfrak{q} + \mathfrak{q}x} =
       \sqrt{\sqrt{\mathfrak{q}} + \sqrt{\mathfrak{q}} \cap \sqrt{x}} =
       \sqrt{\mathfrak{p} + \mathfrak{p}\cap (x)} =
       \sqrt{\mathfrak{p} + \mathfrak{p}x} =
       \sqrt{\mathfrak{p}[x]} = \mathfrak{p}[x]. \]

    Antes de probar que $\mathfrak{q}[x]$ es también primario, recordemos cómo
    se ven los divisores de cero y nilpotentes en el anillo de polinomios.

    \begin{itemize}
    \item \emph{Un polinomio $f = a_n\,x^n + \cdots + a_1\,x + a_0 \in A [x]$ es
        un divisor de cero si y solamente si $cf = 0$ para alguna constante no
        nula $c \in A$.}

      Una de las imlplicaciones es trivial. Para la otra dirección, asumamos que
      $f \in A[x]$ es un divisor de cero y procedamos por inducción sobre
      $\deg f$. Vamos a probar la propiedad más fuerte: si
      $$g = b_m\,x^m + \cdots + b_1\,x + b_0$$
      es un polinomio del mínimo grado posible tal que $fg = 0$, entonces
      $b_m f = 0$. Para la base de inducción, si $\deg f = 0$, entonces $f = a$
      es constante. Luego, $fg = 0$ implica en particular que $b_m\cdot a = 0$.

      Para el paso inductivo cuando $\deg f > 0$, notamos que
      $$fg = a_n\,b_m\,x^{m+n} + \cdots = 0,$$
      y en particular $a_n b_m = 0$, así que $\deg (a_n g) < m$. Luego,
      $a_n g f = 0$, y por la minimalidad de $\deg g$, tenemos $a_n g =
      0$. Podemos entonces escribir
      $$(f - a_n\,x^n)\,g = gf - a_n\,x^n\,g = 0,$$
      donde $\deg (f - a_n\,x^n) < \deg f$. Notamos que $g$ es también un
      polinomio de grado mínimo posible tal que $(f - a_n\,x^n)\cdot g = 0$, así
      que por la hipótesis inductiva, $b_m\cdot (f - a_n\,x^n) = 0$. Luego,
      $$b_m\,f = b_m\,a_n\,x^n = 0.$$

    \item \emph{Los nilpotentes en el anillo de polinomios vienen dados por
        \[ N (A [x]) =
           \{ a_n\,x^n + \cdots + a_1\,x + a_0 \mid a_0, a_1, \ldots, a_n \in N (A) \} =
           N (A) [x]. \]}

      En efecto, dado que $N (A [x])$ es un ideal, si
      $a_i \in N (A) \subset N (A [x])$ para todo $i$, entonces
      ${a_n\,x^n + \cdots + a_0 \in N (A[x])}$. Esto establece la inclusión
      $$N (A) [x] \subseteq N (A[x]).$$
      Viceversa,
      \[ N (A [x]) =
         \bigcap_{\mathfrak{P} \in \Spec A [x]} \mathfrak{P} \subseteq \bigcap_{\mathfrak{p} \in \Spec A} \mathfrak{p} [x] =
         \left(\bigcap_{\mathfrak{p} \in \Spec A} \mathfrak{p}\right) [x] =
         N (A) [x]. \]
      Aquí hemos usado de nuevo que si $\mathfrak{p} \subset A$ es primo,
      entonces $\mathfrak{p} [x] \subset A[x]$ es también primo.
    \end{itemize}

    Ahora bien, si $\mathfrak{q}$ es primario, entonces
    $$A[x] / \mathfrak{q} [x] \isom (A/\mathfrak{q})[x],$$
    donde los divisores de cero en $A/\mathfrak{q}$ son nilpotentes. Un divisor
    de cero en $(A/\mathfrak{q}) [x]$ sería un polinomio
    $$f = a_n\,x^n + \cdots + a_1\,x + a_0 \in (A/\mathfrak{q}) [x]$$
    tal que $c f = 0$ para algún $c \in A/\mathfrak{q}$ no nulo. En particular,
    $$c a_n = c a_1 = \cdots = c a_0 = 0.$$
    Esto significa que $a_n, \ldots, a_0$ son divisores de cero en
    $A/\mathfrak{q}$, y luego son nilpotentes, lo que implica que $f$ es
    nilpotente en $(A/\mathfrak{q}) [x]$. Esto demuestra la parte 2).

    \vspace{1em}

    Ahora para una descomposición primaria minimal
    $$I = \mathfrak{q}_1 \cap \cdots \cap \mathfrak{q}_s$$
    tenemos
    $$I [x] = \mathfrak{q}_1 [x] \cap \cdots \cap \mathfrak{q}_s [x],$$
    que es una descomposición primaria en $A [x]$ según lo que acabamos de
    probar. Hay que verificar su minimalidad. Primero, tenemos
    $$\sqrt{\mathfrak{q}_i [x]} = \sqrt{\mathfrak{q}_i} [x],$$
    así que $\sqrt{\mathfrak{q}_i [x]} = \sqrt{\mathfrak{q}_j [x]}$ implicaría
    que $\sqrt{\mathfrak{q}_i} = \sqrt{\mathfrak{q}_j}$, lo que no es el
    caso. De la misma manera, de
    \[ \bigcap_{j\ne i} \mathfrak{q}_j [x] =
       \left(\bigcap_{j\ne i} \mathfrak{q}_j\right) [x], \]
    se ve la otra condición de minimalidad.

    Ahora sea $\mathfrak{p} \in \Spec A$ un primo minimal asociado a $I$. Esto
    significa que $\mathfrak{p}$ es minimal entre los primos tales que
    $$I \subseteq \mathfrak{p} \subset A.$$
    Luego, tenemos
    $$I [x] \subseteq \mathfrak{p} [x] \subset A [x].$$
    Si $\mathfrak{P} \subset A [x]$ es otro ideal primo tal que
    $$I [x] \subseteq \mathfrak{P} \subseteq \mathfrak{p} [x] \subset A [x],$$
    entonces tenemos
    $$I \subseteq \mathfrak{P} \cap A \subseteq \mathfrak{p} \subset A,$$
    y por la minimalidad de $\mathfrak{p}$ se cumple
    $\mathfrak{P} \cap A = \mathfrak{p}$. En particular,
    $\mathfrak{p} \subseteq \mathfrak{P}$, y luego
    $\mathfrak{p} [x] = \mathfrak{p} + \mathfrak{p} [x] \subseteq
    \mathfrak{P}$. Podemos concluir que $\mathfrak{P} = \mathfrak{p} [x]$.
  \end{solucion}\fi
\end{ejerc}

\begin{ejerc}
  Demuestre que en el anillo de polinomios $k [x_1,\ldots,x_n]$ para los ideales
  primos $\mathfrak{p}_i \dfn (x_1,\ldots,x_i)$ (donde $i = 1,\ldots,n$) todas
  las potencias $\mathfrak{p}_i^m$ son ideales primarios.

  \ifdefined\solutions\begin{solucion}
    Podemos escribir
    \[ \mathfrak{p}_i^m =
       ((\mathfrak{p}_i^m \cap k [x_1,\ldots,x_i]) [x_{i+1}] \cdots) [x_n]. \]
    Luego, gracias al ejercicio anterior, bastaría verificar que
    $(x_1,\ldots,x_i)^m$ es un ideal primario en el anillo $k
    [x_1,\ldots,x_i]$. Tenemos
    $$\sqrt{(x_1,\ldots,x_i)^m} = (x_1,\ldots,x_i),$$
    y este ideal es maximal en $k [x_1,\ldots,x_i]$, lo que implica que
    $(x_1,\ldots,x_i)^m$ es primario en $k [x_1,\ldots,x_i]$.
  \end{solucion}\fi
\end{ejerc}

\pagebreak

\section*{Dimensión}

\begin{ejerc}
  Demuestre que si $X\ne \emptyset$ es un espacio noetheriano y $Z_1,\ldots,Z_s$
  son sus componentes irreducibles, entonces
  $$\dim X = \max \{ \dim Z_1, \ldots, \dim Z_s \}.$$

  \ifdefined\solutions\begin{solucion}
    Dada una cadena de subconjuntos cerrados irreducibles
    $$X_0 \subsetneq X_1 \subsetneq X_2 \subsetneq \cdots \subsetneq X_n \subset X,$$
    tenemos necesariamente $X_n \subseteq Z_i$ para algún $i = 1,\ldots,s$.
  \end{solucion}\fi
\end{ejerc}

\begin{ejerc}
  Demuestre que para un subespacio $Y\subseteq X$ se tiene
  $$\dim Y \le \dim X.$$

  \ifdefined\solutions\begin{solucion}
    Primero recordemos que un subespacio $Z \subseteq X$ es irreducible si y
    solo si su cerradura $\overline{Z} \subseteq X$ es irreducible (véanse los
    apuntes de clase). Ahora si
    \[ Z_0 \subsetneq Z_1 \subsetneq Z_2 \subsetneq \cdots \subsetneq
       Z_n \subseteq Y \]
    es una cadena de subconjuntos cerrados irreducibles en $Y$, entonces tenemos
    $Z_i = Z_i \cap \overline{Z_i}$, así que al pasar a las cerraduras en $X$,
    se obtiene una cadena
    \[ \overline{Z_0} \subsetneq \overline{Z_1} \subsetneq
       \overline{Z_2} \subsetneq \cdots \subsetneq \overline{Z_n} \subseteq X, \]
    donde $\overline{Z_i}$ son irreducibles por el resultado mencionado al
    inicio.
  \end{solucion}\fi
\end{ejerc}

\begin{ejerc}
  Sean $A$ un anillo e $I \subseteq A$ un ideal. Demuestre que
  $\dim (A/I) \le \dim A$.

  \ifdefined\solutions\begin{solucion}
    Una cadena de ideales primos
    \[ \overline{\mathfrak{p}_0} \subsetneq \overline{\mathfrak{p}_1} \subsetneq
       \overline{\mathfrak{p}_2} \subsetneq \cdots \subsetneq
       \overline{\mathfrak{p}_n} \subset A/I \]
    corresponde a una cadena de ideales primos
    \[ I \subseteq \mathfrak{p}_0 \subsetneq \mathfrak{p}_1 \subsetneq
       \mathfrak{p}_2 \subsetneq \cdots \subsetneq \mathfrak{p}_n \subset A. \]
    Podemos también formularlo de otra manera: el homomorfismo $A \epi A/I$
    induce una aplicación continua inyectiva
    $\Spec A/I \hookrightarrow \Spec A$, y se puede usar el ejercicio anterior.
  \end{solucion}\fi
\end{ejerc}

\begin{ejerc}
  Demuestre que para el producto de dos anillos se tiene
  $$\dim (A\times B) = \max \{ \dim A, \dim B \}.$$

  \ifdefined\solutions\begin{solucion}
    Tenemos
    \begin{equation}
      \tag{*} \Spec (A\times B) =
      \{ \mathfrak{p}\times B \mid \mathfrak{p} \in \Spec A \} \cup
      \{ A\times \mathfrak{q} \mid \mathfrak{q} \in \Spec B \}.
    \end{equation}
    Ahora tenemos $\mathfrak{p}\times B \not\subseteq A\times \mathfrak{q}$ y
    $A\times \mathfrak{q} \not\subseteq \mathfrak{p}\times B$, así que toda
    cadena de ideales primos en $A\times B$ viene de una cadena en $A$ o una
    cadena en $B$.

    \vspace{1em}

    Sería instructivo revisar una prueba directa de (*). El producto $A\times B$
    viene con los homomorfismos canónicos
    $$\pi_A\colon A\times B \to A, \quad \pi_B\colon A\times B \to B.$$
    Ahora si
    $$\mathfrak{p} \in \Spec A, ~ \mathfrak{q} \in \Spec B,$$
    entonces
    \[ \pi_A^{-1} (\mathfrak{p}) = \mathfrak{p} \times B, ~
       \pi_A^{-1} (\mathfrak{q}) = A\times \mathfrak{q} \in \Spec (A\times B). \]

    Viceversa, asumamos que $\mathfrak{P} \subset A\times B$ es un ideal
    primo. Luego, para $e_1 \dfn (1,0)$ y $e_2 \dfn (0,1)$ se tiene
    $e_1\,e_2 \in \mathfrak{P}$, así que por la primalidad de $\mathfrak{P}$
    tenemos dos posibilidades:

    \begin{itemize}
    \item $e_1 \in \mathfrak{P}$, y luego
      $\mathfrak{P} = A\times \pi_B (\mathfrak{P})$, donde
      $\pi_B (\mathfrak{P}) \in \Spec B$;

    \item $e_2 \in \mathfrak{P}$, y luego
      $\mathfrak{P} = \pi_A (\mathfrak{P})\times B$, donde
      $\pi_A (\mathfrak{P}) \in \Spec A$. \qedhere
    \end{itemize}
  \end{solucion}\fi
\end{ejerc}

\begin{ejerc}
  Sea $k$ un cuerpo. Demuestre que el anillo de las series formales
  $k [\![x]\!]$ y el anillo de polinomios de Laurent $k [x,x^{-1}]$ tienen
  dimensión $1$.

  \ifdefined\solutions\begin{solucion}
    En el caso de $k [\![x]\!]$, tenemos simplemente
    $$\Spec k [\![x]\!] = \{ (0), (x) \},$$
    de donde es evidente que la dimensión es $1$.

    A saber, es fácil entender cuáles son los ideales en $k
    [\![x]\!]$. Recordemos que una serie $f = \sum_{i\ge 0} a_i\,x^i$ es
    invertible en $k [\![x]\!]$ si y solo si $a_0 \ne 0$.
    Para una serie no nula $f = \sum_{i\ge 0} a_i\,x^i$, pongamos
    $$v (f) \dfn \min \{ i \mid a_i \ne 0 \}.$$
    Este es el ``orden de anulación en $x=0$'' de la serie. Notamos que
    $$v (f) = n \iff f = u\,x^n, \quad u \in k [\![x]\!]^\times.$$

    Para un ideal no nulo $I \subseteq k [\![x]\!]$, sea $f \in I$ una serie no
    nula con el mínimo posible valor de $v (f)$. Consideremos otra serie
    $g \in I$. Escribamos
    $$f = u\,x^{v (f)}, \quad g = w\,x^{v (g)}.$$
    Luego,
    $$g = w\,u^{-1}\,x^{v (g) - v (f)} \cdot (u\,x^{v (f)}) \in (f).$$
    Entonces,
    $$I = (f) = (x^{v (f)}).$$
    Esto demuestra que los ideales no nulos en $k [\![x]\!]$ son $(x)^n = (x^n)$
    para $n = 1,2,3,\ldots$ El único ideal primo entre ellos es $(x)$.

    \vspace{1em}

    En el caso de
    $$k [x,x^{-1}] \isom k [x,y] / (xy-1),$$
    podemos ocupar la caracterización de la dimensión de $k$-álgebras
    finitamente generadas: el ideal $(xy-1) \subset k [x,y]$ no es nulo y
    $$(xy-1) \cap k[x] = (xy-1) \cap k [y] = 0,$$
    así que la dimensión es igual a $1$.

    \vspace{1em}

    También podemos notar que se trata de la localización
    \[ k [x,x^{-1}] = S^{-1} (k [x]), \quad
       \text{donde } S = \{ 1, x, x^2, x^3, \ldots \}.\]
    Recordemos que los ideales primos en la localización vienen dados por
    \[ \Spec S^{-1} A \isom
       \{ \mathfrak{p} \in \Spec A \mid \mathfrak{p} \cap S = \emptyset \}, \]
    y esta correspondencia preserva las inclusiones. En particular, toda cadena
    de ideales primos en $S^{-1} A$ proviene de una cadena de ideales primos en
    $A$, así que
    $$\dim S^{-1} A \le \dim A.$$
    (Otro modo de decir lo mismo: el homomorfismo canónico $A \to S^{-1} A$
    induce una aplicación continua inyectiva
    $\Spec S^{-1} A \hookrightarrow \Spec A$.)

    Este resultado muy general implica que
    $$\dim (k [x,x^{-1}]) \le \dim k [x] = 1,$$
    y basta encontrar alguna cadena específica de longitud $1$, como por ejemplo
    \[ (0) \subsetneq (x + x^{-1}) \subset k [x,x^{-1}]. \qedhere \]
  \end{solucion}\fi
\end{ejerc}

\begin{ejerc}
  Demuestre que el anillo $\ZZ [x]$ tiene dimensión $2$.

  \ifdefined\solutions\begin{solucion}
    Aquí puede ser útil la descripción completa de $\Spec \ZZ [x]$. Los ideales
    primos en $\ZZ [x]$ son los siguientes:

    \begin{enumerate}
    \item[0)] el ideal nulo $(0)$,

    \item[1)] $(p)$ para $p \in \ZZ$ primo,

    \item[2)] $(f)$ para $f \in \ZZ [x]$ irreducible,

    \item[3)] $(p,f)$ para $p \in \ZZ$ primo y $f \in \ZZ [x]$ tal que
      $\overline{f}$ es irreducible en $\FF_p [x]$; estos ideales son maximales.
    \end{enumerate}

    Las cadenas maximales de ideales primos tendrán entonces forma
    \[ (0) \subsetneq (p) \subsetneq (p,f) \subset \ZZ [x]
       \quad\text{o}\quad
       (0) \subsetneq (f) \subsetneq (p,f) \subset \ZZ [x]. \]

    \vspace{1em}

    Creo que sería interesante demostrar la descripción de $\Spec \ZZ [x]$ de
    arriba. Esto tiene mucho que ver con la aritmética.

    Primero, notemos que los ideales de la lista son primos. La parte 0) es
    obvia: $\ZZ [x]$ es un dominio de integridad. Las partes 1) y 2) se siguen
    de la descripción de los elementos irreducibles (primos) en $\ZZ [x]$. Para
    la parte 3), notamos que
    $$\ZZ [x]/(p,f) \isom \FF_p [x] / (\overline{f}),$$
    que es un cuerpo gracias a la hipótesis de que $\overline{f}$ es irreducible
    (primo) en $\FF_p [x]$. Podemos concluir que el ideal
    $(p,f) \subset \ZZ [x]$ es maximal.

    Lo más interesante es probar que todo ideal primo
    $\mathfrak{p} \subset \ZZ [x]$ es de la forma 0)--3). Asumamos que
    $\mathfrak{p} \ne (0)$. Consideremos el ideal primo
    $\mathfrak{p} \cap \ZZ \subset \ZZ$.

    \begin{itemize}
    \item[i)] \textbf{Asumamos primero que $\mathfrak{p} \cap \ZZ \ne (0)$; es
        decir que $\mathfrak{p} \cap \ZZ = (p)$ para algún primo $p\in \ZZ$}.

      Tenemos en particular $p \in \mathfrak{p}$. El ideal
      $\overline{\mathfrak{p}} \dfn \mathfrak{p} / (p\ZZ [x])$ es primo en el
      anillo cociente $\ZZ [x]/(p) \isom \FF_p [x]$, lo que nos lleva a dos
      posibilidades.

      \begin{itemize}
      \item[a)] $\overline{\mathfrak{p}} = (0)$, y luego
        $\mathfrak{p} = p\ZZ[x]$, lo que corresponde al caso 1) de la lista;

      \item[b)] $\overline{\mathfrak{p}} = \overline{f}\,\FF_p [x]$ donde
        $\overline{f} \in \FF_p [x]$ es algún polinomio irreducible, y luego
        $\mathfrak{p} = p\ZZ [x] + f\,\ZZ [x]$, lo que corresponde al caso 3) de
        la lista.
      \end{itemize}

    \item[ii)] \textbf{Asumamos que $\mathfrak{p} \cap \ZZ = (0)$}.

      Sea $h \in \mathfrak{p}$ un polinomio no nulo. Tenemos la factorización
      única
      $$h = \pm f_1 \cdots f_s,$$
      donde $f_i \in \ZZ [x]$ son polinomios irreducibles (primos). Puesto que
      $\mathfrak{p}$ es un ideal primo, se tiene necesariamente
      $f \dfn f_i \in \mathfrak{p}$ para algún $i = 1,\ldots,s$. La hipótesis
      que $\mathfrak{p} \cap \ZZ = (0)$ implica que este polinomio $f$ no es
      constante. Nuestro objetivo es probar que $\mathfrak{p} = f \ZZ [x]$.
      Sea entonces $g\in \mathfrak{p}$ cualquier polinomio no nulo.

      Asumamos que $f$ no divide a $g$ en $\QQ [x]$. Dado que $f$ es un
      polinomio no constante que es irreducible en $\ZZ [x]$, es también
      irreducible en $\QQ [x]$, y por ende $\mcd (f, g) = 1$, lo que significa
      que
      $$h_1\,f + h_2\,g = 1$$
      para algunos $h_1,h_2\in \QQ [x]$. Tomemos $n$ suficientemente grande tal
      que los polinomios $n\,h_1\,f$ y $n\,h_2\,g$ tienen coeficientes
      enteros. Luego,
      $$n\,h_1\,f + n\,h_2\,g = n \in \mathfrak{p},$$
      pero esto contradice nuestra hipótesis de que
      $\mathfrak{p} \cap \ZZ = (0)$.

      Entonces, $f$ tiene que dividir en $\QQ [x]$ a cualquier polinomio
      $g \in \mathfrak{p}$: tenemos
      $$g = f\,r$$
      para algún $r \in \QQ [x]$. Sin embargo,
      $$\cont (g) = \cont (f)\,\cont (r),$$
      Donde $\cont (f) = 1$, dado que $f$ es un polinomio irreducible en
      $\ZZ [x]$. Entonces, $\cont (r) = \cont (g) \in \ZZ$ y $r \in \ZZ
      [x]$. Esto demuestra que $g \in f\ZZ [x]$, lo que corresponde al caso 2)
      de la lista. \qedhere
    \end{itemize}
  \end{solucion}\fi
\end{ejerc}

\begin{ejerc}
  Encuentre la dimensión de las $k$-álgebras
  $$k[x,y,z]/(xz, xy - 1), \quad k[x,y,z,w]/(zw - y^2 , xy - z^3)$$
  usando la eliminación de variables.

  \ifdefined\solutions\begin{solucion}
    Pongamos
    $$I \dfn (xz, \, xy-1) \subset k[x,y,z].$$
    Sería mejor escribir
    $$I = (z, \, xy - 1).$$
    Esto se sigue de la expresión
    $$z = y\,xz - z\,(xy-z) \in (xz, xy - 1).$$

    Ahora se ve que
    $$I \cap k[x] = I \cap k[y] = 0, \quad I \cap k[z] = (z) \ne 0.$$
    Además,
    $$I \cap k[x,y] = (xy-1) \ne 0.$$
    Esto nos permite concluir que
    $$\dim k[x,y,z]/I = 1.$$

    Consideremos ahora
    $$J \dfn (zw - y^2 , xy - z^3) \subset k [x,y,z,w].$$
    Tenemos
    $$J \cap k [x] = J \cap k [y] = J \cap k [z] = J \cap k [w] = 0.$$
    Luego,
    \[ J \cap k [x,y] =
       J \cap k [x,z] =
       J \cap k [x,w] =
       J \cap k [y,z] =
       J \cap k [y,w] =
       J \cap k [z,w] = 0. \]
    Ahora tenemos claramente
    $$xy - z^3 \in J \cap k[x,y,z], \quad zw - y^2 \in J \cap k[y,z,w].$$
    Con ayuda de Macaulay2 también notamos que
    \[ xyw^3 - y^6 =
       (y^4 + y^2 z w + z^2 w^2)\cdot (zw - y^2) + w^3\cdot (xy-z^3)
       \in J \cap k [x,y,w] \]
    y
    \[ x^2 z w - z^6 =
       x^2 \cdot (zw - y^2) + (z^3+xy)\cdot(xy - z^3)
       \in J \cap k [x,z,w]. \]
    Podemos concluir que
    \[ \dim k [x,y,z,w]/I = 2. \qedhere \]
  \end{solucion}\fi
\end{ejerc}

\end{document}
