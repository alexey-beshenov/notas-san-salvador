\documentclass{article}

\def\exercisespersection{}
% TODO : CLEAN UP THIS MESS
% (AND MAKE SURE ALL TEXTS STILL COMPILE)
\usepackage[leqno]{amsmath}
\usepackage{amssymb}
\usepackage{graphicx}

\usepackage{diagbox} % table heads with diagonal lines
\usepackage{relsize}

\usepackage{wasysym}
\usepackage{scrextend}
\usepackage{epigraph}
\setlength\epigraphwidth{.6\textwidth}

\usepackage[utf8]{inputenc}

\usepackage{titlesec}
\titleformat{\chapter}[display]
  {\normalfont\sffamily\huge\bfseries}
  {\chaptertitlename\ \thechapter}{5pt}{\Huge}
\titleformat{\section}
  {\normalfont\sffamily\Large\bfseries}
  {\thesection}{1em}{}
\titleformat{\subsection}
  {\normalfont\sffamily\large\bfseries}
  {\thesubsection}{1em}{}
\titleformat{\part}[display]
  {\normalfont\sffamily\huge\bfseries}
  {\partname\ \thepart}{0pt}{\Huge}

\usepackage[T1]{fontenc}
\usepackage{fourier}
\usepackage{paratype}

\usepackage[symbol,perpage]{footmisc}

\usepackage{perpage}
\MakePerPage{footnote}

\usepackage{array}
\newcolumntype{x}[1]{>{\centering\hspace{0pt}}p{#1}}

% TODO: the following line causes conflict with new texlive (!)
% \usepackage[english,russian,polutonikogreek,spanish]{babel}
% \newcommand{\russian}[1]{{\selectlanguage{russian}#1}}

% Remove conflicting options for the moment:
\usepackage[english,polutonikogreek,spanish]{babel}

\AtBeginDocument{\shorthandoff{"}}
\newcommand{\greek}[1]{{\selectlanguage{polutonikogreek}#1}}

% % % % % % % % % % % % % % % % % % % % % % % % % % % % % %
% Limit/colimit symbols (with accented i: lím / colím)

\usepackage{etoolbox} % \patchcmd

\makeatletter
\patchcmd{\varlim@}{lim}{\lim}{}{}
\makeatother
\DeclareMathOperator*{\colim}{co{\lim}}
\newcommand{\dirlim}{\varinjlim}
\newcommand{\invlim}{\varprojlim}

% % % % % % % % % % % % % % % % % % % % % % % % % % % % % %

\usepackage[all,color]{xy}

\usepackage{pigpen}
\newcommand{\po}{\ar@{}[dr]|(.4){\text{\pigpenfont I}}}
\newcommand{\pb}{\ar@{}[dr]|(.3){\text{\pigpenfont A}}}
\newcommand{\polr}{\ar@{}[dr]|(.65){\text{\pigpenfont A}}}
\newcommand{\pour}{\ar@{}[ur]|(.65){\text{\pigpenfont G}}}
\newcommand{\hstar}{\mathop{\bigstar}}

\newcommand{\bigast}{\mathop{\Huge \mathlarger{\mathlarger{\ast}}}}

\newcommand{\term}{\textbf}

\usepackage{stmaryrd}

\usepackage{cancel}

\usepackage{tikzsymbols}

\newcommand{\open}{\underset{\mathrm{open}}{\hookrightarrow}}
\newcommand{\closed}{\underset{\mathrm{closed}}{\hookrightarrow}}

\newcommand{\tcol}[2]{{#1 \choose #2}}

\newcommand{\homot}{\simeq}
\newcommand{\isom}{\cong}
\newcommand{\cH}{\mathcal{H}}
\renewcommand{\hom}{\mathrm{hom}}
\renewcommand{\div}{\mathop{\mathrm{div}}}
\renewcommand{\Im}{\mathop{\mathrm{Im}}}
\renewcommand{\Re}{\mathop{\mathrm{Re}}}
\newcommand{\id}[1]{\mathrm{id}_{#1}}
\newcommand{\idid}{\mathrm{id}}

\newcommand{\ZG}{{\ZZ G}}
\newcommand{\ZH}{{\ZZ H}}

\newcommand{\quiso}{\simeq}

\newcommand{\personality}[1]{{\sc #1}}

\newcommand{\mono}{\rightarrowtail}
\newcommand{\epi}{\twoheadrightarrow}
\newcommand{\xepi}[1]{\xrightarrow{#1}\mathrel{\mkern-14mu}\rightarrow}

% % % % % % % % % % % % % % % % % % % % % % % % % % % % % %

\DeclareMathOperator{\Ad}{Ad}
\DeclareMathOperator{\Aff}{Aff}
\DeclareMathOperator{\Ann}{Ann}
\DeclareMathOperator{\Aut}{Aut}
\DeclareMathOperator{\Br}{Br}
\DeclareMathOperator{\CH}{CH}
\DeclareMathOperator{\Cl}{Cl}
\DeclareMathOperator{\Coeq}{Coeq}
\DeclareMathOperator{\Coind}{Coind}
\DeclareMathOperator{\Cop}{Cop}
\DeclareMathOperator{\Corr}{Corr}
\DeclareMathOperator{\Cor}{Cor}
\DeclareMathOperator{\Cov}{Cov}
\DeclareMathOperator{\Der}{Der}
\DeclareMathOperator{\Div}{Div}
\DeclareMathOperator{\D}{D}
\DeclareMathOperator{\Ehr}{Ehr}
\DeclareMathOperator{\End}{End}
\DeclareMathOperator{\Eq}{Eq}
\DeclareMathOperator{\Ext}{Ext}
\DeclareMathOperator{\Frac}{Frac}
\DeclareMathOperator{\Frob}{Frob}
\DeclareMathOperator{\Funct}{Funct}
\DeclareMathOperator{\Fun}{Fun}
\DeclareMathOperator{\GL}{GL}
\DeclareMathOperator{\Gal}{Gal}
\DeclareMathOperator{\Gr}{Gr}
\DeclareMathOperator{\Hol}{Hol}
\DeclareMathOperator{\Hom}{Hom}
\DeclareMathOperator{\Ho}{Ho}
\DeclareMathOperator{\Id}{Id}
\DeclareMathOperator{\Ind}{Ind}
\DeclareMathOperator{\Inn}{Inn}
\DeclareMathOperator{\Isom}{Isom}
\DeclareMathOperator{\Ker}{Ker}
\DeclareMathOperator{\Lan}{Lan}
\DeclareMathOperator{\Lie}{Lie}
\DeclareMathOperator{\Map}{Map}
\DeclareMathOperator{\Mat}{Mat}
\DeclareMathOperator{\Max}{Max}
\DeclareMathOperator{\Mor}{Mor}
\DeclareMathOperator{\Nat}{Nat}
\DeclareMathOperator{\Nrd}{Nrd}
\DeclareMathOperator{\Ob}{Ob}
\DeclareMathOperator{\Out}{Out}
\DeclareMathOperator{\PGL}{PGL}
\DeclareMathOperator{\PSL}{PSL}
\DeclareMathOperator{\PSU}{PSU}
\DeclareMathOperator{\Pic}{Pic}
\DeclareMathOperator{\RHom}{RHom}
\DeclareMathOperator{\Rad}{Rad}
\DeclareMathOperator{\Ran}{Ran}
\DeclareMathOperator{\Rep}{Rep}
\DeclareMathOperator{\Res}{Res}
\DeclareMathOperator{\SL}{SL}
\DeclareMathOperator{\SO}{SO}
\DeclareMathOperator{\SU}{SU}
\DeclareMathOperator{\Sh}{Sh}
\DeclareMathOperator{\Sing}{Sing}
\DeclareMathOperator{\Specm}{Specm}
\DeclareMathOperator{\Spec}{Spec}
\DeclareMathOperator{\Sp}{Sp}
\DeclareMathOperator{\Stab}{Stab}
\DeclareMathOperator{\Sym}{Sym}
\DeclareMathOperator{\Tors}{Tors}
\DeclareMathOperator{\Tor}{Tor}
\DeclareMathOperator{\Tot}{Tot}
\DeclareMathOperator{\UUU}{U}

\DeclareMathOperator{\adj}{adj}
\DeclareMathOperator{\ad}{ad}
\DeclareMathOperator{\af}{af}
\DeclareMathOperator{\card}{card}
\DeclareMathOperator{\cm}{cm}
\DeclareMathOperator{\codim}{codim}
\DeclareMathOperator{\cod}{cod}
\DeclareMathOperator{\coeq}{coeq}
\DeclareMathOperator{\coim}{coim}
\DeclareMathOperator{\coker}{coker}
\DeclareMathOperator{\cont}{cont}
\DeclareMathOperator{\conv}{conv}
\DeclareMathOperator{\cor}{cor}
\DeclareMathOperator{\depth}{depth}
\DeclareMathOperator{\diag}{diag}
\DeclareMathOperator{\diam}{diam}
\DeclareMathOperator{\dist}{dist}
\DeclareMathOperator{\dom}{dom}
\DeclareMathOperator{\eq}{eq}
\DeclareMathOperator{\ev}{ev}
\DeclareMathOperator{\ex}{ex}
\DeclareMathOperator{\fchar}{char}
\DeclareMathOperator{\fr}{fr}
\DeclareMathOperator{\gr}{gr}
\DeclareMathOperator{\im}{im}
\DeclareMathOperator{\infl}{inf}
\DeclareMathOperator{\interior}{int}
\DeclareMathOperator{\intrel}{intrel}
\DeclareMathOperator{\inv}{inv}
\DeclareMathOperator{\length}{length}
\DeclareMathOperator{\mcd}{mcd}
\DeclareMathOperator{\mcm}{mcm}
\DeclareMathOperator{\multideg}{multideg}
\DeclareMathOperator{\ord}{ord}
\DeclareMathOperator{\pr}{pr}
\DeclareMathOperator{\rel}{rel}
\DeclareMathOperator{\res}{res}
\DeclareMathOperator{\rkred}{rkred}
\DeclareMathOperator{\rkss}{rkss}
\DeclareMathOperator{\rk}{rk}
\DeclareMathOperator{\sgn}{sgn}
\DeclareMathOperator{\sk}{sk}
\DeclareMathOperator{\supp}{supp}
\DeclareMathOperator{\trdeg}{trdeg}
\DeclareMathOperator{\tr}{tr}
\DeclareMathOperator{\vol}{vol}

\newcommand{\iHom}{\underline{\Hom}}

\renewcommand{\AA}{\mathbb{A}}
\newcommand{\CC}{\mathbb{C}}
\renewcommand{\SS}{\mathbb{S}}
\newcommand{\TT}{\mathbb{T}}
\newcommand{\PP}{\mathbb{P}}
\newcommand{\BB}{\mathbb{B}}
\newcommand{\RR}{\mathbb{R}}
\newcommand{\ZZ}{\mathbb{Z}}
\newcommand{\FF}{\mathbb{F}}
\newcommand{\HH}{\mathbb{H}}
\newcommand{\NN}{\mathbb{N}}
\newcommand{\QQ}{\mathbb{Q}}
\newcommand{\KK}{\mathbb{K}}

% % % % % % % % % % % % % % % % % % % % % % % % % % % % % %

\usepackage{amsthm}

\newcommand{\legendre}[2]{\left(\frac{#1}{#2}\right)}

\newcommand{\examplesymbol}{$\blacktriangle$}
\renewcommand{\qedsymbol}{$\blacksquare$}

\newcommand{\dfn}{\mathrel{\mathop:}=}
\newcommand{\rdfn}{=\mathrel{\mathop:}}

\usepackage{xcolor}
\definecolor{mylinkcolor}{rgb}{0.0,0.4,1.0}
\definecolor{mycitecolor}{rgb}{0.0,0.4,1.0}
\definecolor{shadecolor}{rgb}{0.79,0.78,0.65}
\definecolor{gray}{rgb}{0.6,0.6,0.6}

\usepackage{colortbl}

\definecolor{myred}{rgb}{0.7,0.0,0.0}
\definecolor{mygreen}{rgb}{0.0,0.7,0.0}
\definecolor{myblue}{rgb}{0.0,0.0,0.7}

\definecolor{redshade}{rgb}{0.9,0.5,0.5}
\definecolor{greenshade}{rgb}{0.5,0.9,0.5}

\usepackage[unicode,colorlinks=true,linkcolor=mylinkcolor,citecolor=mycitecolor]{hyperref}
\newcommand{\refref}[2]{\hyperref[#2]{#1~\ref*{#2}}}
\newcommand{\eqnref}[1]{\hyperref[#1]{(\ref*{#1})}}

\newcommand{\tos}{\!\!\to\!\!}

\usepackage{framed}

\newcommand{\cequiv}{\simeq}

\makeatletter
\newcommand\xleftrightarrow[2][]{%
  \ext@arrow 9999{\longleftrightarrowfill@}{#1}{#2}}
\newcommand\longleftrightarrowfill@{%
  \arrowfill@\leftarrow\relbar\rightarrow}
\makeatother

\newcommand{\bsquare}{\textrm{\ding{114}}}

% % % % % % % % % % % % % % % % % % % % % % % % % % % % % %

\newtheoremstyle{myplain}
  {\topsep}   % ABOVESPACE
  {\topsep}   % BELOWSPACE
  {\itshape}  % BODYFONT
  {0pt}       % INDENT (empty value is the same as 0pt)
  {\bfseries} % HEADFONT
  {.}         % HEADPUNCT
  {5pt plus 1pt minus 1pt} % HEADSPACE
  {\thmnumber{#2}. \thmname{#1}\thmnote{ (#3)}}   % CUSTOM-HEAD-SPEC

\newtheoremstyle{myplainnameless}
  {\topsep}   % ABOVESPACE
  {\topsep}   % BELOWSPACE
  {\normalfont}  % BODYFONT
  {0pt}       % INDENT (empty value is the same as 0pt)
  {\bfseries} % HEADFONT
  {.}         % HEADPUNCT
  {5pt plus 1pt minus 1pt} % HEADSPACE
  {\thmnumber{#2}}   % CUSTOM-HEAD-SPEC 

\newtheoremstyle{sectionexercise}
  {\topsep}   % ABOVESPACE
  {\topsep}   % BELOWSPACE
  {\normalfont}  % BODYFONT
  {0pt}       % INDENT (empty value is the same as 0pt)
  {\bfseries} % HEADFONT
  {.}         % HEADPUNCT
  {5pt plus 1pt minus 1pt} % HEADSPACE
  {Ejercicio \thmnumber{#2}\thmnote{ (#3)}}   % CUSTOM-HEAD-SPEC

\newtheoremstyle{mydefinition}
  {\topsep}   % ABOVESPACE
  {\topsep}   % BELOWSPACE
  {\normalfont}  % BODYFONT
  {0pt}       % INDENT (empty value is the same as 0pt)
  {\bfseries} % HEADFONT
  {.}         % HEADPUNCT
  {5pt plus 1pt minus 1pt} % HEADSPACE
  {\thmnumber{#2}. \thmname{#1}\thmnote{ (#3)}}   % CUSTOM-HEAD-SPEC

% EN ESPAÑOL

\newtheorem*{hecho*}{Hecho}
\newtheorem*{corolario*}{Corolario}
\newtheorem*{teorema*}{Teorema}
\newtheorem*{conjetura*}{Conjetura}
\newtheorem*{proyecto*}{Proyecto}
\newtheorem*{observacion*}{Observación}

\newtheorem*{lema*}{Lema}
\newtheorem*{resultado-clave*}{Resultado clave}
\newtheorem*{proposicion*}{Proposición}

\theoremstyle{definition}
\newtheorem*{ejercicio*}{Ejercicio}
\newtheorem*{definicion*}{Definición}
\newtheorem*{comentario*}{Comentario}
\newtheorem*{definicion-alternativa*}{Definición alternativa}
\newtheorem*{ejemploxs}{Ejemplo}
\newenvironment{ejemplo*}
  {\pushQED{\qed}\renewcommand{\qedsymbol}{\examplesymbol}\ejemploxs}
  {\popQED\endejemploxs}

\theoremstyle{myplain}
\newtheorem{proposicion}{Proposición}[section]

\newtheorem{proyecto}[proposicion]{Proyecto}
\newtheorem{teorema}[proposicion]{Teorema}
\newtheorem{corolario}[proposicion]{Corolario}
\newtheorem{hecho}[proposicion]{Hecho}
\newtheorem{lema}[proposicion]{Lema}

\newtheorem{observacion}[proposicion]{Observación}

\newenvironment{observacionejerc}
    {\pushQED{\qed}\renewcommand{\qedsymbol}{$\square$}\csname inner@observacionejerc\endcsname}
    {\popQED\csname endinner@observacionejerc\endcsname}
\newtheorem{inner@observacionejerc}[proposicion]{Observación}

\newenvironment{proposicionejerc}
    {\pushQED{\qed}\renewcommand{\qedsymbol}{$\square$}\csname inner@proposicionejerc\endcsname}
    {\popQED\csname endinner@proposicionejerc\endcsname}
\newtheorem{inner@proposicionejerc}[proposicion]{Proposicion}

\newenvironment{lemaejerc}
    {\pushQED{\qed}\renewcommand{\qedsymbol}{$\square$}\csname inner@lemaejerc\endcsname}
    {\popQED\csname endinner@lemaejerc\endcsname}
\newtheorem{inner@lemaejerc}[proposicion]{Lema}

\newtheorem{calculo}[proposicion]{Cálculo}

\theoremstyle{myplainnameless}
\newtheorem{nameless}[proposicion]{}

\theoremstyle{mydefinition}
\newtheorem{comentario}[proposicion]{Comentario}
\newtheorem{comentarioast}[proposicion]{Comentario ($\clubsuit$)}
\newtheorem{construccion}[proposicion]{Construcción}
\newtheorem{aplicacion}[proposicion]{Aplicación}
\newtheorem{definicion}[proposicion]{Definición}
\newtheorem{definicion-alternativa}[proposicion]{Definición alternativa}
\newtheorem{notacion}[proposicion]{Notación}
\newtheorem{advertencia}[proposicion]{Advertencia}
\newtheorem{digresion}[proposicion]{Digresión}
\newtheorem{ejemplox}[proposicion]{Ejemplo}
\newenvironment{ejemplo}
  {\pushQED{\qed}\renewcommand{\qedsymbol}{\examplesymbol}\ejemplox}
  {\popQED\endejemplox}
\newtheorem{contraejemplox}[proposicion]{Contraejemplo}
\newenvironment{contraejemplo}
  {\pushQED{\qed}\renewcommand{\qedsymbol}{\examplesymbol}\contraejemplox}
  {\popQED\endcontraejemplox}
\newtheorem{noejemplox}[proposicion]{No-ejemplo}
\newenvironment{noejemplo}
  {\pushQED{\qed}\renewcommand{\qedsymbol}{\examplesymbol}\noejemplox}
  {\popQED\endnoejemplox}
 
\newtheorem{ejemploastx}[proposicion]{Ejemplo ($\clubsuit$)}
\newenvironment{ejemploast}
  {\pushQED{\qed}\renewcommand{\qedsymbol}{\examplesymbol}\ejemploastx}
  {\popQED\endejemploastx}

\ifdefined\exercisespersection
  \theoremstyle{sectionexercise}
  \newtheorem{ejercicio}{}[section]
  \theoremstyle{mydefinition}
\else
  \ifdefined\exercisesglobal
    \theoremstyle{sectionexercise}
    \newtheorem{ejercicio}{}
    \theoremstyle{mydefinition}
  \else
    \ifdefined\exercisespersection
      \newtheorem{ejercicio}[proposicion]{Ejercicio}
    \fi
  \fi
\fi

% % % % % % % % % % % % % % % % % % % % % % % % % % % % % %

\theoremstyle{myplain}
\newtheorem{proposition}{Proposition}[section]
\newtheorem*{fact*}{Fact}
\newtheorem*{proposition*}{Proposition}
\newtheorem{lemma}[proposition]{Lemma}
\newtheorem*{lemma*}{Lemma}

\newtheorem{exercise}{Exercise}
\newtheorem*{hint}{Hint}

\newtheorem{theorem}[proposition]{Theorem}
\newtheorem{conjecture}[proposition]{Conjecture}
\newtheorem*{theorem*}{Theorem}
\newtheorem{corollary}[proposition]{Corollary}
\newtheorem{fact}[proposition]{Fact}
\newtheorem*{claim}{Claim}
\newtheorem{definition-theorem}[proposition]{Definition-theorem}

\theoremstyle{mydefinition}
\newtheorem{examplex}[proposition]{Example}
\newenvironment{example}
  {\pushQED{\qed}\renewcommand{\qedsymbol}{\examplesymbol}\examplex}
  {\popQED\endexamplex}

\newtheorem*{examplexx}{Example}
\newenvironment{example*}
  {\pushQED{\qed}\renewcommand{\qedsymbol}{\examplesymbol}\examplexx}
  {\popQED\endexamplexx}

\newtheorem{definition}[proposition]{Definition}
\newtheorem*{definition*}{Definition}
\newtheorem{wrong-definition}[proposition]{Wrong definition}
\newtheorem{remark}[proposition]{Remark}

\makeatletter
\newcommand{\xRightarrow}[2][]{\ext@arrow 0359\Rightarrowfill@{#1}{#2}}
\makeatother

% % % % % % % % % % % % % % % % % % % % % % % % % % % % % %

\newcommand{\Et}{\mathop{\text{\rm Ét}}}

\newcommand{\categ}[1]{\text{\bf #1}}
\newcommand{\vcateg}{\mathcal}
\newcommand{\bone}{{\boldsymbol 1}}
\newcommand{\bDelta}{{\boldsymbol\Delta}}
\newcommand{\bR}{{\mathbf{R}}}

\newcommand{\univ}{\mathfrak}

\newcommand{\TODO}{\colorbox{red}{\textbf{*** TODO ***}}}
\newcommand{\proofreadme}{\colorbox{red}{\textbf{*** NEEDS PROOFREADING ***}}}

\makeatletter
\def\iddots{\mathinner{\mkern1mu\raise\p@
\vbox{\kern7\p@\hbox{.}}\mkern2mu
\raise4\p@\hbox{.}\mkern2mu\raise7\p@\hbox{.}\mkern1mu}}
\makeatother

\newcommand{\ssincl}{\reflectbox{\rotatebox[origin=c]{45}{$\subseteq$}}}
\newcommand{\vsupseteq}{\reflectbox{\rotatebox[origin=c]{-90}{$\supseteq$}}}
\newcommand{\vin}{\reflectbox{\rotatebox[origin=c]{90}{$\in$}}}

\newcommand{\Ga}{\mathbb{G}_\mathrm{a}}
\newcommand{\Gm}{\mathbb{G}_\mathrm{m}}

\renewcommand{\U}{\UUU}

\DeclareRobustCommand{\Stirling}{\genfrac\{\}{0pt}{}}
\DeclareRobustCommand{\stirling}{\genfrac[]{0pt}{}}

% % % % % % % % % % % % % % % % % % % % % % % % % % % % % %
% tikz

\usepackage{tikz-cd}
\usetikzlibrary{babel}
\usetikzlibrary{decorations.pathmorphing}
\usetikzlibrary{arrows}
\usetikzlibrary{calc}
\usetikzlibrary{fit}
\usetikzlibrary{hobby}

% % % % % % % % % % % % % % % % % % % % % % % % % % % % % %
% Banners

\newcommand\mybannerext[3]{{\normalfont\sffamily\bfseries\large\noindent #1

\noindent #2

\noindent #3

}\noindent\rule{\textwidth}{1.25pt}

\vspace{1em}}

\newcommand\mybanner[2]{{\normalfont\sffamily\bfseries\large\noindent #1

\noindent #2

}\noindent\rule{\textwidth}{1.25pt}

\vspace{1em}}

\renewcommand{\O}{\mathcal{O}}


\usepackage{fullpage}

\theoremstyle{definition}
\newtheorem{ejerc}{Ejercicio}
\newenvironment{solucion}{\begin{proof}[Solución]}{\end{proof}}

\begin{document}

\mybanner{Álgebra computacional. Tarea 5. Fecha límite: 13/06/2019}{Universidad de El Salvador, ciclo impar 2019}

Se puede usar Macaulay2 para \emph{comprobar} algunos cálculos, pero hay que
justificar todas las respuestas.

\subsection*{Serie y polinomio de Hilbert}

\begin{ejerc}
  Para $d \in \NN$ consideremos los polinomios
  $${x \choose d} \dfn \frac{x\,(x-1)\cdots (x-d+1)}{d!} \in \QQ [x].$$

  \begin{enumerate}
  \item[a)] Demuestre que
    \[ {x \choose 0} = 1, ~
       {x \choose 1}, ~
       {x \choose 2}, ~
       \ldots, ~
       {x\choose d} \]
    forman una base del $\QQ$-espacio vectorial formado por los polinomios
    racionales de grado $\le d$.

  \item[b)] Demuestre que todo polinomio $f \in \QQ [x]$ tal que $f (n) \in \ZZ$
    para todo $n\in \ZZ$ puede ser escrito como
    \[ f = a_0 \, {x \choose 0} + a_1 \, {x \choose 1} + \cdots +
             a_d\,{x \choose d}, \]
    donde $a_0, a_1,\ldots,a_d \in \ZZ$.
  \end{enumerate}

  \ifdefined\solutions\begin{solucion}
    La parte a) se sigue inmediatamente del hecho de que los polinomios en
    cuestión tienen grado $0,1,2,\ldots,d$. En la parte b), primero escribamos
    \[ f = a_0 \, {x \choose 0} + a_1 \, {x \choose 1} + \cdots +
             a_d\,{x \choose d}, \]
    donde $a_0, a_1,\ldots,a_d \in \QQ$. Ahora
    \[ f (d) = a_0 \, {d \choose 0} + a_1 \, {d \choose 1} + \cdots +
                 a_d \in \ZZ, \]
    de donde por inducción $a_d \in \ZZ$.
  \end{solucion}\fi
\end{ejerc}

\begin{ejerc}
  Demuestre que si $I \subseteq J$, entonces $H_I (t) = H_J (t)$ implica que
  $I = J$.

  \ifdefined\solutions\begin{solucion}
    Denotemos
    $$A \dfn k [x_1,\ldots,x_n]/I, \quad B \dfn k [x_1,\ldots,x_n]/J.$$
    Tenemos
    \[ A_{\le d} \dfn \{ \overline{f} \in A \mid
                         f \in k [x_1,\ldots,x_n], \, \deg f \le d \}, \quad
       B_{\le d} \dfn \{ \overline{f} \in B \mid
                         f \in k [x_1,\ldots,x_n], \, \deg f \le d \}. \]

    Pongamos
    \[ I_{\le d} \dfn \{ f \in I \mid \deg f \le d \}, \quad
       J_{\le d} \dfn \{ f \in I \mid \deg f \le d \}. \]
    Ahora
    \[ A_{\le d} \isom k [x_1,\ldots,x_n]_{\le d}/I_{\le d}, \quad
      B_{\le d} \isom k [x_1,\ldots,x_n]_{\le d}/J_{\le d}. \]
    Ahora la hipótesis $H_I (t) = H_J (t)$ implica que
    $\widetilde{H}_I (t) = \widetilde{H}_J (t)$, y luego para todo $d$ se cumple
    \[ \dim_k (A_{\le d}) = \dim_k (B_{\le d}) \iff
       \dim_k (I_{\le d}) = \dim_k (J_{\le d}). \]
    Por otra parte, $I \subseteq J$ significa que
    $I_{\le d} \subseteq J_{\le d}$ para todo $d$.
  \end{solucion}\fi
\end{ejerc}

\begin{ejerc}
  Demuestre que el número de monomios $x_1^{\alpha_1}\cdots x_n^{\alpha_n}$ con
  $\alpha_1+\cdots+\alpha_n = d$ es igual a ${n+d-1 \choose d}$.

  \ifdefined\solutions\begin{solucion}
    Denotemos el número de estos monomios por $M_{n,d}$. Está claro que si
    $n = 1$, entonces $M_{n,d} = 1$. Ahora por inducción, asumamos que
    $M_{n,d} = {n+d-1 \choose d} = {n+d-1 \choose n-1}$ y sea $x_{n+1}$ una
    variable extra. Cada monomio $x_1^{\alpha_1}\cdots x_n^{\alpha_n} x_{n+1}^i$
    con $\alpha_1+\cdots+\alpha_n+i = d$ se obtiene de un monomio
    $x_1^{\alpha_1}\cdots x_n^{\alpha_n}$ con
    $\alpha_1+\cdots+\alpha_n = d-i$. Entonces,
    \[ M_{n+1,d} = \sum_{0 \le i \le d} M_{n,d-i} =
       \sum_{0 \le i \le d} {n+i-1 \choose n-1} =
       \sum_{n-1 \le i \le n+d-1} {i \choose n-1} =
       {n + d \choose n} = {(n+1) + d-1 \choose d}. \]
    Aquí hemos usado la identidad
    \[ \sum_{n \le i \le m} {i \choose n} =
       {m+1 \choose n+1} \quad\text{para }m > n. \qedhere \]
  \end{solucion}\fi
\end{ejerc}

\begin{ejerc}
  \label{ejerc:series-de-Hilbert-ideales-monomiales}
  Calcule las series de Hilbert de los siguientes ideales monomiales en
  $k [x,y,z]$:
  $$(x^2, \, y^2, \, z^2), \quad (xy, \, xz^3), \quad (xy, \, xz, \, yz)$$
  de manera directa, contando los monomios.

  \ifdefined\solutions\begin{solucion}
    Recordemos que
    $$\widetilde{h}_I (d) = \dim_k \{ f\in k[x,y,z]/I \mid \deg f \le d \}.$$
    Los ideales en cuestión son monomiales, así que
    $$h_I (d) = \dim_k k \left< x^ay^bz^c\notin I \mid \deg f = d \right>.$$
    En el caso de $I = (x^2, \, y^2, \, z^2)$, tenemos entonces las bases de
    monomios
    \begin{align*}
      d=0\colon & 1,\\
      d=1\colon & x, \, y, \, z,\\
      d=2\colon & xy, \, xz, \, yz,\\
      d=3\colon & xyz,\\
      d > 3\colon & \text{no hay}.
    \end{align*}
    Entonces,
    $$H_{(x^2, y^2, z^2)} (t) = 1 + 3\,t + 3\,t^2 + t^3.$$
    En el caso de $I = (xy, \, xz^3)$ se obtiene
    \begin{align*}
      d=0\colon & 1,\\
      d=1\colon & x, \, y, \, z,\\
      d=2\colon & x^2, \, xz, \, y^2, \, yz, \, z^2,\\
      d=3\colon & x^3, \, x^2z, \, xz^2, \, y^3, \, y^2z, \, yz^2, \, z^3,\\
      d=4\colon & x^4, \, x^3z, \, x^2z^2, \, y^4, \, y^3\,z, \, y^2z^2, \, yz^3, \, z^4,\\
                & \cdots
    \end{align*}
    En general, para $d \ge 3$ tendremos los siguientes $d+4$ monomios:
    $$x^d, \, x^{d-1}z, \, x^{d-2}\,z^2, \, y^{d-i}z^i ~ (0 \le i \le d).$$
    Esto nos da
    \[ H_{(xy, xz^3)} (t) =
       1 + 3t + 5t^2 + \sum_{d\ge 3} (d+4)\,t^d =
       1 + 3t + 5t^2 + t^3\,\left(\frac{6}{1-t} + \frac{1}{(1-t)^2}\right) =
       \frac{-t^4+t+1}{(1-t)^2}. \]
    En fin, para el ideal $(xy, \, xz, \, yz)$ se tiene
    \begin{align*}
      d=0\colon & 1,\\
      d\ge 1\colon & x^d, \, y^d, \, z^d,
    \end{align*}
    de donde
    \[ H_{(xy, xz, yz)} (t) =
       1 + 3t\,\sum_{d\ge 0} t^d =
       1 + \frac{3t}{1-t} =
       \frac{2t+1}{1-t}. \qedhere \]
  \end{solucion}\fi
\end{ejerc}

\begin{ejerc}
  Encuentre un ideal particular $I \subset k [x_1,\ldots,x_n]$ y orden monomial
  $\preceq$ que no respeta el grado tales que $H_I (t) \ne H_{LT (I)} (t)$.

  \ifdefined\solutions\begin{solucion}
    Consideremos el ideal $I \dfn (x^2-y) \subset k [x,y]$. Respecto al orden
    graduado lexicográfico se tiene $LT (I) = (x^2)$, así que
    $$H_I (t) = \frac{1 - t^2}{(1-t)^2}.$$
    Ahora respecto al orden lexicográfico con $y \succ x$, tenemos
    $LT (I) = (y)$, y luego
    \[ H_{(y)} (t) = \frac{1-t}{(1-t)^2}. \qedhere \]
  \end{solucion}\fi
\end{ejerc}

\begin{ejerc}
  Demuestre que para un polinomio no nulo $f \in k [x_1,\ldots,x_n]$, el ideal
  principal correspondiente $I = (f) \subseteq k [x_1,\ldots,x_n]$ tiene la
  serie de Hilbert
  $$H_I (t) = \frac{1 - t^{\deg f}}{(1-t)^n}.$$

  \ifdefined\solutions\begin{solucion}
    Respecto a cualquier orden monomial, $G = \{ f \}$ será una base de Gröbner
    de $f$. Ahora si el orden monomial respeta el grado, entonces
    \[ H_{(f)} (t) = H_{(LT (f))} (t) =
       \frac{1 - t^{\deg LT (f)}}{(1-t)^n} =
       \frac{1 - t^{\deg f}}{(1-t)^n}. \qedhere \]
  \end{solucion}\fi
\end{ejerc}

\begin{ejerc}
  Se dice que un ideal $I \subseteq k [x_1,\ldots,x_n]$ es \term{homogéneo} si
  $I$ está generado por \term{polinomios homogéneos}
  $$f = c_{\alpha (1)}\,x^{\alpha (1)} + \cdots + c_{\alpha (s)}\,x^{\alpha (s)},$$
  donde
  $$\deg x^{\alpha (1)} = \cdots = \deg x^{\alpha (s)}.$$
  Por ejemplo, $(x^3 - xz^2 - y^2z, \, x+y)$ es un ideal homogéneo.

  Demuestre que si $I, J \subseteq k [x_1,\ldots,x_n]$ son homogéneos, entonces
  $$H_{I+J} (t) = H_I (t) + H_J (t) - H_{I\cap J} (t).$$
  Encuentre un contraejemplo para el caso no-homogéneo.

  \ifdefined\solutions\begin{solucion}
    Primero, tenemos el siguiente contraejemplo no-homogéneo tonto: en el anillo
    $k [x]$ consideremos los ideales $I = (x)$ y $J = (x+1)$. Luego,
    $I + J = 1$, así que
    $$H_{I+J} (t) = 0.$$
    Por otra parte, $I\cap J = (x^2+x)$, y luego
    $$H_I (t) + H_J (t) - H_{I\cap J} (t) = 1 + 1 - \frac{1-t^2}{1-t} = 1-t,$$
    lo que ni siquiera representa una serie de Hilbert legítima.

    \vspace{1em}

    Ahora para ideales homogéneos $I = (f_1,\ldots,f_r)$ y
    $J = (f_{r+1},\ldots,f_s)$ el ideal $I + J = (f_1,\ldots,f_s)$ es también
    homogéneo. Tenemos la sucesión exacta corta de espacios $k$-vectoriales
    $$0 \to (I \cap J)_{\le d} \to I_{\le d} \to (I+J)_{\le d}/J_{\le d} \to 0$$
    Aquí la sobreyectividad de la segunda aplicación puede ser vista de la
    siguiente manera: todo elemento de $I+J$ tiene forma
    \[ f = \underbrace{g_1 f_1 + \cdots + g_r f_r}_{\in I} +
           \underbrace{g_{r+1} f_{r+1} + \cdots + g_s f_s}_{\in J} \]
    para algunos $g_1,\ldots,g_s \in k [x_1,\ldots,x_n]$. Ahora si
    $\deg f \le d$, entonces podemos quitar de cada $g_i$ la parte homogénea de
    grado $> d - \deg f_i$, y la identidad de arriba se preserva.

    La sucesión exacta de arriba nos da
    \[ \dim_k (I_{\le d}) =
       \dim_k ((I\cap J)_{\le d}) + \dim_k ((I+J)_{\le d}/J_{\le d}) =
       \dim_k ((I\cap J)_{\le d}) + \dim_k ((I+J)_{\le d}) - \dim_k (J_{\le d}). \]
    De aquí se sigue que
    \[ \dim_k \left(\frac{k[x_1,\ldots,x_n]_{\le d}}{(I+J)_{\le d}}\right) =
       \dim_k \left(\frac{k[x_1,\ldots,x_n]_{\le d}}{I_{\le d}}\right) +
       \dim_k \left(\frac{k[x_1,\ldots,x_n]_{\le d}}{J_{\le d}}\right) -
       \dim_k \left(\frac{k[x_1,\ldots,x_n]_{\le d}}{(I\cap J)_{\le d}}\right), \]
    lo que implica
    \[ \widetilde{H}_{I+J} (t) =
         \widetilde{H}_I (t) +
         \widetilde{H}_J (t) -
         \widetilde{H}_{I\cap J} (t). \qedhere \]
  \end{solucion}\fi
\end{ejerc}

\begin{ejerc}
  Comprueba sus cálculos del ejercicio
  \ref{ejerc:series-de-Hilbert-ideales-monomiales} usando el algoritmo
  recursivo.

  \ifdefined\solutions\begin{solucion}
    Para el ideal $(x^2, \, y^2, \, z^2)$ tenemos
    \begin{multline*}
      H_{(x^2, \, y^2, \, z^2)} (t) =
      H_{(x^2, \, y^2)} (t) + H_{(z^2)} (t) - H_{(x^2 z^2, \, y^2 z^2)} (t) \\
      = H_{(x^2)} (t) + H_{(y^2)} (t) -  H_{(x^2 y^2)} (t) + H_{(z^2)} (t) -
        \Bigl(H_{(x^2 z^2)} (t) + H_{(y^2 z^2)} (t) - H_{(x^2 y^2 z^2)} (t) \Bigr) \\
      = \frac{(1-t^2)+(1-t^2)-(1-t^4)+(1-t^2)-(1-t^4)-(1-t^4)+(1-t^6)}{(1-t)^3} =
        \frac{-t^6 + 3t^4 - 3t^2 + 1}{(1-t)^3} = \frac{t^3 + 3t^2 + 3t + 1}{1}.
    \end{multline*}
    De la misma manera, para $(xy, \, xz^3)$ calculamos
    \[ H_{(xy, \, xz^3)} (t) =
       H_{(xy)} (t) + H_{(xz^3)} (t) - H_{(xyz^3)} (t) =
       \frac{(1-t^2) + (1-t^4) - (1-t^5)}{(1-t)^3} =
       \frac{t^5 - t^4 - t^2 + 1}{(1-t)^3} =
       \frac{-t^4 + t + 1}{(1-t)^2}. \]
    Y en fin, para $(xy, \, xz, \, yz)$ se tiene
    \begin{multline*}
      H_{(xy, \, xz, \, yz)} (t) =
      H_{(xy, \, xz)} (t) + H_{(yz)} (t) - H_{(xyz)} (t) =
      H_{(xy)} (t) + H_{(xy)} (t) - H_{(xyz)} (t) + H_{(yz)} (t) - H_{(xyz)} (t) \\
    = \frac{(1-t^2) + (1-t^2) - (1-t^3) + (1-t^2) - (1-t^3)}{(1-t)^3} =
      \frac{2\,t^3 - 3\,t^2 + 1}{(1-t)^3} =
      \frac{2t + 1}{1-t}. \qedhere
    \end{multline*}
  \end{solucion}\fi
\end{ejerc}

\begin{ejerc}
  Calcule los polinomios de Hilbert $p_I (x)$ para los ideales del ejercicio
  \ref{ejerc:series-de-Hilbert-ideales-monomiales}. ¿Para cuáles valores de $d$
  se cumple $p_I (d) = h_I (d)$?

  \ifdefined\solutions\begin{solucion}
    Tenemos
    $$H_{(x^2, \, y^2, \, z^2)} (t) = \frac{t^3 + 3 t^2 + 3 t + 1}{1},$$
    así que
    $$p_{(x^2, \, y^2, \, z^2)} (x) = 0.$$
    En este caso se tiene $h_I (d) = p_I (d)$ para $d \ge 4$.

    Ahora
    \begin{multline*}
      H_{(xy, \, xz^3)} (t) =
      \frac{-t^4 + t + 1}{(1-t)^2} =
      (-t^4 + t + 1)\,\sum_{d\ge 0} (d+1)\,t^d \\
    = \sum_{d\ge 0} (d+1)\,t^d + \sum_{d\ge 1} d\,t^d - \sum_{d\ge 4} (d-3)\,t^d =
      1 + 3t + 5\,t^2 + \sum_{d\ge 3} (d+4)\,t^d.
    \end{multline*}
    De aquí se tiene
    $$p_{(xy, \, xz^3)} (x) = x+4,$$
    y $h_I (d) = p_I (d)$ para $d \ge 3$.

    En fin,
    \[ H_{(xy, \, xz, \, yz)} (t) =
       \frac{2t + 1}{1-t} =
       1 + \frac{3t}{1-t} =
       1 + 3t + 3t^2 + 3t^3 + \cdots \]
    de donde
    $$p_{(xy, \, xz, \, yz)} (x) = 3.$$
    Se cumple $h_I (d) = p_I (d)$ para $d \ge 1$.
  \end{solucion}\fi
\end{ejerc}

\begin{ejerc}
  Calcule la serie de Hilbert del ideal $I \dfn (xy + z, xy^3) \subset k[x,y,z]$
  mediante una base de Gröbner y el algoritmo recursivo.

  \ifdefined\solutions\begin{solucion}
    La base de Gröbner respecto al orden graduado lexicográfico viene dada por
    $$G = \{ xy+z, \, z^3, \, yz^2, \, y^2z \}.$$
    Luego,
    $$H_I (t) = H_{LT (I)} (t) = H_{(xy, z^3, yz^2, y^2z)} (t).$$
    Calculemos la última serie:
    \begin{multline*}
      H_{(xy, z^3, yz^2, y^2z)} (t) =
      H_{(xy, z^3, yz^2)} (t) + H_{(y^2z)} (t) - H_{(xy^2z, y^2 z^2)} (t) \\
      = H_{(xy, z^3)} (t) + H_{(yz^2)} (t) - H_{(xyz^2, yz^3)} (t) + H_{(y^2z)} (t) - \Bigl(H_{(xy^2z)} (t) + H_{(y^2 z^2)} (t) - H_{(x y^2 z^2)} (t)\Bigr) \\
      = H_{(xy)} (t) + H_{(z^3)} (t) - H_{(xyz^3)} (t) + H_{(yz^2)} (t) - \Bigl(H_{(xyz^2)} (t) + H_{(yz^3)} (t) - H_{(xyz^3)} (t)\Bigr) + H_{(y^2z)} (t) \\
      - \Bigl(H_{(xy^2z)} (t) + H_{(y^2 z^2)} (t) - H_{(x y^2 z^2)} (t)\Bigr) \\
      = \frac{(1-t^2) + (1-t^3) - (1-t^5) + (1-t^3) - (1-t^4) - (1-t^4) + (1-t^5) + (1-t^3) - (1-t^4) - (1-t^4) + (1-t^5)}{(1-t)^3} \\
      = \frac{-t^5 + 4\,t^4 - 3\,t^3 - t^2 + 1}{(1-t)^3} =
      \frac{-t^3 + 2t^2 + 2t + 1}{1-t} =
      1 + 3\,t + 5\,t^2 + 4\,t^3 + 4\,t^4 + 4\,t^6 + \cdots
    \end{multline*}
    (El polinomio de Hilbert correspondiente es igual a $4$ y
    $h_I (d) = p_I (d)$ para $d \ge 3$.)
  \end{solucion}\fi
\end{ejerc}

\subsection*{Normalización de Noether}

\begin{ejerc}
  Calcule la subálgebra
  $$k [\overline{x},\overline{y}] \subset k[x,y,z]/(x^2+y^2-z^2, \, x+y+z).$$

  \ifdefined\solutions\begin{solucion} Consideremos el homomorfismo
    \begin{align*}
      \phi\colon k [t,u] & \to k[x,y,z]/(x^2+y^2-z^2, \, x+y+z),\\
      t & \mapsto \overline{x},\\
      u & \mapsto \overline{y}.
    \end{align*}
    Ahora
    $$\ker \phi = (x^2+y^2-z^2, \, x+y+z, \, x-t, \, y-u) \cap k [t,u].$$
    La base de Gröbner del ideal
    $(x^2+y^2-z^2, \, x+y+z, \, x-t, \, y-u) \subset k [x,y,z,t,u]$ respecto al
    orden lexicográfico con $x \succ y \succ z \succ t \succ u$ viene dada por
    $$G = \{ z+t+u, \, y-u, \, x-t, \, tu \}.$$
    Entonces,
    $$\ker \phi = (tu).$$
    Podemos concluir que
    \[ k [\overline{x},\overline{y}] \isom k [t,u]/(tu). \qedhere \]
  \end{solucion}\fi
\end{ejerc}

\begin{ejerc}
  Consideremos las $k$-álgebras
  $$A \dfn k [x,y]/(x^2\,(x+1)-y^2), \quad B \dfn k[x,y,z]/(x-z^2+1, y-z^3+z).$$
  Demuestre que se tiene un homomorfismo inyectivo natural
  $A \hookrightarrow B$, y $B$ es integral sobre $A$.

  \ifdefined\solutions\begin{solucion}
    Consideremos el homomorfismo

    \begin{align*}
      \phi\colon k [t,u] & \to k[x,y,z]/(x-z^2+1, y-z^3+z),\\
      t & \mapsto x,\\
      u & \mapsto y.
    \end{align*}
    Ahora
    $$\ker \phi = (x-z^2+1, \, y-z^3+z, \, t-x, \, u-y) \cap k [t,u].$$
    Una base de Gröbner del ideal $(x-z^2+1, \, y-z^3+z, \, t-x, \, u-y)$
    respecto al orden lexicográfico con $x \succ y \succ z \succ t \succ u$
    viene dada por
    $$G = \{ t^3+t^2-u^2, \, zu-t^2-t, \, zt-u, \, z^2-t-1, \, y-u, \, x-t \},$$
    lo que nos permite concluir que
    $$\ker \phi = (t^3+t^2-u^2).$$
    Ahora $\overline{z}$ es integral sobre $A = k [\overline{x},\overline{y}]$
    gracias a la relación $\overline{z}^3 - \overline{z} - \overline{y} = 0$.
  \end{solucion}\fi
\end{ejerc}

\begin{ejerc}
  Consideremos la $k$-álgebra
  $A \dfn k [x_1,x_2] / (x_1 x_2^2 + x_1^2 x_2 + 1)$. Encuentre

  \begin{enumerate}
  \item[a)] una normalización de Noether para $k = \FF_2$
    (recuerde que esta no puede ser lineal),
  \item[b)] una normalización de Noether para $k = \QQ$ de la forma
    $a = c_1 x_1 + c_2 x_2$ para $c_1, c_2 \in \QQ$.
  \end{enumerate}

  \ifdefined\solutions\begin{solucion}
    Tenemos $\dim A = 1$, así que estamos buscando un elemento $a \in A$ tal que
    la extensión $k [a] \subset A$ es integral. Consideremos el polinomio
    $f = x_1 x_2^2 + x_1^2 x_2 + 1$. Tenemos $\deg f = 3$, así que podemos poner
    $\delta = 4$. Hagamos la sustitución $y_2 = x_2 - x_1^4$. Luego
    \[ f (x_1,x_2) =
       f (x_1, \, y_2 + x_1^4) =
       x_1\,(y_2 + x_1^4)^2 + x_1^2\,(y_2 + x_1^4) + 1 =
       x_1^9 + x_1^6 + 2 y_2\cdot x_1^5 + y_2\cdot x_1^2 + y_2^2\cdot x_1 + 1. \]
    Esto nos da una dependencia integral de $\overline{x_1}$ sobre
    $k [\overline{y_2}]$. Entonces, podemos tomar
    $a = \overline{x_2} - \overline{x_1}^4$.

    Para la sustitución lineal en el caso de $k = \QQ$, escribamos
    $y_1 = x_1 - cx_2$. Luego,
    \[ f (x_1,x_2) =
       f (y_1 + cx_2, \, x_2) =
       (y_1 + cx_2)\,x_2^2 + (y_1 + cx_2)^2\,x_2 + 1 =
       c\,(c+1)\cdot x_2^3 + (2c + 1)\,y_1\cdot x_2^2 + y_1^2\cdot x_2 + 1. \]
    Entonces, basta tomar cualquier $c \ne 0, -1$ para obtener una dependencia
    integral de $\overline{x_2}$ sobre $k [\overline{y_1}]$. Por ejemplo,
    funcionaría $c = +1$.
  \end{solucion}\fi
\end{ejerc}

\begin{ejerc}
  Encuentre una normalización de Noether en la forma lineal para la $k$-álgebra
  $$A \dfn k [x_1,x_2,x_3] / (x_1^2 x_2^2, \, x_1^2 x_3^2).$$

  \ifdefined\solutions\begin{solucion}
    Primero notamos que el ideal $I = (x_1^2 x_2^2, \, x_1^2 x_3^2)$ es
    monomial, y por ende se ve fácilmente que $I \cap k[x_2,x_3] = 0$, así que
    $\dim A = 2$. Entonces, la normalización de Noether debe consistir en dos
    elementos $a_1,a_2 \in A$ tales que la extensión $k [a_1,a_2] \subset A$ es
    integral.

    Hagamos sustituciones
    $$y_1 = x_1 - c_1 x_3, \quad y_2 = x_2 - c_2 x_3.$$
    Consideremos el polinomio $f = x_1^2 x_2^2$. Tenemos
    \[ f (x_1,x_2,x_3) =
       f (y_1 + c_1 x_3, \, y_2 + c_2 x_3, \, x_3) =
       (y_1 + c_1 x_3)^2\,(y_2 + c_2 x_3)^2. \]
    Podemos poner $c_1 = c_2 = 1$, y luego la fórmula de arriba nos dirá que
    $x_3$ es integral sobre la subálgebra
    \[ k [\overline{y_1},\overline{y_2}] =
       k [\overline{x_1 + x_3},\overline{x_2 + x_3}]. \]
    Esta subálgebra es isomorfa a $k [t,u]$. Para comprobarlo, podemos
    considerar el homomorfismo
    \begin{align*}
      \phi\colon k[t,u] & \to k [x_1,x_2,x_3] / (x_1^2 x_2^2, \, x_1^2 x_3^2),\\
      t & \mapsto \overline{x_1 + x_3},\\
      u & \mapsto \overline{x_2 + x_3}.
    \end{align*}
    Ahora
    \[ \ker \phi =
       (x_1^2x_2^2,\, x_1^2x_3^2,\, t-x_1-x_3,\, u-x_2-x_3) \cap k [t,u]. \]
    La base de Gröbner correspondiente respecto al orden lexicográfico con
    $x_1 \succ x_2 \succ x_3 \succ t \succ u$ es
    \[ G = \{ x_3^2 u^3 - 2 x_3 tu^3 + t^2u^3, \,
              2x_3^3u-4x_3^2tu -x_3^2u^2+2x_3t^2u+2x_3tu^2-t^2u^2, \,
              x_3^4-2x_3^3t+x_3^2t^2, \,
              x_2+x_3-u, \,
              x_1+x_3-t \}, \]
    así que $\ker \phi = 0$.
  \end{solucion}\fi
\end{ejerc}

\end{document}
