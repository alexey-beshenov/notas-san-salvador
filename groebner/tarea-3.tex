\documentclass{article}

\def\exercisespersection{}
% TODO : CLEAN UP THIS MESS
% (AND MAKE SURE ALL TEXTS STILL COMPILE)
\usepackage[leqno]{amsmath}
\usepackage{amssymb}
\usepackage{graphicx}

\usepackage{diagbox} % table heads with diagonal lines
\usepackage{relsize}

\usepackage{wasysym}
\usepackage{scrextend}
\usepackage{epigraph}
\setlength\epigraphwidth{.6\textwidth}

\usepackage[utf8]{inputenc}

\usepackage{titlesec}
\titleformat{\chapter}[display]
  {\normalfont\sffamily\huge\bfseries}
  {\chaptertitlename\ \thechapter}{5pt}{\Huge}
\titleformat{\section}
  {\normalfont\sffamily\Large\bfseries}
  {\thesection}{1em}{}
\titleformat{\subsection}
  {\normalfont\sffamily\large\bfseries}
  {\thesubsection}{1em}{}
\titleformat{\part}[display]
  {\normalfont\sffamily\huge\bfseries}
  {\partname\ \thepart}{0pt}{\Huge}

\usepackage[T1]{fontenc}
\usepackage{fourier}
\usepackage{paratype}

\usepackage[symbol,perpage]{footmisc}

\usepackage{perpage}
\MakePerPage{footnote}

\usepackage{array}
\newcolumntype{x}[1]{>{\centering\hspace{0pt}}p{#1}}

% TODO: the following line causes conflict with new texlive (!)
% \usepackage[english,russian,polutonikogreek,spanish]{babel}
% \newcommand{\russian}[1]{{\selectlanguage{russian}#1}}

% Remove conflicting options for the moment:
\usepackage[english,polutonikogreek,spanish]{babel}

\AtBeginDocument{\shorthandoff{"}}
\newcommand{\greek}[1]{{\selectlanguage{polutonikogreek}#1}}

% % % % % % % % % % % % % % % % % % % % % % % % % % % % % %
% Limit/colimit symbols (with accented i: lím / colím)

\usepackage{etoolbox} % \patchcmd

\makeatletter
\patchcmd{\varlim@}{lim}{\lim}{}{}
\makeatother
\DeclareMathOperator*{\colim}{co{\lim}}
\newcommand{\dirlim}{\varinjlim}
\newcommand{\invlim}{\varprojlim}

% % % % % % % % % % % % % % % % % % % % % % % % % % % % % %

\usepackage[all,color]{xy}

\usepackage{pigpen}
\newcommand{\po}{\ar@{}[dr]|(.4){\text{\pigpenfont I}}}
\newcommand{\pb}{\ar@{}[dr]|(.3){\text{\pigpenfont A}}}
\newcommand{\polr}{\ar@{}[dr]|(.65){\text{\pigpenfont A}}}
\newcommand{\pour}{\ar@{}[ur]|(.65){\text{\pigpenfont G}}}
\newcommand{\hstar}{\mathop{\bigstar}}

\newcommand{\bigast}{\mathop{\Huge \mathlarger{\mathlarger{\ast}}}}

\newcommand{\term}{\textbf}

\usepackage{stmaryrd}

\usepackage{cancel}

\usepackage{tikzsymbols}

\newcommand{\open}{\underset{\mathrm{open}}{\hookrightarrow}}
\newcommand{\closed}{\underset{\mathrm{closed}}{\hookrightarrow}}

\newcommand{\tcol}[2]{{#1 \choose #2}}

\newcommand{\homot}{\simeq}
\newcommand{\isom}{\cong}
\newcommand{\cH}{\mathcal{H}}
\renewcommand{\hom}{\mathrm{hom}}
\renewcommand{\div}{\mathop{\mathrm{div}}}
\renewcommand{\Im}{\mathop{\mathrm{Im}}}
\renewcommand{\Re}{\mathop{\mathrm{Re}}}
\newcommand{\id}[1]{\mathrm{id}_{#1}}
\newcommand{\idid}{\mathrm{id}}

\newcommand{\ZG}{{\ZZ G}}
\newcommand{\ZH}{{\ZZ H}}

\newcommand{\quiso}{\simeq}

\newcommand{\personality}[1]{{\sc #1}}

\newcommand{\mono}{\rightarrowtail}
\newcommand{\epi}{\twoheadrightarrow}
\newcommand{\xepi}[1]{\xrightarrow{#1}\mathrel{\mkern-14mu}\rightarrow}

% % % % % % % % % % % % % % % % % % % % % % % % % % % % % %

\DeclareMathOperator{\Ad}{Ad}
\DeclareMathOperator{\Aff}{Aff}
\DeclareMathOperator{\Ann}{Ann}
\DeclareMathOperator{\Aut}{Aut}
\DeclareMathOperator{\Br}{Br}
\DeclareMathOperator{\CH}{CH}
\DeclareMathOperator{\Cl}{Cl}
\DeclareMathOperator{\Coeq}{Coeq}
\DeclareMathOperator{\Coind}{Coind}
\DeclareMathOperator{\Cop}{Cop}
\DeclareMathOperator{\Corr}{Corr}
\DeclareMathOperator{\Cor}{Cor}
\DeclareMathOperator{\Cov}{Cov}
\DeclareMathOperator{\Der}{Der}
\DeclareMathOperator{\Div}{Div}
\DeclareMathOperator{\D}{D}
\DeclareMathOperator{\Ehr}{Ehr}
\DeclareMathOperator{\End}{End}
\DeclareMathOperator{\Eq}{Eq}
\DeclareMathOperator{\Ext}{Ext}
\DeclareMathOperator{\Frac}{Frac}
\DeclareMathOperator{\Frob}{Frob}
\DeclareMathOperator{\Funct}{Funct}
\DeclareMathOperator{\Fun}{Fun}
\DeclareMathOperator{\GL}{GL}
\DeclareMathOperator{\Gal}{Gal}
\DeclareMathOperator{\Gr}{Gr}
\DeclareMathOperator{\Hol}{Hol}
\DeclareMathOperator{\Hom}{Hom}
\DeclareMathOperator{\Ho}{Ho}
\DeclareMathOperator{\Id}{Id}
\DeclareMathOperator{\Ind}{Ind}
\DeclareMathOperator{\Inn}{Inn}
\DeclareMathOperator{\Isom}{Isom}
\DeclareMathOperator{\Ker}{Ker}
\DeclareMathOperator{\Lan}{Lan}
\DeclareMathOperator{\Lie}{Lie}
\DeclareMathOperator{\Map}{Map}
\DeclareMathOperator{\Mat}{Mat}
\DeclareMathOperator{\Max}{Max}
\DeclareMathOperator{\Mor}{Mor}
\DeclareMathOperator{\Nat}{Nat}
\DeclareMathOperator{\Nrd}{Nrd}
\DeclareMathOperator{\Ob}{Ob}
\DeclareMathOperator{\Out}{Out}
\DeclareMathOperator{\PGL}{PGL}
\DeclareMathOperator{\PSL}{PSL}
\DeclareMathOperator{\PSU}{PSU}
\DeclareMathOperator{\Pic}{Pic}
\DeclareMathOperator{\RHom}{RHom}
\DeclareMathOperator{\Rad}{Rad}
\DeclareMathOperator{\Ran}{Ran}
\DeclareMathOperator{\Rep}{Rep}
\DeclareMathOperator{\Res}{Res}
\DeclareMathOperator{\SL}{SL}
\DeclareMathOperator{\SO}{SO}
\DeclareMathOperator{\SU}{SU}
\DeclareMathOperator{\Sh}{Sh}
\DeclareMathOperator{\Sing}{Sing}
\DeclareMathOperator{\Specm}{Specm}
\DeclareMathOperator{\Spec}{Spec}
\DeclareMathOperator{\Sp}{Sp}
\DeclareMathOperator{\Stab}{Stab}
\DeclareMathOperator{\Sym}{Sym}
\DeclareMathOperator{\Tors}{Tors}
\DeclareMathOperator{\Tor}{Tor}
\DeclareMathOperator{\Tot}{Tot}
\DeclareMathOperator{\UUU}{U}

\DeclareMathOperator{\adj}{adj}
\DeclareMathOperator{\ad}{ad}
\DeclareMathOperator{\af}{af}
\DeclareMathOperator{\card}{card}
\DeclareMathOperator{\cm}{cm}
\DeclareMathOperator{\codim}{codim}
\DeclareMathOperator{\cod}{cod}
\DeclareMathOperator{\coeq}{coeq}
\DeclareMathOperator{\coim}{coim}
\DeclareMathOperator{\coker}{coker}
\DeclareMathOperator{\cont}{cont}
\DeclareMathOperator{\conv}{conv}
\DeclareMathOperator{\cor}{cor}
\DeclareMathOperator{\depth}{depth}
\DeclareMathOperator{\diag}{diag}
\DeclareMathOperator{\diam}{diam}
\DeclareMathOperator{\dist}{dist}
\DeclareMathOperator{\dom}{dom}
\DeclareMathOperator{\eq}{eq}
\DeclareMathOperator{\ev}{ev}
\DeclareMathOperator{\ex}{ex}
\DeclareMathOperator{\fchar}{char}
\DeclareMathOperator{\fr}{fr}
\DeclareMathOperator{\gr}{gr}
\DeclareMathOperator{\im}{im}
\DeclareMathOperator{\infl}{inf}
\DeclareMathOperator{\interior}{int}
\DeclareMathOperator{\intrel}{intrel}
\DeclareMathOperator{\inv}{inv}
\DeclareMathOperator{\length}{length}
\DeclareMathOperator{\mcd}{mcd}
\DeclareMathOperator{\mcm}{mcm}
\DeclareMathOperator{\multideg}{multideg}
\DeclareMathOperator{\ord}{ord}
\DeclareMathOperator{\pr}{pr}
\DeclareMathOperator{\rel}{rel}
\DeclareMathOperator{\res}{res}
\DeclareMathOperator{\rkred}{rkred}
\DeclareMathOperator{\rkss}{rkss}
\DeclareMathOperator{\rk}{rk}
\DeclareMathOperator{\sgn}{sgn}
\DeclareMathOperator{\sk}{sk}
\DeclareMathOperator{\supp}{supp}
\DeclareMathOperator{\trdeg}{trdeg}
\DeclareMathOperator{\tr}{tr}
\DeclareMathOperator{\vol}{vol}

\newcommand{\iHom}{\underline{\Hom}}

\renewcommand{\AA}{\mathbb{A}}
\newcommand{\CC}{\mathbb{C}}
\renewcommand{\SS}{\mathbb{S}}
\newcommand{\TT}{\mathbb{T}}
\newcommand{\PP}{\mathbb{P}}
\newcommand{\BB}{\mathbb{B}}
\newcommand{\RR}{\mathbb{R}}
\newcommand{\ZZ}{\mathbb{Z}}
\newcommand{\FF}{\mathbb{F}}
\newcommand{\HH}{\mathbb{H}}
\newcommand{\NN}{\mathbb{N}}
\newcommand{\QQ}{\mathbb{Q}}
\newcommand{\KK}{\mathbb{K}}

% % % % % % % % % % % % % % % % % % % % % % % % % % % % % %

\usepackage{amsthm}

\newcommand{\legendre}[2]{\left(\frac{#1}{#2}\right)}

\newcommand{\examplesymbol}{$\blacktriangle$}
\renewcommand{\qedsymbol}{$\blacksquare$}

\newcommand{\dfn}{\mathrel{\mathop:}=}
\newcommand{\rdfn}{=\mathrel{\mathop:}}

\usepackage{xcolor}
\definecolor{mylinkcolor}{rgb}{0.0,0.4,1.0}
\definecolor{mycitecolor}{rgb}{0.0,0.4,1.0}
\definecolor{shadecolor}{rgb}{0.79,0.78,0.65}
\definecolor{gray}{rgb}{0.6,0.6,0.6}

\usepackage{colortbl}

\definecolor{myred}{rgb}{0.7,0.0,0.0}
\definecolor{mygreen}{rgb}{0.0,0.7,0.0}
\definecolor{myblue}{rgb}{0.0,0.0,0.7}

\definecolor{redshade}{rgb}{0.9,0.5,0.5}
\definecolor{greenshade}{rgb}{0.5,0.9,0.5}

\usepackage[unicode,colorlinks=true,linkcolor=mylinkcolor,citecolor=mycitecolor]{hyperref}
\newcommand{\refref}[2]{\hyperref[#2]{#1~\ref*{#2}}}
\newcommand{\eqnref}[1]{\hyperref[#1]{(\ref*{#1})}}

\newcommand{\tos}{\!\!\to\!\!}

\usepackage{framed}

\newcommand{\cequiv}{\simeq}

\makeatletter
\newcommand\xleftrightarrow[2][]{%
  \ext@arrow 9999{\longleftrightarrowfill@}{#1}{#2}}
\newcommand\longleftrightarrowfill@{%
  \arrowfill@\leftarrow\relbar\rightarrow}
\makeatother

\newcommand{\bsquare}{\textrm{\ding{114}}}

% % % % % % % % % % % % % % % % % % % % % % % % % % % % % %

\newtheoremstyle{myplain}
  {\topsep}   % ABOVESPACE
  {\topsep}   % BELOWSPACE
  {\itshape}  % BODYFONT
  {0pt}       % INDENT (empty value is the same as 0pt)
  {\bfseries} % HEADFONT
  {.}         % HEADPUNCT
  {5pt plus 1pt minus 1pt} % HEADSPACE
  {\thmnumber{#2}. \thmname{#1}\thmnote{ (#3)}}   % CUSTOM-HEAD-SPEC

\newtheoremstyle{myplainnameless}
  {\topsep}   % ABOVESPACE
  {\topsep}   % BELOWSPACE
  {\normalfont}  % BODYFONT
  {0pt}       % INDENT (empty value is the same as 0pt)
  {\bfseries} % HEADFONT
  {.}         % HEADPUNCT
  {5pt plus 1pt minus 1pt} % HEADSPACE
  {\thmnumber{#2}}   % CUSTOM-HEAD-SPEC 

\newtheoremstyle{sectionexercise}
  {\topsep}   % ABOVESPACE
  {\topsep}   % BELOWSPACE
  {\normalfont}  % BODYFONT
  {0pt}       % INDENT (empty value is the same as 0pt)
  {\bfseries} % HEADFONT
  {.}         % HEADPUNCT
  {5pt plus 1pt minus 1pt} % HEADSPACE
  {Ejercicio \thmnumber{#2}\thmnote{ (#3)}}   % CUSTOM-HEAD-SPEC

\newtheoremstyle{mydefinition}
  {\topsep}   % ABOVESPACE
  {\topsep}   % BELOWSPACE
  {\normalfont}  % BODYFONT
  {0pt}       % INDENT (empty value is the same as 0pt)
  {\bfseries} % HEADFONT
  {.}         % HEADPUNCT
  {5pt plus 1pt minus 1pt} % HEADSPACE
  {\thmnumber{#2}. \thmname{#1}\thmnote{ (#3)}}   % CUSTOM-HEAD-SPEC

% EN ESPAÑOL

\newtheorem*{hecho*}{Hecho}
\newtheorem*{corolario*}{Corolario}
\newtheorem*{teorema*}{Teorema}
\newtheorem*{conjetura*}{Conjetura}
\newtheorem*{proyecto*}{Proyecto}
\newtheorem*{observacion*}{Observación}

\newtheorem*{lema*}{Lema}
\newtheorem*{resultado-clave*}{Resultado clave}
\newtheorem*{proposicion*}{Proposición}

\theoremstyle{definition}
\newtheorem*{ejercicio*}{Ejercicio}
\newtheorem*{definicion*}{Definición}
\newtheorem*{comentario*}{Comentario}
\newtheorem*{definicion-alternativa*}{Definición alternativa}
\newtheorem*{ejemploxs}{Ejemplo}
\newenvironment{ejemplo*}
  {\pushQED{\qed}\renewcommand{\qedsymbol}{\examplesymbol}\ejemploxs}
  {\popQED\endejemploxs}

\theoremstyle{myplain}
\newtheorem{proposicion}{Proposición}[section]

\newtheorem{proyecto}[proposicion]{Proyecto}
\newtheorem{teorema}[proposicion]{Teorema}
\newtheorem{corolario}[proposicion]{Corolario}
\newtheorem{hecho}[proposicion]{Hecho}
\newtheorem{lema}[proposicion]{Lema}

\newtheorem{observacion}[proposicion]{Observación}

\newenvironment{observacionejerc}
    {\pushQED{\qed}\renewcommand{\qedsymbol}{$\square$}\csname inner@observacionejerc\endcsname}
    {\popQED\csname endinner@observacionejerc\endcsname}
\newtheorem{inner@observacionejerc}[proposicion]{Observación}

\newenvironment{proposicionejerc}
    {\pushQED{\qed}\renewcommand{\qedsymbol}{$\square$}\csname inner@proposicionejerc\endcsname}
    {\popQED\csname endinner@proposicionejerc\endcsname}
\newtheorem{inner@proposicionejerc}[proposicion]{Proposicion}

\newenvironment{lemaejerc}
    {\pushQED{\qed}\renewcommand{\qedsymbol}{$\square$}\csname inner@lemaejerc\endcsname}
    {\popQED\csname endinner@lemaejerc\endcsname}
\newtheorem{inner@lemaejerc}[proposicion]{Lema}

\newtheorem{calculo}[proposicion]{Cálculo}

\theoremstyle{myplainnameless}
\newtheorem{nameless}[proposicion]{}

\theoremstyle{mydefinition}
\newtheorem{comentario}[proposicion]{Comentario}
\newtheorem{comentarioast}[proposicion]{Comentario ($\clubsuit$)}
\newtheorem{construccion}[proposicion]{Construcción}
\newtheorem{aplicacion}[proposicion]{Aplicación}
\newtheorem{definicion}[proposicion]{Definición}
\newtheorem{definicion-alternativa}[proposicion]{Definición alternativa}
\newtheorem{notacion}[proposicion]{Notación}
\newtheorem{advertencia}[proposicion]{Advertencia}
\newtheorem{digresion}[proposicion]{Digresión}
\newtheorem{ejemplox}[proposicion]{Ejemplo}
\newenvironment{ejemplo}
  {\pushQED{\qed}\renewcommand{\qedsymbol}{\examplesymbol}\ejemplox}
  {\popQED\endejemplox}
\newtheorem{contraejemplox}[proposicion]{Contraejemplo}
\newenvironment{contraejemplo}
  {\pushQED{\qed}\renewcommand{\qedsymbol}{\examplesymbol}\contraejemplox}
  {\popQED\endcontraejemplox}
\newtheorem{noejemplox}[proposicion]{No-ejemplo}
\newenvironment{noejemplo}
  {\pushQED{\qed}\renewcommand{\qedsymbol}{\examplesymbol}\noejemplox}
  {\popQED\endnoejemplox}
 
\newtheorem{ejemploastx}[proposicion]{Ejemplo ($\clubsuit$)}
\newenvironment{ejemploast}
  {\pushQED{\qed}\renewcommand{\qedsymbol}{\examplesymbol}\ejemploastx}
  {\popQED\endejemploastx}

\ifdefined\exercisespersection
  \theoremstyle{sectionexercise}
  \newtheorem{ejercicio}{}[section]
  \theoremstyle{mydefinition}
\else
  \ifdefined\exercisesglobal
    \theoremstyle{sectionexercise}
    \newtheorem{ejercicio}{}
    \theoremstyle{mydefinition}
  \else
    \ifdefined\exercisespersection
      \newtheorem{ejercicio}[proposicion]{Ejercicio}
    \fi
  \fi
\fi

% % % % % % % % % % % % % % % % % % % % % % % % % % % % % %

\theoremstyle{myplain}
\newtheorem{proposition}{Proposition}[section]
\newtheorem*{fact*}{Fact}
\newtheorem*{proposition*}{Proposition}
\newtheorem{lemma}[proposition]{Lemma}
\newtheorem*{lemma*}{Lemma}

\newtheorem{exercise}{Exercise}
\newtheorem*{hint}{Hint}

\newtheorem{theorem}[proposition]{Theorem}
\newtheorem{conjecture}[proposition]{Conjecture}
\newtheorem*{theorem*}{Theorem}
\newtheorem{corollary}[proposition]{Corollary}
\newtheorem{fact}[proposition]{Fact}
\newtheorem*{claim}{Claim}
\newtheorem{definition-theorem}[proposition]{Definition-theorem}

\theoremstyle{mydefinition}
\newtheorem{examplex}[proposition]{Example}
\newenvironment{example}
  {\pushQED{\qed}\renewcommand{\qedsymbol}{\examplesymbol}\examplex}
  {\popQED\endexamplex}

\newtheorem*{examplexx}{Example}
\newenvironment{example*}
  {\pushQED{\qed}\renewcommand{\qedsymbol}{\examplesymbol}\examplexx}
  {\popQED\endexamplexx}

\newtheorem{definition}[proposition]{Definition}
\newtheorem*{definition*}{Definition}
\newtheorem{wrong-definition}[proposition]{Wrong definition}
\newtheorem{remark}[proposition]{Remark}

\makeatletter
\newcommand{\xRightarrow}[2][]{\ext@arrow 0359\Rightarrowfill@{#1}{#2}}
\makeatother

% % % % % % % % % % % % % % % % % % % % % % % % % % % % % %

\newcommand{\Et}{\mathop{\text{\rm Ét}}}

\newcommand{\categ}[1]{\text{\bf #1}}
\newcommand{\vcateg}{\mathcal}
\newcommand{\bone}{{\boldsymbol 1}}
\newcommand{\bDelta}{{\boldsymbol\Delta}}
\newcommand{\bR}{{\mathbf{R}}}

\newcommand{\univ}{\mathfrak}

\newcommand{\TODO}{\colorbox{red}{\textbf{*** TODO ***}}}
\newcommand{\proofreadme}{\colorbox{red}{\textbf{*** NEEDS PROOFREADING ***}}}

\makeatletter
\def\iddots{\mathinner{\mkern1mu\raise\p@
\vbox{\kern7\p@\hbox{.}}\mkern2mu
\raise4\p@\hbox{.}\mkern2mu\raise7\p@\hbox{.}\mkern1mu}}
\makeatother

\newcommand{\ssincl}{\reflectbox{\rotatebox[origin=c]{45}{$\subseteq$}}}
\newcommand{\vsupseteq}{\reflectbox{\rotatebox[origin=c]{-90}{$\supseteq$}}}
\newcommand{\vin}{\reflectbox{\rotatebox[origin=c]{90}{$\in$}}}

\newcommand{\Ga}{\mathbb{G}_\mathrm{a}}
\newcommand{\Gm}{\mathbb{G}_\mathrm{m}}

\renewcommand{\U}{\UUU}

\DeclareRobustCommand{\Stirling}{\genfrac\{\}{0pt}{}}
\DeclareRobustCommand{\stirling}{\genfrac[]{0pt}{}}

% % % % % % % % % % % % % % % % % % % % % % % % % % % % % %
% tikz

\usepackage{tikz-cd}
\usetikzlibrary{babel}
\usetikzlibrary{decorations.pathmorphing}
\usetikzlibrary{arrows}
\usetikzlibrary{calc}
\usetikzlibrary{fit}
\usetikzlibrary{hobby}

% % % % % % % % % % % % % % % % % % % % % % % % % % % % % %
% Banners

\newcommand\mybannerext[3]{{\normalfont\sffamily\bfseries\large\noindent #1

\noindent #2

\noindent #3

}\noindent\rule{\textwidth}{1.25pt}

\vspace{1em}}

\newcommand\mybanner[2]{{\normalfont\sffamily\bfseries\large\noindent #1

\noindent #2

}\noindent\rule{\textwidth}{1.25pt}

\vspace{1em}}

\renewcommand{\O}{\mathcal{O}}


\usepackage{fullpage}

\theoremstyle{definition}
\newtheorem{ejerc}{Ejercicio}
\newenvironment{solucion}{\begin{proof}[Solución]}{\end{proof}}

\def\mystrut(#1,#2){\vrule height #1 depth #2 width 0pt}

\newcolumntype{f}[1]{%
>{\mystrut(3ex,2ex)\centering}%
p{#1}%
<{}}

\begin{document}

\mybanner{Álgebra computacional. Tarea 3. Fecha límite: 03/05/2019}{Universidad de El Salvador, ciclo impar 2019}

\section*{Más bases de Gröbner}

\begin{ejerc}
  Consideremos el ideal
  $$I = (x^3 y - z, \, y^2 - z - 1, \, x^2 + 1) \subset k[x,y,z].$$

  \begin{enumerate}
  \item[1)] Usando la computadora, encuentre una base de Gröbner para $I$
    respecto al orden lexicográfico. Encuentre la base monomial correspondiente
    para $k[x,y,z]/I$ como un espacio vectorial sobre $k$.

  \item[2)] La misma pregunta para el orden graduado lexicográfico.

  \item[3)] Compile las tablas de multiplicación en $k[x,y,z]/I$ respecto a
    estas dos bases.
  \end{enumerate}

  \ifdefined\solutions\begin{solucion}
    Respecto al orden lexicográfico, se obtiene
    $$G = \{ x+yz+y, \, y^2-z-1, \, z^2+z+1 \}.$$
    Entonces,
    $$(LT (I)) = (LT (G)) = (x, y^2, z^2).$$
    Esto nos da la base monomial
    $$1, \, y, \, z, \, yz.$$

    La tabla de multiplicación correspondiente viene dada por
    \begin{center}
\begin{tabular}{f{1.25cm}|f{1.25cm}|f{1.25cm}|f{1.25cm}|f{1.25cm}}
$\cdot$ & $1$ & $y$ & $z$ & $yz$ \tabularnewline
\hline
$1$ & $1$ & $y$ & $z$ & $yz$ \tabularnewline
\hline
$y$ & $y$ & $z+1$ & $yz$ & $-1$ \tabularnewline
\hline
$z$ & $z$ & $yz$ & $-z-1$ & $-yz - y$ \tabularnewline
\hline
$yz$ & $yz$ & $-1$ & $-yz - y$ & $-z$ \tabularnewline
\end{tabular}
    \end{center}
    (Todo esto se calcula tomando los restos de división $\overline{f}^G$.)

    Ahora respecto al orden graduado lexicográfico, se tiene
    $$G = \{ x^2+1, \, xy+z, \, xz-y, \, y^2-z-1, \, yz+x+y, \, z^2+z+1 \}.$$
    Entonces,
    $$(LT (I)) = (LT (G)) = (x^2, \, xy, \, xz, \, y^2, \, yz, \, z^2).$$
    La base monomial correspondiente es
    $$1, \, x, \, y, \, z,$$
    y la tabla de multiplicación viene dada por

    \begin{center}
\begin{tabular}{f{1.25cm}|f{1.25cm}|f{1.25cm}|f{1.25cm}|f{1.25cm}}
$\cdot$ & $1$ & $x$ & $y$ & $z$ \tabularnewline
\hline
$1$ & $1$ & $x$ & $y$ & $z$ \tabularnewline
\hline
$x$ & $x$ & $-1$ & $-z$ & $y$ \tabularnewline
\hline
$y$ & $y$ & $-z$ & $z+1$ & $-x-y$ \tabularnewline
\hline
$z$ & $z$ & $y$ & $-x - y$ & $-z-1$ \tabularnewline
\end{tabular}
    \end{center}

    En particular, se ve que diferentes órdenes monomiales nos dan diferentes
    bases monomiales.
  \end{solucion}\fi
\end{ejerc}

\begin{ejerc}
  Demuestre que las siguientes condiciones son equivalentes:

  \begin{enumerate}
  \item[a)] $\dim_k (k [x_1,\ldots,x_n]/I) < \infty$;

  \item[b)] $\#\{ x^\alpha \mid x^\alpha \notin (LT (I)) \} < \infty$;

  \item[c)] para todo $i = 1,\ldots,n$ existe $\alpha_i \ge 0$ tal que
    $x_i^{\alpha_i} \in (LT (I))$;

  \item[d)] si $G$ es una base de Gröbner para $I$, entonces para todo
    $i = 1,\ldots,n$ existe $\alpha_i \ge 0$ tal que $x_i^{\alpha_i} = LM (g)$
    para algún $g\in G$.
  \end{enumerate}

  \ifdefined\solutions\begin{solucion}
    Los monomios $x^\alpha \notin (LT (I))$ forman una base de
    $k [x_1,\ldots,x_n]/I$ como un espacio vectorial sobre $k$, así que a) y b)
    son equivalentes. Luego, si para todo $i$ existe $\alpha_i$ tal que
    $x_i^{\alpha_i} \in (LT (I))$, entonces el número de monomios tales que
    $x^\beta \notin (LT (I))$ es finito: si
    $x^\beta = x_1^{\beta_1}\cdots x_n^{\beta_n} \notin (LT (I))$, entonces
    necesariamente $\beta_i < \alpha_i$. Esto demuestra la implicación
    {c)$\Rightarrow$b)}. La otra implicación {b)$\Rightarrow$c)} es evidente.

    En fin, relacionemos todo con la condición d). Para una base de Gröbner
    $G = \{ g_1, \ldots, g_s \}$ tenemos
    $$(LT (I)) = (LM (g_1), \ldots, LM (g_s)),$$
    así que
    \[ x_i^{\beta_i} \in (LT (I)) \iff
       LM (g) \mid x_i^{\beta_i} \text{ para algún }g\in G \iff
       x_i^{\alpha_i} = LM (g)\text{ para }g\in G, \, \alpha_i \le \beta_i. \]
    Esto establece la equivalencia entre c) y d).
\end{solucion}\fi
\end{ejerc}

\begin{ejerc}
  Demuestre que $\dim_k (k [x_1,\ldots,x_n]/I) < \infty$ si y solamente si
  $I \cap k [x_i] \ne 0$ para todo $i = 1,\ldots,n$.

  \noindent \emph{Sugerencia: use el ejercicio anterior y el orden lexicográfico
    con $x_j \succ x_i$ para todo $j \ne i$.}

  \ifdefined\solutions\begin{solucion}
    Recordemos que una base de Gröbner para $I \cap k [x_i]$ respecto al orden
    lexicográfico puede ser obtienida a partir de una base de Gröbner para $I$
    respecto al orden lexicográfico con $x_j \succ x_i$ para $j \ne i$. La
    condición $I \cap k[x_i] \ne 0$ quiere decir que en tal base de Gröbner para
    $I$ existe un polinomio $g \in G \cap k [x_i]$. En particular,
    $LM (g) = x_i^{\alpha_i}$. Si esto se cumple para todo $i$, entonces
    $\dim_k (k [x_1,\ldots,x_n]/I) < \infty$ por la condición d) del ejercicio
    anterior.

    Viceversa, si $\dim_k (k [x_1,\ldots,x_n]/I) < \infty$, entonces para todo
    $i$ existe $g \in G$ tal que $LM (g) = x_i^{\alpha_i}$, pero puesto que el
    orden es lexicográfico con $x_j \succ x_i$ para $j \ne i$, esto implica que
    $g \in I \cap k [x_i]$.
  \end{solucion}\fi
\end{ejerc}

\begin{ejerc}
  Consideremos el ideal
  $$I = (x^3 y - z, \, y^2 - z - 1, \, x^2 + 1) \subset k[x,y,z].$$
  Usando Macaulay2, encuentre generadores de los ideales principales
  $$I \cap k[x], \quad I \cap k[y], \quad I \cap k[z].$$

  \ifdefined\solutions\begin{solucion}
    Los cálculos en Macaulay2 nos dan lo siguiente.

    \begin{framed}\small
\begin{verbatim}
i1 : R = QQ[x,y,z];

i2 : I = ideal (x^3*y - z, y^2 - z - 1, x^2 + 1);

o2 : Ideal of R

i3 : eliminate ({y,z},I)

            2
o3 = ideal(x  + 1)

o3 : Ideal of R

i4 : eliminate ({x,z},I)

            4    2
o4 = ideal(y  - y  + 1)

o4 : Ideal of R

i5 : eliminate ({x,y},I)

            2
o5 = ideal(z  + z + 1)

o5 : Ideal of R
\end{verbatim}
    \end{framed}

    Entonces,
    \[ I \cap k [x] =
       (x^2+1), \quad I \cap k [y] =
       (y^4 - y^2 + 1), \quad I \cap k [z] =
       (z^2 + z  + 1). \qedhere \]
  \end{solucion}\fi
\end{ejerc}

\begin{ejerc}
  Demuestre que en $k [x_1,\ldots,x_n]$ se cumple
  $$(f) \cap (g) = (\mcm (f,g)).$$

  \ifdefined\solutions\begin{solucion}
    El anillo de polinomios $k [x_1,\ldots,x_n]$ es un dominio de ideales
    principales, así que siempre existe
    $$\mcm (f,g) = \prod_p p^{\max (v_p (f), v_p (g))},$$
    donde el producto es sobre todos los polinomios irreducibles
    $p \in k [x_1,\ldots,x_n]$ no asociados entre sí.  Puesto que
    $f \mid \mcm (f,g)$ y $g \mid \mcm (f,g)$, tenemos
    $$(\mcm (f,g)) \subseteq (f), \quad (\mcm (f,g)) \subseteq (g).$$
    Ahora si tenemos $I \subseteq (f) \cap (g)$, entonces todo polinomio
    $h \in I$ cumple $f \mid h$ y $g \mid h$, y luego $\mcm (f,g) \mid h$. Esto
    demuestra la inclusión $I \subseteq (\mcd (f,g))$. Entonces, $(\mcd (f,g))$
    es el máximo ideal contenido en $(f)$ y $(g)$, así que
    \[ (\mcd (f,g)) = (f) \cap (g). \qedhere \]
  \end{solucion}\fi
\end{ejerc}

El orden lexicográfico que hemos usado en el teorema de eliminación no es muy
bueno en práctica. En los siguientes ejercicios vamos a investigar otro orden
que también funciona y suele ser mejor en los cálculos.

\begin{ejerc}
  Para $1 \le \ell \le n$ consideremos la relación sobre los monomios en
  $k [x_1,\ldots,x_n]$
  $$x^\alpha \prec_\ell x^\beta \iff \left\{\begin{array}{c}
\alpha_1+\cdots+\alpha_\ell < \beta_1+\cdots+\beta_\ell \\
\text{ o bien }\\
\alpha_1+\cdots+\alpha_\ell = \beta_1+\cdots+\beta_\ell \text{ y } \alpha\prec_{grevlex} \beta \\
\end{array}\right\}$$

  \begin{enumerate}
  \item[1)] Demuestre que $\prec_\ell$ es un orden monomial.

  \item[2)] Demuestre que para un ideal $I \subseteq k [x_1,\ldots,x_n]$, si $G$
    es una base de Gröbner respecto al orden $\prec_\ell$, entonces,
    $G \cap k [x_{\ell+1},\ldots,x_n]$ es una base de Gröbner para
    $I \cap k [x_{\ell+1},\ldots,x_n]$ respecto al orden grevlex.
  \end{enumerate}

  Este orden monomial puede ser especificado en Macaulay2 como
  \begin{center}
    \texttt{MonomialOrder=>Eliminate $\ell$}
  \end{center}

  \ifdefined\solutions\begin{solucion}
    La parte 1) es una verificación directa. En la parte 2), basta revisar
    nuestra prueba para la eliminación ${I \cap k [x_{\ell+1},\ldots,x_n]}$
    respecto al orden lexicográfico. Allí lo único que usamos del orden monomial
    es la siguiente propiedad: si las variables $x_1,\ldots,x_\ell$ no aparecen
    en $LT (g)$, entonces estas tampoco aparecen en otros términos de $g$. El
    orden $\prec_\ell$ cumple la misma propiedad, así que el mismo argumento
    funciona palabra por palabra.
  \end{solucion}\fi
\end{ejerc}

\begin{ejerc}
  Consideremos el ideal
  $$I = (t^2 + x^2 + y^2 + z^2, \, t^2 + 2 x^2 - xy - z^2, \, t + y^3 - z^3) \subset \QQ [t,x,y,z].$$

  \begin{enumerate}
  \item[1)] Calcule en Macaulay2 la base de Gröbner reducida de $I$ respecto al
    orden lexicográfico. Encuentre la base de Gröbner correspondiente para
    $I \cap \QQ [x,y,z]$.

  \item[2)] Haga el mismo cálculo para el orden $\prec_1$ del ejercicio anterior
    (es decir, $\prec_\ell$ con $\ell = 1$). Encuentre la base de Gröbner
    correspondiente para $I \cap \QQ [x,y,z]$.
  \end{enumerate}

  \ifdefined\solutions\begin{solucion}
    La base de Gröbner respecto al orden lexicográfico es bastante grande:

    \begin{framed}\small
\begin{verbatim}
i1 : R = QQ[t,x,y,z, MonomialOrder=>Lex];

i2 : I = ideal (t^2+x^2+y^2+z^2, t^2+2*x^2-x*y-z^2, t+y^3-z^3);

o2 : Ideal of R

i3 : groebnerBasis I

o3 = | y12-4y9z3+5y8+6y6z6+6y6z2-10y5z3+5y4-4y3z9-12y3z5+5y2z6+13y2z2+z12+6z8+9z4
     -----------------------------------------------------------------------------
     xz6+3xz2-y11+4y8z3-5y7-5y5z6-3y5z2+10y4z3-5y3+2y2z9+6y2z5-3yz6-7yz2
     -----------------------------------------------------------------------------
     xy+y6-2y3z3+2y2+z6+3z2 x2+y6-2y3z3+y2+z6+z2 t+y3-z3 |

             1       5
o3 : Matrix R  <--- R
\end{verbatim}
    \end{framed}

    Aquí solo el último polinomio $t+y^3-z^3$ contiene $t$. Hagamos ahora el
    mismo cálculo para el orden $\prec_1$.

    \begin{framed}\small
\begin{verbatim}
i1 : R = QQ[t,x,y,z, MonomialOrder=>Eliminate 1];

i2 : I = ideal (t^2+x^2+y^2+z^2, t^2+2*x^2-x*y-z^2, t+y^3-z^3);

o2 : Ideal of R

i3 : groebnerBasis I

o3 = | x2-xy-y2-2z2 y6-2y3z3+z6+xy+2y2+3z2 t+y3-z3 |

             1       3
o6 : Matrix R  <--- R
\end{verbatim}
    \end{framed}

    Esto nos da una base de $I \cap k[x,y,z]$ respecto al orden grevlex
    \[ G = (x^2-xy-y^2-2z^2, \, y^6-2y^3z^3+z^6+xy+2y^2+3z^2). \qedhere \]
  \end{solucion}\fi
\end{ejerc}

\ifdefined\solutions\pagebreak\fi

\section*{Conjuntos algebraicos}

\begin{ejerc}
  Describa la unión de los $n$ ejes de coordenadas en $\AA^n (k)$ como un
  conjunto algebraico.

  \ifdefined\solutions\begin{solucion}
    El $i$-ésimo eje de coordenadas es el conjunto
    \[ \{ (a_1,\ldots,a_n) \mid a_j = 0 \text{ para }j \ne i \} =
       \bigcap_{j\ne i} \mathbf{V} (x_j) =
       \mathbf{V} (x_1, \ldots, \widehat{x_i}, \ldots, x_n). \]
    La unión de los ejes es
    \[ \bigcup_{1 \le i \le n} \mathbf{V} (x_1, \ldots, \widehat{x_i}, \ldots, x_n) =
       \mathbf{V} \Bigl(\bigcap_i (x_1, \ldots, \widehat{x_i}, \ldots, x_n)\Bigr). \]
    La intersección de los ideales monomiales
    $(x_1, \ldots, \widehat{x_i}, \ldots, x_n)$ fue calculada en la primera
    tarea. Tenemos entonces
    \[ \mathbf{V} (x_i x_j \mid 1 \le i < j \le n). \qedhere \]
  \end{solucion}\fi
\end{ejerc}

\begin{ejerc}
  Demuestre que si $k$ es un cuerpo infinito, entonces la topología de Zariski
  sobre $\AA^2 (k)$ es más fina que la topología producto sobre
  $\AA^1 (k) \times \AA^1 (k)$.

  \ifdefined\solutions\begin{solucion}
    Primero notamos que si $k$ es un cuerpo finito, entonces la topología sobre
    $\AA^1 (k)$, $\AA^1 (k) \times \AA^1 (k)$ y $\AA^2 (k)$ es discreta, así que
    la hipótesis de que $k$ es infinito es importante.

    Los conjuntos cerrados en $\AA^1 (k)$ son ceros de polinomios $f \in k [x]$,
    así que son finitos o coinciden con todo $\AA^1 (k)$. Si $k$ es infinito,
    esto implica que para cualesquiera $U_1, U_2 \subset \AA^1 (k)$ abiertos no
    vacíos se tiene $U_1 \cap U_2 \ne \emptyset$, y en particular $\AA^1 (k)$ no
    es Hausdorff.

    Para dos polinomios $f \in k [x_1]$ y $g \in k [x_2]$ podemos identificar
    \[ \mathbf{V} (f) \times \mathbf{V} (g) =
       \mathbf{V} (f) \cap \mathbf{V} (g) =
       \mathbf{V} (f,g). \]
    Entonces, todo subconjunto cerrado de $\AA^1 (k) \times \AA^1 (k)$ es
    cerrado en $\AA^2 (k)$.

    Bajo la hipótesis de que $k$ es infinito, podemos encontrar un subconjunto
    cerrado de $\AA^2 (k)$ que no es cerrado en la topología producto de
    $\AA^1 (k)\times \AA^1 (k)$: bastaría considerar la diagonal
    $$\Delta = \mathbf{V} (x_1 - x_2) =  \{ (a,a) \mid a \in k \} \subset \AA^2 (k).$$
    Ahora se puede usar el siguiente resultado: un espacio $X$ es Hausdorff si y
    solo si la diagonal $\Delta \in X\times X$ es cerrada.
  \end{solucion}\fi
\end{ejerc}

\begin{ejerc}
  Demuestre que si $k$ es un cuerpo infinito, entonces $I (\AA^n (k)) = 0$.
  ¿Por qué esto es falso para cuerpos finitos?

  \ifdefined\solutions\begin{solucion}
    Si $n = 1$, entonces para un polinomio en una variable
    $f = \sum_i a_i\,x^i \in I (\AA^1 (k))$, dado que $f$ tiene un número
    infinito de raíces, se tiene $f = 0$ (todo polinomio no nulo tiene
    $\le \deg f$ raíces).

    Para un polinomio en $n$ variables
    $$f = \sum_{i_1,\ldots,i_n} a_{i_1,\ldots,i_n}\,x_1^{i_1}\cdots x_n^{i_n} \in I (\AA^n (k)),$$
    para todo $i = 1, \ldots, n$ se puede considerar el polinomio
    $$f_i \dfn f (1, \ldots, x_i, \ldots, 1) \in k [x_i].$$
    Se tiene $f_i \in I (\AA^1 (k))$, y luego $f_i = 0$. Esto significa que todo
    monomio donde aparece $x_i$ tiene coeficiente $0$. Aplicando este argumento
    a $i = 1,\ldots,n$, podemos concluir que $f$ es un polinomio constante, pero
    puesto que $f \in I (\AA^n (k))$, se tiene $f = 0$.

    Si $k = \FF_q$ es un cuerpo finito, entonces para todo $c \in \FF_q^\times$
    se tiene $c^{q-1} = 1$. Entonces, $x^q - x \in I (\AA^1 (\FF_q))$.
  \end{solucion}\fi
\end{ejerc}

\begin{ejerc}
  Sea $k$ un cuerpo algebraicamente cerrado.

  \begin{enumerate}
  \item[a)] Para $\fchar k \ne 2$ consideremos los siguientes conjuntos
    algebraicos en $\AA^2 (k)$:
    $$Z_1 \dfn \mathbf{V} (u\,(t - 1) - 1), \quad Z_2 \dfn \mathbf{V} (y^2 - x^2 (x + 1)).$$
    Demuestre que el morfismo
    $$Z_1 \to Z_2, \quad (t, u) \mapsto (t^2 - 1, t\,(t^2 - 1))$$
    es biyectivo, pero no es un isomorfismo. Demuestre que en general,
    $Z_1 \not\isom Z_2$.

  \item[b)] Demuestre que el morfismo
    \[ \AA^1 (k) \to \mathbf{V} (y^2 - x^3) \subset \AA^2 (k), \quad
       t \mapsto (t^2, t^3) \]
    es biyectivo, pero no es un isomorfismo. Demuestre que en general,
    $\AA^1 (k) \not\isom \mathbf{V} (y^2 - x^3)$.
  \end{enumerate}

  \noindent Sugerencia: demuestre que $\Gamma (Z_1) \not\isom \Gamma (Z_2)$ y
  $\Gamma (\mathbf{V} (y^2 - x^3)) \not\isom k[t]$.

  \ifdefined\solutions\begin{solucion}
    Notamos que $Z_1$ es una hipérbola, mientras que $Z_2$ es una
    \term{cúbica nodal}; en el segundo caso, $\mathbf{V} (y^2 - x^3)$ es una
    \term{cúbica cuspidal}.

    \begin{center}
      \begin{tikzpicture}[x=1cm, y=1cm]

        \foreach \i [count=\j from 1] in {0, ..., 99}
        \draw ({-1 + 1/50*\i}, {sqrt ((-1 + 1/50*\i)^3 + (-1 + 1/50*\i)^2)}) -- ({-1 + 1/50*\j}, {sqrt ((-1 + 1/50*\j)^3 + (-1 + 1/50*\j)^2)});

        \foreach \i [count=\j from 1] in {0, ..., 99}
        \draw ({-1 + 1/50*\i}, {-sqrt ((-1 + 1/50*\i)^3 + (-1 + 1/50*\i)^2)}) -- ({-1 + 1/50*\j}, {-sqrt ((-1 + 1/50*\j)^3 + (-1 + 1/50*\j)^2)});

        \draw[->] (-2,0) -- (2,0) node[right] {$x$};
        \draw[->] (0,-2) -- (0,2) node[above] {$y$};
      \end{tikzpicture}
      \quad\quad \begin{tikzpicture}[x=0.666cm, y=0.666cm]

        \foreach \i [count=\j from 1] in {0, ..., 85}
        \draw ({1/50*\i}, {sqrt ((1/50*\i)^3)}) -- ({1/50*\j}, {sqrt ((1/50*\j)^3)});

        \foreach \i [count=\j from 1] in {0, ..., 85}
        \draw ({1/50*\i}, {-sqrt ((1/50*\i)^3)}) -- ({1/50*\j}, {-sqrt ((1/50*\j)^3)});

        \draw[->] (-3,0) -- (3,0) node[right] {$x$};
        \draw[->] (0,-3) -- (0,3) node[above] {$y$};
      \end{tikzpicture}
    \end{center}
    En ambos casos no habrá isomorfismo precisamente por la singularidad en el
    origen $(0,0)$.

    \begin{enumerate}
    \item[(a)] Consideremos la aplicación
      $$\phi\colon (t, u) \mapsto (t^2 - 1, t\,(t^2 - 1)).$$
      Notamos que
      $$(t\,(t^2 - 1))^2 - (t^2 - 1)^2\,((t^2 - 1) + 1) = 0,$$
      y entonces la imagen de esta aplicación está en $Z_2$. Para ver que la
      aplicación es inyectiva, asumamos que
      $$(t_1^2 - 1, t_1\,(t_1^2 - 1)) = (t_2^2 - 1, t_2\,(t_2^2 - 1)).$$
      Entonces, $t_1^2 = t_2^2$ y $t_1\,(t_1^2 - 1) = t_2\,(t_1^2 - 1)$. Notamos
      que $t \ne 1$ para $(u,t) \in Z_1$.

      \begin{itemize}
      \item Si $t_1 \ne -1$, entonces la ecuación
        $t_1\,(t_1^2 - 1) = t_2\,(t_1^2 - 1)$ implica $t_1 = t_2$. Además,
        $u_1\,(t_1-1) = u_2\,(t_2-1)$ implica $u_1 = u_2$.

      \item El otro caso es $t_1 = t_2 = -1$ y $u_1 = u_2 = -\frac{1}{2}$
        (usando la hipótesis de que $\fchar k \ne 2$).
      \end{itemize}

      Para la sobreyectividad, consideremos un punto $(x,y)$ tal que
      $y^2 = x^2\,(x+1)$. Si $x = 0$, entonces $y = 0$. Tenemos
      $$\phi (-1,-1/2) = (0,0).$$
      Ahora asumamos que $x \ne 0$. Sea $t$ una de las dos soluciones de la
      ecuación $t^2 = x + 1$. Se tiene
      $$(t\,(t^2-1))^2 = t^2\,(t^2-1)^2 = (x+1)\,x^2 = y^2.$$
      Luego,
      $$t\,(t^2-1) = \pm y.$$
      Sea $t$ la raíz de $t^2 = x+1$ que nos da
      $$t\,(t^2-1) = y.$$
      Notamos que $t \ne 1$, y se puede poner
      $$u \dfn \frac{1}{t-1}.$$
      Ahora
      $$\phi (t,u) = (t^2 - 1, t\,(t^2-1)) = (x, y).$$
      Esto establece la sobreyectividad.

      Para ver que $\phi$ no es un isomorfismo, vamos a ver que en general, los
      anillos de coordenadas
      $$k [t,u] / (u(t-1)-1), \quad k [x,y] / (y^2 - x^2\,(x+1))$$
      no son isomorfos. Para esto nos servirá la noción de
      \term{anillo integralmente cerrado} que vimos el año pasado en el curso de
      álgebra conmutativa.

      Notamos que
      \begin{align*}
        k [t,u] / (u(t-1)-1) & \xrightarrow{\isom} k [t,u]/(tu-1) \isom k [t,t^{-1}],\\
        \overline{t} & \mapsto \overline{t}+1,\\
        \overline{u} & \mapsto \overline{u}.
      \end{align*}
      Se trata de la localización del anillo de polinomios $k [t]$ en $t$.
      El anillo $k [t]$ es integralmente cerrado, puesto que es un DFU. Luego,
      $k [t,t^{-1}]$ es también integralmente cerrado, dado que la localización
      preserva la cerradura integral.

      Por otra parte, el anillo $k [x,y] / (y^2 - x^2\,(x+1))$ no es
      integralmente cerrado. Para verlo, consideremos el homomorfismo de
      $k$-álgebras inducido por $\phi$:
      \begin{align*}
        \phi^*\colon k [x,y] / (y^2 - x^2\,(x+1)) & \to k [t,u] / (u(t-1)-1) \xrightarrow{\isom} k [t,u]/(tu-1) \isom k [t,t^{-1}],\\
        \overline{x} & \mapsto \overline{t}^2 - 1 \mapsto \overline{t}^2 + 2\,\overline{t},\\
        \overline{y} & \mapsto \overline{t}\,(\overline{t}^2 - 1) \mapsto (\overline{t}+1)\,((\overline{t}+1)^2-1) = \overline{t}^3 + 3\,\overline{t}^2 + 2\,\overline{t}.
      \end{align*}
      Este es inyectivo. Sin embargo, $\phi^*$ no es sobreyectivo: tenemos por
      ejemplo $t \notin \phi^*$. Entonces, $\phi^*$ no es un isomorfismo, y
      $\phi$ tampoco lo es. Sin embargo, nuestro objetivo es ver que entre los
      dos conjuntos algebraicos no existe \emph{ningún} isomorfismo.

      Notamos que en el cuerpo de fracciones se tiene
      $$\frac{t^3 + 3\,t^2 + 2\,t}{t^2 + 2t} - 1 = t,$$
      así que el cuerpo de fracciones de $\im \phi^*$ es $k (t)$. Ahora
      $t \in k (t)$ cumple la ecuación mónica
      $$t^2 + 2\,t - c = 0, \quad c = t^2 + 2t \in \im \phi_*,$$
      aunque $t \notin \im \phi_*$. Esto significa que el anillo
      $$\im \phi_* \isom k [x,y] / (y^2 - x^2\,(x+1))$$
      no es integralmente cerrado, a diferencia de $k [t,u] / (u(t-1) - 1)$.

    \item[(b)] Para el morfismo $\AA^1 (k) \to V (y^2 - x^3) \subset \AA^2 (k)$,
      $t \mapsto (t^2, t^3)$ funciona el mismo argumento. Es fácil verificar la
      biyectividad. Consideremos los anillos de coordenadas
      $$k [t], \quad k [x,y] / (y^2 - x^3).$$
      El anillo $k [t]$ es integralmente cerrado (es un DFU), mientras que el
      segundo anillo es isomorfo a $k [t^2,t^3]$, cuyo cuerpo de fracciones es
      $k (t)$, pero la ecuación mónica
      $$(t^3/t^2)^2 - c = 0, \quad c = t^2 \in k [t^2,t^3]$$
      demuestra que $k [t^2,t^3]$ no es integralmente cerrado.
    \end{enumerate}

    Entonces, no solamente probamos que las aplicaciones en cuestión no son
    isomorfismos, sino que no puede haber ningún isomorfismo.
  \end{solucion}\fi
\end{ejerc}

\begin{ejerc}
  Demuestre que $X$ es un espacio Hausdorff noetheriano si y solo si $X$ es
  finito con la topología discreta.

  \ifdefined\solutions\begin{solucion}
    Si $X$ es un espacio noetheriano, entonces, $X$ tiene un número finito de
    componentes irreducibles:
    $$X = Z_1 \cup \cdots \cup Z_s.$$
    Ahora si $X$ es Hausdorff, entonces cada $Z_i$ es irreducible y Hausdorff,
    y por ende es unipuntual. Esto demuestra que todo espacio Hausforff
    noetheriano es necesariamente finito.

    Ahora si $X$ es un espacio Hausdorff finito, entonces todo punto de $X$ es
    cerrado: para todo $y \ne x$ existe un entorno abierto $U_y \ni y$ tal que
    $x \notin U_y$, y luego
    $$\{ x \} = \bigcap_{y\ne x} U_y^c$$
    es cerrado. Usando la finitud de $X$, todo subconjunto $Y \subset X$ puede
    ser escrito como una intersección finita de abiertos
    $$Y = \{ x_1, \ldots, x_n \}^c = \{ x_1 \}^c \cap \cdots \cap \{ x_n \}^c.$$
    Entonces, todo subconjunto de $X$ es abierto.

    Viceversa, si $X$ es finito con la topología discreta, entonces $X$ es
    trivialmente Hausdorff (como todo espacio discreto) y noetheriano (dado que
    es finito).
  \end{solucion}\fi
\end{ejerc}

\begin{ejerc}
  Encuentre las componentes irreducibles de los siguientes conjuntos algebraicos
  ($k$ es un cuerpo algebraicamente cerrado):
  \begin{enumerate}
  \item[1)] $\mathbf{V} (x\,(x+1), y) \subset \AA^2 (k)$;
  \item[2)] $\mathbf{V} (xz, yz) \subset \AA^3 (k)$;
  \item[3)] $\mathbf{V} (xy^2-x^2\,(x^2-1)) \subset \AA^2 (k)$.
  \end{enumerate}

  \ifdefined\solutions\begin{solucion}
    Hay que encontrar las descomposiciones primarias de los ideales
    correspondientes.

    \begin{enumerate}
    \item[1)] Tenemos
      $$(x\,(x+1), y) = (x,y) \cap (x+1,y),$$
      donde los ideales $(x,y)$ e $(x+1,y)$ son primos. Entonces, la
      descomposición en componentes irreducibles es
      \[ \mathbf{V} (x\,(x+1), y) =
         \mathbf{V} (x,y) \cup \mathbf{V} (x+1, y) =
         \{ (0,0) \} \cup \{ (-1,0) \}. \]

    \item[2)] Para el ideal monomial $(xz, yz) \subset k [x,y,z]$ está claro que
      $$(xz, yz) = (x,y) \cap (z),$$
      así que
      $$\mathbf{V} (xz, yz) = \mathbf{V} (x,y) \cup \mathbf{V} (z).$$

    \item[3)] Tenemos
      $$xy^2-x^2\,(x^2-1) = x\,(y^2 - x\,(x^2-1)),$$
      donde el polinomio $y^2 - x\,(x^2-1) \in k [x,y]$ es
      irreducible. Entonces,
      $$(xy^2-x^2\,(x^2-1)) = (x) \cap (y^2 - x\,(x^2-1)),$$
      y la descomposición en componentes irreducibles es
      \[ \mathbf{V} (xy^2-x^2\,(x^2-1)) =
         \mathbf{V} (x) \cup \mathbf{V} (y^2 - x\,(x^2-1)). \qedhere \]
    \end{enumerate}
  \end{solucion}\fi
\end{ejerc}

\end{document}
