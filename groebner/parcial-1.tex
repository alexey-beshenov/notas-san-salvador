\documentclass{article}

\def\exercisespersection{}
% TODO : CLEAN UP THIS MESS
% (AND MAKE SURE ALL TEXTS STILL COMPILE)
\usepackage[leqno]{amsmath}
\usepackage{amssymb}
\usepackage{graphicx}

\usepackage{diagbox} % table heads with diagonal lines
\usepackage{relsize}

\usepackage{wasysym}
\usepackage{scrextend}
\usepackage{epigraph}
\setlength\epigraphwidth{.6\textwidth}

\usepackage[utf8]{inputenc}

\usepackage{titlesec}
\titleformat{\chapter}[display]
  {\normalfont\sffamily\huge\bfseries}
  {\chaptertitlename\ \thechapter}{5pt}{\Huge}
\titleformat{\section}
  {\normalfont\sffamily\Large\bfseries}
  {\thesection}{1em}{}
\titleformat{\subsection}
  {\normalfont\sffamily\large\bfseries}
  {\thesubsection}{1em}{}
\titleformat{\part}[display]
  {\normalfont\sffamily\huge\bfseries}
  {\partname\ \thepart}{0pt}{\Huge}

\usepackage[T1]{fontenc}
\usepackage{fourier}
\usepackage{paratype}

\usepackage[symbol,perpage]{footmisc}

\usepackage{perpage}
\MakePerPage{footnote}

\usepackage{array}
\newcolumntype{x}[1]{>{\centering\hspace{0pt}}p{#1}}

% TODO: the following line causes conflict with new texlive (!)
% \usepackage[english,russian,polutonikogreek,spanish]{babel}
% \newcommand{\russian}[1]{{\selectlanguage{russian}#1}}

% Remove conflicting options for the moment:
\usepackage[english,polutonikogreek,spanish]{babel}

\AtBeginDocument{\shorthandoff{"}}
\newcommand{\greek}[1]{{\selectlanguage{polutonikogreek}#1}}

% % % % % % % % % % % % % % % % % % % % % % % % % % % % % %
% Limit/colimit symbols (with accented i: lím / colím)

\usepackage{etoolbox} % \patchcmd

\makeatletter
\patchcmd{\varlim@}{lim}{\lim}{}{}
\makeatother
\DeclareMathOperator*{\colim}{co{\lim}}
\newcommand{\dirlim}{\varinjlim}
\newcommand{\invlim}{\varprojlim}

% % % % % % % % % % % % % % % % % % % % % % % % % % % % % %

\usepackage[all,color]{xy}

\usepackage{pigpen}
\newcommand{\po}{\ar@{}[dr]|(.4){\text{\pigpenfont I}}}
\newcommand{\pb}{\ar@{}[dr]|(.3){\text{\pigpenfont A}}}
\newcommand{\polr}{\ar@{}[dr]|(.65){\text{\pigpenfont A}}}
\newcommand{\pour}{\ar@{}[ur]|(.65){\text{\pigpenfont G}}}
\newcommand{\hstar}{\mathop{\bigstar}}

\newcommand{\bigast}{\mathop{\Huge \mathlarger{\mathlarger{\ast}}}}

\newcommand{\term}{\textbf}

\usepackage{stmaryrd}

\usepackage{cancel}

\usepackage{tikzsymbols}

\newcommand{\open}{\underset{\mathrm{open}}{\hookrightarrow}}
\newcommand{\closed}{\underset{\mathrm{closed}}{\hookrightarrow}}

\newcommand{\tcol}[2]{{#1 \choose #2}}

\newcommand{\homot}{\simeq}
\newcommand{\isom}{\cong}
\newcommand{\cH}{\mathcal{H}}
\renewcommand{\hom}{\mathrm{hom}}
\renewcommand{\div}{\mathop{\mathrm{div}}}
\renewcommand{\Im}{\mathop{\mathrm{Im}}}
\renewcommand{\Re}{\mathop{\mathrm{Re}}}
\newcommand{\id}[1]{\mathrm{id}_{#1}}
\newcommand{\idid}{\mathrm{id}}

\newcommand{\ZG}{{\ZZ G}}
\newcommand{\ZH}{{\ZZ H}}

\newcommand{\quiso}{\simeq}

\newcommand{\personality}[1]{{\sc #1}}

\newcommand{\mono}{\rightarrowtail}
\newcommand{\epi}{\twoheadrightarrow}
\newcommand{\xepi}[1]{\xrightarrow{#1}\mathrel{\mkern-14mu}\rightarrow}

% % % % % % % % % % % % % % % % % % % % % % % % % % % % % %

\DeclareMathOperator{\Ad}{Ad}
\DeclareMathOperator{\Aff}{Aff}
\DeclareMathOperator{\Ann}{Ann}
\DeclareMathOperator{\Aut}{Aut}
\DeclareMathOperator{\Br}{Br}
\DeclareMathOperator{\CH}{CH}
\DeclareMathOperator{\Cl}{Cl}
\DeclareMathOperator{\Coeq}{Coeq}
\DeclareMathOperator{\Coind}{Coind}
\DeclareMathOperator{\Cop}{Cop}
\DeclareMathOperator{\Corr}{Corr}
\DeclareMathOperator{\Cor}{Cor}
\DeclareMathOperator{\Cov}{Cov}
\DeclareMathOperator{\Der}{Der}
\DeclareMathOperator{\Div}{Div}
\DeclareMathOperator{\D}{D}
\DeclareMathOperator{\Ehr}{Ehr}
\DeclareMathOperator{\End}{End}
\DeclareMathOperator{\Eq}{Eq}
\DeclareMathOperator{\Ext}{Ext}
\DeclareMathOperator{\Frac}{Frac}
\DeclareMathOperator{\Frob}{Frob}
\DeclareMathOperator{\Funct}{Funct}
\DeclareMathOperator{\Fun}{Fun}
\DeclareMathOperator{\GL}{GL}
\DeclareMathOperator{\Gal}{Gal}
\DeclareMathOperator{\Gr}{Gr}
\DeclareMathOperator{\Hol}{Hol}
\DeclareMathOperator{\Hom}{Hom}
\DeclareMathOperator{\Ho}{Ho}
\DeclareMathOperator{\Id}{Id}
\DeclareMathOperator{\Ind}{Ind}
\DeclareMathOperator{\Inn}{Inn}
\DeclareMathOperator{\Isom}{Isom}
\DeclareMathOperator{\Ker}{Ker}
\DeclareMathOperator{\Lan}{Lan}
\DeclareMathOperator{\Lie}{Lie}
\DeclareMathOperator{\Map}{Map}
\DeclareMathOperator{\Mat}{Mat}
\DeclareMathOperator{\Max}{Max}
\DeclareMathOperator{\Mor}{Mor}
\DeclareMathOperator{\Nat}{Nat}
\DeclareMathOperator{\Nrd}{Nrd}
\DeclareMathOperator{\Ob}{Ob}
\DeclareMathOperator{\Out}{Out}
\DeclareMathOperator{\PGL}{PGL}
\DeclareMathOperator{\PSL}{PSL}
\DeclareMathOperator{\PSU}{PSU}
\DeclareMathOperator{\Pic}{Pic}
\DeclareMathOperator{\RHom}{RHom}
\DeclareMathOperator{\Rad}{Rad}
\DeclareMathOperator{\Ran}{Ran}
\DeclareMathOperator{\Rep}{Rep}
\DeclareMathOperator{\Res}{Res}
\DeclareMathOperator{\SL}{SL}
\DeclareMathOperator{\SO}{SO}
\DeclareMathOperator{\SU}{SU}
\DeclareMathOperator{\Sh}{Sh}
\DeclareMathOperator{\Sing}{Sing}
\DeclareMathOperator{\Specm}{Specm}
\DeclareMathOperator{\Spec}{Spec}
\DeclareMathOperator{\Sp}{Sp}
\DeclareMathOperator{\Stab}{Stab}
\DeclareMathOperator{\Sym}{Sym}
\DeclareMathOperator{\Tors}{Tors}
\DeclareMathOperator{\Tor}{Tor}
\DeclareMathOperator{\Tot}{Tot}
\DeclareMathOperator{\UUU}{U}

\DeclareMathOperator{\adj}{adj}
\DeclareMathOperator{\ad}{ad}
\DeclareMathOperator{\af}{af}
\DeclareMathOperator{\card}{card}
\DeclareMathOperator{\cm}{cm}
\DeclareMathOperator{\codim}{codim}
\DeclareMathOperator{\cod}{cod}
\DeclareMathOperator{\coeq}{coeq}
\DeclareMathOperator{\coim}{coim}
\DeclareMathOperator{\coker}{coker}
\DeclareMathOperator{\cont}{cont}
\DeclareMathOperator{\conv}{conv}
\DeclareMathOperator{\cor}{cor}
\DeclareMathOperator{\depth}{depth}
\DeclareMathOperator{\diag}{diag}
\DeclareMathOperator{\diam}{diam}
\DeclareMathOperator{\dist}{dist}
\DeclareMathOperator{\dom}{dom}
\DeclareMathOperator{\eq}{eq}
\DeclareMathOperator{\ev}{ev}
\DeclareMathOperator{\ex}{ex}
\DeclareMathOperator{\fchar}{char}
\DeclareMathOperator{\fr}{fr}
\DeclareMathOperator{\gr}{gr}
\DeclareMathOperator{\im}{im}
\DeclareMathOperator{\infl}{inf}
\DeclareMathOperator{\interior}{int}
\DeclareMathOperator{\intrel}{intrel}
\DeclareMathOperator{\inv}{inv}
\DeclareMathOperator{\length}{length}
\DeclareMathOperator{\mcd}{mcd}
\DeclareMathOperator{\mcm}{mcm}
\DeclareMathOperator{\multideg}{multideg}
\DeclareMathOperator{\ord}{ord}
\DeclareMathOperator{\pr}{pr}
\DeclareMathOperator{\rel}{rel}
\DeclareMathOperator{\res}{res}
\DeclareMathOperator{\rkred}{rkred}
\DeclareMathOperator{\rkss}{rkss}
\DeclareMathOperator{\rk}{rk}
\DeclareMathOperator{\sgn}{sgn}
\DeclareMathOperator{\sk}{sk}
\DeclareMathOperator{\supp}{supp}
\DeclareMathOperator{\trdeg}{trdeg}
\DeclareMathOperator{\tr}{tr}
\DeclareMathOperator{\vol}{vol}

\newcommand{\iHom}{\underline{\Hom}}

\renewcommand{\AA}{\mathbb{A}}
\newcommand{\CC}{\mathbb{C}}
\renewcommand{\SS}{\mathbb{S}}
\newcommand{\TT}{\mathbb{T}}
\newcommand{\PP}{\mathbb{P}}
\newcommand{\BB}{\mathbb{B}}
\newcommand{\RR}{\mathbb{R}}
\newcommand{\ZZ}{\mathbb{Z}}
\newcommand{\FF}{\mathbb{F}}
\newcommand{\HH}{\mathbb{H}}
\newcommand{\NN}{\mathbb{N}}
\newcommand{\QQ}{\mathbb{Q}}
\newcommand{\KK}{\mathbb{K}}

% % % % % % % % % % % % % % % % % % % % % % % % % % % % % %

\usepackage{amsthm}

\newcommand{\legendre}[2]{\left(\frac{#1}{#2}\right)}

\newcommand{\examplesymbol}{$\blacktriangle$}
\renewcommand{\qedsymbol}{$\blacksquare$}

\newcommand{\dfn}{\mathrel{\mathop:}=}
\newcommand{\rdfn}{=\mathrel{\mathop:}}

\usepackage{xcolor}
\definecolor{mylinkcolor}{rgb}{0.0,0.4,1.0}
\definecolor{mycitecolor}{rgb}{0.0,0.4,1.0}
\definecolor{shadecolor}{rgb}{0.79,0.78,0.65}
\definecolor{gray}{rgb}{0.6,0.6,0.6}

\usepackage{colortbl}

\definecolor{myred}{rgb}{0.7,0.0,0.0}
\definecolor{mygreen}{rgb}{0.0,0.7,0.0}
\definecolor{myblue}{rgb}{0.0,0.0,0.7}

\definecolor{redshade}{rgb}{0.9,0.5,0.5}
\definecolor{greenshade}{rgb}{0.5,0.9,0.5}

\usepackage[unicode,colorlinks=true,linkcolor=mylinkcolor,citecolor=mycitecolor]{hyperref}
\newcommand{\refref}[2]{\hyperref[#2]{#1~\ref*{#2}}}
\newcommand{\eqnref}[1]{\hyperref[#1]{(\ref*{#1})}}

\newcommand{\tos}{\!\!\to\!\!}

\usepackage{framed}

\newcommand{\cequiv}{\simeq}

\makeatletter
\newcommand\xleftrightarrow[2][]{%
  \ext@arrow 9999{\longleftrightarrowfill@}{#1}{#2}}
\newcommand\longleftrightarrowfill@{%
  \arrowfill@\leftarrow\relbar\rightarrow}
\makeatother

\newcommand{\bsquare}{\textrm{\ding{114}}}

% % % % % % % % % % % % % % % % % % % % % % % % % % % % % %

\newtheoremstyle{myplain}
  {\topsep}   % ABOVESPACE
  {\topsep}   % BELOWSPACE
  {\itshape}  % BODYFONT
  {0pt}       % INDENT (empty value is the same as 0pt)
  {\bfseries} % HEADFONT
  {.}         % HEADPUNCT
  {5pt plus 1pt minus 1pt} % HEADSPACE
  {\thmnumber{#2}. \thmname{#1}\thmnote{ (#3)}}   % CUSTOM-HEAD-SPEC

\newtheoremstyle{myplainnameless}
  {\topsep}   % ABOVESPACE
  {\topsep}   % BELOWSPACE
  {\normalfont}  % BODYFONT
  {0pt}       % INDENT (empty value is the same as 0pt)
  {\bfseries} % HEADFONT
  {.}         % HEADPUNCT
  {5pt plus 1pt minus 1pt} % HEADSPACE
  {\thmnumber{#2}}   % CUSTOM-HEAD-SPEC 

\newtheoremstyle{sectionexercise}
  {\topsep}   % ABOVESPACE
  {\topsep}   % BELOWSPACE
  {\normalfont}  % BODYFONT
  {0pt}       % INDENT (empty value is the same as 0pt)
  {\bfseries} % HEADFONT
  {.}         % HEADPUNCT
  {5pt plus 1pt minus 1pt} % HEADSPACE
  {Ejercicio \thmnumber{#2}\thmnote{ (#3)}}   % CUSTOM-HEAD-SPEC

\newtheoremstyle{mydefinition}
  {\topsep}   % ABOVESPACE
  {\topsep}   % BELOWSPACE
  {\normalfont}  % BODYFONT
  {0pt}       % INDENT (empty value is the same as 0pt)
  {\bfseries} % HEADFONT
  {.}         % HEADPUNCT
  {5pt plus 1pt minus 1pt} % HEADSPACE
  {\thmnumber{#2}. \thmname{#1}\thmnote{ (#3)}}   % CUSTOM-HEAD-SPEC

% EN ESPAÑOL

\newtheorem*{hecho*}{Hecho}
\newtheorem*{corolario*}{Corolario}
\newtheorem*{teorema*}{Teorema}
\newtheorem*{conjetura*}{Conjetura}
\newtheorem*{proyecto*}{Proyecto}
\newtheorem*{observacion*}{Observación}

\newtheorem*{lema*}{Lema}
\newtheorem*{resultado-clave*}{Resultado clave}
\newtheorem*{proposicion*}{Proposición}

\theoremstyle{definition}
\newtheorem*{ejercicio*}{Ejercicio}
\newtheorem*{definicion*}{Definición}
\newtheorem*{comentario*}{Comentario}
\newtheorem*{definicion-alternativa*}{Definición alternativa}
\newtheorem*{ejemploxs}{Ejemplo}
\newenvironment{ejemplo*}
  {\pushQED{\qed}\renewcommand{\qedsymbol}{\examplesymbol}\ejemploxs}
  {\popQED\endejemploxs}

\theoremstyle{myplain}
\newtheorem{proposicion}{Proposición}[section]

\newtheorem{proyecto}[proposicion]{Proyecto}
\newtheorem{teorema}[proposicion]{Teorema}
\newtheorem{corolario}[proposicion]{Corolario}
\newtheorem{hecho}[proposicion]{Hecho}
\newtheorem{lema}[proposicion]{Lema}

\newtheorem{observacion}[proposicion]{Observación}

\newenvironment{observacionejerc}
    {\pushQED{\qed}\renewcommand{\qedsymbol}{$\square$}\csname inner@observacionejerc\endcsname}
    {\popQED\csname endinner@observacionejerc\endcsname}
\newtheorem{inner@observacionejerc}[proposicion]{Observación}

\newenvironment{proposicionejerc}
    {\pushQED{\qed}\renewcommand{\qedsymbol}{$\square$}\csname inner@proposicionejerc\endcsname}
    {\popQED\csname endinner@proposicionejerc\endcsname}
\newtheorem{inner@proposicionejerc}[proposicion]{Proposicion}

\newenvironment{lemaejerc}
    {\pushQED{\qed}\renewcommand{\qedsymbol}{$\square$}\csname inner@lemaejerc\endcsname}
    {\popQED\csname endinner@lemaejerc\endcsname}
\newtheorem{inner@lemaejerc}[proposicion]{Lema}

\newtheorem{calculo}[proposicion]{Cálculo}

\theoremstyle{myplainnameless}
\newtheorem{nameless}[proposicion]{}

\theoremstyle{mydefinition}
\newtheorem{comentario}[proposicion]{Comentario}
\newtheorem{comentarioast}[proposicion]{Comentario ($\clubsuit$)}
\newtheorem{construccion}[proposicion]{Construcción}
\newtheorem{aplicacion}[proposicion]{Aplicación}
\newtheorem{definicion}[proposicion]{Definición}
\newtheorem{definicion-alternativa}[proposicion]{Definición alternativa}
\newtheorem{notacion}[proposicion]{Notación}
\newtheorem{advertencia}[proposicion]{Advertencia}
\newtheorem{digresion}[proposicion]{Digresión}
\newtheorem{ejemplox}[proposicion]{Ejemplo}
\newenvironment{ejemplo}
  {\pushQED{\qed}\renewcommand{\qedsymbol}{\examplesymbol}\ejemplox}
  {\popQED\endejemplox}
\newtheorem{contraejemplox}[proposicion]{Contraejemplo}
\newenvironment{contraejemplo}
  {\pushQED{\qed}\renewcommand{\qedsymbol}{\examplesymbol}\contraejemplox}
  {\popQED\endcontraejemplox}
\newtheorem{noejemplox}[proposicion]{No-ejemplo}
\newenvironment{noejemplo}
  {\pushQED{\qed}\renewcommand{\qedsymbol}{\examplesymbol}\noejemplox}
  {\popQED\endnoejemplox}
 
\newtheorem{ejemploastx}[proposicion]{Ejemplo ($\clubsuit$)}
\newenvironment{ejemploast}
  {\pushQED{\qed}\renewcommand{\qedsymbol}{\examplesymbol}\ejemploastx}
  {\popQED\endejemploastx}

\ifdefined\exercisespersection
  \theoremstyle{sectionexercise}
  \newtheorem{ejercicio}{}[section]
  \theoremstyle{mydefinition}
\else
  \ifdefined\exercisesglobal
    \theoremstyle{sectionexercise}
    \newtheorem{ejercicio}{}
    \theoremstyle{mydefinition}
  \else
    \ifdefined\exercisespersection
      \newtheorem{ejercicio}[proposicion]{Ejercicio}
    \fi
  \fi
\fi

% % % % % % % % % % % % % % % % % % % % % % % % % % % % % %

\theoremstyle{myplain}
\newtheorem{proposition}{Proposition}[section]
\newtheorem*{fact*}{Fact}
\newtheorem*{proposition*}{Proposition}
\newtheorem{lemma}[proposition]{Lemma}
\newtheorem*{lemma*}{Lemma}

\newtheorem{exercise}{Exercise}
\newtheorem*{hint}{Hint}

\newtheorem{theorem}[proposition]{Theorem}
\newtheorem{conjecture}[proposition]{Conjecture}
\newtheorem*{theorem*}{Theorem}
\newtheorem{corollary}[proposition]{Corollary}
\newtheorem{fact}[proposition]{Fact}
\newtheorem*{claim}{Claim}
\newtheorem{definition-theorem}[proposition]{Definition-theorem}

\theoremstyle{mydefinition}
\newtheorem{examplex}[proposition]{Example}
\newenvironment{example}
  {\pushQED{\qed}\renewcommand{\qedsymbol}{\examplesymbol}\examplex}
  {\popQED\endexamplex}

\newtheorem*{examplexx}{Example}
\newenvironment{example*}
  {\pushQED{\qed}\renewcommand{\qedsymbol}{\examplesymbol}\examplexx}
  {\popQED\endexamplexx}

\newtheorem{definition}[proposition]{Definition}
\newtheorem*{definition*}{Definition}
\newtheorem{wrong-definition}[proposition]{Wrong definition}
\newtheorem{remark}[proposition]{Remark}

\makeatletter
\newcommand{\xRightarrow}[2][]{\ext@arrow 0359\Rightarrowfill@{#1}{#2}}
\makeatother

% % % % % % % % % % % % % % % % % % % % % % % % % % % % % %

\newcommand{\Et}{\mathop{\text{\rm Ét}}}

\newcommand{\categ}[1]{\text{\bf #1}}
\newcommand{\vcateg}{\mathcal}
\newcommand{\bone}{{\boldsymbol 1}}
\newcommand{\bDelta}{{\boldsymbol\Delta}}
\newcommand{\bR}{{\mathbf{R}}}

\newcommand{\univ}{\mathfrak}

\newcommand{\TODO}{\colorbox{red}{\textbf{*** TODO ***}}}
\newcommand{\proofreadme}{\colorbox{red}{\textbf{*** NEEDS PROOFREADING ***}}}

\makeatletter
\def\iddots{\mathinner{\mkern1mu\raise\p@
\vbox{\kern7\p@\hbox{.}}\mkern2mu
\raise4\p@\hbox{.}\mkern2mu\raise7\p@\hbox{.}\mkern1mu}}
\makeatother

\newcommand{\ssincl}{\reflectbox{\rotatebox[origin=c]{45}{$\subseteq$}}}
\newcommand{\vsupseteq}{\reflectbox{\rotatebox[origin=c]{-90}{$\supseteq$}}}
\newcommand{\vin}{\reflectbox{\rotatebox[origin=c]{90}{$\in$}}}

\newcommand{\Ga}{\mathbb{G}_\mathrm{a}}
\newcommand{\Gm}{\mathbb{G}_\mathrm{m}}

\renewcommand{\U}{\UUU}

\DeclareRobustCommand{\Stirling}{\genfrac\{\}{0pt}{}}
\DeclareRobustCommand{\stirling}{\genfrac[]{0pt}{}}

% % % % % % % % % % % % % % % % % % % % % % % % % % % % % %
% tikz

\usepackage{tikz-cd}
\usetikzlibrary{babel}
\usetikzlibrary{decorations.pathmorphing}
\usetikzlibrary{arrows}
\usetikzlibrary{calc}
\usetikzlibrary{fit}
\usetikzlibrary{hobby}

% % % % % % % % % % % % % % % % % % % % % % % % % % % % % %
% Banners

\newcommand\mybannerext[3]{{\normalfont\sffamily\bfseries\large\noindent #1

\noindent #2

\noindent #3

}\noindent\rule{\textwidth}{1.25pt}

\vspace{1em}}

\newcommand\mybanner[2]{{\normalfont\sffamily\bfseries\large\noindent #1

\noindent #2

}\noindent\rule{\textwidth}{1.25pt}

\vspace{1em}}

\renewcommand{\O}{\mathcal{O}}


\usepackage{fullpage}

\theoremstyle{definition}
\newtheorem{ejerc}{Ejercicio}

\begin{document}
\pagestyle{empty}

\mybanner{Álgebra computacional. Examen parcial 1\ifdefined\solutions{}.
  Soluciones
\fi}{Universidad de El Salvador, 12/04/2019}

\begin{ejerc}
  Consideremos el ideal
  $$I = (xy, \, x^3-y^2+x) \subset k [x,y].$$

  \begin{enumerate}
  \item[a)] Encuentre la base de Gröbner reducida de $I$ respecto al orden
    graduado lexicográfico.

  \item[b)] Encuentre una base monomial de $k[x,y]/I$ como un espacio vectorial
    sobre $k$.
  \end{enumerate}
\end{ejerc}

\ifdefined\solutions

\hrule
\vspace{1em}

Este ejercicio nada más pone a prueba el manejo de los algoritmos básicos, y
escogí a propósito polinomios que no requieren muchos cálculos. Denotemos
$$f \dfn xy, \quad g \dfn x^3 - y^2 + x.$$

Calculemos el $S$-polinomio de $f$ y $g$. Primero,
$$\mcm (LT (f), LT (g)) = \mcm (xy, x^3) = x^3 y.$$
Ahora
\[ S (f,g) =
   \frac{x^3y}{xy}\,xy - \frac{x^3y}{x^3}\,(x^3 - y^2 + x) =
   y^3 - xy. \]
La división con resto nos da
$$y^3 - xy = (-1)\cdot xy + 0\cdot (x^3 - y^2 + x) + y^3.$$
Entonces, hay que agregar a nuestra base el polinomio
$$h \dfn y^3.$$
Procedamos calculando
\begin{align*}
  S (h, f) & = \frac{xy^3}{y^3}\,y^3 - \frac{xy^3}{xy}\,xy = 0,\\
  S (h, g) & = \frac{x^3y^3}{y^3}\,y^3 - \frac{x^3 y^3}{x^3}\,(x^3 - y^2 + x) = y^5 - xy^3 = (y^2 - x)\cdot y^3.
\end{align*}
Entonces, el criterio de Buchberger nos dice que los polinomios
$$G = \{ xy, \, x^3-y^2+x, \, y^3 \}$$
forman una base de Gröbner. Notamos que esta ya es reducida.

\pagebreak

Ahora
$$(LT (G)) = (xy, \, x^3, \, y^3).$$
Los monomios que no están en este ideal son
$$1, y, y^2, x, x^2,$$
y estos forman una base de $k[x,y]/I$ como un espacio vectorial sobre $k$.

\begin{center}
  \begin{tikzpicture}[x=0.75cm,y=0.75cm]

    \fill[color=blue, fill opacity=0.3] (1,1) -- (5,1) -- (5,5) -- (1,5) -- cycle;
    \fill[color=blue, fill opacity=0.3] (0,3) -- (5,3) -- (5,5) -- (0,5) -- cycle;
    \fill[color=blue, fill opacity=0.3] (3,0) -- (3,5) -- (5,5) -- (5,0) -- cycle;

    \foreach \i in {0,...,4}
    \foreach \j in {0,...,4}
    \draw (\i,\j) node[circle,fill,inner sep=1pt] {};

    \draw[->] (0,0) -- (5,0) node[right] {$m$};
    \draw[->] (0,0) -- (0,5) node[above] {$n$};
  \end{tikzpicture}
\end{center}

\newpage
\fi

% % % % % % % % % % % % % % % % % % % % % % % % % % % % % %

\begin{ejerc}
  Los polinomios de la forma $x^\alpha - x^\beta \in k [x_1,\ldots,x_n]$ se
  llaman \term{binomios}. Se dice que un ideal $I$ es \term{binomial} si $I$
  puede ser generado por algunos binomios. En este ejercicio vamos a probar que
  $I$ es binomial si y solo si su base de Gröbner reducida consiste en binomios.

  \begin{enumerate}
  \item[a)] Demuestre que para dos binomios $f_1 = x^{\alpha(1)} - x^{\beta(1)}$
    y $f_2 = x^{\alpha(2)} - x^{\beta(2)}$ el polinomio $S (f_1,f_2)$ es también
    un binomio si $f_1 \ne f_2$.

  \item[b)] Sean $f = x^\alpha - x^\beta$,
    $f_1 = x^{\alpha (1)} - x^{\beta (1)}, \ldots, f_s = x^{\alpha (s)} -
    x^{\beta (s)}$ binomios. Demuestre que el algoritmo de división con resto de
    $f$ por $(f_1,\ldots,f_s)$ produce
    $$f = q_1 f_1 + \cdots + q_s f_s + r,$$
    donde $r = 0$ o $r$ es también un binomio.

    % \noindent\emph{Sugerencia: demuestre que a cada paso del algoritmo
    % $r+p = x^\gamma - x^\delta$ es un binomio, y además $r = 0$, o $p = 0$, o
    % $r$ y $p$ corresponden a los monomios de este binomio.}

  \item[c)] Demuestre que todo ideal binomial tiene una base de Gröbner que
    consiste en binomios.

  \item[d)] Demuestre que la base de Gröbner reducida de un ideal binomial
    consiste en binomios.
  \end{enumerate}
\end{ejerc}

\ifdefined\solutions
\hrule
\vspace{1em}

Este ejercicio es más interesante y requiere análisis un poco más creativo del
algoritmo de división con resto y el algoritmo de Buchberger.

Primero, por la definición, tenemos para dos binomios
\begin{multline*}
  S (x^{\alpha (1)} - x^{\beta (1)}, \, x^{\alpha (2)} - x^{\beta (2)}) =
  x^{\gamma-\alpha(1)}\,(x^{\alpha (1)} - x^{\beta (1)}) - x^{\gamma-\alpha(2)}\,(x^{\alpha (2)} - x^{\beta (2)}) \\
  = x^{\gamma - \alpha (2) + \beta (2)} - x^{\gamma - \alpha (1) + \beta (1)},
\end{multline*}
donde $x^\gamma = \mcm (x^{\alpha(1)}, x^{\alpha(2)})$. Entonces,
el $S$-polinomio de dos binomios es también un binomio.

\vspace{1em}

Ahora analicemos el algoritmo de división de $x^\alpha - x^\beta$ por binomios
$f_1,\ldots,f_s$. Al inicio del algoritmo, se tiene
$$q_1 \dfn \cdots \dfn q_s \dfn 0, \quad r \dfn 0, \quad p \dfn f,$$
y entonces $r+p$ es un binomio. Vamos a probar por inducción que a cada paso se
cumple una de las siguientes posibilidades:

\begin{itemize}
\item $r = x^\gamma - x^\delta$ es un binomio, $p = 0$ (y en este caso el
  algoritmo se termina),
\item $p = x^\gamma - x^\delta$ es un binomio, $r = 0$,
\item $p = x^\gamma$, $r = -x^\delta$,
\item $r = x^\gamma$, $p = -x^\delta$.
\end{itemize}

Durante la ejecución del algoritmo ocurren dos situaciones.

\begin{enumerate}
\item[i)] Si $LT (f_i) \mid LT (p)$ para algún $i$, entonces en el algoritmo
  $r+p$ se remplaza por
  $$r+p', \quad \text{donde }p' \dfn p - (LT (p)/LT (f_i))\,f_i.$$
  Si $p = x^\gamma - x^\delta$ es un binomio y $r = 0$, entonces
  $$p' = x^{\beta (i) + \gamma - \alpha (i)} - x^\delta.$$
  Si $p = x^\gamma$ y $r = -x^\delta$, entonces
  $$p' = x^{\beta (i) + \gamma - \alpha (i)}.$$
  De la misma manera, si $r = x^\gamma$ y $p = -x^\delta$, entonces
  $$p' = -x^{\beta (i) + \delta - \alpha (i)}.$$

\item[ii)] Si $LT (f_i) \nmid LT (p)$ para ningún $i$, entonces $r$ y $p$ se
  remplazan por
  $$r' \dfn r + LT (p), \quad p' \dfn p - LT (p).$$
  Si $p = x^\gamma - x^\delta$ es un binomio y $r = 0$, entonces
  $$r' = x^\gamma, \quad p' = -x^\delta.$$
  Si $p = x^\gamma$ y $r = -x^\delta$, entonces
  $$r' = x^\gamma - x^\delta, \quad p' = 0.$$
  De la misma manera, si $r = x^\gamma$ y $p = -x^\delta$, entonces
  $$r' = x^\gamma - x^\delta, \quad p' = 0.$$
\end{enumerate}

Estas consideraciones nos permiten concluir que el algoritmo produce el resto
que es también un binomio.

\vspace{1em}

Ahora si $I$ es un ideal binomial, podemos ejecutar el algoritmo de Buchberger
sobre los generadores binomiales de $I$. A cada paso el algoritmo calcula los
$S$-polinomios, y todos estos serán binomios gracias a la parte a). Los
polinomios que se añaden a la base son restos de división de $S$-polinomios,
y estos son binomios gracias a b). Podemos concluir que el algoritmo de
Buchberger construye una base de Gröbner que consiste en binomios.

\vspace{1em}

Quitando los polinomios innecesarios, se obtiene una base de Gröbner mínima
$G = \{ g_1, \ldots, g_s \}$ que también consiste en binomios. El algoritmo que
construye la base reducida a cada paso remplaza $g_i$ por
$\overline{g_i}^{G\setminus \{ g_i \}}$, y todos estos son binomios gracias a la
parte b).

\newpage
\fi

% % % % % % % % % % % % % % % % % % % % % % % % % % % % % %

\begin{ejerc}
  En este ejercicio vamos a calcular el radical de un ideal monomial.

  \begin{enumerate}
  \item[a)] Demuestre que un ideal monomial $I \subset k [x_1,\ldots,x_n]$ es
    primo si y solo si $I = (x_{i_1}, \ldots, x_{i_s})$ es el ideal generado por
    algunas variables
    $\{ x_{i_1}, \ldots, x_{i_s} \} \subseteq \{ x_1, \ldots, x_n \}$.

  \item[b)] Demuestre que si $A$ es cualquier anillo conmutativo e
    $I,J \subseteq A$ son ideales, entonces
    \[ \sqrt{I+J} = \sqrt{\sqrt{I} + \sqrt{J}}, \quad
       \sqrt{I\cap J} = \sqrt{I}\cap \sqrt{J}. \]

  \item[c)] Para un monomio $x^\alpha \dfn x_1^{\alpha_1}\cdots x_n^{\alpha_n}$
    demuestre que $\sqrt{(x^\alpha)} = (\sqrt{x^\alpha})$, donde
    \[ \sqrt{x^\alpha} \dfn
       x_1^{\min(1,\alpha_1)}\cdots x_n^{\min(1,\alpha_n)} =
       \text{producto de las variables que están en }x^\alpha. \]

  \item[d)] Demuestre que el ideal
    $(\sqrt{x^{\alpha (1)}}, \ldots, \sqrt{x^{\alpha (s)}})$ es radical.

    \noindent \emph{Sugerencia: note que si
      $\sqrt{x^{\alpha}} = x_{i_1}\cdots x_{i_k}$, entonces
      $\sqrt{(x^{\alpha})} = (x_{i_1})\cap\cdots\cap (x_{i_k})$. Usando esta
      observación, exprese
      $(\sqrt{x^{\alpha (1)}}, \ldots, \sqrt{x^{\alpha (s)}}) =
       \bigcap_i\mathfrak{p}_i$, donde $\mathfrak{p}_i$
      son algunos ideales monomiales primos.}

  \item[e)] Demuestre que
    $\sqrt{(x^{\alpha (1)}, \ldots, x^{\alpha (s)})} =
     \sqrt{(\sqrt{x^{\alpha (1)}}, \ldots, \sqrt{x^{\alpha (s)}})} =
     (\sqrt{x^{\alpha (1)}}, \ldots, \sqrt{x^{\alpha (s)}})$.
  \end{enumerate}
\end{ejerc}

\ifdefined\solutions
\hrule
\vspace{1em}

Este ejercicio no tiene nada que ver con las bases de Gröbner, sino revisa las
propiedades de ideales monomiales que juegan papel importante en el
curso. Consideremos entonces un ideal monomial propio
$$I = (x^{\alpha (1)}, \ldots, x^{\alpha (s)}),$$
donde $x^{\alpha (1)}, \ldots, x^{\alpha (s)}$ son generadores minimales. Para
todo $i$ existe $j$ tal que $x_j \mid x^{\alpha (i)}$.
Si $x^{\alpha (i)} \ne x_j$, entonces podemos escribir
$$x^{\alpha (i)} = x_j\,x^{\alpha' (i)},$$
y se ve que $x_j, x^{\alpha' (i)} \notin I$ (por la minimalidad de la base),
lo que significa que el ideal no es primo. Entonces, los generadores de un ideal
primo son necesariamente variables.

Viceversa, si un ideal $I$ está generado por las variables $x_1,\ldots,x_\ell$,
entonces
$$k [x_1,\ldots,x_n]/I \isom k [x_{\ell+1},\ldots,x_n],$$
así que $I$ es primo. Esto establece la parte a).

\vspace{1em}

La parte b) es una propiedad general de radicales y la vimos en el curso de
álgebra conmutativa. Para la suma de dos ideales $I,J\subseteq A$, podemos notar
que
\[ \sqrt{I + J} =
   \bigcap_{\substack{\mathfrak{p} \in \Spec A \\ \mathfrak{p} \supseteq I + J}} =
   \bigcap_{\substack{\mathfrak{p} \in \Spec A \\ \mathfrak{p} \supseteq I \text{ y } \mathfrak{p} \supseteq J}} =
   \bigcap_{\substack{\mathfrak{p} \in \Spec A \\ \mathfrak{p} \supseteq \sqrt{I} \text{ y } \mathfrak{p} \supseteq \sqrt{J}}} =
   \bigcap_{\substack{\mathfrak{p} \in \Spec A \\ \mathfrak{p} \supseteq \sqrt{I} + \sqrt{J}}} =
   \sqrt{\sqrt{I} + \sqrt{J}}. \]
 
Aquí hemos usado dos propiedades. Primero, cualquier ideal contiene $I + J$ si y
solo si este contiene $I$ y contiene $J$. Segundo, si $\mathfrak{p}$ es un ideal
primo, entonces $\mathfrak{p} \supseteq I$ si y solo si
$\mathfrak{p} \supseteq \sqrt{I}$ (se sigue de $\sqrt{I} \supseteq I$ y
$\sqrt{\mathfrak{p}} = \mathfrak{p}$).

También, como nos propuso Mario, se podía ocupar directamente la definición del
radical
$$\sqrt{I} \dfn \{ f \in A \mid f^r \in I\text{ para algún }r = 1,2,3,\ldots \}.$$
A saber, tenemos $I + J \subseteq \sqrt{I} + \sqrt{J}$, y luego
$\sqrt{I + J} \subseteq \sqrt{\sqrt{I} + \sqrt{J}}$. Esta es la inclusión
trivial. Para la otra inclusión, notamos que si
$f \in \sqrt{\sqrt{I} + \sqrt{J}}$, esto quiere decir que existen $r,s,t$ tales
que
$$f^r = g + h, \quad g^s \in I, \quad h^t \in J.$$
Ahora
$$f^{r\,(s+t)} = (g+h)^{s+t} = \sum_{i + j = s+t} {s+t \choose i} \, g^i\,h^j.$$
Aquí para cada término $g^i\,h^j$ se cumple $i \ge s$ o $j \ge t$, así que
$g^i\,h^j \in I$ o $g^i\,h^j \in J$. Esto nos permite concluir que
$f \in \sqrt{I + J}$.

Para la intersección de ideales, notamos que $f \in \sqrt{I\cap J}$ si y solo si
existe $r$ tal que $f^r \in I$ y $f^r \in J$, y luego
$f \in \sqrt{I} \cap \sqrt{J}$. Viceversa, si $f \in \sqrt{I} \cap \sqrt{J}$,
entonces existen $r,s$ tales que $f^r \in I$ e $f^s \in J$. Para
$t \dfn \max \{ r, s \}$ tenemos entonces $f^t \in I\cap J$, así que
$f \in \sqrt{I\cap J}$. (Este es el argumento directo propuesto por Mario.)

\vspace{1em}

La parte c) en realidad viene de una propiedad más general: \emph{si $A$ es un
  dominio de factorización única y $f \in A$ es un elemento que se factoriza
  como $f \sim f_1^{m_1}\cdots f_s^{m_s}$, donde $f_1,\ldots,f_s$ son elementos
  irreducibles no asociados entre sí y $m_1,\ldots,m_s\ge 1$, entonces
  $\sqrt{(f)} = (\sqrt{f})$, donde $\sqrt{f} \dfn f_1\cdots f_s$}.

\vspace{1em}

En efecto, primero está claro que
$$f \mid (f_1\cdots f_s)^m, \quad\text{donde }m \dfn \max \{ m_1,\ldots,m_s \},$$
y esto nos permite concluir que $(f_1\cdots f_s) \subseteq \sqrt{(f)}$.
Viceversa, asumamos que $g \in \sqrt{(f)}$. Luego, existe $r$ tal que
$f \mid g^r$. Factoricemos
$$g \sim g_1\cdots g_t,$$
donde $g_1,\ldots, g_t$ son irreducibles (no necesariamente distintos). Tenemos
entonces
$$(f_1^{m_1}\cdots f_s^{m_s}) \mid g_1^r\cdots g_t^r.$$
Por la irreducibilidad de los $f_i$ y $g_j$, esto nos permite concluir que cada
$f_i$ es asociado con algún $g_j$, y luego
$$f_1\cdots f_s \mid g,$$
Lo que establece la otra inclusión $\sqrt{(f)} \subseteq (f_1\cdots f_s)$. \qed

\vspace{1em}

En este ejercicio particular, $f_1,\ldots,f_s$ son algunas variables
$x_{i_1},\ldots,x_{i_s}$. Como vimos arriba, la inclusión
${(\sqrt{x^\alpha}) \subseteq \sqrt{(x^\alpha)}}$ es fácil. Para la otra
inclusión, tenemos que probar que si para un polinomio
$g = \sum_\beta c_\beta\,x^\beta \in k [x_1,\ldots,x_n]$ se tiene
$x^\alpha \mid g^r$ para algún $r$, entonces $\sqrt{x^\alpha} \mid g^r$.
Uno puede tratar de analizar los términos de
$$g^r = \sum_\gamma \left(\sum_{\beta_1 + \cdots + \beta_r = \gamma} c_{\beta_1}\cdots c_{\beta_r}\right)\,x^\gamma,$$
pero no es una buena idea\dots Mejor ocupar las factorizaciones como en el
argumento que acabo de dar.

\vspace{1em}

En la parte d), usando
$$\sqrt{x^{\alpha}} = (x_{i_1})\cap\cdots\cap (x_{i_t}).$$
podemos expresar
$$(\sqrt{x^{\alpha (1)}}, \ldots, \sqrt{x^{\alpha (s)}}) = \sum_{1\le i\le s} (\sqrt{x^{\alpha (i)}})$$
como una suma de intersecciones de la forma $(x_{i_1})\cap\cdots\cap (x_{i_t})$,
y luego usar la distributividad\footnote{En la tarea 1 hemos analizado
  intersecciones de ideales monomiales, y es fácil observar que
  $I\cap (J_1 + J_2) = (I\cap J_1) + (I \cap J_2)$ para ideales monomiales
  $I,J_1,J_2 \subseteq k [x_1,\ldots,x_n]$.} de $\cap$ respecto a la suma $\sum$
para escribir el ideal de arriba como una intersección de sumas
$$(x_{i_1}) + \cdots + (x_{i_t}) = (x_{i_1},\ldots,x_{i_t}),$$
que son ideales primos gracias a la parte a).

Luego,
\[ \sqrt{(\sqrt{x^{\alpha (1)}}, \ldots, \sqrt{x^{\alpha (s)}})} =
   \sqrt{\bigcap_i \mathfrak{p}_i} =
   \bigcap_i \sqrt{\mathfrak{p}_i} =
   \bigcap_i\mathfrak{p}_i =
   (\sqrt{x^{\alpha (1)}}, \ldots, \sqrt{x^{\alpha (s)}}). \]

Ahora combinando las partes b), c), y d), se tiene
\[ \sqrt{(x^{\alpha (1)}, \ldots, x^{\alpha (s)})} =
   \sqrt{(\sqrt{x^{\alpha (1)}}, \ldots, \sqrt{x^{\alpha (s)}})} =
   (\sqrt{x^{\alpha (1)}}, \ldots, \sqrt{x^{\alpha (s)}}). \]
\fi

\end{document}
