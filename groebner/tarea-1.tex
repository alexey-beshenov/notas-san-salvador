\documentclass{article}

\def\exercisespersection{}
% TODO : CLEAN UP THIS MESS
% (AND MAKE SURE ALL TEXTS STILL COMPILE)
\usepackage[leqno]{amsmath}
\usepackage{amssymb}
\usepackage{graphicx}

\usepackage{diagbox} % table heads with diagonal lines
\usepackage{relsize}

\usepackage{wasysym}
\usepackage{scrextend}
\usepackage{epigraph}
\setlength\epigraphwidth{.6\textwidth}

\usepackage[utf8]{inputenc}

\usepackage{titlesec}
\titleformat{\chapter}[display]
  {\normalfont\sffamily\huge\bfseries}
  {\chaptertitlename\ \thechapter}{5pt}{\Huge}
\titleformat{\section}
  {\normalfont\sffamily\Large\bfseries}
  {\thesection}{1em}{}
\titleformat{\subsection}
  {\normalfont\sffamily\large\bfseries}
  {\thesubsection}{1em}{}
\titleformat{\part}[display]
  {\normalfont\sffamily\huge\bfseries}
  {\partname\ \thepart}{0pt}{\Huge}

\usepackage[T1]{fontenc}
\usepackage{fourier}
\usepackage{paratype}

\usepackage[symbol,perpage]{footmisc}

\usepackage{perpage}
\MakePerPage{footnote}

\usepackage{array}
\newcolumntype{x}[1]{>{\centering\hspace{0pt}}p{#1}}

% TODO: the following line causes conflict with new texlive (!)
% \usepackage[english,russian,polutonikogreek,spanish]{babel}
% \newcommand{\russian}[1]{{\selectlanguage{russian}#1}}

% Remove conflicting options for the moment:
\usepackage[english,polutonikogreek,spanish]{babel}

\AtBeginDocument{\shorthandoff{"}}
\newcommand{\greek}[1]{{\selectlanguage{polutonikogreek}#1}}

% % % % % % % % % % % % % % % % % % % % % % % % % % % % % %
% Limit/colimit symbols (with accented i: lím / colím)

\usepackage{etoolbox} % \patchcmd

\makeatletter
\patchcmd{\varlim@}{lim}{\lim}{}{}
\makeatother
\DeclareMathOperator*{\colim}{co{\lim}}
\newcommand{\dirlim}{\varinjlim}
\newcommand{\invlim}{\varprojlim}

% % % % % % % % % % % % % % % % % % % % % % % % % % % % % %

\usepackage[all,color]{xy}

\usepackage{pigpen}
\newcommand{\po}{\ar@{}[dr]|(.4){\text{\pigpenfont I}}}
\newcommand{\pb}{\ar@{}[dr]|(.3){\text{\pigpenfont A}}}
\newcommand{\polr}{\ar@{}[dr]|(.65){\text{\pigpenfont A}}}
\newcommand{\pour}{\ar@{}[ur]|(.65){\text{\pigpenfont G}}}
\newcommand{\hstar}{\mathop{\bigstar}}

\newcommand{\bigast}{\mathop{\Huge \mathlarger{\mathlarger{\ast}}}}

\newcommand{\term}{\textbf}

\usepackage{stmaryrd}

\usepackage{cancel}

\usepackage{tikzsymbols}

\newcommand{\open}{\underset{\mathrm{open}}{\hookrightarrow}}
\newcommand{\closed}{\underset{\mathrm{closed}}{\hookrightarrow}}

\newcommand{\tcol}[2]{{#1 \choose #2}}

\newcommand{\homot}{\simeq}
\newcommand{\isom}{\cong}
\newcommand{\cH}{\mathcal{H}}
\renewcommand{\hom}{\mathrm{hom}}
\renewcommand{\div}{\mathop{\mathrm{div}}}
\renewcommand{\Im}{\mathop{\mathrm{Im}}}
\renewcommand{\Re}{\mathop{\mathrm{Re}}}
\newcommand{\id}[1]{\mathrm{id}_{#1}}
\newcommand{\idid}{\mathrm{id}}

\newcommand{\ZG}{{\ZZ G}}
\newcommand{\ZH}{{\ZZ H}}

\newcommand{\quiso}{\simeq}

\newcommand{\personality}[1]{{\sc #1}}

\newcommand{\mono}{\rightarrowtail}
\newcommand{\epi}{\twoheadrightarrow}
\newcommand{\xepi}[1]{\xrightarrow{#1}\mathrel{\mkern-14mu}\rightarrow}

% % % % % % % % % % % % % % % % % % % % % % % % % % % % % %

\DeclareMathOperator{\Ad}{Ad}
\DeclareMathOperator{\Aff}{Aff}
\DeclareMathOperator{\Ann}{Ann}
\DeclareMathOperator{\Aut}{Aut}
\DeclareMathOperator{\Br}{Br}
\DeclareMathOperator{\CH}{CH}
\DeclareMathOperator{\Cl}{Cl}
\DeclareMathOperator{\Coeq}{Coeq}
\DeclareMathOperator{\Coind}{Coind}
\DeclareMathOperator{\Cop}{Cop}
\DeclareMathOperator{\Corr}{Corr}
\DeclareMathOperator{\Cor}{Cor}
\DeclareMathOperator{\Cov}{Cov}
\DeclareMathOperator{\Der}{Der}
\DeclareMathOperator{\Div}{Div}
\DeclareMathOperator{\D}{D}
\DeclareMathOperator{\Ehr}{Ehr}
\DeclareMathOperator{\End}{End}
\DeclareMathOperator{\Eq}{Eq}
\DeclareMathOperator{\Ext}{Ext}
\DeclareMathOperator{\Frac}{Frac}
\DeclareMathOperator{\Frob}{Frob}
\DeclareMathOperator{\Funct}{Funct}
\DeclareMathOperator{\Fun}{Fun}
\DeclareMathOperator{\GL}{GL}
\DeclareMathOperator{\Gal}{Gal}
\DeclareMathOperator{\Gr}{Gr}
\DeclareMathOperator{\Hol}{Hol}
\DeclareMathOperator{\Hom}{Hom}
\DeclareMathOperator{\Ho}{Ho}
\DeclareMathOperator{\Id}{Id}
\DeclareMathOperator{\Ind}{Ind}
\DeclareMathOperator{\Inn}{Inn}
\DeclareMathOperator{\Isom}{Isom}
\DeclareMathOperator{\Ker}{Ker}
\DeclareMathOperator{\Lan}{Lan}
\DeclareMathOperator{\Lie}{Lie}
\DeclareMathOperator{\Map}{Map}
\DeclareMathOperator{\Mat}{Mat}
\DeclareMathOperator{\Max}{Max}
\DeclareMathOperator{\Mor}{Mor}
\DeclareMathOperator{\Nat}{Nat}
\DeclareMathOperator{\Nrd}{Nrd}
\DeclareMathOperator{\Ob}{Ob}
\DeclareMathOperator{\Out}{Out}
\DeclareMathOperator{\PGL}{PGL}
\DeclareMathOperator{\PSL}{PSL}
\DeclareMathOperator{\PSU}{PSU}
\DeclareMathOperator{\Pic}{Pic}
\DeclareMathOperator{\RHom}{RHom}
\DeclareMathOperator{\Rad}{Rad}
\DeclareMathOperator{\Ran}{Ran}
\DeclareMathOperator{\Rep}{Rep}
\DeclareMathOperator{\Res}{Res}
\DeclareMathOperator{\SL}{SL}
\DeclareMathOperator{\SO}{SO}
\DeclareMathOperator{\SU}{SU}
\DeclareMathOperator{\Sh}{Sh}
\DeclareMathOperator{\Sing}{Sing}
\DeclareMathOperator{\Specm}{Specm}
\DeclareMathOperator{\Spec}{Spec}
\DeclareMathOperator{\Sp}{Sp}
\DeclareMathOperator{\Stab}{Stab}
\DeclareMathOperator{\Sym}{Sym}
\DeclareMathOperator{\Tors}{Tors}
\DeclareMathOperator{\Tor}{Tor}
\DeclareMathOperator{\Tot}{Tot}
\DeclareMathOperator{\UUU}{U}

\DeclareMathOperator{\adj}{adj}
\DeclareMathOperator{\ad}{ad}
\DeclareMathOperator{\af}{af}
\DeclareMathOperator{\card}{card}
\DeclareMathOperator{\cm}{cm}
\DeclareMathOperator{\codim}{codim}
\DeclareMathOperator{\cod}{cod}
\DeclareMathOperator{\coeq}{coeq}
\DeclareMathOperator{\coim}{coim}
\DeclareMathOperator{\coker}{coker}
\DeclareMathOperator{\cont}{cont}
\DeclareMathOperator{\conv}{conv}
\DeclareMathOperator{\cor}{cor}
\DeclareMathOperator{\depth}{depth}
\DeclareMathOperator{\diag}{diag}
\DeclareMathOperator{\diam}{diam}
\DeclareMathOperator{\dist}{dist}
\DeclareMathOperator{\dom}{dom}
\DeclareMathOperator{\eq}{eq}
\DeclareMathOperator{\ev}{ev}
\DeclareMathOperator{\ex}{ex}
\DeclareMathOperator{\fchar}{char}
\DeclareMathOperator{\fr}{fr}
\DeclareMathOperator{\gr}{gr}
\DeclareMathOperator{\im}{im}
\DeclareMathOperator{\infl}{inf}
\DeclareMathOperator{\interior}{int}
\DeclareMathOperator{\intrel}{intrel}
\DeclareMathOperator{\inv}{inv}
\DeclareMathOperator{\length}{length}
\DeclareMathOperator{\mcd}{mcd}
\DeclareMathOperator{\mcm}{mcm}
\DeclareMathOperator{\multideg}{multideg}
\DeclareMathOperator{\ord}{ord}
\DeclareMathOperator{\pr}{pr}
\DeclareMathOperator{\rel}{rel}
\DeclareMathOperator{\res}{res}
\DeclareMathOperator{\rkred}{rkred}
\DeclareMathOperator{\rkss}{rkss}
\DeclareMathOperator{\rk}{rk}
\DeclareMathOperator{\sgn}{sgn}
\DeclareMathOperator{\sk}{sk}
\DeclareMathOperator{\supp}{supp}
\DeclareMathOperator{\trdeg}{trdeg}
\DeclareMathOperator{\tr}{tr}
\DeclareMathOperator{\vol}{vol}

\newcommand{\iHom}{\underline{\Hom}}

\renewcommand{\AA}{\mathbb{A}}
\newcommand{\CC}{\mathbb{C}}
\renewcommand{\SS}{\mathbb{S}}
\newcommand{\TT}{\mathbb{T}}
\newcommand{\PP}{\mathbb{P}}
\newcommand{\BB}{\mathbb{B}}
\newcommand{\RR}{\mathbb{R}}
\newcommand{\ZZ}{\mathbb{Z}}
\newcommand{\FF}{\mathbb{F}}
\newcommand{\HH}{\mathbb{H}}
\newcommand{\NN}{\mathbb{N}}
\newcommand{\QQ}{\mathbb{Q}}
\newcommand{\KK}{\mathbb{K}}

% % % % % % % % % % % % % % % % % % % % % % % % % % % % % %

\usepackage{amsthm}

\newcommand{\legendre}[2]{\left(\frac{#1}{#2}\right)}

\newcommand{\examplesymbol}{$\blacktriangle$}
\renewcommand{\qedsymbol}{$\blacksquare$}

\newcommand{\dfn}{\mathrel{\mathop:}=}
\newcommand{\rdfn}{=\mathrel{\mathop:}}

\usepackage{xcolor}
\definecolor{mylinkcolor}{rgb}{0.0,0.4,1.0}
\definecolor{mycitecolor}{rgb}{0.0,0.4,1.0}
\definecolor{shadecolor}{rgb}{0.79,0.78,0.65}
\definecolor{gray}{rgb}{0.6,0.6,0.6}

\usepackage{colortbl}

\definecolor{myred}{rgb}{0.7,0.0,0.0}
\definecolor{mygreen}{rgb}{0.0,0.7,0.0}
\definecolor{myblue}{rgb}{0.0,0.0,0.7}

\definecolor{redshade}{rgb}{0.9,0.5,0.5}
\definecolor{greenshade}{rgb}{0.5,0.9,0.5}

\usepackage[unicode,colorlinks=true,linkcolor=mylinkcolor,citecolor=mycitecolor]{hyperref}
\newcommand{\refref}[2]{\hyperref[#2]{#1~\ref*{#2}}}
\newcommand{\eqnref}[1]{\hyperref[#1]{(\ref*{#1})}}

\newcommand{\tos}{\!\!\to\!\!}

\usepackage{framed}

\newcommand{\cequiv}{\simeq}

\makeatletter
\newcommand\xleftrightarrow[2][]{%
  \ext@arrow 9999{\longleftrightarrowfill@}{#1}{#2}}
\newcommand\longleftrightarrowfill@{%
  \arrowfill@\leftarrow\relbar\rightarrow}
\makeatother

\newcommand{\bsquare}{\textrm{\ding{114}}}

% % % % % % % % % % % % % % % % % % % % % % % % % % % % % %

\newtheoremstyle{myplain}
  {\topsep}   % ABOVESPACE
  {\topsep}   % BELOWSPACE
  {\itshape}  % BODYFONT
  {0pt}       % INDENT (empty value is the same as 0pt)
  {\bfseries} % HEADFONT
  {.}         % HEADPUNCT
  {5pt plus 1pt minus 1pt} % HEADSPACE
  {\thmnumber{#2}. \thmname{#1}\thmnote{ (#3)}}   % CUSTOM-HEAD-SPEC

\newtheoremstyle{myplainnameless}
  {\topsep}   % ABOVESPACE
  {\topsep}   % BELOWSPACE
  {\normalfont}  % BODYFONT
  {0pt}       % INDENT (empty value is the same as 0pt)
  {\bfseries} % HEADFONT
  {.}         % HEADPUNCT
  {5pt plus 1pt minus 1pt} % HEADSPACE
  {\thmnumber{#2}}   % CUSTOM-HEAD-SPEC 

\newtheoremstyle{sectionexercise}
  {\topsep}   % ABOVESPACE
  {\topsep}   % BELOWSPACE
  {\normalfont}  % BODYFONT
  {0pt}       % INDENT (empty value is the same as 0pt)
  {\bfseries} % HEADFONT
  {.}         % HEADPUNCT
  {5pt plus 1pt minus 1pt} % HEADSPACE
  {Ejercicio \thmnumber{#2}\thmnote{ (#3)}}   % CUSTOM-HEAD-SPEC

\newtheoremstyle{mydefinition}
  {\topsep}   % ABOVESPACE
  {\topsep}   % BELOWSPACE
  {\normalfont}  % BODYFONT
  {0pt}       % INDENT (empty value is the same as 0pt)
  {\bfseries} % HEADFONT
  {.}         % HEADPUNCT
  {5pt plus 1pt minus 1pt} % HEADSPACE
  {\thmnumber{#2}. \thmname{#1}\thmnote{ (#3)}}   % CUSTOM-HEAD-SPEC

% EN ESPAÑOL

\newtheorem*{hecho*}{Hecho}
\newtheorem*{corolario*}{Corolario}
\newtheorem*{teorema*}{Teorema}
\newtheorem*{conjetura*}{Conjetura}
\newtheorem*{proyecto*}{Proyecto}
\newtheorem*{observacion*}{Observación}

\newtheorem*{lema*}{Lema}
\newtheorem*{resultado-clave*}{Resultado clave}
\newtheorem*{proposicion*}{Proposición}

\theoremstyle{definition}
\newtheorem*{ejercicio*}{Ejercicio}
\newtheorem*{definicion*}{Definición}
\newtheorem*{comentario*}{Comentario}
\newtheorem*{definicion-alternativa*}{Definición alternativa}
\newtheorem*{ejemploxs}{Ejemplo}
\newenvironment{ejemplo*}
  {\pushQED{\qed}\renewcommand{\qedsymbol}{\examplesymbol}\ejemploxs}
  {\popQED\endejemploxs}

\theoremstyle{myplain}
\newtheorem{proposicion}{Proposición}[section]

\newtheorem{proyecto}[proposicion]{Proyecto}
\newtheorem{teorema}[proposicion]{Teorema}
\newtheorem{corolario}[proposicion]{Corolario}
\newtheorem{hecho}[proposicion]{Hecho}
\newtheorem{lema}[proposicion]{Lema}

\newtheorem{observacion}[proposicion]{Observación}

\newenvironment{observacionejerc}
    {\pushQED{\qed}\renewcommand{\qedsymbol}{$\square$}\csname inner@observacionejerc\endcsname}
    {\popQED\csname endinner@observacionejerc\endcsname}
\newtheorem{inner@observacionejerc}[proposicion]{Observación}

\newenvironment{proposicionejerc}
    {\pushQED{\qed}\renewcommand{\qedsymbol}{$\square$}\csname inner@proposicionejerc\endcsname}
    {\popQED\csname endinner@proposicionejerc\endcsname}
\newtheorem{inner@proposicionejerc}[proposicion]{Proposicion}

\newenvironment{lemaejerc}
    {\pushQED{\qed}\renewcommand{\qedsymbol}{$\square$}\csname inner@lemaejerc\endcsname}
    {\popQED\csname endinner@lemaejerc\endcsname}
\newtheorem{inner@lemaejerc}[proposicion]{Lema}

\newtheorem{calculo}[proposicion]{Cálculo}

\theoremstyle{myplainnameless}
\newtheorem{nameless}[proposicion]{}

\theoremstyle{mydefinition}
\newtheorem{comentario}[proposicion]{Comentario}
\newtheorem{comentarioast}[proposicion]{Comentario ($\clubsuit$)}
\newtheorem{construccion}[proposicion]{Construcción}
\newtheorem{aplicacion}[proposicion]{Aplicación}
\newtheorem{definicion}[proposicion]{Definición}
\newtheorem{definicion-alternativa}[proposicion]{Definición alternativa}
\newtheorem{notacion}[proposicion]{Notación}
\newtheorem{advertencia}[proposicion]{Advertencia}
\newtheorem{digresion}[proposicion]{Digresión}
\newtheorem{ejemplox}[proposicion]{Ejemplo}
\newenvironment{ejemplo}
  {\pushQED{\qed}\renewcommand{\qedsymbol}{\examplesymbol}\ejemplox}
  {\popQED\endejemplox}
\newtheorem{contraejemplox}[proposicion]{Contraejemplo}
\newenvironment{contraejemplo}
  {\pushQED{\qed}\renewcommand{\qedsymbol}{\examplesymbol}\contraejemplox}
  {\popQED\endcontraejemplox}
\newtheorem{noejemplox}[proposicion]{No-ejemplo}
\newenvironment{noejemplo}
  {\pushQED{\qed}\renewcommand{\qedsymbol}{\examplesymbol}\noejemplox}
  {\popQED\endnoejemplox}
 
\newtheorem{ejemploastx}[proposicion]{Ejemplo ($\clubsuit$)}
\newenvironment{ejemploast}
  {\pushQED{\qed}\renewcommand{\qedsymbol}{\examplesymbol}\ejemploastx}
  {\popQED\endejemploastx}

\ifdefined\exercisespersection
  \theoremstyle{sectionexercise}
  \newtheorem{ejercicio}{}[section]
  \theoremstyle{mydefinition}
\else
  \ifdefined\exercisesglobal
    \theoremstyle{sectionexercise}
    \newtheorem{ejercicio}{}
    \theoremstyle{mydefinition}
  \else
    \ifdefined\exercisespersection
      \newtheorem{ejercicio}[proposicion]{Ejercicio}
    \fi
  \fi
\fi

% % % % % % % % % % % % % % % % % % % % % % % % % % % % % %

\theoremstyle{myplain}
\newtheorem{proposition}{Proposition}[section]
\newtheorem*{fact*}{Fact}
\newtheorem*{proposition*}{Proposition}
\newtheorem{lemma}[proposition]{Lemma}
\newtheorem*{lemma*}{Lemma}

\newtheorem{exercise}{Exercise}
\newtheorem*{hint}{Hint}

\newtheorem{theorem}[proposition]{Theorem}
\newtheorem{conjecture}[proposition]{Conjecture}
\newtheorem*{theorem*}{Theorem}
\newtheorem{corollary}[proposition]{Corollary}
\newtheorem{fact}[proposition]{Fact}
\newtheorem*{claim}{Claim}
\newtheorem{definition-theorem}[proposition]{Definition-theorem}

\theoremstyle{mydefinition}
\newtheorem{examplex}[proposition]{Example}
\newenvironment{example}
  {\pushQED{\qed}\renewcommand{\qedsymbol}{\examplesymbol}\examplex}
  {\popQED\endexamplex}

\newtheorem*{examplexx}{Example}
\newenvironment{example*}
  {\pushQED{\qed}\renewcommand{\qedsymbol}{\examplesymbol}\examplexx}
  {\popQED\endexamplexx}

\newtheorem{definition}[proposition]{Definition}
\newtheorem*{definition*}{Definition}
\newtheorem{wrong-definition}[proposition]{Wrong definition}
\newtheorem{remark}[proposition]{Remark}

\makeatletter
\newcommand{\xRightarrow}[2][]{\ext@arrow 0359\Rightarrowfill@{#1}{#2}}
\makeatother

% % % % % % % % % % % % % % % % % % % % % % % % % % % % % %

\newcommand{\Et}{\mathop{\text{\rm Ét}}}

\newcommand{\categ}[1]{\text{\bf #1}}
\newcommand{\vcateg}{\mathcal}
\newcommand{\bone}{{\boldsymbol 1}}
\newcommand{\bDelta}{{\boldsymbol\Delta}}
\newcommand{\bR}{{\mathbf{R}}}

\newcommand{\univ}{\mathfrak}

\newcommand{\TODO}{\colorbox{red}{\textbf{*** TODO ***}}}
\newcommand{\proofreadme}{\colorbox{red}{\textbf{*** NEEDS PROOFREADING ***}}}

\makeatletter
\def\iddots{\mathinner{\mkern1mu\raise\p@
\vbox{\kern7\p@\hbox{.}}\mkern2mu
\raise4\p@\hbox{.}\mkern2mu\raise7\p@\hbox{.}\mkern1mu}}
\makeatother

\newcommand{\ssincl}{\reflectbox{\rotatebox[origin=c]{45}{$\subseteq$}}}
\newcommand{\vsupseteq}{\reflectbox{\rotatebox[origin=c]{-90}{$\supseteq$}}}
\newcommand{\vin}{\reflectbox{\rotatebox[origin=c]{90}{$\in$}}}

\newcommand{\Ga}{\mathbb{G}_\mathrm{a}}
\newcommand{\Gm}{\mathbb{G}_\mathrm{m}}

\renewcommand{\U}{\UUU}

\DeclareRobustCommand{\Stirling}{\genfrac\{\}{0pt}{}}
\DeclareRobustCommand{\stirling}{\genfrac[]{0pt}{}}

% % % % % % % % % % % % % % % % % % % % % % % % % % % % % %
% tikz

\usepackage{tikz-cd}
\usetikzlibrary{babel}
\usetikzlibrary{decorations.pathmorphing}
\usetikzlibrary{arrows}
\usetikzlibrary{calc}
\usetikzlibrary{fit}
\usetikzlibrary{hobby}

% % % % % % % % % % % % % % % % % % % % % % % % % % % % % %
% Banners

\newcommand\mybannerext[3]{{\normalfont\sffamily\bfseries\large\noindent #1

\noindent #2

\noindent #3

}\noindent\rule{\textwidth}{1.25pt}

\vspace{1em}}

\newcommand\mybanner[2]{{\normalfont\sffamily\bfseries\large\noindent #1

\noindent #2

}\noindent\rule{\textwidth}{1.25pt}

\vspace{1em}}

\renewcommand{\O}{\mathcal{O}}


\usepackage{fullpage}

\theoremstyle{definition}
\newtheorem{ejerc}{Ejercicio}
\newenvironment{solucion}{\begin{proof}[Solución]}{\end{proof}}

\begin{document}

\pagestyle{empty}

\mybanner{Álgebra computacional. Tarea 1. Fecha límite: 21/03/2019}{Universidad de El Salvador, ciclo impar 2019}

\begin{ejerc}
  Implemente en Macaulay2 la función \texttt{mcd($f$,$g$)} que calcula el máximo
  común divisor de $f,g \in k[x]$ usando el algoritmo de Euclides.
  (Para el resto de división, use el operador \verb|%|.)

  \ifdefined\solutions\begin{solucion}
    ~

\begin{verbatim}
mcd = (f,g)-> (
  while g != 0 do (
    h := f;
    f = g;
    g = h%g
  );

  f//leadCoefficient(f)
);
\end{verbatim}

  \end{solucion}\fi
\end{ejerc}

\begin{ejerc}
  Demuestre que sobre los polinomios en una variable $k [x]$ hay un solo orden
  monomial que viene dado por
  $$1 \prec x \prec x^2 \prec x^3 \prec \cdots;$$
  es decir, $x^m \prec x^n \iff m < n$.

  \ifdefined\solutions\begin{solucion}
    Asumamos que
    $$x^m \succ x^n$$
    para algún $m < n$. Luego, por la compatibilidad de $\succ$ con los
    productos, se obtiene
    \begin{align*}
      x^n = x^m\,x^{n-m} & \succ x^n\,x^{n-m} = x^{2n - m},\\
      x^{2n - m} = x^m\,x^{2\,(n-m)} & \succ x^n\,x^{2\,(n-m)} = x^{3\,n - 2m},\\
      x^{3n - 2m} = x^{m}\,x^{3\,(n-m)} & \succ x^n\,x^{3\,(n-m)} = x^{4n - 3m},\\
                         & \cdots
    \end{align*}
    Lo que nos da una cadena infinita decreciente
    $$x^n \succ x^{2n-m} \succ x^{3n-2m} \succ x^{4n-3m} \succ \cdots$$
    Esto contradice la condición de buen orden. Entonces, se tiene
    necesariamente $x^m \prec x^n$ para cualesquiera $m < n$.

    \vspace{1em}

    Otro argumento un poco diferente: notamos que para $x$ hay dos
    posibilidades: o $x \succ 1$, o $x \prec 1$. En el primer caso, la
    compatibilidad con productos implica que
    $$1 \prec x \prec x^2 \prec x^3 \prec \cdots,$$
    mientras que en el segundo caso
    $$1 \succ x \succ x^2 \succ x^3 \succ \cdots.$$
    Pero en el segundo caso no se cumple la propiedad de buen orden.
  \end{solucion}\fi
\end{ejerc}

\begin{ejerc}
  Demuestre que $\prec_{lex}, \prec_{grlex}, \prec_{grevlex}$ son órdenes
  monomiales.

  \ifdefined\solutions\begin{solucion}
    Omitida. (Veo que la mayoría no probó que la relación en cada caso es una
    relación de orden.)
  \end{solucion}\fi
\end{ejerc}

\begin{ejerc}
  Consideremos la siguiente propiedad: para cualesquiera $\alpha,\beta$ existe
  un número finito de monomios $x^\gamma$ tales que
  $x^\alpha \prec x^\gamma \prec x^\beta$.  ¿Para cuáles órdenes monomiales
  entre $\prec_{lex}$, $\prec_{grlex}$ y $\prec_{grevlex}$ esto es cierto?

  \ifdefined\solutions\begin{solucion}
    Esta propiedad no se cumple para el orden lexicográfico. Por ejemplo, para
    el orden lexicográfico sobre $k [x,y]$ con $x \succ y$ se tiene
    $$x \succ y^n \succ 1 \quad \text{para cualquier }n.$$

    Para el orden graduado lexicográfico, la propiedad sí se cumple: si tenemos
    $$x_1^{i_1}\cdots x_n^{i_n} \succ x_1^{j_1}\cdots x_n^{j_n} \succ x_1^{k_1}\cdots x_n^{k_n},$$
    entonces en particular,
    $$i_1+\cdots+i_n \ge j_1+\cdots+j_n \ge k_1+\cdots+k_n.$$
    Entonces, para el grado total del monomio en el medio, hay un número finito
    de posibilidades, y hay un número finito de monomios de grado total fijo.

    Para el orden graduado inverso lexicográfico, la propiedad se cumple por las
    mismas razones.
  \end{solucion}\fi
\end{ejerc}

\begin{ejerc}
  Fijemos algún orden monomial sobre $k [x_1,\ldots,x_n]$. Sean
  $f,g \in k [x_1,\ldots,x_n]$ polinomios no nulos. Demuestre las siguientes
  propiedades.

  \begin{enumerate}
  \item[1)] $\multideg(fg) = \multideg(f) + \multideg(g)$.

  \item[2)] Si $f+g\ne 0$, entonces
    $\multideg (f + g) \le \max (\multideg(f), \multideg(g))$. Además, si
    $\multideg(f) \ne \multideg(g)$, entonces se cumple la igualdad.
  \end{enumerate}

  \ifdefined\solutions\begin{solucion}

    Para $f = \sum_\alpha a_\alpha\,x^\alpha$ y
    $g = \sum_\beta b_\beta x^\beta$, tenemos
    $$fg = \sum_\gamma \left(\sum_{\alpha+\beta = \gamma} a_\alpha \, b_\beta\right)\,x^\gamma.$$
    Ahora sea
    \[ \alpha' \dfn \max \{ \alpha \mid a_\alpha \ne 0 \}, \quad
       \beta' \dfn \max \{ \beta \mid b_\beta \ne 0 \}. \]
    Luego, el término mayor de $fg$ será
    $$a_{\alpha'} b_{\beta'} \, x^{\alpha' + \beta'}.$$

    Para la suma $f+g$, basta notar que allí aparecen los mismos monomios que en
    $f$ y $g$, salvo posible cancelación de términos.
    Si $\multideg(f) \ne \multideg(g)$, entonces los terminos mayores no se
    cancelan.
  \end{solucion}\fi
\end{ejerc}

\begin{ejerc}
  Demuestre que para ideales monomiales $I,J \in k [x_1,\ldots,x_n]$ los ideales
  $IJ, I+J, I\cap J$ son también monomiales.

  \ifdefined\solutions\begin{solucion}
    Pongamos
    $$I = (x^\alpha \mid \alpha \in A), \quad J = (x^\beta \mid \beta \in B).$$

    Por la definición, la suma $I+J$ es el ideal más pequeño que contiene a $I$
    e $J$, y es más o menos inmediato que
    $$I+J = (x^\gamma \mid \gamma \in A \cup B).$$

    El producto $IJ$ es el ideal \emph{generado} por los productos $fg$ donde
    $f\in I$, $g\in J$. Esto implica que los elementos de $IJ$ son las sumas
    finitas
    $$h = f_1 g_1 + \cdots + f_s g_s,$$
    donde $f_1,\ldots,f_s \in I$, $g_1,\ldots,g_s \in J$. Recordamos que los
    elementos de $I$ son precisamente las combinaciones $k$-lineales de monomios
    de $I$, mientras que los elementos de $J$ son combinaciones $k$-lineales de
    monomios en $J$. Al desarrollar la expresión para $h$, se ve que es una
    combinación $k$-lineal de monomios $x^\gamma\,x^\delta$ donde
    $x^\gamma \in I$, $x^\delta \in J$. Notamos que esto implica que
    $x^\alpha x^\beta \mid x^\gamma x^\delta$ para algunos $\alpha \in A$,
    $\beta \in B$. Esto demuestra que $I+J$ es monomial y además que
    $$IJ = (x^\alpha x^\beta \mid \alpha \in A, \, \beta \in B).$$

    \noindent\textbf{* Un recordatorio}: en general, el producto $IJ$ no
    simplemente consiste en los productos $fg$ con $f\in I$, $g\in J$, sino
    \emph{está generado} por estos productos. Por ejemplo, en el anillo
    $\ZZ [x]$ consideremos el ideal
    \[ I = (2,x) = \{ 2f + xg \mid f,g \in \ZZ [x] \} =
       \{ f \in \ZZ [x] \mid f (0)\text{ es par} \}. \]
     Luego,
     $$I^2 = \Bigl\{ \text{sumas finitas }\sum_i f_i g_i \Bigm| f_i, g_i \in I \Bigr\}.$$
     Ahora $2\in I$ y $x \in I$, así que $x^2 + 4 \in I$. Sin embargo, el
     polinomio $x^2+4$ es irreducible y no puede ser escrito como un producto
     $fg$ con $f,g\in I$.

     Otro ejemplo parecido: consideremos el ideal $I = (x,y) \subset \QQ
     [x,y]$. Luego, $x^2 + y^2 \in I^2$, pero este polinomio es irreducible. En
     efecto, asumamos que
     $$x^2 + y^2 = fg$$
     para dos polinomios no constantes $f,g$. Estos polinomios tienen que ser
     lineales y sin pérdida de generalidad, podemos asumir que ambos son
     mónicos. Además, se ve que el término constante debe ser nulo, así que nos
     queda
     $$f = x + a\,y, \quad g \dfn x + b\,y.$$
     Ahora
     $$fg = x^2 + a b \, y^2 + (a + b)\,xy = x^2 + y^2$$
     implica que $a^2 = -1$ o $b^2 = -1$, pero esto es imposible en nuestro
     cuerpo de base $\QQ$. (Por otro lado, en $\QQ (i) [x]$ el polinomio es
     claramente reducible: $x^2 + y^2 = (x+iy)\,(x-iy)$).

     \vspace{1em}

     En fin, para la intersección, recordamos que $I$ es monomial si y solo si
     $f\in I$ implica que todos los monomios de $f$ pertenecen a $I$. Ahora si
     $I$ e $J$ son monomiales y $f\in I\cap J$, entonces todos los monomios de
     $f$ pertenecen a $I\cap J$. Esto es suficiente para concluir que $I\cap J$
     es monomial. Para describir sus generadores, notamos que
     $x^\gamma \in I\cap J$ si y solo si $x^\alpha \mid x^\gamma$ y
     $x^\beta \mid x^\gamma$ para algunos $\alpha \in A$, $\beta \in B$. Luego,
     $\mcm (x^\alpha,x^\beta) \mid x^\gamma$. De aquí se ve que
     $$I \cap J = (\mcm (x^\alpha,x^\beta) \mid \alpha\in A, \, \beta\in B).$$
  \end{solucion}\fi
\end{ejerc}

\begin{ejerc}
  Demuestre que el lema de Dickson es equivalente al siguiente resultado: para
  todo subconjunto $A \subseteq \NN^n$ existe un número finito de elementos
  $\alpha (1), \ldots, \alpha (s) \in A$ tales que para todo $\alpha \in A$ se
  tiene $\alpha = \alpha (i) + \gamma$ para algún $i = 1,\ldots,s$ y
  $\gamma \in \NN^n$.

  \ifdefined\solutions\begin{solucion}
    Omitida.
  \end{solucion}\fi
\end{ejerc}

\begin{ejerc}
  Para un ideal monomial $I$ digamos que un conjunto de generadores
  $\{ x^{\alpha (1)}, \ldots, x^{\alpha (s)} \}$ es \term{minimal} si
  $x^{\alpha (i)} \nmid x^{\alpha (j)}$ para $i\ne j$. Demuestre que todo ideal
  monomial posee un conjunto de generadores minimal y este es único.

  \ifdefined\solutions\begin{solucion}
    Por el lema de Dickson, sabemos que existe un conjunto de generadores finito
    $\{ x^{\alpha (1)}, \ldots, x^{\alpha (s)} \}$. Si no es minimal y
    $x^{\alpha (i)} \nmid x^{\alpha (j)}$, entonces podemos quitar
    $x^{\alpha (j)}$. Esto no cambia el ideal. Procediendo de esta manera,
    se obtiene un conjunto de generadores minimal.

    Ahora si tenemos dos conjuntos de generadores minimales
    \[ \{ x^{\alpha (1)}, \ldots, x^{\alpha (s)} \}, \quad
       \{ x^{\beta (1)}, \ldots, x^{\beta (t)} \}, \]
    entonces
    \[ (x^{\alpha (1)}, \ldots, x^{\alpha (s)}) =
       (x^{\beta (1)}, \ldots, x^{\beta (t)}). \]
    En particular, para cualquier $i = 1,\ldots,s$ tenemos
    $x^{\alpha (i)} \in (x^{\beta (1)}, \ldots, x^{\beta (t)})$, y por ende
    existe $j = 1,\ldots,t$ tal que $x^{\beta (j)} \mid x^{\alpha (i)}$. Pero
    $x^{\beta (j)} \in (x^{\alpha (1)}, \ldots, x^{\alpha (s)})$, lo que
    significa que existe $k = 1,\ldots,s$ tal que
    $$x^{\alpha (k)} \mid x^{\beta (j)} \mid x^{\alpha (i)}.$$
    Pero por la minimalidad, se tiene necesariamente $k = i$ y luego
    $\alpha (i) = \beta (j)$. Esto demuestra que
    \[ \{ x^{\alpha (1)}, \ldots, x^{\alpha (s)} \} \subseteq
       \{ x^{\beta (1)}, \ldots, x^{\beta (t)} \}. \]
    El mismo argumento demuestra la otra inclusión.
  \end{solucion}\fi
\end{ejerc}

\begin{ejerc}
  Consideremos los ideales monomiales
  \[ I_j \dfn (x_1,\ldots, \widehat{x_j}, \ldots, x_n) \subset
     k [x_1,\ldots,x_n], \quad j = 1,\ldots,n, \]
  donde $\widehat{x_j}$ significa que $x_j$ se omite de la lista. Encuentre el
  conjunto de generadores minimal para el ideal
  $$I_1\cap \cdots \cap I_n.$$
  Por ejemplo, para $n = 2$ tenemos $(x_2)\cap (x_1) = (x_1 x_2)$; para $n = 3$
  tenemos
  $$(x_2,x_3) \cap (x_1,x_3) \cap (x_1,x_2) = (x_1 x_2, x_1 x_3, x_2 x_3)$$
  (¡demuéstrelo!), etcétera.

  \ifdefined\solutions\begin{solucion}
    Podemos usar el ejercicio 6, pero con cuidado. La intersección será generada
    por los monomios
    $$x^\alpha = \mcm (x_{i_1},\ldots,x_{i_n}), \quad i_1 \ne 1, \ldots i_n \ne n.$$
    Este mcm será igual al producto de todas las variables distintas que
    aparecen entre $x_{i_1},\ldots,x_{i_n}$. Notamos que siempre hay
    \emph{por lo menos} dos variables distintas, y de esta manera se pueden
    obtener todos los monomios $x_i x_j$ con $i < j$. Por otra parte, para
    cualquier mcm de arriba debe haber $1 \le i < j \le n$ tales que
    $x_i x_j \mid x^\alpha$. Entonces, podemos tomar como un conjunto de
    generadores
    $$\{ x_i x_j \mid 1 \le i < j \le n \}.$$
    Es evidente que es minimal.

    En algunas soluciones la respuesta era que los generadores serán
    $$\mcm (x_1,\ldots,\widehat{x_i},\ldots,x_n) = x_1\cdots\widehat{x_i}\cdots x_n,$$
    estos monomios van a generar un ideal que es más pequeño que la verdadera
    intersección $I_1\cap\cdots\cap I_n$.
  \end{solucion}\fi
\end{ejerc}

\begin{ejerc}
  Fijemos un vector $u = (u_1,\ldots,u_n) \in \RR^n$ tal que los números $u_i$
  son positivos y linealmente independientes sobre $\QQ$. Demuestre que
  $$x^\alpha \prec_u x^\beta \iff u\cdot \alpha < u\cdot \beta,$$
  donde $\cdot$ denota el producto escalar habitual, define un orden
  monomial. ¿Qué sucede si los $u_i$ no son linealmente independientes?

  \ifdefined\solutions\begin{solucion}
    Primero, tenemos un orden total: para cualesquiera $\alpha,\beta \in \NN^n$
    tendremos uno de los siguientes casos:
    \[ u\cdot \alpha < u\cdot \beta, \quad
       u\cdot \alpha > u\cdot \beta, \quad
       u\cdot \alpha = u\cdot \beta. \]
    Ahora si $u\cdot \alpha = u\cdot \beta$, entonces
    $$u\cdot (\alpha-\beta) = \sum_{1 \le i \le n} u_i\,(\alpha_i - \beta_i) = 0,$$
    y luego por la independencia lineal de los $u_i$ sobre $\QQ$, se tiene
    necesariamente $\alpha_i = \beta_i$ para todo $i$.

    Para la compatibilidad con los productos
    $$x^\alpha \prec x^\beta \Longrightarrow x^\alpha x^\gamma \prec x^\beta x^\gamma,$$
    notamos que si $u\cdot\alpha < u\cdot\beta$, entonces
    \[ u\cdot (\alpha+\gamma) =
       u\cdot\alpha + u\cdot\gamma <
       u\cdot\beta + u\cdot\gamma =
       u\cdot (\beta+\gamma). \]

     En fin, hay que verificar la propiedad de buen orden. No es tan fácil
     hacerlo directamente, pero recordemos que es suficiente comprobar que
     $x^\alpha \succeq 1$ para todo $\alpha$. Y en efecto,
     $$u\cdot \alpha \dfn u_1 \alpha_1 + \cdots + u_n \alpha_n \ge 0,$$
     dado que $u_i > 0$ y $\alpha_i \in \NN$.
   \end{solucion}\fi
\end{ejerc}

\end{document}
